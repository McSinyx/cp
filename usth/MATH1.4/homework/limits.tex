\documentclass[a4paper,12pt]{article}
\usepackage[utf8]{inputenc}
\usepackage[english,vietnamese]{babel}
\usepackage{amsmath}
\usepackage{amssymb}
\usepackage{enumerate}
\usepackage{mathtools}
\usepackage{pgfplots}
\usepackage{siunitx}
\title{Calculus Homework}
\author{Nguyễn Gia Phong}
\date{Winter 2018}
\newcommand{\ud}{\,\mathrm{d}}
\newcommand{\leibniz}[3][]{\frac{\mathrm{d} #1 #2}{\mathrm{d} #3 #1}}
\DeclareMathOperator{\erf}{erf}

\begin{document}
\maketitle
\setcounter{section}{1}
\section{Limits}

\setcounter{subsection}{2}
\subsection{Limit Laws}
Evaluate the limit:
\[\lim_{x \to 2}\sqrt{\frac{2x^2 + 1}{3x - 2}}
= \sqrt{\lim_{x \to 2}\frac{2x^2 + 1}{3x - 2}}
= \sqrt{\frac{2 \cdot 2^2 + 1}{3 \cdot 2 - 2}}
= \frac{3}{2} \tag{9}\]

\[\lim_{x \to 4}\frac{x^2 - 4x}{x^2 - 3x - 4}
= \lim_{x \to 4}\frac{x}{x + 1}
= \frac{4}{5} \tag{12}\]

\begin{align*}
   \lim_{t \to 0}\frac{\sqrt{1 + t} - \sqrt{1 - t}}{t}
&= \lim_{t \to 0}\frac{2t}{t(\sqrt{1 + t} + \sqrt{1 - t})} \\
&= \lim_{t \to 0}\frac{2}{(\sqrt{1 + t} + \sqrt{1 - t})} \\
&= \frac{2}{\sqrt{1} + \sqrt{1}} \\
&= 1 \tag{25}
\end{align*}

\noindent\textbf{40. }Prove that
$\lim_{x \to 0^+}\sqrt{x}e^{\sin\frac{\pi}{x}} = 0$.

Given $\varepsilon > 0$, let $\delta = \frac{1}{9}\varepsilon^2$.
If $0 < x < 0 + \delta$ then \[0 < \sqrt{x}e^{\sin\frac{\pi}{x}}
\leq e\sqrt{\delta} < 3\sqrt{\frac{\varepsilon^2}{9}}
\Longrightarrow |\sqrt{x}e^{\sin\frac{\pi}{x}} - 0 | < \varepsilon\]

Thus, by the definition of right-hand limit,
\[\lim_{x \to 0^+}\sqrt{x}e^{\sin\frac{\pi}{x}} = 0\]

\noindent\textbf{59. }Prove that $\lim_{x \to 0}f(x) = 0$ if
\[f(x) = \begin{cases}
           x^2\text{ if } x\text{ is rational} \\
           0\text{ if } x\text{ is irrational}
         \end{cases}\]

Given $\varepsilon > 0$, let $\delta = \sqrt{\varepsilon}$.
If $0 < |x - 0| < \delta$, then $0 < x^2 < \varepsilon$
or $|f(x) - 0| < \varepsilon$. Thus, by the definition of a limit,
\[\lim_{x \to 0}f(x) = 0\]

\noindent\textbf{61. }If $f(x) = \begin{cases}
                                   1\text{ if } x \geq 0 \\
                                   0\text{ if } x < 0
                                 \end{cases}$
and $g(x) = \begin{cases}
              0\text{ if } x \geq 0 \\
              1\text{ if } x < 0
            \end{cases}$
then $f(x)g(x) = 0$. Thus $\lim_{x \to 0}f(x)g(x) = 0$ though neither
$\lim_{x \to 0}f(x)$ nor $\lim_{x \to 0}g(x)$ exists.

\subsection{The precise definition of a limit}
\textbf{3. }Given $f(x) = \sqrt{x}$, if $|x - 4| < 1.44$ then
$|\sqrt{x} - 2| < 0.4$.

\noindent\textbf{21. }Prove that $\lim_{x \to 2}\frac{x^2 + x - 6}{x - 2} = 5$.

Given $\varepsilon > 0$, let $\delta = \varepsilon$.
If $0 < |x - 2| < \delta$, then \[|x + 3 - 5| < \varepsilon
\iff \left|\frac{x^2 + x - 6}{x - 2} - 5\right| < \varepsilon\]

Thus, by the definition of a limit,
$\lim_{x \to 2}\frac{x^2 + x - 6}{x - 2} = 5$.

\noindent\textbf{39. }Prove that $\lim_{x \to 0}f(x)$ does not exist if
\[f(x) = \begin{cases}
           0\text{ if } x\text{ is rational} \\
           1\text{ if } x\text{ is irrational}
         \end{cases}\].

Suppose $\lim_{x \to 0}f(x) = L$, hence by the definition of limit,
for every $\varepsilon > 0$, there exists $\delta > 0$ that
\[0 < |x - 0| < \delta \Rightarrow |f(x) - L| < \varepsilon \tag{$*$}\]

For $L = 0$, consider $\varepsilon = |L - 1|$. For every $\delta$, there is
at least one irrational $x \in (0, \delta)$, which turns $(*)$ into a false
statement: \[0 < |x| < \delta \Rightarrow |1 - L| < |L - 1|\]

For $L \neq 0$, consider $\varepsilon = |L|$. For every $\delta$, there is
at least one rational $x \in (0, \delta)$, which turns $(*)$ into a false
statement: \[0 < |x| < \delta \Rightarrow |L| < |L|\]

Conclusion: The assumption is incorrect; in other words, $\lim_{x \to 0}f(x)$
does not exist.

\subsection{Continuity}
\textbf{22. }Explain why the function $f$ is discontinuous at the given number
$a = 3$.
\begin{align*}
f(x) &= \begin{cases}
          \frac{2x^2 - 5x - 3}{x - 3}\text{ if } x \neq 3 \\
          6\text{ if } x = 3
        \end{cases}\\
     &= \begin{cases}
          2x + 1\text{ if } x \neq 3 \\
          6\text{ if } x = 3
        \end{cases}
\end{align*}

Since $\lim_{x \to 3}f(x) = \lim_{x \to 3}(2x + 1) = 7 \neq 6 = f(3)$,
$f$ is discontinuous at $3$.

\begin{tikzpicture}
  \begin{axis}[
    axis x line=middle, axis y line=middle,
    xlabel={$x$}, ylabel={$f(x)$},
    xlabel style={at=(current axis.right of origin), anchor=west},
    ylabel style={at=(current axis.above origin), anchor=south}]
    \addplot[domain=-1:4,blue]{2*x + 1};
    % This part is a bit hacky but it works XD
    \addplot[white, mark=*, only marks] coordinates {(3,7)};
    \addplot[blue, mark=o, only marks] coordinates {(3,7)};
    \addplot[blue, mark=*, only marks] coordinates {(3,6)};
  \end{axis}
\end{tikzpicture}

\noindent\textbf{26. }$G(x) = \frac{x^2 + 1}{2x^2 - x - 1}$ is a rational
function so it is continuous at every number in its domain.

\noindent\textbf{38. }Since $arctan$ is an inverse trigonometric function and
thus continuous at every number in its domain and
$\lim_{x \to 2}\frac{x^2 - 4}{3x^2 - 6x} = \lim_{x \to 2}\frac{x + 2}{3x} =
\frac{2}{3}$, \[\lim_{x \to 2}\arctan\frac{x^2 - 4}{3x^2 - 6x} =
\arctan\frac{2}{3}\]

\subsection{To Infinity and Beyond!}
Find the limit:
\[\lim_{x \to -\infty}\frac{\sqrt{9x^6 - x}}{x^3 + 1}
= \lim_{x \to -\infty}\frac{\sqrt{9 - \frac{1}{x^5}}}{-1 - \frac{1}{x^3}}
= -3 \tag{24}\]

\subsection{Derivatives}
\textbf{24. }If $g(x) = x^4 - 2$,
\begin{align*}
g'(1) &= \lim_{h \to 0}\frac{g(1 + h) - g(1)}{h}\\
      &= \lim_{h \to 0}\frac{(1 + h)^4 - 2 - (1^4 - 2)}{h}\\
      &= \lim_{h \to 0}\frac{h^4 + 4h^3 + 6h^2 + 4h + 1 - 1}{h}\\
      &= \lim_{h \to 0}(h^3 + 4h^2 + 6h + 4)\\
\end{align*}

An equation of the tangent line to $g$ at $(1, -1)$:
\[y - g(1) = g'(1)(x - 1) \iff y = 4x - 5\]

\noindent Determine whether $f'(0)$ exists.
\[f(x) = \begin{cases}
           x\sin\frac{1}{x}\text{ if } x \neq 0 \\
           0\text{ if } x = 0
         \end{cases}\tag{53}\]
\begin{align*}
f'(0) &= \lim_{h \to 0}\frac{f(0 + h) - f(0)}{h}\\
      &= \lim_{h \to 0}\frac{h\sin\frac{1}{h}}{h}\\
      &= \lim_{h \to 0}\sin\frac{1}{h}\tag{does not exist}\\
\end{align*}

\[f(x) = \begin{cases}
           x^2\sin\frac{1}{x}\text{ if } x \neq 0 \\
           0\text{ if } x = 0
         \end{cases}\tag{54}\]
\begin{align*}
f'(0) &= \lim_{h \to 0}\frac{f(0 + h) - f(0)}{h}\\
      &= \lim_{h \to 0}\frac{h^2\sin\frac{1}{h}}{h}\\
      &= \lim_{h \to 0}h\sin\frac{1}{h}
\end{align*}

Since $\forall h \neq 0, -|h| \leq h\sin\frac{1}{h} \leq |h|$
and $\lim_{h \to 0}(-|h|) = \lim_{h \to 0}|h| = 0$,
according to the Squeeze Theorem, $f'(0) = 0$.

\section{Differentiation}
\setcounter{subsection}{3}
\subsection{The chain rule}
Find the derivative of the function.

\[y = \cos\sqrt{\sin(\tan{\pi x})}\tag{45}\]
\begin{align*}
\dot{y} &= -\sqrt{\sin(\tan{\pi x})}' \cdot \sin\sqrt{\sin(\tan{\pi x})}\\
        &= \frac{\sin'(\tan{\pi x}) \cdot \sin\sqrt{\sin(\tan{\pi x})}}
                {2\sqrt{\sin(\tan{\pi x})}}\\
        &= \frac{\tan'{\pi x} \cdot \cos(\tan{\pi x})
                 \cdot \sin\sqrt{\sin(\tan{\pi x})}}
                {2\sqrt{\sin(\tan{\pi x})}}\\
        &= \frac{\pi\sec^2{\pi x} \cdot \cos(\tan{\pi x})
                 \cdot \sin\sqrt{\sin(\tan{\pi x})}}
                {2\sqrt{\sin(\tan{\pi x})}}
\end{align*}

\[y = [x + (x + \sin^2{x})^3]^4 \tag{46}\]
\begin{align*}
\dot{y} &= 4[x + (x + \sin^2{x})^3]' [x + (x + \sin^2{x})^3]^3\\
        &= 4[1 + 3(x + \sin^2{x})' (x + \sin^2{x})^2]
            [x + (x + \sin^2{x})^3]^3\\
        &= 4[1 + 3(1 + \sin{2x})(x + \sin^2{x})^2]
            [x + (x + \sin^2{x})^3]^3
\end{align*}

\setcounter{subsection}{6}
\subsection{Applications in Sciences}
\textbf{9. }A rock is thrown vertically upward from the surface of Mars, its
height after $t$ seconds is $h = 15t - 1.86t^2$.
\[\frac{\ud h}{\ud t}(2)
= (t \mapsto 15 - 3.72t)(2)
= 7.56\text{ (m/s)}\tag{a}\]
\[h = 25 \iff 15t - 1.86t^2 = 25
  \iff t = \frac{375 \mp 25\sqrt{39}}{93}\tag{b}\]

So at $t \approx 2.35$ s or $t \approx 5.71$ the Rock's height is 25 m. Its
velocity at this point is
\[v = (t \mapsto 15 - 3.72t)\left(\frac{375 \mp 25\sqrt{39}}{93}\right)
    = \pm 6.24\text{ (m/s)}\]

\pagebreak\noindent\textbf{10. }A particle moves with position function
\[s = t^4 - 4t^3 - 20t^2 + 20t \qquad t \geq 0\]
\[v = 20 \iff \dot{s} = 20 \iff 4t^3 - 12t^2 - 40t + 20 = 20 \tag{a}\]

Since $t$ is nonnegative, the particle has a velocity of 20 m/s at $t = 0$ and
$t = 5$ s.
\[a = 0 \iff \dot{v} = 0 \iff 12t^2 - 24t - 40 = 0 \tag{b}\]

Since $t$ is nonnegative, the acceleration is 0 at $t = \sqrt\frac{13}{3} - 1$
s. This is when the instantaneous speed of the particle ($|v|$) reaches its
critical value.

\noindent\textbf{21. }The force $F$ acting on a body with velocity $v$ and mass
$m = m_0 \Big/ \sqrt{1 - \frac{v^2}{c^2}}$ (where $m_0$ is the mass of the particle
at rest and $c$ is the speed of light) is the rate of change of momentum:
\begin{align*}
F &= \frac{\ud(mv)}{\ud t}\\
  &= \frac{\ud}{\ud t}\left(\frac{m_0v}{\sqrt{1 - \frac{v^2}{c^2}}}\right)\\
  &= m_0\frac{\ud v}{\ud t}\cdot
     \frac{\ud}{\ud v}\left(\frac{v}{\sqrt{1 - \frac{v^2}{c^2}}}\right)\\
  &= m_0a\frac{\ud}{\ud v}\left(\frac{v}{\sqrt{1 - \frac{v^2}{c^2}}}\right)\\
  &= m_0a\frac{\sqrt{1 - \frac{v^2}{c^2}}
               - v\frac{\ud\sqrt{1 - v^2/c^2}}{\ud v}}
              {1 - \frac{v^2}{c^2}}\\
  &= m_0a\frac{\sqrt{1 - \frac{v^2}{c^2}}
               - \frac{v}{2}\cdot\frac{-2v}{c^2\sqrt{1 - v^2/c^2}}}
              {1 - \frac{v^2}{c^2}}\\
  &= m_0a\frac{\sqrt{1 - \frac{v^2}{c^2}} + \frac{v^2}{c^2\sqrt{1 - v^2/c^2}}}
              {1 - \frac{v^2}{c^2}}\\
  &= m_0a\frac{1 - \frac{v^2}{c^2} + \frac{v^2}{c^2}}
              {\left(1 - \frac{v^2}{c^2}\right)^\frac{3}{2}}\\
  &= \frac{m_0a}{\left(1 - \frac{v^2}{c^2}\right)^\frac{3}{2}}
\end{align*}

\noindent\textbf{30. }The frequency of vibrations a vibrating violin string is
given by \[f = \frac{1}{2L}\sqrt{\frac{T}{\rho}} \qquad T \geq 0, \rho > 0\]
\begin{enumerate}[(a)]
  \item The rate of change of the frequency with respect to
    \begin{enumerate}[(i)]
      \item The length:
        $\frac{\ud f}{\ud L} = \frac{-1}{L^2}\sqrt{\frac{T}{\rho}}$.
      \item The tension: $\frac{\ud f}{\ud T} = \frac{1}{4L\sqrt{T\rho}}$.
      \item The density:
        $\frac{\ud f}{\ud L} = \frac{-1}{L2}\sqrt{\frac{T}{\rho^3}}$.
    \end{enumerate}
  \item The pitch of a note gets higher when the string is shorter and lower
    when the tension or density is increased.
\end{enumerate}

\noindent\textbf{35. }Applying the gas law
\[PV = nRT \iff T = \frac{PV}{nR}\]

The rate of change of temperature can be easily calculated via differentiation:
\begin{align*}
   \frac{\ud T}{\ud t}
&= \frac{\ud}{\ud t}\left(\frac{PV}{nR}\right)\\
&= \frac{1}{nR}\left(P\frac{\ud V}{\ud t} + V\frac{\ud P}{\ud t}\right)\\
&= \frac{8.0 \cdot 0.15 + 10 \cdot 0.10}{10 \cdot 0.0821}\\
&= \frac{1.2 + 1}{10 \cdot 0.0821}\\
&= \frac{2}{10 \cdot 0.0821}\\
&= 2\text{ (K/s)}
\end{align*}

(In the calculation above, significant figures are taken into consideration.)

\pagebreak\subsection{Exponential Growth and Decay}
\textbf{4. }Let $P(t)$ be the bacteria count after $t$ hours. As the bacteria
culture grows with constant relative growth rate,
\[\frac{\ud P}{\ud t} = kP \Longrightarrow P(t) = P(0)e^{kt}\]

Since $P(2) = 400$ and $P(6) = 25600$,
\begin{align*}
  \begin{cases}
    P(0)e^{2k} = 400\\
    P(0)e^{6k} = 25600
  \end{cases}
  &\iff
  \begin{cases}
    P(0)e^{2k} = 400\\
    e^{4k} = 64
  \end{cases}\\
  &\iff
  \begin{cases}
    P(0)e^{2k} = 400\\
    e^{2k} = 8
  \end{cases}\\
  &\iff
  \begin{cases}
    P(0) = 50\\
    k = \frac{\ln 8}{2} \approx 104\%
  \end{cases}
\end{align*}

Thus (a) the relative growth rate is 104\%, (b) the initial size of the culture
is 50 and (c) the number of bacteria after $t$ hours is $50\sqrt{8^t}$.

The number of cells after 4.5 hours:
\[P(4.5) = 50\sqrt{8^{4.5}} \approx 5382\tag{d}\]

The rate of growth after 4.5 hours:
\begin{align*}
   \frac{\ud P}{\ud t}(4.5)
&= 50\frac{\ud \sqrt{8}^t}{\ud t}(4.5)\\
&= 50\left(t \mapsto \sqrt{8^t}\ln\sqrt{8}\right)(4.5)\\
&= 25\cdot8^{2.25}\ln 8\\
&\approx 5596\text{ (bacteria per minute)}\tag{e}
\end{align*}

The population reach 50000 when
\[50\sqrt{8^t} = 50000 \iff 8^t = 10^6 \iff t = \log_2{100}
\approx 6.64\text{ (days)}\tag{f}\]

\noindent\textbf{8. }Given 50 mg of $^{90}$Sr which has a half-life of 28 days.
\begin{enumerate}[(a)]
  \item Formula of the mass remaining after $t$ days: $m(t) = 50\cdot2^{-t/28}$.
  \item The mass remaining after 40 days:
    $m(40) = 50\cdot\frac{1}{2}^{10/7} \approx 19\text{ (mg)}$.
  \item To decay to a mass of 2 mg, it takes
    $-28\log_2\frac{2}{50} \approx 130\text{ (days)}$.
  \item The graph of the mass function:\\
    \begin{tikzpicture}
      \begin{axis}[
        axis x line=middle, axis y line=middle,
        xlabel={$t$}, ylabel={$m$},
        xlabel style={at=(current axis.right of origin), anchor=west},
        ylabel style={at=(current axis.above origin), anchor=south},
        enlarge y limits={rel=0.1}, enlarge x limits={rel=0.1}]
        \addplot[domain=0:130,magenta]{50*2^(-x/28)};
      \end{axis}
    \end{tikzpicture}
\end{enumerate}

\noindent\textbf{16. }Let $T(t)$ be the temperature of the coffee after $t$
minutes. The surrounding temperature is \ang{20}C, so Newton's Law of Cooling
states that \[\frac{\ud T}{\ud t} = k(T - 20)\]

If we let $y = T - 20$, then $y(0) = T(0) - 20 = 95 - 20 = 75$, so $y$
satisfies \[\frac{\ud y}{\ud t} = ky \iff y(t) = 75e^{kt}\]

When the temperature of the coffee is \ang{70}C, its cooling rate is \ang{1}C
per minute, i.e.
\begin{align*}
  \begin{cases}
    y(t) + 20 = 70\\
    ky(t) = -1
  \end{cases}
  &\iff\begin{cases}
         y(t) = 50\\
         k = \frac{-1}{50}
       \end{cases}\\
  \Longrightarrow 75e^{-t/50} = 50
    &\iff t = 50\ln1.5 \approx 20\text{ (minutes)}
\end{align*}

\subsection{Related rates}
\textbf{10. }A particle is moving along a hyperbola $xy = 8$
\begin{align*}
  \Longrightarrow \frac{\ud (xy)}{\ud t} = \frac{\ud 8}{\ud t}
  &\iff y\frac{\ud x}{\ud t} + x\frac{\ud y}{\ud t} = 0\\
  &\iff 2 \cdot \frac{\ud x}{\ud t} + 4 \cdot 3 = 0\\
  &\iff \frac{\ud x}{\ud t} = -6\text{ (cm/s)}
\end{align*}

\noindent\textbf{12. }Let $D(t)$ (cm) be the diameter of the ball at minute
$t$, its surface area is $A(D) = \pi D^2$ (cm$^2$).
\[\frac{\ud A}{\ud t} = 1 \iff \frac{\ud A}{\ud D} \cdot \frac{\ud D}{\ud t} = 1
  \iff 2\pi D \frac{\ud D}{\ud t} = 1 \iff \frac{\ud D}{\ud t}
                                  = \frac{1}{2\pi D}\]

Thus the decreasing rate of the diameter when it is 10 cm:
\[\frac{\ud D}{\ud t}(10) = \frac{1}{20\pi}\text{ (cm/s)}\]

\noindent\textbf{14. }

\begin{tikzpicture}
  \begin{axis}[nodes near coords align=right, xlabel={W -- E}, ylabel={S -- N}]
    \node[label=A, shape=circle, fill, inner sep=1.5pt] at (0,0) {};
    \addplot[->] plot coordinates {(0,0) (35,0)};
    \node[label={180:B}, shape=circle, fill, inner sep=1.5pt] at (150,0) {};
    \addplot[->] plot coordinates {(150,0) (150,25)};
  \end{axis}
\end{tikzpicture}
\[\begin{cases}
    \Delta x(t) = x_B(t) - x_A(t)\\
    \Delta y(t) = y_B(t) - y_A(t)
  \end{cases}
\iff\begin{cases}
      \Delta x(t) = 150 - 35t\\
      \Delta y(t) = 25t
  \end{cases}\]
\[\Longrightarrow\Delta s(t) = \sqrt{1850t^2 - 10500t + 22500}
  \Longrightarrow\frac{\ud s}{\ud t} = \frac{1850t - 5250}
                                       {\sqrt{1850t^2 - 10500t + 22500}}\]
\[\Longrightarrow\frac{\ud s}{\ud t}(4)
               = \frac{1850\cdot4 - 5250}
                      {\sqrt{1850\cdot16 - 10500\cdot4 + 22500}}
               = \frac{2150}{\sqrt{10100}}
               = \frac{215\sqrt{101}}{101}
               \approx 21\text{ (km/h)}\]

\noindent\textbf{27. }Let $h(t)$ (ft) be the height of the cone at minute $t$.
Volume of the cone is
\begin{align*}
  V(h) = \frac{\pi h^2}{12}
  &\Longrightarrow\frac{\ud V}{\ud t} = \frac{\pi h}{6}\cdot\frac{\ud h}{\ud t}
                                      = 30
  \iff\frac{\ud h}{\ud t} = \frac{180}{\pi h}\\
  &\Longrightarrow\frac{\ud h}{\ud t}(10) = \frac{180}{10\pi} = \frac{18}{\pi}
  \approx 5.7\text{ (ft/s)}
\end{align*}

\section{Applications of derivative}
\subsection{Max and Min}
Find the absolute min and max values of $f$.

\[f(x) = 3x^4 - 4x^3 - 12x^2 + 1, \qquad [-2, 3]\tag{51}\]

Since $f'(x) = 12x^3 - 12x^2 - 24x$, we have $f'(x) = 0$ when
$x \in \{-1, 0, 2\}$. The values of $f$ at these critical numbers are
\begin{align*}
  f(-1) &= 3 + 4 - 12 + 1 = -4\\
  f(0)  &= 1\\
  f(2)  &= 48 - 32 - 48 + 1 = -31
\end{align*}

The values of $f$ at endpoints are
\begin{align*}
  f(-2) &= 48 + 32 - 48 + 1 = 33\\
  f(3)  &= 243 - 108 - 108 + 1 = 28
\end{align*}

Comparing these five numbers and using the Closed Interval Method, we see that
the absolute minimum value is $f(2) = -31$ and the absolute maximum value is
$f(-2) = 33$.

\[f(t) = t\sqrt{4 - t^2}, \qquad [-1, 2]\tag{55}\]

Since $f'(t) = \frac{4 - 2t^2}{\sqrt{4 - t^2}}$, with $t \in [-1, 2]$, we have
$f'(t) = 0$ when $t = \sqrt{2}$. The value of $f$ at this critical number is
$f(\sqrt{2}) = 2$. The value of $f$ at endpoints are $f(-1) = -\sqrt{3}$ and
$f(2) = 0$. Comparing these 3 numbers and using the Closed Interval Method, we
see that the absolute minimum value is $f(-1) = -\sqrt{3}$ and the absolute
maximum value is $f(\sqrt{2}) = 2$.

\[f(x) = xe^{-x^2/8}, \qquad [-1, 4]\tag{59}\]

With $x \in [-1, 4]$, we have $f'(x) =  \left(1-\frac{x^2}{4}\right)e^{-x^2/8}$
when $x = 2$. Comparing values of $f$ at this critical number and endpoints,
the minimum value is $f(-1) = -e^{-1/8}$ and
the maximum value is $f(2) = 2e^{-1/2}$.

\subsection{The Mean Theorem}
\textbf{26. }Let $h$ be the function that $h(x) = f(x) - g(x)$. Since both $f$
and $g$ are continuous on $[a, b]$ and differentiable on $(a, b)$, $h$ inherits
the same properties. By applying the Mean Value Theorem to $h$ on the interval
$[a, b]$, we get a number $c \in (a, b)$ such that
\begin{align*}
    &h(b) - h(a) = (b - a)h'(c)\\
\iff&f(b) - g(b) - f(a) + g(a) = (b - a)(f'(c) - g'(c))\\
\iff&f(b) - g(b) = (b - a)(f'(c) - g'(c))
\end{align*}
$b - a > 0$ and $f'(c) - g'(c) < 0$ so $f(b) - g(b) < 0$ or $f(b) < g(b)$.

\begin{align*}
x > 0 &\iff x + 1 > 1\\
      &\iff \sqrt{x + 1} > 1\\
      &\iff \sqrt{x + 1} - 1 > 0\\
      &\Longrightarrow \left(\sqrt{x + 1} - 1\right)^2 > 0\\
      &\iff x + 1 - 2\sqrt{x + 1} + 1 > 0\\
      &\iff x + 2 > 2\sqrt{x + 1}\\
      &\iff \sqrt{1 + x} < 1 + \frac{1}{2}x\tag{27}
\end{align*}

\noindent\textbf{33. }Prove the identity
\[\arcsin\frac{x - 1}{x + 1} = 2\arctan\sqrt{x} - \frac{\pi}{2}\]

Let $\frac{-\pi}{2} \leq y = \arcsin\frac{x - 1}{x + 1} \leq \frac{\pi}{2}$ and
$z = \arctan\sqrt{x}$, then
\begin{align*}
\begin{cases}
  \sin{y} = \frac{x - 1}{x + 1}\\
  \tan{z} = \sqrt{x}
\end{cases}
\Longrightarrow&
\begin{cases}
  \frac{\ud\sin{y}}{\ud x} = \frac{\ud}{\ud x}\left(\frac{x - 1}{x + 1}\right)\\
  \frac{\ud\tan{z}}{\ud x} = \frac{\ud\sqrt{x}}{\ud x}
\end{cases}\\
\iff&
\begin{cases}
  \cos{y}\cdot\frac{\ud y}{\ud x} = \frac{2}{(x + 1)^2}\\
  \left(\tan^2{z} + 1\right)\frac{\ud z}{\ud x} = \frac{1}{2\sqrt{x}}
\end{cases}\\
\iff&
\begin{cases}
  \sqrt{1 - \sin^2{y}}\cdot\frac{\ud y}{\ud x} = \frac{2}{(x + 1)^2}\\
  \left(\sqrt{x}^2 + 1\right)\frac{\ud z}{\ud x} = \frac{1}{2\sqrt{x}}
\end{cases}\\
\iff&
\begin{cases}
  \sqrt{\frac{4x}{(x + 1)^2}}\cdot\frac{\ud y}{\ud x} = \frac{2}{(x + 1)^2}\\
  (x + 1)\frac{\ud z}{\ud x} = \frac{1}{2\sqrt{x}}
\end{cases}\\
\iff&
\begin{cases}
  \frac{\ud y}{\ud x} = \frac{1}{|x + 1|\sqrt{x}}\\
  \frac{\ud z}{\ud x} = \frac{1}{2(x + 1)\sqrt{x}}
\end{cases}\\
\end{align*}

For all $x \geq 0$ or $x + 1 \geq 1 > 0$
\begin{align*}
   \frac{\ud}{\ud x}\left(2\arctan\sqrt{x} - \arcsin\frac{x - 1}{x + 1}\right)
&= 2\frac{\ud z}{\ud x} - \frac{\ud y}{\ud x}\\
&= \frac{2}{2(x + 1)\sqrt{x}} - \frac{1}{(x + 1)\sqrt{x}}\\
&= 0
\end{align*}

Thus the function
$x \mapsto 2\arctan\sqrt{x} - \arcsin\frac{x - 1}{x + 1}$ is constant on its
domain $[0, \infty)$. Consequently, in $[0, \infty)$
\begin{align*}
   2\arctan\sqrt{x} - \arcsin\frac{x - 1}{x + 1}
&= \left(x \mapsto 2\arctan\sqrt{x} - \arcsin\frac{x - 1}{x + 1}\right)(0)\\
&= 2\arctan\sqrt{0} - \arcsin\frac{-1}{1}\\
&= 0 - \frac{-\pi}{2}\\
&= \frac{\pi}{2}\\
\iff \arcsin\frac{x - 1}{x + 1} &= 2\arctan\sqrt{x} - \frac{\pi}{2}
\end{align*}

\subsection{Shape of a graph}
\textbf{75. }Given two funtions $f$ and $g$ which are positive and concave
upward on $I$, i.e. for all $x$ in $I$
\[\begin{cases}
    f(x) > 0\\
    f''(x) > 0\\
    g(x) > 0\\
    g''(x) > 0
  \end{cases}\]

Second derivative of the product function $fg$:
\[(f(x)g(x))'' = (f'(x)g(x) + f(x)g'(x))'
  = f''(x)g(x) + 2f'(x)g'(x) + f(x)g''(x)\]

If $f$ and $g$ are both increasing or decreasing, then $f'(x)g'(x) > 0$, which
means $\forall x \in I, (f(x)g(x))'' > 0$, or $fg$ is concave upward on $I$.
Otherwise, $f$ is increasing and $g$ is decreasing for instance, $fg$ may be
either concave upward, concave downward or linear:

\begin{center}
  \begin{tikzpicture}[domain=0:8]
    \begin{axis}[
      axis x line = middle, axis y line = middle,
      xlabel={$x$}, ylabel={$y$}, ymin=-8,
      xlabel style={at=(current axis.right of origin), anchor=west},
      ylabel style={at=(current axis.above origin), anchor=south},
      enlarge y limits={rel=0.1}, enlarge x limits={rel=0.1},
      legend pos=south east]
      \addplot[color=blue, samples=100, smooth]{-1/x};
      \addplot[color=cyan, samples=100, smooth]{-x^(3/2)};
      \addplot[color=red, samples=100, smooth]{sqrt(x)};
      \legend{$f(x) = \frac{-1}{x}$, $g(x) = -x\sqrt{x}$,
              $f(x)g(x) = \sqrt{x}$}
    \end{axis}
  \end{tikzpicture}

  \begin{tikzpicture}[domain=0:4]
    \begin{axis}[
      axis x line = middle, axis y line = middle,
      xlabel={$x$}, ylabel={$y$}, ymin=-8,
      xlabel style={at=(current axis.right of origin), anchor=west},
      ylabel style={at=(current axis.above origin), anchor=south},
      enlarge y limits={rel=0.1}, enlarge x limits={rel=0.1}]
      \addplot[color=blue, samples=100, smooth]{-1/x};
      \addplot[color=cyan, samples=100, smooth]{-x^3};
      \addplot[color=red, samples=100, smooth]{x^2};
      \legend{$f(x) = \frac{-1}{x}$, $g(x) = -x^3$, $f(x)g(x) = x^2$}
    \end{axis}
  \end{tikzpicture}

  \begin{tikzpicture}[domain=0:4]
    \begin{axis}[
      axis x line = middle, axis y line = middle,
      xlabel={$x$}, ylabel={$y$}, ymin=-8,
      xlabel style={at=(current axis.right of origin), anchor=west},
      ylabel style={at=(current axis.above origin), anchor=south},
      enlarge y limits={rel=0.1}, enlarge x limits={rel=0.1},
      legend pos=south east]
      \addplot[color=blue, samples=100, smooth]{-1/x};
      \addplot[color=cyan, samples=100, smooth]{-x^2};
      \addplot[color=red, samples=100, smooth]{x};
      \legend{$f(x) = \frac{-1}{x}$, $g(x) = -x^2$, $f(x)g(x) = x$}
    \end{axis}
  \end{tikzpicture}
\end{center}

\pagebreak\noindent\textbf{76. }In order for $h = f(g(x))$ to be concave
upward on $\mathbb R$
\begin{align*}
  h'' > 0 &\iff (f \circ g)'' > 0\\
          &\iff ((f' \circ g) \cdot g')' > 0\\
          &\iff (f' \circ g)' \cdot g' + (f' \circ g) \cdot g'' > 0\\
          &\iff (f'' \circ g) \cdot (g')^2 + (f' \circ g) \cdot g'' > 0
\end{align*}

Because $f$ and $g$ are given to be concave upward on $\mathbb R$, i.e.
$f'' > 0$ and $g'' > 0$, and $\forall x \in \mathbb R, g^2(x) \geq 0$, so if
$f' > 0$ or $f$ is an increasing function, $h$ will be concave upward.

\noindent\textbf{77. }Show that $\tan{x} > x$ for $0 < x < \frac{\pi}{2}$.

Let $f$ be the function that $f(x) = \tan{x} - x$. On $(0, \frac{\pi}{2})$,
$\sin{x}\cos{x} \neq 0$ thus $\tan{x}$ exists and is nonzero. Therefore,
$f'(x) = \tan^2(x) > 0$ or $f$ is increasing on $[0, \frac{\pi}{2}]$, which
means for all x in $(0, \frac{\pi}{2})$,
\[f(x) > f(0) \iff \tan{x} - x > \tan0 - 0 \iff \tan{x} > x\]

\noindent\textbf{78. } Use mathematical induction to prove that for all
positive integer $n$,
\[\forall x \geq 0, \qquad e^x \geq 1 + \sum_{i=1}^n\frac{x^i}{i!}\tag{$*$}\]

Let $f$ be the function of domain $[0, \infty)$ that $f(x) = e^x - x$, then
for all $x \geq 0$, $f'(x) = e^x - 1 > f'(0) = 0$ (since it is obvious that $f'$
is an increasing function). Hence
$\forall x \geq 0, e^x - x > e^0 - 0 = 1 \iff \forall x \geq 0, e^x > 1 + x$,
i.e.  $(*)$ is true for $n = 1$.

Suppose that $(*)$ is also true for $n = k$ ($k \in \mathbb N^*$). For all
nonnegative $x$,
\[e^x \geq 1 + \sum_{i=1}^k\frac{x^i}{i!}
  \iff e^x - 1 - \sum_{i=1}^k\frac{x^i}{i!} \geq 0\]

Let $g$ be the function of domain $[0, \infty)$ that
$g(x) = e^x - \sum_{i=1}^{k+1}\frac{x^{i}}{i!}$, then for all positive $x$
\[g'(x) = e^x - \sum_{i=1}^{k+1}\frac{ix^{i-1}}{i!}
        = e^x - 1 - \sum_{i=2}^{k+1}\frac{x^{i-1}}{(i - 1)!}
        = e^x - 1 - \sum_{i=1}^{k}\frac{x^{i}}{i!} \geq 0\]

This means $g$ in a non-decreasing function on $[0, \infty)$
\[e^x - \sum_{i=1}^{k+1}\frac{x^{i}}{i!}
  \geq e^0 - \sum_{i=1}^{k+1}\frac{0^{i}}{i!} = 1\]

This expression shows that $(*)$ is true for $n = k + 1$. Therefore, by
mathematical induction, it is true for all positive integers $n$.

\subsection{Rule of the Hospital}
Find the limit

\noindent\textbf{54. }Since $\lim_{x \to 0^+}\ln x = -\infty$
and $\lim_{x \to 0^+}\frac{1}{\sqrt x} = \infty$
\[\lim_{x \to 0^+}x^{\sqrt x}
= \lim_{x \to 0^+}e^{\sqrt{x}\cdot\ln{x}}
= e^{\lim_{x \to 0^+}\frac{\ln x}{1/\sqrt x}}
= e^{\lim_{x \to 0^+}\frac{-2x\sqrt x}{x}}
= e^{-2\lim_{x \to 0^+}\sqrt x}
= e^0
= 1\]

\noindent\textbf{60. }Since
$\lim_{x \to \infty}\ln{2}\ln x = \lim_{x \to \infty}(1 + \ln x) = \infty$
\[\lim_{x \to \infty}x^\frac{\ln 2}{1 + \ln x}
= e^{\lim_{x \to \infty}\frac{\ln{2}\ln x}{1 + \ln x}}
= e^{\lim_{x \to \infty}\ln{2}}
= e^{\ln 2}
= 2\]

\setcounter{subsection}{6}
\subsection{Optimization Problems}
\textbf{44. }Given $E(v) = \frac{aLv^3}{v - u}$
\[\frac{\ud E}{\ud v}
= aL\frac{3v^2(v - u) - v^3}{(v - u)^2}
= aL\frac{2v^3 - 3uv^2}{(v - u)^2}\]

Since $v > u > 0$, $E$ has only one absolute extreme value, at the only
critical number $v = 1.5u$. Applying the First Derivative Test for Absolute
Extreme Values, $v = 1.5u$ is shown to be the value of $v$ that minimizes $E$.

\noindent\textbf{45. }Given
$S = 6sh + \frac{3}{2}s^2(\sqrt{3}\cdot\csc\theta - \cot\theta)$.
\[\frac{\ud S}{\ud\theta}
= \frac{3}{2}s^2\frac{\ud}{\ud\theta}(\sqrt{3}\cdot\csc\theta - \cot\theta)
= \frac{3s^2(1 - {\sqrt{3}\cdot\cos\theta})}{2\sin^2\theta}\]

We have $\frac{\ud S}{\ud\theta} = 0$ when $\theta = \arccos\frac{\sqrt 3}{3}$.
Applying the First Derivative Test for Absolute Extreme Values, this value of
$\theta$ is shown to minimize $S$ to $6sh + \frac{3s^2}{\sqrt 2}$.

\noindent\textbf{76. }Using Poiseuille's Law, we have the total resistance of
the blood along the path ABC is
\begin{multline*}
R = R_{AB} + R_{BC}
  = C\frac{a - b\cot\theta}{r_1^4} + C\frac{b}{r_2^4 \sin\theta}
  = C\left(\frac{a - b\cot\theta}{r_1^4} + \frac{b\csc\theta}{r_2^4}\right)\\
\Longrightarrow\frac{\ud R}{\ud\theta}
= \frac{Cb}{r_1^4 \sin^2\theta} - \frac{Cb\cos\theta}{r_2^4\sin^2\theta}
= \frac{Cb}{sin^2\theta}\left(\frac{1}{r_1^4} - \frac{\cos\theta}{r_2^4}\right)
\end{multline*}

We have $\frac{\ud R}{\ud\theta} = 0$ when $\cos\theta = (r_2/r_1)^4$. At this
angle, the resistance is minimized (can be shown using the First Derivative
Test for Absolute Extreme Values, but like in the two previous exercises, I'm
too lazy to evaluate it). When $\frac{r_2}{r_1} = \frac{2}{3}$, the optimal
branching angle is $\theta \approx \ang{79}$.

\section{Integral}
\subsection{Areas}
\textbf{4. }Estimate the area under the graph of $f(x) = \sqrt{x}$ from $x = 0$
to $x = 4$.
\[\lim_{n \to \infty}R_n
= \lim_{n \to \infty}\frac{4 - 0}{n}\sum_{i = 1}^n\sqrt\frac{4i}{n}
= \lim_{n \to \infty}\frac{8}{n}\sum_{i = 1}^n\sqrt\frac{i}{n}\]
\[\lim_{n \to \infty}L_n
= \lim_{n \to \infty}\frac{4 - 0}{n}\sum_{i = 0}^{n - 1}\sqrt\frac{4i}{n}
= \lim_{n \to \infty}\frac{8}{n}\sum_{i = 0}^{n - 1}\sqrt\frac{i}{n}\]

For estimation, consider $n \to 4$:
\[\lim_{n \to 4}R_n
= \frac{8}{4}\sum_{i = 1}^4\sqrt\frac{i}{4}
= \sum_{i = 1}^4\sqrt i
= 1 + \sqrt 2 + \sqrt 3 + 2
\approx 6.1463\]
\[\lim_{n \to 4}L_n
= \frac{8}{4}\sum_{i = 0}^3\sqrt\frac{i}{4}
= \sum_{i = 0}^3\sqrt i
= 0 + 1 + \sqrt 2 + \sqrt 3
\approx 4.1463\]

\noindent\textbf{5. }Estimate the area under the graph of $f(x) = 1 + x^2$ from
$x = -1$ to $x = 2$.
\begin{align*}
  \lim_{n \to \infty}R_n
=&\lim_{n \to \infty}\frac{2 + 1}{n}\sum_{i = 1}^n
  f\left(-1 + i\frac{2 + 1}{n}\right)\\
=&\lim_{n \to \infty}\frac{3}{n}\sum_{i = 1}^n
  \left(1 + \left(\frac{3i}{n} - 1\right)^2\right)
\end{align*}

Similarly,
\begin{align*}
  \lim_{n \to \infty}L_n
=&\lim_{n \to \infty}\frac{3}{n}\sum_{i = 0}^{n - 1}
  \left(1 + \left(\frac{3i}{n} - 1\right)^2\right)\\
  \lim_{n \to \infty}M_n
=&\lim_{n \to \infty}\frac{3}{n}\sum_{i = 1}^n
  \left(1 + \left(\frac{3i - 3/2}{n} - 1\right)^2\right)
\end{align*}

For $n \to 3$,
\begin{align*}
  \lim_{n \to 3}R_n
=&\sum_{i = 1}^3\left(1 + (i - 1)^2\right)
= 1 + 2 + 5
= 8\\
  \lim_{n \to 3}M_n
=&\sum_{i = 1}^3\left(1 + \left(i - \frac{3}{2}\right)^2\right)
= \frac{5}{4} + \frac{5}{4} + \frac{13}{4}
= 5.75\\
  \lim_{n \to 3}L_n
=&\sum_{i = 0}^2\left(1 + (i - 1)^2\right)
= 2 + 1 + 2
= 5
\end{align*}

For $n \to 6$,
\begin{align*}
  \lim_{n \to 6}R_n
=&\frac{1}{2}\sum_{i = 1}^6\left(1 + \left(\frac{i}{2} - 1\right)^2\right)
= \frac{1}{2}\left(\frac{5}{4} + 1 + \frac{5}{4} + 2 + \frac{13}{4} + 5\right)
= 6.875\\
  \lim_{n \to 6}M_n
=&\frac{1}{2}\sum_{i = 1}^6\left(1 + \left(\frac{2i - 5}{4}\right)^2\right)
= \frac{1}{2}\left(\frac{25}{16} + \frac{17}{16} + \frac{17}{16} +
                   \frac{25}{16} + \frac{41}{16} + \frac{65}{16}\right)
= 5.9375\\
  \lim_{n \to 6}L_n
=&\frac{1}{2}\sum_{i = 0}^5\left(1 + \left(\frac{i}{2} - 1\right)^2\right)
= \frac{1}{2}\left(2 + \frac{5}{4} + 1 + \frac{5}{4} + 2 + \frac{13}{4}\right)
= 5.375
\end{align*}

\noindent\textbf{16. }The height (in feet) above the earth's surface of the
\textit{Endeavour}, 62 seconds after liftoff, can be estimated with the assist
of Python (which, coincidentally, has been utilized by NASA recently):
\begin{verbatim}
>>> time = 0, 10, 15, 20, 32, 59, 62, 125
>>> velocity = 0, 185, 319, 447, 742, 1325, 1445, 4151
>>> sum(map(int.__mul__, velocity,
...         map(int.__sub__, time[1:], time[:-1])))
122928
\end{verbatim}

\subsection{The Definite Integral}
Evaluate the integral.
\begin{align*}
  \int_2^5(4 - 2x)\ud x &= \left.4x\right]_2^5 - \left.x^2\right]_2^5
= 12 - 21 = -9\tag{21}\\
  \int_0^2(2x - x^3)\ud x &= \left.x^3\right]_0^2 - \left.\frac{x^4}{4}\right]_0^2
= 9 - 16 = -7\tag{24}
\end{align*}

\noindent\textbf{33. }Evaluate integral by interpreting it in terms of areas.
\begin{align*}
  \int_0^2 f(x)\ud x &= 4\tag{a}\\
  \int_0^5 f(x)\ud x &= 10\tag{b}\\
  \int_5^7 f(x)\ud x &= -3\tag{c}\\
  \int_0^9 f(x)\ud x &= \int_0^5 f(x)\ud x + \int_5^9 f(x)\ud x
                      = 10 - 8 = 2\tag{d}
\end{align*}

\noindent\textbf{50. }Given $f(x) = \begin{cases}
                                      3\text{ for } x < 3\\
                                      x\text{ for } x \geq 3
                                    \end{cases}$
\[\int_0^5 f(x)\ud x = \int_0^3 3\ud x + \int_3^5 x\ud x
= \left.3x\right]_0^3 + \left.\frac{x^2}{2}\right]_3^5
= 9 + 8 = 17\]

\subsection{The Fundamental Theorem of Calculus}
\textbf{3. }Let $g(x) = \int_0^x f(t)\ud t$.
\begin{enumerate}[(a)]
\item By interpreting the above integral in terms of areas, we get $g(0) = 0$,
  $g(1) = 2$, $g(2) = 5$, $g(3) = 7$ and $g(6) = 3$.
\item $g$ is increasing on $(0, 3)$.
\item $g$ has a maximum value of 7 at $x = 3$.
\item \begin{tikzpicture}
        \begin{axis}[
          axis x line = middle, axis y line = middle,
          xlabel={$x$}, ylabel={$y$},
          xlabel style={at=(current axis.right of origin), anchor=west},
          ylabel style={at=(current axis.above origin), anchor=south},
          enlarge y limits={rel=0.1}, enlarge x limits={rel=0.1}]
          \addplot[color=magenta, domain=0:1]{2};
          \addplot[color=blue, domain=0:1]{2*x};
          \addplot[color=magenta, domain=1:2]{2*x};
          \addplot[color=blue, domain=2:3]{-2*x^2 + 12*x -11};
          \addplot[color=magenta, domain=2:3]{-4*x + 12};
          \addplot[color=blue, domain=1:2]{x^2 + 1};
          \addplot[color=magenta, domain=3:5]{-x + 3};
          \addplot[color=blue, domain=3:5]{-x^2/2 + 3*x + 2.5};
          \addplot[color=magenta, domain=5:6]{-2};
          \addplot[color=blue, domain=5:6]{-2*x + 15};
          \addplot[color=magenta, domain=6:7]{2*x - 14};
          \addplot[color=blue, domain=6:7]{x^2 - 14*x + 51};
          \legend{$f(x)$, $g(x)$}
        \end{axis}
      \end{tikzpicture}
\end{enumerate}

\noindent Find the derivative of the function.
\begin{align*}
   \frac{\ud}{\ud x}\int_1^{\sqrt x}\frac{z^2}{z^4 + 1}\ud z
&= \frac{\ud}{\ud\sqrt x}\left(\int_1^{\sqrt x}\frac{z^2}{z^4 + 1}dz\right)
   \frac{\ud\sqrt x}{\ud x}\\
&= \frac{x}{x^2 + 1} \cdot \frac{1}{2\sqrt x}\\
&= \frac{\sqrt x}{2x^2 + 2}\tag{14}
\end{align*}

\begin{align*}
   \frac{\ud}{\ud x}\int_0^{\tan x}\sqrt{t + \sqrt t}\ud t
&= \frac{\ud}{\ud\tan x}\left(\int_0^{\tan x}\sqrt{t + \sqrt t}\ud t\right)
   \frac{\ud\tan x}{\ud x}\\
&= \frac{\sqrt{\tan x + \sqrt{\tan x}}}{\cos^2 x}\tag{15}
\end{align*}

\noindent\textbf{64. }Given the \textbf{error function}
\[\erf(x) = \frac{2}{\sqrt\pi}\int_0^x e^{-t^2}\ud t
  \Longrightarrow \erf'(x) = \frac{2e^{-x^2}}{\sqrt\pi}\]

\[\int_a^b e^{-t^2}\ud t = \frac{\sqrt\pi}{2}\int_a^b \erf'(t)\ud t
                      = \frac{\sqrt\pi}{2}[\erf(b) - \erf(a)]\tag{a}\]

With $y = e^{x^2}\erf(x)$
\begin{align*}
y' &= \left(e^{x^2}\right)'\erf(x) + e^{x^2}\erf'(x)\\
   &= 2xe^{x^2}\erf(x) + e^{x^2}\frac{2e^{-x^2}}{\sqrt\pi}\\
   &= 2xe^{x^2}\erf(x) + \frac{2}{\sqrt\pi}\\
   &= 2xy + \frac{2}{\sqrt\pi}\tag{b}
\end{align*}

\subsection{Infinite Integral}
\textbf{56. }Let $y(x)$ be the vertical postion at a distance of $x$ miles from
the start of the trail, then $y'(x) = f(x)$.
Thus, $\int_3^5 f(x)\ud x = y(5) - y(3)$, which is the vertical displacement from
3 to 5 miles.

\noindent\textbf{63. }Total mass of the rod:
\[\int_0^4 (9 + 2\sqrt x)\ud x
= \left.9x\right]_0^4 + \left.\frac{2\sqrt{x^3}}{3}\right]_0^4
= 36 + \frac{16}{3} = 41\frac{1}{3}\text{ (kg)}\]

\noindent\textbf{64. }Amount of water flowing from the tank during the first 10
minutes:
\[\int_0^{10} (200 - 4t)\ud t
= \left.200t\right]_0^{10} - \left.2t^2\right]_0^{10}
= 2000 - 200 = 1800\text{ (l)}\]

\subsection{The Substitution Rule}
\textbf{74. }Given $f(x) = \sin\sqrt[3]x$.

Since $f(-x) = \sin\sqrt[3]{-x} = \sin-\sqrt[3]x = -\sin\sqrt[3]x = -f(x)$,
$f$ is an odd function. Hence $\int_{-2}^3 f(x)\ud x = \int_2^3 f(x)\ud x$.

For $2 \leq x \leq 3$, $0 \leq \sqrt[3]2 \leq \sqrt[3]x \leq \sqrt[3]3 \pi$,
thus $\sin\sqrt[3]x \geq 0$ and $\int_2^3 f(x)\ud x \geq 0$. Futhermore,
$\sin\sqrt[3]x \leq 1$ so $\int_2^3 f(x)\ud x \leq \int_2^3 \ud x = 1$.

\noindent Evaluate the integral.
\begin{align*}
   \int_{-2}^2(x + 3)\sqrt{4 - x^2}\ud x
&= \int_{-2}^2 x\sqrt{4 - x^2}\ud x + 3\int_{-2}^2\sqrt{4 - x^2}\ud x\\
&= 0 + 3 \cdot 2\pi\\
&= 6\pi\tag{77}
\end{align*}

\begin{align*}
   \int_0^{24}\left(85 - 0.18\cos\frac{\pi t}{12}\right)\ud t
&= \left.85t\right]_0^{24}
 - \frac{54}{25\pi}\int_0^{24}\frac{\pi t}{12}'\cos\frac{\pi t}{12}\ud t\\
&= 2040 - \frac{54}{25\pi}\int_0^{2\pi}\cos x \ud x\\
&= 2040 - \left.\frac{54\sin x}{25\pi}\right]_0^{2\pi}\\
&= 2040\tag{80}
\end{align*}

\begin{align*}
   400 + \int_0^3 450.268e^{1.12567t}\ud t
&= 400 + 400\int_0^3 1.12567e^{1.12567t}\ud t\\
&= 400 + \left.400e^{1.12567t}\right]_0^3\\
&= 400e^{1.12567 \cdot 3}\\
&\approx 11713\tag{82}
\end{align*}

\section{Applications of Integration}
\subsection{Areas Between Curves}
Evaluate the integral

\begin{align*}
  \int_{-1}^1\left|e^x - x^2 + 1\right|\ud x
&=\int_{-1}^1\left(e^x - x^2 + 1\right)\ud x\\
&=\left[e^x - \frac{x^3}{3} + x\right]_{-1}^1\\
&=e - \frac{1}{3} + 1 - \frac{1}{e} - \frac{1}{3} + 1\\
&=e - \frac{1}{e} + \frac{4}{3}\tag{5}
\end{align*}

\begin{align*}
  \int_1^4\left|x^2 - 3x + 4\right|\ud x
&=\int_1^4\left(x^2 - 3x + 4\right)\ud x\\
&=\left[\frac{x^3}{3} - \frac{3x^2}{2} + 4x\right]_1^4\\
&=\frac{64 - 1}{3} - \frac{48 - 3}{2} + 16 - 4\\
&=\frac{21}{2}\tag{7}
\end{align*}

\begin{align*}
  \int_1^2\left|\frac{1}{x} - \frac{1}{x^2}\right|\ud x
&=\int_1^2\left(\frac{1}{x} - \frac{1}{x^2}\right)\ud x\\
&=\left[\frac{1}{3x^3} - \frac{1}{2x^2}\right]_1^2\\
&=\frac{1}{24} - \frac{1}{8} - \frac{1}{3} + \frac{1}{2}\\
&=\frac{1}{12}\tag{9}
\end{align*}

\noindent\textbf{53. }Find the values of $c$ such that
\begin{align*}
      \int_{-|c|}^{|c|}\left|x^2 - c^2 - c^2 + x^2\right|\ud x = 576
&\iff \int_{-|c|}^{|c|}\left(c^2 - x^2\right)\ud x = 288\\
&\iff \left[c^2 x - \frac{x^3}{3}\right]_{-|c|}^{|c|} = 288\\
&\iff \frac{4\left|c^3\right|}{3} = 288\\
&\iff |c| = 6\\
&\iff c = \pm 6
\end{align*}

\noindent\textbf{54. }Find the area of the region enclosed by the line $y = mx$
and the curve $y = \frac{x}{x^2 + 1}$.

Those two curves enclose a region if and only if the following equations has
two unique solutions, i.e. $m \in (0, 1)$
\[mx = \frac{x}{x^2 + 1} \iff mx^3 + (m - 1)x = 0
  \iff x \in \left\{0, \pm\frac{1 - m}{m}\right\}\]

The area of the region would then be
\begin{align*}
A&=\int_{\frac{m-1}{m}}^{\frac{1-m}{m}}\left|mx - \frac{x}{x^2 + 1}\right|\ud x\\
 &=\int_{\frac{m-1}{m}}^0\left(\frac{x}{x^2 + 1} - mx\right)\ud x +
   \int_0^{\frac{1-m}{m}}\left(mx - \frac{x}{x^2 + 1}\right)\ud x\\
 &=\int_{\frac{m-1}{m}}^0\left(\frac{x}{x^2 + 1} - mx\right)\ud x +
   \int_0^{\frac{1-m}{m}}\left(mx - \frac{x}{x^2 + 1}\right)\ud x\\
 &=\int_{\frac{m-1}{m}}^0\frac{1}{x^2 + 1}\cdot\frac{\ud x^2}{\ud x}\ud x -
   \left.\frac{mx^2}{2}\right]_{\frac{m-1}{m}}^0 +
   \left.\frac{mx^2}{2}\right]_0^{\frac{1-m}{m}} -
   \int_0^{\frac{1-m}{m}}\frac{1}{x^2 + 1}\cdot\frac{\ud x^2}{\ud x}\ud x\\
 &=\frac{(m - 1)^2}{m} +
   2\int_{\left(\frac{m-1}{m}\right)^2}^0\frac{1}{x + 1}\ud x\\
 &=\frac{m^2 - 2m + 1}{m} +
   2\left.\ln(|x + 1|)\right]_{\frac{m^2 - 2m + 1}{m^2}}^0\\
 &=\frac{m^2 - 2m + 1}{m} -
   2\ln\left(\frac{2m^2 - 2m + 1}{m^2}\right)
\end{align*}

\subsection{Volumes}
Evaluate the integral

\begin{align*}
  \int_1^2\pi\left(2 - \frac{x}{2}\right)^2 \ud x
&=\pi\int_1^2\left(4 - x + \frac{x^2}{4}\right)\ud x\\
&=\pi\left[4x - x^2 + \frac{x^3}{12}\right]_1^2\\
&=\pi\left(4 - 3 + \frac{7}{12}\right)\\
&=\frac{19\pi}{12}\tag{1}
\end{align*}

\[\int_1^5\pi(x - 1)\ud x
= \pi\left[\frac{x^2}{2} - x\right]_1^5
= \pi(12 - 4)
= 8\pi\tag{3}\]

\[\int_0^9 4\pi y \ud y = \left.2\pi y^2\right]_0^9 = 162\pi\tag{5}\]

\[\int_0^1\pi\left|x^2 - x^6\right|\ud x
= \pi\left[\frac{x^3}{3} - \frac{x^7}{7}\right]_0^1
= \frac{4\pi}{21}\tag{7}\]

\begin{align*}
   \int_{-2}^2\pi\left|\frac{x^4}{16} - 25 + 10x^2 - x^4\right|\ud x
&= 2\pi\int_0^2\left(\frac{15x^4}{16} - 10x^2 + 25\right)\ud x\\
&= 2\pi\left[-\frac{10x^3}{3} + 25x + \frac{3x^5}{16}\right]_0^2\\
&= \frac{88\pi}{3}\tag{8}
\end{align*}

\begin{align*}
   \int_0^1\pi\left|\left(\sqrt x - 1\right)^2 - \left(x^2 - 1\right)^2\right|\ud x
&= \int_0^1\pi\left|x - 2\sqrt x - x^4 + 2x^2\right|\ud x\\
&= \int_0^1\pi\left(x^4 - 2x^2 - x + 2\sqrt x\right)\ud x\\
&= \pi\left[\frac{x^5}{5} - \frac{2x^3}{3}
            - \frac{x^2}{2} + \frac{4\sqrt x^3}{3}\right]_0^1\\
&= \pi\left(\frac{1}{5} - \frac{2}{3} - \frac{1}{2} + \frac{4}{3}\right)\\
&= \frac{11\pi}{30}\tag{11}
\end{align*}

\begin{align*}
   \int_0^h\left(a + \frac{x}{h}(b-a)\right)^2\ud x
&= \int_0^h\left(a^2 - \frac{ax}{h}(b-a) + \frac{x^2}{h^2}(b-a)^2\right)\ud x\\
&= \left[a^2 x - \frac{ax^2}{2h}(b-a) + \frac{x^3}{3h^2}(b-a)^2\right]_0^h\\
&= ha^2 - \frac{ha}{2}(b-a) + \frac{h}{3}(b-a)^2\\
&= ha^2 - \frac{hab}{2} + \frac{ha^2}{2}
 + \frac{ha^2}{3} - \frac{2hab}{3} + \frac{hb^2}{3}\\
&= \frac{11ha^2 - 7hab + 2hb^2}{6}\tag{50}
\end{align*}

\begin{align*}
A&=\int_{-r}^r\left(\pi\left(R + \sqrt{r^2 - x^2}\right)^2 -
                    \pi\left(R - \sqrt{r^2 - x^2}\right)^2\right)\ud x\\
 &=\int_{-r}^r 4\pi R\sqrt{r^2 - x^2}\ud x\\
 &=2\pi R\int_{-r}^r 2\sqrt{r^2 - x^2}\ud x\\
 &=2\pi R \cdot \pi r^2\\
 &=2\pi^2 R r^2\tag{61}
\end{align*}

\setcounter{subsection}{3}
\subsection{Work}
\textbf{7. }Spring constant: $k = F(4) / 4 = 10g / 4 = 2.5g$ (lbf/in)

Work done by stretching the spring from its natural length to 6 in:
\[\int_0^6 kx\ud x = \left.k\frac{x^2}{2}\right]_0^6 = 18k = 45g\text{ (lbf.in)}\]

\noindent\textbf{9. }Suppose that 2 J of work is needed to stretch a spring from its
natural length of 30 cm to a length of 42 cm.

\begin{align*}
      \int_0^{0.12}kx\ud x = 2
&\iff \left.\frac{kx^2}{2}\right]_0^{0.12} = 2\\
&\iff 0.0072k = 2\\
&\iff k = \frac{2500}{9}\text{ (N/m)}
\end{align*}

\begin{align*}
  \int_{0.05}^{0.1}kx\ud x
&=\int_{0.05}^{0.1}\frac{2500x}{9}\ud x\\
&=\left.\frac{1250x^2}{9}\right]_{0.05}^{0.1}\\
&=\frac{1250(0.01 - 0.0025)}{9}\\
&=\frac{25}{24}\text{ (J)}\tag{a}
\end{align*}

\[x = \frac{F}{k} = \frac{30 \cdot 9}{2500} = \frac{27}{250}\text{ (m)}\tag{b}\]

\noindent Evaluate the integral
\[\int_0^{50}mgx\ud x = \left.\frac{25gx^2}{2}\right]_0^{50} = 31250g\text{ (ft.lbf)}\tag{13}\]

\begin{align*}
W&=\lim_{n \to \infty}\sum_{i = 1}^n F_i\frac{3i}{n}\\
 &=\lim_{n \to \infty}\sum_{i = 1}^n m_i g\frac{3i}{n}\\
 &=\lim_{n \to \infty}\sum_{i = 1}^n A_i\frac{3}{n}\rho g\frac{3i}{n}\\
 &=\lim_{n \to \infty}\sum_{i = 1}^n
   3\left(1 - \frac{i}{n}\right)8\left(1 - \frac{i}{n}\right)
   1000g\frac{3i}{n}\cdot\frac{3}{n}\\
 &=\lim_{n \to \infty}\sum_{i = 1}^n
   \frac{80000}{3}\left(3 - \frac{3i}{n}\right)^2\frac{3i}{n}\cdot\frac{3}{n}\\
 &=\int_0^3 \frac{80000}{3}\left(9x - 6x^2 + x^3\right)\ud x\\
 &=\frac{80000}{3}\left[\frac{9x^2}{2} - 2x^3 + \frac{x^4}{4}\right]_0^3\\
 &=180000\text{ (J)}\tag{21}
\end{align*}

\begin{align*}
W&=\lim_{n \to \infty}\sum_{i = 1}^n m_i g\frac{8i}{n}\\
 &=\lim_{n \to \infty}\sum_{i = 1}^n A_i\frac{8}{n}\rho g\frac{8i}{n}\\
 &=\lim_{n \to \infty}\sum_{i = 1}^n
   \pi\left(6 - 3\frac{i}{n}\right)^2 62.5g\frac{8i}{n}\cdot\frac{8}{n}\\
 &=\lim_{n \to \infty}\sum_{i = 1}^n
   \frac{1125\pi g}{128}\left(16 - \frac{8i}{n}\right)^2\frac{8i}{n}\cdot\frac{8}{n}\\
 &=\frac{1125\pi g}{128}\int_0^8\left(16 - x\right)^2 x\ud x\\
 &=\frac{1125\pi g}{128}\left[128x^2 - \frac{32x^3}{3} + \frac{x^4}{4}\right]_0^8\\
 &=33000\pi g\text{ (ft.lbf)}\tag{23}
\end{align*}

\subsection{Average Value of a Function}
\textbf{9. }Given the function $f(x) = (x - 3)^2$ on $[2, 5]$.
\[f_{ave}
= \frac{1}{5 - 2}\int_2^5 (x - 3)^2 \ud x
= \frac{1}{3}\left[\frac{x^3}{3} - 3x^2 + 9x\right]_2^5
= 1\tag{a}\]

Since $x \in [2, 5]$,
\[f(c) = f_{ave} \iff (c - 3)^2 = 1 \iff c = 4\tag{b}\]

\begin{tikzpicture}
  \begin{axis}[
    axis x line=middle, axis y line=middle,
    xlabel={$x$}, ylabel={$y$}, area style,
    xlabel style={at=(current axis.right of origin), anchor=west},
    ylabel style={at=(current axis.above origin), anchor=south}]
    \addplot[domain=2:5,blue]{x^2 - 6*x + 9};
    \addplot[domain=2:5]{1};
  \end{axis}
\end{tikzpicture}

\noindent\textbf{12. }Given $f(x) = 2\sin x - \sin 2x$ on $[0, \pi]$.
\begin{align*}
f_{ave} &= \frac{1}{\pi}\int_0^\pi\left(2\sin x - \sin 2x\right)\ud x\\
        &= \frac{1}{\pi}\left[\frac{\cos 2x}{2} - 2\cos x\right]_0^\pi\\
        &= \frac{4}{\pi}
\end{align*}

$f(c) = f_{ave} \iff 2\sin x - \sin 2x = \frac{4}{\pi}$, i.e.
$x \approx 1.24$ or $x \approx 2.81$ on $[0, \pi]$.

\noindent\textbf{13. }Since $f$ is continuous, apply Mean Value Theorem on
$[1, 3]$,

\[\exists c \in [1, 3], f(c) = \frac{1}{3 - 1}\int_1^3 f(x)\ud x
= \frac{8}{2} = 4\]

\section{Techniques of Integration}
\subsection{Integration by Parts}
Evaluate the integral
\begin{align*}
   \int x \cos 5x \ud x
&= \int\frac{x}{5} \ud\sin 5x\\
&= \frac{x\sin 5x}{5} - \int\sin 5x \ud\frac{x}{5}\\
&= \frac{x\sin 5x}{5} + \frac{x\cos 5x}{25} + C\tag{3}
\end{align*}

\begin{align*}
   \int e^{2\theta}\sin 3\theta \ud\theta
&= \int\frac{\sin 3\theta}{2} \ud e^{2\theta}\\
&= \frac{e^{2\theta}\sin 3\theta}{2}
 - \int\frac{3}{2}e^{2\theta}\cos 3\theta \ud\theta\\
&= \frac{e^{2\theta}\sin 3\theta}{2}
 - \frac{3}{2}\int\frac{\cos 3\theta}{2} \ud e^{2\theta}\\
&= \frac{e^{2\theta}\sin 3\theta}{2}
 - \frac{3e^{2\theta}\cos 3\theta}{4}
 - \frac{9}{4}\int e^{2\theta}\sin 3\theta \ud\theta\\
&= \frac{4}{13}e^{2\theta}
   \left(\frac{\sin 3\theta}{2} - \frac{\cos 3\theta}{4}\right) + C\\
&= \frac{e^{2\theta}}{13}(2\sin 3\theta - 3\cos 3\theta) + C\tag{17}
\end{align*}

\setcounter{section}{8}
\section{Differential Equations}
\setcounter{subsection}{2}
\subsection{Separable Equations}
Solve the equation.
\begin{align*}
  \leibniz{y}{x} = xy^2
  &\iff y^{-2}\ud y = x\ud x\\
  &\;\Longrightarrow \int\frac{\ud y}{y^2} = \int x\ud x
  \qquad\text{(for $y\neq 0$)}\\
  &\iff C_y - \frac{1}{y} = C_x + \frac{x^2}{2}\\
  &\iff \frac{C - x^2}{2} = \frac{1}{y}\\
  &\iff y = \frac{2}{C - x^2}\tag{1}
\end{align*}
\begin{align*}
  (y + \sin y)\leibniz{y}{x} = x + x^3
  &\iff \int(y + \sin y)\ud y = \int(x + x^3)\ud x\\
  &\iff \frac{y^2}{2} - \cos y = \frac{x^2}{2} + \frac{x^4}{4} + C\tag{5}\\
\end{align*}
\begin{align*}
  \leibniz{y}{t} = \frac{t}{y\exp(y + t^2)}
  &\iff \int ye^y\ud y = \int te^{-t^2}\ud t\\
  &\iff (y - 1)e^y = C - \frac{1}{2e^{t^2}}\tag{7}
\end{align*}
\begin{align*}
  \leibniz{u}{t} = \frac{2t + \sec^2 t}{2u}
  &\iff \int 2u\ud u = \int(2t + \sec^2 t)\ud t\\
  &\iff u^2 = t^2 + \tan t + C\tag{13}
\end{align*}
Since $u(0) = -5$, $u = -\sqrt{t^2 + \tan t + 25}$.
\begin{align*}
  \leibniz{y}{x} = xy
  &\iff \int\frac{\ud y}{y} = \int x\ud x\qquad\text{(since $y\neq 0$)}\\
  &\iff \ln|y| = \frac{x^2}{2} + C\\
  &\iff |y| = \exp\left(\frac{x^2}{2} + C\right)\tag{19}
\end{align*}
Since $y(0) = 1$, $y = \exp(x^2/2)$.\pagebreak
\begin{align*}
  y(x) = 2 + \int_2^x[t - ty(t)]\ud t
  &\;\Longrightarrow y - 2 = \int(x - xy)\ud x\\
  &\iff \leibniz{(y - 2)}{x} = x - xy\\
  &\iff \int\frac{\ud y}{1 - y} = \int x\ud x\\
  &\iff C - \frac{x^2}{2} = \ln|1 - y|\\
  &\iff y = 1 \pm \exp\left(C - \frac{x^2}{2}\right)\tag{33}
\end{align*}
Since $y(2) = 2$ (which can be trivially obtained from the original condition),
$y = 1 + \exp(2 - x^2/2)$.

\subsection{Models for Population Growth}
\textbf{3.} The Pacific halibut fishery has been modeled
by the differential equation
\begin{align*}
  \leibniz{y}{t} = ky\left(1 - \frac{y}{M}\right)
  &\;\Longrightarrow \int\left(\frac{1}{y} + \frac{1}{M-y}\right)\ud y
  = \int k\ud t\\
  &\iff \ln|y| - \ln|M - y| = kt + C\\
  &\iff \left|\frac{M}{y} - 1\right| = e^{-kt-C}\\
  &\iff \frac{M}{y} = 1 \pm e^{-kt-C}\\
  &\iff y = \frac{M}{1 \pm e^{-kt-C}}
  \tag{$*$}
\end{align*}

As $M = 8\times 10^7$, $k = 0.71$ and $y(0) = 2\times 10^7$,
from $(*)$ we get $\pm e^{-C} = 3$ and thus
\[y = \frac{M}{1 + 3e^{-kt}}\]

For $t = 1$, $y \approx 3.2\times 10^7$.
For $y = 4\times 10^7$, $t = (\ln 3)/0.71$.

\noindent\textbf{5.} Suppose a population grows according to a logistic model
\[\leibniz{P}{t} = kP\left(1 - \frac{P}{M}\right)
\iff P(t) = \frac{M}{1 \pm e^{-kt-C}}\]
with initial population $P(0) = 1000$ and carrying capacity $M = 10000$.

Suppose $P(1) = 2500$,
\[\begin{dcases}
  \frac{10000}{1\pm e^{-C}} &= 1000\\
  \frac{10000}{1\pm e^{-k-C}} &= 2500
\end{dcases}
\iff\begin{cases}
  \pm e^{-C} &= 9\\
  \pm e^{-k-C} &= 3
\end{cases}
\iff\begin{cases}
  \pm := +\\
  C = -\ln 9\\
  k = \ln 3
\end{cases}\]

After another 3 years, the population will be
\[P(3) = \left(t\mapsto\frac{10000}{1+3^{2-t}}\right)(1+3) = 9000\]

\subsection{Linear Equations}
Solve the differential equation.
\begin{align*}
  \leibniz{y}{x} + y = x
  &\iff e^x\leibniz{y}{x} + y\leibniz{e^x}{x} = xe^x\\
  &\iff \int\ud ye^x = \int x\ud e^x\\
  &\iff ye^x = e^x(x - 1) + C\\
  &\iff y = x - 1 + Ce^{-x}\tag{7}
\end{align*}
\begin{align*}
  x\leibniz{y}{x} + y = \sqrt x
  &\iff \int\ud xy = \int\sqrt x\ud x\\
  &\iff xy = \frac{2x\sqrt x}{3} + C\\
  &\iff y = \frac{2\sqrt x}{3} + \frac{C}{x}\tag{9}
\end{align*}
\begin{align*}
  x^2\leibniz{y}{x} + 2xy = \ln x
  &\iff \int\ud yx^2 = \int\ln x\ud x\\
  &\iff yx^2 = x(\ln x - 1) + C\\
  &\iff y = \frac{\ln x - 1}{x} + \frac{C}{x^2}
\end{align*}
Since $y(1) = 2$, $C = 3$.
\begin{align*}
  L\leibniz{I}{t} + RI = \mathcal E
  &\iff e^{Rt/L}\left(\leibniz{I}{t} + \frac{R}{L}I\right)
  = \frac{\mathcal E}{L}e^{Rt/L}\\
  &\iff \int\ud Ie^{Rt/L} = \frac{\mathcal E}{L}\int e^{Rt/L}\ud t\\
  &\iff Ie^{Rt/L} = \frac{\mathcal E}{R}e^{Rt/L} + C\\
  &\iff I = \frac{\mathcal E}{R} + \frac{C}{\exp(Rt/L)}
\end{align*}
Since $\mathcal E = 40$ V, $L = 2$ H, $R = 10\,\Omega$ and $I(0) = 0$,
$I(t) = 4 - 4/\exp 5t$ and $I(0.1) = 4 - 4/\sqrt e$.

\allowdisplaybreaks
\setcounter{section}{10}
\section{Lazy Evaluation}
\setcounter{subsection}{2}
\subsection{The Integral Test and Estimates of Sums}
\textbf{34. }Using Leonhard Euler's calculation of the exact sum of
the $p$-series with $p = 2$:
\[\zeta(2) = \sum_{n=1}^\infty\frac{1}{n^2}
= \lim_{n\to\infty}\sum_{i=1}^n\frac{1}{i^2} = \frac{\pi^2}{6}\]
\[\sum_{n=2}^\infty\frac{1}{n^2} = \lim_{n\to\infty}\sum_{i=2}^n\frac{1}{i^2}
= \lim_{n\to\infty}\sum_{i=1}^n\frac{1}{i^2} - \frac{1}{1^2}
= \frac{\pi^2}{6} - 1\tag{a}\]
\[\sum_{n=3}^\infty\frac{1}{(n + 1)^2}
= \lim_{n\to\infty}\sum_{i=4}^{n+1}\frac{1}{i^2}
= \lim_{n\to\infty}\sum_{i=1}^n\frac{1}{i^2} - \sum_{i=1}^3\frac{1}{i^2}
= \frac{\pi^2}{6} - \frac{49}{36}\tag{b}\]
\[\sum_{n=1}^\infty\frac{1}{(2n)^2}
= \lim_{n\to\infty}\sum_{i=1}^n\frac{1}{4i^2}
= \frac{1}{4}\lim_{n\to\infty}\sum_{i=1}^n\frac{1}{i^2}
= \frac{\pi^2}{24}\tag{c}\]

\noindent Determine if the series is convergent or divergent using
the Integral Test.

\[\sum_{n=2}^\infty \frac{1}{n(\ln n)^2}\tag{22}\]

\begin{align*}
  \int_2^\infty\frac{1}{x(\ln x)^2}
&= \lim_{t\to\infty}\int_2^t\frac{1}{x(\ln x)^2}\ud x\\
&= \lim_{t\to\infty}\int_2^t\frac{1}{(\ln x)^2}\ud\ln x\\
&= \lim_{t\to\infty}\int_{\ln 2}^{\ln t}\frac{1}{x^2}\ud x\\
&= \lim_{t\to\infty}\left.\frac{-1}{x}\right]_{\ln 2}^{\ln t}\\
&= \lim_{t\to\infty}\left(\frac{1}{\ln{2}} - \frac{1}{\ln t}\right)\\
&= \frac{1}{\ln 2}
\end{align*}

Thus by the Integral Test, the given series is convergent.

\[\sum_{n=3}^\infty\frac{n^2}{e^n}\tag{24}\]

\begin{align*}
  \int_3^\infty\frac{x^2}{e^x}
&= \lim_{t\to\infty}\int_3^t\frac{x^2}{e^x}\ud x\\
&= \lim_{t\to\infty}\int_3^t -x^2 \ud e^{-x}\\
&= \lim_{t\to\infty}\left(\int_3^t e^{-x}\ud x^2
 - \left.x^2 e^{-x}\right]_3^t\right)\\
&= \lim_{t\to\infty}\left(-\int_3^t 2x\ud e^{-x}
 + \left.\frac{x^2}{e^x}\right]_t^3\right)\\
&= \lim_{t\to\infty}\left(\int_3^t e^{-x}\ud 2x
 + \left[\frac{2x}{e^x} + \frac{x^2}{e^x}\right]_t^3\right)\\
&= \lim_{t\to\infty}\left(2\int_3^t e^{-x}\ud x
 + \left.\frac{2x + x^2}{e^x}\right]_t^3\right)\\
&= \lim_{t\to\infty}\left.\frac{2 + 2x + x^2}{e^x}\right]_t^3\\
&= \lim_{t\to\infty}\left(\frac{17}{e^3} - \frac{2 + 2t + t^2}{e^t}\right)\\
&= \frac{17}{e^3}
\end{align*}

Thus by the Integral Test, the given series is convergent.\pagebreak

\[\sum_{n=1}^\infty\frac{\cos(\pi n)}{\sqrt n}\tag{27}\]

Since $x \mapsto \cos(\pi x)/\sqrt n$ is neither positive
(e.g. $\cos 3\pi/\sqrt 3 = -1$) nor ultimately decreasing, the Integral Test
cannot be used to determine whether the series is convergent.

\subsection{The Comparison Test}
Determine whether the series is convergent or divergent.
\[\sum_{n=1}^\infty\frac{\sqrt{n^4 + 1}}{n^3 + n^2}\tag{25}\]

We use the Limit Comparison Test with
\[a_n = \frac{\sqrt{n^4 + 1}}{n^3 + n^2} \qquad b_n = \frac{1}{n}\]
and obtain
\[\lim_{n\to\infty}\frac{a_n}{b_n}
= \lim_{n\to\infty}\frac{\sqrt{n^4 + 1}}{n^2 + n}
= \lim_{n\to\infty}\frac{\sqrt{1 + \frac{1}{n^4}}}{1 + \frac{1}{n}}
= 1 > 0\]

Since this limit exists and $\sum\frac{1}{n}$ is divergent ($p$-series with
$p = 1$), the given series diverges by the Limit Comparison Test.

\[\sum_{n=1}^\infty\frac{1}{n!}\tag{29}\]
\[\lim_{n\to\infty}\frac{1/(n+1)!}{1/n!}
= \lim_{n\to\infty}\frac{1}{n + 1} = 0 < 1\]

Thus by the Ratio Test, the given series is absolutely convergent.

\[\sum_{n=1}^\infty\frac{n!}{n^n}\tag{30}\]
\[\frac{n!}{n^n} = \frac{2}{n^2}\cdot\frac{n!}{2n^{n-2}} \leq \frac{2}{n^2}\]

Since both $\sum n!/n^n$ and $\sum 2/n^2$ are series with positive terms and
$\sum 2/n^2$ converges because it is a constant time of $p$-series with
$p = 2$, by the Comparison Test, $\sum n!/n^n$ is convergent.

\[\sum_{n=1}^\infty\frac{1}{n\sqrt[n]n}\tag{32}\]

We use the Limit Comparison Test with
\[a_n = \frac{1}{n\sqrt[n]n} \qquad b_n = \frac{1}{n}\]
and obtain
\[\lim_{n\to\infty}\frac{a_n}{b_n} = \lim_{n\to\infty}\frac{1}{\sqrt[n]n}\]

Since $\frac{1}{\sqrt[n]n}' = \frac{1 - \ln n}{n^2\sqrt[n]n}$ is negative on
$(e, \infty)$, $n \mapsto \frac{1}{\sqrt[n]n}$ is ultimately decreasing.
Additionally, $\frac{1}{\sqrt[n]n} \geq \frac{1}{\sqrt[n]1} = 1$ on this
interval, thus
\[\lim_{n\to\infty}\frac{1}{\sqrt[n]n}
= \inf\left\{\frac{1}{\sqrt[n]n}: n \in \mathbb N_3\right\} = 1 > 0\]

Therefore, the given series diverges by the Limit Comparison Test,
as $\sum\frac{1}{n}$ is divergent ($p$-series with $p = 1$).

\subsection{Alternating Series}
Test the series for convergence or divergence.

\[\sum_{n=1}^\infty (-1)^n\frac{n^n}{n!}\tag{19}\]

Since $\frac{(n + 1)^{n + 1}}{(n + 1)!} > \frac{n^n}{n!}$, the given
alternating series diverges.

\[\sum_{n=1}^\infty (-1)^n\left(\sqrt{n + 1} - \sqrt n\right)\tag{20}\]

For all $n$,
\begin{align*}
      n^2 + 2n < n^2 + 2n + 1
&\iff \sqrt{n(n + 2)} < n + 1\\
&\iff n + \sqrt{n(n + 2)} + n + 2 < 4n + 4\\
&\iff \sqrt{n + 2} + \sqrt n < 2\sqrt{n + 1}\\
&\iff \sqrt{n + 2} - \sqrt{n + 1} < \sqrt{n + 1} - \sqrt n\tag{i}
\end{align*}
\[\lim_{n\to\infty}\left(\sqrt{n + 1} - \sqrt n\right)
= \lim_{n\to\infty}\frac{1}{\sqrt{n + 1} + \sqrt n}
= \lim_{n\to\infty}\frac{1/\sqrt n}{\sqrt{1 + 1/\sqrt{n}} + 1}
= 0\tag{ii}\]

Thus, by the Alternating Series Test, the given series is convergent.

\subsection{Absolute Convergence}
Determine whether the series is absolutely convergent,
conditionally convergent, or divergent.

\[\sum_{n=2}^\infty\left(\frac{-2n}{n + 1}\right)^{5n}\tag{22}\]
\[\lim_{n\to\infty}\sqrt[n]{\left(\frac{2n}{n + 1}\right)^{5n}}
= \lim_{n\to\infty}\left(\frac{2n}{n + 1}\right)^5\\
= \left(\lim_{n\to\infty}\frac{2}{1 + 1/n}\right)^5\\
= 32 > 1\]

Thus the given series diverges by the Root Test.

\[\sum_{n=1}^\infty\prod_{i=1}^n\frac{2i}{3i + 2}\tag{30}\]
\[\lim_{n\to\infty}\frac{\prod_{i=1}^{n+1}\frac{2i}{3i + 2}}
                        {\prod_{i=1}^n\frac{2i}{3i + 2}}
= \lim_{n\to\infty}\frac{2n + 2}{3n + 5}
= \lim_{n\to\infty}\frac{2 + 2/n}{3 + 5/n}
= \frac{2}{3} < 1\]

Thus by the Ratio Test, the given series is absolutely convergent.

\setcounter{subsection}{7}
\subsection{Power Series}
Find the radius of convergence and the interval of convergence of the series.

\[\sum_{n=0}^\infty (-1)^n\frac{x^{2n+1}}{(2n + 1)!}\tag{14}\]

Let $a_n = (-1)^n x^{2n+1}/(2n + 1)!$,
\[\lim_{n\to\infty}\left|\frac{a_{n+1}}{a_n}\right|
= \lim_{n\to\infty}\left|\frac{x^2}{4n^2 + 10n + 6}\right|
= 0 < 1\]

Thus by the Ratio Test, the series is convergent for all $x$ and the radius of
convergence is $R = \infty$.

\[\sum_{n=1}^\infty\frac{3^n (x+4)^n}{\sqrt n}\tag{17}\]

Let $a_n = 3^n (x+4)^n / \sqrt n$,
\[\lim_{n\to\infty}\left|\frac{a_{n+1}}{a_n}\right|
= \lim_{n\to\infty}\left|\frac{3(x + 4)\sqrt n}{\sqrt{n + 1}}\right|
= 3|x + 4|\]

Using the Ratio Test, we see that the series converges if $|x + 4| < 1/3$ and
it diverges if $|x + 4| > 1/3$, thus the radius of convergence is $R = 1/3$.

When $|x + 4| = 1/3$, the series is either $\sum (-3)^n / \sqrt n$ or
$\sum 3^n / \sqrt n$, both of which diverge by the Test for Divergence.
Therefore the interval of convergence is $(-13/3, -11/3)$.

\[\sum_{n=1}^\infty\frac{b^n}{\ln n}(x - a)^n,\qquad b > 0\tag{22}\]

Let $a_n = b^n (x - a)^n / \ln n$,
\[\lim_{n\to\infty}\left|\frac{a_{n+1}}{a_n}\right|
= \lim_{n\to\infty}\left|\frac{b(x - a)\ln n}{\ln(n + 1)}\right|
= b|x - a|\]

Using the Ratio Test, we see that the series converges if $|x - a| < b^{-1}$
and it diverges if $|x - a| > b^{-1}$, thus the radius of convergence is
$R = b^{-1}$.

When $|x - a| = b^{-1}$, the series is $\sum (\pm b)^n / \ln n$, which diverges
by the Test for Divergence. Therefore the interval of convergence is
$(a - b^{-1}, a + b^{-1})$.

\[\sum_{n=2}^\infty\frac{x^{2n}}{n(\ln n)^2}\tag{26}\]

Let $a_n = x^{2n} / n / (\ln n)^2$,
\[\lim_{n\to\infty}\left|\frac{a_{n+1}}{a_n}\right|
= \lim_{n\to\infty}\left|\frac{x^2 n \ln^2 n}{(n + 1)\ln^2(n + 1)}\right|
= x^2\]

Using the Ratio Test, we see that the series converges if $|x| < 1$ and it
diverges if $|x| > 1$, therefore the radius of convergence is $R = 1$.

When $x = \pm 1$, $a_n = n^{-1}/(\ln n)^2$, which is defined by a continuous,
positive and decreasing function $x \mapsto x^{-1}/(\ln x)^2$ on $[2, \infty)$.
\begin{align*}
   \int_2^\infty\frac{1}{x(\ln x)^2}\ud x
&= \lim_{t\to\infty}\int_2^t\frac{1}{(\ln x)^2}\ud\ln x\\
&= \lim_{t\to\infty}\int_{\ln 2}^{\ln t}\frac{1}{x^2}\ud x\\
&= \lim_{t\to\infty}\left.\frac{1}{3x^3}\right]_{\ln t}^{\ln 2}\\
&= \frac{1}{3(\ln 2)^3}
\end{align*}

By the Integral Test, $\sum_{n=2}^\infty n^{-1}/(\ln n)^2$ converges, and thus
the interval if of convergence of the given power series is $[-1, 1]$.

\subsection{Representations of Functions as Power Series}
Find a power series representation for the function and determine the interval
of convergence.

\[f(x) = \frac{5}{1 - 4x^2}
       = 5\sum_{n=0}^\infty\left(4x^2\right)^n
       = \sum_{n=0}^\infty 5 \cdot 2^{2n} x^{2n}\tag{4}\]

Interval of convergence is $(-1, 1)$.

\[f(x) = \frac{x^2}{a^3 - x^3}
       = \frac{x^2}{a^3}\sum_{n=0}^\infty\left(\frac{x}{a}\right)^{3n}
       = \sum_{n=0}^\infty\frac{x^{3n + 2}}{a^{3n + 3}}\tag{10}\]

Interval of convergence is $(-a, a)$.

\begin{align*}
f(x) &= \frac{x + 2}{2x^2 - x - 1}\\
     &= \frac{1}{x - 1} - \frac{1}{2x + 1}\\
     &= -\sum_{n=0}^\infty x^n - \sum_{n=0}^\infty (-2x)^n\\
     &= \sum_{n=0}^\infty (-1 - (-2)^n)x^n\tag{12}
\end{align*}

Interval of convergence is $(-1, 1) \cap (-1/2, 1/2) = (-1/2, 1/2)$.

\noindent\textbf{40. }Find the sum of the series when $|x| < 1$.
\begin{align*}
   \sum_{n=1}^\infty nx^{n - 1}
&= \sum_{n=1}^\infty x^{n - 1} + \sum_{n=1}^\infty (n - 1)x^{n - 1}\\
&= \sum_{n=0}^\infty x^n + \sum_{n=0}^\infty nx^n\\
&= \frac{1}{1 - x} + x\sum_{n=1}^\infty nx^{n - 1}\\
&= \frac{1}{(1 - x)^2}\tag{a}
\end{align*}

\[\sum_{n=1}^\infty nx^n
= x\sum_{n=1}^\infty nx^{n - 1}
= \frac{x}{(1 - x)^2}\tag{b.i}\]

\[\sum_{n=1}^\infty \frac{n}{2^n}
= \left(x \mapsto \frac{x}{(1 - x)^2}\right)\left(\frac{1}{2}\right)
= 2\tag{b.ii}\]

\begin{align*}
   \sum_{n=2}^\infty n(n - 1)x^n
&= \sum_{n=2}^\infty 2(n - 1)x^n + \sum_{n=2}^\infty (n - 1)(n - 2)x^n\\
&= 2\sum_{n=1}^\infty (n - 1)x^n + x\sum_{n=1}^\infty n(n - 1)x^n\\
&= 2\left(\sum_{n=1}^\infty nx^n + 1 - \sum_{n=0}^\infty x^n\right) : (1 - x)\\
&= 2\left(\frac{x}{(1 - x)^2} + 1 - \frac{1}{1 - x}\right) : (1 - x)\\
&= \frac{2x^2}{(1 - x)^3}\tag{c.i}
\end{align*}

\[\sum_{n=2}^\infty \frac{n^2 - n}{2^n}
= \left(x \mapsto \frac{2x^2}{(1 - x)^3}\right)\left(\frac{1}{2}\right)
= 4\tag{c.ii}\]

\[\sum_{n=1}^\infty \frac{n^2}{2^n}
= \sum_{n=2}^\infty \frac{n^2 - n}{2^n} + \sum_{n=1}^\infty nx^n
= 4 + 2 = 6\tag{c.iii}\]

\subsection{Taylor and Maclaurin Series}
Find the Taylor series for $f$ centered at the given value of $a$ and the
associative radius of convergence.

\[f(x) = \ln x,\qquad a = 2\tag{15}\]
\begin{align*}
   f(x)
&= \sum_{n=0}^\infty\frac{f^{(n)}(a)}{n!}(x - a)^n\\
&= \ln 2 + \sum_{n=1}^\infty\left(x\mapsto\binom{-1}{n-1}\frac{1}{nx^n}\right)
                            (2)\cdot(x - 2)^n\\
  &= \ln 2 + \sum_{n=1}^\infty(-1)^{n-1}\frac{(x - 2)^n}{n2^n}
\end{align*}

Let $a_n = (-1)^{n-1}(x - 2)^n/(n2^n)$,
\[\lim_{n\to\infty}\left|\frac{a_{n+1}}{a_n}\right|
= \lim_{n\to\infty}\frac{n}{2n + 2}|x - 2|
= \frac{|x - 2|}{2}\]

Using the Ratio Test, we see $f(x) = \ln 2 + \sum a_n$ converges if
$|x - 2| < 2$ and it diverges if $|x - 2| > 2$, therefore the associative
radius of convergence is $R = 2$.

\[f(x) = \frac{1}{x},\qquad a = -3\tag{16}\]
\begin{align*}
   f(x)
&= \sum_{n=0}^\infty\frac{f^{(n)}(a)}{n!}(x - a)^n\\
&= \sum_{n=0}^\infty
   \left(x\mapsto\binom{-1}{n}\frac{1}{x^{n+1}}\right)(-3)\cdot(x + 3)^n\\
&= \sum_{n=0}^\infty(-1)^n\frac{(x + 3)^n}{(-3)^{n+1}}\\
&= \sum_{n=0}^\infty\frac{-(x + 3)^n}{3^{n+1}}
\end{align*}

Let $a_n = (x + 3)^n/3^{n+1}$,
\[\lim_{n\to\infty}\left|\frac{a_{n+1}}{a_n}\right|
= \lim_{n\to\infty}\frac{|x + 3|}{3}
= \frac{|x + 3|}{3}\]

Using the Ratio Test, we see that the series converges if $|x + 3| < 3$ and it
diverges if $|x + 3| > 3$, therefore the associative radius of convergence is
$R = 3$.

\[f(x) = \sin x,\qquad a = \frac{\pi}{2}\tag{18}\]
\begin{align*}
f(x) &= \sum_{n=0}^\infty\frac{f^{(n)}(a)}{n!}(x - a)^n\\
     &= \sum_{n=0}^\infty\frac{\cos\frac{n\pi}{2}}{n!}
                         \left(x - \frac{\pi}{2}\right)^n\\
     &= \sum_{n=0}^\infty\frac{(-1)^n}{(2n)!}
                         \left(x - \frac{\pi}{2}\right)^{2n}
\end{align*}

Let $a_n = (-1)^n(x - \pi/2)^{2n}/(2n)!$,
\[\lim_{n\to\infty}\left|\frac{a_{n+1}}{a_n}\right|
= \lim_{n\to\infty}\frac{(x - \pi/2)^2}{(2n + 2)(2n + 1)}
= 0 < 1\]

Using the Ratio Test, we see that the series converges for all $x$, thus the
associative radius of convergence is $R = \infty$.

\[f(x) = \sqrt x,\qquad x = 16\tag{20}\]
\begin{align*}
   f(x)
&= \sum_{n=0}^\infty\frac{f^{(n)}(a)}{n!}(x - a)^n\\
&= \sum_{n=0}^\infty\left(x\mapsto\binom{\frac{1}{2}}{n}x^{1/2-n}\right)(16)
                    \cdot (x - 16)^n\\
&= \sum_{n=0}^\infty\binom{\frac{1}{2}}{n}\frac{4(x - 16)^n}{16^n}
\end{align*}

Let $a_n = 4\binom{1/2}{n}(x - 16)^n/16^n$,
\[\lim_{n\to\infty}\left|\frac{a_{n+1}}{a_n}\right|
= \lim_{n\to\infty}\left|\frac{1/2 - n}{n + 1}\right|\frac{|x - 16|}{16}
= \frac{|x - 16|}{16}\]

Using the Ratio Test, we see that the series converges if $|x - 16| < 16$ and
it diverges if $|x - 16| > 16$, therefore the associative radius of convergence
is $R = 16$.

\noindent\textbf{55. }Use series to evaluate the limit.
\begin{align*}
   \lim_{x \to 0}\frac{x - \ln(1 + x)}{x^2}
&= \lim_{x \to 0}\frac{x - \sum_{n=1}^\infty (-1)^{n-1}\frac{x^n}{n}}{x^2}\\
&= \lim_{x \to 0}\frac{\sum_{n=2}^\infty (-1)^n\frac{x^n}{n}}{x^2}\\
&= \lim_{x \to 0}\sum_{n=0}^\infty (-1)^n\frac{x^n}{n + 2}\\
&= \lim_{x \to 0}\frac{1}{2}
 + \lim_{x \to 0}\sum_{n=1}^\infty (-1)^n\frac{x^n}{n + 2}\\
&= \frac{1}{2}
\end{align*}

\newpage\noindent Find the sum of the series.
\begin{align*}
   \sum_{n=0}^\infty\frac{(-\ln 2)^n}{n!}
&= \left(x \mapsto \sum_{n=0}^\infty\frac{x^n}{n!}\right)(-\ln 2)\\
&= (x \mapsto e^x)\left(\ln\frac{1}{2}\right)\\
&= \exp\left(\ln\frac{1}{2}\right)\\
&= \frac{1}{2}\tag{68}
\end{align*}

\begin{align*}
\sum_{n=0}^\infty\frac{(-1)^n}{(2n + 1)2^{2n + 1}}
&= \left(x \mapsto \sum_{n=0}^\infty(-1)^n\frac{2^{2n + 1}}{2n + 1}\right)
   \left(\frac{1}{2}\right)\\
&= \left(x \mapsto \tan^{-1}x\right)\left(\frac{1}{2}\right)\\
  &= \tan^{-1}\frac{1}{2}\tag{70}
\end{align*}

\noindent\textbf{72. }If $f(x) = \left(1 + x^3\right)^{30}$, what is
$f^{(58)}(0)$?
\begin{align*}
  f(x) = \sum_{n=0}^{30}\binom{30}{n}x^{3n}
&\Longrightarrow f'(x) = \sum_{n=0}^{30}\binom{30}{n}x^{3n - 1}3n\\
&\Longrightarrow f''(x) = \sum_{n=1}^{30}\binom{30}{n}x^{3n - 2}3n(3n - 1)\\
&\Longrightarrow f^{(58)}(x) = \sum_{n=20}^{30}\binom{30}{n}x^{3n - 58}
                                               \prod_{i=0}^{57}(3n - i)\\
&\Longrightarrow f^{(58)}(0) = \sum_{n=20}^{30}\binom{30}{n}0^{3n - 58}
                                               \prod_{i=0}^{57}(3n - i) = 0
\end{align*}
\end{document}
