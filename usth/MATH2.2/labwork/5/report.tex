\documentclass[a4paper,12pt]{article}
\usepackage[english,vietnamese]{babel}
\usepackage{amsmath}
\usepackage{booktabs}
\usepackage{enumerate}
\usepackage{lmodern}
\usepackage{siunitx}
\usepackage{tikz}

\newcommand{\exercise}[1]{\noindent\textbf{#1.}}

\title{Numerical Methods: Linear Programming}
\author{Nguyễn Gia Phong--BI9-184}
\date{\dateenglish\today}

\begin{document}
\maketitle

Given the production contraints and profits of two grades of heating gas
in the table below.
\begin{center}
  \begin{tabular}{l c c c}
    \toprule
    & \multicolumn{2}{c}{Product}\\
    Resource & Regular & Premium & Availability\\
    \midrule
    Raw gas (\si{\cubic\metre\per\tonne}) & 7 & 11 & 77\\
    Production time (\si{\hour\per\tonne}) & 10 & 8 & 80\\
    Storage (\si{\tonne}) & 9 & 6\\
    \midrule
    Profit (\si{\per\tonne}) & 150 & 175\\
    \bottomrule
  \end{tabular}
\end{center}

\begin{enumerate}[(a)]
  \item Let two nonnegative numbers $x_1$ and $x_2$ respectively be
    the quantities in tonnes of regular and premium gas to be produced.
    The constraints can then be expressed as
    \[\begin{cases}
      7x_1 + 11x^2 &\le 77\\
      10x_1 + 8x_2 &\le 80\\
      x_1 &\le 9\\
      x_2 &\le 6
    \end{cases}
    \iff \begin{bmatrix}
      7 & 11\\
      10 & 8\\
      1 & 0\\
      0 & 1
    \end{bmatrix}\begin{bmatrix}
      x_1\\ x_2
    \end{bmatrix}
    \le \begin{bmatrix}
      77\\ 80\\ 9\\ 6
    \end{bmatrix}\]

    The total profit is the linear function to be maximized:
    \[\Pi(x_1, x_2) = 150x_1 + 175x_2 = \begin{bmatrix}
      150\\ 175
    \end{bmatrix}\cdot\begin{bmatrix}
      x_1\\ x_2
    \end{bmatrix}\]

    Let $\mathbf x = \begin{bmatrix}
      x_1\\ x_2
    \end{bmatrix}$, $A = \begin{bmatrix}
      7 & 11\\
      10 & 8\\
      1 & 0\\
      0 & 1
    \end{bmatrix}$, $\mathbf b = \begin{bmatrix}
      77\\ 80\\ 9\\ 6
    \end{bmatrix}$ and $\mathbf c = \begin{bmatrix}
      150\\ 175
    \end{bmatrix}$, the canonical form of the problem is
    \[\max\left\{\mathbf c^\mathrm T
                 \mid A\mathbf x\le b\land\mathbf x\ge 0\right\}\]
  \item Due to the absense of \verb|linprog| in Octave, we instead use GNU GLPK:
\begin{verbatim}
octave> x = glpk (c, A, b, [], [], "UUUU", "CC", -1)
x =
   4.8889
   3.8889
\end{verbatim}
    Contraint type \verb|UUUU| is used because all contraints are inequalities
    with an upper bound and \verb|CC| indicates continous values of $\mathbf x$.
    With the sense of -1, GLPK looks for the maximization\footnote{I believe
    \emph{minimization} was a typo in the assignment, since it is trivial that
    $\Pi$ has the minimum value of 0 at $\mathbf x = \mathbf 0$.} of
    $\Pi(4.8889, 3.8889) = 1413.9$.

    The two blank arguments are for the lower and upper bounds of $\mathbf x$,
    default to zero and infinite respectively.  Alternatively we can use
    the following to obtain the same result
\begin{verbatim}
glpk (c, [7 11; 10 8], [77; 80], [], [9 6], "UU", "CC", -1)
\end{verbatim}

  \item Within the constrains, the profit can be calculated using the following
    function, which takes two meshes of $x$ and $y$ as arguments
\begin{verbatim}
function z = profit (x, y)
  A = [7 11; 10 8];
  b = [77; 80];
  c = [150; 175];
  [m n] = size (x);
  z = -inf (m, n);
  for s = 1 : m
    for t = 1 : n
      r = [x(s, t); y(s, t)];
      if A * r <= b
        z(s, t) = dot (c, r);
      endif
    endfor
  endfor
endfunction
\end{verbatim}

    Using this, the solution space is then plotted using \verb|ezsurf|,
    which color each grid by their relative values (the smallest is dark purple
    and the largest is bright yellow):
\begin{verbatim}
ezsurf (@(x1, x2) constraints (x1, x2), [0 9 0 6], 58)
\end{verbatim}
\pagebreak

    Since the plot is just a part of a plane, we can rotate it for a better view
    without losing any information about its shape.

    \scalebox{0.69}{% Title: gl2ps_renderer figure
% Creator: GL2PS 1.4.0, (C) 1999-2017 C. Geuzaine
% For: Octave
% CreationDate: Fri Nov  8 22:10:22 2019
\begin{pgfpicture}
\color[rgb]{1.000000,1.000000,1.000000}
\pgfpathrectanglecorners{\pgfpoint{0pt}{0pt}}{\pgfpoint{576pt}{432pt}}
\pgfusepath{fill}
\begin{pgfscope}
\pgfpathrectangle{\pgfpoint{0pt}{0pt}}{\pgfpoint{576pt}{432pt}}
\pgfusepath{fill}
\pgfpathrectangle{\pgfpoint{0pt}{0pt}}{\pgfpoint{576pt}{432pt}}
\pgfusepath{clip}
\pgfpathmoveto{\pgfpoint{74.880005pt}{399.600037pt}}
\pgflineto{\pgfpoint{521.279968pt}{47.519989pt}}
\pgflineto{\pgfpoint{74.880005pt}{47.519989pt}}
\pgfpathclose
\pgfusepath{fill,stroke}
\pgfpathmoveto{\pgfpoint{74.880005pt}{399.600037pt}}
\pgflineto{\pgfpoint{521.279968pt}{399.600037pt}}
\pgflineto{\pgfpoint{521.279968pt}{47.519989pt}}
\pgfpathclose
\pgfusepath{fill,stroke}
\color[rgb]{0.150000,0.150000,0.150000}
\pgfsetlinewidth{0.500000pt}
\pgfpathmoveto{\pgfpoint{74.880005pt}{51.985016pt}}
\pgflineto{\pgfpoint{74.880005pt}{51.210175pt}}
\pgfusepath{stroke}
\pgfpathmoveto{\pgfpoint{187.967987pt}{51.985016pt}}
\pgflineto{\pgfpoint{187.967987pt}{47.519989pt}}
\pgfusepath{stroke}
\pgfpathmoveto{\pgfpoint{301.055969pt}{51.985016pt}}
\pgflineto{\pgfpoint{301.055969pt}{47.519989pt}}
\pgfusepath{stroke}
\pgfpathmoveto{\pgfpoint{414.143982pt}{51.985016pt}}
\pgflineto{\pgfpoint{414.143982pt}{47.519989pt}}
\pgfusepath{stroke}
\pgfpathmoveto{\pgfpoint{79.347992pt}{47.519989pt}}
\pgflineto{\pgfpoint{79.027298pt}{47.519989pt}}
\pgfusepath{stroke}
\pgfpathmoveto{\pgfpoint{79.347992pt}{106.200005pt}}
\pgflineto{\pgfpoint{74.880005pt}{106.200005pt}}
\pgfusepath{stroke}
\pgfpathmoveto{\pgfpoint{79.347992pt}{164.880005pt}}
\pgflineto{\pgfpoint{74.880005pt}{164.880005pt}}
\pgfusepath{stroke}
\pgfpathmoveto{\pgfpoint{79.347992pt}{223.560013pt}}
\pgflineto{\pgfpoint{74.880005pt}{223.560013pt}}
\pgfusepath{stroke}
\pgfpathmoveto{\pgfpoint{79.347992pt}{282.239990pt}}
\pgflineto{\pgfpoint{74.880005pt}{282.239990pt}}
\pgfusepath{stroke}
\pgfpathmoveto{\pgfpoint{79.347992pt}{340.919983pt}}
\pgflineto{\pgfpoint{74.880005pt}{340.919983pt}}
\pgfusepath{stroke}
\pgfpathmoveto{\pgfpoint{79.347992pt}{399.600037pt}}
\pgflineto{\pgfpoint{74.880005pt}{399.600037pt}}
\pgfusepath{stroke}
\pgfsetrectcap
\pgfsetdash{{16pt}{0pt}}{0pt}
\pgfpathmoveto{\pgfpoint{521.279968pt}{47.519989pt}}
\pgflineto{\pgfpoint{79.027298pt}{47.519989pt}}
\pgfusepath{stroke}
\pgfpathmoveto{\pgfpoint{74.880005pt}{399.600037pt}}
\pgflineto{\pgfpoint{74.880005pt}{51.210175pt}}
\pgfusepath{stroke}
{
\pgftransformshift{\pgfpoint{187.967987pt}{40.018295pt}}
\pgfnode{rectangle}{north}{\fontsize{10}{0}\selectfont\textcolor[rgb]{0.15,0.15,0.15}{{2}}}{}{\pgfusepath{discard}}}
{
\pgftransformshift{\pgfpoint{301.056000pt}{40.018295pt}}
\pgfnode{rectangle}{north}{\fontsize{10}{0}\selectfont\textcolor[rgb]{0.15,0.15,0.15}{{4}}}{}{\pgfusepath{discard}}}
{
\pgftransformshift{\pgfpoint{414.143982pt}{40.018295pt}}
\pgfnode{rectangle}{north}{\fontsize{10}{0}\selectfont\textcolor[rgb]{0.15,0.15,0.15}{{6}}}{}{\pgfusepath{discard}}}
{
\pgftransformshift{\pgfpoint{69.875519pt}{106.200005pt}}
\pgfnode{rectangle}{east}{\fontsize{10}{0}\selectfont\textcolor[rgb]{0.15,0.15,0.15}{{1}}}{}{\pgfusepath{discard}}}
{
\pgftransformshift{\pgfpoint{69.875519pt}{164.880005pt}}
\pgfnode{rectangle}{east}{\fontsize{10}{0}\selectfont\textcolor[rgb]{0.15,0.15,0.15}{{2}}}{}{\pgfusepath{discard}}}
{
\pgftransformshift{\pgfpoint{69.875519pt}{223.559998pt}}
\pgfnode{rectangle}{east}{\fontsize{10}{0}\selectfont\textcolor[rgb]{0.15,0.15,0.15}{{3}}}{}{\pgfusepath{discard}}}
{
\pgftransformshift{\pgfpoint{69.875519pt}{282.239990pt}}
\pgfnode{rectangle}{east}{\fontsize{10}{0}\selectfont\textcolor[rgb]{0.15,0.15,0.15}{{4}}}{}{\pgfusepath{discard}}}
{
\pgftransformshift{\pgfpoint{69.875519pt}{340.920013pt}}
\pgfnode{rectangle}{east}{\fontsize{10}{0}\selectfont\textcolor[rgb]{0.15,0.15,0.15}{{5}}}{}{\pgfusepath{discard}}}
{
\pgftransformshift{\pgfpoint{69.875519pt}{399.599976pt}}
\pgfnode{rectangle}{east}{\fontsize{10}{0}\selectfont\textcolor[rgb]{0.15,0.15,0.15}{{6}}}{}{\pgfusepath{discard}}}
{
\pgftransformshift{\pgfpoint{58.875519pt}{223.559998pt}}
\pgftransformrotate{90.000000}{\pgfnode{rectangle}{south}{\fontsize{11}{0}\selectfont\textcolor[rgb]{0.15,0.15,0.15}{{x2}}}{}{\pgfusepath{discard}}}}
{
\pgftransformshift{\pgfpoint{298.079987pt}{27.018295pt}}
\pgfnode{rectangle}{north}{\fontsize{11}{0}\selectfont\textcolor[rgb]{0.15,0.15,0.15}{{x1}}}{}{\pgfusepath{discard}}}
{
\pgftransformshift{\pgfpoint{298.079987pt}{409.600006pt}}
\pgfnode{rectangle}{south}{\fontsize{11}{0}\selectfont\textcolor[rgb]{0,0,0}{{profit (x1, x2)}}}{}{\pgfusepath{discard}}}
\pgfsetlinewidth{0.01pt}
\color[rgb]{0.280356,0.074201,0.397901}
\pgfpathmoveto{\pgfpoint{101.664001pt}{53.696838pt}}
\pgflineto{\pgfpoint{110.591980pt}{53.696838pt}}
\pgflineto{\pgfpoint{110.591980pt}{47.519989pt}}
\pgfpathclose
\pgfusepath{fill,stroke}
\color[rgb]{0.282390,0.095954,0.418251}
\pgfpathmoveto{\pgfpoint{101.664001pt}{59.873672pt}}
\pgflineto{\pgfpoint{110.591980pt}{53.696838pt}}
\pgflineto{\pgfpoint{101.664001pt}{53.696838pt}}
\pgfpathclose
\pgfusepath{fill,stroke}
\color[rgb]{0.283205,0.116893,0.437179}
\pgfpathmoveto{\pgfpoint{119.519989pt}{53.696838pt}}
\pgflineto{\pgfpoint{128.447998pt}{47.519989pt}}
\pgflineto{\pgfpoint{119.519989pt}{47.519989pt}}
\pgfpathclose
\pgfusepath{fill,stroke}
\color[rgb]{0.277106,0.050914,0.376236}
\pgfpathmoveto{\pgfpoint{83.807999pt}{66.050522pt}}
\pgflineto{\pgfpoint{92.735992pt}{66.050522pt}}
\pgflineto{\pgfpoint{92.735992pt}{59.873672pt}}
\pgfpathclose
\pgfusepath{fill,stroke}
\color[rgb]{0.280356,0.074201,0.397901}
\pgfpathmoveto{\pgfpoint{83.807999pt}{72.227356pt}}
\pgflineto{\pgfpoint{92.735992pt}{66.050522pt}}
\pgflineto{\pgfpoint{83.807999pt}{66.050522pt}}
\pgfpathclose
\pgfusepath{fill,stroke}
\pgfpathmoveto{\pgfpoint{83.807999pt}{72.227356pt}}
\pgflineto{\pgfpoint{92.735992pt}{72.227356pt}}
\pgflineto{\pgfpoint{92.735992pt}{66.050522pt}}
\pgfpathclose
\pgfusepath{fill,stroke}
\color[rgb]{0.277106,0.050914,0.376236}
\pgfpathmoveto{\pgfpoint{92.735992pt}{53.696838pt}}
\pgflineto{\pgfpoint{101.664001pt}{53.696838pt}}
\pgflineto{\pgfpoint{101.664001pt}{47.519989pt}}
\pgfpathclose
\pgfusepath{fill,stroke}
\color[rgb]{0.280356,0.074201,0.397901}
\pgfpathmoveto{\pgfpoint{92.735992pt}{59.873672pt}}
\pgflineto{\pgfpoint{101.664001pt}{53.696838pt}}
\pgflineto{\pgfpoint{92.735992pt}{53.696838pt}}
\pgfpathclose
\pgfusepath{fill,stroke}
\pgfpathmoveto{\pgfpoint{92.735992pt}{59.873672pt}}
\pgflineto{\pgfpoint{101.664001pt}{59.873672pt}}
\pgflineto{\pgfpoint{101.664001pt}{53.696838pt}}
\pgfpathclose
\pgfusepath{fill,stroke}
\pgfpathmoveto{\pgfpoint{92.735992pt}{66.050522pt}}
\pgflineto{\pgfpoint{101.664001pt}{59.873672pt}}
\pgflineto{\pgfpoint{92.735992pt}{59.873672pt}}
\pgfpathclose
\pgfusepath{fill,stroke}
\pgfpathmoveto{\pgfpoint{92.735992pt}{66.050522pt}}
\pgflineto{\pgfpoint{101.664001pt}{66.050522pt}}
\pgflineto{\pgfpoint{101.664001pt}{59.873672pt}}
\pgfpathclose
\pgfusepath{fill,stroke}
\color[rgb]{0.282390,0.095954,0.418251}
\pgfpathmoveto{\pgfpoint{92.735992pt}{72.227356pt}}
\pgflineto{\pgfpoint{101.664001pt}{66.050522pt}}
\pgflineto{\pgfpoint{92.735992pt}{66.050522pt}}
\pgfpathclose
\pgfusepath{fill,stroke}
\color[rgb]{0.280356,0.074201,0.397901}
\pgfpathmoveto{\pgfpoint{101.664001pt}{53.696838pt}}
\pgflineto{\pgfpoint{110.591980pt}{47.519989pt}}
\pgflineto{\pgfpoint{101.664001pt}{47.519989pt}}
\pgfpathclose
\pgfusepath{fill,stroke}
\color[rgb]{0.282390,0.095954,0.418251}
\pgfpathmoveto{\pgfpoint{101.664001pt}{59.873672pt}}
\pgflineto{\pgfpoint{110.591980pt}{59.873672pt}}
\pgflineto{\pgfpoint{110.591980pt}{53.696838pt}}
\pgfpathclose
\pgfusepath{fill,stroke}
\pgfpathmoveto{\pgfpoint{101.664001pt}{66.050522pt}}
\pgflineto{\pgfpoint{110.591980pt}{59.873672pt}}
\pgflineto{\pgfpoint{101.664001pt}{59.873672pt}}
\pgfpathclose
\pgfusepath{fill,stroke}
\pgfpathmoveto{\pgfpoint{110.591980pt}{53.696838pt}}
\pgflineto{\pgfpoint{119.519989pt}{47.519989pt}}
\pgflineto{\pgfpoint{110.591980pt}{47.519989pt}}
\pgfpathclose
\pgfusepath{fill,stroke}
\pgfpathmoveto{\pgfpoint{110.591980pt}{53.696838pt}}
\pgflineto{\pgfpoint{119.519989pt}{53.696838pt}}
\pgflineto{\pgfpoint{119.519989pt}{47.519989pt}}
\pgfpathclose
\pgfusepath{fill,stroke}
\color[rgb]{0.283205,0.116893,0.437179}
\pgfpathmoveto{\pgfpoint{110.591980pt}{59.873672pt}}
\pgflineto{\pgfpoint{119.519989pt}{53.696838pt}}
\pgflineto{\pgfpoint{110.591980pt}{53.696838pt}}
\pgfpathclose
\pgfusepath{fill,stroke}
\pgfpathmoveto{\pgfpoint{110.591980pt}{59.873672pt}}
\pgflineto{\pgfpoint{119.519989pt}{59.873672pt}}
\pgflineto{\pgfpoint{119.519989pt}{53.696838pt}}
\pgfpathclose
\pgfusepath{fill,stroke}
\pgfpathmoveto{\pgfpoint{119.519989pt}{53.696838pt}}
\pgflineto{\pgfpoint{128.447998pt}{53.696838pt}}
\pgflineto{\pgfpoint{128.447998pt}{47.519989pt}}
\pgfpathclose
\pgfusepath{fill,stroke}
\color[rgb]{0.282809,0.137350,0.454596}
\pgfpathmoveto{\pgfpoint{119.519989pt}{59.873672pt}}
\pgflineto{\pgfpoint{128.447998pt}{53.696838pt}}
\pgflineto{\pgfpoint{119.519989pt}{53.696838pt}}
\pgfpathclose
\pgfusepath{fill,stroke}
\pgfpathmoveto{\pgfpoint{128.447998pt}{53.696838pt}}
\pgflineto{\pgfpoint{137.376007pt}{47.519989pt}}
\pgflineto{\pgfpoint{128.447998pt}{47.519989pt}}
\pgfpathclose
\pgfusepath{fill,stroke}
\pgfpathmoveto{\pgfpoint{128.447998pt}{53.696838pt}}
\pgflineto{\pgfpoint{137.376007pt}{53.696838pt}}
\pgflineto{\pgfpoint{137.376007pt}{47.519989pt}}
\pgfpathclose
\pgfusepath{fill,stroke}
\color[rgb]{0.281231,0.157480,0.470434}
\pgfpathmoveto{\pgfpoint{137.376007pt}{53.696838pt}}
\pgflineto{\pgfpoint{146.303986pt}{47.519989pt}}
\pgflineto{\pgfpoint{137.376007pt}{47.519989pt}}
\pgfpathclose
\pgfusepath{fill,stroke}
\color[rgb]{0.282390,0.095954,0.418251}
\pgfpathmoveto{\pgfpoint{92.735992pt}{72.227356pt}}
\pgflineto{\pgfpoint{101.664001pt}{72.227356pt}}
\pgflineto{\pgfpoint{101.664001pt}{66.050522pt}}
\pgfpathclose
\pgfusepath{fill,stroke}
\color[rgb]{0.283205,0.116893,0.437179}
\pgfpathmoveto{\pgfpoint{92.735992pt}{78.404205pt}}
\pgflineto{\pgfpoint{101.664001pt}{72.227356pt}}
\pgflineto{\pgfpoint{92.735992pt}{72.227356pt}}
\pgfpathclose
\pgfusepath{fill,stroke}
\color[rgb]{0.282390,0.095954,0.418251}
\pgfpathmoveto{\pgfpoint{74.880005pt}{84.581039pt}}
\pgflineto{\pgfpoint{83.807999pt}{84.581039pt}}
\pgflineto{\pgfpoint{83.807999pt}{78.404205pt}}
\pgfpathclose
\pgfusepath{fill,stroke}
\color[rgb]{0.283205,0.116893,0.437179}
\pgfpathmoveto{\pgfpoint{74.880005pt}{90.757896pt}}
\pgflineto{\pgfpoint{83.807999pt}{84.581039pt}}
\pgflineto{\pgfpoint{74.880005pt}{84.581039pt}}
\pgfpathclose
\pgfusepath{fill,stroke}
\pgfpathmoveto{\pgfpoint{74.880005pt}{90.757896pt}}
\pgflineto{\pgfpoint{83.807999pt}{90.757896pt}}
\pgflineto{\pgfpoint{83.807999pt}{84.581039pt}}
\pgfpathclose
\pgfusepath{fill,stroke}
\color[rgb]{0.282390,0.095954,0.418251}
\pgfpathmoveto{\pgfpoint{83.807999pt}{78.404205pt}}
\pgflineto{\pgfpoint{92.735992pt}{72.227356pt}}
\pgflineto{\pgfpoint{83.807999pt}{72.227356pt}}
\pgfpathclose
\pgfusepath{fill,stroke}
\pgfpathmoveto{\pgfpoint{83.807999pt}{78.404205pt}}
\pgflineto{\pgfpoint{92.735992pt}{78.404205pt}}
\pgflineto{\pgfpoint{92.735992pt}{72.227356pt}}
\pgfpathclose
\pgfusepath{fill,stroke}
\color[rgb]{0.283205,0.116893,0.437179}
\pgfpathmoveto{\pgfpoint{83.807999pt}{84.581039pt}}
\pgflineto{\pgfpoint{92.735992pt}{78.404205pt}}
\pgflineto{\pgfpoint{83.807999pt}{78.404205pt}}
\pgfpathclose
\pgfusepath{fill,stroke}
\pgfpathmoveto{\pgfpoint{83.807999pt}{84.581039pt}}
\pgflineto{\pgfpoint{92.735992pt}{84.581039pt}}
\pgflineto{\pgfpoint{92.735992pt}{78.404205pt}}
\pgfpathclose
\pgfusepath{fill,stroke}
\color[rgb]{0.282809,0.137350,0.454596}
\pgfpathmoveto{\pgfpoint{83.807999pt}{90.757896pt}}
\pgflineto{\pgfpoint{92.735992pt}{84.581039pt}}
\pgflineto{\pgfpoint{83.807999pt}{84.581039pt}}
\pgfpathclose
\pgfusepath{fill,stroke}
\color[rgb]{0.283205,0.116893,0.437179}
\pgfpathmoveto{\pgfpoint{92.735992pt}{78.404205pt}}
\pgflineto{\pgfpoint{101.664001pt}{78.404205pt}}
\pgflineto{\pgfpoint{101.664001pt}{72.227356pt}}
\pgfpathclose
\pgfusepath{fill,stroke}
\color[rgb]{0.282809,0.137350,0.454596}
\pgfpathmoveto{\pgfpoint{92.735992pt}{84.581039pt}}
\pgflineto{\pgfpoint{101.664001pt}{78.404205pt}}
\pgflineto{\pgfpoint{92.735992pt}{78.404205pt}}
\pgfpathclose
\pgfusepath{fill,stroke}
\pgfpathmoveto{\pgfpoint{92.735992pt}{84.581039pt}}
\pgflineto{\pgfpoint{101.664001pt}{84.581039pt}}
\pgflineto{\pgfpoint{101.664001pt}{78.404205pt}}
\pgfpathclose
\pgfusepath{fill,stroke}
\color[rgb]{0.282390,0.095954,0.418251}
\pgfpathmoveto{\pgfpoint{101.664001pt}{66.050522pt}}
\pgflineto{\pgfpoint{110.591980pt}{66.050522pt}}
\pgflineto{\pgfpoint{110.591980pt}{59.873672pt}}
\pgfpathclose
\pgfusepath{fill,stroke}
\color[rgb]{0.283205,0.116893,0.437179}
\pgfpathmoveto{\pgfpoint{101.664001pt}{72.227356pt}}
\pgflineto{\pgfpoint{110.591980pt}{66.050522pt}}
\pgflineto{\pgfpoint{101.664001pt}{66.050522pt}}
\pgfpathclose
\pgfusepath{fill,stroke}
\pgfpathmoveto{\pgfpoint{101.664001pt}{72.227356pt}}
\pgflineto{\pgfpoint{110.591980pt}{72.227356pt}}
\pgflineto{\pgfpoint{110.591980pt}{66.050522pt}}
\pgfpathclose
\pgfusepath{fill,stroke}
\color[rgb]{0.282809,0.137350,0.454596}
\pgfpathmoveto{\pgfpoint{101.664001pt}{78.404205pt}}
\pgflineto{\pgfpoint{110.591980pt}{72.227356pt}}
\pgflineto{\pgfpoint{101.664001pt}{72.227356pt}}
\pgfpathclose
\pgfusepath{fill,stroke}
\pgfpathmoveto{\pgfpoint{101.664001pt}{78.404205pt}}
\pgflineto{\pgfpoint{110.591980pt}{78.404205pt}}
\pgflineto{\pgfpoint{110.591980pt}{72.227356pt}}
\pgfpathclose
\pgfusepath{fill,stroke}
\color[rgb]{0.281231,0.157480,0.470434}
\pgfpathmoveto{\pgfpoint{101.664001pt}{84.581039pt}}
\pgflineto{\pgfpoint{110.591980pt}{78.404205pt}}
\pgflineto{\pgfpoint{101.664001pt}{78.404205pt}}
\pgfpathclose
\pgfusepath{fill,stroke}
\color[rgb]{0.282809,0.137350,0.454596}
\pgfpathmoveto{\pgfpoint{110.591980pt}{66.050522pt}}
\pgflineto{\pgfpoint{119.519989pt}{59.873672pt}}
\pgflineto{\pgfpoint{110.591980pt}{59.873672pt}}
\pgfpathclose
\pgfusepath{fill,stroke}
\pgfpathmoveto{\pgfpoint{110.591980pt}{66.050522pt}}
\pgflineto{\pgfpoint{119.519989pt}{66.050522pt}}
\pgflineto{\pgfpoint{119.519989pt}{59.873672pt}}
\pgfpathclose
\pgfusepath{fill,stroke}
\pgfpathmoveto{\pgfpoint{110.591980pt}{72.227356pt}}
\pgflineto{\pgfpoint{119.519989pt}{66.050522pt}}
\pgflineto{\pgfpoint{110.591980pt}{66.050522pt}}
\pgfpathclose
\pgfusepath{fill,stroke}
\pgfpathmoveto{\pgfpoint{110.591980pt}{72.227356pt}}
\pgflineto{\pgfpoint{119.519989pt}{72.227356pt}}
\pgflineto{\pgfpoint{119.519989pt}{66.050522pt}}
\pgfpathclose
\pgfusepath{fill,stroke}
\color[rgb]{0.281231,0.157480,0.470434}
\pgfpathmoveto{\pgfpoint{110.591980pt}{78.404205pt}}
\pgflineto{\pgfpoint{119.519989pt}{72.227356pt}}
\pgflineto{\pgfpoint{110.591980pt}{72.227356pt}}
\pgfpathclose
\pgfusepath{fill,stroke}
\pgfpathmoveto{\pgfpoint{110.591980pt}{78.404205pt}}
\pgflineto{\pgfpoint{119.519989pt}{78.404205pt}}
\pgflineto{\pgfpoint{119.519989pt}{72.227356pt}}
\pgfpathclose
\pgfusepath{fill,stroke}
\color[rgb]{0.282809,0.137350,0.454596}
\pgfpathmoveto{\pgfpoint{119.519989pt}{59.873672pt}}
\pgflineto{\pgfpoint{128.447998pt}{59.873672pt}}
\pgflineto{\pgfpoint{128.447998pt}{53.696838pt}}
\pgfpathclose
\pgfusepath{fill,stroke}
\color[rgb]{0.281231,0.157480,0.470434}
\pgfpathmoveto{\pgfpoint{119.519989pt}{66.050522pt}}
\pgflineto{\pgfpoint{128.447998pt}{59.873672pt}}
\pgflineto{\pgfpoint{119.519989pt}{59.873672pt}}
\pgfpathclose
\pgfusepath{fill,stroke}
\pgfpathmoveto{\pgfpoint{119.519989pt}{66.050522pt}}
\pgflineto{\pgfpoint{128.447998pt}{66.050522pt}}
\pgflineto{\pgfpoint{128.447998pt}{59.873672pt}}
\pgfpathclose
\pgfusepath{fill,stroke}
\pgfpathmoveto{\pgfpoint{119.519989pt}{72.227356pt}}
\pgflineto{\pgfpoint{128.447998pt}{66.050522pt}}
\pgflineto{\pgfpoint{119.519989pt}{66.050522pt}}
\pgfpathclose
\pgfusepath{fill,stroke}
\pgfpathmoveto{\pgfpoint{119.519989pt}{72.227356pt}}
\pgflineto{\pgfpoint{128.447998pt}{72.227356pt}}
\pgflineto{\pgfpoint{128.447998pt}{66.050522pt}}
\pgfpathclose
\pgfusepath{fill,stroke}
\color[rgb]{0.278516,0.177348,0.484654}
\pgfpathmoveto{\pgfpoint{119.519989pt}{78.404205pt}}
\pgflineto{\pgfpoint{128.447998pt}{72.227356pt}}
\pgflineto{\pgfpoint{119.519989pt}{72.227356pt}}
\pgfpathclose
\pgfusepath{fill,stroke}
\color[rgb]{0.281231,0.157480,0.470434}
\pgfpathmoveto{\pgfpoint{128.447998pt}{59.873672pt}}
\pgflineto{\pgfpoint{137.376007pt}{53.696838pt}}
\pgflineto{\pgfpoint{128.447998pt}{53.696838pt}}
\pgfpathclose
\pgfusepath{fill,stroke}
\pgfpathmoveto{\pgfpoint{128.447998pt}{59.873672pt}}
\pgflineto{\pgfpoint{137.376007pt}{59.873672pt}}
\pgflineto{\pgfpoint{137.376007pt}{53.696838pt}}
\pgfpathclose
\pgfusepath{fill,stroke}
\color[rgb]{0.278516,0.177348,0.484654}
\pgfpathmoveto{\pgfpoint{128.447998pt}{66.050522pt}}
\pgflineto{\pgfpoint{137.376007pt}{59.873672pt}}
\pgflineto{\pgfpoint{128.447998pt}{59.873672pt}}
\pgfpathclose
\pgfusepath{fill,stroke}
\pgfpathmoveto{\pgfpoint{128.447998pt}{66.050522pt}}
\pgflineto{\pgfpoint{137.376007pt}{66.050522pt}}
\pgflineto{\pgfpoint{137.376007pt}{59.873672pt}}
\pgfpathclose
\pgfusepath{fill,stroke}
\color[rgb]{0.274736,0.196969,0.497250}
\pgfpathmoveto{\pgfpoint{128.447998pt}{72.227356pt}}
\pgflineto{\pgfpoint{137.376007pt}{66.050522pt}}
\pgflineto{\pgfpoint{128.447998pt}{66.050522pt}}
\pgfpathclose
\pgfusepath{fill,stroke}
\pgfpathmoveto{\pgfpoint{128.447998pt}{72.227356pt}}
\pgflineto{\pgfpoint{137.376007pt}{72.227356pt}}
\pgflineto{\pgfpoint{137.376007pt}{66.050522pt}}
\pgfpathclose
\pgfusepath{fill,stroke}
\color[rgb]{0.281231,0.157480,0.470434}
\pgfpathmoveto{\pgfpoint{137.376007pt}{53.696838pt}}
\pgflineto{\pgfpoint{146.303986pt}{53.696838pt}}
\pgflineto{\pgfpoint{146.303986pt}{47.519989pt}}
\pgfpathclose
\pgfusepath{fill,stroke}
\color[rgb]{0.278516,0.177348,0.484654}
\pgfpathmoveto{\pgfpoint{137.376007pt}{59.873672pt}}
\pgflineto{\pgfpoint{146.303986pt}{53.696838pt}}
\pgflineto{\pgfpoint{137.376007pt}{53.696838pt}}
\pgfpathclose
\pgfusepath{fill,stroke}
\pgfpathmoveto{\pgfpoint{137.376007pt}{59.873672pt}}
\pgflineto{\pgfpoint{146.303986pt}{59.873672pt}}
\pgflineto{\pgfpoint{146.303986pt}{53.696838pt}}
\pgfpathclose
\pgfusepath{fill,stroke}
\color[rgb]{0.274736,0.196969,0.497250}
\pgfpathmoveto{\pgfpoint{137.376007pt}{66.050522pt}}
\pgflineto{\pgfpoint{146.303986pt}{59.873672pt}}
\pgflineto{\pgfpoint{137.376007pt}{59.873672pt}}
\pgfpathclose
\pgfusepath{fill,stroke}
\pgfpathmoveto{\pgfpoint{137.376007pt}{66.050522pt}}
\pgflineto{\pgfpoint{146.303986pt}{66.050522pt}}
\pgflineto{\pgfpoint{146.303986pt}{59.873672pt}}
\pgfpathclose
\pgfusepath{fill,stroke}
\color[rgb]{0.269982,0.216330,0.508255}
\pgfpathmoveto{\pgfpoint{137.376007pt}{72.227356pt}}
\pgflineto{\pgfpoint{146.303986pt}{66.050522pt}}
\pgflineto{\pgfpoint{137.376007pt}{66.050522pt}}
\pgfpathclose
\pgfusepath{fill,stroke}
\color[rgb]{0.278516,0.177348,0.484654}
\pgfpathmoveto{\pgfpoint{146.303986pt}{53.696838pt}}
\pgflineto{\pgfpoint{155.231979pt}{47.519989pt}}
\pgflineto{\pgfpoint{146.303986pt}{47.519989pt}}
\pgfpathclose
\pgfusepath{fill,stroke}
\pgfpathmoveto{\pgfpoint{146.303986pt}{53.696838pt}}
\pgflineto{\pgfpoint{155.231979pt}{53.696838pt}}
\pgflineto{\pgfpoint{155.231979pt}{47.519989pt}}
\pgfpathclose
\pgfusepath{fill,stroke}
\color[rgb]{0.274736,0.196969,0.497250}
\pgfpathmoveto{\pgfpoint{146.303986pt}{59.873672pt}}
\pgflineto{\pgfpoint{155.231979pt}{53.696838pt}}
\pgflineto{\pgfpoint{146.303986pt}{53.696838pt}}
\pgfpathclose
\pgfusepath{fill,stroke}
\pgfpathmoveto{\pgfpoint{146.303986pt}{59.873672pt}}
\pgflineto{\pgfpoint{155.231979pt}{59.873672pt}}
\pgflineto{\pgfpoint{155.231979pt}{53.696838pt}}
\pgfpathclose
\pgfusepath{fill,stroke}
\color[rgb]{0.269982,0.216330,0.508255}
\pgfpathmoveto{\pgfpoint{146.303986pt}{66.050522pt}}
\pgflineto{\pgfpoint{155.231979pt}{59.873672pt}}
\pgflineto{\pgfpoint{146.303986pt}{59.873672pt}}
\pgfpathclose
\pgfusepath{fill,stroke}
\color[rgb]{0.274736,0.196969,0.497250}
\pgfpathmoveto{\pgfpoint{155.231979pt}{53.696838pt}}
\pgflineto{\pgfpoint{164.160004pt}{47.519989pt}}
\pgflineto{\pgfpoint{155.231979pt}{47.519989pt}}
\pgfpathclose
\pgfusepath{fill,stroke}
\pgfpathmoveto{\pgfpoint{155.231979pt}{53.696838pt}}
\pgflineto{\pgfpoint{164.160004pt}{53.696838pt}}
\pgflineto{\pgfpoint{164.160004pt}{47.519989pt}}
\pgfpathclose
\pgfusepath{fill,stroke}
\color[rgb]{0.269982,0.216330,0.508255}
\pgfpathmoveto{\pgfpoint{155.231979pt}{59.873672pt}}
\pgflineto{\pgfpoint{164.160004pt}{53.696838pt}}
\pgflineto{\pgfpoint{155.231979pt}{53.696838pt}}
\pgfpathclose
\pgfusepath{fill,stroke}
\pgfpathmoveto{\pgfpoint{155.231979pt}{59.873672pt}}
\pgflineto{\pgfpoint{164.160004pt}{59.873672pt}}
\pgflineto{\pgfpoint{164.160004pt}{53.696838pt}}
\pgfpathclose
\pgfusepath{fill,stroke}
\pgfpathmoveto{\pgfpoint{164.160004pt}{53.696838pt}}
\pgflineto{\pgfpoint{173.087997pt}{47.519989pt}}
\pgflineto{\pgfpoint{164.160004pt}{47.519989pt}}
\pgfpathclose
\pgfusepath{fill,stroke}
\pgfpathmoveto{\pgfpoint{164.160004pt}{53.696838pt}}
\pgflineto{\pgfpoint{173.087997pt}{53.696838pt}}
\pgflineto{\pgfpoint{173.087997pt}{47.519989pt}}
\pgfpathclose
\pgfusepath{fill,stroke}
\color[rgb]{0.264369,0.235405,0.517732}
\pgfpathmoveto{\pgfpoint{164.160004pt}{59.873672pt}}
\pgflineto{\pgfpoint{173.087997pt}{53.696838pt}}
\pgflineto{\pgfpoint{164.160004pt}{53.696838pt}}
\pgfpathclose
\pgfusepath{fill,stroke}
\pgfpathmoveto{\pgfpoint{173.087997pt}{53.696838pt}}
\pgflineto{\pgfpoint{182.015991pt}{47.519989pt}}
\pgflineto{\pgfpoint{173.087997pt}{47.519989pt}}
\pgfpathclose
\pgfusepath{fill,stroke}
\pgfpathmoveto{\pgfpoint{173.087997pt}{53.696838pt}}
\pgflineto{\pgfpoint{182.015991pt}{53.696838pt}}
\pgflineto{\pgfpoint{182.015991pt}{47.519989pt}}
\pgfpathclose
\pgfusepath{fill,stroke}
\color[rgb]{0.258026,0.254162,0.525780}
\pgfpathmoveto{\pgfpoint{182.015991pt}{53.696838pt}}
\pgflineto{\pgfpoint{190.943985pt}{47.519989pt}}
\pgflineto{\pgfpoint{182.015991pt}{47.519989pt}}
\pgfpathclose
\pgfusepath{fill,stroke}
\color[rgb]{0.269982,0.216330,0.508255}
\pgfpathmoveto{\pgfpoint{74.880005pt}{127.818947pt}}
\pgflineto{\pgfpoint{83.807999pt}{121.642097pt}}
\pgflineto{\pgfpoint{74.880005pt}{121.642097pt}}
\pgfpathclose
\pgfusepath{fill,stroke}
\pgfpathmoveto{\pgfpoint{74.880005pt}{127.818947pt}}
\pgflineto{\pgfpoint{83.807999pt}{127.818947pt}}
\pgflineto{\pgfpoint{83.807999pt}{121.642097pt}}
\pgfpathclose
\pgfusepath{fill,stroke}
\pgfpathmoveto{\pgfpoint{74.880005pt}{133.995789pt}}
\pgflineto{\pgfpoint{83.807999pt}{127.818947pt}}
\pgflineto{\pgfpoint{74.880005pt}{127.818947pt}}
\pgfpathclose
\pgfusepath{fill,stroke}
\color[rgb]{0.228263,0.325586,0.546335}
\pgfpathmoveto{\pgfpoint{92.735992pt}{158.703156pt}}
\pgflineto{\pgfpoint{101.664001pt}{158.703156pt}}
\pgflineto{\pgfpoint{101.664001pt}{152.526306pt}}
\pgfpathclose
\pgfusepath{fill,stroke}
\color[rgb]{0.220425,0.342517,0.549287}
\pgfpathmoveto{\pgfpoint{92.735992pt}{164.880005pt}}
\pgflineto{\pgfpoint{101.664001pt}{158.703156pt}}
\pgflineto{\pgfpoint{92.735992pt}{158.703156pt}}
\pgfpathclose
\pgfusepath{fill,stroke}
\color[rgb]{0.236073,0.308291,0.542652}
\pgfpathmoveto{\pgfpoint{74.880005pt}{171.056854pt}}
\pgflineto{\pgfpoint{83.807999pt}{171.056854pt}}
\pgflineto{\pgfpoint{83.807999pt}{164.880005pt}}
\pgfpathclose
\pgfusepath{fill,stroke}
\color[rgb]{0.228263,0.325586,0.546335}
\pgfpathmoveto{\pgfpoint{74.880005pt}{177.233673pt}}
\pgflineto{\pgfpoint{83.807999pt}{171.056854pt}}
\pgflineto{\pgfpoint{74.880005pt}{171.056854pt}}
\pgfpathclose
\pgfusepath{fill,stroke}
\pgfpathmoveto{\pgfpoint{74.880005pt}{177.233673pt}}
\pgflineto{\pgfpoint{83.807999pt}{177.233673pt}}
\pgflineto{\pgfpoint{83.807999pt}{171.056854pt}}
\pgfpathclose
\pgfusepath{fill,stroke}
\color[rgb]{0.236073,0.308291,0.542652}
\pgfpathmoveto{\pgfpoint{83.807999pt}{158.703156pt}}
\pgflineto{\pgfpoint{92.735992pt}{158.703156pt}}
\pgflineto{\pgfpoint{92.735992pt}{152.526306pt}}
\pgfpathclose
\pgfusepath{fill,stroke}
\color[rgb]{0.228263,0.325586,0.546335}
\pgfpathmoveto{\pgfpoint{83.807999pt}{164.880005pt}}
\pgflineto{\pgfpoint{92.735992pt}{158.703156pt}}
\pgflineto{\pgfpoint{83.807999pt}{158.703156pt}}
\pgfpathclose
\pgfusepath{fill,stroke}
\pgfpathmoveto{\pgfpoint{83.807999pt}{164.880005pt}}
\pgflineto{\pgfpoint{92.735992pt}{164.880005pt}}
\pgflineto{\pgfpoint{92.735992pt}{158.703156pt}}
\pgfpathclose
\pgfusepath{fill,stroke}
\color[rgb]{0.220425,0.342517,0.549287}
\pgfpathmoveto{\pgfpoint{83.807999pt}{171.056854pt}}
\pgflineto{\pgfpoint{92.735992pt}{164.880005pt}}
\pgflineto{\pgfpoint{83.807999pt}{164.880005pt}}
\pgfpathclose
\pgfusepath{fill,stroke}
\pgfpathmoveto{\pgfpoint{83.807999pt}{171.056854pt}}
\pgflineto{\pgfpoint{92.735992pt}{171.056854pt}}
\pgflineto{\pgfpoint{92.735992pt}{164.880005pt}}
\pgfpathclose
\pgfusepath{fill,stroke}
\pgfpathmoveto{\pgfpoint{83.807999pt}{177.233673pt}}
\pgflineto{\pgfpoint{92.735992pt}{171.056854pt}}
\pgflineto{\pgfpoint{83.807999pt}{171.056854pt}}
\pgfpathclose
\pgfusepath{fill,stroke}
\color[rgb]{0.228263,0.325586,0.546335}
\pgfpathmoveto{\pgfpoint{92.735992pt}{158.703156pt}}
\pgflineto{\pgfpoint{101.664001pt}{152.526306pt}}
\pgflineto{\pgfpoint{92.735992pt}{152.526306pt}}
\pgfpathclose
\pgfusepath{fill,stroke}
\color[rgb]{0.220425,0.342517,0.549287}
\pgfpathmoveto{\pgfpoint{92.735992pt}{164.880005pt}}
\pgflineto{\pgfpoint{101.664001pt}{164.880005pt}}
\pgflineto{\pgfpoint{101.664001pt}{158.703156pt}}
\pgfpathclose
\pgfusepath{fill,stroke}
\color[rgb]{0.212667,0.359102,0.551635}
\pgfpathmoveto{\pgfpoint{92.735992pt}{171.056854pt}}
\pgflineto{\pgfpoint{101.664001pt}{164.880005pt}}
\pgflineto{\pgfpoint{92.735992pt}{164.880005pt}}
\pgfpathclose
\pgfusepath{fill,stroke}
\pgfpathmoveto{\pgfpoint{92.735992pt}{171.056854pt}}
\pgflineto{\pgfpoint{101.664001pt}{171.056854pt}}
\pgflineto{\pgfpoint{101.664001pt}{164.880005pt}}
\pgfpathclose
\pgfusepath{fill,stroke}
\color[rgb]{0.205079,0.375366,0.553493}
\pgfpathmoveto{\pgfpoint{101.664001pt}{171.056854pt}}
\pgflineto{\pgfpoint{110.591980pt}{164.880005pt}}
\pgflineto{\pgfpoint{101.664001pt}{164.880005pt}}
\pgfpathclose
\pgfusepath{fill,stroke}
\color[rgb]{0.183819,0.422564,0.556952}
\pgfpathmoveto{\pgfpoint{74.880005pt}{220.471588pt}}
\pgflineto{\pgfpoint{83.807999pt}{214.294739pt}}
\pgflineto{\pgfpoint{74.880005pt}{214.294739pt}}
\pgfpathclose
\pgfusepath{fill,stroke}
\pgfpathmoveto{\pgfpoint{74.880005pt}{220.471588pt}}
\pgflineto{\pgfpoint{83.807999pt}{220.471588pt}}
\pgflineto{\pgfpoint{83.807999pt}{214.294739pt}}
\pgfpathclose
\pgfusepath{fill,stroke}
\color[rgb]{0.177272,0.437886,0.557576}
\pgfpathmoveto{\pgfpoint{74.880005pt}{226.648422pt}}
\pgflineto{\pgfpoint{83.807999pt}{220.471588pt}}
\pgflineto{\pgfpoint{74.880005pt}{220.471588pt}}
\pgfpathclose
\pgfusepath{fill,stroke}
\color[rgb]{0.152951,0.498053,0.557685}
\pgfpathmoveto{\pgfpoint{83.807999pt}{245.178955pt}}
\pgflineto{\pgfpoint{92.735992pt}{245.178955pt}}
\pgflineto{\pgfpoint{92.735992pt}{239.002106pt}}
\pgfpathclose
\pgfusepath{fill,stroke}
\pgfpathmoveto{\pgfpoint{83.807999pt}{251.355804pt}}
\pgflineto{\pgfpoint{92.735992pt}{245.178955pt}}
\pgflineto{\pgfpoint{83.807999pt}{245.178955pt}}
\pgfpathclose
\pgfusepath{fill,stroke}
\pgfpathmoveto{\pgfpoint{83.807999pt}{251.355804pt}}
\pgflineto{\pgfpoint{92.735992pt}{251.355804pt}}
\pgflineto{\pgfpoint{92.735992pt}{245.178955pt}}
\pgfpathclose
\pgfusepath{fill,stroke}
\color[rgb]{0.147132,0.512959,0.556973}
\pgfpathmoveto{\pgfpoint{92.735992pt}{245.178955pt}}
\pgflineto{\pgfpoint{101.664001pt}{239.002106pt}}
\pgflineto{\pgfpoint{92.735992pt}{239.002106pt}}
\pgfpathclose
\pgfusepath{fill,stroke}
\pgfpathmoveto{\pgfpoint{92.735992pt}{245.178955pt}}
\pgflineto{\pgfpoint{101.664001pt}{245.178955pt}}
\pgflineto{\pgfpoint{101.664001pt}{239.002106pt}}
\pgfpathclose
\pgfusepath{fill,stroke}
\color[rgb]{0.141402,0.527854,0.555864}
\pgfpathmoveto{\pgfpoint{92.735992pt}{251.355804pt}}
\pgflineto{\pgfpoint{101.664001pt}{245.178955pt}}
\pgflineto{\pgfpoint{92.735992pt}{245.178955pt}}
\pgfpathclose
\pgfusepath{fill,stroke}
\pgfpathmoveto{\pgfpoint{92.735992pt}{251.355804pt}}
\pgflineto{\pgfpoint{101.664001pt}{251.355804pt}}
\pgflineto{\pgfpoint{101.664001pt}{245.178955pt}}
\pgfpathclose
\pgfusepath{fill,stroke}
\pgfpathmoveto{\pgfpoint{92.735992pt}{257.532623pt}}
\pgflineto{\pgfpoint{101.664001pt}{251.355804pt}}
\pgflineto{\pgfpoint{92.735992pt}{251.355804pt}}
\pgfpathclose
\pgfusepath{fill,stroke}
\color[rgb]{0.147132,0.512959,0.556973}
\pgfpathmoveto{\pgfpoint{74.880005pt}{263.709473pt}}
\pgflineto{\pgfpoint{83.807999pt}{263.709473pt}}
\pgflineto{\pgfpoint{83.807999pt}{257.532623pt}}
\pgfpathclose
\pgfusepath{fill,stroke}
\color[rgb]{0.141402,0.527854,0.555864}
\pgfpathmoveto{\pgfpoint{74.880005pt}{269.886322pt}}
\pgflineto{\pgfpoint{83.807999pt}{263.709473pt}}
\pgflineto{\pgfpoint{74.880005pt}{263.709473pt}}
\pgfpathclose
\pgfusepath{fill,stroke}
\pgfpathmoveto{\pgfpoint{74.880005pt}{269.886322pt}}
\pgflineto{\pgfpoint{83.807999pt}{269.886322pt}}
\pgflineto{\pgfpoint{83.807999pt}{263.709473pt}}
\pgfpathclose
\pgfusepath{fill,stroke}
\color[rgb]{0.135833,0.542750,0.554289}
\pgfpathmoveto{\pgfpoint{74.880005pt}{276.063141pt}}
\pgflineto{\pgfpoint{83.807999pt}{269.886322pt}}
\pgflineto{\pgfpoint{74.880005pt}{269.886322pt}}
\pgfpathclose
\pgfusepath{fill,stroke}
\color[rgb]{0.147132,0.512959,0.556973}
\pgfpathmoveto{\pgfpoint{83.807999pt}{257.532623pt}}
\pgflineto{\pgfpoint{92.735992pt}{251.355804pt}}
\pgflineto{\pgfpoint{83.807999pt}{251.355804pt}}
\pgfpathclose
\pgfusepath{fill,stroke}
\pgfpathmoveto{\pgfpoint{83.807999pt}{257.532623pt}}
\pgflineto{\pgfpoint{92.735992pt}{257.532623pt}}
\pgflineto{\pgfpoint{92.735992pt}{251.355804pt}}
\pgfpathclose
\pgfusepath{fill,stroke}
\color[rgb]{0.141402,0.527854,0.555864}
\pgfpathmoveto{\pgfpoint{83.807999pt}{263.709473pt}}
\pgflineto{\pgfpoint{92.735992pt}{257.532623pt}}
\pgflineto{\pgfpoint{83.807999pt}{257.532623pt}}
\pgfpathclose
\pgfusepath{fill,stroke}
\pgfpathmoveto{\pgfpoint{83.807999pt}{263.709473pt}}
\pgflineto{\pgfpoint{92.735992pt}{263.709473pt}}
\pgflineto{\pgfpoint{92.735992pt}{257.532623pt}}
\pgfpathclose
\pgfusepath{fill,stroke}
\color[rgb]{0.135833,0.542750,0.554289}
\pgfpathmoveto{\pgfpoint{83.807999pt}{269.886322pt}}
\pgflineto{\pgfpoint{92.735992pt}{263.709473pt}}
\pgflineto{\pgfpoint{83.807999pt}{263.709473pt}}
\pgfpathclose
\pgfusepath{fill,stroke}
\color[rgb]{0.141402,0.527854,0.555864}
\pgfpathmoveto{\pgfpoint{92.735992pt}{257.532623pt}}
\pgflineto{\pgfpoint{101.664001pt}{257.532623pt}}
\pgflineto{\pgfpoint{101.664001pt}{251.355804pt}}
\pgfpathclose
\pgfusepath{fill,stroke}
\color[rgb]{0.135833,0.542750,0.554289}
\pgfpathmoveto{\pgfpoint{92.735992pt}{263.709473pt}}
\pgflineto{\pgfpoint{101.664001pt}{257.532623pt}}
\pgflineto{\pgfpoint{92.735992pt}{257.532623pt}}
\pgfpathclose
\pgfusepath{fill,stroke}
\pgfpathmoveto{\pgfpoint{92.735992pt}{263.709473pt}}
\pgflineto{\pgfpoint{101.664001pt}{263.709473pt}}
\pgflineto{\pgfpoint{101.664001pt}{257.532623pt}}
\pgfpathclose
\pgfusepath{fill,stroke}
\pgfpathmoveto{\pgfpoint{101.664001pt}{257.532623pt}}
\pgflineto{\pgfpoint{110.591980pt}{251.355804pt}}
\pgflineto{\pgfpoint{101.664001pt}{251.355804pt}}
\pgfpathclose
\pgfusepath{fill,stroke}
\pgfpathmoveto{\pgfpoint{101.664001pt}{257.532623pt}}
\pgflineto{\pgfpoint{110.591980pt}{257.532623pt}}
\pgflineto{\pgfpoint{110.591980pt}{251.355804pt}}
\pgfpathclose
\pgfusepath{fill,stroke}
\color[rgb]{0.130582,0.557652,0.552176}
\pgfpathmoveto{\pgfpoint{101.664001pt}{263.709473pt}}
\pgflineto{\pgfpoint{110.591980pt}{257.532623pt}}
\pgflineto{\pgfpoint{101.664001pt}{257.532623pt}}
\pgfpathclose
\pgfusepath{fill,stroke}
\color[rgb]{0.125898,0.572563,0.549445}
\pgfpathmoveto{\pgfpoint{101.664001pt}{269.886322pt}}
\pgflineto{\pgfpoint{110.591980pt}{269.886322pt}}
\pgflineto{\pgfpoint{110.591980pt}{263.709473pt}}
\pgfpathclose
\pgfusepath{fill,stroke}
\color[rgb]{0.122163,0.587476,0.546023}
\pgfpathmoveto{\pgfpoint{101.664001pt}{276.063141pt}}
\pgflineto{\pgfpoint{110.591980pt}{269.886322pt}}
\pgflineto{\pgfpoint{101.664001pt}{269.886322pt}}
\pgfpathclose
\pgfusepath{fill,stroke}
\pgfpathmoveto{\pgfpoint{101.664001pt}{276.063141pt}}
\pgflineto{\pgfpoint{110.591980pt}{276.063141pt}}
\pgflineto{\pgfpoint{110.591980pt}{269.886322pt}}
\pgfpathclose
\pgfusepath{fill,stroke}
\pgfpathmoveto{\pgfpoint{110.591980pt}{269.886322pt}}
\pgflineto{\pgfpoint{119.519989pt}{263.709473pt}}
\pgflineto{\pgfpoint{110.591980pt}{263.709473pt}}
\pgfpathclose
\pgfusepath{fill,stroke}
\pgfpathmoveto{\pgfpoint{110.591980pt}{269.886322pt}}
\pgflineto{\pgfpoint{119.519989pt}{269.886322pt}}
\pgflineto{\pgfpoint{119.519989pt}{263.709473pt}}
\pgfpathclose
\pgfusepath{fill,stroke}
\color[rgb]{0.119872,0.602382,0.541831}
\pgfpathmoveto{\pgfpoint{110.591980pt}{276.063141pt}}
\pgflineto{\pgfpoint{119.519989pt}{269.886322pt}}
\pgflineto{\pgfpoint{110.591980pt}{269.886322pt}}
\pgfpathclose
\pgfusepath{fill,stroke}
\pgfpathmoveto{\pgfpoint{110.591980pt}{276.063141pt}}
\pgflineto{\pgfpoint{119.519989pt}{276.063141pt}}
\pgflineto{\pgfpoint{119.519989pt}{269.886322pt}}
\pgfpathclose
\pgfusepath{fill,stroke}
\color[rgb]{0.119627,0.617266,0.536796}
\pgfpathmoveto{\pgfpoint{110.591980pt}{282.239990pt}}
\pgflineto{\pgfpoint{119.519989pt}{276.063141pt}}
\pgflineto{\pgfpoint{110.591980pt}{276.063141pt}}
\pgfpathclose
\pgfusepath{fill,stroke}
\color[rgb]{0.119872,0.602382,0.541831}
\pgfpathmoveto{\pgfpoint{92.735992pt}{288.416840pt}}
\pgflineto{\pgfpoint{101.664001pt}{288.416840pt}}
\pgflineto{\pgfpoint{101.664001pt}{282.239990pt}}
\pgfpathclose
\pgfusepath{fill,stroke}
\pgfpathmoveto{\pgfpoint{92.735992pt}{294.593689pt}}
\pgflineto{\pgfpoint{101.664001pt}{288.416840pt}}
\pgflineto{\pgfpoint{92.735992pt}{288.416840pt}}
\pgfpathclose
\pgfusepath{fill,stroke}
\pgfpathmoveto{\pgfpoint{92.735992pt}{294.593689pt}}
\pgflineto{\pgfpoint{101.664001pt}{294.593689pt}}
\pgflineto{\pgfpoint{101.664001pt}{288.416840pt}}
\pgfpathclose
\pgfusepath{fill,stroke}
\pgfpathmoveto{\pgfpoint{101.664001pt}{282.239990pt}}
\pgflineto{\pgfpoint{110.591980pt}{276.063141pt}}
\pgflineto{\pgfpoint{101.664001pt}{276.063141pt}}
\pgfpathclose
\pgfusepath{fill,stroke}
\pgfpathmoveto{\pgfpoint{101.664001pt}{282.239990pt}}
\pgflineto{\pgfpoint{110.591980pt}{282.239990pt}}
\pgflineto{\pgfpoint{110.591980pt}{276.063141pt}}
\pgfpathclose
\pgfusepath{fill,stroke}
\color[rgb]{0.119627,0.617266,0.536796}
\pgfpathmoveto{\pgfpoint{101.664001pt}{288.416840pt}}
\pgflineto{\pgfpoint{110.591980pt}{282.239990pt}}
\pgflineto{\pgfpoint{101.664001pt}{282.239990pt}}
\pgfpathclose
\pgfusepath{fill,stroke}
\pgfpathmoveto{\pgfpoint{101.664001pt}{288.416840pt}}
\pgflineto{\pgfpoint{110.591980pt}{288.416840pt}}
\pgflineto{\pgfpoint{110.591980pt}{282.239990pt}}
\pgfpathclose
\pgfusepath{fill,stroke}
\color[rgb]{0.122046,0.632107,0.530848}
\pgfpathmoveto{\pgfpoint{101.664001pt}{294.593689pt}}
\pgflineto{\pgfpoint{110.591980pt}{288.416840pt}}
\pgflineto{\pgfpoint{101.664001pt}{288.416840pt}}
\pgfpathclose
\pgfusepath{fill,stroke}
\color[rgb]{0.119627,0.617266,0.536796}
\pgfpathmoveto{\pgfpoint{110.591980pt}{282.239990pt}}
\pgflineto{\pgfpoint{119.519989pt}{282.239990pt}}
\pgflineto{\pgfpoint{119.519989pt}{276.063141pt}}
\pgfpathclose
\pgfusepath{fill,stroke}
\color[rgb]{0.122046,0.632107,0.530848}
\pgfpathmoveto{\pgfpoint{110.591980pt}{288.416840pt}}
\pgflineto{\pgfpoint{119.519989pt}{282.239990pt}}
\pgflineto{\pgfpoint{110.591980pt}{282.239990pt}}
\pgfpathclose
\pgfusepath{fill,stroke}
\pgfpathmoveto{\pgfpoint{110.591980pt}{288.416840pt}}
\pgflineto{\pgfpoint{119.519989pt}{288.416840pt}}
\pgflineto{\pgfpoint{119.519989pt}{282.239990pt}}
\pgfpathclose
\pgfusepath{fill,stroke}
\pgfpathmoveto{\pgfpoint{119.519989pt}{282.239990pt}}
\pgflineto{\pgfpoint{128.447998pt}{276.063141pt}}
\pgflineto{\pgfpoint{119.519989pt}{276.063141pt}}
\pgfpathclose
\pgfusepath{fill,stroke}
\pgfpathmoveto{\pgfpoint{119.519989pt}{282.239990pt}}
\pgflineto{\pgfpoint{128.447998pt}{282.239990pt}}
\pgflineto{\pgfpoint{128.447998pt}{276.063141pt}}
\pgfpathclose
\pgfusepath{fill,stroke}
\color[rgb]{0.127668,0.646882,0.523924}
\pgfpathmoveto{\pgfpoint{119.519989pt}{288.416840pt}}
\pgflineto{\pgfpoint{128.447998pt}{282.239990pt}}
\pgflineto{\pgfpoint{119.519989pt}{282.239990pt}}
\pgfpathclose
\pgfusepath{fill,stroke}
\color[rgb]{0.119627,0.617266,0.536796}
\pgfpathmoveto{\pgfpoint{92.735992pt}{300.770538pt}}
\pgflineto{\pgfpoint{101.664001pt}{294.593689pt}}
\pgflineto{\pgfpoint{92.735992pt}{294.593689pt}}
\pgfpathclose
\pgfusepath{fill,stroke}
\pgfpathmoveto{\pgfpoint{92.735992pt}{300.770538pt}}
\pgflineto{\pgfpoint{101.664001pt}{300.770538pt}}
\pgflineto{\pgfpoint{101.664001pt}{294.593689pt}}
\pgfpathclose
\pgfusepath{fill,stroke}
\color[rgb]{0.122046,0.632107,0.530848}
\pgfpathmoveto{\pgfpoint{92.735992pt}{306.947388pt}}
\pgflineto{\pgfpoint{101.664001pt}{306.947388pt}}
\pgflineto{\pgfpoint{101.664001pt}{300.770538pt}}
\pgfpathclose
\pgfusepath{fill,stroke}
\color[rgb]{0.127668,0.646882,0.523924}
\pgfpathmoveto{\pgfpoint{92.735992pt}{313.124207pt}}
\pgflineto{\pgfpoint{101.664001pt}{306.947388pt}}
\pgflineto{\pgfpoint{92.735992pt}{306.947388pt}}
\pgfpathclose
\pgfusepath{fill,stroke}
\color[rgb]{0.122046,0.632107,0.530848}
\pgfpathmoveto{\pgfpoint{101.664001pt}{294.593689pt}}
\pgflineto{\pgfpoint{110.591980pt}{294.593689pt}}
\pgflineto{\pgfpoint{110.591980pt}{288.416840pt}}
\pgfpathclose
\pgfusepath{fill,stroke}
\pgfpathmoveto{\pgfpoint{101.664001pt}{300.770538pt}}
\pgflineto{\pgfpoint{110.591980pt}{294.593689pt}}
\pgflineto{\pgfpoint{101.664001pt}{294.593689pt}}
\pgfpathclose
\pgfusepath{fill,stroke}
\color[rgb]{0.127668,0.646882,0.523924}
\pgfpathmoveto{\pgfpoint{110.591980pt}{294.593689pt}}
\pgflineto{\pgfpoint{119.519989pt}{288.416840pt}}
\pgflineto{\pgfpoint{110.591980pt}{288.416840pt}}
\pgfpathclose
\pgfusepath{fill,stroke}
\pgfpathmoveto{\pgfpoint{110.591980pt}{294.593689pt}}
\pgflineto{\pgfpoint{119.519989pt}{294.593689pt}}
\pgflineto{\pgfpoint{119.519989pt}{288.416840pt}}
\pgfpathclose
\pgfusepath{fill,stroke}
\pgfpathmoveto{\pgfpoint{110.591980pt}{300.770538pt}}
\pgflineto{\pgfpoint{119.519989pt}{300.770538pt}}
\pgflineto{\pgfpoint{119.519989pt}{294.593689pt}}
\pgfpathclose
\pgfusepath{fill,stroke}
\color[rgb]{0.136835,0.661563,0.515967}
\pgfpathmoveto{\pgfpoint{110.591980pt}{306.947388pt}}
\pgflineto{\pgfpoint{119.519989pt}{300.770538pt}}
\pgflineto{\pgfpoint{110.591980pt}{300.770538pt}}
\pgfpathclose
\pgfusepath{fill,stroke}
\color[rgb]{0.127668,0.646882,0.523924}
\pgfpathmoveto{\pgfpoint{119.519989pt}{288.416840pt}}
\pgflineto{\pgfpoint{128.447998pt}{288.416840pt}}
\pgflineto{\pgfpoint{128.447998pt}{282.239990pt}}
\pgfpathclose
\pgfusepath{fill,stroke}
\color[rgb]{0.136835,0.661563,0.515967}
\pgfpathmoveto{\pgfpoint{119.519989pt}{294.593689pt}}
\pgflineto{\pgfpoint{128.447998pt}{288.416840pt}}
\pgflineto{\pgfpoint{119.519989pt}{288.416840pt}}
\pgfpathclose
\pgfusepath{fill,stroke}
\pgfpathmoveto{\pgfpoint{119.519989pt}{294.593689pt}}
\pgflineto{\pgfpoint{128.447998pt}{294.593689pt}}
\pgflineto{\pgfpoint{128.447998pt}{288.416840pt}}
\pgfpathclose
\pgfusepath{fill,stroke}
\color[rgb]{0.149643,0.676120,0.506924}
\pgfpathmoveto{\pgfpoint{119.519989pt}{300.770538pt}}
\pgflineto{\pgfpoint{128.447998pt}{294.593689pt}}
\pgflineto{\pgfpoint{119.519989pt}{294.593689pt}}
\pgfpathclose
\pgfusepath{fill,stroke}
\color[rgb]{0.122046,0.632107,0.530848}
\pgfpathmoveto{\pgfpoint{74.880005pt}{319.301056pt}}
\pgflineto{\pgfpoint{83.807999pt}{319.301056pt}}
\pgflineto{\pgfpoint{83.807999pt}{313.124207pt}}
\pgfpathclose
\pgfusepath{fill,stroke}
\pgfpathmoveto{\pgfpoint{74.880005pt}{325.477905pt}}
\pgflineto{\pgfpoint{83.807999pt}{319.301056pt}}
\pgflineto{\pgfpoint{74.880005pt}{319.301056pt}}
\pgfpathclose
\pgfusepath{fill,stroke}
\pgfpathmoveto{\pgfpoint{74.880005pt}{325.477905pt}}
\pgflineto{\pgfpoint{83.807999pt}{325.477905pt}}
\pgflineto{\pgfpoint{83.807999pt}{319.301056pt}}
\pgfpathclose
\pgfusepath{fill,stroke}
\color[rgb]{0.119627,0.617266,0.536796}
\pgfpathmoveto{\pgfpoint{83.807999pt}{306.947388pt}}
\pgflineto{\pgfpoint{92.735992pt}{306.947388pt}}
\pgflineto{\pgfpoint{92.735992pt}{300.770538pt}}
\pgfpathclose
\pgfusepath{fill,stroke}
\color[rgb]{0.122046,0.632107,0.530848}
\pgfpathmoveto{\pgfpoint{83.807999pt}{313.124207pt}}
\pgflineto{\pgfpoint{92.735992pt}{306.947388pt}}
\pgflineto{\pgfpoint{83.807999pt}{306.947388pt}}
\pgfpathclose
\pgfusepath{fill,stroke}
\pgfpathmoveto{\pgfpoint{83.807999pt}{313.124207pt}}
\pgflineto{\pgfpoint{92.735992pt}{313.124207pt}}
\pgflineto{\pgfpoint{92.735992pt}{306.947388pt}}
\pgfpathclose
\pgfusepath{fill,stroke}
\color[rgb]{0.127668,0.646882,0.523924}
\pgfpathmoveto{\pgfpoint{83.807999pt}{319.301056pt}}
\pgflineto{\pgfpoint{92.735992pt}{313.124207pt}}
\pgflineto{\pgfpoint{83.807999pt}{313.124207pt}}
\pgfpathclose
\pgfusepath{fill,stroke}
\pgfpathmoveto{\pgfpoint{83.807999pt}{319.301056pt}}
\pgflineto{\pgfpoint{92.735992pt}{319.301056pt}}
\pgflineto{\pgfpoint{92.735992pt}{313.124207pt}}
\pgfpathclose
\pgfusepath{fill,stroke}
\color[rgb]{0.136835,0.661563,0.515967}
\pgfpathmoveto{\pgfpoint{83.807999pt}{325.477905pt}}
\pgflineto{\pgfpoint{92.735992pt}{319.301056pt}}
\pgflineto{\pgfpoint{83.807999pt}{319.301056pt}}
\pgfpathclose
\pgfusepath{fill,stroke}
\color[rgb]{0.122046,0.632107,0.530848}
\pgfpathmoveto{\pgfpoint{92.735992pt}{306.947388pt}}
\pgflineto{\pgfpoint{101.664001pt}{300.770538pt}}
\pgflineto{\pgfpoint{92.735992pt}{300.770538pt}}
\pgfpathclose
\pgfusepath{fill,stroke}
\color[rgb]{0.127668,0.646882,0.523924}
\pgfpathmoveto{\pgfpoint{92.735992pt}{313.124207pt}}
\pgflineto{\pgfpoint{101.664001pt}{313.124207pt}}
\pgflineto{\pgfpoint{101.664001pt}{306.947388pt}}
\pgfpathclose
\pgfusepath{fill,stroke}
\color[rgb]{0.136835,0.661563,0.515967}
\pgfpathmoveto{\pgfpoint{92.735992pt}{319.301056pt}}
\pgflineto{\pgfpoint{101.664001pt}{313.124207pt}}
\pgflineto{\pgfpoint{92.735992pt}{313.124207pt}}
\pgfpathclose
\pgfusepath{fill,stroke}
\pgfpathmoveto{\pgfpoint{92.735992pt}{319.301056pt}}
\pgflineto{\pgfpoint{101.664001pt}{319.301056pt}}
\pgflineto{\pgfpoint{101.664001pt}{313.124207pt}}
\pgfpathclose
\pgfusepath{fill,stroke}
\color[rgb]{0.122046,0.632107,0.530848}
\pgfpathmoveto{\pgfpoint{101.664001pt}{300.770538pt}}
\pgflineto{\pgfpoint{110.591980pt}{300.770538pt}}
\pgflineto{\pgfpoint{110.591980pt}{294.593689pt}}
\pgfpathclose
\pgfusepath{fill,stroke}
\color[rgb]{0.127668,0.646882,0.523924}
\pgfpathmoveto{\pgfpoint{101.664001pt}{306.947388pt}}
\pgflineto{\pgfpoint{110.591980pt}{300.770538pt}}
\pgflineto{\pgfpoint{101.664001pt}{300.770538pt}}
\pgfpathclose
\pgfusepath{fill,stroke}
\pgfpathmoveto{\pgfpoint{101.664001pt}{306.947388pt}}
\pgflineto{\pgfpoint{110.591980pt}{306.947388pt}}
\pgflineto{\pgfpoint{110.591980pt}{300.770538pt}}
\pgfpathclose
\pgfusepath{fill,stroke}
\color[rgb]{0.136835,0.661563,0.515967}
\pgfpathmoveto{\pgfpoint{101.664001pt}{313.124207pt}}
\pgflineto{\pgfpoint{110.591980pt}{306.947388pt}}
\pgflineto{\pgfpoint{101.664001pt}{306.947388pt}}
\pgfpathclose
\pgfusepath{fill,stroke}
\pgfpathmoveto{\pgfpoint{101.664001pt}{313.124207pt}}
\pgflineto{\pgfpoint{110.591980pt}{313.124207pt}}
\pgflineto{\pgfpoint{110.591980pt}{306.947388pt}}
\pgfpathclose
\pgfusepath{fill,stroke}
\color[rgb]{0.149643,0.676120,0.506924}
\pgfpathmoveto{\pgfpoint{101.664001pt}{319.301056pt}}
\pgflineto{\pgfpoint{110.591980pt}{313.124207pt}}
\pgflineto{\pgfpoint{101.664001pt}{313.124207pt}}
\pgfpathclose
\pgfusepath{fill,stroke}
\color[rgb]{0.127668,0.646882,0.523924}
\pgfpathmoveto{\pgfpoint{110.591980pt}{300.770538pt}}
\pgflineto{\pgfpoint{119.519989pt}{294.593689pt}}
\pgflineto{\pgfpoint{110.591980pt}{294.593689pt}}
\pgfpathclose
\pgfusepath{fill,stroke}
\color[rgb]{0.136835,0.661563,0.515967}
\pgfpathmoveto{\pgfpoint{110.591980pt}{306.947388pt}}
\pgflineto{\pgfpoint{119.519989pt}{306.947388pt}}
\pgflineto{\pgfpoint{119.519989pt}{300.770538pt}}
\pgfpathclose
\pgfusepath{fill,stroke}
\color[rgb]{0.149643,0.676120,0.506924}
\pgfpathmoveto{\pgfpoint{110.591980pt}{313.124207pt}}
\pgflineto{\pgfpoint{119.519989pt}{306.947388pt}}
\pgflineto{\pgfpoint{110.591980pt}{306.947388pt}}
\pgfpathclose
\pgfusepath{fill,stroke}
\pgfpathmoveto{\pgfpoint{110.591980pt}{313.124207pt}}
\pgflineto{\pgfpoint{119.519989pt}{313.124207pt}}
\pgflineto{\pgfpoint{119.519989pt}{306.947388pt}}
\pgfpathclose
\pgfusepath{fill,stroke}
\pgfpathmoveto{\pgfpoint{119.519989pt}{300.770538pt}}
\pgflineto{\pgfpoint{128.447998pt}{300.770538pt}}
\pgflineto{\pgfpoint{128.447998pt}{294.593689pt}}
\pgfpathclose
\pgfusepath{fill,stroke}
\pgfpathmoveto{\pgfpoint{119.519989pt}{306.947388pt}}
\pgflineto{\pgfpoint{128.447998pt}{300.770538pt}}
\pgflineto{\pgfpoint{119.519989pt}{300.770538pt}}
\pgfpathclose
\pgfusepath{fill,stroke}
\pgfpathmoveto{\pgfpoint{119.519989pt}{306.947388pt}}
\pgflineto{\pgfpoint{128.447998pt}{306.947388pt}}
\pgflineto{\pgfpoint{128.447998pt}{300.770538pt}}
\pgfpathclose
\pgfusepath{fill,stroke}
\color[rgb]{0.165967,0.690519,0.496752}
\pgfpathmoveto{\pgfpoint{119.519989pt}{313.124207pt}}
\pgflineto{\pgfpoint{128.447998pt}{306.947388pt}}
\pgflineto{\pgfpoint{119.519989pt}{306.947388pt}}
\pgfpathclose
\pgfusepath{fill,stroke}
\pgfpathmoveto{\pgfpoint{128.447998pt}{300.770538pt}}
\pgflineto{\pgfpoint{137.376007pt}{294.593689pt}}
\pgflineto{\pgfpoint{128.447998pt}{294.593689pt}}
\pgfpathclose
\pgfusepath{fill,stroke}
\pgfpathmoveto{\pgfpoint{128.447998pt}{300.770538pt}}
\pgflineto{\pgfpoint{137.376007pt}{300.770538pt}}
\pgflineto{\pgfpoint{137.376007pt}{294.593689pt}}
\pgfpathclose
\pgfusepath{fill,stroke}
\pgfpathmoveto{\pgfpoint{128.447998pt}{306.947388pt}}
\pgflineto{\pgfpoint{137.376007pt}{300.770538pt}}
\pgflineto{\pgfpoint{128.447998pt}{300.770538pt}}
\pgfpathclose
\pgfusepath{fill,stroke}
\color[rgb]{0.208030,0.718701,0.472873}
\pgfpathmoveto{\pgfpoint{74.880005pt}{368.715820pt}}
\pgflineto{\pgfpoint{83.807999pt}{362.538940pt}}
\pgflineto{\pgfpoint{74.880005pt}{362.538940pt}}
\pgfpathclose
\pgfusepath{fill,stroke}
\pgfpathmoveto{\pgfpoint{74.880005pt}{368.715820pt}}
\pgflineto{\pgfpoint{83.807999pt}{368.715820pt}}
\pgflineto{\pgfpoint{83.807999pt}{362.538940pt}}
\pgfpathclose
\pgfusepath{fill,stroke}
\color[rgb]{0.233127,0.732406,0.459106}
\pgfpathmoveto{\pgfpoint{74.880005pt}{374.892639pt}}
\pgflineto{\pgfpoint{83.807999pt}{368.715820pt}}
\pgflineto{\pgfpoint{74.880005pt}{368.715820pt}}
\pgfpathclose
\pgfusepath{fill,stroke}
\pgfpathmoveto{\pgfpoint{74.880005pt}{374.892639pt}}
\pgflineto{\pgfpoint{83.807999pt}{374.892639pt}}
\pgflineto{\pgfpoint{83.807999pt}{368.715820pt}}
\pgfpathclose
\pgfusepath{fill,stroke}
\color[rgb]{0.260531,0.745802,0.444096}
\pgfpathmoveto{\pgfpoint{83.807999pt}{374.892639pt}}
\pgflineto{\pgfpoint{92.735992pt}{368.715820pt}}
\pgflineto{\pgfpoint{83.807999pt}{368.715820pt}}
\pgfpathclose
\pgfusepath{fill,stroke}
\pgfpathmoveto{\pgfpoint{74.880005pt}{381.069458pt}}
\pgflineto{\pgfpoint{83.807999pt}{374.892639pt}}
\pgflineto{\pgfpoint{74.880005pt}{374.892639pt}}
\pgfpathclose
\pgfusepath{fill,stroke}
\pgfpathmoveto{\pgfpoint{74.880005pt}{381.069458pt}}
\pgflineto{\pgfpoint{83.807999pt}{381.069458pt}}
\pgflineto{\pgfpoint{83.807999pt}{374.892639pt}}
\pgfpathclose
\pgfusepath{fill,stroke}
\color[rgb]{0.290001,0.758846,0.427826}
\pgfpathmoveto{\pgfpoint{74.880005pt}{387.246338pt}}
\pgflineto{\pgfpoint{83.807999pt}{381.069458pt}}
\pgflineto{\pgfpoint{74.880005pt}{381.069458pt}}
\pgfpathclose
\pgfusepath{fill,stroke}
\pgfpathmoveto{\pgfpoint{74.880005pt}{387.246338pt}}
\pgflineto{\pgfpoint{83.807999pt}{387.246338pt}}
\pgflineto{\pgfpoint{83.807999pt}{381.069458pt}}
\pgfpathclose
\pgfusepath{fill,stroke}
\color[rgb]{0.321330,0.771498,0.410293}
\pgfpathmoveto{\pgfpoint{74.880005pt}{393.423157pt}}
\pgflineto{\pgfpoint{83.807999pt}{387.246338pt}}
\pgflineto{\pgfpoint{74.880005pt}{387.246338pt}}
\pgfpathclose
\pgfusepath{fill,stroke}
\pgfpathmoveto{\pgfpoint{74.880005pt}{393.423157pt}}
\pgflineto{\pgfpoint{83.807999pt}{393.423157pt}}
\pgflineto{\pgfpoint{83.807999pt}{387.246338pt}}
\pgfpathclose
\pgfusepath{fill,stroke}
\color[rgb]{0.354355,0.783714,0.391488}
\pgfpathmoveto{\pgfpoint{74.880005pt}{399.600037pt}}
\pgflineto{\pgfpoint{83.807999pt}{393.423157pt}}
\pgflineto{\pgfpoint{74.880005pt}{393.423157pt}}
\pgfpathclose
\pgfusepath{fill,stroke}
\pgfpathmoveto{\pgfpoint{74.880005pt}{399.600037pt}}
\pgflineto{\pgfpoint{83.807999pt}{399.600037pt}}
\pgflineto{\pgfpoint{83.807999pt}{393.423157pt}}
\pgfpathclose
\pgfusepath{fill,stroke}
\color[rgb]{0.260531,0.745802,0.444096}
\pgfpathmoveto{\pgfpoint{83.807999pt}{374.892639pt}}
\pgflineto{\pgfpoint{92.735992pt}{374.892639pt}}
\pgflineto{\pgfpoint{92.735992pt}{368.715820pt}}
\pgfpathclose
\pgfusepath{fill,stroke}
\color[rgb]{0.290001,0.758846,0.427826}
\pgfpathmoveto{\pgfpoint{83.807999pt}{381.069458pt}}
\pgflineto{\pgfpoint{92.735992pt}{374.892639pt}}
\pgflineto{\pgfpoint{83.807999pt}{374.892639pt}}
\pgfpathclose
\pgfusepath{fill,stroke}
\pgfpathmoveto{\pgfpoint{83.807999pt}{381.069458pt}}
\pgflineto{\pgfpoint{92.735992pt}{381.069458pt}}
\pgflineto{\pgfpoint{92.735992pt}{374.892639pt}}
\pgfpathclose
\pgfusepath{fill,stroke}
\color[rgb]{0.321330,0.771498,0.410293}
\pgfpathmoveto{\pgfpoint{83.807999pt}{387.246338pt}}
\pgflineto{\pgfpoint{92.735992pt}{381.069458pt}}
\pgflineto{\pgfpoint{83.807999pt}{381.069458pt}}
\pgfpathclose
\pgfusepath{fill,stroke}
\pgfpathmoveto{\pgfpoint{83.807999pt}{387.246338pt}}
\pgflineto{\pgfpoint{92.735992pt}{387.246338pt}}
\pgflineto{\pgfpoint{92.735992pt}{381.069458pt}}
\pgfpathclose
\pgfusepath{fill,stroke}
\color[rgb]{0.354355,0.783714,0.391488}
\pgfpathmoveto{\pgfpoint{83.807999pt}{393.423157pt}}
\pgflineto{\pgfpoint{92.735992pt}{387.246338pt}}
\pgflineto{\pgfpoint{83.807999pt}{387.246338pt}}
\pgfpathclose
\pgfusepath{fill,stroke}
\pgfpathmoveto{\pgfpoint{83.807999pt}{393.423157pt}}
\pgflineto{\pgfpoint{92.735992pt}{393.423157pt}}
\pgflineto{\pgfpoint{92.735992pt}{387.246338pt}}
\pgfpathclose
\pgfusepath{fill,stroke}
\color[rgb]{0.388930,0.795453,0.371421}
\pgfpathmoveto{\pgfpoint{83.807999pt}{399.600037pt}}
\pgflineto{\pgfpoint{92.735992pt}{393.423157pt}}
\pgflineto{\pgfpoint{83.807999pt}{393.423157pt}}
\pgfpathclose
\pgfusepath{fill,stroke}
\pgfpathmoveto{\pgfpoint{83.807999pt}{399.600037pt}}
\pgflineto{\pgfpoint{92.735992pt}{399.600037pt}}
\pgflineto{\pgfpoint{92.735992pt}{393.423157pt}}
\pgfpathclose
\pgfusepath{fill,stroke}
\color[rgb]{0.354355,0.783714,0.391488}
\pgfpathmoveto{\pgfpoint{92.735992pt}{387.246338pt}}
\pgflineto{\pgfpoint{101.664001pt}{381.069458pt}}
\pgflineto{\pgfpoint{92.735992pt}{381.069458pt}}
\pgfpathclose
\pgfusepath{fill,stroke}
\pgfpathmoveto{\pgfpoint{92.735992pt}{387.246338pt}}
\pgflineto{\pgfpoint{101.664001pt}{387.246338pt}}
\pgflineto{\pgfpoint{101.664001pt}{381.069458pt}}
\pgfpathclose
\pgfusepath{fill,stroke}
\color[rgb]{0.388930,0.795453,0.371421}
\pgfpathmoveto{\pgfpoint{92.735992pt}{393.423157pt}}
\pgflineto{\pgfpoint{101.664001pt}{387.246338pt}}
\pgflineto{\pgfpoint{92.735992pt}{387.246338pt}}
\pgfpathclose
\pgfusepath{fill,stroke}
\pgfpathmoveto{\pgfpoint{92.735992pt}{393.423157pt}}
\pgflineto{\pgfpoint{101.664001pt}{393.423157pt}}
\pgflineto{\pgfpoint{101.664001pt}{387.246338pt}}
\pgfpathclose
\pgfusepath{fill,stroke}
\color[rgb]{0.424933,0.806674,0.350099}
\pgfpathmoveto{\pgfpoint{92.735992pt}{399.600037pt}}
\pgflineto{\pgfpoint{101.664001pt}{393.423157pt}}
\pgflineto{\pgfpoint{92.735992pt}{393.423157pt}}
\pgfpathclose
\pgfusepath{fill,stroke}
\pgfpathmoveto{\pgfpoint{92.735992pt}{399.600037pt}}
\pgflineto{\pgfpoint{101.664001pt}{399.600037pt}}
\pgflineto{\pgfpoint{101.664001pt}{393.423157pt}}
\pgfpathclose
\pgfusepath{fill,stroke}
\color[rgb]{0.462247,0.817338,0.327545}
\pgfpathmoveto{\pgfpoint{101.664001pt}{399.600037pt}}
\pgflineto{\pgfpoint{110.591980pt}{393.423157pt}}
\pgflineto{\pgfpoint{101.664001pt}{393.423157pt}}
\pgfpathclose
\pgfusepath{fill,stroke}
\pgfpathmoveto{\pgfpoint{101.664001pt}{399.600037pt}}
\pgflineto{\pgfpoint{110.591980pt}{399.600037pt}}
\pgflineto{\pgfpoint{110.591980pt}{393.423157pt}}
\pgfpathclose
\pgfusepath{fill,stroke}
\color[rgb]{0.127668,0.646882,0.523924}
\pgfpathmoveto{\pgfpoint{74.880005pt}{331.654724pt}}
\pgflineto{\pgfpoint{83.807999pt}{325.477905pt}}
\pgflineto{\pgfpoint{74.880005pt}{325.477905pt}}
\pgfpathclose
\pgfusepath{fill,stroke}
\pgfpathmoveto{\pgfpoint{74.880005pt}{331.654724pt}}
\pgflineto{\pgfpoint{83.807999pt}{331.654724pt}}
\pgflineto{\pgfpoint{83.807999pt}{325.477905pt}}
\pgfpathclose
\pgfusepath{fill,stroke}
\color[rgb]{0.136835,0.661563,0.515967}
\pgfpathmoveto{\pgfpoint{74.880005pt}{337.831604pt}}
\pgflineto{\pgfpoint{83.807999pt}{331.654724pt}}
\pgflineto{\pgfpoint{74.880005pt}{331.654724pt}}
\pgfpathclose
\pgfusepath{fill,stroke}
\pgfpathmoveto{\pgfpoint{74.880005pt}{337.831604pt}}
\pgflineto{\pgfpoint{83.807999pt}{337.831604pt}}
\pgflineto{\pgfpoint{83.807999pt}{331.654724pt}}
\pgfpathclose
\pgfusepath{fill,stroke}
\color[rgb]{0.149643,0.676120,0.506924}
\pgfpathmoveto{\pgfpoint{74.880005pt}{344.008423pt}}
\pgflineto{\pgfpoint{83.807999pt}{337.831604pt}}
\pgflineto{\pgfpoint{74.880005pt}{337.831604pt}}
\pgfpathclose
\pgfusepath{fill,stroke}
\pgfpathmoveto{\pgfpoint{74.880005pt}{344.008423pt}}
\pgflineto{\pgfpoint{83.807999pt}{344.008423pt}}
\pgflineto{\pgfpoint{83.807999pt}{337.831604pt}}
\pgfpathclose
\pgfusepath{fill,stroke}
\color[rgb]{0.165967,0.690519,0.496752}
\pgfpathmoveto{\pgfpoint{74.880005pt}{350.185242pt}}
\pgflineto{\pgfpoint{83.807999pt}{344.008423pt}}
\pgflineto{\pgfpoint{74.880005pt}{344.008423pt}}
\pgfpathclose
\pgfusepath{fill,stroke}
\pgfpathmoveto{\pgfpoint{74.880005pt}{350.185242pt}}
\pgflineto{\pgfpoint{83.807999pt}{350.185242pt}}
\pgflineto{\pgfpoint{83.807999pt}{344.008423pt}}
\pgfpathclose
\pgfusepath{fill,stroke}
\color[rgb]{0.185538,0.704725,0.485412}
\pgfpathmoveto{\pgfpoint{74.880005pt}{356.362122pt}}
\pgflineto{\pgfpoint{83.807999pt}{350.185242pt}}
\pgflineto{\pgfpoint{74.880005pt}{350.185242pt}}
\pgfpathclose
\pgfusepath{fill,stroke}
\pgfpathmoveto{\pgfpoint{74.880005pt}{356.362122pt}}
\pgflineto{\pgfpoint{83.807999pt}{356.362122pt}}
\pgflineto{\pgfpoint{83.807999pt}{350.185242pt}}
\pgfpathclose
\pgfusepath{fill,stroke}
\color[rgb]{0.208030,0.718701,0.472873}
\pgfpathmoveto{\pgfpoint{74.880005pt}{362.538940pt}}
\pgflineto{\pgfpoint{83.807999pt}{356.362122pt}}
\pgflineto{\pgfpoint{74.880005pt}{356.362122pt}}
\pgfpathclose
\pgfusepath{fill,stroke}
\pgfpathmoveto{\pgfpoint{74.880005pt}{362.538940pt}}
\pgflineto{\pgfpoint{83.807999pt}{362.538940pt}}
\pgflineto{\pgfpoint{83.807999pt}{356.362122pt}}
\pgfpathclose
\pgfusepath{fill,stroke}
\color[rgb]{0.136835,0.661563,0.515967}
\pgfpathmoveto{\pgfpoint{83.807999pt}{325.477905pt}}
\pgflineto{\pgfpoint{92.735992pt}{325.477905pt}}
\pgflineto{\pgfpoint{92.735992pt}{319.301056pt}}
\pgfpathclose
\pgfusepath{fill,stroke}
\pgfpathmoveto{\pgfpoint{83.807999pt}{331.654724pt}}
\pgflineto{\pgfpoint{92.735992pt}{325.477905pt}}
\pgflineto{\pgfpoint{83.807999pt}{325.477905pt}}
\pgfpathclose
\pgfusepath{fill,stroke}
\pgfpathmoveto{\pgfpoint{83.807999pt}{331.654724pt}}
\pgflineto{\pgfpoint{92.735992pt}{331.654724pt}}
\pgflineto{\pgfpoint{92.735992pt}{325.477905pt}}
\pgfpathclose
\pgfusepath{fill,stroke}
\color[rgb]{0.149643,0.676120,0.506924}
\pgfpathmoveto{\pgfpoint{83.807999pt}{337.831604pt}}
\pgflineto{\pgfpoint{92.735992pt}{331.654724pt}}
\pgflineto{\pgfpoint{83.807999pt}{331.654724pt}}
\pgfpathclose
\pgfusepath{fill,stroke}
\pgfpathmoveto{\pgfpoint{83.807999pt}{337.831604pt}}
\pgflineto{\pgfpoint{92.735992pt}{337.831604pt}}
\pgflineto{\pgfpoint{92.735992pt}{331.654724pt}}
\pgfpathclose
\pgfusepath{fill,stroke}
\color[rgb]{0.165967,0.690519,0.496752}
\pgfpathmoveto{\pgfpoint{83.807999pt}{344.008423pt}}
\pgflineto{\pgfpoint{92.735992pt}{337.831604pt}}
\pgflineto{\pgfpoint{83.807999pt}{337.831604pt}}
\pgfpathclose
\pgfusepath{fill,stroke}
\pgfpathmoveto{\pgfpoint{83.807999pt}{344.008423pt}}
\pgflineto{\pgfpoint{92.735992pt}{344.008423pt}}
\pgflineto{\pgfpoint{92.735992pt}{337.831604pt}}
\pgfpathclose
\pgfusepath{fill,stroke}
\color[rgb]{0.185538,0.704725,0.485412}
\pgfpathmoveto{\pgfpoint{83.807999pt}{350.185242pt}}
\pgflineto{\pgfpoint{92.735992pt}{344.008423pt}}
\pgflineto{\pgfpoint{83.807999pt}{344.008423pt}}
\pgfpathclose
\pgfusepath{fill,stroke}
\pgfpathmoveto{\pgfpoint{83.807999pt}{350.185242pt}}
\pgflineto{\pgfpoint{92.735992pt}{350.185242pt}}
\pgflineto{\pgfpoint{92.735992pt}{344.008423pt}}
\pgfpathclose
\pgfusepath{fill,stroke}
\color[rgb]{0.208030,0.718701,0.472873}
\pgfpathmoveto{\pgfpoint{83.807999pt}{356.362122pt}}
\pgflineto{\pgfpoint{92.735992pt}{350.185242pt}}
\pgflineto{\pgfpoint{83.807999pt}{350.185242pt}}
\pgfpathclose
\pgfusepath{fill,stroke}
\pgfpathmoveto{\pgfpoint{83.807999pt}{356.362122pt}}
\pgflineto{\pgfpoint{92.735992pt}{356.362122pt}}
\pgflineto{\pgfpoint{92.735992pt}{350.185242pt}}
\pgfpathclose
\pgfusepath{fill,stroke}
\color[rgb]{0.233127,0.732406,0.459106}
\pgfpathmoveto{\pgfpoint{83.807999pt}{362.538940pt}}
\pgflineto{\pgfpoint{92.735992pt}{356.362122pt}}
\pgflineto{\pgfpoint{83.807999pt}{356.362122pt}}
\pgfpathclose
\pgfusepath{fill,stroke}
\pgfpathmoveto{\pgfpoint{83.807999pt}{362.538940pt}}
\pgflineto{\pgfpoint{92.735992pt}{362.538940pt}}
\pgflineto{\pgfpoint{92.735992pt}{356.362122pt}}
\pgfpathclose
\pgfusepath{fill,stroke}
\pgfpathmoveto{\pgfpoint{83.807999pt}{368.715820pt}}
\pgflineto{\pgfpoint{92.735992pt}{362.538940pt}}
\pgflineto{\pgfpoint{83.807999pt}{362.538940pt}}
\pgfpathclose
\pgfusepath{fill,stroke}
\pgfpathmoveto{\pgfpoint{83.807999pt}{368.715820pt}}
\pgflineto{\pgfpoint{92.735992pt}{368.715820pt}}
\pgflineto{\pgfpoint{92.735992pt}{362.538940pt}}
\pgfpathclose
\pgfusepath{fill,stroke}
\color[rgb]{0.149643,0.676120,0.506924}
\pgfpathmoveto{\pgfpoint{92.735992pt}{325.477905pt}}
\pgflineto{\pgfpoint{101.664001pt}{319.301056pt}}
\pgflineto{\pgfpoint{92.735992pt}{319.301056pt}}
\pgfpathclose
\pgfusepath{fill,stroke}
\pgfpathmoveto{\pgfpoint{92.735992pt}{325.477905pt}}
\pgflineto{\pgfpoint{101.664001pt}{325.477905pt}}
\pgflineto{\pgfpoint{101.664001pt}{319.301056pt}}
\pgfpathclose
\pgfusepath{fill,stroke}
\pgfpathmoveto{\pgfpoint{92.735992pt}{331.654724pt}}
\pgflineto{\pgfpoint{101.664001pt}{325.477905pt}}
\pgflineto{\pgfpoint{92.735992pt}{325.477905pt}}
\pgfpathclose
\pgfusepath{fill,stroke}
\pgfpathmoveto{\pgfpoint{92.735992pt}{331.654724pt}}
\pgflineto{\pgfpoint{101.664001pt}{331.654724pt}}
\pgflineto{\pgfpoint{101.664001pt}{325.477905pt}}
\pgfpathclose
\pgfusepath{fill,stroke}
\color[rgb]{0.165967,0.690519,0.496752}
\pgfpathmoveto{\pgfpoint{92.735992pt}{337.831604pt}}
\pgflineto{\pgfpoint{101.664001pt}{331.654724pt}}
\pgflineto{\pgfpoint{92.735992pt}{331.654724pt}}
\pgfpathclose
\pgfusepath{fill,stroke}
\pgfpathmoveto{\pgfpoint{92.735992pt}{337.831604pt}}
\pgflineto{\pgfpoint{101.664001pt}{337.831604pt}}
\pgflineto{\pgfpoint{101.664001pt}{331.654724pt}}
\pgfpathclose
\pgfusepath{fill,stroke}
\color[rgb]{0.185538,0.704725,0.485412}
\pgfpathmoveto{\pgfpoint{92.735992pt}{344.008423pt}}
\pgflineto{\pgfpoint{101.664001pt}{337.831604pt}}
\pgflineto{\pgfpoint{92.735992pt}{337.831604pt}}
\pgfpathclose
\pgfusepath{fill,stroke}
\pgfpathmoveto{\pgfpoint{92.735992pt}{344.008423pt}}
\pgflineto{\pgfpoint{101.664001pt}{344.008423pt}}
\pgflineto{\pgfpoint{101.664001pt}{337.831604pt}}
\pgfpathclose
\pgfusepath{fill,stroke}
\color[rgb]{0.208030,0.718701,0.472873}
\pgfpathmoveto{\pgfpoint{92.735992pt}{350.185242pt}}
\pgflineto{\pgfpoint{101.664001pt}{344.008423pt}}
\pgflineto{\pgfpoint{92.735992pt}{344.008423pt}}
\pgfpathclose
\pgfusepath{fill,stroke}
\pgfpathmoveto{\pgfpoint{92.735992pt}{350.185242pt}}
\pgflineto{\pgfpoint{101.664001pt}{350.185242pt}}
\pgflineto{\pgfpoint{101.664001pt}{344.008423pt}}
\pgfpathclose
\pgfusepath{fill,stroke}
\color[rgb]{0.233127,0.732406,0.459106}
\pgfpathmoveto{\pgfpoint{92.735992pt}{356.362122pt}}
\pgflineto{\pgfpoint{101.664001pt}{350.185242pt}}
\pgflineto{\pgfpoint{92.735992pt}{350.185242pt}}
\pgfpathclose
\pgfusepath{fill,stroke}
\pgfpathmoveto{\pgfpoint{92.735992pt}{356.362122pt}}
\pgflineto{\pgfpoint{101.664001pt}{356.362122pt}}
\pgflineto{\pgfpoint{101.664001pt}{350.185242pt}}
\pgfpathclose
\pgfusepath{fill,stroke}
\color[rgb]{0.260531,0.745802,0.444096}
\pgfpathmoveto{\pgfpoint{92.735992pt}{362.538940pt}}
\pgflineto{\pgfpoint{101.664001pt}{356.362122pt}}
\pgflineto{\pgfpoint{92.735992pt}{356.362122pt}}
\pgfpathclose
\pgfusepath{fill,stroke}
\pgfpathmoveto{\pgfpoint{92.735992pt}{362.538940pt}}
\pgflineto{\pgfpoint{101.664001pt}{362.538940pt}}
\pgflineto{\pgfpoint{101.664001pt}{356.362122pt}}
\pgfpathclose
\pgfusepath{fill,stroke}
\color[rgb]{0.290001,0.758846,0.427826}
\pgfpathmoveto{\pgfpoint{92.735992pt}{368.715820pt}}
\pgflineto{\pgfpoint{101.664001pt}{362.538940pt}}
\pgflineto{\pgfpoint{92.735992pt}{362.538940pt}}
\pgfpathclose
\pgfusepath{fill,stroke}
\pgfpathmoveto{\pgfpoint{92.735992pt}{368.715820pt}}
\pgflineto{\pgfpoint{101.664001pt}{368.715820pt}}
\pgflineto{\pgfpoint{101.664001pt}{362.538940pt}}
\pgfpathclose
\pgfusepath{fill,stroke}
\pgfpathmoveto{\pgfpoint{92.735992pt}{374.892639pt}}
\pgflineto{\pgfpoint{101.664001pt}{368.715820pt}}
\pgflineto{\pgfpoint{92.735992pt}{368.715820pt}}
\pgfpathclose
\pgfusepath{fill,stroke}
\pgfpathmoveto{\pgfpoint{92.735992pt}{374.892639pt}}
\pgflineto{\pgfpoint{101.664001pt}{374.892639pt}}
\pgflineto{\pgfpoint{101.664001pt}{368.715820pt}}
\pgfpathclose
\pgfusepath{fill,stroke}
\color[rgb]{0.321330,0.771498,0.410293}
\pgfpathmoveto{\pgfpoint{92.735992pt}{381.069458pt}}
\pgflineto{\pgfpoint{101.664001pt}{374.892639pt}}
\pgflineto{\pgfpoint{92.735992pt}{374.892639pt}}
\pgfpathclose
\pgfusepath{fill,stroke}
\pgfpathmoveto{\pgfpoint{92.735992pt}{381.069458pt}}
\pgflineto{\pgfpoint{101.664001pt}{381.069458pt}}
\pgflineto{\pgfpoint{101.664001pt}{374.892639pt}}
\pgfpathclose
\pgfusepath{fill,stroke}
\color[rgb]{0.185538,0.704725,0.485412}
\pgfpathmoveto{\pgfpoint{101.664001pt}{337.831604pt}}
\pgflineto{\pgfpoint{110.591980pt}{331.654724pt}}
\pgflineto{\pgfpoint{101.664001pt}{331.654724pt}}
\pgfpathclose
\pgfusepath{fill,stroke}
\pgfpathmoveto{\pgfpoint{101.664001pt}{337.831604pt}}
\pgflineto{\pgfpoint{110.591980pt}{337.831604pt}}
\pgflineto{\pgfpoint{110.591980pt}{331.654724pt}}
\pgfpathclose
\pgfusepath{fill,stroke}
\color[rgb]{0.208030,0.718701,0.472873}
\pgfpathmoveto{\pgfpoint{101.664001pt}{344.008423pt}}
\pgflineto{\pgfpoint{110.591980pt}{337.831604pt}}
\pgflineto{\pgfpoint{101.664001pt}{337.831604pt}}
\pgfpathclose
\pgfusepath{fill,stroke}
\pgfpathmoveto{\pgfpoint{101.664001pt}{344.008423pt}}
\pgflineto{\pgfpoint{110.591980pt}{344.008423pt}}
\pgflineto{\pgfpoint{110.591980pt}{337.831604pt}}
\pgfpathclose
\pgfusepath{fill,stroke}
\color[rgb]{0.233127,0.732406,0.459106}
\pgfpathmoveto{\pgfpoint{101.664001pt}{350.185242pt}}
\pgflineto{\pgfpoint{110.591980pt}{344.008423pt}}
\pgflineto{\pgfpoint{101.664001pt}{344.008423pt}}
\pgfpathclose
\pgfusepath{fill,stroke}
\pgfpathmoveto{\pgfpoint{101.664001pt}{350.185242pt}}
\pgflineto{\pgfpoint{110.591980pt}{350.185242pt}}
\pgflineto{\pgfpoint{110.591980pt}{344.008423pt}}
\pgfpathclose
\pgfusepath{fill,stroke}
\color[rgb]{0.260531,0.745802,0.444096}
\pgfpathmoveto{\pgfpoint{101.664001pt}{356.362122pt}}
\pgflineto{\pgfpoint{110.591980pt}{350.185242pt}}
\pgflineto{\pgfpoint{101.664001pt}{350.185242pt}}
\pgfpathclose
\pgfusepath{fill,stroke}
\pgfpathmoveto{\pgfpoint{101.664001pt}{356.362122pt}}
\pgflineto{\pgfpoint{110.591980pt}{356.362122pt}}
\pgflineto{\pgfpoint{110.591980pt}{350.185242pt}}
\pgfpathclose
\pgfusepath{fill,stroke}
\color[rgb]{0.290001,0.758846,0.427826}
\pgfpathmoveto{\pgfpoint{101.664001pt}{362.538940pt}}
\pgflineto{\pgfpoint{110.591980pt}{356.362122pt}}
\pgflineto{\pgfpoint{101.664001pt}{356.362122pt}}
\pgfpathclose
\pgfusepath{fill,stroke}
\pgfpathmoveto{\pgfpoint{101.664001pt}{362.538940pt}}
\pgflineto{\pgfpoint{110.591980pt}{362.538940pt}}
\pgflineto{\pgfpoint{110.591980pt}{356.362122pt}}
\pgfpathclose
\pgfusepath{fill,stroke}
\color[rgb]{0.321330,0.771498,0.410293}
\pgfpathmoveto{\pgfpoint{101.664001pt}{368.715820pt}}
\pgflineto{\pgfpoint{110.591980pt}{362.538940pt}}
\pgflineto{\pgfpoint{101.664001pt}{362.538940pt}}
\pgfpathclose
\pgfusepath{fill,stroke}
\pgfpathmoveto{\pgfpoint{101.664001pt}{368.715820pt}}
\pgflineto{\pgfpoint{110.591980pt}{368.715820pt}}
\pgflineto{\pgfpoint{110.591980pt}{362.538940pt}}
\pgfpathclose
\pgfusepath{fill,stroke}
\pgfpathmoveto{\pgfpoint{101.664001pt}{374.892639pt}}
\pgflineto{\pgfpoint{110.591980pt}{368.715820pt}}
\pgflineto{\pgfpoint{101.664001pt}{368.715820pt}}
\pgfpathclose
\pgfusepath{fill,stroke}
\pgfpathmoveto{\pgfpoint{101.664001pt}{374.892639pt}}
\pgflineto{\pgfpoint{110.591980pt}{374.892639pt}}
\pgflineto{\pgfpoint{110.591980pt}{368.715820pt}}
\pgfpathclose
\pgfusepath{fill,stroke}
\color[rgb]{0.354355,0.783714,0.391488}
\pgfpathmoveto{\pgfpoint{101.664001pt}{381.069458pt}}
\pgflineto{\pgfpoint{110.591980pt}{374.892639pt}}
\pgflineto{\pgfpoint{101.664001pt}{374.892639pt}}
\pgfpathclose
\pgfusepath{fill,stroke}
\pgfpathmoveto{\pgfpoint{101.664001pt}{381.069458pt}}
\pgflineto{\pgfpoint{110.591980pt}{381.069458pt}}
\pgflineto{\pgfpoint{110.591980pt}{374.892639pt}}
\pgfpathclose
\pgfusepath{fill,stroke}
\color[rgb]{0.388930,0.795453,0.371421}
\pgfpathmoveto{\pgfpoint{101.664001pt}{387.246338pt}}
\pgflineto{\pgfpoint{110.591980pt}{381.069458pt}}
\pgflineto{\pgfpoint{101.664001pt}{381.069458pt}}
\pgfpathclose
\pgfusepath{fill,stroke}
\pgfpathmoveto{\pgfpoint{101.664001pt}{387.246338pt}}
\pgflineto{\pgfpoint{110.591980pt}{387.246338pt}}
\pgflineto{\pgfpoint{110.591980pt}{381.069458pt}}
\pgfpathclose
\pgfusepath{fill,stroke}
\color[rgb]{0.424933,0.806674,0.350099}
\pgfpathmoveto{\pgfpoint{101.664001pt}{393.423157pt}}
\pgflineto{\pgfpoint{110.591980pt}{387.246338pt}}
\pgflineto{\pgfpoint{101.664001pt}{387.246338pt}}
\pgfpathclose
\pgfusepath{fill,stroke}
\pgfpathmoveto{\pgfpoint{101.664001pt}{393.423157pt}}
\pgflineto{\pgfpoint{110.591980pt}{393.423157pt}}
\pgflineto{\pgfpoint{110.591980pt}{387.246338pt}}
\pgfpathclose
\pgfusepath{fill,stroke}
\color[rgb]{0.260531,0.745802,0.444096}
\pgfpathmoveto{\pgfpoint{110.591980pt}{350.185242pt}}
\pgflineto{\pgfpoint{119.519989pt}{344.008423pt}}
\pgflineto{\pgfpoint{110.591980pt}{344.008423pt}}
\pgfpathclose
\pgfusepath{fill,stroke}
\pgfpathmoveto{\pgfpoint{110.591980pt}{350.185242pt}}
\pgflineto{\pgfpoint{119.519989pt}{350.185242pt}}
\pgflineto{\pgfpoint{119.519989pt}{344.008423pt}}
\pgfpathclose
\pgfusepath{fill,stroke}
\color[rgb]{0.290001,0.758846,0.427826}
\pgfpathmoveto{\pgfpoint{110.591980pt}{356.362122pt}}
\pgflineto{\pgfpoint{119.519989pt}{350.185242pt}}
\pgflineto{\pgfpoint{110.591980pt}{350.185242pt}}
\pgfpathclose
\pgfusepath{fill,stroke}
\pgfpathmoveto{\pgfpoint{110.591980pt}{356.362122pt}}
\pgflineto{\pgfpoint{119.519989pt}{356.362122pt}}
\pgflineto{\pgfpoint{119.519989pt}{350.185242pt}}
\pgfpathclose
\pgfusepath{fill,stroke}
\color[rgb]{0.321330,0.771498,0.410293}
\pgfpathmoveto{\pgfpoint{110.591980pt}{362.538940pt}}
\pgflineto{\pgfpoint{119.519989pt}{356.362122pt}}
\pgflineto{\pgfpoint{110.591980pt}{356.362122pt}}
\pgfpathclose
\pgfusepath{fill,stroke}
\pgfpathmoveto{\pgfpoint{110.591980pt}{362.538940pt}}
\pgflineto{\pgfpoint{119.519989pt}{362.538940pt}}
\pgflineto{\pgfpoint{119.519989pt}{356.362122pt}}
\pgfpathclose
\pgfusepath{fill,stroke}
\color[rgb]{0.354355,0.783714,0.391488}
\pgfpathmoveto{\pgfpoint{110.591980pt}{368.715820pt}}
\pgflineto{\pgfpoint{119.519989pt}{362.538940pt}}
\pgflineto{\pgfpoint{110.591980pt}{362.538940pt}}
\pgfpathclose
\pgfusepath{fill,stroke}
\pgfpathmoveto{\pgfpoint{110.591980pt}{368.715820pt}}
\pgflineto{\pgfpoint{119.519989pt}{368.715820pt}}
\pgflineto{\pgfpoint{119.519989pt}{362.538940pt}}
\pgfpathclose
\pgfusepath{fill,stroke}
\color[rgb]{0.388930,0.795453,0.371421}
\pgfpathmoveto{\pgfpoint{110.591980pt}{374.892639pt}}
\pgflineto{\pgfpoint{119.519989pt}{368.715820pt}}
\pgflineto{\pgfpoint{110.591980pt}{368.715820pt}}
\pgfpathclose
\pgfusepath{fill,stroke}
\pgfpathmoveto{\pgfpoint{110.591980pt}{374.892639pt}}
\pgflineto{\pgfpoint{119.519989pt}{374.892639pt}}
\pgflineto{\pgfpoint{119.519989pt}{368.715820pt}}
\pgfpathclose
\pgfusepath{fill,stroke}
\pgfpathmoveto{\pgfpoint{110.591980pt}{381.069458pt}}
\pgflineto{\pgfpoint{119.519989pt}{374.892639pt}}
\pgflineto{\pgfpoint{110.591980pt}{374.892639pt}}
\pgfpathclose
\pgfusepath{fill,stroke}
\pgfpathmoveto{\pgfpoint{110.591980pt}{381.069458pt}}
\pgflineto{\pgfpoint{119.519989pt}{381.069458pt}}
\pgflineto{\pgfpoint{119.519989pt}{374.892639pt}}
\pgfpathclose
\pgfusepath{fill,stroke}
\color[rgb]{0.424933,0.806674,0.350099}
\pgfpathmoveto{\pgfpoint{110.591980pt}{387.246338pt}}
\pgflineto{\pgfpoint{119.519989pt}{381.069458pt}}
\pgflineto{\pgfpoint{110.591980pt}{381.069458pt}}
\pgfpathclose
\pgfusepath{fill,stroke}
\pgfpathmoveto{\pgfpoint{110.591980pt}{387.246338pt}}
\pgflineto{\pgfpoint{119.519989pt}{387.246338pt}}
\pgflineto{\pgfpoint{119.519989pt}{381.069458pt}}
\pgfpathclose
\pgfusepath{fill,stroke}
\color[rgb]{0.462247,0.817338,0.327545}
\pgfpathmoveto{\pgfpoint{110.591980pt}{393.423157pt}}
\pgflineto{\pgfpoint{119.519989pt}{387.246338pt}}
\pgflineto{\pgfpoint{110.591980pt}{387.246338pt}}
\pgfpathclose
\pgfusepath{fill,stroke}
\pgfpathmoveto{\pgfpoint{110.591980pt}{393.423157pt}}
\pgflineto{\pgfpoint{119.519989pt}{393.423157pt}}
\pgflineto{\pgfpoint{119.519989pt}{387.246338pt}}
\pgfpathclose
\pgfusepath{fill,stroke}
\color[rgb]{0.500754,0.827409,0.303799}
\pgfpathmoveto{\pgfpoint{110.591980pt}{399.600037pt}}
\pgflineto{\pgfpoint{119.519989pt}{393.423157pt}}
\pgflineto{\pgfpoint{110.591980pt}{393.423157pt}}
\pgfpathclose
\pgfusepath{fill,stroke}
\pgfpathmoveto{\pgfpoint{110.591980pt}{399.600037pt}}
\pgflineto{\pgfpoint{119.519989pt}{399.600037pt}}
\pgflineto{\pgfpoint{119.519989pt}{393.423157pt}}
\pgfpathclose
\pgfusepath{fill,stroke}
\color[rgb]{0.388930,0.795453,0.371421}
\pgfpathmoveto{\pgfpoint{119.519989pt}{368.715820pt}}
\pgflineto{\pgfpoint{128.447998pt}{362.538940pt}}
\pgflineto{\pgfpoint{119.519989pt}{362.538940pt}}
\pgfpathclose
\pgfusepath{fill,stroke}
\pgfpathmoveto{\pgfpoint{119.519989pt}{368.715820pt}}
\pgflineto{\pgfpoint{128.447998pt}{368.715820pt}}
\pgflineto{\pgfpoint{128.447998pt}{362.538940pt}}
\pgfpathclose
\pgfusepath{fill,stroke}
\color[rgb]{0.424933,0.806674,0.350099}
\pgfpathmoveto{\pgfpoint{119.519989pt}{374.892639pt}}
\pgflineto{\pgfpoint{128.447998pt}{368.715820pt}}
\pgflineto{\pgfpoint{119.519989pt}{368.715820pt}}
\pgfpathclose
\pgfusepath{fill,stroke}
\pgfpathmoveto{\pgfpoint{119.519989pt}{374.892639pt}}
\pgflineto{\pgfpoint{128.447998pt}{374.892639pt}}
\pgflineto{\pgfpoint{128.447998pt}{368.715820pt}}
\pgfpathclose
\pgfusepath{fill,stroke}
\pgfpathmoveto{\pgfpoint{119.519989pt}{381.069458pt}}
\pgflineto{\pgfpoint{128.447998pt}{374.892639pt}}
\pgflineto{\pgfpoint{119.519989pt}{374.892639pt}}
\pgfpathclose
\pgfusepath{fill,stroke}
\pgfpathmoveto{\pgfpoint{119.519989pt}{381.069458pt}}
\pgflineto{\pgfpoint{128.447998pt}{381.069458pt}}
\pgflineto{\pgfpoint{128.447998pt}{374.892639pt}}
\pgfpathclose
\pgfusepath{fill,stroke}
\color[rgb]{0.462247,0.817338,0.327545}
\pgfpathmoveto{\pgfpoint{119.519989pt}{387.246338pt}}
\pgflineto{\pgfpoint{128.447998pt}{381.069458pt}}
\pgflineto{\pgfpoint{119.519989pt}{381.069458pt}}
\pgfpathclose
\pgfusepath{fill,stroke}
\pgfpathmoveto{\pgfpoint{119.519989pt}{387.246338pt}}
\pgflineto{\pgfpoint{128.447998pt}{387.246338pt}}
\pgflineto{\pgfpoint{128.447998pt}{381.069458pt}}
\pgfpathclose
\pgfusepath{fill,stroke}
\color[rgb]{0.500754,0.827409,0.303799}
\pgfpathmoveto{\pgfpoint{119.519989pt}{393.423157pt}}
\pgflineto{\pgfpoint{128.447998pt}{387.246338pt}}
\pgflineto{\pgfpoint{119.519989pt}{387.246338pt}}
\pgfpathclose
\pgfusepath{fill,stroke}
\pgfpathmoveto{\pgfpoint{119.519989pt}{393.423157pt}}
\pgflineto{\pgfpoint{128.447998pt}{393.423157pt}}
\pgflineto{\pgfpoint{128.447998pt}{387.246338pt}}
\pgfpathclose
\pgfusepath{fill,stroke}
\color[rgb]{0.540337,0.836858,0.278917}
\pgfpathmoveto{\pgfpoint{119.519989pt}{399.600037pt}}
\pgflineto{\pgfpoint{128.447998pt}{393.423157pt}}
\pgflineto{\pgfpoint{119.519989pt}{393.423157pt}}
\pgfpathclose
\pgfusepath{fill,stroke}
\pgfpathmoveto{\pgfpoint{119.519989pt}{399.600037pt}}
\pgflineto{\pgfpoint{128.447998pt}{399.600037pt}}
\pgflineto{\pgfpoint{128.447998pt}{393.423157pt}}
\pgfpathclose
\pgfusepath{fill,stroke}
\color[rgb]{0.500754,0.827409,0.303799}
\pgfpathmoveto{\pgfpoint{128.447998pt}{381.069458pt}}
\pgflineto{\pgfpoint{137.376007pt}{374.892639pt}}
\pgflineto{\pgfpoint{128.447998pt}{374.892639pt}}
\pgfpathclose
\pgfusepath{fill,stroke}
\pgfpathmoveto{\pgfpoint{128.447998pt}{381.069458pt}}
\pgflineto{\pgfpoint{137.376007pt}{381.069458pt}}
\pgflineto{\pgfpoint{137.376007pt}{374.892639pt}}
\pgfpathclose
\pgfusepath{fill,stroke}
\pgfpathmoveto{\pgfpoint{128.447998pt}{387.246338pt}}
\pgflineto{\pgfpoint{137.376007pt}{381.069458pt}}
\pgflineto{\pgfpoint{128.447998pt}{381.069458pt}}
\pgfpathclose
\pgfusepath{fill,stroke}
\pgfpathmoveto{\pgfpoint{128.447998pt}{387.246338pt}}
\pgflineto{\pgfpoint{137.376007pt}{387.246338pt}}
\pgflineto{\pgfpoint{137.376007pt}{381.069458pt}}
\pgfpathclose
\pgfusepath{fill,stroke}
\color[rgb]{0.540337,0.836858,0.278917}
\pgfpathmoveto{\pgfpoint{128.447998pt}{393.423157pt}}
\pgflineto{\pgfpoint{137.376007pt}{387.246338pt}}
\pgflineto{\pgfpoint{128.447998pt}{387.246338pt}}
\pgfpathclose
\pgfusepath{fill,stroke}
\pgfpathmoveto{\pgfpoint{128.447998pt}{393.423157pt}}
\pgflineto{\pgfpoint{137.376007pt}{393.423157pt}}
\pgflineto{\pgfpoint{137.376007pt}{387.246338pt}}
\pgfpathclose
\pgfusepath{fill,stroke}
\color[rgb]{0.580861,0.845663,0.253001}
\pgfpathmoveto{\pgfpoint{128.447998pt}{399.600037pt}}
\pgflineto{\pgfpoint{137.376007pt}{393.423157pt}}
\pgflineto{\pgfpoint{128.447998pt}{393.423157pt}}
\pgfpathclose
\pgfusepath{fill,stroke}
\pgfpathmoveto{\pgfpoint{128.447998pt}{399.600037pt}}
\pgflineto{\pgfpoint{137.376007pt}{399.600037pt}}
\pgflineto{\pgfpoint{137.376007pt}{393.423157pt}}
\pgfpathclose
\pgfusepath{fill,stroke}
\color[rgb]{0.622171,0.853816,0.226224}
\pgfpathmoveto{\pgfpoint{137.376007pt}{399.600037pt}}
\pgflineto{\pgfpoint{146.303986pt}{393.423157pt}}
\pgflineto{\pgfpoint{137.376007pt}{393.423157pt}}
\pgfpathclose
\pgfusepath{fill,stroke}
\pgfpathmoveto{\pgfpoint{137.376007pt}{399.600037pt}}
\pgflineto{\pgfpoint{146.303986pt}{399.600037pt}}
\pgflineto{\pgfpoint{146.303986pt}{393.423157pt}}
\pgfpathclose
\pgfusepath{fill,stroke}
\color[rgb]{0.208030,0.718701,0.472873}
\pgfpathmoveto{\pgfpoint{137.376007pt}{313.124207pt}}
\pgflineto{\pgfpoint{146.303986pt}{313.124207pt}}
\pgflineto{\pgfpoint{146.303986pt}{306.947388pt}}
\pgfpathclose
\pgfusepath{fill,stroke}
\color[rgb]{0.233127,0.732406,0.459106}
\pgfpathmoveto{\pgfpoint{137.376007pt}{319.301056pt}}
\pgflineto{\pgfpoint{146.303986pt}{313.124207pt}}
\pgflineto{\pgfpoint{137.376007pt}{313.124207pt}}
\pgfpathclose
\pgfusepath{fill,stroke}
\color[rgb]{0.208030,0.718701,0.472873}
\pgfpathmoveto{\pgfpoint{119.519989pt}{325.477905pt}}
\pgflineto{\pgfpoint{128.447998pt}{325.477905pt}}
\pgflineto{\pgfpoint{128.447998pt}{319.301056pt}}
\pgfpathclose
\pgfusepath{fill,stroke}
\color[rgb]{0.233127,0.732406,0.459106}
\pgfpathmoveto{\pgfpoint{119.519989pt}{331.654724pt}}
\pgflineto{\pgfpoint{128.447998pt}{325.477905pt}}
\pgflineto{\pgfpoint{119.519989pt}{325.477905pt}}
\pgfpathclose
\pgfusepath{fill,stroke}
\pgfpathmoveto{\pgfpoint{119.519989pt}{331.654724pt}}
\pgflineto{\pgfpoint{128.447998pt}{331.654724pt}}
\pgflineto{\pgfpoint{128.447998pt}{325.477905pt}}
\pgfpathclose
\pgfusepath{fill,stroke}
\color[rgb]{0.185538,0.704725,0.485412}
\pgfpathmoveto{\pgfpoint{128.447998pt}{313.124207pt}}
\pgflineto{\pgfpoint{137.376007pt}{313.124207pt}}
\pgflineto{\pgfpoint{137.376007pt}{306.947388pt}}
\pgfpathclose
\pgfusepath{fill,stroke}
\color[rgb]{0.208030,0.718701,0.472873}
\pgfpathmoveto{\pgfpoint{128.447998pt}{319.301056pt}}
\pgflineto{\pgfpoint{137.376007pt}{313.124207pt}}
\pgflineto{\pgfpoint{128.447998pt}{313.124207pt}}
\pgfpathclose
\pgfusepath{fill,stroke}
\pgfpathmoveto{\pgfpoint{128.447998pt}{319.301056pt}}
\pgflineto{\pgfpoint{137.376007pt}{319.301056pt}}
\pgflineto{\pgfpoint{137.376007pt}{313.124207pt}}
\pgfpathclose
\pgfusepath{fill,stroke}
\color[rgb]{0.233127,0.732406,0.459106}
\pgfpathmoveto{\pgfpoint{128.447998pt}{325.477905pt}}
\pgflineto{\pgfpoint{137.376007pt}{319.301056pt}}
\pgflineto{\pgfpoint{128.447998pt}{319.301056pt}}
\pgfpathclose
\pgfusepath{fill,stroke}
\pgfpathmoveto{\pgfpoint{128.447998pt}{325.477905pt}}
\pgflineto{\pgfpoint{137.376007pt}{325.477905pt}}
\pgflineto{\pgfpoint{137.376007pt}{319.301056pt}}
\pgfpathclose
\pgfusepath{fill,stroke}
\color[rgb]{0.260531,0.745802,0.444096}
\pgfpathmoveto{\pgfpoint{128.447998pt}{331.654724pt}}
\pgflineto{\pgfpoint{137.376007pt}{325.477905pt}}
\pgflineto{\pgfpoint{128.447998pt}{325.477905pt}}
\pgfpathclose
\pgfusepath{fill,stroke}
\color[rgb]{0.208030,0.718701,0.472873}
\pgfpathmoveto{\pgfpoint{137.376007pt}{313.124207pt}}
\pgflineto{\pgfpoint{146.303986pt}{306.947388pt}}
\pgflineto{\pgfpoint{137.376007pt}{306.947388pt}}
\pgfpathclose
\pgfusepath{fill,stroke}
\color[rgb]{0.233127,0.732406,0.459106}
\pgfpathmoveto{\pgfpoint{137.376007pt}{319.301056pt}}
\pgflineto{\pgfpoint{146.303986pt}{319.301056pt}}
\pgflineto{\pgfpoint{146.303986pt}{313.124207pt}}
\pgfpathclose
\pgfusepath{fill,stroke}
\color[rgb]{0.260531,0.745802,0.444096}
\pgfpathmoveto{\pgfpoint{137.376007pt}{325.477905pt}}
\pgflineto{\pgfpoint{146.303986pt}{319.301056pt}}
\pgflineto{\pgfpoint{137.376007pt}{319.301056pt}}
\pgfpathclose
\pgfusepath{fill,stroke}
\pgfpathmoveto{\pgfpoint{146.303986pt}{319.301056pt}}
\pgflineto{\pgfpoint{155.231979pt}{313.124207pt}}
\pgflineto{\pgfpoint{146.303986pt}{313.124207pt}}
\pgfpathclose
\pgfusepath{fill,stroke}
\pgfpathmoveto{\pgfpoint{146.303986pt}{319.301056pt}}
\pgflineto{\pgfpoint{155.231979pt}{319.301056pt}}
\pgflineto{\pgfpoint{155.231979pt}{313.124207pt}}
\pgfpathclose
\pgfusepath{fill,stroke}
\pgfpathmoveto{\pgfpoint{137.376007pt}{325.477905pt}}
\pgflineto{\pgfpoint{146.303986pt}{325.477905pt}}
\pgflineto{\pgfpoint{146.303986pt}{319.301056pt}}
\pgfpathclose
\pgfusepath{fill,stroke}
\color[rgb]{0.290001,0.758846,0.427826}
\pgfpathmoveto{\pgfpoint{146.303986pt}{325.477905pt}}
\pgflineto{\pgfpoint{155.231979pt}{319.301056pt}}
\pgflineto{\pgfpoint{146.303986pt}{319.301056pt}}
\pgfpathclose
\pgfusepath{fill,stroke}
\color[rgb]{0.233127,0.732406,0.459106}
\pgfpathmoveto{\pgfpoint{110.591980pt}{344.008423pt}}
\pgflineto{\pgfpoint{119.519989pt}{344.008423pt}}
\pgflineto{\pgfpoint{119.519989pt}{337.831604pt}}
\pgfpathclose
\pgfusepath{fill,stroke}
\color[rgb]{0.260531,0.745802,0.444096}
\pgfpathmoveto{\pgfpoint{119.519989pt}{337.831604pt}}
\pgflineto{\pgfpoint{128.447998pt}{331.654724pt}}
\pgflineto{\pgfpoint{119.519989pt}{331.654724pt}}
\pgfpathclose
\pgfusepath{fill,stroke}
\pgfpathmoveto{\pgfpoint{119.519989pt}{337.831604pt}}
\pgflineto{\pgfpoint{128.447998pt}{337.831604pt}}
\pgflineto{\pgfpoint{128.447998pt}{331.654724pt}}
\pgfpathclose
\pgfusepath{fill,stroke}
\pgfpathmoveto{\pgfpoint{119.519989pt}{344.008423pt}}
\pgflineto{\pgfpoint{128.447998pt}{337.831604pt}}
\pgflineto{\pgfpoint{119.519989pt}{337.831604pt}}
\pgfpathclose
\pgfusepath{fill,stroke}
\pgfpathmoveto{\pgfpoint{119.519989pt}{344.008423pt}}
\pgflineto{\pgfpoint{128.447998pt}{344.008423pt}}
\pgflineto{\pgfpoint{128.447998pt}{337.831604pt}}
\pgfpathclose
\pgfusepath{fill,stroke}
\color[rgb]{0.290001,0.758846,0.427826}
\pgfpathmoveto{\pgfpoint{119.519989pt}{350.185242pt}}
\pgflineto{\pgfpoint{128.447998pt}{344.008423pt}}
\pgflineto{\pgfpoint{119.519989pt}{344.008423pt}}
\pgfpathclose
\pgfusepath{fill,stroke}
\color[rgb]{0.260531,0.745802,0.444096}
\pgfpathmoveto{\pgfpoint{128.447998pt}{331.654724pt}}
\pgflineto{\pgfpoint{137.376007pt}{331.654724pt}}
\pgflineto{\pgfpoint{137.376007pt}{325.477905pt}}
\pgfpathclose
\pgfusepath{fill,stroke}
\color[rgb]{0.290001,0.758846,0.427826}
\pgfpathmoveto{\pgfpoint{128.447998pt}{337.831604pt}}
\pgflineto{\pgfpoint{137.376007pt}{331.654724pt}}
\pgflineto{\pgfpoint{128.447998pt}{331.654724pt}}
\pgfpathclose
\pgfusepath{fill,stroke}
\pgfpathmoveto{\pgfpoint{128.447998pt}{337.831604pt}}
\pgflineto{\pgfpoint{137.376007pt}{337.831604pt}}
\pgflineto{\pgfpoint{137.376007pt}{331.654724pt}}
\pgfpathclose
\pgfusepath{fill,stroke}
\pgfpathmoveto{\pgfpoint{128.447998pt}{344.008423pt}}
\pgflineto{\pgfpoint{137.376007pt}{337.831604pt}}
\pgflineto{\pgfpoint{128.447998pt}{337.831604pt}}
\pgfpathclose
\pgfusepath{fill,stroke}
\pgfpathmoveto{\pgfpoint{137.376007pt}{331.654724pt}}
\pgflineto{\pgfpoint{146.303986pt}{325.477905pt}}
\pgflineto{\pgfpoint{137.376007pt}{325.477905pt}}
\pgfpathclose
\pgfusepath{fill,stroke}
\pgfpathmoveto{\pgfpoint{137.376007pt}{331.654724pt}}
\pgflineto{\pgfpoint{146.303986pt}{331.654724pt}}
\pgflineto{\pgfpoint{146.303986pt}{325.477905pt}}
\pgfpathclose
\pgfusepath{fill,stroke}
\color[rgb]{0.321330,0.771498,0.410293}
\pgfpathmoveto{\pgfpoint{137.376007pt}{337.831604pt}}
\pgflineto{\pgfpoint{146.303986pt}{331.654724pt}}
\pgflineto{\pgfpoint{137.376007pt}{331.654724pt}}
\pgfpathclose
\pgfusepath{fill,stroke}
\pgfpathmoveto{\pgfpoint{137.376007pt}{337.831604pt}}
\pgflineto{\pgfpoint{146.303986pt}{337.831604pt}}
\pgflineto{\pgfpoint{146.303986pt}{331.654724pt}}
\pgfpathclose
\pgfusepath{fill,stroke}
\color[rgb]{0.290001,0.758846,0.427826}
\pgfpathmoveto{\pgfpoint{146.303986pt}{325.477905pt}}
\pgflineto{\pgfpoint{155.231979pt}{325.477905pt}}
\pgflineto{\pgfpoint{155.231979pt}{319.301056pt}}
\pgfpathclose
\pgfusepath{fill,stroke}
\color[rgb]{0.321330,0.771498,0.410293}
\pgfpathmoveto{\pgfpoint{146.303986pt}{331.654724pt}}
\pgflineto{\pgfpoint{155.231979pt}{325.477905pt}}
\pgflineto{\pgfpoint{146.303986pt}{325.477905pt}}
\pgfpathclose
\pgfusepath{fill,stroke}
\pgfpathmoveto{\pgfpoint{146.303986pt}{331.654724pt}}
\pgflineto{\pgfpoint{155.231979pt}{331.654724pt}}
\pgflineto{\pgfpoint{155.231979pt}{325.477905pt}}
\pgfpathclose
\pgfusepath{fill,stroke}
\color[rgb]{0.354355,0.783714,0.391488}
\pgfpathmoveto{\pgfpoint{146.303986pt}{337.831604pt}}
\pgflineto{\pgfpoint{155.231979pt}{331.654724pt}}
\pgflineto{\pgfpoint{146.303986pt}{331.654724pt}}
\pgfpathclose
\pgfusepath{fill,stroke}
\pgfpathmoveto{\pgfpoint{155.231979pt}{331.654724pt}}
\pgflineto{\pgfpoint{164.160004pt}{325.477905pt}}
\pgflineto{\pgfpoint{155.231979pt}{325.477905pt}}
\pgfpathclose
\pgfusepath{fill,stroke}
\pgfpathmoveto{\pgfpoint{155.231979pt}{331.654724pt}}
\pgflineto{\pgfpoint{164.160004pt}{331.654724pt}}
\pgflineto{\pgfpoint{164.160004pt}{325.477905pt}}
\pgfpathclose
\pgfusepath{fill,stroke}
\color[rgb]{0.290001,0.758846,0.427826}
\pgfpathmoveto{\pgfpoint{119.519989pt}{350.185242pt}}
\pgflineto{\pgfpoint{128.447998pt}{350.185242pt}}
\pgflineto{\pgfpoint{128.447998pt}{344.008423pt}}
\pgfpathclose
\pgfusepath{fill,stroke}
\color[rgb]{0.321330,0.771498,0.410293}
\pgfpathmoveto{\pgfpoint{119.519989pt}{356.362122pt}}
\pgflineto{\pgfpoint{128.447998pt}{350.185242pt}}
\pgflineto{\pgfpoint{119.519989pt}{350.185242pt}}
\pgfpathclose
\pgfusepath{fill,stroke}
\pgfpathmoveto{\pgfpoint{119.519989pt}{356.362122pt}}
\pgflineto{\pgfpoint{128.447998pt}{356.362122pt}}
\pgflineto{\pgfpoint{128.447998pt}{350.185242pt}}
\pgfpathclose
\pgfusepath{fill,stroke}
\color[rgb]{0.354355,0.783714,0.391488}
\pgfpathmoveto{\pgfpoint{119.519989pt}{362.538940pt}}
\pgflineto{\pgfpoint{128.447998pt}{356.362122pt}}
\pgflineto{\pgfpoint{119.519989pt}{356.362122pt}}
\pgfpathclose
\pgfusepath{fill,stroke}
\pgfpathmoveto{\pgfpoint{119.519989pt}{362.538940pt}}
\pgflineto{\pgfpoint{128.447998pt}{362.538940pt}}
\pgflineto{\pgfpoint{128.447998pt}{356.362122pt}}
\pgfpathclose
\pgfusepath{fill,stroke}
\color[rgb]{0.290001,0.758846,0.427826}
\pgfpathmoveto{\pgfpoint{128.447998pt}{344.008423pt}}
\pgflineto{\pgfpoint{137.376007pt}{344.008423pt}}
\pgflineto{\pgfpoint{137.376007pt}{337.831604pt}}
\pgfpathclose
\pgfusepath{fill,stroke}
\color[rgb]{0.321330,0.771498,0.410293}
\pgfpathmoveto{\pgfpoint{128.447998pt}{350.185242pt}}
\pgflineto{\pgfpoint{137.376007pt}{344.008423pt}}
\pgflineto{\pgfpoint{128.447998pt}{344.008423pt}}
\pgfpathclose
\pgfusepath{fill,stroke}
\pgfpathmoveto{\pgfpoint{128.447998pt}{350.185242pt}}
\pgflineto{\pgfpoint{137.376007pt}{350.185242pt}}
\pgflineto{\pgfpoint{137.376007pt}{344.008423pt}}
\pgfpathclose
\pgfusepath{fill,stroke}
\color[rgb]{0.354355,0.783714,0.391488}
\pgfpathmoveto{\pgfpoint{128.447998pt}{356.362122pt}}
\pgflineto{\pgfpoint{137.376007pt}{350.185242pt}}
\pgflineto{\pgfpoint{128.447998pt}{350.185242pt}}
\pgfpathclose
\pgfusepath{fill,stroke}
\pgfpathmoveto{\pgfpoint{128.447998pt}{356.362122pt}}
\pgflineto{\pgfpoint{137.376007pt}{356.362122pt}}
\pgflineto{\pgfpoint{137.376007pt}{350.185242pt}}
\pgfpathclose
\pgfusepath{fill,stroke}
\color[rgb]{0.388930,0.795453,0.371421}
\pgfpathmoveto{\pgfpoint{128.447998pt}{362.538940pt}}
\pgflineto{\pgfpoint{137.376007pt}{356.362122pt}}
\pgflineto{\pgfpoint{128.447998pt}{356.362122pt}}
\pgfpathclose
\pgfusepath{fill,stroke}
\pgfpathmoveto{\pgfpoint{128.447998pt}{362.538940pt}}
\pgflineto{\pgfpoint{137.376007pt}{362.538940pt}}
\pgflineto{\pgfpoint{137.376007pt}{356.362122pt}}
\pgfpathclose
\pgfusepath{fill,stroke}
\color[rgb]{0.424933,0.806674,0.350099}
\pgfpathmoveto{\pgfpoint{128.447998pt}{368.715820pt}}
\pgflineto{\pgfpoint{137.376007pt}{362.538940pt}}
\pgflineto{\pgfpoint{128.447998pt}{362.538940pt}}
\pgfpathclose
\pgfusepath{fill,stroke}
\pgfpathmoveto{\pgfpoint{128.447998pt}{368.715820pt}}
\pgflineto{\pgfpoint{137.376007pt}{368.715820pt}}
\pgflineto{\pgfpoint{137.376007pt}{362.538940pt}}
\pgfpathclose
\pgfusepath{fill,stroke}
\color[rgb]{0.354355,0.783714,0.391488}
\pgfpathmoveto{\pgfpoint{137.376007pt}{344.008423pt}}
\pgflineto{\pgfpoint{146.303986pt}{337.831604pt}}
\pgflineto{\pgfpoint{137.376007pt}{337.831604pt}}
\pgfpathclose
\pgfusepath{fill,stroke}
\pgfpathmoveto{\pgfpoint{137.376007pt}{344.008423pt}}
\pgflineto{\pgfpoint{146.303986pt}{344.008423pt}}
\pgflineto{\pgfpoint{146.303986pt}{337.831604pt}}
\pgfpathclose
\pgfusepath{fill,stroke}
\pgfpathmoveto{\pgfpoint{137.376007pt}{350.185242pt}}
\pgflineto{\pgfpoint{146.303986pt}{344.008423pt}}
\pgflineto{\pgfpoint{137.376007pt}{344.008423pt}}
\pgfpathclose
\pgfusepath{fill,stroke}
\pgfpathmoveto{\pgfpoint{137.376007pt}{350.185242pt}}
\pgflineto{\pgfpoint{146.303986pt}{350.185242pt}}
\pgflineto{\pgfpoint{146.303986pt}{344.008423pt}}
\pgfpathclose
\pgfusepath{fill,stroke}
\color[rgb]{0.388930,0.795453,0.371421}
\pgfpathmoveto{\pgfpoint{137.376007pt}{356.362122pt}}
\pgflineto{\pgfpoint{146.303986pt}{350.185242pt}}
\pgflineto{\pgfpoint{137.376007pt}{350.185242pt}}
\pgfpathclose
\pgfusepath{fill,stroke}
\pgfpathmoveto{\pgfpoint{137.376007pt}{356.362122pt}}
\pgflineto{\pgfpoint{146.303986pt}{356.362122pt}}
\pgflineto{\pgfpoint{146.303986pt}{350.185242pt}}
\pgfpathclose
\pgfusepath{fill,stroke}
\color[rgb]{0.424933,0.806674,0.350099}
\pgfpathmoveto{\pgfpoint{137.376007pt}{362.538940pt}}
\pgflineto{\pgfpoint{146.303986pt}{356.362122pt}}
\pgflineto{\pgfpoint{137.376007pt}{356.362122pt}}
\pgfpathclose
\pgfusepath{fill,stroke}
\pgfpathmoveto{\pgfpoint{137.376007pt}{362.538940pt}}
\pgflineto{\pgfpoint{146.303986pt}{362.538940pt}}
\pgflineto{\pgfpoint{146.303986pt}{356.362122pt}}
\pgfpathclose
\pgfusepath{fill,stroke}
\color[rgb]{0.462247,0.817338,0.327545}
\pgfpathmoveto{\pgfpoint{137.376007pt}{368.715820pt}}
\pgflineto{\pgfpoint{146.303986pt}{362.538940pt}}
\pgflineto{\pgfpoint{137.376007pt}{362.538940pt}}
\pgfpathclose
\pgfusepath{fill,stroke}
\pgfpathmoveto{\pgfpoint{137.376007pt}{368.715820pt}}
\pgflineto{\pgfpoint{146.303986pt}{368.715820pt}}
\pgflineto{\pgfpoint{146.303986pt}{362.538940pt}}
\pgfpathclose
\pgfusepath{fill,stroke}
\color[rgb]{0.500754,0.827409,0.303799}
\pgfpathmoveto{\pgfpoint{137.376007pt}{374.892639pt}}
\pgflineto{\pgfpoint{146.303986pt}{368.715820pt}}
\pgflineto{\pgfpoint{137.376007pt}{368.715820pt}}
\pgfpathclose
\pgfusepath{fill,stroke}
\pgfpathmoveto{\pgfpoint{137.376007pt}{374.892639pt}}
\pgflineto{\pgfpoint{146.303986pt}{374.892639pt}}
\pgflineto{\pgfpoint{146.303986pt}{368.715820pt}}
\pgfpathclose
\pgfusepath{fill,stroke}
\color[rgb]{0.354355,0.783714,0.391488}
\pgfpathmoveto{\pgfpoint{146.303986pt}{337.831604pt}}
\pgflineto{\pgfpoint{155.231979pt}{337.831604pt}}
\pgflineto{\pgfpoint{155.231979pt}{331.654724pt}}
\pgfpathclose
\pgfusepath{fill,stroke}
\color[rgb]{0.388930,0.795453,0.371421}
\pgfpathmoveto{\pgfpoint{146.303986pt}{344.008423pt}}
\pgflineto{\pgfpoint{155.231979pt}{337.831604pt}}
\pgflineto{\pgfpoint{146.303986pt}{337.831604pt}}
\pgfpathclose
\pgfusepath{fill,stroke}
\pgfpathmoveto{\pgfpoint{146.303986pt}{344.008423pt}}
\pgflineto{\pgfpoint{155.231979pt}{344.008423pt}}
\pgflineto{\pgfpoint{155.231979pt}{337.831604pt}}
\pgfpathclose
\pgfusepath{fill,stroke}
\pgfpathmoveto{\pgfpoint{146.303986pt}{350.185242pt}}
\pgflineto{\pgfpoint{155.231979pt}{344.008423pt}}
\pgflineto{\pgfpoint{146.303986pt}{344.008423pt}}
\pgfpathclose
\pgfusepath{fill,stroke}
\pgfpathmoveto{\pgfpoint{146.303986pt}{350.185242pt}}
\pgflineto{\pgfpoint{155.231979pt}{350.185242pt}}
\pgflineto{\pgfpoint{155.231979pt}{344.008423pt}}
\pgfpathclose
\pgfusepath{fill,stroke}
\color[rgb]{0.424933,0.806674,0.350099}
\pgfpathmoveto{\pgfpoint{146.303986pt}{356.362122pt}}
\pgflineto{\pgfpoint{155.231979pt}{350.185242pt}}
\pgflineto{\pgfpoint{146.303986pt}{350.185242pt}}
\pgfpathclose
\pgfusepath{fill,stroke}
\pgfpathmoveto{\pgfpoint{146.303986pt}{356.362122pt}}
\pgflineto{\pgfpoint{155.231979pt}{356.362122pt}}
\pgflineto{\pgfpoint{155.231979pt}{350.185242pt}}
\pgfpathclose
\pgfusepath{fill,stroke}
\color[rgb]{0.462247,0.817338,0.327545}
\pgfpathmoveto{\pgfpoint{146.303986pt}{362.538940pt}}
\pgflineto{\pgfpoint{155.231979pt}{356.362122pt}}
\pgflineto{\pgfpoint{146.303986pt}{356.362122pt}}
\pgfpathclose
\pgfusepath{fill,stroke}
\pgfpathmoveto{\pgfpoint{146.303986pt}{362.538940pt}}
\pgflineto{\pgfpoint{155.231979pt}{362.538940pt}}
\pgflineto{\pgfpoint{155.231979pt}{356.362122pt}}
\pgfpathclose
\pgfusepath{fill,stroke}
\color[rgb]{0.500754,0.827409,0.303799}
\pgfpathmoveto{\pgfpoint{146.303986pt}{368.715820pt}}
\pgflineto{\pgfpoint{155.231979pt}{362.538940pt}}
\pgflineto{\pgfpoint{146.303986pt}{362.538940pt}}
\pgfpathclose
\pgfusepath{fill,stroke}
\pgfpathmoveto{\pgfpoint{146.303986pt}{368.715820pt}}
\pgflineto{\pgfpoint{155.231979pt}{368.715820pt}}
\pgflineto{\pgfpoint{155.231979pt}{362.538940pt}}
\pgfpathclose
\pgfusepath{fill,stroke}
\color[rgb]{0.540337,0.836858,0.278917}
\pgfpathmoveto{\pgfpoint{146.303986pt}{374.892639pt}}
\pgflineto{\pgfpoint{155.231979pt}{368.715820pt}}
\pgflineto{\pgfpoint{146.303986pt}{368.715820pt}}
\pgfpathclose
\pgfusepath{fill,stroke}
\pgfpathmoveto{\pgfpoint{146.303986pt}{374.892639pt}}
\pgflineto{\pgfpoint{155.231979pt}{374.892639pt}}
\pgflineto{\pgfpoint{155.231979pt}{368.715820pt}}
\pgfpathclose
\pgfusepath{fill,stroke}
\color[rgb]{0.388930,0.795453,0.371421}
\pgfpathmoveto{\pgfpoint{155.231979pt}{337.831604pt}}
\pgflineto{\pgfpoint{164.160004pt}{331.654724pt}}
\pgflineto{\pgfpoint{155.231979pt}{331.654724pt}}
\pgfpathclose
\pgfusepath{fill,stroke}
\pgfpathmoveto{\pgfpoint{155.231979pt}{337.831604pt}}
\pgflineto{\pgfpoint{164.160004pt}{337.831604pt}}
\pgflineto{\pgfpoint{164.160004pt}{331.654724pt}}
\pgfpathclose
\pgfusepath{fill,stroke}
\color[rgb]{0.424933,0.806674,0.350099}
\pgfpathmoveto{\pgfpoint{155.231979pt}{344.008423pt}}
\pgflineto{\pgfpoint{164.160004pt}{337.831604pt}}
\pgflineto{\pgfpoint{155.231979pt}{337.831604pt}}
\pgfpathclose
\pgfusepath{fill,stroke}
\pgfpathmoveto{\pgfpoint{155.231979pt}{344.008423pt}}
\pgflineto{\pgfpoint{164.160004pt}{344.008423pt}}
\pgflineto{\pgfpoint{164.160004pt}{337.831604pt}}
\pgfpathclose
\pgfusepath{fill,stroke}
\color[rgb]{0.462247,0.817338,0.327545}
\pgfpathmoveto{\pgfpoint{155.231979pt}{350.185242pt}}
\pgflineto{\pgfpoint{164.160004pt}{344.008423pt}}
\pgflineto{\pgfpoint{155.231979pt}{344.008423pt}}
\pgfpathclose
\pgfusepath{fill,stroke}
\pgfpathmoveto{\pgfpoint{155.231979pt}{350.185242pt}}
\pgflineto{\pgfpoint{164.160004pt}{350.185242pt}}
\pgflineto{\pgfpoint{164.160004pt}{344.008423pt}}
\pgfpathclose
\pgfusepath{fill,stroke}
\pgfpathmoveto{\pgfpoint{155.231979pt}{356.362122pt}}
\pgflineto{\pgfpoint{164.160004pt}{350.185242pt}}
\pgflineto{\pgfpoint{155.231979pt}{350.185242pt}}
\pgfpathclose
\pgfusepath{fill,stroke}
\pgfpathmoveto{\pgfpoint{155.231979pt}{356.362122pt}}
\pgflineto{\pgfpoint{164.160004pt}{356.362122pt}}
\pgflineto{\pgfpoint{164.160004pt}{350.185242pt}}
\pgfpathclose
\pgfusepath{fill,stroke}
\color[rgb]{0.500754,0.827409,0.303799}
\pgfpathmoveto{\pgfpoint{155.231979pt}{362.538940pt}}
\pgflineto{\pgfpoint{164.160004pt}{356.362122pt}}
\pgflineto{\pgfpoint{155.231979pt}{356.362122pt}}
\pgfpathclose
\pgfusepath{fill,stroke}
\pgfpathmoveto{\pgfpoint{155.231979pt}{362.538940pt}}
\pgflineto{\pgfpoint{164.160004pt}{362.538940pt}}
\pgflineto{\pgfpoint{164.160004pt}{356.362122pt}}
\pgfpathclose
\pgfusepath{fill,stroke}
\color[rgb]{0.540337,0.836858,0.278917}
\pgfpathmoveto{\pgfpoint{155.231979pt}{368.715820pt}}
\pgflineto{\pgfpoint{164.160004pt}{362.538940pt}}
\pgflineto{\pgfpoint{155.231979pt}{362.538940pt}}
\pgfpathclose
\pgfusepath{fill,stroke}
\pgfpathmoveto{\pgfpoint{155.231979pt}{368.715820pt}}
\pgflineto{\pgfpoint{164.160004pt}{368.715820pt}}
\pgflineto{\pgfpoint{164.160004pt}{362.538940pt}}
\pgfpathclose
\pgfusepath{fill,stroke}
\color[rgb]{0.580861,0.845663,0.253001}
\pgfpathmoveto{\pgfpoint{155.231979pt}{374.892639pt}}
\pgflineto{\pgfpoint{164.160004pt}{368.715820pt}}
\pgflineto{\pgfpoint{155.231979pt}{368.715820pt}}
\pgfpathclose
\pgfusepath{fill,stroke}
\pgfpathmoveto{\pgfpoint{155.231979pt}{374.892639pt}}
\pgflineto{\pgfpoint{164.160004pt}{374.892639pt}}
\pgflineto{\pgfpoint{164.160004pt}{368.715820pt}}
\pgfpathclose
\pgfusepath{fill,stroke}
\color[rgb]{0.462247,0.817338,0.327545}
\pgfpathmoveto{\pgfpoint{164.160004pt}{344.008423pt}}
\pgflineto{\pgfpoint{173.087997pt}{337.831604pt}}
\pgflineto{\pgfpoint{164.160004pt}{337.831604pt}}
\pgfpathclose
\pgfusepath{fill,stroke}
\pgfpathmoveto{\pgfpoint{164.160004pt}{344.008423pt}}
\pgflineto{\pgfpoint{173.087997pt}{344.008423pt}}
\pgflineto{\pgfpoint{173.087997pt}{337.831604pt}}
\pgfpathclose
\pgfusepath{fill,stroke}
\color[rgb]{0.500754,0.827409,0.303799}
\pgfpathmoveto{\pgfpoint{164.160004pt}{350.185242pt}}
\pgflineto{\pgfpoint{173.087997pt}{344.008423pt}}
\pgflineto{\pgfpoint{164.160004pt}{344.008423pt}}
\pgfpathclose
\pgfusepath{fill,stroke}
\pgfpathmoveto{\pgfpoint{164.160004pt}{350.185242pt}}
\pgflineto{\pgfpoint{173.087997pt}{350.185242pt}}
\pgflineto{\pgfpoint{173.087997pt}{344.008423pt}}
\pgfpathclose
\pgfusepath{fill,stroke}
\pgfpathmoveto{\pgfpoint{164.160004pt}{356.362122pt}}
\pgflineto{\pgfpoint{173.087997pt}{350.185242pt}}
\pgflineto{\pgfpoint{164.160004pt}{350.185242pt}}
\pgfpathclose
\pgfusepath{fill,stroke}
\pgfpathmoveto{\pgfpoint{164.160004pt}{356.362122pt}}
\pgflineto{\pgfpoint{173.087997pt}{356.362122pt}}
\pgflineto{\pgfpoint{173.087997pt}{350.185242pt}}
\pgfpathclose
\pgfusepath{fill,stroke}
\color[rgb]{0.540337,0.836858,0.278917}
\pgfpathmoveto{\pgfpoint{164.160004pt}{362.538940pt}}
\pgflineto{\pgfpoint{173.087997pt}{356.362122pt}}
\pgflineto{\pgfpoint{164.160004pt}{356.362122pt}}
\pgfpathclose
\pgfusepath{fill,stroke}
\pgfpathmoveto{\pgfpoint{164.160004pt}{362.538940pt}}
\pgflineto{\pgfpoint{173.087997pt}{362.538940pt}}
\pgflineto{\pgfpoint{173.087997pt}{356.362122pt}}
\pgfpathclose
\pgfusepath{fill,stroke}
\color[rgb]{0.580861,0.845663,0.253001}
\pgfpathmoveto{\pgfpoint{164.160004pt}{368.715820pt}}
\pgflineto{\pgfpoint{173.087997pt}{362.538940pt}}
\pgflineto{\pgfpoint{164.160004pt}{362.538940pt}}
\pgfpathclose
\pgfusepath{fill,stroke}
\pgfpathmoveto{\pgfpoint{164.160004pt}{368.715820pt}}
\pgflineto{\pgfpoint{173.087997pt}{368.715820pt}}
\pgflineto{\pgfpoint{173.087997pt}{362.538940pt}}
\pgfpathclose
\pgfusepath{fill,stroke}
\color[rgb]{0.622171,0.853816,0.226224}
\pgfpathmoveto{\pgfpoint{164.160004pt}{374.892639pt}}
\pgflineto{\pgfpoint{173.087997pt}{368.715820pt}}
\pgflineto{\pgfpoint{164.160004pt}{368.715820pt}}
\pgfpathclose
\pgfusepath{fill,stroke}
\pgfpathmoveto{\pgfpoint{164.160004pt}{374.892639pt}}
\pgflineto{\pgfpoint{173.087997pt}{374.892639pt}}
\pgflineto{\pgfpoint{173.087997pt}{368.715820pt}}
\pgfpathclose
\pgfusepath{fill,stroke}
\color[rgb]{0.580861,0.845663,0.253001}
\pgfpathmoveto{\pgfpoint{173.087997pt}{356.362122pt}}
\pgflineto{\pgfpoint{182.015991pt}{350.185242pt}}
\pgflineto{\pgfpoint{173.087997pt}{350.185242pt}}
\pgfpathclose
\pgfusepath{fill,stroke}
\pgfpathmoveto{\pgfpoint{173.087997pt}{356.362122pt}}
\pgflineto{\pgfpoint{182.015991pt}{356.362122pt}}
\pgflineto{\pgfpoint{182.015991pt}{350.185242pt}}
\pgfpathclose
\pgfusepath{fill,stroke}
\pgfpathmoveto{\pgfpoint{173.087997pt}{362.538940pt}}
\pgflineto{\pgfpoint{182.015991pt}{356.362122pt}}
\pgflineto{\pgfpoint{173.087997pt}{356.362122pt}}
\pgfpathclose
\pgfusepath{fill,stroke}
\pgfpathmoveto{\pgfpoint{173.087997pt}{362.538940pt}}
\pgflineto{\pgfpoint{182.015991pt}{362.538940pt}}
\pgflineto{\pgfpoint{182.015991pt}{356.362122pt}}
\pgfpathclose
\pgfusepath{fill,stroke}
\color[rgb]{0.622171,0.853816,0.226224}
\pgfpathmoveto{\pgfpoint{173.087997pt}{368.715820pt}}
\pgflineto{\pgfpoint{182.015991pt}{362.538940pt}}
\pgflineto{\pgfpoint{173.087997pt}{362.538940pt}}
\pgfpathclose
\pgfusepath{fill,stroke}
\pgfpathmoveto{\pgfpoint{173.087997pt}{368.715820pt}}
\pgflineto{\pgfpoint{182.015991pt}{368.715820pt}}
\pgflineto{\pgfpoint{182.015991pt}{362.538940pt}}
\pgfpathclose
\pgfusepath{fill,stroke}
\color[rgb]{0.664087,0.861321,0.198879}
\pgfpathmoveto{\pgfpoint{173.087997pt}{374.892639pt}}
\pgflineto{\pgfpoint{182.015991pt}{368.715820pt}}
\pgflineto{\pgfpoint{173.087997pt}{368.715820pt}}
\pgfpathclose
\pgfusepath{fill,stroke}
\pgfpathmoveto{\pgfpoint{173.087997pt}{374.892639pt}}
\pgflineto{\pgfpoint{182.015991pt}{374.892639pt}}
\pgflineto{\pgfpoint{182.015991pt}{368.715820pt}}
\pgfpathclose
\pgfusepath{fill,stroke}
\pgfpathmoveto{\pgfpoint{182.015991pt}{368.715820pt}}
\pgflineto{\pgfpoint{190.943985pt}{362.538940pt}}
\pgflineto{\pgfpoint{182.015991pt}{362.538940pt}}
\pgfpathclose
\pgfusepath{fill,stroke}
\pgfpathmoveto{\pgfpoint{182.015991pt}{368.715820pt}}
\pgflineto{\pgfpoint{190.943985pt}{368.715820pt}}
\pgflineto{\pgfpoint{190.943985pt}{362.538940pt}}
\pgfpathclose
\pgfusepath{fill,stroke}
\color[rgb]{0.706404,0.868206,0.171495}
\pgfpathmoveto{\pgfpoint{182.015991pt}{374.892639pt}}
\pgflineto{\pgfpoint{190.943985pt}{368.715820pt}}
\pgflineto{\pgfpoint{182.015991pt}{368.715820pt}}
\pgfpathclose
\pgfusepath{fill,stroke}
\pgfpathmoveto{\pgfpoint{182.015991pt}{374.892639pt}}
\pgflineto{\pgfpoint{190.943985pt}{374.892639pt}}
\pgflineto{\pgfpoint{190.943985pt}{368.715820pt}}
\pgfpathclose
\pgfusepath{fill,stroke}
\color[rgb]{0.748885,0.874522,0.145038}
\pgfpathmoveto{\pgfpoint{182.015991pt}{381.069458pt}}
\pgflineto{\pgfpoint{190.943985pt}{374.892639pt}}
\pgflineto{\pgfpoint{182.015991pt}{374.892639pt}}
\pgfpathclose
\pgfusepath{fill,stroke}
\pgfpathmoveto{\pgfpoint{182.015991pt}{381.069458pt}}
\pgflineto{\pgfpoint{190.943985pt}{381.069458pt}}
\pgflineto{\pgfpoint{190.943985pt}{374.892639pt}}
\pgfpathclose
\pgfusepath{fill,stroke}
\color[rgb]{0.462247,0.817338,0.327545}
\pgfpathmoveto{\pgfpoint{128.447998pt}{374.892639pt}}
\pgflineto{\pgfpoint{137.376007pt}{368.715820pt}}
\pgflineto{\pgfpoint{128.447998pt}{368.715820pt}}
\pgfpathclose
\pgfusepath{fill,stroke}
\pgfpathmoveto{\pgfpoint{128.447998pt}{374.892639pt}}
\pgflineto{\pgfpoint{137.376007pt}{374.892639pt}}
\pgflineto{\pgfpoint{137.376007pt}{368.715820pt}}
\pgfpathclose
\pgfusepath{fill,stroke}
\color[rgb]{0.540337,0.836858,0.278917}
\pgfpathmoveto{\pgfpoint{137.376007pt}{381.069458pt}}
\pgflineto{\pgfpoint{146.303986pt}{374.892639pt}}
\pgflineto{\pgfpoint{137.376007pt}{374.892639pt}}
\pgfpathclose
\pgfusepath{fill,stroke}
\pgfpathmoveto{\pgfpoint{137.376007pt}{381.069458pt}}
\pgflineto{\pgfpoint{146.303986pt}{381.069458pt}}
\pgflineto{\pgfpoint{146.303986pt}{374.892639pt}}
\pgfpathclose
\pgfusepath{fill,stroke}
\pgfpathmoveto{\pgfpoint{137.376007pt}{387.246338pt}}
\pgflineto{\pgfpoint{146.303986pt}{381.069458pt}}
\pgflineto{\pgfpoint{137.376007pt}{381.069458pt}}
\pgfpathclose
\pgfusepath{fill,stroke}
\pgfpathmoveto{\pgfpoint{137.376007pt}{387.246338pt}}
\pgflineto{\pgfpoint{146.303986pt}{387.246338pt}}
\pgflineto{\pgfpoint{146.303986pt}{381.069458pt}}
\pgfpathclose
\pgfusepath{fill,stroke}
\color[rgb]{0.580861,0.845663,0.253001}
\pgfpathmoveto{\pgfpoint{137.376007pt}{393.423157pt}}
\pgflineto{\pgfpoint{146.303986pt}{387.246338pt}}
\pgflineto{\pgfpoint{137.376007pt}{387.246338pt}}
\pgfpathclose
\pgfusepath{fill,stroke}
\pgfpathmoveto{\pgfpoint{137.376007pt}{393.423157pt}}
\pgflineto{\pgfpoint{146.303986pt}{393.423157pt}}
\pgflineto{\pgfpoint{146.303986pt}{387.246338pt}}
\pgfpathclose
\pgfusepath{fill,stroke}
\pgfpathmoveto{\pgfpoint{146.303986pt}{381.069458pt}}
\pgflineto{\pgfpoint{155.231979pt}{374.892639pt}}
\pgflineto{\pgfpoint{146.303986pt}{374.892639pt}}
\pgfpathclose
\pgfusepath{fill,stroke}
\pgfpathmoveto{\pgfpoint{146.303986pt}{381.069458pt}}
\pgflineto{\pgfpoint{155.231979pt}{381.069458pt}}
\pgflineto{\pgfpoint{155.231979pt}{374.892639pt}}
\pgfpathclose
\pgfusepath{fill,stroke}
\color[rgb]{0.622171,0.853816,0.226224}
\pgfpathmoveto{\pgfpoint{146.303986pt}{387.246338pt}}
\pgflineto{\pgfpoint{155.231979pt}{381.069458pt}}
\pgflineto{\pgfpoint{146.303986pt}{381.069458pt}}
\pgfpathclose
\pgfusepath{fill,stroke}
\pgfpathmoveto{\pgfpoint{146.303986pt}{387.246338pt}}
\pgflineto{\pgfpoint{155.231979pt}{387.246338pt}}
\pgflineto{\pgfpoint{155.231979pt}{381.069458pt}}
\pgfpathclose
\pgfusepath{fill,stroke}
\pgfpathmoveto{\pgfpoint{146.303986pt}{393.423157pt}}
\pgflineto{\pgfpoint{155.231979pt}{387.246338pt}}
\pgflineto{\pgfpoint{146.303986pt}{387.246338pt}}
\pgfpathclose
\pgfusepath{fill,stroke}
\pgfpathmoveto{\pgfpoint{146.303986pt}{393.423157pt}}
\pgflineto{\pgfpoint{155.231979pt}{393.423157pt}}
\pgflineto{\pgfpoint{155.231979pt}{387.246338pt}}
\pgfpathclose
\pgfusepath{fill,stroke}
\color[rgb]{0.664087,0.861321,0.198879}
\pgfpathmoveto{\pgfpoint{146.303986pt}{399.600037pt}}
\pgflineto{\pgfpoint{155.231979pt}{393.423157pt}}
\pgflineto{\pgfpoint{146.303986pt}{393.423157pt}}
\pgfpathclose
\pgfusepath{fill,stroke}
\pgfpathmoveto{\pgfpoint{146.303986pt}{399.600037pt}}
\pgflineto{\pgfpoint{155.231979pt}{399.600037pt}}
\pgflineto{\pgfpoint{155.231979pt}{393.423157pt}}
\pgfpathclose
\pgfusepath{fill,stroke}
\color[rgb]{0.622171,0.853816,0.226224}
\pgfpathmoveto{\pgfpoint{155.231979pt}{381.069458pt}}
\pgflineto{\pgfpoint{164.160004pt}{374.892639pt}}
\pgflineto{\pgfpoint{155.231979pt}{374.892639pt}}
\pgfpathclose
\pgfusepath{fill,stroke}
\pgfpathmoveto{\pgfpoint{155.231979pt}{381.069458pt}}
\pgflineto{\pgfpoint{164.160004pt}{381.069458pt}}
\pgflineto{\pgfpoint{164.160004pt}{374.892639pt}}
\pgfpathclose
\pgfusepath{fill,stroke}
\color[rgb]{0.664087,0.861321,0.198879}
\pgfpathmoveto{\pgfpoint{155.231979pt}{387.246338pt}}
\pgflineto{\pgfpoint{164.160004pt}{381.069458pt}}
\pgflineto{\pgfpoint{155.231979pt}{381.069458pt}}
\pgfpathclose
\pgfusepath{fill,stroke}
\pgfpathmoveto{\pgfpoint{155.231979pt}{387.246338pt}}
\pgflineto{\pgfpoint{164.160004pt}{387.246338pt}}
\pgflineto{\pgfpoint{164.160004pt}{381.069458pt}}
\pgfpathclose
\pgfusepath{fill,stroke}
\pgfpathmoveto{\pgfpoint{155.231979pt}{393.423157pt}}
\pgflineto{\pgfpoint{164.160004pt}{387.246338pt}}
\pgflineto{\pgfpoint{155.231979pt}{387.246338pt}}
\pgfpathclose
\pgfusepath{fill,stroke}
\pgfpathmoveto{\pgfpoint{155.231979pt}{393.423157pt}}
\pgflineto{\pgfpoint{164.160004pt}{393.423157pt}}
\pgflineto{\pgfpoint{164.160004pt}{387.246338pt}}
\pgfpathclose
\pgfusepath{fill,stroke}
\pgfpathmoveto{\pgfpoint{164.160004pt}{381.069458pt}}
\pgflineto{\pgfpoint{173.087997pt}{374.892639pt}}
\pgflineto{\pgfpoint{164.160004pt}{374.892639pt}}
\pgfpathclose
\pgfusepath{fill,stroke}
\pgfpathmoveto{\pgfpoint{164.160004pt}{381.069458pt}}
\pgflineto{\pgfpoint{173.087997pt}{381.069458pt}}
\pgflineto{\pgfpoint{173.087997pt}{374.892639pt}}
\pgfpathclose
\pgfusepath{fill,stroke}
\color[rgb]{0.706404,0.868206,0.171495}
\pgfpathmoveto{\pgfpoint{164.160004pt}{387.246338pt}}
\pgflineto{\pgfpoint{173.087997pt}{381.069458pt}}
\pgflineto{\pgfpoint{164.160004pt}{381.069458pt}}
\pgfpathclose
\pgfusepath{fill,stroke}
\pgfpathmoveto{\pgfpoint{164.160004pt}{387.246338pt}}
\pgflineto{\pgfpoint{173.087997pt}{387.246338pt}}
\pgflineto{\pgfpoint{173.087997pt}{381.069458pt}}
\pgfpathclose
\pgfusepath{fill,stroke}
\color[rgb]{0.748885,0.874522,0.145038}
\pgfpathmoveto{\pgfpoint{164.160004pt}{393.423157pt}}
\pgflineto{\pgfpoint{173.087997pt}{387.246338pt}}
\pgflineto{\pgfpoint{164.160004pt}{387.246338pt}}
\pgfpathclose
\pgfusepath{fill,stroke}
\pgfpathmoveto{\pgfpoint{164.160004pt}{393.423157pt}}
\pgflineto{\pgfpoint{173.087997pt}{393.423157pt}}
\pgflineto{\pgfpoint{173.087997pt}{387.246338pt}}
\pgfpathclose
\pgfusepath{fill,stroke}
\color[rgb]{0.706404,0.868206,0.171495}
\pgfpathmoveto{\pgfpoint{173.087997pt}{381.069458pt}}
\pgflineto{\pgfpoint{182.015991pt}{374.892639pt}}
\pgflineto{\pgfpoint{173.087997pt}{374.892639pt}}
\pgfpathclose
\pgfusepath{fill,stroke}
\pgfpathmoveto{\pgfpoint{173.087997pt}{381.069458pt}}
\pgflineto{\pgfpoint{182.015991pt}{381.069458pt}}
\pgflineto{\pgfpoint{182.015991pt}{374.892639pt}}
\pgfpathclose
\pgfusepath{fill,stroke}
\color[rgb]{0.748885,0.874522,0.145038}
\pgfpathmoveto{\pgfpoint{173.087997pt}{387.246338pt}}
\pgflineto{\pgfpoint{182.015991pt}{381.069458pt}}
\pgflineto{\pgfpoint{173.087997pt}{381.069458pt}}
\pgfpathclose
\pgfusepath{fill,stroke}
\pgfpathmoveto{\pgfpoint{173.087997pt}{387.246338pt}}
\pgflineto{\pgfpoint{182.015991pt}{387.246338pt}}
\pgflineto{\pgfpoint{182.015991pt}{381.069458pt}}
\pgfpathclose
\pgfusepath{fill,stroke}
\color[rgb]{0.149643,0.676120,0.506924}
\pgfpathmoveto{\pgfpoint{101.664001pt}{319.301056pt}}
\pgflineto{\pgfpoint{110.591980pt}{319.301056pt}}
\pgflineto{\pgfpoint{110.591980pt}{313.124207pt}}
\pgfpathclose
\pgfusepath{fill,stroke}
\color[rgb]{0.165967,0.690519,0.496752}
\pgfpathmoveto{\pgfpoint{101.664001pt}{325.477905pt}}
\pgflineto{\pgfpoint{110.591980pt}{319.301056pt}}
\pgflineto{\pgfpoint{101.664001pt}{319.301056pt}}
\pgfpathclose
\pgfusepath{fill,stroke}
\pgfpathmoveto{\pgfpoint{101.664001pt}{325.477905pt}}
\pgflineto{\pgfpoint{110.591980pt}{325.477905pt}}
\pgflineto{\pgfpoint{110.591980pt}{319.301056pt}}
\pgfpathclose
\pgfusepath{fill,stroke}
\color[rgb]{0.185538,0.704725,0.485412}
\pgfpathmoveto{\pgfpoint{101.664001pt}{331.654724pt}}
\pgflineto{\pgfpoint{110.591980pt}{325.477905pt}}
\pgflineto{\pgfpoint{101.664001pt}{325.477905pt}}
\pgfpathclose
\pgfusepath{fill,stroke}
\pgfpathmoveto{\pgfpoint{101.664001pt}{331.654724pt}}
\pgflineto{\pgfpoint{110.591980pt}{331.654724pt}}
\pgflineto{\pgfpoint{110.591980pt}{325.477905pt}}
\pgfpathclose
\pgfusepath{fill,stroke}
\color[rgb]{0.165967,0.690519,0.496752}
\pgfpathmoveto{\pgfpoint{110.591980pt}{319.301056pt}}
\pgflineto{\pgfpoint{119.519989pt}{313.124207pt}}
\pgflineto{\pgfpoint{110.591980pt}{313.124207pt}}
\pgfpathclose
\pgfusepath{fill,stroke}
\pgfpathmoveto{\pgfpoint{110.591980pt}{319.301056pt}}
\pgflineto{\pgfpoint{119.519989pt}{319.301056pt}}
\pgflineto{\pgfpoint{119.519989pt}{313.124207pt}}
\pgfpathclose
\pgfusepath{fill,stroke}
\color[rgb]{0.185538,0.704725,0.485412}
\pgfpathmoveto{\pgfpoint{110.591980pt}{325.477905pt}}
\pgflineto{\pgfpoint{119.519989pt}{319.301056pt}}
\pgflineto{\pgfpoint{110.591980pt}{319.301056pt}}
\pgfpathclose
\pgfusepath{fill,stroke}
\pgfpathmoveto{\pgfpoint{110.591980pt}{325.477905pt}}
\pgflineto{\pgfpoint{119.519989pt}{325.477905pt}}
\pgflineto{\pgfpoint{119.519989pt}{319.301056pt}}
\pgfpathclose
\pgfusepath{fill,stroke}
\color[rgb]{0.208030,0.718701,0.472873}
\pgfpathmoveto{\pgfpoint{110.591980pt}{331.654724pt}}
\pgflineto{\pgfpoint{119.519989pt}{325.477905pt}}
\pgflineto{\pgfpoint{110.591980pt}{325.477905pt}}
\pgfpathclose
\pgfusepath{fill,stroke}
\pgfpathmoveto{\pgfpoint{110.591980pt}{331.654724pt}}
\pgflineto{\pgfpoint{119.519989pt}{331.654724pt}}
\pgflineto{\pgfpoint{119.519989pt}{325.477905pt}}
\pgfpathclose
\pgfusepath{fill,stroke}
\pgfpathmoveto{\pgfpoint{110.591980pt}{337.831604pt}}
\pgflineto{\pgfpoint{119.519989pt}{331.654724pt}}
\pgflineto{\pgfpoint{110.591980pt}{331.654724pt}}
\pgfpathclose
\pgfusepath{fill,stroke}
\pgfpathmoveto{\pgfpoint{110.591980pt}{337.831604pt}}
\pgflineto{\pgfpoint{119.519989pt}{337.831604pt}}
\pgflineto{\pgfpoint{119.519989pt}{331.654724pt}}
\pgfpathclose
\pgfusepath{fill,stroke}
\color[rgb]{0.233127,0.732406,0.459106}
\pgfpathmoveto{\pgfpoint{110.591980pt}{344.008423pt}}
\pgflineto{\pgfpoint{119.519989pt}{337.831604pt}}
\pgflineto{\pgfpoint{110.591980pt}{337.831604pt}}
\pgfpathclose
\pgfusepath{fill,stroke}
\color[rgb]{0.165967,0.690519,0.496752}
\pgfpathmoveto{\pgfpoint{119.519989pt}{313.124207pt}}
\pgflineto{\pgfpoint{128.447998pt}{313.124207pt}}
\pgflineto{\pgfpoint{128.447998pt}{306.947388pt}}
\pgfpathclose
\pgfusepath{fill,stroke}
\color[rgb]{0.185538,0.704725,0.485412}
\pgfpathmoveto{\pgfpoint{119.519989pt}{319.301056pt}}
\pgflineto{\pgfpoint{128.447998pt}{313.124207pt}}
\pgflineto{\pgfpoint{119.519989pt}{313.124207pt}}
\pgfpathclose
\pgfusepath{fill,stroke}
\pgfpathmoveto{\pgfpoint{119.519989pt}{319.301056pt}}
\pgflineto{\pgfpoint{128.447998pt}{319.301056pt}}
\pgflineto{\pgfpoint{128.447998pt}{313.124207pt}}
\pgfpathclose
\pgfusepath{fill,stroke}
\color[rgb]{0.208030,0.718701,0.472873}
\pgfpathmoveto{\pgfpoint{119.519989pt}{325.477905pt}}
\pgflineto{\pgfpoint{128.447998pt}{319.301056pt}}
\pgflineto{\pgfpoint{119.519989pt}{319.301056pt}}
\pgfpathclose
\pgfusepath{fill,stroke}
\color[rgb]{0.165967,0.690519,0.496752}
\pgfpathmoveto{\pgfpoint{128.447998pt}{306.947388pt}}
\pgflineto{\pgfpoint{137.376007pt}{306.947388pt}}
\pgflineto{\pgfpoint{137.376007pt}{300.770538pt}}
\pgfpathclose
\pgfusepath{fill,stroke}
\color[rgb]{0.185538,0.704725,0.485412}
\pgfpathmoveto{\pgfpoint{128.447998pt}{313.124207pt}}
\pgflineto{\pgfpoint{137.376007pt}{306.947388pt}}
\pgflineto{\pgfpoint{128.447998pt}{306.947388pt}}
\pgfpathclose
\pgfusepath{fill,stroke}
\color[rgb]{0.135833,0.542750,0.554289}
\pgfpathmoveto{\pgfpoint{74.880005pt}{276.063141pt}}
\pgflineto{\pgfpoint{83.807999pt}{276.063141pt}}
\pgflineto{\pgfpoint{83.807999pt}{269.886322pt}}
\pgfpathclose
\pgfusepath{fill,stroke}
\color[rgb]{0.130582,0.557652,0.552176}
\pgfpathmoveto{\pgfpoint{74.880005pt}{282.239990pt}}
\pgflineto{\pgfpoint{83.807999pt}{276.063141pt}}
\pgflineto{\pgfpoint{74.880005pt}{276.063141pt}}
\pgfpathclose
\pgfusepath{fill,stroke}
\pgfpathmoveto{\pgfpoint{74.880005pt}{282.239990pt}}
\pgflineto{\pgfpoint{83.807999pt}{282.239990pt}}
\pgflineto{\pgfpoint{83.807999pt}{276.063141pt}}
\pgfpathclose
\pgfusepath{fill,stroke}
\pgfpathmoveto{\pgfpoint{74.880005pt}{288.416840pt}}
\pgflineto{\pgfpoint{83.807999pt}{282.239990pt}}
\pgflineto{\pgfpoint{74.880005pt}{282.239990pt}}
\pgfpathclose
\pgfusepath{fill,stroke}
\pgfpathmoveto{\pgfpoint{74.880005pt}{288.416840pt}}
\pgflineto{\pgfpoint{83.807999pt}{288.416840pt}}
\pgflineto{\pgfpoint{83.807999pt}{282.239990pt}}
\pgfpathclose
\pgfusepath{fill,stroke}
\color[rgb]{0.125898,0.572563,0.549445}
\pgfpathmoveto{\pgfpoint{74.880005pt}{294.593689pt}}
\pgflineto{\pgfpoint{83.807999pt}{288.416840pt}}
\pgflineto{\pgfpoint{74.880005pt}{288.416840pt}}
\pgfpathclose
\pgfusepath{fill,stroke}
\pgfpathmoveto{\pgfpoint{74.880005pt}{294.593689pt}}
\pgflineto{\pgfpoint{83.807999pt}{294.593689pt}}
\pgflineto{\pgfpoint{83.807999pt}{288.416840pt}}
\pgfpathclose
\pgfusepath{fill,stroke}
\color[rgb]{0.122163,0.587476,0.546023}
\pgfpathmoveto{\pgfpoint{74.880005pt}{300.770538pt}}
\pgflineto{\pgfpoint{83.807999pt}{294.593689pt}}
\pgflineto{\pgfpoint{74.880005pt}{294.593689pt}}
\pgfpathclose
\pgfusepath{fill,stroke}
\pgfpathmoveto{\pgfpoint{74.880005pt}{300.770538pt}}
\pgflineto{\pgfpoint{83.807999pt}{300.770538pt}}
\pgflineto{\pgfpoint{83.807999pt}{294.593689pt}}
\pgfpathclose
\pgfusepath{fill,stroke}
\color[rgb]{0.119872,0.602382,0.541831}
\pgfpathmoveto{\pgfpoint{74.880005pt}{306.947388pt}}
\pgflineto{\pgfpoint{83.807999pt}{300.770538pt}}
\pgflineto{\pgfpoint{74.880005pt}{300.770538pt}}
\pgfpathclose
\pgfusepath{fill,stroke}
\pgfpathmoveto{\pgfpoint{74.880005pt}{306.947388pt}}
\pgflineto{\pgfpoint{83.807999pt}{306.947388pt}}
\pgflineto{\pgfpoint{83.807999pt}{300.770538pt}}
\pgfpathclose
\pgfusepath{fill,stroke}
\color[rgb]{0.119627,0.617266,0.536796}
\pgfpathmoveto{\pgfpoint{74.880005pt}{313.124207pt}}
\pgflineto{\pgfpoint{83.807999pt}{306.947388pt}}
\pgflineto{\pgfpoint{74.880005pt}{306.947388pt}}
\pgfpathclose
\pgfusepath{fill,stroke}
\pgfpathmoveto{\pgfpoint{74.880005pt}{313.124207pt}}
\pgflineto{\pgfpoint{83.807999pt}{313.124207pt}}
\pgflineto{\pgfpoint{83.807999pt}{306.947388pt}}
\pgfpathclose
\pgfusepath{fill,stroke}
\color[rgb]{0.122046,0.632107,0.530848}
\pgfpathmoveto{\pgfpoint{74.880005pt}{319.301056pt}}
\pgflineto{\pgfpoint{83.807999pt}{313.124207pt}}
\pgflineto{\pgfpoint{74.880005pt}{313.124207pt}}
\pgfpathclose
\pgfusepath{fill,stroke}
\color[rgb]{0.135833,0.542750,0.554289}
\pgfpathmoveto{\pgfpoint{83.807999pt}{269.886322pt}}
\pgflineto{\pgfpoint{92.735992pt}{269.886322pt}}
\pgflineto{\pgfpoint{92.735992pt}{263.709473pt}}
\pgfpathclose
\pgfusepath{fill,stroke}
\color[rgb]{0.130582,0.557652,0.552176}
\pgfpathmoveto{\pgfpoint{83.807999pt}{276.063141pt}}
\pgflineto{\pgfpoint{92.735992pt}{269.886322pt}}
\pgflineto{\pgfpoint{83.807999pt}{269.886322pt}}
\pgfpathclose
\pgfusepath{fill,stroke}
\pgfpathmoveto{\pgfpoint{83.807999pt}{276.063141pt}}
\pgflineto{\pgfpoint{92.735992pt}{276.063141pt}}
\pgflineto{\pgfpoint{92.735992pt}{269.886322pt}}
\pgfpathclose
\pgfusepath{fill,stroke}
\color[rgb]{0.125898,0.572563,0.549445}
\pgfpathmoveto{\pgfpoint{83.807999pt}{282.239990pt}}
\pgflineto{\pgfpoint{92.735992pt}{276.063141pt}}
\pgflineto{\pgfpoint{83.807999pt}{276.063141pt}}
\pgfpathclose
\pgfusepath{fill,stroke}
\pgfpathmoveto{\pgfpoint{83.807999pt}{282.239990pt}}
\pgflineto{\pgfpoint{92.735992pt}{282.239990pt}}
\pgflineto{\pgfpoint{92.735992pt}{276.063141pt}}
\pgfpathclose
\pgfusepath{fill,stroke}
\color[rgb]{0.122163,0.587476,0.546023}
\pgfpathmoveto{\pgfpoint{83.807999pt}{288.416840pt}}
\pgflineto{\pgfpoint{92.735992pt}{282.239990pt}}
\pgflineto{\pgfpoint{83.807999pt}{282.239990pt}}
\pgfpathclose
\pgfusepath{fill,stroke}
\pgfpathmoveto{\pgfpoint{83.807999pt}{288.416840pt}}
\pgflineto{\pgfpoint{92.735992pt}{288.416840pt}}
\pgflineto{\pgfpoint{92.735992pt}{282.239990pt}}
\pgfpathclose
\pgfusepath{fill,stroke}
\pgfpathmoveto{\pgfpoint{83.807999pt}{294.593689pt}}
\pgflineto{\pgfpoint{92.735992pt}{288.416840pt}}
\pgflineto{\pgfpoint{83.807999pt}{288.416840pt}}
\pgfpathclose
\pgfusepath{fill,stroke}
\pgfpathmoveto{\pgfpoint{83.807999pt}{294.593689pt}}
\pgflineto{\pgfpoint{92.735992pt}{294.593689pt}}
\pgflineto{\pgfpoint{92.735992pt}{288.416840pt}}
\pgfpathclose
\pgfusepath{fill,stroke}
\color[rgb]{0.119872,0.602382,0.541831}
\pgfpathmoveto{\pgfpoint{83.807999pt}{300.770538pt}}
\pgflineto{\pgfpoint{92.735992pt}{294.593689pt}}
\pgflineto{\pgfpoint{83.807999pt}{294.593689pt}}
\pgfpathclose
\pgfusepath{fill,stroke}
\pgfpathmoveto{\pgfpoint{83.807999pt}{300.770538pt}}
\pgflineto{\pgfpoint{92.735992pt}{300.770538pt}}
\pgflineto{\pgfpoint{92.735992pt}{294.593689pt}}
\pgfpathclose
\pgfusepath{fill,stroke}
\color[rgb]{0.119627,0.617266,0.536796}
\pgfpathmoveto{\pgfpoint{83.807999pt}{306.947388pt}}
\pgflineto{\pgfpoint{92.735992pt}{300.770538pt}}
\pgflineto{\pgfpoint{83.807999pt}{300.770538pt}}
\pgfpathclose
\pgfusepath{fill,stroke}
\color[rgb]{0.130582,0.557652,0.552176}
\pgfpathmoveto{\pgfpoint{92.735992pt}{269.886322pt}}
\pgflineto{\pgfpoint{101.664001pt}{263.709473pt}}
\pgflineto{\pgfpoint{92.735992pt}{263.709473pt}}
\pgfpathclose
\pgfusepath{fill,stroke}
\pgfpathmoveto{\pgfpoint{92.735992pt}{269.886322pt}}
\pgflineto{\pgfpoint{101.664001pt}{269.886322pt}}
\pgflineto{\pgfpoint{101.664001pt}{263.709473pt}}
\pgfpathclose
\pgfusepath{fill,stroke}
\color[rgb]{0.125898,0.572563,0.549445}
\pgfpathmoveto{\pgfpoint{92.735992pt}{276.063141pt}}
\pgflineto{\pgfpoint{101.664001pt}{269.886322pt}}
\pgflineto{\pgfpoint{92.735992pt}{269.886322pt}}
\pgfpathclose
\pgfusepath{fill,stroke}
\pgfpathmoveto{\pgfpoint{92.735992pt}{276.063141pt}}
\pgflineto{\pgfpoint{101.664001pt}{276.063141pt}}
\pgflineto{\pgfpoint{101.664001pt}{269.886322pt}}
\pgfpathclose
\pgfusepath{fill,stroke}
\color[rgb]{0.122163,0.587476,0.546023}
\pgfpathmoveto{\pgfpoint{92.735992pt}{282.239990pt}}
\pgflineto{\pgfpoint{101.664001pt}{276.063141pt}}
\pgflineto{\pgfpoint{92.735992pt}{276.063141pt}}
\pgfpathclose
\pgfusepath{fill,stroke}
\pgfpathmoveto{\pgfpoint{92.735992pt}{282.239990pt}}
\pgflineto{\pgfpoint{101.664001pt}{282.239990pt}}
\pgflineto{\pgfpoint{101.664001pt}{276.063141pt}}
\pgfpathclose
\pgfusepath{fill,stroke}
\color[rgb]{0.119872,0.602382,0.541831}
\pgfpathmoveto{\pgfpoint{92.735992pt}{288.416840pt}}
\pgflineto{\pgfpoint{101.664001pt}{282.239990pt}}
\pgflineto{\pgfpoint{92.735992pt}{282.239990pt}}
\pgfpathclose
\pgfusepath{fill,stroke}
\color[rgb]{0.130582,0.557652,0.552176}
\pgfpathmoveto{\pgfpoint{101.664001pt}{263.709473pt}}
\pgflineto{\pgfpoint{110.591980pt}{263.709473pt}}
\pgflineto{\pgfpoint{110.591980pt}{257.532623pt}}
\pgfpathclose
\pgfusepath{fill,stroke}
\color[rgb]{0.125898,0.572563,0.549445}
\pgfpathmoveto{\pgfpoint{101.664001pt}{269.886322pt}}
\pgflineto{\pgfpoint{110.591980pt}{263.709473pt}}
\pgflineto{\pgfpoint{101.664001pt}{263.709473pt}}
\pgfpathclose
\pgfusepath{fill,stroke}
\color[rgb]{0.177272,0.437886,0.557576}
\pgfpathmoveto{\pgfpoint{74.880005pt}{226.648422pt}}
\pgflineto{\pgfpoint{83.807999pt}{226.648422pt}}
\pgflineto{\pgfpoint{83.807999pt}{220.471588pt}}
\pgfpathclose
\pgfusepath{fill,stroke}
\color[rgb]{0.170958,0.453063,0.557974}
\pgfpathmoveto{\pgfpoint{74.880005pt}{232.825272pt}}
\pgflineto{\pgfpoint{83.807999pt}{226.648422pt}}
\pgflineto{\pgfpoint{74.880005pt}{226.648422pt}}
\pgfpathclose
\pgfusepath{fill,stroke}
\pgfpathmoveto{\pgfpoint{74.880005pt}{232.825272pt}}
\pgflineto{\pgfpoint{83.807999pt}{232.825272pt}}
\pgflineto{\pgfpoint{83.807999pt}{226.648422pt}}
\pgfpathclose
\pgfusepath{fill,stroke}
\color[rgb]{0.164833,0.468130,0.558143}
\pgfpathmoveto{\pgfpoint{74.880005pt}{239.002106pt}}
\pgflineto{\pgfpoint{83.807999pt}{232.825272pt}}
\pgflineto{\pgfpoint{74.880005pt}{232.825272pt}}
\pgfpathclose
\pgfusepath{fill,stroke}
\pgfpathmoveto{\pgfpoint{74.880005pt}{239.002106pt}}
\pgflineto{\pgfpoint{83.807999pt}{239.002106pt}}
\pgflineto{\pgfpoint{83.807999pt}{232.825272pt}}
\pgfpathclose
\pgfusepath{fill,stroke}
\color[rgb]{0.158845,0.483117,0.558059}
\pgfpathmoveto{\pgfpoint{74.880005pt}{245.178955pt}}
\pgflineto{\pgfpoint{83.807999pt}{239.002106pt}}
\pgflineto{\pgfpoint{74.880005pt}{239.002106pt}}
\pgfpathclose
\pgfusepath{fill,stroke}
\pgfpathmoveto{\pgfpoint{74.880005pt}{245.178955pt}}
\pgflineto{\pgfpoint{83.807999pt}{245.178955pt}}
\pgflineto{\pgfpoint{83.807999pt}{239.002106pt}}
\pgfpathclose
\pgfusepath{fill,stroke}
\pgfpathmoveto{\pgfpoint{74.880005pt}{251.355804pt}}
\pgflineto{\pgfpoint{83.807999pt}{245.178955pt}}
\pgflineto{\pgfpoint{74.880005pt}{245.178955pt}}
\pgfpathclose
\pgfusepath{fill,stroke}
\pgfpathmoveto{\pgfpoint{74.880005pt}{251.355804pt}}
\pgflineto{\pgfpoint{83.807999pt}{251.355804pt}}
\pgflineto{\pgfpoint{83.807999pt}{245.178955pt}}
\pgfpathclose
\pgfusepath{fill,stroke}
\color[rgb]{0.152951,0.498053,0.557685}
\pgfpathmoveto{\pgfpoint{74.880005pt}{257.532623pt}}
\pgflineto{\pgfpoint{83.807999pt}{251.355804pt}}
\pgflineto{\pgfpoint{74.880005pt}{251.355804pt}}
\pgfpathclose
\pgfusepath{fill,stroke}
\pgfpathmoveto{\pgfpoint{74.880005pt}{257.532623pt}}
\pgflineto{\pgfpoint{83.807999pt}{257.532623pt}}
\pgflineto{\pgfpoint{83.807999pt}{251.355804pt}}
\pgfpathclose
\pgfusepath{fill,stroke}
\color[rgb]{0.147132,0.512959,0.556973}
\pgfpathmoveto{\pgfpoint{74.880005pt}{263.709473pt}}
\pgflineto{\pgfpoint{83.807999pt}{257.532623pt}}
\pgflineto{\pgfpoint{74.880005pt}{257.532623pt}}
\pgfpathclose
\pgfusepath{fill,stroke}
\color[rgb]{0.164833,0.468130,0.558143}
\pgfpathmoveto{\pgfpoint{83.807999pt}{232.825272pt}}
\pgflineto{\pgfpoint{92.735992pt}{226.648422pt}}
\pgflineto{\pgfpoint{83.807999pt}{226.648422pt}}
\pgfpathclose
\pgfusepath{fill,stroke}
\pgfpathmoveto{\pgfpoint{83.807999pt}{232.825272pt}}
\pgflineto{\pgfpoint{92.735992pt}{232.825272pt}}
\pgflineto{\pgfpoint{92.735992pt}{226.648422pt}}
\pgfpathclose
\pgfusepath{fill,stroke}
\color[rgb]{0.158845,0.483117,0.558059}
\pgfpathmoveto{\pgfpoint{83.807999pt}{239.002106pt}}
\pgflineto{\pgfpoint{92.735992pt}{232.825272pt}}
\pgflineto{\pgfpoint{83.807999pt}{232.825272pt}}
\pgfpathclose
\pgfusepath{fill,stroke}
\pgfpathmoveto{\pgfpoint{83.807999pt}{239.002106pt}}
\pgflineto{\pgfpoint{92.735992pt}{239.002106pt}}
\pgflineto{\pgfpoint{92.735992pt}{232.825272pt}}
\pgfpathclose
\pgfusepath{fill,stroke}
\color[rgb]{0.152951,0.498053,0.557685}
\pgfpathmoveto{\pgfpoint{83.807999pt}{245.178955pt}}
\pgflineto{\pgfpoint{92.735992pt}{239.002106pt}}
\pgflineto{\pgfpoint{83.807999pt}{239.002106pt}}
\pgfpathclose
\pgfusepath{fill,stroke}
\color[rgb]{0.220425,0.342517,0.549287}
\pgfpathmoveto{\pgfpoint{74.880005pt}{183.410522pt}}
\pgflineto{\pgfpoint{83.807999pt}{177.233673pt}}
\pgflineto{\pgfpoint{74.880005pt}{177.233673pt}}
\pgfpathclose
\pgfusepath{fill,stroke}
\pgfpathmoveto{\pgfpoint{74.880005pt}{183.410522pt}}
\pgflineto{\pgfpoint{83.807999pt}{183.410522pt}}
\pgflineto{\pgfpoint{83.807999pt}{177.233673pt}}
\pgfpathclose
\pgfusepath{fill,stroke}
\color[rgb]{0.212667,0.359102,0.551635}
\pgfpathmoveto{\pgfpoint{74.880005pt}{189.587372pt}}
\pgflineto{\pgfpoint{83.807999pt}{183.410522pt}}
\pgflineto{\pgfpoint{74.880005pt}{183.410522pt}}
\pgfpathclose
\pgfusepath{fill,stroke}
\pgfpathmoveto{\pgfpoint{74.880005pt}{189.587372pt}}
\pgflineto{\pgfpoint{83.807999pt}{189.587372pt}}
\pgflineto{\pgfpoint{83.807999pt}{183.410522pt}}
\pgfpathclose
\pgfusepath{fill,stroke}
\color[rgb]{0.205079,0.375366,0.553493}
\pgfpathmoveto{\pgfpoint{74.880005pt}{195.764206pt}}
\pgflineto{\pgfpoint{83.807999pt}{189.587372pt}}
\pgflineto{\pgfpoint{74.880005pt}{189.587372pt}}
\pgfpathclose
\pgfusepath{fill,stroke}
\pgfpathmoveto{\pgfpoint{74.880005pt}{195.764206pt}}
\pgflineto{\pgfpoint{83.807999pt}{195.764206pt}}
\pgflineto{\pgfpoint{83.807999pt}{189.587372pt}}
\pgfpathclose
\pgfusepath{fill,stroke}
\color[rgb]{0.197722,0.391341,0.554953}
\pgfpathmoveto{\pgfpoint{74.880005pt}{201.941055pt}}
\pgflineto{\pgfpoint{83.807999pt}{195.764206pt}}
\pgflineto{\pgfpoint{74.880005pt}{195.764206pt}}
\pgfpathclose
\pgfusepath{fill,stroke}
\pgfpathmoveto{\pgfpoint{74.880005pt}{201.941055pt}}
\pgflineto{\pgfpoint{83.807999pt}{201.941055pt}}
\pgflineto{\pgfpoint{83.807999pt}{195.764206pt}}
\pgfpathclose
\pgfusepath{fill,stroke}
\color[rgb]{0.190631,0.407061,0.556089}
\pgfpathmoveto{\pgfpoint{74.880005pt}{208.117905pt}}
\pgflineto{\pgfpoint{83.807999pt}{201.941055pt}}
\pgflineto{\pgfpoint{74.880005pt}{201.941055pt}}
\pgfpathclose
\pgfusepath{fill,stroke}
\pgfpathmoveto{\pgfpoint{74.880005pt}{208.117905pt}}
\pgflineto{\pgfpoint{83.807999pt}{208.117905pt}}
\pgflineto{\pgfpoint{83.807999pt}{201.941055pt}}
\pgfpathclose
\pgfusepath{fill,stroke}
\pgfpathmoveto{\pgfpoint{74.880005pt}{214.294739pt}}
\pgflineto{\pgfpoint{83.807999pt}{208.117905pt}}
\pgflineto{\pgfpoint{74.880005pt}{208.117905pt}}
\pgfpathclose
\pgfusepath{fill,stroke}
\pgfpathmoveto{\pgfpoint{74.880005pt}{214.294739pt}}
\pgflineto{\pgfpoint{83.807999pt}{214.294739pt}}
\pgflineto{\pgfpoint{83.807999pt}{208.117905pt}}
\pgfpathclose
\pgfusepath{fill,stroke}
\color[rgb]{0.220425,0.342517,0.549287}
\pgfpathmoveto{\pgfpoint{83.807999pt}{177.233673pt}}
\pgflineto{\pgfpoint{92.735992pt}{177.233673pt}}
\pgflineto{\pgfpoint{92.735992pt}{171.056854pt}}
\pgfpathclose
\pgfusepath{fill,stroke}
\color[rgb]{0.212667,0.359102,0.551635}
\pgfpathmoveto{\pgfpoint{83.807999pt}{183.410522pt}}
\pgflineto{\pgfpoint{92.735992pt}{177.233673pt}}
\pgflineto{\pgfpoint{83.807999pt}{177.233673pt}}
\pgfpathclose
\pgfusepath{fill,stroke}
\pgfpathmoveto{\pgfpoint{83.807999pt}{183.410522pt}}
\pgflineto{\pgfpoint{92.735992pt}{183.410522pt}}
\pgflineto{\pgfpoint{92.735992pt}{177.233673pt}}
\pgfpathclose
\pgfusepath{fill,stroke}
\color[rgb]{0.205079,0.375366,0.553493}
\pgfpathmoveto{\pgfpoint{83.807999pt}{189.587372pt}}
\pgflineto{\pgfpoint{92.735992pt}{183.410522pt}}
\pgflineto{\pgfpoint{83.807999pt}{183.410522pt}}
\pgfpathclose
\pgfusepath{fill,stroke}
\pgfpathmoveto{\pgfpoint{83.807999pt}{189.587372pt}}
\pgflineto{\pgfpoint{92.735992pt}{189.587372pt}}
\pgflineto{\pgfpoint{92.735992pt}{183.410522pt}}
\pgfpathclose
\pgfusepath{fill,stroke}
\color[rgb]{0.197722,0.391341,0.554953}
\pgfpathmoveto{\pgfpoint{83.807999pt}{195.764206pt}}
\pgflineto{\pgfpoint{92.735992pt}{189.587372pt}}
\pgflineto{\pgfpoint{83.807999pt}{189.587372pt}}
\pgfpathclose
\pgfusepath{fill,stroke}
\pgfpathmoveto{\pgfpoint{83.807999pt}{195.764206pt}}
\pgflineto{\pgfpoint{92.735992pt}{195.764206pt}}
\pgflineto{\pgfpoint{92.735992pt}{189.587372pt}}
\pgfpathclose
\pgfusepath{fill,stroke}
\color[rgb]{0.190631,0.407061,0.556089}
\pgfpathmoveto{\pgfpoint{83.807999pt}{201.941055pt}}
\pgflineto{\pgfpoint{92.735992pt}{195.764206pt}}
\pgflineto{\pgfpoint{83.807999pt}{195.764206pt}}
\pgfpathclose
\pgfusepath{fill,stroke}
\pgfpathmoveto{\pgfpoint{83.807999pt}{201.941055pt}}
\pgflineto{\pgfpoint{92.735992pt}{201.941055pt}}
\pgflineto{\pgfpoint{92.735992pt}{195.764206pt}}
\pgfpathclose
\pgfusepath{fill,stroke}
\color[rgb]{0.183819,0.422564,0.556952}
\pgfpathmoveto{\pgfpoint{83.807999pt}{208.117905pt}}
\pgflineto{\pgfpoint{92.735992pt}{201.941055pt}}
\pgflineto{\pgfpoint{83.807999pt}{201.941055pt}}
\pgfpathclose
\pgfusepath{fill,stroke}
\pgfpathmoveto{\pgfpoint{83.807999pt}{208.117905pt}}
\pgflineto{\pgfpoint{92.735992pt}{208.117905pt}}
\pgflineto{\pgfpoint{92.735992pt}{201.941055pt}}
\pgfpathclose
\pgfusepath{fill,stroke}
\pgfpathmoveto{\pgfpoint{83.807999pt}{214.294739pt}}
\pgflineto{\pgfpoint{92.735992pt}{208.117905pt}}
\pgflineto{\pgfpoint{83.807999pt}{208.117905pt}}
\pgfpathclose
\pgfusepath{fill,stroke}
\pgfpathmoveto{\pgfpoint{83.807999pt}{214.294739pt}}
\pgflineto{\pgfpoint{92.735992pt}{214.294739pt}}
\pgflineto{\pgfpoint{92.735992pt}{208.117905pt}}
\pgfpathclose
\pgfusepath{fill,stroke}
\color[rgb]{0.177272,0.437886,0.557576}
\pgfpathmoveto{\pgfpoint{83.807999pt}{220.471588pt}}
\pgflineto{\pgfpoint{92.735992pt}{214.294739pt}}
\pgflineto{\pgfpoint{83.807999pt}{214.294739pt}}
\pgfpathclose
\pgfusepath{fill,stroke}
\pgfpathmoveto{\pgfpoint{83.807999pt}{220.471588pt}}
\pgflineto{\pgfpoint{92.735992pt}{220.471588pt}}
\pgflineto{\pgfpoint{92.735992pt}{214.294739pt}}
\pgfpathclose
\pgfusepath{fill,stroke}
\color[rgb]{0.170958,0.453063,0.557974}
\pgfpathmoveto{\pgfpoint{83.807999pt}{226.648422pt}}
\pgflineto{\pgfpoint{92.735992pt}{220.471588pt}}
\pgflineto{\pgfpoint{83.807999pt}{220.471588pt}}
\pgfpathclose
\pgfusepath{fill,stroke}
\pgfpathmoveto{\pgfpoint{83.807999pt}{226.648422pt}}
\pgflineto{\pgfpoint{92.735992pt}{226.648422pt}}
\pgflineto{\pgfpoint{92.735992pt}{220.471588pt}}
\pgfpathclose
\pgfusepath{fill,stroke}
\color[rgb]{0.212667,0.359102,0.551635}
\pgfpathmoveto{\pgfpoint{92.735992pt}{177.233673pt}}
\pgflineto{\pgfpoint{101.664001pt}{171.056854pt}}
\pgflineto{\pgfpoint{92.735992pt}{171.056854pt}}
\pgfpathclose
\pgfusepath{fill,stroke}
\pgfpathmoveto{\pgfpoint{92.735992pt}{177.233673pt}}
\pgflineto{\pgfpoint{101.664001pt}{177.233673pt}}
\pgflineto{\pgfpoint{101.664001pt}{171.056854pt}}
\pgfpathclose
\pgfusepath{fill,stroke}
\color[rgb]{0.205079,0.375366,0.553493}
\pgfpathmoveto{\pgfpoint{92.735992pt}{183.410522pt}}
\pgflineto{\pgfpoint{101.664001pt}{177.233673pt}}
\pgflineto{\pgfpoint{92.735992pt}{177.233673pt}}
\pgfpathclose
\pgfusepath{fill,stroke}
\pgfpathmoveto{\pgfpoint{92.735992pt}{183.410522pt}}
\pgflineto{\pgfpoint{101.664001pt}{183.410522pt}}
\pgflineto{\pgfpoint{101.664001pt}{177.233673pt}}
\pgfpathclose
\pgfusepath{fill,stroke}
\color[rgb]{0.197722,0.391341,0.554953}
\pgfpathmoveto{\pgfpoint{92.735992pt}{189.587372pt}}
\pgflineto{\pgfpoint{101.664001pt}{183.410522pt}}
\pgflineto{\pgfpoint{92.735992pt}{183.410522pt}}
\pgfpathclose
\pgfusepath{fill,stroke}
\pgfpathmoveto{\pgfpoint{92.735992pt}{189.587372pt}}
\pgflineto{\pgfpoint{101.664001pt}{189.587372pt}}
\pgflineto{\pgfpoint{101.664001pt}{183.410522pt}}
\pgfpathclose
\pgfusepath{fill,stroke}
\color[rgb]{0.190631,0.407061,0.556089}
\pgfpathmoveto{\pgfpoint{92.735992pt}{195.764206pt}}
\pgflineto{\pgfpoint{101.664001pt}{189.587372pt}}
\pgflineto{\pgfpoint{92.735992pt}{189.587372pt}}
\pgfpathclose
\pgfusepath{fill,stroke}
\pgfpathmoveto{\pgfpoint{92.735992pt}{195.764206pt}}
\pgflineto{\pgfpoint{101.664001pt}{195.764206pt}}
\pgflineto{\pgfpoint{101.664001pt}{189.587372pt}}
\pgfpathclose
\pgfusepath{fill,stroke}
\color[rgb]{0.183819,0.422564,0.556952}
\pgfpathmoveto{\pgfpoint{92.735992pt}{201.941055pt}}
\pgflineto{\pgfpoint{101.664001pt}{195.764206pt}}
\pgflineto{\pgfpoint{92.735992pt}{195.764206pt}}
\pgfpathclose
\pgfusepath{fill,stroke}
\pgfpathmoveto{\pgfpoint{92.735992pt}{201.941055pt}}
\pgflineto{\pgfpoint{101.664001pt}{201.941055pt}}
\pgflineto{\pgfpoint{101.664001pt}{195.764206pt}}
\pgfpathclose
\pgfusepath{fill,stroke}
\color[rgb]{0.177272,0.437886,0.557576}
\pgfpathmoveto{\pgfpoint{92.735992pt}{208.117905pt}}
\pgflineto{\pgfpoint{101.664001pt}{201.941055pt}}
\pgflineto{\pgfpoint{92.735992pt}{201.941055pt}}
\pgfpathclose
\pgfusepath{fill,stroke}
\pgfpathmoveto{\pgfpoint{92.735992pt}{208.117905pt}}
\pgflineto{\pgfpoint{101.664001pt}{208.117905pt}}
\pgflineto{\pgfpoint{101.664001pt}{201.941055pt}}
\pgfpathclose
\pgfusepath{fill,stroke}
\color[rgb]{0.170958,0.453063,0.557974}
\pgfpathmoveto{\pgfpoint{92.735992pt}{214.294739pt}}
\pgflineto{\pgfpoint{101.664001pt}{208.117905pt}}
\pgflineto{\pgfpoint{92.735992pt}{208.117905pt}}
\pgfpathclose
\pgfusepath{fill,stroke}
\pgfpathmoveto{\pgfpoint{92.735992pt}{214.294739pt}}
\pgflineto{\pgfpoint{101.664001pt}{214.294739pt}}
\pgflineto{\pgfpoint{101.664001pt}{208.117905pt}}
\pgfpathclose
\pgfusepath{fill,stroke}
\pgfpathmoveto{\pgfpoint{92.735992pt}{220.471588pt}}
\pgflineto{\pgfpoint{101.664001pt}{214.294739pt}}
\pgflineto{\pgfpoint{92.735992pt}{214.294739pt}}
\pgfpathclose
\pgfusepath{fill,stroke}
\pgfpathmoveto{\pgfpoint{92.735992pt}{220.471588pt}}
\pgflineto{\pgfpoint{101.664001pt}{220.471588pt}}
\pgflineto{\pgfpoint{101.664001pt}{214.294739pt}}
\pgfpathclose
\pgfusepath{fill,stroke}
\color[rgb]{0.164833,0.468130,0.558143}
\pgfpathmoveto{\pgfpoint{92.735992pt}{226.648422pt}}
\pgflineto{\pgfpoint{101.664001pt}{220.471588pt}}
\pgflineto{\pgfpoint{92.735992pt}{220.471588pt}}
\pgfpathclose
\pgfusepath{fill,stroke}
\pgfpathmoveto{\pgfpoint{92.735992pt}{226.648422pt}}
\pgflineto{\pgfpoint{101.664001pt}{226.648422pt}}
\pgflineto{\pgfpoint{101.664001pt}{220.471588pt}}
\pgfpathclose
\pgfusepath{fill,stroke}
\color[rgb]{0.158845,0.483117,0.558059}
\pgfpathmoveto{\pgfpoint{92.735992pt}{232.825272pt}}
\pgflineto{\pgfpoint{101.664001pt}{226.648422pt}}
\pgflineto{\pgfpoint{92.735992pt}{226.648422pt}}
\pgfpathclose
\pgfusepath{fill,stroke}
\pgfpathmoveto{\pgfpoint{92.735992pt}{232.825272pt}}
\pgflineto{\pgfpoint{101.664001pt}{232.825272pt}}
\pgflineto{\pgfpoint{101.664001pt}{226.648422pt}}
\pgfpathclose
\pgfusepath{fill,stroke}
\color[rgb]{0.152951,0.498053,0.557685}
\pgfpathmoveto{\pgfpoint{92.735992pt}{239.002106pt}}
\pgflineto{\pgfpoint{101.664001pt}{232.825272pt}}
\pgflineto{\pgfpoint{92.735992pt}{232.825272pt}}
\pgfpathclose
\pgfusepath{fill,stroke}
\pgfpathmoveto{\pgfpoint{92.735992pt}{239.002106pt}}
\pgflineto{\pgfpoint{101.664001pt}{239.002106pt}}
\pgflineto{\pgfpoint{101.664001pt}{232.825272pt}}
\pgfpathclose
\pgfusepath{fill,stroke}
\color[rgb]{0.190631,0.407061,0.556089}
\pgfpathmoveto{\pgfpoint{101.664001pt}{189.587372pt}}
\pgflineto{\pgfpoint{110.591980pt}{183.410522pt}}
\pgflineto{\pgfpoint{101.664001pt}{183.410522pt}}
\pgfpathclose
\pgfusepath{fill,stroke}
\pgfpathmoveto{\pgfpoint{101.664001pt}{189.587372pt}}
\pgflineto{\pgfpoint{110.591980pt}{189.587372pt}}
\pgflineto{\pgfpoint{110.591980pt}{183.410522pt}}
\pgfpathclose
\pgfusepath{fill,stroke}
\color[rgb]{0.183819,0.422564,0.556952}
\pgfpathmoveto{\pgfpoint{101.664001pt}{195.764206pt}}
\pgflineto{\pgfpoint{110.591980pt}{189.587372pt}}
\pgflineto{\pgfpoint{101.664001pt}{189.587372pt}}
\pgfpathclose
\pgfusepath{fill,stroke}
\pgfpathmoveto{\pgfpoint{101.664001pt}{195.764206pt}}
\pgflineto{\pgfpoint{110.591980pt}{195.764206pt}}
\pgflineto{\pgfpoint{110.591980pt}{189.587372pt}}
\pgfpathclose
\pgfusepath{fill,stroke}
\color[rgb]{0.177272,0.437886,0.557576}
\pgfpathmoveto{\pgfpoint{101.664001pt}{201.941055pt}}
\pgflineto{\pgfpoint{110.591980pt}{195.764206pt}}
\pgflineto{\pgfpoint{101.664001pt}{195.764206pt}}
\pgfpathclose
\pgfusepath{fill,stroke}
\pgfpathmoveto{\pgfpoint{101.664001pt}{201.941055pt}}
\pgflineto{\pgfpoint{110.591980pt}{201.941055pt}}
\pgflineto{\pgfpoint{110.591980pt}{195.764206pt}}
\pgfpathclose
\pgfusepath{fill,stroke}
\color[rgb]{0.170958,0.453063,0.557974}
\pgfpathmoveto{\pgfpoint{101.664001pt}{208.117905pt}}
\pgflineto{\pgfpoint{110.591980pt}{201.941055pt}}
\pgflineto{\pgfpoint{101.664001pt}{201.941055pt}}
\pgfpathclose
\pgfusepath{fill,stroke}
\pgfpathmoveto{\pgfpoint{101.664001pt}{208.117905pt}}
\pgflineto{\pgfpoint{110.591980pt}{208.117905pt}}
\pgflineto{\pgfpoint{110.591980pt}{201.941055pt}}
\pgfpathclose
\pgfusepath{fill,stroke}
\color[rgb]{0.164833,0.468130,0.558143}
\pgfpathmoveto{\pgfpoint{101.664001pt}{214.294739pt}}
\pgflineto{\pgfpoint{110.591980pt}{208.117905pt}}
\pgflineto{\pgfpoint{101.664001pt}{208.117905pt}}
\pgfpathclose
\pgfusepath{fill,stroke}
\pgfpathmoveto{\pgfpoint{101.664001pt}{214.294739pt}}
\pgflineto{\pgfpoint{110.591980pt}{214.294739pt}}
\pgflineto{\pgfpoint{110.591980pt}{208.117905pt}}
\pgfpathclose
\pgfusepath{fill,stroke}
\pgfpathmoveto{\pgfpoint{101.664001pt}{220.471588pt}}
\pgflineto{\pgfpoint{110.591980pt}{214.294739pt}}
\pgflineto{\pgfpoint{101.664001pt}{214.294739pt}}
\pgfpathclose
\pgfusepath{fill,stroke}
\pgfpathmoveto{\pgfpoint{101.664001pt}{220.471588pt}}
\pgflineto{\pgfpoint{110.591980pt}{220.471588pt}}
\pgflineto{\pgfpoint{110.591980pt}{214.294739pt}}
\pgfpathclose
\pgfusepath{fill,stroke}
\color[rgb]{0.158845,0.483117,0.558059}
\pgfpathmoveto{\pgfpoint{101.664001pt}{226.648422pt}}
\pgflineto{\pgfpoint{110.591980pt}{220.471588pt}}
\pgflineto{\pgfpoint{101.664001pt}{220.471588pt}}
\pgfpathclose
\pgfusepath{fill,stroke}
\pgfpathmoveto{\pgfpoint{101.664001pt}{226.648422pt}}
\pgflineto{\pgfpoint{110.591980pt}{226.648422pt}}
\pgflineto{\pgfpoint{110.591980pt}{220.471588pt}}
\pgfpathclose
\pgfusepath{fill,stroke}
\color[rgb]{0.152951,0.498053,0.557685}
\pgfpathmoveto{\pgfpoint{101.664001pt}{232.825272pt}}
\pgflineto{\pgfpoint{110.591980pt}{226.648422pt}}
\pgflineto{\pgfpoint{101.664001pt}{226.648422pt}}
\pgfpathclose
\pgfusepath{fill,stroke}
\pgfpathmoveto{\pgfpoint{101.664001pt}{232.825272pt}}
\pgflineto{\pgfpoint{110.591980pt}{232.825272pt}}
\pgflineto{\pgfpoint{110.591980pt}{226.648422pt}}
\pgfpathclose
\pgfusepath{fill,stroke}
\color[rgb]{0.147132,0.512959,0.556973}
\pgfpathmoveto{\pgfpoint{101.664001pt}{239.002106pt}}
\pgflineto{\pgfpoint{110.591980pt}{232.825272pt}}
\pgflineto{\pgfpoint{101.664001pt}{232.825272pt}}
\pgfpathclose
\pgfusepath{fill,stroke}
\pgfpathmoveto{\pgfpoint{101.664001pt}{239.002106pt}}
\pgflineto{\pgfpoint{110.591980pt}{239.002106pt}}
\pgflineto{\pgfpoint{110.591980pt}{232.825272pt}}
\pgfpathclose
\pgfusepath{fill,stroke}
\color[rgb]{0.141402,0.527854,0.555864}
\pgfpathmoveto{\pgfpoint{101.664001pt}{245.178955pt}}
\pgflineto{\pgfpoint{110.591980pt}{239.002106pt}}
\pgflineto{\pgfpoint{101.664001pt}{239.002106pt}}
\pgfpathclose
\pgfusepath{fill,stroke}
\pgfpathmoveto{\pgfpoint{101.664001pt}{245.178955pt}}
\pgflineto{\pgfpoint{110.591980pt}{245.178955pt}}
\pgflineto{\pgfpoint{110.591980pt}{239.002106pt}}
\pgfpathclose
\pgfusepath{fill,stroke}
\color[rgb]{0.135833,0.542750,0.554289}
\pgfpathmoveto{\pgfpoint{101.664001pt}{251.355804pt}}
\pgflineto{\pgfpoint{110.591980pt}{245.178955pt}}
\pgflineto{\pgfpoint{101.664001pt}{245.178955pt}}
\pgfpathclose
\pgfusepath{fill,stroke}
\pgfpathmoveto{\pgfpoint{101.664001pt}{251.355804pt}}
\pgflineto{\pgfpoint{110.591980pt}{251.355804pt}}
\pgflineto{\pgfpoint{110.591980pt}{245.178955pt}}
\pgfpathclose
\pgfusepath{fill,stroke}
\color[rgb]{0.170958,0.453063,0.557974}
\pgfpathmoveto{\pgfpoint{110.591980pt}{201.941055pt}}
\pgflineto{\pgfpoint{119.519989pt}{195.764206pt}}
\pgflineto{\pgfpoint{110.591980pt}{195.764206pt}}
\pgfpathclose
\pgfusepath{fill,stroke}
\pgfpathmoveto{\pgfpoint{110.591980pt}{201.941055pt}}
\pgflineto{\pgfpoint{119.519989pt}{201.941055pt}}
\pgflineto{\pgfpoint{119.519989pt}{195.764206pt}}
\pgfpathclose
\pgfusepath{fill,stroke}
\color[rgb]{0.164833,0.468130,0.558143}
\pgfpathmoveto{\pgfpoint{110.591980pt}{208.117905pt}}
\pgflineto{\pgfpoint{119.519989pt}{201.941055pt}}
\pgflineto{\pgfpoint{110.591980pt}{201.941055pt}}
\pgfpathclose
\pgfusepath{fill,stroke}
\pgfpathmoveto{\pgfpoint{110.591980pt}{208.117905pt}}
\pgflineto{\pgfpoint{119.519989pt}{208.117905pt}}
\pgflineto{\pgfpoint{119.519989pt}{201.941055pt}}
\pgfpathclose
\pgfusepath{fill,stroke}
\color[rgb]{0.158845,0.483117,0.558059}
\pgfpathmoveto{\pgfpoint{110.591980pt}{214.294739pt}}
\pgflineto{\pgfpoint{119.519989pt}{208.117905pt}}
\pgflineto{\pgfpoint{110.591980pt}{208.117905pt}}
\pgfpathclose
\pgfusepath{fill,stroke}
\pgfpathmoveto{\pgfpoint{110.591980pt}{214.294739pt}}
\pgflineto{\pgfpoint{119.519989pt}{214.294739pt}}
\pgflineto{\pgfpoint{119.519989pt}{208.117905pt}}
\pgfpathclose
\pgfusepath{fill,stroke}
\color[rgb]{0.152951,0.498053,0.557685}
\pgfpathmoveto{\pgfpoint{110.591980pt}{220.471588pt}}
\pgflineto{\pgfpoint{119.519989pt}{214.294739pt}}
\pgflineto{\pgfpoint{110.591980pt}{214.294739pt}}
\pgfpathclose
\pgfusepath{fill,stroke}
\pgfpathmoveto{\pgfpoint{110.591980pt}{220.471588pt}}
\pgflineto{\pgfpoint{119.519989pt}{220.471588pt}}
\pgflineto{\pgfpoint{119.519989pt}{214.294739pt}}
\pgfpathclose
\pgfusepath{fill,stroke}
\pgfpathmoveto{\pgfpoint{110.591980pt}{226.648422pt}}
\pgflineto{\pgfpoint{119.519989pt}{220.471588pt}}
\pgflineto{\pgfpoint{110.591980pt}{220.471588pt}}
\pgfpathclose
\pgfusepath{fill,stroke}
\pgfpathmoveto{\pgfpoint{110.591980pt}{226.648422pt}}
\pgflineto{\pgfpoint{119.519989pt}{226.648422pt}}
\pgflineto{\pgfpoint{119.519989pt}{220.471588pt}}
\pgfpathclose
\pgfusepath{fill,stroke}
\color[rgb]{0.147132,0.512959,0.556973}
\pgfpathmoveto{\pgfpoint{110.591980pt}{232.825272pt}}
\pgflineto{\pgfpoint{119.519989pt}{226.648422pt}}
\pgflineto{\pgfpoint{110.591980pt}{226.648422pt}}
\pgfpathclose
\pgfusepath{fill,stroke}
\pgfpathmoveto{\pgfpoint{110.591980pt}{232.825272pt}}
\pgflineto{\pgfpoint{119.519989pt}{232.825272pt}}
\pgflineto{\pgfpoint{119.519989pt}{226.648422pt}}
\pgfpathclose
\pgfusepath{fill,stroke}
\color[rgb]{0.141402,0.527854,0.555864}
\pgfpathmoveto{\pgfpoint{110.591980pt}{239.002106pt}}
\pgflineto{\pgfpoint{119.519989pt}{232.825272pt}}
\pgflineto{\pgfpoint{110.591980pt}{232.825272pt}}
\pgfpathclose
\pgfusepath{fill,stroke}
\pgfpathmoveto{\pgfpoint{110.591980pt}{239.002106pt}}
\pgflineto{\pgfpoint{119.519989pt}{239.002106pt}}
\pgflineto{\pgfpoint{119.519989pt}{232.825272pt}}
\pgfpathclose
\pgfusepath{fill,stroke}
\color[rgb]{0.135833,0.542750,0.554289}
\pgfpathmoveto{\pgfpoint{110.591980pt}{245.178955pt}}
\pgflineto{\pgfpoint{119.519989pt}{239.002106pt}}
\pgflineto{\pgfpoint{110.591980pt}{239.002106pt}}
\pgfpathclose
\pgfusepath{fill,stroke}
\pgfpathmoveto{\pgfpoint{110.591980pt}{245.178955pt}}
\pgflineto{\pgfpoint{119.519989pt}{245.178955pt}}
\pgflineto{\pgfpoint{119.519989pt}{239.002106pt}}
\pgfpathclose
\pgfusepath{fill,stroke}
\color[rgb]{0.130582,0.557652,0.552176}
\pgfpathmoveto{\pgfpoint{110.591980pt}{251.355804pt}}
\pgflineto{\pgfpoint{119.519989pt}{245.178955pt}}
\pgflineto{\pgfpoint{110.591980pt}{245.178955pt}}
\pgfpathclose
\pgfusepath{fill,stroke}
\pgfpathmoveto{\pgfpoint{110.591980pt}{251.355804pt}}
\pgflineto{\pgfpoint{119.519989pt}{251.355804pt}}
\pgflineto{\pgfpoint{119.519989pt}{245.178955pt}}
\pgfpathclose
\pgfusepath{fill,stroke}
\color[rgb]{0.125898,0.572563,0.549445}
\pgfpathmoveto{\pgfpoint{110.591980pt}{257.532623pt}}
\pgflineto{\pgfpoint{119.519989pt}{251.355804pt}}
\pgflineto{\pgfpoint{110.591980pt}{251.355804pt}}
\pgfpathclose
\pgfusepath{fill,stroke}
\pgfpathmoveto{\pgfpoint{110.591980pt}{257.532623pt}}
\pgflineto{\pgfpoint{119.519989pt}{257.532623pt}}
\pgflineto{\pgfpoint{119.519989pt}{251.355804pt}}
\pgfpathclose
\pgfusepath{fill,stroke}
\pgfpathmoveto{\pgfpoint{110.591980pt}{263.709473pt}}
\pgflineto{\pgfpoint{119.519989pt}{257.532623pt}}
\pgflineto{\pgfpoint{110.591980pt}{257.532623pt}}
\pgfpathclose
\pgfusepath{fill,stroke}
\pgfpathmoveto{\pgfpoint{110.591980pt}{263.709473pt}}
\pgflineto{\pgfpoint{119.519989pt}{263.709473pt}}
\pgflineto{\pgfpoint{119.519989pt}{257.532623pt}}
\pgfpathclose
\pgfusepath{fill,stroke}
\color[rgb]{0.152951,0.498053,0.557685}
\pgfpathmoveto{\pgfpoint{119.519989pt}{214.294739pt}}
\pgflineto{\pgfpoint{128.447998pt}{208.117905pt}}
\pgflineto{\pgfpoint{119.519989pt}{208.117905pt}}
\pgfpathclose
\pgfusepath{fill,stroke}
\pgfpathmoveto{\pgfpoint{119.519989pt}{214.294739pt}}
\pgflineto{\pgfpoint{128.447998pt}{214.294739pt}}
\pgflineto{\pgfpoint{128.447998pt}{208.117905pt}}
\pgfpathclose
\pgfusepath{fill,stroke}
\color[rgb]{0.147132,0.512959,0.556973}
\pgfpathmoveto{\pgfpoint{119.519989pt}{220.471588pt}}
\pgflineto{\pgfpoint{128.447998pt}{214.294739pt}}
\pgflineto{\pgfpoint{119.519989pt}{214.294739pt}}
\pgfpathclose
\pgfusepath{fill,stroke}
\pgfpathmoveto{\pgfpoint{119.519989pt}{220.471588pt}}
\pgflineto{\pgfpoint{128.447998pt}{220.471588pt}}
\pgflineto{\pgfpoint{128.447998pt}{214.294739pt}}
\pgfpathclose
\pgfusepath{fill,stroke}
\pgfpathmoveto{\pgfpoint{119.519989pt}{226.648422pt}}
\pgflineto{\pgfpoint{128.447998pt}{220.471588pt}}
\pgflineto{\pgfpoint{119.519989pt}{220.471588pt}}
\pgfpathclose
\pgfusepath{fill,stroke}
\pgfpathmoveto{\pgfpoint{119.519989pt}{226.648422pt}}
\pgflineto{\pgfpoint{128.447998pt}{226.648422pt}}
\pgflineto{\pgfpoint{128.447998pt}{220.471588pt}}
\pgfpathclose
\pgfusepath{fill,stroke}
\color[rgb]{0.141402,0.527854,0.555864}
\pgfpathmoveto{\pgfpoint{119.519989pt}{232.825272pt}}
\pgflineto{\pgfpoint{128.447998pt}{226.648422pt}}
\pgflineto{\pgfpoint{119.519989pt}{226.648422pt}}
\pgfpathclose
\pgfusepath{fill,stroke}
\pgfpathmoveto{\pgfpoint{119.519989pt}{232.825272pt}}
\pgflineto{\pgfpoint{128.447998pt}{232.825272pt}}
\pgflineto{\pgfpoint{128.447998pt}{226.648422pt}}
\pgfpathclose
\pgfusepath{fill,stroke}
\color[rgb]{0.135833,0.542750,0.554289}
\pgfpathmoveto{\pgfpoint{119.519989pt}{239.002106pt}}
\pgflineto{\pgfpoint{128.447998pt}{232.825272pt}}
\pgflineto{\pgfpoint{119.519989pt}{232.825272pt}}
\pgfpathclose
\pgfusepath{fill,stroke}
\pgfpathmoveto{\pgfpoint{119.519989pt}{239.002106pt}}
\pgflineto{\pgfpoint{128.447998pt}{239.002106pt}}
\pgflineto{\pgfpoint{128.447998pt}{232.825272pt}}
\pgfpathclose
\pgfusepath{fill,stroke}
\color[rgb]{0.130582,0.557652,0.552176}
\pgfpathmoveto{\pgfpoint{119.519989pt}{245.178955pt}}
\pgflineto{\pgfpoint{128.447998pt}{239.002106pt}}
\pgflineto{\pgfpoint{119.519989pt}{239.002106pt}}
\pgfpathclose
\pgfusepath{fill,stroke}
\pgfpathmoveto{\pgfpoint{119.519989pt}{245.178955pt}}
\pgflineto{\pgfpoint{128.447998pt}{245.178955pt}}
\pgflineto{\pgfpoint{128.447998pt}{239.002106pt}}
\pgfpathclose
\pgfusepath{fill,stroke}
\color[rgb]{0.125898,0.572563,0.549445}
\pgfpathmoveto{\pgfpoint{119.519989pt}{251.355804pt}}
\pgflineto{\pgfpoint{128.447998pt}{245.178955pt}}
\pgflineto{\pgfpoint{119.519989pt}{245.178955pt}}
\pgfpathclose
\pgfusepath{fill,stroke}
\pgfpathmoveto{\pgfpoint{119.519989pt}{251.355804pt}}
\pgflineto{\pgfpoint{128.447998pt}{251.355804pt}}
\pgflineto{\pgfpoint{128.447998pt}{245.178955pt}}
\pgfpathclose
\pgfusepath{fill,stroke}
\color[rgb]{0.122163,0.587476,0.546023}
\pgfpathmoveto{\pgfpoint{119.519989pt}{257.532623pt}}
\pgflineto{\pgfpoint{128.447998pt}{251.355804pt}}
\pgflineto{\pgfpoint{119.519989pt}{251.355804pt}}
\pgfpathclose
\pgfusepath{fill,stroke}
\pgfpathmoveto{\pgfpoint{119.519989pt}{257.532623pt}}
\pgflineto{\pgfpoint{128.447998pt}{257.532623pt}}
\pgflineto{\pgfpoint{128.447998pt}{251.355804pt}}
\pgfpathclose
\pgfusepath{fill,stroke}
\pgfpathmoveto{\pgfpoint{119.519989pt}{263.709473pt}}
\pgflineto{\pgfpoint{128.447998pt}{257.532623pt}}
\pgflineto{\pgfpoint{119.519989pt}{257.532623pt}}
\pgfpathclose
\pgfusepath{fill,stroke}
\pgfpathmoveto{\pgfpoint{119.519989pt}{263.709473pt}}
\pgflineto{\pgfpoint{128.447998pt}{263.709473pt}}
\pgflineto{\pgfpoint{128.447998pt}{257.532623pt}}
\pgfpathclose
\pgfusepath{fill,stroke}
\color[rgb]{0.119872,0.602382,0.541831}
\pgfpathmoveto{\pgfpoint{119.519989pt}{269.886322pt}}
\pgflineto{\pgfpoint{128.447998pt}{263.709473pt}}
\pgflineto{\pgfpoint{119.519989pt}{263.709473pt}}
\pgfpathclose
\pgfusepath{fill,stroke}
\pgfpathmoveto{\pgfpoint{119.519989pt}{269.886322pt}}
\pgflineto{\pgfpoint{128.447998pt}{269.886322pt}}
\pgflineto{\pgfpoint{128.447998pt}{263.709473pt}}
\pgfpathclose
\pgfusepath{fill,stroke}
\color[rgb]{0.119627,0.617266,0.536796}
\pgfpathmoveto{\pgfpoint{119.519989pt}{276.063141pt}}
\pgflineto{\pgfpoint{128.447998pt}{269.886322pt}}
\pgflineto{\pgfpoint{119.519989pt}{269.886322pt}}
\pgfpathclose
\pgfusepath{fill,stroke}
\pgfpathmoveto{\pgfpoint{119.519989pt}{276.063141pt}}
\pgflineto{\pgfpoint{128.447998pt}{276.063141pt}}
\pgflineto{\pgfpoint{128.447998pt}{269.886322pt}}
\pgfpathclose
\pgfusepath{fill,stroke}
\color[rgb]{0.135833,0.542750,0.554289}
\pgfpathmoveto{\pgfpoint{128.447998pt}{226.648422pt}}
\pgflineto{\pgfpoint{137.376007pt}{220.471588pt}}
\pgflineto{\pgfpoint{128.447998pt}{220.471588pt}}
\pgfpathclose
\pgfusepath{fill,stroke}
\pgfpathmoveto{\pgfpoint{128.447998pt}{226.648422pt}}
\pgflineto{\pgfpoint{137.376007pt}{226.648422pt}}
\pgflineto{\pgfpoint{137.376007pt}{220.471588pt}}
\pgfpathclose
\pgfusepath{fill,stroke}
\pgfpathmoveto{\pgfpoint{128.447998pt}{232.825272pt}}
\pgflineto{\pgfpoint{137.376007pt}{226.648422pt}}
\pgflineto{\pgfpoint{128.447998pt}{226.648422pt}}
\pgfpathclose
\pgfusepath{fill,stroke}
\pgfpathmoveto{\pgfpoint{128.447998pt}{232.825272pt}}
\pgflineto{\pgfpoint{137.376007pt}{232.825272pt}}
\pgflineto{\pgfpoint{137.376007pt}{226.648422pt}}
\pgfpathclose
\pgfusepath{fill,stroke}
\color[rgb]{0.130582,0.557652,0.552176}
\pgfpathmoveto{\pgfpoint{128.447998pt}{239.002106pt}}
\pgflineto{\pgfpoint{137.376007pt}{232.825272pt}}
\pgflineto{\pgfpoint{128.447998pt}{232.825272pt}}
\pgfpathclose
\pgfusepath{fill,stroke}
\pgfpathmoveto{\pgfpoint{128.447998pt}{239.002106pt}}
\pgflineto{\pgfpoint{137.376007pt}{239.002106pt}}
\pgflineto{\pgfpoint{137.376007pt}{232.825272pt}}
\pgfpathclose
\pgfusepath{fill,stroke}
\color[rgb]{0.125898,0.572563,0.549445}
\pgfpathmoveto{\pgfpoint{128.447998pt}{245.178955pt}}
\pgflineto{\pgfpoint{137.376007pt}{239.002106pt}}
\pgflineto{\pgfpoint{128.447998pt}{239.002106pt}}
\pgfpathclose
\pgfusepath{fill,stroke}
\pgfpathmoveto{\pgfpoint{128.447998pt}{245.178955pt}}
\pgflineto{\pgfpoint{137.376007pt}{245.178955pt}}
\pgflineto{\pgfpoint{137.376007pt}{239.002106pt}}
\pgfpathclose
\pgfusepath{fill,stroke}
\color[rgb]{0.122163,0.587476,0.546023}
\pgfpathmoveto{\pgfpoint{128.447998pt}{251.355804pt}}
\pgflineto{\pgfpoint{137.376007pt}{245.178955pt}}
\pgflineto{\pgfpoint{128.447998pt}{245.178955pt}}
\pgfpathclose
\pgfusepath{fill,stroke}
\pgfpathmoveto{\pgfpoint{128.447998pt}{251.355804pt}}
\pgflineto{\pgfpoint{137.376007pt}{251.355804pt}}
\pgflineto{\pgfpoint{137.376007pt}{245.178955pt}}
\pgfpathclose
\pgfusepath{fill,stroke}
\color[rgb]{0.119872,0.602382,0.541831}
\pgfpathmoveto{\pgfpoint{128.447998pt}{257.532623pt}}
\pgflineto{\pgfpoint{137.376007pt}{251.355804pt}}
\pgflineto{\pgfpoint{128.447998pt}{251.355804pt}}
\pgfpathclose
\pgfusepath{fill,stroke}
\pgfpathmoveto{\pgfpoint{128.447998pt}{257.532623pt}}
\pgflineto{\pgfpoint{137.376007pt}{257.532623pt}}
\pgflineto{\pgfpoint{137.376007pt}{251.355804pt}}
\pgfpathclose
\pgfusepath{fill,stroke}
\color[rgb]{0.119627,0.617266,0.536796}
\pgfpathmoveto{\pgfpoint{128.447998pt}{263.709473pt}}
\pgflineto{\pgfpoint{137.376007pt}{257.532623pt}}
\pgflineto{\pgfpoint{128.447998pt}{257.532623pt}}
\pgfpathclose
\pgfusepath{fill,stroke}
\pgfpathmoveto{\pgfpoint{128.447998pt}{263.709473pt}}
\pgflineto{\pgfpoint{137.376007pt}{263.709473pt}}
\pgflineto{\pgfpoint{137.376007pt}{257.532623pt}}
\pgfpathclose
\pgfusepath{fill,stroke}
\pgfpathmoveto{\pgfpoint{128.447998pt}{269.886322pt}}
\pgflineto{\pgfpoint{137.376007pt}{263.709473pt}}
\pgflineto{\pgfpoint{128.447998pt}{263.709473pt}}
\pgfpathclose
\pgfusepath{fill,stroke}
\pgfpathmoveto{\pgfpoint{128.447998pt}{269.886322pt}}
\pgflineto{\pgfpoint{137.376007pt}{269.886322pt}}
\pgflineto{\pgfpoint{137.376007pt}{263.709473pt}}
\pgfpathclose
\pgfusepath{fill,stroke}
\color[rgb]{0.122046,0.632107,0.530848}
\pgfpathmoveto{\pgfpoint{128.447998pt}{276.063141pt}}
\pgflineto{\pgfpoint{137.376007pt}{269.886322pt}}
\pgflineto{\pgfpoint{128.447998pt}{269.886322pt}}
\pgfpathclose
\pgfusepath{fill,stroke}
\pgfpathmoveto{\pgfpoint{128.447998pt}{276.063141pt}}
\pgflineto{\pgfpoint{137.376007pt}{276.063141pt}}
\pgflineto{\pgfpoint{137.376007pt}{269.886322pt}}
\pgfpathclose
\pgfusepath{fill,stroke}
\color[rgb]{0.127668,0.646882,0.523924}
\pgfpathmoveto{\pgfpoint{128.447998pt}{282.239990pt}}
\pgflineto{\pgfpoint{137.376007pt}{276.063141pt}}
\pgflineto{\pgfpoint{128.447998pt}{276.063141pt}}
\pgfpathclose
\pgfusepath{fill,stroke}
\pgfpathmoveto{\pgfpoint{128.447998pt}{282.239990pt}}
\pgflineto{\pgfpoint{137.376007pt}{282.239990pt}}
\pgflineto{\pgfpoint{137.376007pt}{276.063141pt}}
\pgfpathclose
\pgfusepath{fill,stroke}
\color[rgb]{0.136835,0.661563,0.515967}
\pgfpathmoveto{\pgfpoint{128.447998pt}{288.416840pt}}
\pgflineto{\pgfpoint{137.376007pt}{282.239990pt}}
\pgflineto{\pgfpoint{128.447998pt}{282.239990pt}}
\pgfpathclose
\pgfusepath{fill,stroke}
\pgfpathmoveto{\pgfpoint{128.447998pt}{288.416840pt}}
\pgflineto{\pgfpoint{137.376007pt}{288.416840pt}}
\pgflineto{\pgfpoint{137.376007pt}{282.239990pt}}
\pgfpathclose
\pgfusepath{fill,stroke}
\color[rgb]{0.149643,0.676120,0.506924}
\pgfpathmoveto{\pgfpoint{128.447998pt}{294.593689pt}}
\pgflineto{\pgfpoint{137.376007pt}{288.416840pt}}
\pgflineto{\pgfpoint{128.447998pt}{288.416840pt}}
\pgfpathclose
\pgfusepath{fill,stroke}
\pgfpathmoveto{\pgfpoint{128.447998pt}{294.593689pt}}
\pgflineto{\pgfpoint{137.376007pt}{294.593689pt}}
\pgflineto{\pgfpoint{137.376007pt}{288.416840pt}}
\pgfpathclose
\pgfusepath{fill,stroke}
\color[rgb]{0.125898,0.572563,0.549445}
\pgfpathmoveto{\pgfpoint{137.376007pt}{239.002106pt}}
\pgflineto{\pgfpoint{146.303986pt}{232.825272pt}}
\pgflineto{\pgfpoint{137.376007pt}{232.825272pt}}
\pgfpathclose
\pgfusepath{fill,stroke}
\pgfpathmoveto{\pgfpoint{137.376007pt}{239.002106pt}}
\pgflineto{\pgfpoint{146.303986pt}{239.002106pt}}
\pgflineto{\pgfpoint{146.303986pt}{232.825272pt}}
\pgfpathclose
\pgfusepath{fill,stroke}
\color[rgb]{0.122163,0.587476,0.546023}
\pgfpathmoveto{\pgfpoint{137.376007pt}{245.178955pt}}
\pgflineto{\pgfpoint{146.303986pt}{239.002106pt}}
\pgflineto{\pgfpoint{137.376007pt}{239.002106pt}}
\pgfpathclose
\pgfusepath{fill,stroke}
\pgfpathmoveto{\pgfpoint{137.376007pt}{245.178955pt}}
\pgflineto{\pgfpoint{146.303986pt}{245.178955pt}}
\pgflineto{\pgfpoint{146.303986pt}{239.002106pt}}
\pgfpathclose
\pgfusepath{fill,stroke}
\color[rgb]{0.119872,0.602382,0.541831}
\pgfpathmoveto{\pgfpoint{137.376007pt}{251.355804pt}}
\pgflineto{\pgfpoint{146.303986pt}{245.178955pt}}
\pgflineto{\pgfpoint{137.376007pt}{245.178955pt}}
\pgfpathclose
\pgfusepath{fill,stroke}
\pgfpathmoveto{\pgfpoint{137.376007pt}{251.355804pt}}
\pgflineto{\pgfpoint{146.303986pt}{251.355804pt}}
\pgflineto{\pgfpoint{146.303986pt}{245.178955pt}}
\pgfpathclose
\pgfusepath{fill,stroke}
\color[rgb]{0.119627,0.617266,0.536796}
\pgfpathmoveto{\pgfpoint{137.376007pt}{257.532623pt}}
\pgflineto{\pgfpoint{146.303986pt}{251.355804pt}}
\pgflineto{\pgfpoint{137.376007pt}{251.355804pt}}
\pgfpathclose
\pgfusepath{fill,stroke}
\pgfpathmoveto{\pgfpoint{137.376007pt}{257.532623pt}}
\pgflineto{\pgfpoint{146.303986pt}{257.532623pt}}
\pgflineto{\pgfpoint{146.303986pt}{251.355804pt}}
\pgfpathclose
\pgfusepath{fill,stroke}
\color[rgb]{0.122046,0.632107,0.530848}
\pgfpathmoveto{\pgfpoint{137.376007pt}{263.709473pt}}
\pgflineto{\pgfpoint{146.303986pt}{257.532623pt}}
\pgflineto{\pgfpoint{137.376007pt}{257.532623pt}}
\pgfpathclose
\pgfusepath{fill,stroke}
\pgfpathmoveto{\pgfpoint{137.376007pt}{263.709473pt}}
\pgflineto{\pgfpoint{146.303986pt}{263.709473pt}}
\pgflineto{\pgfpoint{146.303986pt}{257.532623pt}}
\pgfpathclose
\pgfusepath{fill,stroke}
\pgfpathmoveto{\pgfpoint{137.376007pt}{269.886322pt}}
\pgflineto{\pgfpoint{146.303986pt}{263.709473pt}}
\pgflineto{\pgfpoint{137.376007pt}{263.709473pt}}
\pgfpathclose
\pgfusepath{fill,stroke}
\pgfpathmoveto{\pgfpoint{137.376007pt}{269.886322pt}}
\pgflineto{\pgfpoint{146.303986pt}{269.886322pt}}
\pgflineto{\pgfpoint{146.303986pt}{263.709473pt}}
\pgfpathclose
\pgfusepath{fill,stroke}
\color[rgb]{0.127668,0.646882,0.523924}
\pgfpathmoveto{\pgfpoint{137.376007pt}{276.063141pt}}
\pgflineto{\pgfpoint{146.303986pt}{269.886322pt}}
\pgflineto{\pgfpoint{137.376007pt}{269.886322pt}}
\pgfpathclose
\pgfusepath{fill,stroke}
\pgfpathmoveto{\pgfpoint{137.376007pt}{276.063141pt}}
\pgflineto{\pgfpoint{146.303986pt}{276.063141pt}}
\pgflineto{\pgfpoint{146.303986pt}{269.886322pt}}
\pgfpathclose
\pgfusepath{fill,stroke}
\color[rgb]{0.136835,0.661563,0.515967}
\pgfpathmoveto{\pgfpoint{137.376007pt}{282.239990pt}}
\pgflineto{\pgfpoint{146.303986pt}{276.063141pt}}
\pgflineto{\pgfpoint{137.376007pt}{276.063141pt}}
\pgfpathclose
\pgfusepath{fill,stroke}
\pgfpathmoveto{\pgfpoint{137.376007pt}{282.239990pt}}
\pgflineto{\pgfpoint{146.303986pt}{282.239990pt}}
\pgflineto{\pgfpoint{146.303986pt}{276.063141pt}}
\pgfpathclose
\pgfusepath{fill,stroke}
\color[rgb]{0.149643,0.676120,0.506924}
\pgfpathmoveto{\pgfpoint{137.376007pt}{288.416840pt}}
\pgflineto{\pgfpoint{146.303986pt}{282.239990pt}}
\pgflineto{\pgfpoint{137.376007pt}{282.239990pt}}
\pgfpathclose
\pgfusepath{fill,stroke}
\pgfpathmoveto{\pgfpoint{137.376007pt}{288.416840pt}}
\pgflineto{\pgfpoint{146.303986pt}{288.416840pt}}
\pgflineto{\pgfpoint{146.303986pt}{282.239990pt}}
\pgfpathclose
\pgfusepath{fill,stroke}
\color[rgb]{0.165967,0.690519,0.496752}
\pgfpathmoveto{\pgfpoint{137.376007pt}{294.593689pt}}
\pgflineto{\pgfpoint{146.303986pt}{288.416840pt}}
\pgflineto{\pgfpoint{137.376007pt}{288.416840pt}}
\pgfpathclose
\pgfusepath{fill,stroke}
\pgfpathmoveto{\pgfpoint{137.376007pt}{294.593689pt}}
\pgflineto{\pgfpoint{146.303986pt}{294.593689pt}}
\pgflineto{\pgfpoint{146.303986pt}{288.416840pt}}
\pgfpathclose
\pgfusepath{fill,stroke}
\color[rgb]{0.185538,0.704725,0.485412}
\pgfpathmoveto{\pgfpoint{137.376007pt}{300.770538pt}}
\pgflineto{\pgfpoint{146.303986pt}{294.593689pt}}
\pgflineto{\pgfpoint{137.376007pt}{294.593689pt}}
\pgfpathclose
\pgfusepath{fill,stroke}
\pgfpathmoveto{\pgfpoint{137.376007pt}{300.770538pt}}
\pgflineto{\pgfpoint{146.303986pt}{300.770538pt}}
\pgflineto{\pgfpoint{146.303986pt}{294.593689pt}}
\pgfpathclose
\pgfusepath{fill,stroke}
\color[rgb]{0.208030,0.718701,0.472873}
\pgfpathmoveto{\pgfpoint{137.376007pt}{306.947388pt}}
\pgflineto{\pgfpoint{146.303986pt}{300.770538pt}}
\pgflineto{\pgfpoint{137.376007pt}{300.770538pt}}
\pgfpathclose
\pgfusepath{fill,stroke}
\pgfpathmoveto{\pgfpoint{137.376007pt}{306.947388pt}}
\pgflineto{\pgfpoint{146.303986pt}{306.947388pt}}
\pgflineto{\pgfpoint{146.303986pt}{300.770538pt}}
\pgfpathclose
\pgfusepath{fill,stroke}
\color[rgb]{0.122046,0.632107,0.530848}
\pgfpathmoveto{\pgfpoint{146.303986pt}{257.532623pt}}
\pgflineto{\pgfpoint{155.231979pt}{251.355804pt}}
\pgflineto{\pgfpoint{146.303986pt}{251.355804pt}}
\pgfpathclose
\pgfusepath{fill,stroke}
\pgfpathmoveto{\pgfpoint{146.303986pt}{257.532623pt}}
\pgflineto{\pgfpoint{155.231979pt}{257.532623pt}}
\pgflineto{\pgfpoint{155.231979pt}{251.355804pt}}
\pgfpathclose
\pgfusepath{fill,stroke}
\color[rgb]{0.127668,0.646882,0.523924}
\pgfpathmoveto{\pgfpoint{146.303986pt}{263.709473pt}}
\pgflineto{\pgfpoint{155.231979pt}{257.532623pt}}
\pgflineto{\pgfpoint{146.303986pt}{257.532623pt}}
\pgfpathclose
\pgfusepath{fill,stroke}
\pgfpathmoveto{\pgfpoint{146.303986pt}{263.709473pt}}
\pgflineto{\pgfpoint{155.231979pt}{263.709473pt}}
\pgflineto{\pgfpoint{155.231979pt}{257.532623pt}}
\pgfpathclose
\pgfusepath{fill,stroke}
\color[rgb]{0.136835,0.661563,0.515967}
\pgfpathmoveto{\pgfpoint{146.303986pt}{269.886322pt}}
\pgflineto{\pgfpoint{155.231979pt}{263.709473pt}}
\pgflineto{\pgfpoint{146.303986pt}{263.709473pt}}
\pgfpathclose
\pgfusepath{fill,stroke}
\pgfpathmoveto{\pgfpoint{146.303986pt}{269.886322pt}}
\pgflineto{\pgfpoint{155.231979pt}{269.886322pt}}
\pgflineto{\pgfpoint{155.231979pt}{263.709473pt}}
\pgfpathclose
\pgfusepath{fill,stroke}
\pgfpathmoveto{\pgfpoint{146.303986pt}{276.063141pt}}
\pgflineto{\pgfpoint{155.231979pt}{269.886322pt}}
\pgflineto{\pgfpoint{146.303986pt}{269.886322pt}}
\pgfpathclose
\pgfusepath{fill,stroke}
\pgfpathmoveto{\pgfpoint{146.303986pt}{276.063141pt}}
\pgflineto{\pgfpoint{155.231979pt}{276.063141pt}}
\pgflineto{\pgfpoint{155.231979pt}{269.886322pt}}
\pgfpathclose
\pgfusepath{fill,stroke}
\color[rgb]{0.149643,0.676120,0.506924}
\pgfpathmoveto{\pgfpoint{146.303986pt}{282.239990pt}}
\pgflineto{\pgfpoint{155.231979pt}{276.063141pt}}
\pgflineto{\pgfpoint{146.303986pt}{276.063141pt}}
\pgfpathclose
\pgfusepath{fill,stroke}
\pgfpathmoveto{\pgfpoint{146.303986pt}{282.239990pt}}
\pgflineto{\pgfpoint{155.231979pt}{282.239990pt}}
\pgflineto{\pgfpoint{155.231979pt}{276.063141pt}}
\pgfpathclose
\pgfusepath{fill,stroke}
\color[rgb]{0.165967,0.690519,0.496752}
\pgfpathmoveto{\pgfpoint{146.303986pt}{288.416840pt}}
\pgflineto{\pgfpoint{155.231979pt}{282.239990pt}}
\pgflineto{\pgfpoint{146.303986pt}{282.239990pt}}
\pgfpathclose
\pgfusepath{fill,stroke}
\pgfpathmoveto{\pgfpoint{146.303986pt}{288.416840pt}}
\pgflineto{\pgfpoint{155.231979pt}{288.416840pt}}
\pgflineto{\pgfpoint{155.231979pt}{282.239990pt}}
\pgfpathclose
\pgfusepath{fill,stroke}
\color[rgb]{0.185538,0.704725,0.485412}
\pgfpathmoveto{\pgfpoint{146.303986pt}{294.593689pt}}
\pgflineto{\pgfpoint{155.231979pt}{288.416840pt}}
\pgflineto{\pgfpoint{146.303986pt}{288.416840pt}}
\pgfpathclose
\pgfusepath{fill,stroke}
\pgfpathmoveto{\pgfpoint{146.303986pt}{294.593689pt}}
\pgflineto{\pgfpoint{155.231979pt}{294.593689pt}}
\pgflineto{\pgfpoint{155.231979pt}{288.416840pt}}
\pgfpathclose
\pgfusepath{fill,stroke}
\color[rgb]{0.208030,0.718701,0.472873}
\pgfpathmoveto{\pgfpoint{146.303986pt}{300.770538pt}}
\pgflineto{\pgfpoint{155.231979pt}{294.593689pt}}
\pgflineto{\pgfpoint{146.303986pt}{294.593689pt}}
\pgfpathclose
\pgfusepath{fill,stroke}
\pgfpathmoveto{\pgfpoint{146.303986pt}{300.770538pt}}
\pgflineto{\pgfpoint{155.231979pt}{300.770538pt}}
\pgflineto{\pgfpoint{155.231979pt}{294.593689pt}}
\pgfpathclose
\pgfusepath{fill,stroke}
\color[rgb]{0.233127,0.732406,0.459106}
\pgfpathmoveto{\pgfpoint{146.303986pt}{306.947388pt}}
\pgflineto{\pgfpoint{155.231979pt}{300.770538pt}}
\pgflineto{\pgfpoint{146.303986pt}{300.770538pt}}
\pgfpathclose
\pgfusepath{fill,stroke}
\pgfpathmoveto{\pgfpoint{146.303986pt}{306.947388pt}}
\pgflineto{\pgfpoint{155.231979pt}{306.947388pt}}
\pgflineto{\pgfpoint{155.231979pt}{300.770538pt}}
\pgfpathclose
\pgfusepath{fill,stroke}
\pgfpathmoveto{\pgfpoint{146.303986pt}{313.124207pt}}
\pgflineto{\pgfpoint{155.231979pt}{306.947388pt}}
\pgflineto{\pgfpoint{146.303986pt}{306.947388pt}}
\pgfpathclose
\pgfusepath{fill,stroke}
\pgfpathmoveto{\pgfpoint{146.303986pt}{313.124207pt}}
\pgflineto{\pgfpoint{155.231979pt}{313.124207pt}}
\pgflineto{\pgfpoint{155.231979pt}{306.947388pt}}
\pgfpathclose
\pgfusepath{fill,stroke}
\color[rgb]{0.149643,0.676120,0.506924}
\pgfpathmoveto{\pgfpoint{155.231979pt}{269.886322pt}}
\pgflineto{\pgfpoint{164.160004pt}{263.709473pt}}
\pgflineto{\pgfpoint{155.231979pt}{263.709473pt}}
\pgfpathclose
\pgfusepath{fill,stroke}
\pgfpathmoveto{\pgfpoint{155.231979pt}{269.886322pt}}
\pgflineto{\pgfpoint{164.160004pt}{269.886322pt}}
\pgflineto{\pgfpoint{164.160004pt}{263.709473pt}}
\pgfpathclose
\pgfusepath{fill,stroke}
\pgfpathmoveto{\pgfpoint{155.231979pt}{276.063141pt}}
\pgflineto{\pgfpoint{164.160004pt}{269.886322pt}}
\pgflineto{\pgfpoint{155.231979pt}{269.886322pt}}
\pgfpathclose
\pgfusepath{fill,stroke}
\pgfpathmoveto{\pgfpoint{155.231979pt}{276.063141pt}}
\pgflineto{\pgfpoint{164.160004pt}{276.063141pt}}
\pgflineto{\pgfpoint{164.160004pt}{269.886322pt}}
\pgfpathclose
\pgfusepath{fill,stroke}
\color[rgb]{0.165967,0.690519,0.496752}
\pgfpathmoveto{\pgfpoint{155.231979pt}{282.239990pt}}
\pgflineto{\pgfpoint{164.160004pt}{276.063141pt}}
\pgflineto{\pgfpoint{155.231979pt}{276.063141pt}}
\pgfpathclose
\pgfusepath{fill,stroke}
\pgfpathmoveto{\pgfpoint{155.231979pt}{282.239990pt}}
\pgflineto{\pgfpoint{164.160004pt}{282.239990pt}}
\pgflineto{\pgfpoint{164.160004pt}{276.063141pt}}
\pgfpathclose
\pgfusepath{fill,stroke}
\color[rgb]{0.185538,0.704725,0.485412}
\pgfpathmoveto{\pgfpoint{155.231979pt}{288.416840pt}}
\pgflineto{\pgfpoint{164.160004pt}{282.239990pt}}
\pgflineto{\pgfpoint{155.231979pt}{282.239990pt}}
\pgfpathclose
\pgfusepath{fill,stroke}
\pgfpathmoveto{\pgfpoint{155.231979pt}{288.416840pt}}
\pgflineto{\pgfpoint{164.160004pt}{288.416840pt}}
\pgflineto{\pgfpoint{164.160004pt}{282.239990pt}}
\pgfpathclose
\pgfusepath{fill,stroke}
\color[rgb]{0.208030,0.718701,0.472873}
\pgfpathmoveto{\pgfpoint{155.231979pt}{294.593689pt}}
\pgflineto{\pgfpoint{164.160004pt}{288.416840pt}}
\pgflineto{\pgfpoint{155.231979pt}{288.416840pt}}
\pgfpathclose
\pgfusepath{fill,stroke}
\pgfpathmoveto{\pgfpoint{155.231979pt}{294.593689pt}}
\pgflineto{\pgfpoint{164.160004pt}{294.593689pt}}
\pgflineto{\pgfpoint{164.160004pt}{288.416840pt}}
\pgfpathclose
\pgfusepath{fill,stroke}
\color[rgb]{0.233127,0.732406,0.459106}
\pgfpathmoveto{\pgfpoint{155.231979pt}{300.770538pt}}
\pgflineto{\pgfpoint{164.160004pt}{294.593689pt}}
\pgflineto{\pgfpoint{155.231979pt}{294.593689pt}}
\pgfpathclose
\pgfusepath{fill,stroke}
\pgfpathmoveto{\pgfpoint{155.231979pt}{300.770538pt}}
\pgflineto{\pgfpoint{164.160004pt}{300.770538pt}}
\pgflineto{\pgfpoint{164.160004pt}{294.593689pt}}
\pgfpathclose
\pgfusepath{fill,stroke}
\color[rgb]{0.260531,0.745802,0.444096}
\pgfpathmoveto{\pgfpoint{155.231979pt}{306.947388pt}}
\pgflineto{\pgfpoint{164.160004pt}{300.770538pt}}
\pgflineto{\pgfpoint{155.231979pt}{300.770538pt}}
\pgfpathclose
\pgfusepath{fill,stroke}
\pgfpathmoveto{\pgfpoint{155.231979pt}{306.947388pt}}
\pgflineto{\pgfpoint{164.160004pt}{306.947388pt}}
\pgflineto{\pgfpoint{164.160004pt}{300.770538pt}}
\pgfpathclose
\pgfusepath{fill,stroke}
\color[rgb]{0.290001,0.758846,0.427826}
\pgfpathmoveto{\pgfpoint{155.231979pt}{313.124207pt}}
\pgflineto{\pgfpoint{164.160004pt}{306.947388pt}}
\pgflineto{\pgfpoint{155.231979pt}{306.947388pt}}
\pgfpathclose
\pgfusepath{fill,stroke}
\pgfpathmoveto{\pgfpoint{155.231979pt}{313.124207pt}}
\pgflineto{\pgfpoint{164.160004pt}{313.124207pt}}
\pgflineto{\pgfpoint{164.160004pt}{306.947388pt}}
\pgfpathclose
\pgfusepath{fill,stroke}
\pgfpathmoveto{\pgfpoint{155.231979pt}{319.301056pt}}
\pgflineto{\pgfpoint{164.160004pt}{313.124207pt}}
\pgflineto{\pgfpoint{155.231979pt}{313.124207pt}}
\pgfpathclose
\pgfusepath{fill,stroke}
\pgfpathmoveto{\pgfpoint{155.231979pt}{319.301056pt}}
\pgflineto{\pgfpoint{164.160004pt}{319.301056pt}}
\pgflineto{\pgfpoint{164.160004pt}{313.124207pt}}
\pgfpathclose
\pgfusepath{fill,stroke}
\color[rgb]{0.321330,0.771498,0.410293}
\pgfpathmoveto{\pgfpoint{155.231979pt}{325.477905pt}}
\pgflineto{\pgfpoint{164.160004pt}{319.301056pt}}
\pgflineto{\pgfpoint{155.231979pt}{319.301056pt}}
\pgfpathclose
\pgfusepath{fill,stroke}
\pgfpathmoveto{\pgfpoint{155.231979pt}{325.477905pt}}
\pgflineto{\pgfpoint{164.160004pt}{325.477905pt}}
\pgflineto{\pgfpoint{164.160004pt}{319.301056pt}}
\pgfpathclose
\pgfusepath{fill,stroke}
\color[rgb]{0.185538,0.704725,0.485412}
\pgfpathmoveto{\pgfpoint{164.160004pt}{282.239990pt}}
\pgflineto{\pgfpoint{173.087997pt}{276.063141pt}}
\pgflineto{\pgfpoint{164.160004pt}{276.063141pt}}
\pgfpathclose
\pgfusepath{fill,stroke}
\pgfpathmoveto{\pgfpoint{164.160004pt}{282.239990pt}}
\pgflineto{\pgfpoint{173.087997pt}{282.239990pt}}
\pgflineto{\pgfpoint{173.087997pt}{276.063141pt}}
\pgfpathclose
\pgfusepath{fill,stroke}
\color[rgb]{0.208030,0.718701,0.472873}
\pgfpathmoveto{\pgfpoint{164.160004pt}{288.416840pt}}
\pgflineto{\pgfpoint{173.087997pt}{282.239990pt}}
\pgflineto{\pgfpoint{164.160004pt}{282.239990pt}}
\pgfpathclose
\pgfusepath{fill,stroke}
\pgfpathmoveto{\pgfpoint{164.160004pt}{288.416840pt}}
\pgflineto{\pgfpoint{173.087997pt}{288.416840pt}}
\pgflineto{\pgfpoint{173.087997pt}{282.239990pt}}
\pgfpathclose
\pgfusepath{fill,stroke}
\color[rgb]{0.233127,0.732406,0.459106}
\pgfpathmoveto{\pgfpoint{164.160004pt}{294.593689pt}}
\pgflineto{\pgfpoint{173.087997pt}{288.416840pt}}
\pgflineto{\pgfpoint{164.160004pt}{288.416840pt}}
\pgfpathclose
\pgfusepath{fill,stroke}
\pgfpathmoveto{\pgfpoint{164.160004pt}{294.593689pt}}
\pgflineto{\pgfpoint{173.087997pt}{294.593689pt}}
\pgflineto{\pgfpoint{173.087997pt}{288.416840pt}}
\pgfpathclose
\pgfusepath{fill,stroke}
\color[rgb]{0.260531,0.745802,0.444096}
\pgfpathmoveto{\pgfpoint{164.160004pt}{300.770538pt}}
\pgflineto{\pgfpoint{173.087997pt}{294.593689pt}}
\pgflineto{\pgfpoint{164.160004pt}{294.593689pt}}
\pgfpathclose
\pgfusepath{fill,stroke}
\pgfpathmoveto{\pgfpoint{164.160004pt}{300.770538pt}}
\pgflineto{\pgfpoint{173.087997pt}{300.770538pt}}
\pgflineto{\pgfpoint{173.087997pt}{294.593689pt}}
\pgfpathclose
\pgfusepath{fill,stroke}
\color[rgb]{0.290001,0.758846,0.427826}
\pgfpathmoveto{\pgfpoint{164.160004pt}{306.947388pt}}
\pgflineto{\pgfpoint{173.087997pt}{300.770538pt}}
\pgflineto{\pgfpoint{164.160004pt}{300.770538pt}}
\pgfpathclose
\pgfusepath{fill,stroke}
\pgfpathmoveto{\pgfpoint{164.160004pt}{306.947388pt}}
\pgflineto{\pgfpoint{173.087997pt}{306.947388pt}}
\pgflineto{\pgfpoint{173.087997pt}{300.770538pt}}
\pgfpathclose
\pgfusepath{fill,stroke}
\color[rgb]{0.321330,0.771498,0.410293}
\pgfpathmoveto{\pgfpoint{164.160004pt}{313.124207pt}}
\pgflineto{\pgfpoint{173.087997pt}{306.947388pt}}
\pgflineto{\pgfpoint{164.160004pt}{306.947388pt}}
\pgfpathclose
\pgfusepath{fill,stroke}
\pgfpathmoveto{\pgfpoint{164.160004pt}{313.124207pt}}
\pgflineto{\pgfpoint{173.087997pt}{313.124207pt}}
\pgflineto{\pgfpoint{173.087997pt}{306.947388pt}}
\pgfpathclose
\pgfusepath{fill,stroke}
\pgfpathmoveto{\pgfpoint{164.160004pt}{319.301056pt}}
\pgflineto{\pgfpoint{173.087997pt}{313.124207pt}}
\pgflineto{\pgfpoint{164.160004pt}{313.124207pt}}
\pgfpathclose
\pgfusepath{fill,stroke}
\pgfpathmoveto{\pgfpoint{164.160004pt}{319.301056pt}}
\pgflineto{\pgfpoint{173.087997pt}{319.301056pt}}
\pgflineto{\pgfpoint{173.087997pt}{313.124207pt}}
\pgfpathclose
\pgfusepath{fill,stroke}
\color[rgb]{0.354355,0.783714,0.391488}
\pgfpathmoveto{\pgfpoint{164.160004pt}{325.477905pt}}
\pgflineto{\pgfpoint{173.087997pt}{319.301056pt}}
\pgflineto{\pgfpoint{164.160004pt}{319.301056pt}}
\pgfpathclose
\pgfusepath{fill,stroke}
\pgfpathmoveto{\pgfpoint{164.160004pt}{325.477905pt}}
\pgflineto{\pgfpoint{173.087997pt}{325.477905pt}}
\pgflineto{\pgfpoint{173.087997pt}{319.301056pt}}
\pgfpathclose
\pgfusepath{fill,stroke}
\color[rgb]{0.388930,0.795453,0.371421}
\pgfpathmoveto{\pgfpoint{164.160004pt}{331.654724pt}}
\pgflineto{\pgfpoint{173.087997pt}{325.477905pt}}
\pgflineto{\pgfpoint{164.160004pt}{325.477905pt}}
\pgfpathclose
\pgfusepath{fill,stroke}
\pgfpathmoveto{\pgfpoint{164.160004pt}{331.654724pt}}
\pgflineto{\pgfpoint{173.087997pt}{331.654724pt}}
\pgflineto{\pgfpoint{173.087997pt}{325.477905pt}}
\pgfpathclose
\pgfusepath{fill,stroke}
\color[rgb]{0.424933,0.806674,0.350099}
\pgfpathmoveto{\pgfpoint{164.160004pt}{337.831604pt}}
\pgflineto{\pgfpoint{173.087997pt}{331.654724pt}}
\pgflineto{\pgfpoint{164.160004pt}{331.654724pt}}
\pgfpathclose
\pgfusepath{fill,stroke}
\pgfpathmoveto{\pgfpoint{164.160004pt}{337.831604pt}}
\pgflineto{\pgfpoint{173.087997pt}{337.831604pt}}
\pgflineto{\pgfpoint{173.087997pt}{331.654724pt}}
\pgfpathclose
\pgfusepath{fill,stroke}
\color[rgb]{0.260531,0.745802,0.444096}
\pgfpathmoveto{\pgfpoint{173.087997pt}{294.593689pt}}
\pgflineto{\pgfpoint{182.015991pt}{288.416840pt}}
\pgflineto{\pgfpoint{173.087997pt}{288.416840pt}}
\pgfpathclose
\pgfusepath{fill,stroke}
\pgfpathmoveto{\pgfpoint{173.087997pt}{294.593689pt}}
\pgflineto{\pgfpoint{182.015991pt}{294.593689pt}}
\pgflineto{\pgfpoint{182.015991pt}{288.416840pt}}
\pgfpathclose
\pgfusepath{fill,stroke}
\color[rgb]{0.290001,0.758846,0.427826}
\pgfpathmoveto{\pgfpoint{173.087997pt}{300.770538pt}}
\pgflineto{\pgfpoint{182.015991pt}{294.593689pt}}
\pgflineto{\pgfpoint{173.087997pt}{294.593689pt}}
\pgfpathclose
\pgfusepath{fill,stroke}
\pgfpathmoveto{\pgfpoint{173.087997pt}{300.770538pt}}
\pgflineto{\pgfpoint{182.015991pt}{300.770538pt}}
\pgflineto{\pgfpoint{182.015991pt}{294.593689pt}}
\pgfpathclose
\pgfusepath{fill,stroke}
\color[rgb]{0.321330,0.771498,0.410293}
\pgfpathmoveto{\pgfpoint{173.087997pt}{306.947388pt}}
\pgflineto{\pgfpoint{182.015991pt}{300.770538pt}}
\pgflineto{\pgfpoint{173.087997pt}{300.770538pt}}
\pgfpathclose
\pgfusepath{fill,stroke}
\pgfpathmoveto{\pgfpoint{173.087997pt}{306.947388pt}}
\pgflineto{\pgfpoint{182.015991pt}{306.947388pt}}
\pgflineto{\pgfpoint{182.015991pt}{300.770538pt}}
\pgfpathclose
\pgfusepath{fill,stroke}
\color[rgb]{0.354355,0.783714,0.391488}
\pgfpathmoveto{\pgfpoint{173.087997pt}{313.124207pt}}
\pgflineto{\pgfpoint{182.015991pt}{306.947388pt}}
\pgflineto{\pgfpoint{173.087997pt}{306.947388pt}}
\pgfpathclose
\pgfusepath{fill,stroke}
\pgfpathmoveto{\pgfpoint{173.087997pt}{313.124207pt}}
\pgflineto{\pgfpoint{182.015991pt}{313.124207pt}}
\pgflineto{\pgfpoint{182.015991pt}{306.947388pt}}
\pgfpathclose
\pgfusepath{fill,stroke}
\color[rgb]{0.388930,0.795453,0.371421}
\pgfpathmoveto{\pgfpoint{173.087997pt}{319.301056pt}}
\pgflineto{\pgfpoint{182.015991pt}{313.124207pt}}
\pgflineto{\pgfpoint{173.087997pt}{313.124207pt}}
\pgfpathclose
\pgfusepath{fill,stroke}
\pgfpathmoveto{\pgfpoint{173.087997pt}{319.301056pt}}
\pgflineto{\pgfpoint{182.015991pt}{319.301056pt}}
\pgflineto{\pgfpoint{182.015991pt}{313.124207pt}}
\pgfpathclose
\pgfusepath{fill,stroke}
\pgfpathmoveto{\pgfpoint{173.087997pt}{325.477905pt}}
\pgflineto{\pgfpoint{182.015991pt}{319.301056pt}}
\pgflineto{\pgfpoint{173.087997pt}{319.301056pt}}
\pgfpathclose
\pgfusepath{fill,stroke}
\pgfpathmoveto{\pgfpoint{173.087997pt}{325.477905pt}}
\pgflineto{\pgfpoint{182.015991pt}{325.477905pt}}
\pgflineto{\pgfpoint{182.015991pt}{319.301056pt}}
\pgfpathclose
\pgfusepath{fill,stroke}
\color[rgb]{0.424933,0.806674,0.350099}
\pgfpathmoveto{\pgfpoint{173.087997pt}{331.654724pt}}
\pgflineto{\pgfpoint{182.015991pt}{325.477905pt}}
\pgflineto{\pgfpoint{173.087997pt}{325.477905pt}}
\pgfpathclose
\pgfusepath{fill,stroke}
\pgfpathmoveto{\pgfpoint{173.087997pt}{331.654724pt}}
\pgflineto{\pgfpoint{182.015991pt}{331.654724pt}}
\pgflineto{\pgfpoint{182.015991pt}{325.477905pt}}
\pgfpathclose
\pgfusepath{fill,stroke}
\color[rgb]{0.462247,0.817338,0.327545}
\pgfpathmoveto{\pgfpoint{173.087997pt}{337.831604pt}}
\pgflineto{\pgfpoint{182.015991pt}{331.654724pt}}
\pgflineto{\pgfpoint{173.087997pt}{331.654724pt}}
\pgfpathclose
\pgfusepath{fill,stroke}
\pgfpathmoveto{\pgfpoint{173.087997pt}{337.831604pt}}
\pgflineto{\pgfpoint{182.015991pt}{337.831604pt}}
\pgflineto{\pgfpoint{182.015991pt}{331.654724pt}}
\pgfpathclose
\pgfusepath{fill,stroke}
\color[rgb]{0.500754,0.827409,0.303799}
\pgfpathmoveto{\pgfpoint{173.087997pt}{344.008423pt}}
\pgflineto{\pgfpoint{182.015991pt}{337.831604pt}}
\pgflineto{\pgfpoint{173.087997pt}{337.831604pt}}
\pgfpathclose
\pgfusepath{fill,stroke}
\pgfpathmoveto{\pgfpoint{173.087997pt}{344.008423pt}}
\pgflineto{\pgfpoint{182.015991pt}{344.008423pt}}
\pgflineto{\pgfpoint{182.015991pt}{337.831604pt}}
\pgfpathclose
\pgfusepath{fill,stroke}
\color[rgb]{0.540337,0.836858,0.278917}
\pgfpathmoveto{\pgfpoint{173.087997pt}{350.185242pt}}
\pgflineto{\pgfpoint{182.015991pt}{344.008423pt}}
\pgflineto{\pgfpoint{173.087997pt}{344.008423pt}}
\pgfpathclose
\pgfusepath{fill,stroke}
\pgfpathmoveto{\pgfpoint{173.087997pt}{350.185242pt}}
\pgflineto{\pgfpoint{182.015991pt}{350.185242pt}}
\pgflineto{\pgfpoint{182.015991pt}{344.008423pt}}
\pgfpathclose
\pgfusepath{fill,stroke}
\color[rgb]{0.388930,0.795453,0.371421}
\pgfpathmoveto{\pgfpoint{182.015991pt}{313.124207pt}}
\pgflineto{\pgfpoint{190.943985pt}{306.947388pt}}
\pgflineto{\pgfpoint{182.015991pt}{306.947388pt}}
\pgfpathclose
\pgfusepath{fill,stroke}
\pgfpathmoveto{\pgfpoint{182.015991pt}{313.124207pt}}
\pgflineto{\pgfpoint{190.943985pt}{313.124207pt}}
\pgflineto{\pgfpoint{190.943985pt}{306.947388pt}}
\pgfpathclose
\pgfusepath{fill,stroke}
\color[rgb]{0.424933,0.806674,0.350099}
\pgfpathmoveto{\pgfpoint{182.015991pt}{319.301056pt}}
\pgflineto{\pgfpoint{190.943985pt}{313.124207pt}}
\pgflineto{\pgfpoint{182.015991pt}{313.124207pt}}
\pgfpathclose
\pgfusepath{fill,stroke}
\pgfpathmoveto{\pgfpoint{182.015991pt}{319.301056pt}}
\pgflineto{\pgfpoint{190.943985pt}{319.301056pt}}
\pgflineto{\pgfpoint{190.943985pt}{313.124207pt}}
\pgfpathclose
\pgfusepath{fill,stroke}
\pgfpathmoveto{\pgfpoint{182.015991pt}{325.477905pt}}
\pgflineto{\pgfpoint{190.943985pt}{319.301056pt}}
\pgflineto{\pgfpoint{182.015991pt}{319.301056pt}}
\pgfpathclose
\pgfusepath{fill,stroke}
\pgfpathmoveto{\pgfpoint{182.015991pt}{325.477905pt}}
\pgflineto{\pgfpoint{190.943985pt}{325.477905pt}}
\pgflineto{\pgfpoint{190.943985pt}{319.301056pt}}
\pgfpathclose
\pgfusepath{fill,stroke}
\color[rgb]{0.462247,0.817338,0.327545}
\pgfpathmoveto{\pgfpoint{182.015991pt}{331.654724pt}}
\pgflineto{\pgfpoint{190.943985pt}{325.477905pt}}
\pgflineto{\pgfpoint{182.015991pt}{325.477905pt}}
\pgfpathclose
\pgfusepath{fill,stroke}
\pgfpathmoveto{\pgfpoint{182.015991pt}{331.654724pt}}
\pgflineto{\pgfpoint{190.943985pt}{331.654724pt}}
\pgflineto{\pgfpoint{190.943985pt}{325.477905pt}}
\pgfpathclose
\pgfusepath{fill,stroke}
\color[rgb]{0.500754,0.827409,0.303799}
\pgfpathmoveto{\pgfpoint{182.015991pt}{337.831604pt}}
\pgflineto{\pgfpoint{190.943985pt}{331.654724pt}}
\pgflineto{\pgfpoint{182.015991pt}{331.654724pt}}
\pgfpathclose
\pgfusepath{fill,stroke}
\pgfpathmoveto{\pgfpoint{182.015991pt}{337.831604pt}}
\pgflineto{\pgfpoint{190.943985pt}{337.831604pt}}
\pgflineto{\pgfpoint{190.943985pt}{331.654724pt}}
\pgfpathclose
\pgfusepath{fill,stroke}
\color[rgb]{0.540337,0.836858,0.278917}
\pgfpathmoveto{\pgfpoint{182.015991pt}{344.008423pt}}
\pgflineto{\pgfpoint{190.943985pt}{337.831604pt}}
\pgflineto{\pgfpoint{182.015991pt}{337.831604pt}}
\pgfpathclose
\pgfusepath{fill,stroke}
\pgfpathmoveto{\pgfpoint{182.015991pt}{344.008423pt}}
\pgflineto{\pgfpoint{190.943985pt}{344.008423pt}}
\pgflineto{\pgfpoint{190.943985pt}{337.831604pt}}
\pgfpathclose
\pgfusepath{fill,stroke}
\color[rgb]{0.580861,0.845663,0.253001}
\pgfpathmoveto{\pgfpoint{182.015991pt}{350.185242pt}}
\pgflineto{\pgfpoint{190.943985pt}{344.008423pt}}
\pgflineto{\pgfpoint{182.015991pt}{344.008423pt}}
\pgfpathclose
\pgfusepath{fill,stroke}
\pgfpathmoveto{\pgfpoint{182.015991pt}{350.185242pt}}
\pgflineto{\pgfpoint{190.943985pt}{350.185242pt}}
\pgflineto{\pgfpoint{190.943985pt}{344.008423pt}}
\pgfpathclose
\pgfusepath{fill,stroke}
\color[rgb]{0.622171,0.853816,0.226224}
\pgfpathmoveto{\pgfpoint{182.015991pt}{356.362122pt}}
\pgflineto{\pgfpoint{190.943985pt}{350.185242pt}}
\pgflineto{\pgfpoint{182.015991pt}{350.185242pt}}
\pgfpathclose
\pgfusepath{fill,stroke}
\pgfpathmoveto{\pgfpoint{182.015991pt}{356.362122pt}}
\pgflineto{\pgfpoint{190.943985pt}{356.362122pt}}
\pgflineto{\pgfpoint{190.943985pt}{350.185242pt}}
\pgfpathclose
\pgfusepath{fill,stroke}
\pgfpathmoveto{\pgfpoint{182.015991pt}{362.538940pt}}
\pgflineto{\pgfpoint{190.943985pt}{356.362122pt}}
\pgflineto{\pgfpoint{182.015991pt}{356.362122pt}}
\pgfpathclose
\pgfusepath{fill,stroke}
\pgfpathmoveto{\pgfpoint{182.015991pt}{362.538940pt}}
\pgflineto{\pgfpoint{190.943985pt}{362.538940pt}}
\pgflineto{\pgfpoint{190.943985pt}{356.362122pt}}
\pgfpathclose
\pgfusepath{fill,stroke}
\color[rgb]{0.500754,0.827409,0.303799}
\pgfpathmoveto{\pgfpoint{190.943985pt}{325.477905pt}}
\pgflineto{\pgfpoint{199.871979pt}{319.301056pt}}
\pgflineto{\pgfpoint{190.943985pt}{319.301056pt}}
\pgfpathclose
\pgfusepath{fill,stroke}
\pgfpathmoveto{\pgfpoint{190.943985pt}{325.477905pt}}
\pgflineto{\pgfpoint{199.871979pt}{325.477905pt}}
\pgflineto{\pgfpoint{199.871979pt}{319.301056pt}}
\pgfpathclose
\pgfusepath{fill,stroke}
\pgfpathmoveto{\pgfpoint{190.943985pt}{331.654724pt}}
\pgflineto{\pgfpoint{199.871979pt}{325.477905pt}}
\pgflineto{\pgfpoint{190.943985pt}{325.477905pt}}
\pgfpathclose
\pgfusepath{fill,stroke}
\pgfpathmoveto{\pgfpoint{190.943985pt}{331.654724pt}}
\pgflineto{\pgfpoint{199.871979pt}{331.654724pt}}
\pgflineto{\pgfpoint{199.871979pt}{325.477905pt}}
\pgfpathclose
\pgfusepath{fill,stroke}
\color[rgb]{0.540337,0.836858,0.278917}
\pgfpathmoveto{\pgfpoint{190.943985pt}{337.831604pt}}
\pgflineto{\pgfpoint{199.871979pt}{331.654724pt}}
\pgflineto{\pgfpoint{190.943985pt}{331.654724pt}}
\pgfpathclose
\pgfusepath{fill,stroke}
\pgfpathmoveto{\pgfpoint{190.943985pt}{337.831604pt}}
\pgflineto{\pgfpoint{199.871979pt}{337.831604pt}}
\pgflineto{\pgfpoint{199.871979pt}{331.654724pt}}
\pgfpathclose
\pgfusepath{fill,stroke}
\color[rgb]{0.580861,0.845663,0.253001}
\pgfpathmoveto{\pgfpoint{190.943985pt}{344.008423pt}}
\pgflineto{\pgfpoint{199.871979pt}{337.831604pt}}
\pgflineto{\pgfpoint{190.943985pt}{337.831604pt}}
\pgfpathclose
\pgfusepath{fill,stroke}
\pgfpathmoveto{\pgfpoint{190.943985pt}{344.008423pt}}
\pgflineto{\pgfpoint{199.871979pt}{344.008423pt}}
\pgflineto{\pgfpoint{199.871979pt}{337.831604pt}}
\pgfpathclose
\pgfusepath{fill,stroke}
\color[rgb]{0.622171,0.853816,0.226224}
\pgfpathmoveto{\pgfpoint{190.943985pt}{350.185242pt}}
\pgflineto{\pgfpoint{199.871979pt}{344.008423pt}}
\pgflineto{\pgfpoint{190.943985pt}{344.008423pt}}
\pgfpathclose
\pgfusepath{fill,stroke}
\pgfpathmoveto{\pgfpoint{190.943985pt}{350.185242pt}}
\pgflineto{\pgfpoint{199.871979pt}{350.185242pt}}
\pgflineto{\pgfpoint{199.871979pt}{344.008423pt}}
\pgfpathclose
\pgfusepath{fill,stroke}
\color[rgb]{0.664087,0.861321,0.198879}
\pgfpathmoveto{\pgfpoint{190.943985pt}{356.362122pt}}
\pgflineto{\pgfpoint{199.871979pt}{350.185242pt}}
\pgflineto{\pgfpoint{190.943985pt}{350.185242pt}}
\pgfpathclose
\pgfusepath{fill,stroke}
\pgfpathmoveto{\pgfpoint{190.943985pt}{356.362122pt}}
\pgflineto{\pgfpoint{199.871979pt}{356.362122pt}}
\pgflineto{\pgfpoint{199.871979pt}{350.185242pt}}
\pgfpathclose
\pgfusepath{fill,stroke}
\color[rgb]{0.706404,0.868206,0.171495}
\pgfpathmoveto{\pgfpoint{190.943985pt}{362.538940pt}}
\pgflineto{\pgfpoint{199.871979pt}{356.362122pt}}
\pgflineto{\pgfpoint{190.943985pt}{356.362122pt}}
\pgfpathclose
\pgfusepath{fill,stroke}
\pgfpathmoveto{\pgfpoint{190.943985pt}{362.538940pt}}
\pgflineto{\pgfpoint{199.871979pt}{362.538940pt}}
\pgflineto{\pgfpoint{199.871979pt}{356.362122pt}}
\pgfpathclose
\pgfusepath{fill,stroke}
\pgfpathmoveto{\pgfpoint{190.943985pt}{368.715820pt}}
\pgflineto{\pgfpoint{199.871979pt}{362.538940pt}}
\pgflineto{\pgfpoint{190.943985pt}{362.538940pt}}
\pgfpathclose
\pgfusepath{fill,stroke}
\pgfpathmoveto{\pgfpoint{190.943985pt}{368.715820pt}}
\pgflineto{\pgfpoint{199.871979pt}{368.715820pt}}
\pgflineto{\pgfpoint{199.871979pt}{362.538940pt}}
\pgfpathclose
\pgfusepath{fill,stroke}
\color[rgb]{0.748885,0.874522,0.145038}
\pgfpathmoveto{\pgfpoint{190.943985pt}{374.892639pt}}
\pgflineto{\pgfpoint{199.871979pt}{368.715820pt}}
\pgflineto{\pgfpoint{190.943985pt}{368.715820pt}}
\pgfpathclose
\pgfusepath{fill,stroke}
\pgfpathmoveto{\pgfpoint{190.943985pt}{374.892639pt}}
\pgflineto{\pgfpoint{199.871979pt}{374.892639pt}}
\pgflineto{\pgfpoint{199.871979pt}{368.715820pt}}
\pgfpathclose
\pgfusepath{fill,stroke}
\color[rgb]{0.622171,0.853816,0.226224}
\pgfpathmoveto{\pgfpoint{199.871979pt}{344.008423pt}}
\pgflineto{\pgfpoint{208.799988pt}{337.831604pt}}
\pgflineto{\pgfpoint{199.871979pt}{337.831604pt}}
\pgfpathclose
\pgfusepath{fill,stroke}
\pgfpathmoveto{\pgfpoint{199.871979pt}{344.008423pt}}
\pgflineto{\pgfpoint{208.799988pt}{344.008423pt}}
\pgflineto{\pgfpoint{208.799988pt}{337.831604pt}}
\pgfpathclose
\pgfusepath{fill,stroke}
\color[rgb]{0.664087,0.861321,0.198879}
\pgfpathmoveto{\pgfpoint{199.871979pt}{350.185242pt}}
\pgflineto{\pgfpoint{208.799988pt}{344.008423pt}}
\pgflineto{\pgfpoint{199.871979pt}{344.008423pt}}
\pgfpathclose
\pgfusepath{fill,stroke}
\pgfpathmoveto{\pgfpoint{199.871979pt}{350.185242pt}}
\pgflineto{\pgfpoint{208.799988pt}{350.185242pt}}
\pgflineto{\pgfpoint{208.799988pt}{344.008423pt}}
\pgfpathclose
\pgfusepath{fill,stroke}
\color[rgb]{0.706404,0.868206,0.171495}
\pgfpathmoveto{\pgfpoint{199.871979pt}{356.362122pt}}
\pgflineto{\pgfpoint{208.799988pt}{350.185242pt}}
\pgflineto{\pgfpoint{199.871979pt}{350.185242pt}}
\pgfpathclose
\pgfusepath{fill,stroke}
\pgfpathmoveto{\pgfpoint{199.871979pt}{356.362122pt}}
\pgflineto{\pgfpoint{208.799988pt}{356.362122pt}}
\pgflineto{\pgfpoint{208.799988pt}{350.185242pt}}
\pgfpathclose
\pgfusepath{fill,stroke}
\color[rgb]{0.748885,0.874522,0.145038}
\pgfpathmoveto{\pgfpoint{199.871979pt}{362.538940pt}}
\pgflineto{\pgfpoint{208.799988pt}{356.362122pt}}
\pgflineto{\pgfpoint{199.871979pt}{356.362122pt}}
\pgfpathclose
\pgfusepath{fill,stroke}
\pgfpathmoveto{\pgfpoint{199.871979pt}{362.538940pt}}
\pgflineto{\pgfpoint{208.799988pt}{362.538940pt}}
\pgflineto{\pgfpoint{208.799988pt}{356.362122pt}}
\pgfpathclose
\pgfusepath{fill,stroke}
\pgfpathmoveto{\pgfpoint{199.871979pt}{368.715820pt}}
\pgflineto{\pgfpoint{208.799988pt}{362.538940pt}}
\pgflineto{\pgfpoint{199.871979pt}{362.538940pt}}
\pgfpathclose
\pgfusepath{fill,stroke}
\pgfpathmoveto{\pgfpoint{199.871979pt}{368.715820pt}}
\pgflineto{\pgfpoint{208.799988pt}{368.715820pt}}
\pgflineto{\pgfpoint{208.799988pt}{362.538940pt}}
\pgfpathclose
\pgfusepath{fill,stroke}
\pgfpathmoveto{\pgfpoint{208.799988pt}{356.362122pt}}
\pgflineto{\pgfpoint{217.727982pt}{350.185242pt}}
\pgflineto{\pgfpoint{208.799988pt}{350.185242pt}}
\pgfpathclose
\pgfusepath{fill,stroke}
\pgfpathmoveto{\pgfpoint{208.799988pt}{356.362122pt}}
\pgflineto{\pgfpoint{217.727982pt}{356.362122pt}}
\pgflineto{\pgfpoint{217.727982pt}{350.185242pt}}
\pgfpathclose
\pgfusepath{fill,stroke}
\color[rgb]{0.791273,0.880346,0.121291}
\pgfpathmoveto{\pgfpoint{208.799988pt}{362.538940pt}}
\pgflineto{\pgfpoint{217.727982pt}{356.362122pt}}
\pgflineto{\pgfpoint{208.799988pt}{356.362122pt}}
\pgfpathclose
\pgfusepath{fill,stroke}
\pgfpathmoveto{\pgfpoint{208.799988pt}{362.538940pt}}
\pgflineto{\pgfpoint{217.727982pt}{362.538940pt}}
\pgflineto{\pgfpoint{217.727982pt}{356.362122pt}}
\pgfpathclose
\pgfusepath{fill,stroke}
\color[rgb]{0.177272,0.437886,0.557576}
\pgfpathmoveto{\pgfpoint{110.591980pt}{195.764206pt}}
\pgflineto{\pgfpoint{119.519989pt}{195.764206pt}}
\pgflineto{\pgfpoint{119.519989pt}{189.587372pt}}
\pgfpathclose
\pgfusepath{fill,stroke}
\color[rgb]{0.170958,0.453063,0.557974}
\pgfpathmoveto{\pgfpoint{119.519989pt}{195.764206pt}}
\pgflineto{\pgfpoint{128.447998pt}{189.587372pt}}
\pgflineto{\pgfpoint{119.519989pt}{189.587372pt}}
\pgfpathclose
\pgfusepath{fill,stroke}
\pgfpathmoveto{\pgfpoint{119.519989pt}{195.764206pt}}
\pgflineto{\pgfpoint{128.447998pt}{195.764206pt}}
\pgflineto{\pgfpoint{128.447998pt}{189.587372pt}}
\pgfpathclose
\pgfusepath{fill,stroke}
\color[rgb]{0.164833,0.468130,0.558143}
\pgfpathmoveto{\pgfpoint{119.519989pt}{201.941055pt}}
\pgflineto{\pgfpoint{128.447998pt}{195.764206pt}}
\pgflineto{\pgfpoint{119.519989pt}{195.764206pt}}
\pgfpathclose
\pgfusepath{fill,stroke}
\color[rgb]{0.141402,0.527854,0.555864}
\pgfpathmoveto{\pgfpoint{128.447998pt}{220.471588pt}}
\pgflineto{\pgfpoint{137.376007pt}{220.471588pt}}
\pgflineto{\pgfpoint{137.376007pt}{214.294739pt}}
\pgfpathclose
\pgfusepath{fill,stroke}
\color[rgb]{0.135833,0.542750,0.554289}
\pgfpathmoveto{\pgfpoint{137.376007pt}{220.471588pt}}
\pgflineto{\pgfpoint{146.303986pt}{214.294739pt}}
\pgflineto{\pgfpoint{137.376007pt}{214.294739pt}}
\pgfpathclose
\pgfusepath{fill,stroke}
\pgfpathmoveto{\pgfpoint{137.376007pt}{220.471588pt}}
\pgflineto{\pgfpoint{146.303986pt}{220.471588pt}}
\pgflineto{\pgfpoint{146.303986pt}{214.294739pt}}
\pgfpathclose
\pgfusepath{fill,stroke}
\color[rgb]{0.130582,0.557652,0.552176}
\pgfpathmoveto{\pgfpoint{137.376007pt}{226.648422pt}}
\pgflineto{\pgfpoint{146.303986pt}{220.471588pt}}
\pgflineto{\pgfpoint{137.376007pt}{220.471588pt}}
\pgfpathclose
\pgfusepath{fill,stroke}
\color[rgb]{0.119627,0.617266,0.536796}
\pgfpathmoveto{\pgfpoint{146.303986pt}{251.355804pt}}
\pgflineto{\pgfpoint{155.231979pt}{251.355804pt}}
\pgflineto{\pgfpoint{155.231979pt}{245.178955pt}}
\pgfpathclose
\pgfusepath{fill,stroke}
\color[rgb]{0.122046,0.632107,0.530848}
\pgfpathmoveto{\pgfpoint{155.231979pt}{251.355804pt}}
\pgflineto{\pgfpoint{164.160004pt}{245.178955pt}}
\pgflineto{\pgfpoint{155.231979pt}{245.178955pt}}
\pgfpathclose
\pgfusepath{fill,stroke}
\pgfpathmoveto{\pgfpoint{155.231979pt}{251.355804pt}}
\pgflineto{\pgfpoint{164.160004pt}{251.355804pt}}
\pgflineto{\pgfpoint{164.160004pt}{245.178955pt}}
\pgfpathclose
\pgfusepath{fill,stroke}
\color[rgb]{0.127668,0.646882,0.523924}
\pgfpathmoveto{\pgfpoint{155.231979pt}{257.532623pt}}
\pgflineto{\pgfpoint{164.160004pt}{251.355804pt}}
\pgflineto{\pgfpoint{155.231979pt}{251.355804pt}}
\pgfpathclose
\pgfusepath{fill,stroke}
\color[rgb]{0.185538,0.704725,0.485412}
\pgfpathmoveto{\pgfpoint{164.160004pt}{276.063141pt}}
\pgflineto{\pgfpoint{173.087997pt}{276.063141pt}}
\pgflineto{\pgfpoint{173.087997pt}{269.886322pt}}
\pgfpathclose
\pgfusepath{fill,stroke}
\color[rgb]{0.208030,0.718701,0.472873}
\pgfpathmoveto{\pgfpoint{173.087997pt}{276.063141pt}}
\pgflineto{\pgfpoint{182.015991pt}{269.886322pt}}
\pgflineto{\pgfpoint{173.087997pt}{269.886322pt}}
\pgfpathclose
\pgfusepath{fill,stroke}
\pgfpathmoveto{\pgfpoint{173.087997pt}{276.063141pt}}
\pgflineto{\pgfpoint{182.015991pt}{276.063141pt}}
\pgflineto{\pgfpoint{182.015991pt}{269.886322pt}}
\pgfpathclose
\pgfusepath{fill,stroke}
\pgfpathmoveto{\pgfpoint{173.087997pt}{282.239990pt}}
\pgflineto{\pgfpoint{182.015991pt}{276.063141pt}}
\pgflineto{\pgfpoint{173.087997pt}{276.063141pt}}
\pgfpathclose
\pgfusepath{fill,stroke}
\color[rgb]{0.290001,0.758846,0.427826}
\pgfpathmoveto{\pgfpoint{182.015991pt}{294.593689pt}}
\pgflineto{\pgfpoint{190.943985pt}{294.593689pt}}
\pgflineto{\pgfpoint{190.943985pt}{288.416840pt}}
\pgfpathclose
\pgfusepath{fill,stroke}
\color[rgb]{0.321330,0.771498,0.410293}
\pgfpathmoveto{\pgfpoint{182.015991pt}{300.770538pt}}
\pgflineto{\pgfpoint{190.943985pt}{294.593689pt}}
\pgflineto{\pgfpoint{182.015991pt}{294.593689pt}}
\pgfpathclose
\pgfusepath{fill,stroke}
\pgfpathmoveto{\pgfpoint{182.015991pt}{300.770538pt}}
\pgflineto{\pgfpoint{190.943985pt}{300.770538pt}}
\pgflineto{\pgfpoint{190.943985pt}{294.593689pt}}
\pgfpathclose
\pgfusepath{fill,stroke}
\color[rgb]{0.354355,0.783714,0.391488}
\pgfpathmoveto{\pgfpoint{182.015991pt}{306.947388pt}}
\pgflineto{\pgfpoint{190.943985pt}{300.770538pt}}
\pgflineto{\pgfpoint{182.015991pt}{300.770538pt}}
\pgfpathclose
\pgfusepath{fill,stroke}
\pgfpathmoveto{\pgfpoint{182.015991pt}{306.947388pt}}
\pgflineto{\pgfpoint{190.943985pt}{306.947388pt}}
\pgflineto{\pgfpoint{190.943985pt}{300.770538pt}}
\pgfpathclose
\pgfusepath{fill,stroke}
\color[rgb]{0.321330,0.771498,0.410293}
\pgfpathmoveto{\pgfpoint{190.943985pt}{294.593689pt}}
\pgflineto{\pgfpoint{199.871979pt}{288.416840pt}}
\pgflineto{\pgfpoint{190.943985pt}{288.416840pt}}
\pgfpathclose
\pgfusepath{fill,stroke}
\pgfpathmoveto{\pgfpoint{190.943985pt}{294.593689pt}}
\pgflineto{\pgfpoint{199.871979pt}{294.593689pt}}
\pgflineto{\pgfpoint{199.871979pt}{288.416840pt}}
\pgfpathclose
\pgfusepath{fill,stroke}
\color[rgb]{0.354355,0.783714,0.391488}
\pgfpathmoveto{\pgfpoint{190.943985pt}{300.770538pt}}
\pgflineto{\pgfpoint{199.871979pt}{294.593689pt}}
\pgflineto{\pgfpoint{190.943985pt}{294.593689pt}}
\pgfpathclose
\pgfusepath{fill,stroke}
\pgfpathmoveto{\pgfpoint{190.943985pt}{300.770538pt}}
\pgflineto{\pgfpoint{199.871979pt}{300.770538pt}}
\pgflineto{\pgfpoint{199.871979pt}{294.593689pt}}
\pgfpathclose
\pgfusepath{fill,stroke}
\color[rgb]{0.388930,0.795453,0.371421}
\pgfpathmoveto{\pgfpoint{190.943985pt}{306.947388pt}}
\pgflineto{\pgfpoint{199.871979pt}{300.770538pt}}
\pgflineto{\pgfpoint{190.943985pt}{300.770538pt}}
\pgfpathclose
\pgfusepath{fill,stroke}
\pgfpathmoveto{\pgfpoint{190.943985pt}{306.947388pt}}
\pgflineto{\pgfpoint{199.871979pt}{306.947388pt}}
\pgflineto{\pgfpoint{199.871979pt}{300.770538pt}}
\pgfpathclose
\pgfusepath{fill,stroke}
\color[rgb]{0.424933,0.806674,0.350099}
\pgfpathmoveto{\pgfpoint{190.943985pt}{313.124207pt}}
\pgflineto{\pgfpoint{199.871979pt}{306.947388pt}}
\pgflineto{\pgfpoint{190.943985pt}{306.947388pt}}
\pgfpathclose
\pgfusepath{fill,stroke}
\pgfpathmoveto{\pgfpoint{190.943985pt}{313.124207pt}}
\pgflineto{\pgfpoint{199.871979pt}{313.124207pt}}
\pgflineto{\pgfpoint{199.871979pt}{306.947388pt}}
\pgfpathclose
\pgfusepath{fill,stroke}
\pgfpathmoveto{\pgfpoint{199.871979pt}{306.947388pt}}
\pgflineto{\pgfpoint{208.799988pt}{300.770538pt}}
\pgflineto{\pgfpoint{199.871979pt}{300.770538pt}}
\pgfpathclose
\pgfusepath{fill,stroke}
\pgfpathmoveto{\pgfpoint{199.871979pt}{306.947388pt}}
\pgflineto{\pgfpoint{208.799988pt}{306.947388pt}}
\pgflineto{\pgfpoint{208.799988pt}{300.770538pt}}
\pgfpathclose
\pgfusepath{fill,stroke}
\color[rgb]{0.462247,0.817338,0.327545}
\pgfpathmoveto{\pgfpoint{199.871979pt}{313.124207pt}}
\pgflineto{\pgfpoint{208.799988pt}{306.947388pt}}
\pgflineto{\pgfpoint{199.871979pt}{306.947388pt}}
\pgfpathclose
\pgfusepath{fill,stroke}
\pgfpathmoveto{\pgfpoint{199.871979pt}{313.124207pt}}
\pgflineto{\pgfpoint{208.799988pt}{313.124207pt}}
\pgflineto{\pgfpoint{208.799988pt}{306.947388pt}}
\pgfpathclose
\pgfusepath{fill,stroke}
\color[rgb]{0.500754,0.827409,0.303799}
\pgfpathmoveto{\pgfpoint{199.871979pt}{319.301056pt}}
\pgflineto{\pgfpoint{208.799988pt}{313.124207pt}}
\pgflineto{\pgfpoint{199.871979pt}{313.124207pt}}
\pgfpathclose
\pgfusepath{fill,stroke}
\pgfpathmoveto{\pgfpoint{199.871979pt}{319.301056pt}}
\pgflineto{\pgfpoint{208.799988pt}{319.301056pt}}
\pgflineto{\pgfpoint{208.799988pt}{313.124207pt}}
\pgfpathclose
\pgfusepath{fill,stroke}
\color[rgb]{0.540337,0.836858,0.278917}
\pgfpathmoveto{\pgfpoint{208.799988pt}{319.301056pt}}
\pgflineto{\pgfpoint{217.727982pt}{313.124207pt}}
\pgflineto{\pgfpoint{208.799988pt}{313.124207pt}}
\pgfpathclose
\pgfusepath{fill,stroke}
\pgfpathmoveto{\pgfpoint{208.799988pt}{319.301056pt}}
\pgflineto{\pgfpoint{217.727982pt}{319.301056pt}}
\pgflineto{\pgfpoint{217.727982pt}{313.124207pt}}
\pgfpathclose
\pgfusepath{fill,stroke}
\color[rgb]{0.706404,0.868206,0.171495}
\pgfpathmoveto{\pgfpoint{208.799988pt}{350.185242pt}}
\pgflineto{\pgfpoint{217.727982pt}{344.008423pt}}
\pgflineto{\pgfpoint{208.799988pt}{344.008423pt}}
\pgfpathclose
\pgfusepath{fill,stroke}
\pgfpathmoveto{\pgfpoint{208.799988pt}{350.185242pt}}
\pgflineto{\pgfpoint{217.727982pt}{350.185242pt}}
\pgflineto{\pgfpoint{217.727982pt}{344.008423pt}}
\pgfpathclose
\pgfusepath{fill,stroke}
\pgfpathmoveto{\pgfpoint{217.727982pt}{344.008423pt}}
\pgflineto{\pgfpoint{226.655975pt}{344.008423pt}}
\pgflineto{\pgfpoint{226.655975pt}{337.831604pt}}
\pgfpathclose
\pgfusepath{fill,stroke}
\color[rgb]{0.748885,0.874522,0.145038}
\pgfpathmoveto{\pgfpoint{217.727982pt}{350.185242pt}}
\pgflineto{\pgfpoint{226.655975pt}{344.008423pt}}
\pgflineto{\pgfpoint{217.727982pt}{344.008423pt}}
\pgfpathclose
\pgfusepath{fill,stroke}
\pgfpathmoveto{\pgfpoint{217.727982pt}{350.185242pt}}
\pgflineto{\pgfpoint{226.655975pt}{350.185242pt}}
\pgflineto{\pgfpoint{226.655975pt}{344.008423pt}}
\pgfpathclose
\pgfusepath{fill,stroke}
\color[rgb]{0.791273,0.880346,0.121291}
\pgfpathmoveto{\pgfpoint{217.727982pt}{356.362122pt}}
\pgflineto{\pgfpoint{226.655975pt}{350.185242pt}}
\pgflineto{\pgfpoint{217.727982pt}{350.185242pt}}
\pgfpathclose
\pgfusepath{fill,stroke}
\pgfpathmoveto{\pgfpoint{217.727982pt}{356.362122pt}}
\pgflineto{\pgfpoint{226.655975pt}{356.362122pt}}
\pgflineto{\pgfpoint{226.655975pt}{350.185242pt}}
\pgfpathclose
\pgfusepath{fill,stroke}
\color[rgb]{0.748885,0.874522,0.145038}
\pgfpathmoveto{\pgfpoint{226.655975pt}{344.008423pt}}
\pgflineto{\pgfpoint{235.583969pt}{337.831604pt}}
\pgflineto{\pgfpoint{226.655975pt}{337.831604pt}}
\pgfpathclose
\pgfusepath{fill,stroke}
\pgfpathmoveto{\pgfpoint{226.655975pt}{344.008423pt}}
\pgflineto{\pgfpoint{235.583969pt}{344.008423pt}}
\pgflineto{\pgfpoint{235.583969pt}{337.831604pt}}
\pgfpathclose
\pgfusepath{fill,stroke}
\color[rgb]{0.791273,0.880346,0.121291}
\pgfpathmoveto{\pgfpoint{226.655975pt}{350.185242pt}}
\pgflineto{\pgfpoint{235.583969pt}{344.008423pt}}
\pgflineto{\pgfpoint{226.655975pt}{344.008423pt}}
\pgfpathclose
\pgfusepath{fill,stroke}
\pgfpathmoveto{\pgfpoint{226.655975pt}{350.185242pt}}
\pgflineto{\pgfpoint{235.583969pt}{350.185242pt}}
\pgflineto{\pgfpoint{235.583969pt}{344.008423pt}}
\pgfpathclose
\pgfusepath{fill,stroke}
\color[rgb]{0.462247,0.817338,0.327545}
\pgfpathmoveto{\pgfpoint{190.943985pt}{319.301056pt}}
\pgflineto{\pgfpoint{199.871979pt}{313.124207pt}}
\pgflineto{\pgfpoint{190.943985pt}{313.124207pt}}
\pgfpathclose
\pgfusepath{fill,stroke}
\pgfpathmoveto{\pgfpoint{190.943985pt}{319.301056pt}}
\pgflineto{\pgfpoint{199.871979pt}{319.301056pt}}
\pgflineto{\pgfpoint{199.871979pt}{313.124207pt}}
\pgfpathclose
\pgfusepath{fill,stroke}
\color[rgb]{0.540337,0.836858,0.278917}
\pgfpathmoveto{\pgfpoint{199.871979pt}{325.477905pt}}
\pgflineto{\pgfpoint{208.799988pt}{319.301056pt}}
\pgflineto{\pgfpoint{199.871979pt}{319.301056pt}}
\pgfpathclose
\pgfusepath{fill,stroke}
\pgfpathmoveto{\pgfpoint{199.871979pt}{325.477905pt}}
\pgflineto{\pgfpoint{208.799988pt}{325.477905pt}}
\pgflineto{\pgfpoint{208.799988pt}{319.301056pt}}
\pgfpathclose
\pgfusepath{fill,stroke}
\pgfpathmoveto{\pgfpoint{199.871979pt}{331.654724pt}}
\pgflineto{\pgfpoint{208.799988pt}{325.477905pt}}
\pgflineto{\pgfpoint{199.871979pt}{325.477905pt}}
\pgfpathclose
\pgfusepath{fill,stroke}
\pgfpathmoveto{\pgfpoint{199.871979pt}{331.654724pt}}
\pgflineto{\pgfpoint{208.799988pt}{331.654724pt}}
\pgflineto{\pgfpoint{208.799988pt}{325.477905pt}}
\pgfpathclose
\pgfusepath{fill,stroke}
\color[rgb]{0.580861,0.845663,0.253001}
\pgfpathmoveto{\pgfpoint{199.871979pt}{337.831604pt}}
\pgflineto{\pgfpoint{208.799988pt}{331.654724pt}}
\pgflineto{\pgfpoint{199.871979pt}{331.654724pt}}
\pgfpathclose
\pgfusepath{fill,stroke}
\pgfpathmoveto{\pgfpoint{199.871979pt}{337.831604pt}}
\pgflineto{\pgfpoint{208.799988pt}{337.831604pt}}
\pgflineto{\pgfpoint{208.799988pt}{331.654724pt}}
\pgfpathclose
\pgfusepath{fill,stroke}
\pgfpathmoveto{\pgfpoint{208.799988pt}{325.477905pt}}
\pgflineto{\pgfpoint{217.727982pt}{319.301056pt}}
\pgflineto{\pgfpoint{208.799988pt}{319.301056pt}}
\pgfpathclose
\pgfusepath{fill,stroke}
\pgfpathmoveto{\pgfpoint{208.799988pt}{325.477905pt}}
\pgflineto{\pgfpoint{217.727982pt}{325.477905pt}}
\pgflineto{\pgfpoint{217.727982pt}{319.301056pt}}
\pgfpathclose
\pgfusepath{fill,stroke}
\color[rgb]{0.622171,0.853816,0.226224}
\pgfpathmoveto{\pgfpoint{208.799988pt}{331.654724pt}}
\pgflineto{\pgfpoint{217.727982pt}{325.477905pt}}
\pgflineto{\pgfpoint{208.799988pt}{325.477905pt}}
\pgfpathclose
\pgfusepath{fill,stroke}
\pgfpathmoveto{\pgfpoint{208.799988pt}{331.654724pt}}
\pgflineto{\pgfpoint{217.727982pt}{331.654724pt}}
\pgflineto{\pgfpoint{217.727982pt}{325.477905pt}}
\pgfpathclose
\pgfusepath{fill,stroke}
\pgfpathmoveto{\pgfpoint{208.799988pt}{337.831604pt}}
\pgflineto{\pgfpoint{217.727982pt}{331.654724pt}}
\pgflineto{\pgfpoint{208.799988pt}{331.654724pt}}
\pgfpathclose
\pgfusepath{fill,stroke}
\pgfpathmoveto{\pgfpoint{208.799988pt}{337.831604pt}}
\pgflineto{\pgfpoint{217.727982pt}{337.831604pt}}
\pgflineto{\pgfpoint{217.727982pt}{331.654724pt}}
\pgfpathclose
\pgfusepath{fill,stroke}
\color[rgb]{0.664087,0.861321,0.198879}
\pgfpathmoveto{\pgfpoint{208.799988pt}{344.008423pt}}
\pgflineto{\pgfpoint{217.727982pt}{337.831604pt}}
\pgflineto{\pgfpoint{208.799988pt}{337.831604pt}}
\pgfpathclose
\pgfusepath{fill,stroke}
\pgfpathmoveto{\pgfpoint{208.799988pt}{344.008423pt}}
\pgflineto{\pgfpoint{217.727982pt}{344.008423pt}}
\pgflineto{\pgfpoint{217.727982pt}{337.831604pt}}
\pgfpathclose
\pgfusepath{fill,stroke}
\pgfpathmoveto{\pgfpoint{217.727982pt}{331.654724pt}}
\pgflineto{\pgfpoint{226.655975pt}{325.477905pt}}
\pgflineto{\pgfpoint{217.727982pt}{325.477905pt}}
\pgfpathclose
\pgfusepath{fill,stroke}
\pgfpathmoveto{\pgfpoint{217.727982pt}{331.654724pt}}
\pgflineto{\pgfpoint{226.655975pt}{331.654724pt}}
\pgflineto{\pgfpoint{226.655975pt}{325.477905pt}}
\pgfpathclose
\pgfusepath{fill,stroke}
\pgfpathmoveto{\pgfpoint{217.727982pt}{337.831604pt}}
\pgflineto{\pgfpoint{226.655975pt}{331.654724pt}}
\pgflineto{\pgfpoint{217.727982pt}{331.654724pt}}
\pgfpathclose
\pgfusepath{fill,stroke}
\pgfpathmoveto{\pgfpoint{217.727982pt}{337.831604pt}}
\pgflineto{\pgfpoint{226.655975pt}{337.831604pt}}
\pgflineto{\pgfpoint{226.655975pt}{331.654724pt}}
\pgfpathclose
\pgfusepath{fill,stroke}
\color[rgb]{0.706404,0.868206,0.171495}
\pgfpathmoveto{\pgfpoint{217.727982pt}{344.008423pt}}
\pgflineto{\pgfpoint{226.655975pt}{337.831604pt}}
\pgflineto{\pgfpoint{217.727982pt}{337.831604pt}}
\pgfpathclose
\pgfusepath{fill,stroke}
\color[rgb]{0.208030,0.718701,0.472873}
\pgfpathmoveto{\pgfpoint{173.087997pt}{282.239990pt}}
\pgflineto{\pgfpoint{182.015991pt}{282.239990pt}}
\pgflineto{\pgfpoint{182.015991pt}{276.063141pt}}
\pgfpathclose
\pgfusepath{fill,stroke}
\color[rgb]{0.233127,0.732406,0.459106}
\pgfpathmoveto{\pgfpoint{173.087997pt}{288.416840pt}}
\pgflineto{\pgfpoint{182.015991pt}{282.239990pt}}
\pgflineto{\pgfpoint{173.087997pt}{282.239990pt}}
\pgfpathclose
\pgfusepath{fill,stroke}
\pgfpathmoveto{\pgfpoint{173.087997pt}{288.416840pt}}
\pgflineto{\pgfpoint{182.015991pt}{288.416840pt}}
\pgflineto{\pgfpoint{182.015991pt}{282.239990pt}}
\pgfpathclose
\pgfusepath{fill,stroke}
\color[rgb]{0.260531,0.745802,0.444096}
\pgfpathmoveto{\pgfpoint{182.015991pt}{288.416840pt}}
\pgflineto{\pgfpoint{190.943985pt}{282.239990pt}}
\pgflineto{\pgfpoint{182.015991pt}{282.239990pt}}
\pgfpathclose
\pgfusepath{fill,stroke}
\pgfpathmoveto{\pgfpoint{182.015991pt}{288.416840pt}}
\pgflineto{\pgfpoint{190.943985pt}{288.416840pt}}
\pgflineto{\pgfpoint{190.943985pt}{282.239990pt}}
\pgfpathclose
\pgfusepath{fill,stroke}
\color[rgb]{0.290001,0.758846,0.427826}
\pgfpathmoveto{\pgfpoint{182.015991pt}{294.593689pt}}
\pgflineto{\pgfpoint{190.943985pt}{288.416840pt}}
\pgflineto{\pgfpoint{182.015991pt}{288.416840pt}}
\pgfpathclose
\pgfusepath{fill,stroke}
\color[rgb]{0.127668,0.646882,0.523924}
\pgfpathmoveto{\pgfpoint{155.231979pt}{257.532623pt}}
\pgflineto{\pgfpoint{164.160004pt}{257.532623pt}}
\pgflineto{\pgfpoint{164.160004pt}{251.355804pt}}
\pgfpathclose
\pgfusepath{fill,stroke}
\color[rgb]{0.136835,0.661563,0.515967}
\pgfpathmoveto{\pgfpoint{155.231979pt}{263.709473pt}}
\pgflineto{\pgfpoint{164.160004pt}{257.532623pt}}
\pgflineto{\pgfpoint{155.231979pt}{257.532623pt}}
\pgfpathclose
\pgfusepath{fill,stroke}
\pgfpathmoveto{\pgfpoint{155.231979pt}{263.709473pt}}
\pgflineto{\pgfpoint{164.160004pt}{263.709473pt}}
\pgflineto{\pgfpoint{164.160004pt}{257.532623pt}}
\pgfpathclose
\pgfusepath{fill,stroke}
\color[rgb]{0.149643,0.676120,0.506924}
\pgfpathmoveto{\pgfpoint{164.160004pt}{263.709473pt}}
\pgflineto{\pgfpoint{173.087997pt}{257.532623pt}}
\pgflineto{\pgfpoint{164.160004pt}{257.532623pt}}
\pgfpathclose
\pgfusepath{fill,stroke}
\pgfpathmoveto{\pgfpoint{164.160004pt}{263.709473pt}}
\pgflineto{\pgfpoint{173.087997pt}{263.709473pt}}
\pgflineto{\pgfpoint{173.087997pt}{257.532623pt}}
\pgfpathclose
\pgfusepath{fill,stroke}
\color[rgb]{0.165967,0.690519,0.496752}
\pgfpathmoveto{\pgfpoint{164.160004pt}{269.886322pt}}
\pgflineto{\pgfpoint{173.087997pt}{263.709473pt}}
\pgflineto{\pgfpoint{164.160004pt}{263.709473pt}}
\pgfpathclose
\pgfusepath{fill,stroke}
\pgfpathmoveto{\pgfpoint{164.160004pt}{269.886322pt}}
\pgflineto{\pgfpoint{173.087997pt}{269.886322pt}}
\pgflineto{\pgfpoint{173.087997pt}{263.709473pt}}
\pgfpathclose
\pgfusepath{fill,stroke}
\color[rgb]{0.185538,0.704725,0.485412}
\pgfpathmoveto{\pgfpoint{164.160004pt}{276.063141pt}}
\pgflineto{\pgfpoint{173.087997pt}{269.886322pt}}
\pgflineto{\pgfpoint{164.160004pt}{269.886322pt}}
\pgfpathclose
\pgfusepath{fill,stroke}
\color[rgb]{0.130582,0.557652,0.552176}
\pgfpathmoveto{\pgfpoint{137.376007pt}{226.648422pt}}
\pgflineto{\pgfpoint{146.303986pt}{226.648422pt}}
\pgflineto{\pgfpoint{146.303986pt}{220.471588pt}}
\pgfpathclose
\pgfusepath{fill,stroke}
\pgfpathmoveto{\pgfpoint{137.376007pt}{232.825272pt}}
\pgflineto{\pgfpoint{146.303986pt}{226.648422pt}}
\pgflineto{\pgfpoint{137.376007pt}{226.648422pt}}
\pgfpathclose
\pgfusepath{fill,stroke}
\pgfpathmoveto{\pgfpoint{137.376007pt}{232.825272pt}}
\pgflineto{\pgfpoint{146.303986pt}{232.825272pt}}
\pgflineto{\pgfpoint{146.303986pt}{226.648422pt}}
\pgfpathclose
\pgfusepath{fill,stroke}
\color[rgb]{0.122163,0.587476,0.546023}
\pgfpathmoveto{\pgfpoint{146.303986pt}{232.825272pt}}
\pgflineto{\pgfpoint{155.231979pt}{226.648422pt}}
\pgflineto{\pgfpoint{146.303986pt}{226.648422pt}}
\pgfpathclose
\pgfusepath{fill,stroke}
\pgfpathmoveto{\pgfpoint{146.303986pt}{232.825272pt}}
\pgflineto{\pgfpoint{155.231979pt}{232.825272pt}}
\pgflineto{\pgfpoint{155.231979pt}{226.648422pt}}
\pgfpathclose
\pgfusepath{fill,stroke}
\pgfpathmoveto{\pgfpoint{146.303986pt}{239.002106pt}}
\pgflineto{\pgfpoint{155.231979pt}{232.825272pt}}
\pgflineto{\pgfpoint{146.303986pt}{232.825272pt}}
\pgfpathclose
\pgfusepath{fill,stroke}
\pgfpathmoveto{\pgfpoint{146.303986pt}{239.002106pt}}
\pgflineto{\pgfpoint{155.231979pt}{239.002106pt}}
\pgflineto{\pgfpoint{155.231979pt}{232.825272pt}}
\pgfpathclose
\pgfusepath{fill,stroke}
\color[rgb]{0.119872,0.602382,0.541831}
\pgfpathmoveto{\pgfpoint{146.303986pt}{245.178955pt}}
\pgflineto{\pgfpoint{155.231979pt}{239.002106pt}}
\pgflineto{\pgfpoint{146.303986pt}{239.002106pt}}
\pgfpathclose
\pgfusepath{fill,stroke}
\pgfpathmoveto{\pgfpoint{146.303986pt}{245.178955pt}}
\pgflineto{\pgfpoint{155.231979pt}{245.178955pt}}
\pgflineto{\pgfpoint{155.231979pt}{239.002106pt}}
\pgfpathclose
\pgfusepath{fill,stroke}
\color[rgb]{0.119627,0.617266,0.536796}
\pgfpathmoveto{\pgfpoint{146.303986pt}{251.355804pt}}
\pgflineto{\pgfpoint{155.231979pt}{245.178955pt}}
\pgflineto{\pgfpoint{146.303986pt}{245.178955pt}}
\pgfpathclose
\pgfusepath{fill,stroke}
\color[rgb]{0.164833,0.468130,0.558143}
\pgfpathmoveto{\pgfpoint{119.519989pt}{201.941055pt}}
\pgflineto{\pgfpoint{128.447998pt}{201.941055pt}}
\pgflineto{\pgfpoint{128.447998pt}{195.764206pt}}
\pgfpathclose
\pgfusepath{fill,stroke}
\color[rgb]{0.158845,0.483117,0.558059}
\pgfpathmoveto{\pgfpoint{119.519989pt}{208.117905pt}}
\pgflineto{\pgfpoint{128.447998pt}{201.941055pt}}
\pgflineto{\pgfpoint{119.519989pt}{201.941055pt}}
\pgfpathclose
\pgfusepath{fill,stroke}
\pgfpathmoveto{\pgfpoint{119.519989pt}{208.117905pt}}
\pgflineto{\pgfpoint{128.447998pt}{208.117905pt}}
\pgflineto{\pgfpoint{128.447998pt}{201.941055pt}}
\pgfpathclose
\pgfusepath{fill,stroke}
\color[rgb]{0.152951,0.498053,0.557685}
\pgfpathmoveto{\pgfpoint{128.447998pt}{208.117905pt}}
\pgflineto{\pgfpoint{137.376007pt}{201.941055pt}}
\pgflineto{\pgfpoint{128.447998pt}{201.941055pt}}
\pgfpathclose
\pgfusepath{fill,stroke}
\pgfpathmoveto{\pgfpoint{128.447998pt}{208.117905pt}}
\pgflineto{\pgfpoint{137.376007pt}{208.117905pt}}
\pgflineto{\pgfpoint{137.376007pt}{201.941055pt}}
\pgfpathclose
\pgfusepath{fill,stroke}
\color[rgb]{0.147132,0.512959,0.556973}
\pgfpathmoveto{\pgfpoint{128.447998pt}{214.294739pt}}
\pgflineto{\pgfpoint{137.376007pt}{208.117905pt}}
\pgflineto{\pgfpoint{128.447998pt}{208.117905pt}}
\pgfpathclose
\pgfusepath{fill,stroke}
\pgfpathmoveto{\pgfpoint{128.447998pt}{214.294739pt}}
\pgflineto{\pgfpoint{137.376007pt}{214.294739pt}}
\pgflineto{\pgfpoint{137.376007pt}{208.117905pt}}
\pgfpathclose
\pgfusepath{fill,stroke}
\color[rgb]{0.141402,0.527854,0.555864}
\pgfpathmoveto{\pgfpoint{128.447998pt}{220.471588pt}}
\pgflineto{\pgfpoint{137.376007pt}{214.294739pt}}
\pgflineto{\pgfpoint{128.447998pt}{214.294739pt}}
\pgfpathclose
\pgfusepath{fill,stroke}
\color[rgb]{0.205079,0.375366,0.553493}
\pgfpathmoveto{\pgfpoint{101.664001pt}{171.056854pt}}
\pgflineto{\pgfpoint{110.591980pt}{171.056854pt}}
\pgflineto{\pgfpoint{110.591980pt}{164.880005pt}}
\pgfpathclose
\pgfusepath{fill,stroke}
\color[rgb]{0.197722,0.391341,0.554953}
\pgfpathmoveto{\pgfpoint{101.664001pt}{177.233673pt}}
\pgflineto{\pgfpoint{110.591980pt}{171.056854pt}}
\pgflineto{\pgfpoint{101.664001pt}{171.056854pt}}
\pgfpathclose
\pgfusepath{fill,stroke}
\pgfpathmoveto{\pgfpoint{101.664001pt}{177.233673pt}}
\pgflineto{\pgfpoint{110.591980pt}{177.233673pt}}
\pgflineto{\pgfpoint{110.591980pt}{171.056854pt}}
\pgfpathclose
\pgfusepath{fill,stroke}
\pgfpathmoveto{\pgfpoint{101.664001pt}{183.410522pt}}
\pgflineto{\pgfpoint{110.591980pt}{177.233673pt}}
\pgflineto{\pgfpoint{101.664001pt}{177.233673pt}}
\pgfpathclose
\pgfusepath{fill,stroke}
\pgfpathmoveto{\pgfpoint{101.664001pt}{183.410522pt}}
\pgflineto{\pgfpoint{110.591980pt}{183.410522pt}}
\pgflineto{\pgfpoint{110.591980pt}{177.233673pt}}
\pgfpathclose
\pgfusepath{fill,stroke}
\color[rgb]{0.190631,0.407061,0.556089}
\pgfpathmoveto{\pgfpoint{110.591980pt}{183.410522pt}}
\pgflineto{\pgfpoint{119.519989pt}{177.233673pt}}
\pgflineto{\pgfpoint{110.591980pt}{177.233673pt}}
\pgfpathclose
\pgfusepath{fill,stroke}
\pgfpathmoveto{\pgfpoint{110.591980pt}{183.410522pt}}
\pgflineto{\pgfpoint{119.519989pt}{183.410522pt}}
\pgflineto{\pgfpoint{119.519989pt}{177.233673pt}}
\pgfpathclose
\pgfusepath{fill,stroke}
\color[rgb]{0.183819,0.422564,0.556952}
\pgfpathmoveto{\pgfpoint{110.591980pt}{189.587372pt}}
\pgflineto{\pgfpoint{119.519989pt}{183.410522pt}}
\pgflineto{\pgfpoint{110.591980pt}{183.410522pt}}
\pgfpathclose
\pgfusepath{fill,stroke}
\pgfpathmoveto{\pgfpoint{110.591980pt}{189.587372pt}}
\pgflineto{\pgfpoint{119.519989pt}{189.587372pt}}
\pgflineto{\pgfpoint{119.519989pt}{183.410522pt}}
\pgfpathclose
\pgfusepath{fill,stroke}
\color[rgb]{0.177272,0.437886,0.557576}
\pgfpathmoveto{\pgfpoint{110.591980pt}{195.764206pt}}
\pgflineto{\pgfpoint{119.519989pt}{189.587372pt}}
\pgflineto{\pgfpoint{110.591980pt}{189.587372pt}}
\pgfpathclose
\pgfusepath{fill,stroke}
\color[rgb]{0.269982,0.216330,0.508255}
\pgfpathmoveto{\pgfpoint{74.880005pt}{133.995789pt}}
\pgflineto{\pgfpoint{83.807999pt}{133.995789pt}}
\pgflineto{\pgfpoint{83.807999pt}{127.818947pt}}
\pgfpathclose
\pgfusepath{fill,stroke}
\color[rgb]{0.264369,0.235405,0.517732}
\pgfpathmoveto{\pgfpoint{74.880005pt}{140.172638pt}}
\pgflineto{\pgfpoint{83.807999pt}{133.995789pt}}
\pgflineto{\pgfpoint{74.880005pt}{133.995789pt}}
\pgfpathclose
\pgfusepath{fill,stroke}
\pgfpathmoveto{\pgfpoint{74.880005pt}{140.172638pt}}
\pgflineto{\pgfpoint{83.807999pt}{140.172638pt}}
\pgflineto{\pgfpoint{83.807999pt}{133.995789pt}}
\pgfpathclose
\pgfusepath{fill,stroke}
\color[rgb]{0.258026,0.254162,0.525780}
\pgfpathmoveto{\pgfpoint{74.880005pt}{146.349472pt}}
\pgflineto{\pgfpoint{83.807999pt}{140.172638pt}}
\pgflineto{\pgfpoint{74.880005pt}{140.172638pt}}
\pgfpathclose
\pgfusepath{fill,stroke}
\pgfpathmoveto{\pgfpoint{74.880005pt}{146.349472pt}}
\pgflineto{\pgfpoint{83.807999pt}{146.349472pt}}
\pgflineto{\pgfpoint{83.807999pt}{140.172638pt}}
\pgfpathclose
\pgfusepath{fill,stroke}
\color[rgb]{0.251099,0.272573,0.532522}
\pgfpathmoveto{\pgfpoint{74.880005pt}{152.526306pt}}
\pgflineto{\pgfpoint{83.807999pt}{146.349472pt}}
\pgflineto{\pgfpoint{74.880005pt}{146.349472pt}}
\pgfpathclose
\pgfusepath{fill,stroke}
\pgfpathmoveto{\pgfpoint{74.880005pt}{152.526306pt}}
\pgflineto{\pgfpoint{83.807999pt}{152.526306pt}}
\pgflineto{\pgfpoint{83.807999pt}{146.349472pt}}
\pgfpathclose
\pgfusepath{fill,stroke}
\color[rgb]{0.243733,0.290620,0.538097}
\pgfpathmoveto{\pgfpoint{74.880005pt}{158.703156pt}}
\pgflineto{\pgfpoint{83.807999pt}{152.526306pt}}
\pgflineto{\pgfpoint{74.880005pt}{152.526306pt}}
\pgfpathclose
\pgfusepath{fill,stroke}
\pgfpathmoveto{\pgfpoint{74.880005pt}{158.703156pt}}
\pgflineto{\pgfpoint{83.807999pt}{158.703156pt}}
\pgflineto{\pgfpoint{83.807999pt}{152.526306pt}}
\pgfpathclose
\pgfusepath{fill,stroke}
\color[rgb]{0.236073,0.308291,0.542652}
\pgfpathmoveto{\pgfpoint{74.880005pt}{164.880005pt}}
\pgflineto{\pgfpoint{83.807999pt}{158.703156pt}}
\pgflineto{\pgfpoint{74.880005pt}{158.703156pt}}
\pgfpathclose
\pgfusepath{fill,stroke}
\pgfpathmoveto{\pgfpoint{74.880005pt}{164.880005pt}}
\pgflineto{\pgfpoint{83.807999pt}{164.880005pt}}
\pgflineto{\pgfpoint{83.807999pt}{158.703156pt}}
\pgfpathclose
\pgfusepath{fill,stroke}
\pgfpathmoveto{\pgfpoint{74.880005pt}{171.056854pt}}
\pgflineto{\pgfpoint{83.807999pt}{164.880005pt}}
\pgflineto{\pgfpoint{74.880005pt}{164.880005pt}}
\pgfpathclose
\pgfusepath{fill,stroke}
\color[rgb]{0.258026,0.254162,0.525780}
\pgfpathmoveto{\pgfpoint{83.807999pt}{140.172638pt}}
\pgflineto{\pgfpoint{92.735992pt}{133.995789pt}}
\pgflineto{\pgfpoint{83.807999pt}{133.995789pt}}
\pgfpathclose
\pgfusepath{fill,stroke}
\pgfpathmoveto{\pgfpoint{83.807999pt}{140.172638pt}}
\pgflineto{\pgfpoint{92.735992pt}{140.172638pt}}
\pgflineto{\pgfpoint{92.735992pt}{133.995789pt}}
\pgfpathclose
\pgfusepath{fill,stroke}
\color[rgb]{0.251099,0.272573,0.532522}
\pgfpathmoveto{\pgfpoint{83.807999pt}{146.349472pt}}
\pgflineto{\pgfpoint{92.735992pt}{140.172638pt}}
\pgflineto{\pgfpoint{83.807999pt}{140.172638pt}}
\pgfpathclose
\pgfusepath{fill,stroke}
\pgfpathmoveto{\pgfpoint{83.807999pt}{146.349472pt}}
\pgflineto{\pgfpoint{92.735992pt}{146.349472pt}}
\pgflineto{\pgfpoint{92.735992pt}{140.172638pt}}
\pgfpathclose
\pgfusepath{fill,stroke}
\color[rgb]{0.243733,0.290620,0.538097}
\pgfpathmoveto{\pgfpoint{83.807999pt}{152.526306pt}}
\pgflineto{\pgfpoint{92.735992pt}{146.349472pt}}
\pgflineto{\pgfpoint{83.807999pt}{146.349472pt}}
\pgfpathclose
\pgfusepath{fill,stroke}
\pgfpathmoveto{\pgfpoint{83.807999pt}{152.526306pt}}
\pgflineto{\pgfpoint{92.735992pt}{152.526306pt}}
\pgflineto{\pgfpoint{92.735992pt}{146.349472pt}}
\pgfpathclose
\pgfusepath{fill,stroke}
\color[rgb]{0.236073,0.308291,0.542652}
\pgfpathmoveto{\pgfpoint{83.807999pt}{158.703156pt}}
\pgflineto{\pgfpoint{92.735992pt}{152.526306pt}}
\pgflineto{\pgfpoint{83.807999pt}{152.526306pt}}
\pgfpathclose
\pgfusepath{fill,stroke}
\color[rgb]{0.283205,0.116893,0.437179}
\pgfpathmoveto{\pgfpoint{74.880005pt}{96.934731pt}}
\pgflineto{\pgfpoint{83.807999pt}{90.757896pt}}
\pgflineto{\pgfpoint{74.880005pt}{90.757896pt}}
\pgfpathclose
\pgfusepath{fill,stroke}
\pgfpathmoveto{\pgfpoint{74.880005pt}{96.934731pt}}
\pgflineto{\pgfpoint{83.807999pt}{96.934731pt}}
\pgflineto{\pgfpoint{83.807999pt}{90.757896pt}}
\pgfpathclose
\pgfusepath{fill,stroke}
\color[rgb]{0.282809,0.137350,0.454596}
\pgfpathmoveto{\pgfpoint{74.880005pt}{103.111580pt}}
\pgflineto{\pgfpoint{83.807999pt}{96.934731pt}}
\pgflineto{\pgfpoint{74.880005pt}{96.934731pt}}
\pgfpathclose
\pgfusepath{fill,stroke}
\pgfpathmoveto{\pgfpoint{74.880005pt}{103.111580pt}}
\pgflineto{\pgfpoint{83.807999pt}{103.111580pt}}
\pgflineto{\pgfpoint{83.807999pt}{96.934731pt}}
\pgfpathclose
\pgfusepath{fill,stroke}
\color[rgb]{0.281231,0.157480,0.470434}
\pgfpathmoveto{\pgfpoint{74.880005pt}{109.288422pt}}
\pgflineto{\pgfpoint{83.807999pt}{103.111580pt}}
\pgflineto{\pgfpoint{74.880005pt}{103.111580pt}}
\pgfpathclose
\pgfusepath{fill,stroke}
\pgfpathmoveto{\pgfpoint{74.880005pt}{109.288422pt}}
\pgflineto{\pgfpoint{83.807999pt}{109.288422pt}}
\pgflineto{\pgfpoint{83.807999pt}{103.111580pt}}
\pgfpathclose
\pgfusepath{fill,stroke}
\color[rgb]{0.278516,0.177348,0.484654}
\pgfpathmoveto{\pgfpoint{74.880005pt}{115.465263pt}}
\pgflineto{\pgfpoint{83.807999pt}{109.288422pt}}
\pgflineto{\pgfpoint{74.880005pt}{109.288422pt}}
\pgfpathclose
\pgfusepath{fill,stroke}
\pgfpathmoveto{\pgfpoint{74.880005pt}{115.465263pt}}
\pgflineto{\pgfpoint{83.807999pt}{115.465263pt}}
\pgflineto{\pgfpoint{83.807999pt}{109.288422pt}}
\pgfpathclose
\pgfusepath{fill,stroke}
\color[rgb]{0.274736,0.196969,0.497250}
\pgfpathmoveto{\pgfpoint{74.880005pt}{121.642097pt}}
\pgflineto{\pgfpoint{83.807999pt}{115.465263pt}}
\pgflineto{\pgfpoint{74.880005pt}{115.465263pt}}
\pgfpathclose
\pgfusepath{fill,stroke}
\pgfpathmoveto{\pgfpoint{74.880005pt}{121.642097pt}}
\pgflineto{\pgfpoint{83.807999pt}{121.642097pt}}
\pgflineto{\pgfpoint{83.807999pt}{115.465263pt}}
\pgfpathclose
\pgfusepath{fill,stroke}
\color[rgb]{0.282809,0.137350,0.454596}
\pgfpathmoveto{\pgfpoint{83.807999pt}{90.757896pt}}
\pgflineto{\pgfpoint{92.735992pt}{90.757896pt}}
\pgflineto{\pgfpoint{92.735992pt}{84.581039pt}}
\pgfpathclose
\pgfusepath{fill,stroke}
\pgfpathmoveto{\pgfpoint{83.807999pt}{96.934731pt}}
\pgflineto{\pgfpoint{92.735992pt}{90.757896pt}}
\pgflineto{\pgfpoint{83.807999pt}{90.757896pt}}
\pgfpathclose
\pgfusepath{fill,stroke}
\pgfpathmoveto{\pgfpoint{83.807999pt}{96.934731pt}}
\pgflineto{\pgfpoint{92.735992pt}{96.934731pt}}
\pgflineto{\pgfpoint{92.735992pt}{90.757896pt}}
\pgfpathclose
\pgfusepath{fill,stroke}
\color[rgb]{0.281231,0.157480,0.470434}
\pgfpathmoveto{\pgfpoint{83.807999pt}{103.111580pt}}
\pgflineto{\pgfpoint{92.735992pt}{96.934731pt}}
\pgflineto{\pgfpoint{83.807999pt}{96.934731pt}}
\pgfpathclose
\pgfusepath{fill,stroke}
\pgfpathmoveto{\pgfpoint{83.807999pt}{103.111580pt}}
\pgflineto{\pgfpoint{92.735992pt}{103.111580pt}}
\pgflineto{\pgfpoint{92.735992pt}{96.934731pt}}
\pgfpathclose
\pgfusepath{fill,stroke}
\color[rgb]{0.278516,0.177348,0.484654}
\pgfpathmoveto{\pgfpoint{83.807999pt}{109.288422pt}}
\pgflineto{\pgfpoint{92.735992pt}{103.111580pt}}
\pgflineto{\pgfpoint{83.807999pt}{103.111580pt}}
\pgfpathclose
\pgfusepath{fill,stroke}
\pgfpathmoveto{\pgfpoint{83.807999pt}{109.288422pt}}
\pgflineto{\pgfpoint{92.735992pt}{109.288422pt}}
\pgflineto{\pgfpoint{92.735992pt}{103.111580pt}}
\pgfpathclose
\pgfusepath{fill,stroke}
\color[rgb]{0.274736,0.196969,0.497250}
\pgfpathmoveto{\pgfpoint{83.807999pt}{115.465263pt}}
\pgflineto{\pgfpoint{92.735992pt}{109.288422pt}}
\pgflineto{\pgfpoint{83.807999pt}{109.288422pt}}
\pgfpathclose
\pgfusepath{fill,stroke}
\pgfpathmoveto{\pgfpoint{83.807999pt}{115.465263pt}}
\pgflineto{\pgfpoint{92.735992pt}{115.465263pt}}
\pgflineto{\pgfpoint{92.735992pt}{109.288422pt}}
\pgfpathclose
\pgfusepath{fill,stroke}
\color[rgb]{0.269982,0.216330,0.508255}
\pgfpathmoveto{\pgfpoint{83.807999pt}{121.642097pt}}
\pgflineto{\pgfpoint{92.735992pt}{115.465263pt}}
\pgflineto{\pgfpoint{83.807999pt}{115.465263pt}}
\pgfpathclose
\pgfusepath{fill,stroke}
\pgfpathmoveto{\pgfpoint{83.807999pt}{121.642097pt}}
\pgflineto{\pgfpoint{92.735992pt}{121.642097pt}}
\pgflineto{\pgfpoint{92.735992pt}{115.465263pt}}
\pgfpathclose
\pgfusepath{fill,stroke}
\color[rgb]{0.264369,0.235405,0.517732}
\pgfpathmoveto{\pgfpoint{83.807999pt}{127.818947pt}}
\pgflineto{\pgfpoint{92.735992pt}{121.642097pt}}
\pgflineto{\pgfpoint{83.807999pt}{121.642097pt}}
\pgfpathclose
\pgfusepath{fill,stroke}
\pgfpathmoveto{\pgfpoint{83.807999pt}{127.818947pt}}
\pgflineto{\pgfpoint{92.735992pt}{127.818947pt}}
\pgflineto{\pgfpoint{92.735992pt}{121.642097pt}}
\pgfpathclose
\pgfusepath{fill,stroke}
\color[rgb]{0.258026,0.254162,0.525780}
\pgfpathmoveto{\pgfpoint{83.807999pt}{133.995789pt}}
\pgflineto{\pgfpoint{92.735992pt}{127.818947pt}}
\pgflineto{\pgfpoint{83.807999pt}{127.818947pt}}
\pgfpathclose
\pgfusepath{fill,stroke}
\pgfpathmoveto{\pgfpoint{83.807999pt}{133.995789pt}}
\pgflineto{\pgfpoint{92.735992pt}{133.995789pt}}
\pgflineto{\pgfpoint{92.735992pt}{127.818947pt}}
\pgfpathclose
\pgfusepath{fill,stroke}
\color[rgb]{0.281231,0.157480,0.470434}
\pgfpathmoveto{\pgfpoint{92.735992pt}{90.757896pt}}
\pgflineto{\pgfpoint{101.664001pt}{84.581039pt}}
\pgflineto{\pgfpoint{92.735992pt}{84.581039pt}}
\pgfpathclose
\pgfusepath{fill,stroke}
\pgfpathmoveto{\pgfpoint{92.735992pt}{90.757896pt}}
\pgflineto{\pgfpoint{101.664001pt}{90.757896pt}}
\pgflineto{\pgfpoint{101.664001pt}{84.581039pt}}
\pgfpathclose
\pgfusepath{fill,stroke}
\color[rgb]{0.278516,0.177348,0.484654}
\pgfpathmoveto{\pgfpoint{92.735992pt}{96.934731pt}}
\pgflineto{\pgfpoint{101.664001pt}{90.757896pt}}
\pgflineto{\pgfpoint{92.735992pt}{90.757896pt}}
\pgfpathclose
\pgfusepath{fill,stroke}
\pgfpathmoveto{\pgfpoint{92.735992pt}{96.934731pt}}
\pgflineto{\pgfpoint{101.664001pt}{96.934731pt}}
\pgflineto{\pgfpoint{101.664001pt}{90.757896pt}}
\pgfpathclose
\pgfusepath{fill,stroke}
\pgfpathmoveto{\pgfpoint{92.735992pt}{103.111580pt}}
\pgflineto{\pgfpoint{101.664001pt}{96.934731pt}}
\pgflineto{\pgfpoint{92.735992pt}{96.934731pt}}
\pgfpathclose
\pgfusepath{fill,stroke}
\pgfpathmoveto{\pgfpoint{92.735992pt}{103.111580pt}}
\pgflineto{\pgfpoint{101.664001pt}{103.111580pt}}
\pgflineto{\pgfpoint{101.664001pt}{96.934731pt}}
\pgfpathclose
\pgfusepath{fill,stroke}
\color[rgb]{0.274736,0.196969,0.497250}
\pgfpathmoveto{\pgfpoint{92.735992pt}{109.288422pt}}
\pgflineto{\pgfpoint{101.664001pt}{103.111580pt}}
\pgflineto{\pgfpoint{92.735992pt}{103.111580pt}}
\pgfpathclose
\pgfusepath{fill,stroke}
\pgfpathmoveto{\pgfpoint{92.735992pt}{109.288422pt}}
\pgflineto{\pgfpoint{101.664001pt}{109.288422pt}}
\pgflineto{\pgfpoint{101.664001pt}{103.111580pt}}
\pgfpathclose
\pgfusepath{fill,stroke}
\color[rgb]{0.269982,0.216330,0.508255}
\pgfpathmoveto{\pgfpoint{92.735992pt}{115.465263pt}}
\pgflineto{\pgfpoint{101.664001pt}{109.288422pt}}
\pgflineto{\pgfpoint{92.735992pt}{109.288422pt}}
\pgfpathclose
\pgfusepath{fill,stroke}
\pgfpathmoveto{\pgfpoint{92.735992pt}{115.465263pt}}
\pgflineto{\pgfpoint{101.664001pt}{115.465263pt}}
\pgflineto{\pgfpoint{101.664001pt}{109.288422pt}}
\pgfpathclose
\pgfusepath{fill,stroke}
\color[rgb]{0.264369,0.235405,0.517732}
\pgfpathmoveto{\pgfpoint{92.735992pt}{121.642097pt}}
\pgflineto{\pgfpoint{101.664001pt}{115.465263pt}}
\pgflineto{\pgfpoint{92.735992pt}{115.465263pt}}
\pgfpathclose
\pgfusepath{fill,stroke}
\pgfpathmoveto{\pgfpoint{92.735992pt}{121.642097pt}}
\pgflineto{\pgfpoint{101.664001pt}{121.642097pt}}
\pgflineto{\pgfpoint{101.664001pt}{115.465263pt}}
\pgfpathclose
\pgfusepath{fill,stroke}
\color[rgb]{0.258026,0.254162,0.525780}
\pgfpathmoveto{\pgfpoint{92.735992pt}{127.818947pt}}
\pgflineto{\pgfpoint{101.664001pt}{121.642097pt}}
\pgflineto{\pgfpoint{92.735992pt}{121.642097pt}}
\pgfpathclose
\pgfusepath{fill,stroke}
\pgfpathmoveto{\pgfpoint{92.735992pt}{127.818947pt}}
\pgflineto{\pgfpoint{101.664001pt}{127.818947pt}}
\pgflineto{\pgfpoint{101.664001pt}{121.642097pt}}
\pgfpathclose
\pgfusepath{fill,stroke}
\color[rgb]{0.251099,0.272573,0.532522}
\pgfpathmoveto{\pgfpoint{92.735992pt}{133.995789pt}}
\pgflineto{\pgfpoint{101.664001pt}{127.818947pt}}
\pgflineto{\pgfpoint{92.735992pt}{127.818947pt}}
\pgfpathclose
\pgfusepath{fill,stroke}
\pgfpathmoveto{\pgfpoint{92.735992pt}{133.995789pt}}
\pgflineto{\pgfpoint{101.664001pt}{133.995789pt}}
\pgflineto{\pgfpoint{101.664001pt}{127.818947pt}}
\pgfpathclose
\pgfusepath{fill,stroke}
\pgfpathmoveto{\pgfpoint{92.735992pt}{140.172638pt}}
\pgflineto{\pgfpoint{101.664001pt}{133.995789pt}}
\pgflineto{\pgfpoint{92.735992pt}{133.995789pt}}
\pgfpathclose
\pgfusepath{fill,stroke}
\pgfpathmoveto{\pgfpoint{92.735992pt}{140.172638pt}}
\pgflineto{\pgfpoint{101.664001pt}{140.172638pt}}
\pgflineto{\pgfpoint{101.664001pt}{133.995789pt}}
\pgfpathclose
\pgfusepath{fill,stroke}
\color[rgb]{0.243733,0.290620,0.538097}
\pgfpathmoveto{\pgfpoint{92.735992pt}{146.349472pt}}
\pgflineto{\pgfpoint{101.664001pt}{140.172638pt}}
\pgflineto{\pgfpoint{92.735992pt}{140.172638pt}}
\pgfpathclose
\pgfusepath{fill,stroke}
\pgfpathmoveto{\pgfpoint{92.735992pt}{146.349472pt}}
\pgflineto{\pgfpoint{101.664001pt}{146.349472pt}}
\pgflineto{\pgfpoint{101.664001pt}{140.172638pt}}
\pgfpathclose
\pgfusepath{fill,stroke}
\color[rgb]{0.236073,0.308291,0.542652}
\pgfpathmoveto{\pgfpoint{92.735992pt}{152.526306pt}}
\pgflineto{\pgfpoint{101.664001pt}{146.349472pt}}
\pgflineto{\pgfpoint{92.735992pt}{146.349472pt}}
\pgfpathclose
\pgfusepath{fill,stroke}
\pgfpathmoveto{\pgfpoint{92.735992pt}{152.526306pt}}
\pgflineto{\pgfpoint{101.664001pt}{152.526306pt}}
\pgflineto{\pgfpoint{101.664001pt}{146.349472pt}}
\pgfpathclose
\pgfusepath{fill,stroke}
\color[rgb]{0.274736,0.196969,0.497250}
\pgfpathmoveto{\pgfpoint{101.664001pt}{103.111580pt}}
\pgflineto{\pgfpoint{110.591980pt}{96.934731pt}}
\pgflineto{\pgfpoint{101.664001pt}{96.934731pt}}
\pgfpathclose
\pgfusepath{fill,stroke}
\pgfpathmoveto{\pgfpoint{101.664001pt}{103.111580pt}}
\pgflineto{\pgfpoint{110.591980pt}{103.111580pt}}
\pgflineto{\pgfpoint{110.591980pt}{96.934731pt}}
\pgfpathclose
\pgfusepath{fill,stroke}
\color[rgb]{0.269982,0.216330,0.508255}
\pgfpathmoveto{\pgfpoint{101.664001pt}{109.288422pt}}
\pgflineto{\pgfpoint{110.591980pt}{103.111580pt}}
\pgflineto{\pgfpoint{101.664001pt}{103.111580pt}}
\pgfpathclose
\pgfusepath{fill,stroke}
\pgfpathmoveto{\pgfpoint{101.664001pt}{109.288422pt}}
\pgflineto{\pgfpoint{110.591980pt}{109.288422pt}}
\pgflineto{\pgfpoint{110.591980pt}{103.111580pt}}
\pgfpathclose
\pgfusepath{fill,stroke}
\color[rgb]{0.264369,0.235405,0.517732}
\pgfpathmoveto{\pgfpoint{101.664001pt}{115.465263pt}}
\pgflineto{\pgfpoint{110.591980pt}{109.288422pt}}
\pgflineto{\pgfpoint{101.664001pt}{109.288422pt}}
\pgfpathclose
\pgfusepath{fill,stroke}
\pgfpathmoveto{\pgfpoint{101.664001pt}{115.465263pt}}
\pgflineto{\pgfpoint{110.591980pt}{115.465263pt}}
\pgflineto{\pgfpoint{110.591980pt}{109.288422pt}}
\pgfpathclose
\pgfusepath{fill,stroke}
\color[rgb]{0.258026,0.254162,0.525780}
\pgfpathmoveto{\pgfpoint{101.664001pt}{121.642097pt}}
\pgflineto{\pgfpoint{110.591980pt}{115.465263pt}}
\pgflineto{\pgfpoint{101.664001pt}{115.465263pt}}
\pgfpathclose
\pgfusepath{fill,stroke}
\pgfpathmoveto{\pgfpoint{101.664001pt}{121.642097pt}}
\pgflineto{\pgfpoint{110.591980pt}{121.642097pt}}
\pgflineto{\pgfpoint{110.591980pt}{115.465263pt}}
\pgfpathclose
\pgfusepath{fill,stroke}
\color[rgb]{0.251099,0.272573,0.532522}
\pgfpathmoveto{\pgfpoint{101.664001pt}{127.818947pt}}
\pgflineto{\pgfpoint{110.591980pt}{121.642097pt}}
\pgflineto{\pgfpoint{101.664001pt}{121.642097pt}}
\pgfpathclose
\pgfusepath{fill,stroke}
\pgfpathmoveto{\pgfpoint{101.664001pt}{127.818947pt}}
\pgflineto{\pgfpoint{110.591980pt}{127.818947pt}}
\pgflineto{\pgfpoint{110.591980pt}{121.642097pt}}
\pgfpathclose
\pgfusepath{fill,stroke}
\color[rgb]{0.243733,0.290620,0.538097}
\pgfpathmoveto{\pgfpoint{101.664001pt}{133.995789pt}}
\pgflineto{\pgfpoint{110.591980pt}{127.818947pt}}
\pgflineto{\pgfpoint{101.664001pt}{127.818947pt}}
\pgfpathclose
\pgfusepath{fill,stroke}
\pgfpathmoveto{\pgfpoint{101.664001pt}{133.995789pt}}
\pgflineto{\pgfpoint{110.591980pt}{133.995789pt}}
\pgflineto{\pgfpoint{110.591980pt}{127.818947pt}}
\pgfpathclose
\pgfusepath{fill,stroke}
\color[rgb]{0.236073,0.308291,0.542652}
\pgfpathmoveto{\pgfpoint{101.664001pt}{140.172638pt}}
\pgflineto{\pgfpoint{110.591980pt}{133.995789pt}}
\pgflineto{\pgfpoint{101.664001pt}{133.995789pt}}
\pgfpathclose
\pgfusepath{fill,stroke}
\pgfpathmoveto{\pgfpoint{101.664001pt}{140.172638pt}}
\pgflineto{\pgfpoint{110.591980pt}{140.172638pt}}
\pgflineto{\pgfpoint{110.591980pt}{133.995789pt}}
\pgfpathclose
\pgfusepath{fill,stroke}
\pgfpathmoveto{\pgfpoint{101.664001pt}{146.349472pt}}
\pgflineto{\pgfpoint{110.591980pt}{140.172638pt}}
\pgflineto{\pgfpoint{101.664001pt}{140.172638pt}}
\pgfpathclose
\pgfusepath{fill,stroke}
\pgfpathmoveto{\pgfpoint{101.664001pt}{146.349472pt}}
\pgflineto{\pgfpoint{110.591980pt}{146.349472pt}}
\pgflineto{\pgfpoint{110.591980pt}{140.172638pt}}
\pgfpathclose
\pgfusepath{fill,stroke}
\color[rgb]{0.228263,0.325586,0.546335}
\pgfpathmoveto{\pgfpoint{101.664001pt}{152.526306pt}}
\pgflineto{\pgfpoint{110.591980pt}{146.349472pt}}
\pgflineto{\pgfpoint{101.664001pt}{146.349472pt}}
\pgfpathclose
\pgfusepath{fill,stroke}
\pgfpathmoveto{\pgfpoint{101.664001pt}{152.526306pt}}
\pgflineto{\pgfpoint{110.591980pt}{152.526306pt}}
\pgflineto{\pgfpoint{110.591980pt}{146.349472pt}}
\pgfpathclose
\pgfusepath{fill,stroke}
\color[rgb]{0.220425,0.342517,0.549287}
\pgfpathmoveto{\pgfpoint{101.664001pt}{158.703156pt}}
\pgflineto{\pgfpoint{110.591980pt}{152.526306pt}}
\pgflineto{\pgfpoint{101.664001pt}{152.526306pt}}
\pgfpathclose
\pgfusepath{fill,stroke}
\pgfpathmoveto{\pgfpoint{101.664001pt}{158.703156pt}}
\pgflineto{\pgfpoint{110.591980pt}{158.703156pt}}
\pgflineto{\pgfpoint{110.591980pt}{152.526306pt}}
\pgfpathclose
\pgfusepath{fill,stroke}
\color[rgb]{0.212667,0.359102,0.551635}
\pgfpathmoveto{\pgfpoint{101.664001pt}{164.880005pt}}
\pgflineto{\pgfpoint{110.591980pt}{158.703156pt}}
\pgflineto{\pgfpoint{101.664001pt}{158.703156pt}}
\pgfpathclose
\pgfusepath{fill,stroke}
\pgfpathmoveto{\pgfpoint{101.664001pt}{164.880005pt}}
\pgflineto{\pgfpoint{110.591980pt}{164.880005pt}}
\pgflineto{\pgfpoint{110.591980pt}{158.703156pt}}
\pgfpathclose
\pgfusepath{fill,stroke}
\color[rgb]{0.258026,0.254162,0.525780}
\pgfpathmoveto{\pgfpoint{110.591980pt}{115.465263pt}}
\pgflineto{\pgfpoint{119.519989pt}{109.288422pt}}
\pgflineto{\pgfpoint{110.591980pt}{109.288422pt}}
\pgfpathclose
\pgfusepath{fill,stroke}
\pgfpathmoveto{\pgfpoint{110.591980pt}{115.465263pt}}
\pgflineto{\pgfpoint{119.519989pt}{115.465263pt}}
\pgflineto{\pgfpoint{119.519989pt}{109.288422pt}}
\pgfpathclose
\pgfusepath{fill,stroke}
\color[rgb]{0.251099,0.272573,0.532522}
\pgfpathmoveto{\pgfpoint{110.591980pt}{121.642097pt}}
\pgflineto{\pgfpoint{119.519989pt}{115.465263pt}}
\pgflineto{\pgfpoint{110.591980pt}{115.465263pt}}
\pgfpathclose
\pgfusepath{fill,stroke}
\pgfpathmoveto{\pgfpoint{110.591980pt}{121.642097pt}}
\pgflineto{\pgfpoint{119.519989pt}{121.642097pt}}
\pgflineto{\pgfpoint{119.519989pt}{115.465263pt}}
\pgfpathclose
\pgfusepath{fill,stroke}
\color[rgb]{0.243733,0.290620,0.538097}
\pgfpathmoveto{\pgfpoint{110.591980pt}{127.818947pt}}
\pgflineto{\pgfpoint{119.519989pt}{121.642097pt}}
\pgflineto{\pgfpoint{110.591980pt}{121.642097pt}}
\pgfpathclose
\pgfusepath{fill,stroke}
\pgfpathmoveto{\pgfpoint{110.591980pt}{127.818947pt}}
\pgflineto{\pgfpoint{119.519989pt}{127.818947pt}}
\pgflineto{\pgfpoint{119.519989pt}{121.642097pt}}
\pgfpathclose
\pgfusepath{fill,stroke}
\color[rgb]{0.236073,0.308291,0.542652}
\pgfpathmoveto{\pgfpoint{110.591980pt}{133.995789pt}}
\pgflineto{\pgfpoint{119.519989pt}{127.818947pt}}
\pgflineto{\pgfpoint{110.591980pt}{127.818947pt}}
\pgfpathclose
\pgfusepath{fill,stroke}
\pgfpathmoveto{\pgfpoint{110.591980pt}{133.995789pt}}
\pgflineto{\pgfpoint{119.519989pt}{133.995789pt}}
\pgflineto{\pgfpoint{119.519989pt}{127.818947pt}}
\pgfpathclose
\pgfusepath{fill,stroke}
\color[rgb]{0.228263,0.325586,0.546335}
\pgfpathmoveto{\pgfpoint{110.591980pt}{140.172638pt}}
\pgflineto{\pgfpoint{119.519989pt}{133.995789pt}}
\pgflineto{\pgfpoint{110.591980pt}{133.995789pt}}
\pgfpathclose
\pgfusepath{fill,stroke}
\pgfpathmoveto{\pgfpoint{110.591980pt}{140.172638pt}}
\pgflineto{\pgfpoint{119.519989pt}{140.172638pt}}
\pgflineto{\pgfpoint{119.519989pt}{133.995789pt}}
\pgfpathclose
\pgfusepath{fill,stroke}
\pgfpathmoveto{\pgfpoint{110.591980pt}{146.349472pt}}
\pgflineto{\pgfpoint{119.519989pt}{140.172638pt}}
\pgflineto{\pgfpoint{110.591980pt}{140.172638pt}}
\pgfpathclose
\pgfusepath{fill,stroke}
\pgfpathmoveto{\pgfpoint{110.591980pt}{146.349472pt}}
\pgflineto{\pgfpoint{119.519989pt}{146.349472pt}}
\pgflineto{\pgfpoint{119.519989pt}{140.172638pt}}
\pgfpathclose
\pgfusepath{fill,stroke}
\color[rgb]{0.220425,0.342517,0.549287}
\pgfpathmoveto{\pgfpoint{110.591980pt}{152.526306pt}}
\pgflineto{\pgfpoint{119.519989pt}{146.349472pt}}
\pgflineto{\pgfpoint{110.591980pt}{146.349472pt}}
\pgfpathclose
\pgfusepath{fill,stroke}
\pgfpathmoveto{\pgfpoint{110.591980pt}{152.526306pt}}
\pgflineto{\pgfpoint{119.519989pt}{152.526306pt}}
\pgflineto{\pgfpoint{119.519989pt}{146.349472pt}}
\pgfpathclose
\pgfusepath{fill,stroke}
\color[rgb]{0.212667,0.359102,0.551635}
\pgfpathmoveto{\pgfpoint{110.591980pt}{158.703156pt}}
\pgflineto{\pgfpoint{119.519989pt}{152.526306pt}}
\pgflineto{\pgfpoint{110.591980pt}{152.526306pt}}
\pgfpathclose
\pgfusepath{fill,stroke}
\pgfpathmoveto{\pgfpoint{110.591980pt}{158.703156pt}}
\pgflineto{\pgfpoint{119.519989pt}{158.703156pt}}
\pgflineto{\pgfpoint{119.519989pt}{152.526306pt}}
\pgfpathclose
\pgfusepath{fill,stroke}
\color[rgb]{0.205079,0.375366,0.553493}
\pgfpathmoveto{\pgfpoint{110.591980pt}{164.880005pt}}
\pgflineto{\pgfpoint{119.519989pt}{158.703156pt}}
\pgflineto{\pgfpoint{110.591980pt}{158.703156pt}}
\pgfpathclose
\pgfusepath{fill,stroke}
\pgfpathmoveto{\pgfpoint{110.591980pt}{164.880005pt}}
\pgflineto{\pgfpoint{119.519989pt}{164.880005pt}}
\pgflineto{\pgfpoint{119.519989pt}{158.703156pt}}
\pgfpathclose
\pgfusepath{fill,stroke}
\color[rgb]{0.197722,0.391341,0.554953}
\pgfpathmoveto{\pgfpoint{110.591980pt}{171.056854pt}}
\pgflineto{\pgfpoint{119.519989pt}{164.880005pt}}
\pgflineto{\pgfpoint{110.591980pt}{164.880005pt}}
\pgfpathclose
\pgfusepath{fill,stroke}
\pgfpathmoveto{\pgfpoint{110.591980pt}{171.056854pt}}
\pgflineto{\pgfpoint{119.519989pt}{171.056854pt}}
\pgflineto{\pgfpoint{119.519989pt}{164.880005pt}}
\pgfpathclose
\pgfusepath{fill,stroke}
\color[rgb]{0.190631,0.407061,0.556089}
\pgfpathmoveto{\pgfpoint{110.591980pt}{177.233673pt}}
\pgflineto{\pgfpoint{119.519989pt}{171.056854pt}}
\pgflineto{\pgfpoint{110.591980pt}{171.056854pt}}
\pgfpathclose
\pgfusepath{fill,stroke}
\pgfpathmoveto{\pgfpoint{110.591980pt}{177.233673pt}}
\pgflineto{\pgfpoint{119.519989pt}{177.233673pt}}
\pgflineto{\pgfpoint{119.519989pt}{171.056854pt}}
\pgfpathclose
\pgfusepath{fill,stroke}
\color[rgb]{0.236073,0.308291,0.542652}
\pgfpathmoveto{\pgfpoint{119.519989pt}{127.818947pt}}
\pgflineto{\pgfpoint{128.447998pt}{121.642097pt}}
\pgflineto{\pgfpoint{119.519989pt}{121.642097pt}}
\pgfpathclose
\pgfusepath{fill,stroke}
\pgfpathmoveto{\pgfpoint{119.519989pt}{127.818947pt}}
\pgflineto{\pgfpoint{128.447998pt}{127.818947pt}}
\pgflineto{\pgfpoint{128.447998pt}{121.642097pt}}
\pgfpathclose
\pgfusepath{fill,stroke}
\color[rgb]{0.228263,0.325586,0.546335}
\pgfpathmoveto{\pgfpoint{119.519989pt}{133.995789pt}}
\pgflineto{\pgfpoint{128.447998pt}{127.818947pt}}
\pgflineto{\pgfpoint{119.519989pt}{127.818947pt}}
\pgfpathclose
\pgfusepath{fill,stroke}
\pgfpathmoveto{\pgfpoint{119.519989pt}{133.995789pt}}
\pgflineto{\pgfpoint{128.447998pt}{133.995789pt}}
\pgflineto{\pgfpoint{128.447998pt}{127.818947pt}}
\pgfpathclose
\pgfusepath{fill,stroke}
\color[rgb]{0.220425,0.342517,0.549287}
\pgfpathmoveto{\pgfpoint{119.519989pt}{140.172638pt}}
\pgflineto{\pgfpoint{128.447998pt}{133.995789pt}}
\pgflineto{\pgfpoint{119.519989pt}{133.995789pt}}
\pgfpathclose
\pgfusepath{fill,stroke}
\pgfpathmoveto{\pgfpoint{119.519989pt}{140.172638pt}}
\pgflineto{\pgfpoint{128.447998pt}{140.172638pt}}
\pgflineto{\pgfpoint{128.447998pt}{133.995789pt}}
\pgfpathclose
\pgfusepath{fill,stroke}
\color[rgb]{0.212667,0.359102,0.551635}
\pgfpathmoveto{\pgfpoint{119.519989pt}{146.349472pt}}
\pgflineto{\pgfpoint{128.447998pt}{140.172638pt}}
\pgflineto{\pgfpoint{119.519989pt}{140.172638pt}}
\pgfpathclose
\pgfusepath{fill,stroke}
\pgfpathmoveto{\pgfpoint{119.519989pt}{146.349472pt}}
\pgflineto{\pgfpoint{128.447998pt}{146.349472pt}}
\pgflineto{\pgfpoint{128.447998pt}{140.172638pt}}
\pgfpathclose
\pgfusepath{fill,stroke}
\pgfpathmoveto{\pgfpoint{119.519989pt}{152.526306pt}}
\pgflineto{\pgfpoint{128.447998pt}{146.349472pt}}
\pgflineto{\pgfpoint{119.519989pt}{146.349472pt}}
\pgfpathclose
\pgfusepath{fill,stroke}
\pgfpathmoveto{\pgfpoint{119.519989pt}{152.526306pt}}
\pgflineto{\pgfpoint{128.447998pt}{152.526306pt}}
\pgflineto{\pgfpoint{128.447998pt}{146.349472pt}}
\pgfpathclose
\pgfusepath{fill,stroke}
\color[rgb]{0.205079,0.375366,0.553493}
\pgfpathmoveto{\pgfpoint{119.519989pt}{158.703156pt}}
\pgflineto{\pgfpoint{128.447998pt}{152.526306pt}}
\pgflineto{\pgfpoint{119.519989pt}{152.526306pt}}
\pgfpathclose
\pgfusepath{fill,stroke}
\pgfpathmoveto{\pgfpoint{119.519989pt}{158.703156pt}}
\pgflineto{\pgfpoint{128.447998pt}{158.703156pt}}
\pgflineto{\pgfpoint{128.447998pt}{152.526306pt}}
\pgfpathclose
\pgfusepath{fill,stroke}
\color[rgb]{0.197722,0.391341,0.554953}
\pgfpathmoveto{\pgfpoint{119.519989pt}{164.880005pt}}
\pgflineto{\pgfpoint{128.447998pt}{158.703156pt}}
\pgflineto{\pgfpoint{119.519989pt}{158.703156pt}}
\pgfpathclose
\pgfusepath{fill,stroke}
\pgfpathmoveto{\pgfpoint{119.519989pt}{164.880005pt}}
\pgflineto{\pgfpoint{128.447998pt}{164.880005pt}}
\pgflineto{\pgfpoint{128.447998pt}{158.703156pt}}
\pgfpathclose
\pgfusepath{fill,stroke}
\color[rgb]{0.190631,0.407061,0.556089}
\pgfpathmoveto{\pgfpoint{119.519989pt}{171.056854pt}}
\pgflineto{\pgfpoint{128.447998pt}{164.880005pt}}
\pgflineto{\pgfpoint{119.519989pt}{164.880005pt}}
\pgfpathclose
\pgfusepath{fill,stroke}
\pgfpathmoveto{\pgfpoint{119.519989pt}{171.056854pt}}
\pgflineto{\pgfpoint{128.447998pt}{171.056854pt}}
\pgflineto{\pgfpoint{128.447998pt}{164.880005pt}}
\pgfpathclose
\pgfusepath{fill,stroke}
\color[rgb]{0.183819,0.422564,0.556952}
\pgfpathmoveto{\pgfpoint{119.519989pt}{177.233673pt}}
\pgflineto{\pgfpoint{128.447998pt}{171.056854pt}}
\pgflineto{\pgfpoint{119.519989pt}{171.056854pt}}
\pgfpathclose
\pgfusepath{fill,stroke}
\pgfpathmoveto{\pgfpoint{119.519989pt}{177.233673pt}}
\pgflineto{\pgfpoint{128.447998pt}{177.233673pt}}
\pgflineto{\pgfpoint{128.447998pt}{171.056854pt}}
\pgfpathclose
\pgfusepath{fill,stroke}
\color[rgb]{0.177272,0.437886,0.557576}
\pgfpathmoveto{\pgfpoint{119.519989pt}{183.410522pt}}
\pgflineto{\pgfpoint{128.447998pt}{177.233673pt}}
\pgflineto{\pgfpoint{119.519989pt}{177.233673pt}}
\pgfpathclose
\pgfusepath{fill,stroke}
\pgfpathmoveto{\pgfpoint{119.519989pt}{183.410522pt}}
\pgflineto{\pgfpoint{128.447998pt}{183.410522pt}}
\pgflineto{\pgfpoint{128.447998pt}{177.233673pt}}
\pgfpathclose
\pgfusepath{fill,stroke}
\pgfpathmoveto{\pgfpoint{119.519989pt}{189.587372pt}}
\pgflineto{\pgfpoint{128.447998pt}{183.410522pt}}
\pgflineto{\pgfpoint{119.519989pt}{183.410522pt}}
\pgfpathclose
\pgfusepath{fill,stroke}
\pgfpathmoveto{\pgfpoint{119.519989pt}{189.587372pt}}
\pgflineto{\pgfpoint{128.447998pt}{189.587372pt}}
\pgflineto{\pgfpoint{128.447998pt}{183.410522pt}}
\pgfpathclose
\pgfusepath{fill,stroke}
\color[rgb]{0.212667,0.359102,0.551635}
\pgfpathmoveto{\pgfpoint{128.447998pt}{140.172638pt}}
\pgflineto{\pgfpoint{137.376007pt}{133.995789pt}}
\pgflineto{\pgfpoint{128.447998pt}{133.995789pt}}
\pgfpathclose
\pgfusepath{fill,stroke}
\pgfpathmoveto{\pgfpoint{128.447998pt}{140.172638pt}}
\pgflineto{\pgfpoint{137.376007pt}{140.172638pt}}
\pgflineto{\pgfpoint{137.376007pt}{133.995789pt}}
\pgfpathclose
\pgfusepath{fill,stroke}
\color[rgb]{0.205079,0.375366,0.553493}
\pgfpathmoveto{\pgfpoint{128.447998pt}{146.349472pt}}
\pgflineto{\pgfpoint{137.376007pt}{140.172638pt}}
\pgflineto{\pgfpoint{128.447998pt}{140.172638pt}}
\pgfpathclose
\pgfusepath{fill,stroke}
\pgfpathmoveto{\pgfpoint{128.447998pt}{146.349472pt}}
\pgflineto{\pgfpoint{137.376007pt}{146.349472pt}}
\pgflineto{\pgfpoint{137.376007pt}{140.172638pt}}
\pgfpathclose
\pgfusepath{fill,stroke}
\pgfpathmoveto{\pgfpoint{128.447998pt}{152.526306pt}}
\pgflineto{\pgfpoint{137.376007pt}{146.349472pt}}
\pgflineto{\pgfpoint{128.447998pt}{146.349472pt}}
\pgfpathclose
\pgfusepath{fill,stroke}
\pgfpathmoveto{\pgfpoint{128.447998pt}{152.526306pt}}
\pgflineto{\pgfpoint{137.376007pt}{152.526306pt}}
\pgflineto{\pgfpoint{137.376007pt}{146.349472pt}}
\pgfpathclose
\pgfusepath{fill,stroke}
\color[rgb]{0.197722,0.391341,0.554953}
\pgfpathmoveto{\pgfpoint{128.447998pt}{158.703156pt}}
\pgflineto{\pgfpoint{137.376007pt}{152.526306pt}}
\pgflineto{\pgfpoint{128.447998pt}{152.526306pt}}
\pgfpathclose
\pgfusepath{fill,stroke}
\pgfpathmoveto{\pgfpoint{128.447998pt}{158.703156pt}}
\pgflineto{\pgfpoint{137.376007pt}{158.703156pt}}
\pgflineto{\pgfpoint{137.376007pt}{152.526306pt}}
\pgfpathclose
\pgfusepath{fill,stroke}
\color[rgb]{0.190631,0.407061,0.556089}
\pgfpathmoveto{\pgfpoint{128.447998pt}{164.880005pt}}
\pgflineto{\pgfpoint{137.376007pt}{158.703156pt}}
\pgflineto{\pgfpoint{128.447998pt}{158.703156pt}}
\pgfpathclose
\pgfusepath{fill,stroke}
\pgfpathmoveto{\pgfpoint{128.447998pt}{164.880005pt}}
\pgflineto{\pgfpoint{137.376007pt}{164.880005pt}}
\pgflineto{\pgfpoint{137.376007pt}{158.703156pt}}
\pgfpathclose
\pgfusepath{fill,stroke}
\color[rgb]{0.183819,0.422564,0.556952}
\pgfpathmoveto{\pgfpoint{128.447998pt}{171.056854pt}}
\pgflineto{\pgfpoint{137.376007pt}{164.880005pt}}
\pgflineto{\pgfpoint{128.447998pt}{164.880005pt}}
\pgfpathclose
\pgfusepath{fill,stroke}
\pgfpathmoveto{\pgfpoint{128.447998pt}{171.056854pt}}
\pgflineto{\pgfpoint{137.376007pt}{171.056854pt}}
\pgflineto{\pgfpoint{137.376007pt}{164.880005pt}}
\pgfpathclose
\pgfusepath{fill,stroke}
\color[rgb]{0.177272,0.437886,0.557576}
\pgfpathmoveto{\pgfpoint{128.447998pt}{177.233673pt}}
\pgflineto{\pgfpoint{137.376007pt}{171.056854pt}}
\pgflineto{\pgfpoint{128.447998pt}{171.056854pt}}
\pgfpathclose
\pgfusepath{fill,stroke}
\pgfpathmoveto{\pgfpoint{128.447998pt}{177.233673pt}}
\pgflineto{\pgfpoint{137.376007pt}{177.233673pt}}
\pgflineto{\pgfpoint{137.376007pt}{171.056854pt}}
\pgfpathclose
\pgfusepath{fill,stroke}
\color[rgb]{0.170958,0.453063,0.557974}
\pgfpathmoveto{\pgfpoint{128.447998pt}{183.410522pt}}
\pgflineto{\pgfpoint{137.376007pt}{177.233673pt}}
\pgflineto{\pgfpoint{128.447998pt}{177.233673pt}}
\pgfpathclose
\pgfusepath{fill,stroke}
\pgfpathmoveto{\pgfpoint{128.447998pt}{183.410522pt}}
\pgflineto{\pgfpoint{137.376007pt}{183.410522pt}}
\pgflineto{\pgfpoint{137.376007pt}{177.233673pt}}
\pgfpathclose
\pgfusepath{fill,stroke}
\pgfpathmoveto{\pgfpoint{128.447998pt}{189.587372pt}}
\pgflineto{\pgfpoint{137.376007pt}{183.410522pt}}
\pgflineto{\pgfpoint{128.447998pt}{183.410522pt}}
\pgfpathclose
\pgfusepath{fill,stroke}
\pgfpathmoveto{\pgfpoint{128.447998pt}{189.587372pt}}
\pgflineto{\pgfpoint{137.376007pt}{189.587372pt}}
\pgflineto{\pgfpoint{137.376007pt}{183.410522pt}}
\pgfpathclose
\pgfusepath{fill,stroke}
\color[rgb]{0.164833,0.468130,0.558143}
\pgfpathmoveto{\pgfpoint{128.447998pt}{195.764206pt}}
\pgflineto{\pgfpoint{137.376007pt}{189.587372pt}}
\pgflineto{\pgfpoint{128.447998pt}{189.587372pt}}
\pgfpathclose
\pgfusepath{fill,stroke}
\pgfpathmoveto{\pgfpoint{128.447998pt}{195.764206pt}}
\pgflineto{\pgfpoint{137.376007pt}{195.764206pt}}
\pgflineto{\pgfpoint{137.376007pt}{189.587372pt}}
\pgfpathclose
\pgfusepath{fill,stroke}
\color[rgb]{0.158845,0.483117,0.558059}
\pgfpathmoveto{\pgfpoint{128.447998pt}{201.941055pt}}
\pgflineto{\pgfpoint{137.376007pt}{195.764206pt}}
\pgflineto{\pgfpoint{128.447998pt}{195.764206pt}}
\pgfpathclose
\pgfusepath{fill,stroke}
\pgfpathmoveto{\pgfpoint{128.447998pt}{201.941055pt}}
\pgflineto{\pgfpoint{137.376007pt}{201.941055pt}}
\pgflineto{\pgfpoint{137.376007pt}{195.764206pt}}
\pgfpathclose
\pgfusepath{fill,stroke}
\color[rgb]{0.190631,0.407061,0.556089}
\pgfpathmoveto{\pgfpoint{137.376007pt}{152.526306pt}}
\pgflineto{\pgfpoint{146.303986pt}{146.349472pt}}
\pgflineto{\pgfpoint{137.376007pt}{146.349472pt}}
\pgfpathclose
\pgfusepath{fill,stroke}
\pgfpathmoveto{\pgfpoint{137.376007pt}{152.526306pt}}
\pgflineto{\pgfpoint{146.303986pt}{152.526306pt}}
\pgflineto{\pgfpoint{146.303986pt}{146.349472pt}}
\pgfpathclose
\pgfusepath{fill,stroke}
\pgfpathmoveto{\pgfpoint{137.376007pt}{158.703156pt}}
\pgflineto{\pgfpoint{146.303986pt}{152.526306pt}}
\pgflineto{\pgfpoint{137.376007pt}{152.526306pt}}
\pgfpathclose
\pgfusepath{fill,stroke}
\pgfpathmoveto{\pgfpoint{137.376007pt}{158.703156pt}}
\pgflineto{\pgfpoint{146.303986pt}{158.703156pt}}
\pgflineto{\pgfpoint{146.303986pt}{152.526306pt}}
\pgfpathclose
\pgfusepath{fill,stroke}
\color[rgb]{0.183819,0.422564,0.556952}
\pgfpathmoveto{\pgfpoint{137.376007pt}{164.880005pt}}
\pgflineto{\pgfpoint{146.303986pt}{158.703156pt}}
\pgflineto{\pgfpoint{137.376007pt}{158.703156pt}}
\pgfpathclose
\pgfusepath{fill,stroke}
\pgfpathmoveto{\pgfpoint{137.376007pt}{164.880005pt}}
\pgflineto{\pgfpoint{146.303986pt}{164.880005pt}}
\pgflineto{\pgfpoint{146.303986pt}{158.703156pt}}
\pgfpathclose
\pgfusepath{fill,stroke}
\color[rgb]{0.177272,0.437886,0.557576}
\pgfpathmoveto{\pgfpoint{137.376007pt}{171.056854pt}}
\pgflineto{\pgfpoint{146.303986pt}{164.880005pt}}
\pgflineto{\pgfpoint{137.376007pt}{164.880005pt}}
\pgfpathclose
\pgfusepath{fill,stroke}
\pgfpathmoveto{\pgfpoint{137.376007pt}{171.056854pt}}
\pgflineto{\pgfpoint{146.303986pt}{171.056854pt}}
\pgflineto{\pgfpoint{146.303986pt}{164.880005pt}}
\pgfpathclose
\pgfusepath{fill,stroke}
\color[rgb]{0.170958,0.453063,0.557974}
\pgfpathmoveto{\pgfpoint{137.376007pt}{177.233673pt}}
\pgflineto{\pgfpoint{146.303986pt}{171.056854pt}}
\pgflineto{\pgfpoint{137.376007pt}{171.056854pt}}
\pgfpathclose
\pgfusepath{fill,stroke}
\pgfpathmoveto{\pgfpoint{137.376007pt}{177.233673pt}}
\pgflineto{\pgfpoint{146.303986pt}{177.233673pt}}
\pgflineto{\pgfpoint{146.303986pt}{171.056854pt}}
\pgfpathclose
\pgfusepath{fill,stroke}
\color[rgb]{0.164833,0.468130,0.558143}
\pgfpathmoveto{\pgfpoint{137.376007pt}{183.410522pt}}
\pgflineto{\pgfpoint{146.303986pt}{177.233673pt}}
\pgflineto{\pgfpoint{137.376007pt}{177.233673pt}}
\pgfpathclose
\pgfusepath{fill,stroke}
\pgfpathmoveto{\pgfpoint{137.376007pt}{183.410522pt}}
\pgflineto{\pgfpoint{146.303986pt}{183.410522pt}}
\pgflineto{\pgfpoint{146.303986pt}{177.233673pt}}
\pgfpathclose
\pgfusepath{fill,stroke}
\color[rgb]{0.158845,0.483117,0.558059}
\pgfpathmoveto{\pgfpoint{137.376007pt}{189.587372pt}}
\pgflineto{\pgfpoint{146.303986pt}{183.410522pt}}
\pgflineto{\pgfpoint{137.376007pt}{183.410522pt}}
\pgfpathclose
\pgfusepath{fill,stroke}
\pgfpathmoveto{\pgfpoint{137.376007pt}{189.587372pt}}
\pgflineto{\pgfpoint{146.303986pt}{189.587372pt}}
\pgflineto{\pgfpoint{146.303986pt}{183.410522pt}}
\pgfpathclose
\pgfusepath{fill,stroke}
\pgfpathmoveto{\pgfpoint{137.376007pt}{195.764206pt}}
\pgflineto{\pgfpoint{146.303986pt}{189.587372pt}}
\pgflineto{\pgfpoint{137.376007pt}{189.587372pt}}
\pgfpathclose
\pgfusepath{fill,stroke}
\pgfpathmoveto{\pgfpoint{137.376007pt}{195.764206pt}}
\pgflineto{\pgfpoint{146.303986pt}{195.764206pt}}
\pgflineto{\pgfpoint{146.303986pt}{189.587372pt}}
\pgfpathclose
\pgfusepath{fill,stroke}
\color[rgb]{0.152951,0.498053,0.557685}
\pgfpathmoveto{\pgfpoint{137.376007pt}{201.941055pt}}
\pgflineto{\pgfpoint{146.303986pt}{195.764206pt}}
\pgflineto{\pgfpoint{137.376007pt}{195.764206pt}}
\pgfpathclose
\pgfusepath{fill,stroke}
\pgfpathmoveto{\pgfpoint{137.376007pt}{201.941055pt}}
\pgflineto{\pgfpoint{146.303986pt}{201.941055pt}}
\pgflineto{\pgfpoint{146.303986pt}{195.764206pt}}
\pgfpathclose
\pgfusepath{fill,stroke}
\color[rgb]{0.147132,0.512959,0.556973}
\pgfpathmoveto{\pgfpoint{137.376007pt}{208.117905pt}}
\pgflineto{\pgfpoint{146.303986pt}{201.941055pt}}
\pgflineto{\pgfpoint{137.376007pt}{201.941055pt}}
\pgfpathclose
\pgfusepath{fill,stroke}
\pgfpathmoveto{\pgfpoint{137.376007pt}{208.117905pt}}
\pgflineto{\pgfpoint{146.303986pt}{208.117905pt}}
\pgflineto{\pgfpoint{146.303986pt}{201.941055pt}}
\pgfpathclose
\pgfusepath{fill,stroke}
\color[rgb]{0.141402,0.527854,0.555864}
\pgfpathmoveto{\pgfpoint{137.376007pt}{214.294739pt}}
\pgflineto{\pgfpoint{146.303986pt}{208.117905pt}}
\pgflineto{\pgfpoint{137.376007pt}{208.117905pt}}
\pgfpathclose
\pgfusepath{fill,stroke}
\pgfpathmoveto{\pgfpoint{137.376007pt}{214.294739pt}}
\pgflineto{\pgfpoint{146.303986pt}{214.294739pt}}
\pgflineto{\pgfpoint{146.303986pt}{208.117905pt}}
\pgfpathclose
\pgfusepath{fill,stroke}
\color[rgb]{0.177272,0.437886,0.557576}
\pgfpathmoveto{\pgfpoint{146.303986pt}{164.880005pt}}
\pgflineto{\pgfpoint{155.231979pt}{158.703156pt}}
\pgflineto{\pgfpoint{146.303986pt}{158.703156pt}}
\pgfpathclose
\pgfusepath{fill,stroke}
\pgfpathmoveto{\pgfpoint{146.303986pt}{164.880005pt}}
\pgflineto{\pgfpoint{155.231979pt}{164.880005pt}}
\pgflineto{\pgfpoint{155.231979pt}{158.703156pt}}
\pgfpathclose
\pgfusepath{fill,stroke}
\color[rgb]{0.170958,0.453063,0.557974}
\pgfpathmoveto{\pgfpoint{146.303986pt}{171.056854pt}}
\pgflineto{\pgfpoint{155.231979pt}{164.880005pt}}
\pgflineto{\pgfpoint{146.303986pt}{164.880005pt}}
\pgfpathclose
\pgfusepath{fill,stroke}
\pgfpathmoveto{\pgfpoint{146.303986pt}{171.056854pt}}
\pgflineto{\pgfpoint{155.231979pt}{171.056854pt}}
\pgflineto{\pgfpoint{155.231979pt}{164.880005pt}}
\pgfpathclose
\pgfusepath{fill,stroke}
\color[rgb]{0.164833,0.468130,0.558143}
\pgfpathmoveto{\pgfpoint{146.303986pt}{177.233673pt}}
\pgflineto{\pgfpoint{155.231979pt}{171.056854pt}}
\pgflineto{\pgfpoint{146.303986pt}{171.056854pt}}
\pgfpathclose
\pgfusepath{fill,stroke}
\pgfpathmoveto{\pgfpoint{146.303986pt}{177.233673pt}}
\pgflineto{\pgfpoint{155.231979pt}{177.233673pt}}
\pgflineto{\pgfpoint{155.231979pt}{171.056854pt}}
\pgfpathclose
\pgfusepath{fill,stroke}
\color[rgb]{0.158845,0.483117,0.558059}
\pgfpathmoveto{\pgfpoint{146.303986pt}{183.410522pt}}
\pgflineto{\pgfpoint{155.231979pt}{177.233673pt}}
\pgflineto{\pgfpoint{146.303986pt}{177.233673pt}}
\pgfpathclose
\pgfusepath{fill,stroke}
\pgfpathmoveto{\pgfpoint{146.303986pt}{183.410522pt}}
\pgflineto{\pgfpoint{155.231979pt}{183.410522pt}}
\pgflineto{\pgfpoint{155.231979pt}{177.233673pt}}
\pgfpathclose
\pgfusepath{fill,stroke}
\color[rgb]{0.152951,0.498053,0.557685}
\pgfpathmoveto{\pgfpoint{146.303986pt}{189.587372pt}}
\pgflineto{\pgfpoint{155.231979pt}{183.410522pt}}
\pgflineto{\pgfpoint{146.303986pt}{183.410522pt}}
\pgfpathclose
\pgfusepath{fill,stroke}
\pgfpathmoveto{\pgfpoint{146.303986pt}{189.587372pt}}
\pgflineto{\pgfpoint{155.231979pt}{189.587372pt}}
\pgflineto{\pgfpoint{155.231979pt}{183.410522pt}}
\pgfpathclose
\pgfusepath{fill,stroke}
\pgfpathmoveto{\pgfpoint{146.303986pt}{195.764206pt}}
\pgflineto{\pgfpoint{155.231979pt}{189.587372pt}}
\pgflineto{\pgfpoint{146.303986pt}{189.587372pt}}
\pgfpathclose
\pgfusepath{fill,stroke}
\pgfpathmoveto{\pgfpoint{146.303986pt}{195.764206pt}}
\pgflineto{\pgfpoint{155.231979pt}{195.764206pt}}
\pgflineto{\pgfpoint{155.231979pt}{189.587372pt}}
\pgfpathclose
\pgfusepath{fill,stroke}
\color[rgb]{0.147132,0.512959,0.556973}
\pgfpathmoveto{\pgfpoint{146.303986pt}{201.941055pt}}
\pgflineto{\pgfpoint{155.231979pt}{195.764206pt}}
\pgflineto{\pgfpoint{146.303986pt}{195.764206pt}}
\pgfpathclose
\pgfusepath{fill,stroke}
\pgfpathmoveto{\pgfpoint{146.303986pt}{201.941055pt}}
\pgflineto{\pgfpoint{155.231979pt}{201.941055pt}}
\pgflineto{\pgfpoint{155.231979pt}{195.764206pt}}
\pgfpathclose
\pgfusepath{fill,stroke}
\color[rgb]{0.141402,0.527854,0.555864}
\pgfpathmoveto{\pgfpoint{146.303986pt}{208.117905pt}}
\pgflineto{\pgfpoint{155.231979pt}{201.941055pt}}
\pgflineto{\pgfpoint{146.303986pt}{201.941055pt}}
\pgfpathclose
\pgfusepath{fill,stroke}
\pgfpathmoveto{\pgfpoint{146.303986pt}{208.117905pt}}
\pgflineto{\pgfpoint{155.231979pt}{208.117905pt}}
\pgflineto{\pgfpoint{155.231979pt}{201.941055pt}}
\pgfpathclose
\pgfusepath{fill,stroke}
\color[rgb]{0.135833,0.542750,0.554289}
\pgfpathmoveto{\pgfpoint{146.303986pt}{214.294739pt}}
\pgflineto{\pgfpoint{155.231979pt}{208.117905pt}}
\pgflineto{\pgfpoint{146.303986pt}{208.117905pt}}
\pgfpathclose
\pgfusepath{fill,stroke}
\pgfpathmoveto{\pgfpoint{146.303986pt}{214.294739pt}}
\pgflineto{\pgfpoint{155.231979pt}{214.294739pt}}
\pgflineto{\pgfpoint{155.231979pt}{208.117905pt}}
\pgfpathclose
\pgfusepath{fill,stroke}
\color[rgb]{0.130582,0.557652,0.552176}
\pgfpathmoveto{\pgfpoint{146.303986pt}{220.471588pt}}
\pgflineto{\pgfpoint{155.231979pt}{214.294739pt}}
\pgflineto{\pgfpoint{146.303986pt}{214.294739pt}}
\pgfpathclose
\pgfusepath{fill,stroke}
\pgfpathmoveto{\pgfpoint{146.303986pt}{220.471588pt}}
\pgflineto{\pgfpoint{155.231979pt}{220.471588pt}}
\pgflineto{\pgfpoint{155.231979pt}{214.294739pt}}
\pgfpathclose
\pgfusepath{fill,stroke}
\color[rgb]{0.125898,0.572563,0.549445}
\pgfpathmoveto{\pgfpoint{146.303986pt}{226.648422pt}}
\pgflineto{\pgfpoint{155.231979pt}{220.471588pt}}
\pgflineto{\pgfpoint{146.303986pt}{220.471588pt}}
\pgfpathclose
\pgfusepath{fill,stroke}
\pgfpathmoveto{\pgfpoint{146.303986pt}{226.648422pt}}
\pgflineto{\pgfpoint{155.231979pt}{226.648422pt}}
\pgflineto{\pgfpoint{155.231979pt}{220.471588pt}}
\pgfpathclose
\pgfusepath{fill,stroke}
\color[rgb]{0.158845,0.483117,0.558059}
\pgfpathmoveto{\pgfpoint{155.231979pt}{177.233673pt}}
\pgflineto{\pgfpoint{164.160004pt}{171.056854pt}}
\pgflineto{\pgfpoint{155.231979pt}{171.056854pt}}
\pgfpathclose
\pgfusepath{fill,stroke}
\pgfpathmoveto{\pgfpoint{155.231979pt}{177.233673pt}}
\pgflineto{\pgfpoint{164.160004pt}{177.233673pt}}
\pgflineto{\pgfpoint{164.160004pt}{171.056854pt}}
\pgfpathclose
\pgfusepath{fill,stroke}
\color[rgb]{0.152951,0.498053,0.557685}
\pgfpathmoveto{\pgfpoint{155.231979pt}{183.410522pt}}
\pgflineto{\pgfpoint{164.160004pt}{177.233673pt}}
\pgflineto{\pgfpoint{155.231979pt}{177.233673pt}}
\pgfpathclose
\pgfusepath{fill,stroke}
\pgfpathmoveto{\pgfpoint{155.231979pt}{183.410522pt}}
\pgflineto{\pgfpoint{164.160004pt}{183.410522pt}}
\pgflineto{\pgfpoint{164.160004pt}{177.233673pt}}
\pgfpathclose
\pgfusepath{fill,stroke}
\color[rgb]{0.147132,0.512959,0.556973}
\pgfpathmoveto{\pgfpoint{155.231979pt}{189.587372pt}}
\pgflineto{\pgfpoint{164.160004pt}{183.410522pt}}
\pgflineto{\pgfpoint{155.231979pt}{183.410522pt}}
\pgfpathclose
\pgfusepath{fill,stroke}
\pgfpathmoveto{\pgfpoint{155.231979pt}{189.587372pt}}
\pgflineto{\pgfpoint{164.160004pt}{189.587372pt}}
\pgflineto{\pgfpoint{164.160004pt}{183.410522pt}}
\pgfpathclose
\pgfusepath{fill,stroke}
\color[rgb]{0.141402,0.527854,0.555864}
\pgfpathmoveto{\pgfpoint{155.231979pt}{195.764206pt}}
\pgflineto{\pgfpoint{164.160004pt}{189.587372pt}}
\pgflineto{\pgfpoint{155.231979pt}{189.587372pt}}
\pgfpathclose
\pgfusepath{fill,stroke}
\pgfpathmoveto{\pgfpoint{155.231979pt}{195.764206pt}}
\pgflineto{\pgfpoint{164.160004pt}{195.764206pt}}
\pgflineto{\pgfpoint{164.160004pt}{189.587372pt}}
\pgfpathclose
\pgfusepath{fill,stroke}
\pgfpathmoveto{\pgfpoint{155.231979pt}{201.941055pt}}
\pgflineto{\pgfpoint{164.160004pt}{195.764206pt}}
\pgflineto{\pgfpoint{155.231979pt}{195.764206pt}}
\pgfpathclose
\pgfusepath{fill,stroke}
\pgfpathmoveto{\pgfpoint{155.231979pt}{201.941055pt}}
\pgflineto{\pgfpoint{164.160004pt}{201.941055pt}}
\pgflineto{\pgfpoint{164.160004pt}{195.764206pt}}
\pgfpathclose
\pgfusepath{fill,stroke}
\color[rgb]{0.135833,0.542750,0.554289}
\pgfpathmoveto{\pgfpoint{155.231979pt}{208.117905pt}}
\pgflineto{\pgfpoint{164.160004pt}{201.941055pt}}
\pgflineto{\pgfpoint{155.231979pt}{201.941055pt}}
\pgfpathclose
\pgfusepath{fill,stroke}
\pgfpathmoveto{\pgfpoint{155.231979pt}{208.117905pt}}
\pgflineto{\pgfpoint{164.160004pt}{208.117905pt}}
\pgflineto{\pgfpoint{164.160004pt}{201.941055pt}}
\pgfpathclose
\pgfusepath{fill,stroke}
\color[rgb]{0.130582,0.557652,0.552176}
\pgfpathmoveto{\pgfpoint{155.231979pt}{214.294739pt}}
\pgflineto{\pgfpoint{164.160004pt}{208.117905pt}}
\pgflineto{\pgfpoint{155.231979pt}{208.117905pt}}
\pgfpathclose
\pgfusepath{fill,stroke}
\pgfpathmoveto{\pgfpoint{155.231979pt}{214.294739pt}}
\pgflineto{\pgfpoint{164.160004pt}{214.294739pt}}
\pgflineto{\pgfpoint{164.160004pt}{208.117905pt}}
\pgfpathclose
\pgfusepath{fill,stroke}
\color[rgb]{0.125898,0.572563,0.549445}
\pgfpathmoveto{\pgfpoint{155.231979pt}{220.471588pt}}
\pgflineto{\pgfpoint{164.160004pt}{214.294739pt}}
\pgflineto{\pgfpoint{155.231979pt}{214.294739pt}}
\pgfpathclose
\pgfusepath{fill,stroke}
\pgfpathmoveto{\pgfpoint{155.231979pt}{220.471588pt}}
\pgflineto{\pgfpoint{164.160004pt}{220.471588pt}}
\pgflineto{\pgfpoint{164.160004pt}{214.294739pt}}
\pgfpathclose
\pgfusepath{fill,stroke}
\color[rgb]{0.122163,0.587476,0.546023}
\pgfpathmoveto{\pgfpoint{155.231979pt}{226.648422pt}}
\pgflineto{\pgfpoint{164.160004pt}{220.471588pt}}
\pgflineto{\pgfpoint{155.231979pt}{220.471588pt}}
\pgfpathclose
\pgfusepath{fill,stroke}
\pgfpathmoveto{\pgfpoint{155.231979pt}{226.648422pt}}
\pgflineto{\pgfpoint{164.160004pt}{226.648422pt}}
\pgflineto{\pgfpoint{164.160004pt}{220.471588pt}}
\pgfpathclose
\pgfusepath{fill,stroke}
\color[rgb]{0.119872,0.602382,0.541831}
\pgfpathmoveto{\pgfpoint{155.231979pt}{232.825272pt}}
\pgflineto{\pgfpoint{164.160004pt}{226.648422pt}}
\pgflineto{\pgfpoint{155.231979pt}{226.648422pt}}
\pgfpathclose
\pgfusepath{fill,stroke}
\pgfpathmoveto{\pgfpoint{155.231979pt}{232.825272pt}}
\pgflineto{\pgfpoint{164.160004pt}{232.825272pt}}
\pgflineto{\pgfpoint{164.160004pt}{226.648422pt}}
\pgfpathclose
\pgfusepath{fill,stroke}
\pgfpathmoveto{\pgfpoint{155.231979pt}{239.002106pt}}
\pgflineto{\pgfpoint{164.160004pt}{232.825272pt}}
\pgflineto{\pgfpoint{155.231979pt}{232.825272pt}}
\pgfpathclose
\pgfusepath{fill,stroke}
\pgfpathmoveto{\pgfpoint{155.231979pt}{239.002106pt}}
\pgflineto{\pgfpoint{164.160004pt}{239.002106pt}}
\pgflineto{\pgfpoint{164.160004pt}{232.825272pt}}
\pgfpathclose
\pgfusepath{fill,stroke}
\color[rgb]{0.119627,0.617266,0.536796}
\pgfpathmoveto{\pgfpoint{155.231979pt}{245.178955pt}}
\pgflineto{\pgfpoint{164.160004pt}{239.002106pt}}
\pgflineto{\pgfpoint{155.231979pt}{239.002106pt}}
\pgfpathclose
\pgfusepath{fill,stroke}
\pgfpathmoveto{\pgfpoint{155.231979pt}{245.178955pt}}
\pgflineto{\pgfpoint{164.160004pt}{245.178955pt}}
\pgflineto{\pgfpoint{164.160004pt}{239.002106pt}}
\pgfpathclose
\pgfusepath{fill,stroke}
\color[rgb]{0.141402,0.527854,0.555864}
\pgfpathmoveto{\pgfpoint{164.160004pt}{189.587372pt}}
\pgflineto{\pgfpoint{173.087997pt}{183.410522pt}}
\pgflineto{\pgfpoint{164.160004pt}{183.410522pt}}
\pgfpathclose
\pgfusepath{fill,stroke}
\pgfpathmoveto{\pgfpoint{164.160004pt}{189.587372pt}}
\pgflineto{\pgfpoint{173.087997pt}{189.587372pt}}
\pgflineto{\pgfpoint{173.087997pt}{183.410522pt}}
\pgfpathclose
\pgfusepath{fill,stroke}
\color[rgb]{0.135833,0.542750,0.554289}
\pgfpathmoveto{\pgfpoint{164.160004pt}{195.764206pt}}
\pgflineto{\pgfpoint{173.087997pt}{189.587372pt}}
\pgflineto{\pgfpoint{164.160004pt}{189.587372pt}}
\pgfpathclose
\pgfusepath{fill,stroke}
\pgfpathmoveto{\pgfpoint{164.160004pt}{195.764206pt}}
\pgflineto{\pgfpoint{173.087997pt}{195.764206pt}}
\pgflineto{\pgfpoint{173.087997pt}{189.587372pt}}
\pgfpathclose
\pgfusepath{fill,stroke}
\pgfpathmoveto{\pgfpoint{164.160004pt}{201.941055pt}}
\pgflineto{\pgfpoint{173.087997pt}{195.764206pt}}
\pgflineto{\pgfpoint{164.160004pt}{195.764206pt}}
\pgfpathclose
\pgfusepath{fill,stroke}
\pgfpathmoveto{\pgfpoint{164.160004pt}{201.941055pt}}
\pgflineto{\pgfpoint{173.087997pt}{201.941055pt}}
\pgflineto{\pgfpoint{173.087997pt}{195.764206pt}}
\pgfpathclose
\pgfusepath{fill,stroke}
\color[rgb]{0.130582,0.557652,0.552176}
\pgfpathmoveto{\pgfpoint{164.160004pt}{208.117905pt}}
\pgflineto{\pgfpoint{173.087997pt}{201.941055pt}}
\pgflineto{\pgfpoint{164.160004pt}{201.941055pt}}
\pgfpathclose
\pgfusepath{fill,stroke}
\pgfpathmoveto{\pgfpoint{164.160004pt}{208.117905pt}}
\pgflineto{\pgfpoint{173.087997pt}{208.117905pt}}
\pgflineto{\pgfpoint{173.087997pt}{201.941055pt}}
\pgfpathclose
\pgfusepath{fill,stroke}
\color[rgb]{0.125898,0.572563,0.549445}
\pgfpathmoveto{\pgfpoint{164.160004pt}{214.294739pt}}
\pgflineto{\pgfpoint{173.087997pt}{208.117905pt}}
\pgflineto{\pgfpoint{164.160004pt}{208.117905pt}}
\pgfpathclose
\pgfusepath{fill,stroke}
\pgfpathmoveto{\pgfpoint{164.160004pt}{214.294739pt}}
\pgflineto{\pgfpoint{173.087997pt}{214.294739pt}}
\pgflineto{\pgfpoint{173.087997pt}{208.117905pt}}
\pgfpathclose
\pgfusepath{fill,stroke}
\color[rgb]{0.122163,0.587476,0.546023}
\pgfpathmoveto{\pgfpoint{164.160004pt}{220.471588pt}}
\pgflineto{\pgfpoint{173.087997pt}{214.294739pt}}
\pgflineto{\pgfpoint{164.160004pt}{214.294739pt}}
\pgfpathclose
\pgfusepath{fill,stroke}
\pgfpathmoveto{\pgfpoint{164.160004pt}{220.471588pt}}
\pgflineto{\pgfpoint{173.087997pt}{220.471588pt}}
\pgflineto{\pgfpoint{173.087997pt}{214.294739pt}}
\pgfpathclose
\pgfusepath{fill,stroke}
\color[rgb]{0.119872,0.602382,0.541831}
\pgfpathmoveto{\pgfpoint{164.160004pt}{226.648422pt}}
\pgflineto{\pgfpoint{173.087997pt}{220.471588pt}}
\pgflineto{\pgfpoint{164.160004pt}{220.471588pt}}
\pgfpathclose
\pgfusepath{fill,stroke}
\pgfpathmoveto{\pgfpoint{164.160004pt}{226.648422pt}}
\pgflineto{\pgfpoint{173.087997pt}{226.648422pt}}
\pgflineto{\pgfpoint{173.087997pt}{220.471588pt}}
\pgfpathclose
\pgfusepath{fill,stroke}
\color[rgb]{0.119627,0.617266,0.536796}
\pgfpathmoveto{\pgfpoint{164.160004pt}{232.825272pt}}
\pgflineto{\pgfpoint{173.087997pt}{226.648422pt}}
\pgflineto{\pgfpoint{164.160004pt}{226.648422pt}}
\pgfpathclose
\pgfusepath{fill,stroke}
\pgfpathmoveto{\pgfpoint{164.160004pt}{232.825272pt}}
\pgflineto{\pgfpoint{173.087997pt}{232.825272pt}}
\pgflineto{\pgfpoint{173.087997pt}{226.648422pt}}
\pgfpathclose
\pgfusepath{fill,stroke}
\color[rgb]{0.122046,0.632107,0.530848}
\pgfpathmoveto{\pgfpoint{164.160004pt}{239.002106pt}}
\pgflineto{\pgfpoint{173.087997pt}{232.825272pt}}
\pgflineto{\pgfpoint{164.160004pt}{232.825272pt}}
\pgfpathclose
\pgfusepath{fill,stroke}
\pgfpathmoveto{\pgfpoint{164.160004pt}{239.002106pt}}
\pgflineto{\pgfpoint{173.087997pt}{239.002106pt}}
\pgflineto{\pgfpoint{173.087997pt}{232.825272pt}}
\pgfpathclose
\pgfusepath{fill,stroke}
\pgfpathmoveto{\pgfpoint{164.160004pt}{245.178955pt}}
\pgflineto{\pgfpoint{173.087997pt}{239.002106pt}}
\pgflineto{\pgfpoint{164.160004pt}{239.002106pt}}
\pgfpathclose
\pgfusepath{fill,stroke}
\pgfpathmoveto{\pgfpoint{164.160004pt}{245.178955pt}}
\pgflineto{\pgfpoint{173.087997pt}{245.178955pt}}
\pgflineto{\pgfpoint{173.087997pt}{239.002106pt}}
\pgfpathclose
\pgfusepath{fill,stroke}
\color[rgb]{0.127668,0.646882,0.523924}
\pgfpathmoveto{\pgfpoint{164.160004pt}{251.355804pt}}
\pgflineto{\pgfpoint{173.087997pt}{245.178955pt}}
\pgflineto{\pgfpoint{164.160004pt}{245.178955pt}}
\pgfpathclose
\pgfusepath{fill,stroke}
\pgfpathmoveto{\pgfpoint{164.160004pt}{251.355804pt}}
\pgflineto{\pgfpoint{173.087997pt}{251.355804pt}}
\pgflineto{\pgfpoint{173.087997pt}{245.178955pt}}
\pgfpathclose
\pgfusepath{fill,stroke}
\color[rgb]{0.136835,0.661563,0.515967}
\pgfpathmoveto{\pgfpoint{164.160004pt}{257.532623pt}}
\pgflineto{\pgfpoint{173.087997pt}{251.355804pt}}
\pgflineto{\pgfpoint{164.160004pt}{251.355804pt}}
\pgfpathclose
\pgfusepath{fill,stroke}
\pgfpathmoveto{\pgfpoint{164.160004pt}{257.532623pt}}
\pgflineto{\pgfpoint{173.087997pt}{257.532623pt}}
\pgflineto{\pgfpoint{173.087997pt}{251.355804pt}}
\pgfpathclose
\pgfusepath{fill,stroke}
\color[rgb]{0.125898,0.572563,0.549445}
\pgfpathmoveto{\pgfpoint{173.087997pt}{208.117905pt}}
\pgflineto{\pgfpoint{182.015991pt}{201.941055pt}}
\pgflineto{\pgfpoint{173.087997pt}{201.941055pt}}
\pgfpathclose
\pgfusepath{fill,stroke}
\pgfpathmoveto{\pgfpoint{173.087997pt}{208.117905pt}}
\pgflineto{\pgfpoint{182.015991pt}{208.117905pt}}
\pgflineto{\pgfpoint{182.015991pt}{201.941055pt}}
\pgfpathclose
\pgfusepath{fill,stroke}
\color[rgb]{0.122163,0.587476,0.546023}
\pgfpathmoveto{\pgfpoint{173.087997pt}{214.294739pt}}
\pgflineto{\pgfpoint{182.015991pt}{208.117905pt}}
\pgflineto{\pgfpoint{173.087997pt}{208.117905pt}}
\pgfpathclose
\pgfusepath{fill,stroke}
\pgfpathmoveto{\pgfpoint{173.087997pt}{214.294739pt}}
\pgflineto{\pgfpoint{182.015991pt}{214.294739pt}}
\pgflineto{\pgfpoint{182.015991pt}{208.117905pt}}
\pgfpathclose
\pgfusepath{fill,stroke}
\color[rgb]{0.119872,0.602382,0.541831}
\pgfpathmoveto{\pgfpoint{173.087997pt}{220.471588pt}}
\pgflineto{\pgfpoint{182.015991pt}{214.294739pt}}
\pgflineto{\pgfpoint{173.087997pt}{214.294739pt}}
\pgfpathclose
\pgfusepath{fill,stroke}
\pgfpathmoveto{\pgfpoint{173.087997pt}{220.471588pt}}
\pgflineto{\pgfpoint{182.015991pt}{220.471588pt}}
\pgflineto{\pgfpoint{182.015991pt}{214.294739pt}}
\pgfpathclose
\pgfusepath{fill,stroke}
\color[rgb]{0.119627,0.617266,0.536796}
\pgfpathmoveto{\pgfpoint{173.087997pt}{226.648422pt}}
\pgflineto{\pgfpoint{182.015991pt}{220.471588pt}}
\pgflineto{\pgfpoint{173.087997pt}{220.471588pt}}
\pgfpathclose
\pgfusepath{fill,stroke}
\pgfpathmoveto{\pgfpoint{173.087997pt}{226.648422pt}}
\pgflineto{\pgfpoint{182.015991pt}{226.648422pt}}
\pgflineto{\pgfpoint{182.015991pt}{220.471588pt}}
\pgfpathclose
\pgfusepath{fill,stroke}
\color[rgb]{0.122046,0.632107,0.530848}
\pgfpathmoveto{\pgfpoint{173.087997pt}{232.825272pt}}
\pgflineto{\pgfpoint{182.015991pt}{226.648422pt}}
\pgflineto{\pgfpoint{173.087997pt}{226.648422pt}}
\pgfpathclose
\pgfusepath{fill,stroke}
\pgfpathmoveto{\pgfpoint{173.087997pt}{232.825272pt}}
\pgflineto{\pgfpoint{182.015991pt}{232.825272pt}}
\pgflineto{\pgfpoint{182.015991pt}{226.648422pt}}
\pgfpathclose
\pgfusepath{fill,stroke}
\color[rgb]{0.127668,0.646882,0.523924}
\pgfpathmoveto{\pgfpoint{173.087997pt}{239.002106pt}}
\pgflineto{\pgfpoint{182.015991pt}{232.825272pt}}
\pgflineto{\pgfpoint{173.087997pt}{232.825272pt}}
\pgfpathclose
\pgfusepath{fill,stroke}
\pgfpathmoveto{\pgfpoint{173.087997pt}{239.002106pt}}
\pgflineto{\pgfpoint{182.015991pt}{239.002106pt}}
\pgflineto{\pgfpoint{182.015991pt}{232.825272pt}}
\pgfpathclose
\pgfusepath{fill,stroke}
\pgfpathmoveto{\pgfpoint{173.087997pt}{245.178955pt}}
\pgflineto{\pgfpoint{182.015991pt}{239.002106pt}}
\pgflineto{\pgfpoint{173.087997pt}{239.002106pt}}
\pgfpathclose
\pgfusepath{fill,stroke}
\pgfpathmoveto{\pgfpoint{173.087997pt}{245.178955pt}}
\pgflineto{\pgfpoint{182.015991pt}{245.178955pt}}
\pgflineto{\pgfpoint{182.015991pt}{239.002106pt}}
\pgfpathclose
\pgfusepath{fill,stroke}
\color[rgb]{0.136835,0.661563,0.515967}
\pgfpathmoveto{\pgfpoint{173.087997pt}{251.355804pt}}
\pgflineto{\pgfpoint{182.015991pt}{245.178955pt}}
\pgflineto{\pgfpoint{173.087997pt}{245.178955pt}}
\pgfpathclose
\pgfusepath{fill,stroke}
\pgfpathmoveto{\pgfpoint{173.087997pt}{251.355804pt}}
\pgflineto{\pgfpoint{182.015991pt}{251.355804pt}}
\pgflineto{\pgfpoint{182.015991pt}{245.178955pt}}
\pgfpathclose
\pgfusepath{fill,stroke}
\color[rgb]{0.149643,0.676120,0.506924}
\pgfpathmoveto{\pgfpoint{173.087997pt}{257.532623pt}}
\pgflineto{\pgfpoint{182.015991pt}{251.355804pt}}
\pgflineto{\pgfpoint{173.087997pt}{251.355804pt}}
\pgfpathclose
\pgfusepath{fill,stroke}
\pgfpathmoveto{\pgfpoint{173.087997pt}{257.532623pt}}
\pgflineto{\pgfpoint{182.015991pt}{257.532623pt}}
\pgflineto{\pgfpoint{182.015991pt}{251.355804pt}}
\pgfpathclose
\pgfusepath{fill,stroke}
\color[rgb]{0.165967,0.690519,0.496752}
\pgfpathmoveto{\pgfpoint{173.087997pt}{263.709473pt}}
\pgflineto{\pgfpoint{182.015991pt}{257.532623pt}}
\pgflineto{\pgfpoint{173.087997pt}{257.532623pt}}
\pgfpathclose
\pgfusepath{fill,stroke}
\pgfpathmoveto{\pgfpoint{173.087997pt}{263.709473pt}}
\pgflineto{\pgfpoint{182.015991pt}{263.709473pt}}
\pgflineto{\pgfpoint{182.015991pt}{257.532623pt}}
\pgfpathclose
\pgfusepath{fill,stroke}
\color[rgb]{0.185538,0.704725,0.485412}
\pgfpathmoveto{\pgfpoint{173.087997pt}{269.886322pt}}
\pgflineto{\pgfpoint{182.015991pt}{263.709473pt}}
\pgflineto{\pgfpoint{173.087997pt}{263.709473pt}}
\pgfpathclose
\pgfusepath{fill,stroke}
\pgfpathmoveto{\pgfpoint{173.087997pt}{269.886322pt}}
\pgflineto{\pgfpoint{182.015991pt}{269.886322pt}}
\pgflineto{\pgfpoint{182.015991pt}{263.709473pt}}
\pgfpathclose
\pgfusepath{fill,stroke}
\color[rgb]{0.119627,0.617266,0.536796}
\pgfpathmoveto{\pgfpoint{182.015991pt}{220.471588pt}}
\pgflineto{\pgfpoint{190.943985pt}{214.294739pt}}
\pgflineto{\pgfpoint{182.015991pt}{214.294739pt}}
\pgfpathclose
\pgfusepath{fill,stroke}
\pgfpathmoveto{\pgfpoint{182.015991pt}{220.471588pt}}
\pgflineto{\pgfpoint{190.943985pt}{220.471588pt}}
\pgflineto{\pgfpoint{190.943985pt}{214.294739pt}}
\pgfpathclose
\pgfusepath{fill,stroke}
\color[rgb]{0.122046,0.632107,0.530848}
\pgfpathmoveto{\pgfpoint{182.015991pt}{226.648422pt}}
\pgflineto{\pgfpoint{190.943985pt}{220.471588pt}}
\pgflineto{\pgfpoint{182.015991pt}{220.471588pt}}
\pgfpathclose
\pgfusepath{fill,stroke}
\pgfpathmoveto{\pgfpoint{182.015991pt}{226.648422pt}}
\pgflineto{\pgfpoint{190.943985pt}{226.648422pt}}
\pgflineto{\pgfpoint{190.943985pt}{220.471588pt}}
\pgfpathclose
\pgfusepath{fill,stroke}
\color[rgb]{0.127668,0.646882,0.523924}
\pgfpathmoveto{\pgfpoint{182.015991pt}{232.825272pt}}
\pgflineto{\pgfpoint{190.943985pt}{226.648422pt}}
\pgflineto{\pgfpoint{182.015991pt}{226.648422pt}}
\pgfpathclose
\pgfusepath{fill,stroke}
\pgfpathmoveto{\pgfpoint{182.015991pt}{232.825272pt}}
\pgflineto{\pgfpoint{190.943985pt}{232.825272pt}}
\pgflineto{\pgfpoint{190.943985pt}{226.648422pt}}
\pgfpathclose
\pgfusepath{fill,stroke}
\color[rgb]{0.136835,0.661563,0.515967}
\pgfpathmoveto{\pgfpoint{182.015991pt}{239.002106pt}}
\pgflineto{\pgfpoint{190.943985pt}{232.825272pt}}
\pgflineto{\pgfpoint{182.015991pt}{232.825272pt}}
\pgfpathclose
\pgfusepath{fill,stroke}
\pgfpathmoveto{\pgfpoint{182.015991pt}{239.002106pt}}
\pgflineto{\pgfpoint{190.943985pt}{239.002106pt}}
\pgflineto{\pgfpoint{190.943985pt}{232.825272pt}}
\pgfpathclose
\pgfusepath{fill,stroke}
\color[rgb]{0.149643,0.676120,0.506924}
\pgfpathmoveto{\pgfpoint{182.015991pt}{245.178955pt}}
\pgflineto{\pgfpoint{190.943985pt}{239.002106pt}}
\pgflineto{\pgfpoint{182.015991pt}{239.002106pt}}
\pgfpathclose
\pgfusepath{fill,stroke}
\pgfpathmoveto{\pgfpoint{182.015991pt}{245.178955pt}}
\pgflineto{\pgfpoint{190.943985pt}{245.178955pt}}
\pgflineto{\pgfpoint{190.943985pt}{239.002106pt}}
\pgfpathclose
\pgfusepath{fill,stroke}
\pgfpathmoveto{\pgfpoint{182.015991pt}{251.355804pt}}
\pgflineto{\pgfpoint{190.943985pt}{245.178955pt}}
\pgflineto{\pgfpoint{182.015991pt}{245.178955pt}}
\pgfpathclose
\pgfusepath{fill,stroke}
\pgfpathmoveto{\pgfpoint{182.015991pt}{251.355804pt}}
\pgflineto{\pgfpoint{190.943985pt}{251.355804pt}}
\pgflineto{\pgfpoint{190.943985pt}{245.178955pt}}
\pgfpathclose
\pgfusepath{fill,stroke}
\color[rgb]{0.165967,0.690519,0.496752}
\pgfpathmoveto{\pgfpoint{182.015991pt}{257.532623pt}}
\pgflineto{\pgfpoint{190.943985pt}{251.355804pt}}
\pgflineto{\pgfpoint{182.015991pt}{251.355804pt}}
\pgfpathclose
\pgfusepath{fill,stroke}
\pgfpathmoveto{\pgfpoint{182.015991pt}{257.532623pt}}
\pgflineto{\pgfpoint{190.943985pt}{257.532623pt}}
\pgflineto{\pgfpoint{190.943985pt}{251.355804pt}}
\pgfpathclose
\pgfusepath{fill,stroke}
\color[rgb]{0.185538,0.704725,0.485412}
\pgfpathmoveto{\pgfpoint{182.015991pt}{263.709473pt}}
\pgflineto{\pgfpoint{190.943985pt}{257.532623pt}}
\pgflineto{\pgfpoint{182.015991pt}{257.532623pt}}
\pgfpathclose
\pgfusepath{fill,stroke}
\pgfpathmoveto{\pgfpoint{182.015991pt}{263.709473pt}}
\pgflineto{\pgfpoint{190.943985pt}{263.709473pt}}
\pgflineto{\pgfpoint{190.943985pt}{257.532623pt}}
\pgfpathclose
\pgfusepath{fill,stroke}
\color[rgb]{0.208030,0.718701,0.472873}
\pgfpathmoveto{\pgfpoint{182.015991pt}{269.886322pt}}
\pgflineto{\pgfpoint{190.943985pt}{263.709473pt}}
\pgflineto{\pgfpoint{182.015991pt}{263.709473pt}}
\pgfpathclose
\pgfusepath{fill,stroke}
\pgfpathmoveto{\pgfpoint{182.015991pt}{269.886322pt}}
\pgflineto{\pgfpoint{190.943985pt}{269.886322pt}}
\pgflineto{\pgfpoint{190.943985pt}{263.709473pt}}
\pgfpathclose
\pgfusepath{fill,stroke}
\color[rgb]{0.233127,0.732406,0.459106}
\pgfpathmoveto{\pgfpoint{182.015991pt}{276.063141pt}}
\pgflineto{\pgfpoint{190.943985pt}{269.886322pt}}
\pgflineto{\pgfpoint{182.015991pt}{269.886322pt}}
\pgfpathclose
\pgfusepath{fill,stroke}
\pgfpathmoveto{\pgfpoint{182.015991pt}{276.063141pt}}
\pgflineto{\pgfpoint{190.943985pt}{276.063141pt}}
\pgflineto{\pgfpoint{190.943985pt}{269.886322pt}}
\pgfpathclose
\pgfusepath{fill,stroke}
\color[rgb]{0.260531,0.745802,0.444096}
\pgfpathmoveto{\pgfpoint{182.015991pt}{282.239990pt}}
\pgflineto{\pgfpoint{190.943985pt}{276.063141pt}}
\pgflineto{\pgfpoint{182.015991pt}{276.063141pt}}
\pgfpathclose
\pgfusepath{fill,stroke}
\pgfpathmoveto{\pgfpoint{182.015991pt}{282.239990pt}}
\pgflineto{\pgfpoint{190.943985pt}{282.239990pt}}
\pgflineto{\pgfpoint{190.943985pt}{276.063141pt}}
\pgfpathclose
\pgfusepath{fill,stroke}
\color[rgb]{0.136835,0.661563,0.515967}
\pgfpathmoveto{\pgfpoint{190.943985pt}{232.825272pt}}
\pgflineto{\pgfpoint{199.871979pt}{226.648422pt}}
\pgflineto{\pgfpoint{190.943985pt}{226.648422pt}}
\pgfpathclose
\pgfusepath{fill,stroke}
\pgfpathmoveto{\pgfpoint{190.943985pt}{232.825272pt}}
\pgflineto{\pgfpoint{199.871979pt}{232.825272pt}}
\pgflineto{\pgfpoint{199.871979pt}{226.648422pt}}
\pgfpathclose
\pgfusepath{fill,stroke}
\color[rgb]{0.149643,0.676120,0.506924}
\pgfpathmoveto{\pgfpoint{190.943985pt}{239.002106pt}}
\pgflineto{\pgfpoint{199.871979pt}{232.825272pt}}
\pgflineto{\pgfpoint{190.943985pt}{232.825272pt}}
\pgfpathclose
\pgfusepath{fill,stroke}
\pgfpathmoveto{\pgfpoint{190.943985pt}{239.002106pt}}
\pgflineto{\pgfpoint{199.871979pt}{239.002106pt}}
\pgflineto{\pgfpoint{199.871979pt}{232.825272pt}}
\pgfpathclose
\pgfusepath{fill,stroke}
\color[rgb]{0.165967,0.690519,0.496752}
\pgfpathmoveto{\pgfpoint{190.943985pt}{245.178955pt}}
\pgflineto{\pgfpoint{199.871979pt}{239.002106pt}}
\pgflineto{\pgfpoint{190.943985pt}{239.002106pt}}
\pgfpathclose
\pgfusepath{fill,stroke}
\pgfpathmoveto{\pgfpoint{190.943985pt}{245.178955pt}}
\pgflineto{\pgfpoint{199.871979pt}{245.178955pt}}
\pgflineto{\pgfpoint{199.871979pt}{239.002106pt}}
\pgfpathclose
\pgfusepath{fill,stroke}
\pgfpathmoveto{\pgfpoint{190.943985pt}{251.355804pt}}
\pgflineto{\pgfpoint{199.871979pt}{245.178955pt}}
\pgflineto{\pgfpoint{190.943985pt}{245.178955pt}}
\pgfpathclose
\pgfusepath{fill,stroke}
\pgfpathmoveto{\pgfpoint{190.943985pt}{251.355804pt}}
\pgflineto{\pgfpoint{199.871979pt}{251.355804pt}}
\pgflineto{\pgfpoint{199.871979pt}{245.178955pt}}
\pgfpathclose
\pgfusepath{fill,stroke}
\color[rgb]{0.185538,0.704725,0.485412}
\pgfpathmoveto{\pgfpoint{190.943985pt}{257.532623pt}}
\pgflineto{\pgfpoint{199.871979pt}{251.355804pt}}
\pgflineto{\pgfpoint{190.943985pt}{251.355804pt}}
\pgfpathclose
\pgfusepath{fill,stroke}
\pgfpathmoveto{\pgfpoint{190.943985pt}{257.532623pt}}
\pgflineto{\pgfpoint{199.871979pt}{257.532623pt}}
\pgflineto{\pgfpoint{199.871979pt}{251.355804pt}}
\pgfpathclose
\pgfusepath{fill,stroke}
\color[rgb]{0.208030,0.718701,0.472873}
\pgfpathmoveto{\pgfpoint{190.943985pt}{263.709473pt}}
\pgflineto{\pgfpoint{199.871979pt}{257.532623pt}}
\pgflineto{\pgfpoint{190.943985pt}{257.532623pt}}
\pgfpathclose
\pgfusepath{fill,stroke}
\pgfpathmoveto{\pgfpoint{190.943985pt}{263.709473pt}}
\pgflineto{\pgfpoint{199.871979pt}{263.709473pt}}
\pgflineto{\pgfpoint{199.871979pt}{257.532623pt}}
\pgfpathclose
\pgfusepath{fill,stroke}
\color[rgb]{0.233127,0.732406,0.459106}
\pgfpathmoveto{\pgfpoint{190.943985pt}{269.886322pt}}
\pgflineto{\pgfpoint{199.871979pt}{263.709473pt}}
\pgflineto{\pgfpoint{190.943985pt}{263.709473pt}}
\pgfpathclose
\pgfusepath{fill,stroke}
\pgfpathmoveto{\pgfpoint{190.943985pt}{269.886322pt}}
\pgflineto{\pgfpoint{199.871979pt}{269.886322pt}}
\pgflineto{\pgfpoint{199.871979pt}{263.709473pt}}
\pgfpathclose
\pgfusepath{fill,stroke}
\color[rgb]{0.260531,0.745802,0.444096}
\pgfpathmoveto{\pgfpoint{190.943985pt}{276.063141pt}}
\pgflineto{\pgfpoint{199.871979pt}{269.886322pt}}
\pgflineto{\pgfpoint{190.943985pt}{269.886322pt}}
\pgfpathclose
\pgfusepath{fill,stroke}
\pgfpathmoveto{\pgfpoint{190.943985pt}{276.063141pt}}
\pgflineto{\pgfpoint{199.871979pt}{276.063141pt}}
\pgflineto{\pgfpoint{199.871979pt}{269.886322pt}}
\pgfpathclose
\pgfusepath{fill,stroke}
\color[rgb]{0.290001,0.758846,0.427826}
\pgfpathmoveto{\pgfpoint{190.943985pt}{282.239990pt}}
\pgflineto{\pgfpoint{199.871979pt}{276.063141pt}}
\pgflineto{\pgfpoint{190.943985pt}{276.063141pt}}
\pgfpathclose
\pgfusepath{fill,stroke}
\pgfpathmoveto{\pgfpoint{190.943985pt}{282.239990pt}}
\pgflineto{\pgfpoint{199.871979pt}{282.239990pt}}
\pgflineto{\pgfpoint{199.871979pt}{276.063141pt}}
\pgfpathclose
\pgfusepath{fill,stroke}
\pgfpathmoveto{\pgfpoint{190.943985pt}{288.416840pt}}
\pgflineto{\pgfpoint{199.871979pt}{282.239990pt}}
\pgflineto{\pgfpoint{190.943985pt}{282.239990pt}}
\pgfpathclose
\pgfusepath{fill,stroke}
\pgfpathmoveto{\pgfpoint{190.943985pt}{288.416840pt}}
\pgflineto{\pgfpoint{199.871979pt}{288.416840pt}}
\pgflineto{\pgfpoint{199.871979pt}{282.239990pt}}
\pgfpathclose
\pgfusepath{fill,stroke}
\color[rgb]{0.185538,0.704725,0.485412}
\pgfpathmoveto{\pgfpoint{199.871979pt}{245.178955pt}}
\pgflineto{\pgfpoint{208.799988pt}{239.002106pt}}
\pgflineto{\pgfpoint{199.871979pt}{239.002106pt}}
\pgfpathclose
\pgfusepath{fill,stroke}
\pgfpathmoveto{\pgfpoint{199.871979pt}{245.178955pt}}
\pgflineto{\pgfpoint{208.799988pt}{245.178955pt}}
\pgflineto{\pgfpoint{208.799988pt}{239.002106pt}}
\pgfpathclose
\pgfusepath{fill,stroke}
\color[rgb]{0.208030,0.718701,0.472873}
\pgfpathmoveto{\pgfpoint{199.871979pt}{251.355804pt}}
\pgflineto{\pgfpoint{208.799988pt}{245.178955pt}}
\pgflineto{\pgfpoint{199.871979pt}{245.178955pt}}
\pgfpathclose
\pgfusepath{fill,stroke}
\pgfpathmoveto{\pgfpoint{199.871979pt}{251.355804pt}}
\pgflineto{\pgfpoint{208.799988pt}{251.355804pt}}
\pgflineto{\pgfpoint{208.799988pt}{245.178955pt}}
\pgfpathclose
\pgfusepath{fill,stroke}
\pgfpathmoveto{\pgfpoint{199.871979pt}{257.532623pt}}
\pgflineto{\pgfpoint{208.799988pt}{251.355804pt}}
\pgflineto{\pgfpoint{199.871979pt}{251.355804pt}}
\pgfpathclose
\pgfusepath{fill,stroke}
\pgfpathmoveto{\pgfpoint{199.871979pt}{257.532623pt}}
\pgflineto{\pgfpoint{208.799988pt}{257.532623pt}}
\pgflineto{\pgfpoint{208.799988pt}{251.355804pt}}
\pgfpathclose
\pgfusepath{fill,stroke}
\color[rgb]{0.233127,0.732406,0.459106}
\pgfpathmoveto{\pgfpoint{199.871979pt}{263.709473pt}}
\pgflineto{\pgfpoint{208.799988pt}{257.532623pt}}
\pgflineto{\pgfpoint{199.871979pt}{257.532623pt}}
\pgfpathclose
\pgfusepath{fill,stroke}
\pgfpathmoveto{\pgfpoint{199.871979pt}{263.709473pt}}
\pgflineto{\pgfpoint{208.799988pt}{263.709473pt}}
\pgflineto{\pgfpoint{208.799988pt}{257.532623pt}}
\pgfpathclose
\pgfusepath{fill,stroke}
\color[rgb]{0.260531,0.745802,0.444096}
\pgfpathmoveto{\pgfpoint{199.871979pt}{269.886322pt}}
\pgflineto{\pgfpoint{208.799988pt}{263.709473pt}}
\pgflineto{\pgfpoint{199.871979pt}{263.709473pt}}
\pgfpathclose
\pgfusepath{fill,stroke}
\pgfpathmoveto{\pgfpoint{199.871979pt}{269.886322pt}}
\pgflineto{\pgfpoint{208.799988pt}{269.886322pt}}
\pgflineto{\pgfpoint{208.799988pt}{263.709473pt}}
\pgfpathclose
\pgfusepath{fill,stroke}
\color[rgb]{0.290001,0.758846,0.427826}
\pgfpathmoveto{\pgfpoint{199.871979pt}{276.063141pt}}
\pgflineto{\pgfpoint{208.799988pt}{269.886322pt}}
\pgflineto{\pgfpoint{199.871979pt}{269.886322pt}}
\pgfpathclose
\pgfusepath{fill,stroke}
\pgfpathmoveto{\pgfpoint{199.871979pt}{276.063141pt}}
\pgflineto{\pgfpoint{208.799988pt}{276.063141pt}}
\pgflineto{\pgfpoint{208.799988pt}{269.886322pt}}
\pgfpathclose
\pgfusepath{fill,stroke}
\color[rgb]{0.321330,0.771498,0.410293}
\pgfpathmoveto{\pgfpoint{199.871979pt}{282.239990pt}}
\pgflineto{\pgfpoint{208.799988pt}{276.063141pt}}
\pgflineto{\pgfpoint{199.871979pt}{276.063141pt}}
\pgfpathclose
\pgfusepath{fill,stroke}
\pgfpathmoveto{\pgfpoint{199.871979pt}{282.239990pt}}
\pgflineto{\pgfpoint{208.799988pt}{282.239990pt}}
\pgflineto{\pgfpoint{208.799988pt}{276.063141pt}}
\pgfpathclose
\pgfusepath{fill,stroke}
\color[rgb]{0.354355,0.783714,0.391488}
\pgfpathmoveto{\pgfpoint{199.871979pt}{288.416840pt}}
\pgflineto{\pgfpoint{208.799988pt}{282.239990pt}}
\pgflineto{\pgfpoint{199.871979pt}{282.239990pt}}
\pgfpathclose
\pgfusepath{fill,stroke}
\pgfpathmoveto{\pgfpoint{199.871979pt}{288.416840pt}}
\pgflineto{\pgfpoint{208.799988pt}{288.416840pt}}
\pgflineto{\pgfpoint{208.799988pt}{282.239990pt}}
\pgfpathclose
\pgfusepath{fill,stroke}
\pgfpathmoveto{\pgfpoint{199.871979pt}{294.593689pt}}
\pgflineto{\pgfpoint{208.799988pt}{288.416840pt}}
\pgflineto{\pgfpoint{199.871979pt}{288.416840pt}}
\pgfpathclose
\pgfusepath{fill,stroke}
\pgfpathmoveto{\pgfpoint{199.871979pt}{294.593689pt}}
\pgflineto{\pgfpoint{208.799988pt}{294.593689pt}}
\pgflineto{\pgfpoint{208.799988pt}{288.416840pt}}
\pgfpathclose
\pgfusepath{fill,stroke}
\color[rgb]{0.388930,0.795453,0.371421}
\pgfpathmoveto{\pgfpoint{199.871979pt}{300.770538pt}}
\pgflineto{\pgfpoint{208.799988pt}{294.593689pt}}
\pgflineto{\pgfpoint{199.871979pt}{294.593689pt}}
\pgfpathclose
\pgfusepath{fill,stroke}
\pgfpathmoveto{\pgfpoint{199.871979pt}{300.770538pt}}
\pgflineto{\pgfpoint{208.799988pt}{300.770538pt}}
\pgflineto{\pgfpoint{208.799988pt}{294.593689pt}}
\pgfpathclose
\pgfusepath{fill,stroke}
\color[rgb]{0.233127,0.732406,0.459106}
\pgfpathmoveto{\pgfpoint{208.799988pt}{257.532623pt}}
\pgflineto{\pgfpoint{217.727982pt}{251.355804pt}}
\pgflineto{\pgfpoint{208.799988pt}{251.355804pt}}
\pgfpathclose
\pgfusepath{fill,stroke}
\pgfpathmoveto{\pgfpoint{208.799988pt}{257.532623pt}}
\pgflineto{\pgfpoint{217.727982pt}{257.532623pt}}
\pgflineto{\pgfpoint{217.727982pt}{251.355804pt}}
\pgfpathclose
\pgfusepath{fill,stroke}
\color[rgb]{0.260531,0.745802,0.444096}
\pgfpathmoveto{\pgfpoint{208.799988pt}{263.709473pt}}
\pgflineto{\pgfpoint{217.727982pt}{257.532623pt}}
\pgflineto{\pgfpoint{208.799988pt}{257.532623pt}}
\pgfpathclose
\pgfusepath{fill,stroke}
\pgfpathmoveto{\pgfpoint{208.799988pt}{263.709473pt}}
\pgflineto{\pgfpoint{217.727982pt}{263.709473pt}}
\pgflineto{\pgfpoint{217.727982pt}{257.532623pt}}
\pgfpathclose
\pgfusepath{fill,stroke}
\color[rgb]{0.290001,0.758846,0.427826}
\pgfpathmoveto{\pgfpoint{208.799988pt}{269.886322pt}}
\pgflineto{\pgfpoint{217.727982pt}{263.709473pt}}
\pgflineto{\pgfpoint{208.799988pt}{263.709473pt}}
\pgfpathclose
\pgfusepath{fill,stroke}
\pgfpathmoveto{\pgfpoint{208.799988pt}{269.886322pt}}
\pgflineto{\pgfpoint{217.727982pt}{269.886322pt}}
\pgflineto{\pgfpoint{217.727982pt}{263.709473pt}}
\pgfpathclose
\pgfusepath{fill,stroke}
\color[rgb]{0.321330,0.771498,0.410293}
\pgfpathmoveto{\pgfpoint{208.799988pt}{276.063141pt}}
\pgflineto{\pgfpoint{217.727982pt}{269.886322pt}}
\pgflineto{\pgfpoint{208.799988pt}{269.886322pt}}
\pgfpathclose
\pgfusepath{fill,stroke}
\pgfpathmoveto{\pgfpoint{208.799988pt}{276.063141pt}}
\pgflineto{\pgfpoint{217.727982pt}{276.063141pt}}
\pgflineto{\pgfpoint{217.727982pt}{269.886322pt}}
\pgfpathclose
\pgfusepath{fill,stroke}
\color[rgb]{0.354355,0.783714,0.391488}
\pgfpathmoveto{\pgfpoint{208.799988pt}{282.239990pt}}
\pgflineto{\pgfpoint{217.727982pt}{276.063141pt}}
\pgflineto{\pgfpoint{208.799988pt}{276.063141pt}}
\pgfpathclose
\pgfusepath{fill,stroke}
\pgfpathmoveto{\pgfpoint{208.799988pt}{282.239990pt}}
\pgflineto{\pgfpoint{217.727982pt}{282.239990pt}}
\pgflineto{\pgfpoint{217.727982pt}{276.063141pt}}
\pgfpathclose
\pgfusepath{fill,stroke}
\color[rgb]{0.388930,0.795453,0.371421}
\pgfpathmoveto{\pgfpoint{208.799988pt}{288.416840pt}}
\pgflineto{\pgfpoint{217.727982pt}{282.239990pt}}
\pgflineto{\pgfpoint{208.799988pt}{282.239990pt}}
\pgfpathclose
\pgfusepath{fill,stroke}
\pgfpathmoveto{\pgfpoint{208.799988pt}{288.416840pt}}
\pgflineto{\pgfpoint{217.727982pt}{288.416840pt}}
\pgflineto{\pgfpoint{217.727982pt}{282.239990pt}}
\pgfpathclose
\pgfusepath{fill,stroke}
\pgfpathmoveto{\pgfpoint{208.799988pt}{294.593689pt}}
\pgflineto{\pgfpoint{217.727982pt}{288.416840pt}}
\pgflineto{\pgfpoint{208.799988pt}{288.416840pt}}
\pgfpathclose
\pgfusepath{fill,stroke}
\pgfpathmoveto{\pgfpoint{208.799988pt}{294.593689pt}}
\pgflineto{\pgfpoint{217.727982pt}{294.593689pt}}
\pgflineto{\pgfpoint{217.727982pt}{288.416840pt}}
\pgfpathclose
\pgfusepath{fill,stroke}
\color[rgb]{0.424933,0.806674,0.350099}
\pgfpathmoveto{\pgfpoint{208.799988pt}{300.770538pt}}
\pgflineto{\pgfpoint{217.727982pt}{294.593689pt}}
\pgflineto{\pgfpoint{208.799988pt}{294.593689pt}}
\pgfpathclose
\pgfusepath{fill,stroke}
\pgfpathmoveto{\pgfpoint{208.799988pt}{300.770538pt}}
\pgflineto{\pgfpoint{217.727982pt}{300.770538pt}}
\pgflineto{\pgfpoint{217.727982pt}{294.593689pt}}
\pgfpathclose
\pgfusepath{fill,stroke}
\color[rgb]{0.462247,0.817338,0.327545}
\pgfpathmoveto{\pgfpoint{208.799988pt}{306.947388pt}}
\pgflineto{\pgfpoint{217.727982pt}{300.770538pt}}
\pgflineto{\pgfpoint{208.799988pt}{300.770538pt}}
\pgfpathclose
\pgfusepath{fill,stroke}
\pgfpathmoveto{\pgfpoint{208.799988pt}{306.947388pt}}
\pgflineto{\pgfpoint{217.727982pt}{306.947388pt}}
\pgflineto{\pgfpoint{217.727982pt}{300.770538pt}}
\pgfpathclose
\pgfusepath{fill,stroke}
\color[rgb]{0.500754,0.827409,0.303799}
\pgfpathmoveto{\pgfpoint{208.799988pt}{313.124207pt}}
\pgflineto{\pgfpoint{217.727982pt}{306.947388pt}}
\pgflineto{\pgfpoint{208.799988pt}{306.947388pt}}
\pgfpathclose
\pgfusepath{fill,stroke}
\pgfpathmoveto{\pgfpoint{208.799988pt}{313.124207pt}}
\pgflineto{\pgfpoint{217.727982pt}{313.124207pt}}
\pgflineto{\pgfpoint{217.727982pt}{306.947388pt}}
\pgfpathclose
\pgfusepath{fill,stroke}
\color[rgb]{0.354355,0.783714,0.391488}
\pgfpathmoveto{\pgfpoint{217.727982pt}{276.063141pt}}
\pgflineto{\pgfpoint{226.655975pt}{269.886322pt}}
\pgflineto{\pgfpoint{217.727982pt}{269.886322pt}}
\pgfpathclose
\pgfusepath{fill,stroke}
\pgfpathmoveto{\pgfpoint{217.727982pt}{276.063141pt}}
\pgflineto{\pgfpoint{226.655975pt}{276.063141pt}}
\pgflineto{\pgfpoint{226.655975pt}{269.886322pt}}
\pgfpathclose
\pgfusepath{fill,stroke}
\color[rgb]{0.388930,0.795453,0.371421}
\pgfpathmoveto{\pgfpoint{217.727982pt}{282.239990pt}}
\pgflineto{\pgfpoint{226.655975pt}{276.063141pt}}
\pgflineto{\pgfpoint{217.727982pt}{276.063141pt}}
\pgfpathclose
\pgfusepath{fill,stroke}
\pgfpathmoveto{\pgfpoint{217.727982pt}{282.239990pt}}
\pgflineto{\pgfpoint{226.655975pt}{282.239990pt}}
\pgflineto{\pgfpoint{226.655975pt}{276.063141pt}}
\pgfpathclose
\pgfusepath{fill,stroke}
\color[rgb]{0.424933,0.806674,0.350099}
\pgfpathmoveto{\pgfpoint{217.727982pt}{288.416840pt}}
\pgflineto{\pgfpoint{226.655975pt}{282.239990pt}}
\pgflineto{\pgfpoint{217.727982pt}{282.239990pt}}
\pgfpathclose
\pgfusepath{fill,stroke}
\pgfpathmoveto{\pgfpoint{217.727982pt}{288.416840pt}}
\pgflineto{\pgfpoint{226.655975pt}{288.416840pt}}
\pgflineto{\pgfpoint{226.655975pt}{282.239990pt}}
\pgfpathclose
\pgfusepath{fill,stroke}
\color[rgb]{0.462247,0.817338,0.327545}
\pgfpathmoveto{\pgfpoint{217.727982pt}{294.593689pt}}
\pgflineto{\pgfpoint{226.655975pt}{288.416840pt}}
\pgflineto{\pgfpoint{217.727982pt}{288.416840pt}}
\pgfpathclose
\pgfusepath{fill,stroke}
\pgfpathmoveto{\pgfpoint{217.727982pt}{294.593689pt}}
\pgflineto{\pgfpoint{226.655975pt}{294.593689pt}}
\pgflineto{\pgfpoint{226.655975pt}{288.416840pt}}
\pgfpathclose
\pgfusepath{fill,stroke}
\pgfpathmoveto{\pgfpoint{217.727982pt}{300.770538pt}}
\pgflineto{\pgfpoint{226.655975pt}{294.593689pt}}
\pgflineto{\pgfpoint{217.727982pt}{294.593689pt}}
\pgfpathclose
\pgfusepath{fill,stroke}
\pgfpathmoveto{\pgfpoint{217.727982pt}{300.770538pt}}
\pgflineto{\pgfpoint{226.655975pt}{300.770538pt}}
\pgflineto{\pgfpoint{226.655975pt}{294.593689pt}}
\pgfpathclose
\pgfusepath{fill,stroke}
\color[rgb]{0.500754,0.827409,0.303799}
\pgfpathmoveto{\pgfpoint{217.727982pt}{306.947388pt}}
\pgflineto{\pgfpoint{226.655975pt}{300.770538pt}}
\pgflineto{\pgfpoint{217.727982pt}{300.770538pt}}
\pgfpathclose
\pgfusepath{fill,stroke}
\pgfpathmoveto{\pgfpoint{217.727982pt}{306.947388pt}}
\pgflineto{\pgfpoint{226.655975pt}{306.947388pt}}
\pgflineto{\pgfpoint{226.655975pt}{300.770538pt}}
\pgfpathclose
\pgfusepath{fill,stroke}
\color[rgb]{0.540337,0.836858,0.278917}
\pgfpathmoveto{\pgfpoint{217.727982pt}{313.124207pt}}
\pgflineto{\pgfpoint{226.655975pt}{306.947388pt}}
\pgflineto{\pgfpoint{217.727982pt}{306.947388pt}}
\pgfpathclose
\pgfusepath{fill,stroke}
\pgfpathmoveto{\pgfpoint{217.727982pt}{313.124207pt}}
\pgflineto{\pgfpoint{226.655975pt}{313.124207pt}}
\pgflineto{\pgfpoint{226.655975pt}{306.947388pt}}
\pgfpathclose
\pgfusepath{fill,stroke}
\color[rgb]{0.580861,0.845663,0.253001}
\pgfpathmoveto{\pgfpoint{217.727982pt}{319.301056pt}}
\pgflineto{\pgfpoint{226.655975pt}{313.124207pt}}
\pgflineto{\pgfpoint{217.727982pt}{313.124207pt}}
\pgfpathclose
\pgfusepath{fill,stroke}
\pgfpathmoveto{\pgfpoint{217.727982pt}{319.301056pt}}
\pgflineto{\pgfpoint{226.655975pt}{319.301056pt}}
\pgflineto{\pgfpoint{226.655975pt}{313.124207pt}}
\pgfpathclose
\pgfusepath{fill,stroke}
\color[rgb]{0.622171,0.853816,0.226224}
\pgfpathmoveto{\pgfpoint{217.727982pt}{325.477905pt}}
\pgflineto{\pgfpoint{226.655975pt}{319.301056pt}}
\pgflineto{\pgfpoint{217.727982pt}{319.301056pt}}
\pgfpathclose
\pgfusepath{fill,stroke}
\pgfpathmoveto{\pgfpoint{217.727982pt}{325.477905pt}}
\pgflineto{\pgfpoint{226.655975pt}{325.477905pt}}
\pgflineto{\pgfpoint{226.655975pt}{319.301056pt}}
\pgfpathclose
\pgfusepath{fill,stroke}
\color[rgb]{0.500754,0.827409,0.303799}
\pgfpathmoveto{\pgfpoint{226.655975pt}{294.593689pt}}
\pgflineto{\pgfpoint{235.583969pt}{288.416840pt}}
\pgflineto{\pgfpoint{226.655975pt}{288.416840pt}}
\pgfpathclose
\pgfusepath{fill,stroke}
\pgfpathmoveto{\pgfpoint{226.655975pt}{294.593689pt}}
\pgflineto{\pgfpoint{235.583969pt}{294.593689pt}}
\pgflineto{\pgfpoint{235.583969pt}{288.416840pt}}
\pgfpathclose
\pgfusepath{fill,stroke}
\pgfpathmoveto{\pgfpoint{226.655975pt}{300.770538pt}}
\pgflineto{\pgfpoint{235.583969pt}{294.593689pt}}
\pgflineto{\pgfpoint{226.655975pt}{294.593689pt}}
\pgfpathclose
\pgfusepath{fill,stroke}
\pgfpathmoveto{\pgfpoint{226.655975pt}{300.770538pt}}
\pgflineto{\pgfpoint{235.583969pt}{300.770538pt}}
\pgflineto{\pgfpoint{235.583969pt}{294.593689pt}}
\pgfpathclose
\pgfusepath{fill,stroke}
\color[rgb]{0.540337,0.836858,0.278917}
\pgfpathmoveto{\pgfpoint{226.655975pt}{306.947388pt}}
\pgflineto{\pgfpoint{235.583969pt}{300.770538pt}}
\pgflineto{\pgfpoint{226.655975pt}{300.770538pt}}
\pgfpathclose
\pgfusepath{fill,stroke}
\pgfpathmoveto{\pgfpoint{226.655975pt}{306.947388pt}}
\pgflineto{\pgfpoint{235.583969pt}{306.947388pt}}
\pgflineto{\pgfpoint{235.583969pt}{300.770538pt}}
\pgfpathclose
\pgfusepath{fill,stroke}
\color[rgb]{0.580861,0.845663,0.253001}
\pgfpathmoveto{\pgfpoint{226.655975pt}{313.124207pt}}
\pgflineto{\pgfpoint{235.583969pt}{306.947388pt}}
\pgflineto{\pgfpoint{226.655975pt}{306.947388pt}}
\pgfpathclose
\pgfusepath{fill,stroke}
\pgfpathmoveto{\pgfpoint{226.655975pt}{313.124207pt}}
\pgflineto{\pgfpoint{235.583969pt}{313.124207pt}}
\pgflineto{\pgfpoint{235.583969pt}{306.947388pt}}
\pgfpathclose
\pgfusepath{fill,stroke}
\color[rgb]{0.622171,0.853816,0.226224}
\pgfpathmoveto{\pgfpoint{226.655975pt}{319.301056pt}}
\pgflineto{\pgfpoint{235.583969pt}{313.124207pt}}
\pgflineto{\pgfpoint{226.655975pt}{313.124207pt}}
\pgfpathclose
\pgfusepath{fill,stroke}
\pgfpathmoveto{\pgfpoint{226.655975pt}{319.301056pt}}
\pgflineto{\pgfpoint{235.583969pt}{319.301056pt}}
\pgflineto{\pgfpoint{235.583969pt}{313.124207pt}}
\pgfpathclose
\pgfusepath{fill,stroke}
\color[rgb]{0.664087,0.861321,0.198879}
\pgfpathmoveto{\pgfpoint{226.655975pt}{325.477905pt}}
\pgflineto{\pgfpoint{235.583969pt}{319.301056pt}}
\pgflineto{\pgfpoint{226.655975pt}{319.301056pt}}
\pgfpathclose
\pgfusepath{fill,stroke}
\pgfpathmoveto{\pgfpoint{226.655975pt}{325.477905pt}}
\pgflineto{\pgfpoint{235.583969pt}{325.477905pt}}
\pgflineto{\pgfpoint{235.583969pt}{319.301056pt}}
\pgfpathclose
\pgfusepath{fill,stroke}
\color[rgb]{0.706404,0.868206,0.171495}
\pgfpathmoveto{\pgfpoint{226.655975pt}{331.654724pt}}
\pgflineto{\pgfpoint{235.583969pt}{325.477905pt}}
\pgflineto{\pgfpoint{226.655975pt}{325.477905pt}}
\pgfpathclose
\pgfusepath{fill,stroke}
\pgfpathmoveto{\pgfpoint{226.655975pt}{331.654724pt}}
\pgflineto{\pgfpoint{235.583969pt}{331.654724pt}}
\pgflineto{\pgfpoint{235.583969pt}{325.477905pt}}
\pgfpathclose
\pgfusepath{fill,stroke}
\color[rgb]{0.748885,0.874522,0.145038}
\pgfpathmoveto{\pgfpoint{226.655975pt}{337.831604pt}}
\pgflineto{\pgfpoint{235.583969pt}{331.654724pt}}
\pgflineto{\pgfpoint{226.655975pt}{331.654724pt}}
\pgfpathclose
\pgfusepath{fill,stroke}
\pgfpathmoveto{\pgfpoint{226.655975pt}{337.831604pt}}
\pgflineto{\pgfpoint{235.583969pt}{337.831604pt}}
\pgflineto{\pgfpoint{235.583969pt}{331.654724pt}}
\pgfpathclose
\pgfusepath{fill,stroke}
\color[rgb]{0.580861,0.845663,0.253001}
\pgfpathmoveto{\pgfpoint{235.583969pt}{306.947388pt}}
\pgflineto{\pgfpoint{244.511993pt}{300.770538pt}}
\pgflineto{\pgfpoint{235.583969pt}{300.770538pt}}
\pgfpathclose
\pgfusepath{fill,stroke}
\pgfpathmoveto{\pgfpoint{235.583969pt}{306.947388pt}}
\pgflineto{\pgfpoint{244.511993pt}{306.947388pt}}
\pgflineto{\pgfpoint{244.511993pt}{300.770538pt}}
\pgfpathclose
\pgfusepath{fill,stroke}
\color[rgb]{0.622171,0.853816,0.226224}
\pgfpathmoveto{\pgfpoint{235.583969pt}{313.124207pt}}
\pgflineto{\pgfpoint{244.511993pt}{306.947388pt}}
\pgflineto{\pgfpoint{235.583969pt}{306.947388pt}}
\pgfpathclose
\pgfusepath{fill,stroke}
\pgfpathmoveto{\pgfpoint{235.583969pt}{313.124207pt}}
\pgflineto{\pgfpoint{244.511993pt}{313.124207pt}}
\pgflineto{\pgfpoint{244.511993pt}{306.947388pt}}
\pgfpathclose
\pgfusepath{fill,stroke}
\color[rgb]{0.664087,0.861321,0.198879}
\pgfpathmoveto{\pgfpoint{235.583969pt}{319.301056pt}}
\pgflineto{\pgfpoint{244.511993pt}{313.124207pt}}
\pgflineto{\pgfpoint{235.583969pt}{313.124207pt}}
\pgfpathclose
\pgfusepath{fill,stroke}
\pgfpathmoveto{\pgfpoint{235.583969pt}{319.301056pt}}
\pgflineto{\pgfpoint{244.511993pt}{319.301056pt}}
\pgflineto{\pgfpoint{244.511993pt}{313.124207pt}}
\pgfpathclose
\pgfusepath{fill,stroke}
\color[rgb]{0.706404,0.868206,0.171495}
\pgfpathmoveto{\pgfpoint{235.583969pt}{325.477905pt}}
\pgflineto{\pgfpoint{244.511993pt}{319.301056pt}}
\pgflineto{\pgfpoint{235.583969pt}{319.301056pt}}
\pgfpathclose
\pgfusepath{fill,stroke}
\pgfpathmoveto{\pgfpoint{235.583969pt}{325.477905pt}}
\pgflineto{\pgfpoint{244.511993pt}{325.477905pt}}
\pgflineto{\pgfpoint{244.511993pt}{319.301056pt}}
\pgfpathclose
\pgfusepath{fill,stroke}
\color[rgb]{0.748885,0.874522,0.145038}
\pgfpathmoveto{\pgfpoint{235.583969pt}{331.654724pt}}
\pgflineto{\pgfpoint{244.511993pt}{325.477905pt}}
\pgflineto{\pgfpoint{235.583969pt}{325.477905pt}}
\pgfpathclose
\pgfusepath{fill,stroke}
\pgfpathmoveto{\pgfpoint{235.583969pt}{331.654724pt}}
\pgflineto{\pgfpoint{244.511993pt}{331.654724pt}}
\pgflineto{\pgfpoint{244.511993pt}{325.477905pt}}
\pgfpathclose
\pgfusepath{fill,stroke}
\color[rgb]{0.791273,0.880346,0.121291}
\pgfpathmoveto{\pgfpoint{235.583969pt}{337.831604pt}}
\pgflineto{\pgfpoint{244.511993pt}{331.654724pt}}
\pgflineto{\pgfpoint{235.583969pt}{331.654724pt}}
\pgfpathclose
\pgfusepath{fill,stroke}
\pgfpathmoveto{\pgfpoint{235.583969pt}{337.831604pt}}
\pgflineto{\pgfpoint{244.511993pt}{337.831604pt}}
\pgflineto{\pgfpoint{244.511993pt}{331.654724pt}}
\pgfpathclose
\pgfusepath{fill,stroke}
\pgfpathmoveto{\pgfpoint{235.583969pt}{344.008423pt}}
\pgflineto{\pgfpoint{244.511993pt}{337.831604pt}}
\pgflineto{\pgfpoint{235.583969pt}{337.831604pt}}
\pgfpathclose
\pgfusepath{fill,stroke}
\pgfpathmoveto{\pgfpoint{235.583969pt}{344.008423pt}}
\pgflineto{\pgfpoint{244.511993pt}{344.008423pt}}
\pgflineto{\pgfpoint{244.511993pt}{337.831604pt}}
\pgfpathclose
\pgfusepath{fill,stroke}
\color[rgb]{0.706404,0.868206,0.171495}
\pgfpathmoveto{\pgfpoint{244.511993pt}{319.301056pt}}
\pgflineto{\pgfpoint{253.440002pt}{313.124207pt}}
\pgflineto{\pgfpoint{244.511993pt}{313.124207pt}}
\pgfpathclose
\pgfusepath{fill,stroke}
\pgfpathmoveto{\pgfpoint{244.511993pt}{319.301056pt}}
\pgflineto{\pgfpoint{253.440002pt}{319.301056pt}}
\pgflineto{\pgfpoint{253.440002pt}{313.124207pt}}
\pgfpathclose
\pgfusepath{fill,stroke}
\color[rgb]{0.748885,0.874522,0.145038}
\pgfpathmoveto{\pgfpoint{244.511993pt}{325.477905pt}}
\pgflineto{\pgfpoint{253.440002pt}{319.301056pt}}
\pgflineto{\pgfpoint{244.511993pt}{319.301056pt}}
\pgfpathclose
\pgfusepath{fill,stroke}
\pgfpathmoveto{\pgfpoint{244.511993pt}{325.477905pt}}
\pgflineto{\pgfpoint{253.440002pt}{325.477905pt}}
\pgflineto{\pgfpoint{253.440002pt}{319.301056pt}}
\pgfpathclose
\pgfusepath{fill,stroke}
\color[rgb]{0.791273,0.880346,0.121291}
\pgfpathmoveto{\pgfpoint{244.511993pt}{331.654724pt}}
\pgflineto{\pgfpoint{253.440002pt}{325.477905pt}}
\pgflineto{\pgfpoint{244.511993pt}{325.477905pt}}
\pgfpathclose
\pgfusepath{fill,stroke}
\pgfpathmoveto{\pgfpoint{244.511993pt}{331.654724pt}}
\pgflineto{\pgfpoint{253.440002pt}{331.654724pt}}
\pgflineto{\pgfpoint{253.440002pt}{325.477905pt}}
\pgfpathclose
\pgfusepath{fill,stroke}
\color[rgb]{0.833302,0.885780,0.103326}
\pgfpathmoveto{\pgfpoint{244.511993pt}{337.831604pt}}
\pgflineto{\pgfpoint{253.440002pt}{331.654724pt}}
\pgflineto{\pgfpoint{244.511993pt}{331.654724pt}}
\pgfpathclose
\pgfusepath{fill,stroke}
\pgfpathmoveto{\pgfpoint{244.511993pt}{337.831604pt}}
\pgflineto{\pgfpoint{253.440002pt}{337.831604pt}}
\pgflineto{\pgfpoint{253.440002pt}{331.654724pt}}
\pgfpathclose
\pgfusepath{fill,stroke}
\pgfpathmoveto{\pgfpoint{253.440002pt}{331.654724pt}}
\pgflineto{\pgfpoint{262.367981pt}{325.477905pt}}
\pgflineto{\pgfpoint{253.440002pt}{325.477905pt}}
\pgfpathclose
\pgfusepath{fill,stroke}
\pgfpathmoveto{\pgfpoint{253.440002pt}{331.654724pt}}
\pgflineto{\pgfpoint{262.367981pt}{331.654724pt}}
\pgflineto{\pgfpoint{262.367981pt}{325.477905pt}}
\pgfpathclose
\pgfusepath{fill,stroke}
\color[rgb]{0.274736,0.196969,0.497250}
\pgfpathmoveto{\pgfpoint{128.447998pt}{78.404205pt}}
\pgflineto{\pgfpoint{137.376007pt}{72.227356pt}}
\pgflineto{\pgfpoint{128.447998pt}{72.227356pt}}
\pgfpathclose
\pgfusepath{fill,stroke}
\pgfpathmoveto{\pgfpoint{128.447998pt}{78.404205pt}}
\pgflineto{\pgfpoint{137.376007pt}{78.404205pt}}
\pgflineto{\pgfpoint{137.376007pt}{72.227356pt}}
\pgfpathclose
\pgfusepath{fill,stroke}
\color[rgb]{0.269982,0.216330,0.508255}
\pgfpathmoveto{\pgfpoint{137.376007pt}{72.227356pt}}
\pgflineto{\pgfpoint{146.303986pt}{72.227356pt}}
\pgflineto{\pgfpoint{146.303986pt}{66.050522pt}}
\pgfpathclose
\pgfusepath{fill,stroke}
\pgfpathmoveto{\pgfpoint{137.376007pt}{78.404205pt}}
\pgflineto{\pgfpoint{146.303986pt}{72.227356pt}}
\pgflineto{\pgfpoint{137.376007pt}{72.227356pt}}
\pgfpathclose
\pgfusepath{fill,stroke}
\pgfpathmoveto{\pgfpoint{137.376007pt}{78.404205pt}}
\pgflineto{\pgfpoint{146.303986pt}{78.404205pt}}
\pgflineto{\pgfpoint{146.303986pt}{72.227356pt}}
\pgfpathclose
\pgfusepath{fill,stroke}
\color[rgb]{0.264369,0.235405,0.517732}
\pgfpathmoveto{\pgfpoint{137.376007pt}{84.581039pt}}
\pgflineto{\pgfpoint{146.303986pt}{78.404205pt}}
\pgflineto{\pgfpoint{137.376007pt}{78.404205pt}}
\pgfpathclose
\pgfusepath{fill,stroke}
\color[rgb]{0.258026,0.254162,0.525780}
\pgfpathmoveto{\pgfpoint{173.087997pt}{59.873672pt}}
\pgflineto{\pgfpoint{182.015991pt}{53.696838pt}}
\pgflineto{\pgfpoint{173.087997pt}{53.696838pt}}
\pgfpathclose
\pgfusepath{fill,stroke}
\pgfpathmoveto{\pgfpoint{173.087997pt}{59.873672pt}}
\pgflineto{\pgfpoint{182.015991pt}{59.873672pt}}
\pgflineto{\pgfpoint{182.015991pt}{53.696838pt}}
\pgfpathclose
\pgfusepath{fill,stroke}
\color[rgb]{0.251099,0.272573,0.532522}
\pgfpathmoveto{\pgfpoint{173.087997pt}{66.050522pt}}
\pgflineto{\pgfpoint{182.015991pt}{66.050522pt}}
\pgflineto{\pgfpoint{182.015991pt}{59.873672pt}}
\pgfpathclose
\pgfusepath{fill,stroke}
\color[rgb]{0.243733,0.290620,0.538097}
\pgfpathmoveto{\pgfpoint{173.087997pt}{72.227356pt}}
\pgflineto{\pgfpoint{182.015991pt}{66.050522pt}}
\pgflineto{\pgfpoint{173.087997pt}{66.050522pt}}
\pgfpathclose
\pgfusepath{fill,stroke}
\color[rgb]{0.258026,0.254162,0.525780}
\pgfpathmoveto{\pgfpoint{182.015991pt}{53.696838pt}}
\pgflineto{\pgfpoint{190.943985pt}{53.696838pt}}
\pgflineto{\pgfpoint{190.943985pt}{47.519989pt}}
\pgfpathclose
\pgfusepath{fill,stroke}
\color[rgb]{0.251099,0.272573,0.532522}
\pgfpathmoveto{\pgfpoint{182.015991pt}{59.873672pt}}
\pgflineto{\pgfpoint{190.943985pt}{53.696838pt}}
\pgflineto{\pgfpoint{182.015991pt}{53.696838pt}}
\pgfpathclose
\pgfusepath{fill,stroke}
\pgfpathmoveto{\pgfpoint{182.015991pt}{59.873672pt}}
\pgflineto{\pgfpoint{190.943985pt}{59.873672pt}}
\pgflineto{\pgfpoint{190.943985pt}{53.696838pt}}
\pgfpathclose
\pgfusepath{fill,stroke}
\color[rgb]{0.243733,0.290620,0.538097}
\pgfpathmoveto{\pgfpoint{182.015991pt}{66.050522pt}}
\pgflineto{\pgfpoint{190.943985pt}{59.873672pt}}
\pgflineto{\pgfpoint{182.015991pt}{59.873672pt}}
\pgfpathclose
\pgfusepath{fill,stroke}
\pgfpathmoveto{\pgfpoint{190.943985pt}{53.696838pt}}
\pgflineto{\pgfpoint{199.871979pt}{47.519989pt}}
\pgflineto{\pgfpoint{190.943985pt}{47.519989pt}}
\pgfpathclose
\pgfusepath{fill,stroke}
\pgfpathmoveto{\pgfpoint{190.943985pt}{53.696838pt}}
\pgflineto{\pgfpoint{199.871979pt}{53.696838pt}}
\pgflineto{\pgfpoint{199.871979pt}{47.519989pt}}
\pgfpathclose
\pgfusepath{fill,stroke}
\color[rgb]{0.236073,0.308291,0.542652}
\pgfpathmoveto{\pgfpoint{199.871979pt}{53.696838pt}}
\pgflineto{\pgfpoint{208.799988pt}{47.519989pt}}
\pgflineto{\pgfpoint{199.871979pt}{47.519989pt}}
\pgfpathclose
\pgfusepath{fill,stroke}
\pgfpathmoveto{\pgfpoint{199.871979pt}{53.696838pt}}
\pgflineto{\pgfpoint{208.799988pt}{53.696838pt}}
\pgflineto{\pgfpoint{208.799988pt}{47.519989pt}}
\pgfpathclose
\pgfusepath{fill,stroke}
\pgfpathmoveto{\pgfpoint{199.871979pt}{59.873672pt}}
\pgflineto{\pgfpoint{208.799988pt}{53.696838pt}}
\pgflineto{\pgfpoint{199.871979pt}{53.696838pt}}
\pgfpathclose
\pgfusepath{fill,stroke}
\color[rgb]{0.220425,0.342517,0.549287}
\pgfpathmoveto{\pgfpoint{217.727982pt}{53.696838pt}}
\pgflineto{\pgfpoint{226.655975pt}{47.519989pt}}
\pgflineto{\pgfpoint{217.727982pt}{47.519989pt}}
\pgfpathclose
\pgfusepath{fill,stroke}
\color[rgb]{0.269982,0.216330,0.508255}
\pgfpathmoveto{\pgfpoint{146.303986pt}{66.050522pt}}
\pgflineto{\pgfpoint{155.231979pt}{66.050522pt}}
\pgflineto{\pgfpoint{155.231979pt}{59.873672pt}}
\pgfpathclose
\pgfusepath{fill,stroke}
\color[rgb]{0.264369,0.235405,0.517732}
\pgfpathmoveto{\pgfpoint{146.303986pt}{72.227356pt}}
\pgflineto{\pgfpoint{155.231979pt}{66.050522pt}}
\pgflineto{\pgfpoint{146.303986pt}{66.050522pt}}
\pgfpathclose
\pgfusepath{fill,stroke}
\pgfpathmoveto{\pgfpoint{155.231979pt}{66.050522pt}}
\pgflineto{\pgfpoint{164.160004pt}{59.873672pt}}
\pgflineto{\pgfpoint{155.231979pt}{59.873672pt}}
\pgfpathclose
\pgfusepath{fill,stroke}
\pgfpathmoveto{\pgfpoint{155.231979pt}{66.050522pt}}
\pgflineto{\pgfpoint{164.160004pt}{66.050522pt}}
\pgflineto{\pgfpoint{164.160004pt}{59.873672pt}}
\pgfpathclose
\pgfusepath{fill,stroke}
\color[rgb]{0.258026,0.254162,0.525780}
\pgfpathmoveto{\pgfpoint{155.231979pt}{72.227356pt}}
\pgflineto{\pgfpoint{164.160004pt}{72.227356pt}}
\pgflineto{\pgfpoint{164.160004pt}{66.050522pt}}
\pgfpathclose
\pgfusepath{fill,stroke}
\color[rgb]{0.251099,0.272573,0.532522}
\pgfpathmoveto{\pgfpoint{155.231979pt}{78.404205pt}}
\pgflineto{\pgfpoint{164.160004pt}{72.227356pt}}
\pgflineto{\pgfpoint{155.231979pt}{72.227356pt}}
\pgfpathclose
\pgfusepath{fill,stroke}
\color[rgb]{0.264369,0.235405,0.517732}
\pgfpathmoveto{\pgfpoint{164.160004pt}{59.873672pt}}
\pgflineto{\pgfpoint{173.087997pt}{59.873672pt}}
\pgflineto{\pgfpoint{173.087997pt}{53.696838pt}}
\pgfpathclose
\pgfusepath{fill,stroke}
\color[rgb]{0.258026,0.254162,0.525780}
\pgfpathmoveto{\pgfpoint{164.160004pt}{66.050522pt}}
\pgflineto{\pgfpoint{173.087997pt}{59.873672pt}}
\pgflineto{\pgfpoint{164.160004pt}{59.873672pt}}
\pgfpathclose
\pgfusepath{fill,stroke}
\color[rgb]{0.269982,0.216330,0.508255}
\pgfpathmoveto{\pgfpoint{119.519989pt}{90.757896pt}}
\pgflineto{\pgfpoint{128.447998pt}{90.757896pt}}
\pgflineto{\pgfpoint{128.447998pt}{84.581039pt}}
\pgfpathclose
\pgfusepath{fill,stroke}
\color[rgb]{0.264369,0.235405,0.517732}
\pgfpathmoveto{\pgfpoint{119.519989pt}{96.934731pt}}
\pgflineto{\pgfpoint{128.447998pt}{90.757896pt}}
\pgflineto{\pgfpoint{119.519989pt}{90.757896pt}}
\pgfpathclose
\pgfusepath{fill,stroke}
\pgfpathmoveto{\pgfpoint{119.519989pt}{96.934731pt}}
\pgflineto{\pgfpoint{128.447998pt}{96.934731pt}}
\pgflineto{\pgfpoint{128.447998pt}{90.757896pt}}
\pgfpathclose
\pgfusepath{fill,stroke}
\color[rgb]{0.269982,0.216330,0.508255}
\pgfpathmoveto{\pgfpoint{128.447998pt}{84.581039pt}}
\pgflineto{\pgfpoint{137.376007pt}{78.404205pt}}
\pgflineto{\pgfpoint{128.447998pt}{78.404205pt}}
\pgfpathclose
\pgfusepath{fill,stroke}
\pgfpathmoveto{\pgfpoint{128.447998pt}{84.581039pt}}
\pgflineto{\pgfpoint{137.376007pt}{84.581039pt}}
\pgflineto{\pgfpoint{137.376007pt}{78.404205pt}}
\pgfpathclose
\pgfusepath{fill,stroke}
\color[rgb]{0.264369,0.235405,0.517732}
\pgfpathmoveto{\pgfpoint{128.447998pt}{90.757896pt}}
\pgflineto{\pgfpoint{137.376007pt}{84.581039pt}}
\pgflineto{\pgfpoint{128.447998pt}{84.581039pt}}
\pgfpathclose
\pgfusepath{fill,stroke}
\pgfpathmoveto{\pgfpoint{128.447998pt}{90.757896pt}}
\pgflineto{\pgfpoint{137.376007pt}{90.757896pt}}
\pgflineto{\pgfpoint{137.376007pt}{84.581039pt}}
\pgfpathclose
\pgfusepath{fill,stroke}
\color[rgb]{0.258026,0.254162,0.525780}
\pgfpathmoveto{\pgfpoint{128.447998pt}{96.934731pt}}
\pgflineto{\pgfpoint{137.376007pt}{90.757896pt}}
\pgflineto{\pgfpoint{128.447998pt}{90.757896pt}}
\pgfpathclose
\pgfusepath{fill,stroke}
\color[rgb]{0.264369,0.235405,0.517732}
\pgfpathmoveto{\pgfpoint{137.376007pt}{84.581039pt}}
\pgflineto{\pgfpoint{146.303986pt}{84.581039pt}}
\pgflineto{\pgfpoint{146.303986pt}{78.404205pt}}
\pgfpathclose
\pgfusepath{fill,stroke}
\color[rgb]{0.258026,0.254162,0.525780}
\pgfpathmoveto{\pgfpoint{137.376007pt}{90.757896pt}}
\pgflineto{\pgfpoint{146.303986pt}{84.581039pt}}
\pgflineto{\pgfpoint{137.376007pt}{84.581039pt}}
\pgfpathclose
\pgfusepath{fill,stroke}
\pgfpathmoveto{\pgfpoint{137.376007pt}{90.757896pt}}
\pgflineto{\pgfpoint{146.303986pt}{90.757896pt}}
\pgflineto{\pgfpoint{146.303986pt}{84.581039pt}}
\pgfpathclose
\pgfusepath{fill,stroke}
\color[rgb]{0.264369,0.235405,0.517732}
\pgfpathmoveto{\pgfpoint{146.303986pt}{72.227356pt}}
\pgflineto{\pgfpoint{155.231979pt}{72.227356pt}}
\pgflineto{\pgfpoint{155.231979pt}{66.050522pt}}
\pgfpathclose
\pgfusepath{fill,stroke}
\color[rgb]{0.258026,0.254162,0.525780}
\pgfpathmoveto{\pgfpoint{146.303986pt}{78.404205pt}}
\pgflineto{\pgfpoint{155.231979pt}{72.227356pt}}
\pgflineto{\pgfpoint{146.303986pt}{72.227356pt}}
\pgfpathclose
\pgfusepath{fill,stroke}
\pgfpathmoveto{\pgfpoint{146.303986pt}{78.404205pt}}
\pgflineto{\pgfpoint{155.231979pt}{78.404205pt}}
\pgflineto{\pgfpoint{155.231979pt}{72.227356pt}}
\pgfpathclose
\pgfusepath{fill,stroke}
\pgfpathmoveto{\pgfpoint{146.303986pt}{84.581039pt}}
\pgflineto{\pgfpoint{155.231979pt}{78.404205pt}}
\pgflineto{\pgfpoint{146.303986pt}{78.404205pt}}
\pgfpathclose
\pgfusepath{fill,stroke}
\pgfpathmoveto{\pgfpoint{146.303986pt}{84.581039pt}}
\pgflineto{\pgfpoint{155.231979pt}{84.581039pt}}
\pgflineto{\pgfpoint{155.231979pt}{78.404205pt}}
\pgfpathclose
\pgfusepath{fill,stroke}
\color[rgb]{0.251099,0.272573,0.532522}
\pgfpathmoveto{\pgfpoint{146.303986pt}{90.757896pt}}
\pgflineto{\pgfpoint{155.231979pt}{84.581039pt}}
\pgflineto{\pgfpoint{146.303986pt}{84.581039pt}}
\pgfpathclose
\pgfusepath{fill,stroke}
\color[rgb]{0.258026,0.254162,0.525780}
\pgfpathmoveto{\pgfpoint{155.231979pt}{72.227356pt}}
\pgflineto{\pgfpoint{164.160004pt}{66.050522pt}}
\pgflineto{\pgfpoint{155.231979pt}{66.050522pt}}
\pgfpathclose
\pgfusepath{fill,stroke}
\color[rgb]{0.251099,0.272573,0.532522}
\pgfpathmoveto{\pgfpoint{155.231979pt}{78.404205pt}}
\pgflineto{\pgfpoint{164.160004pt}{78.404205pt}}
\pgflineto{\pgfpoint{164.160004pt}{72.227356pt}}
\pgfpathclose
\pgfusepath{fill,stroke}
\pgfpathmoveto{\pgfpoint{155.231979pt}{84.581039pt}}
\pgflineto{\pgfpoint{164.160004pt}{78.404205pt}}
\pgflineto{\pgfpoint{155.231979pt}{78.404205pt}}
\pgfpathclose
\pgfusepath{fill,stroke}
\color[rgb]{0.258026,0.254162,0.525780}
\pgfpathmoveto{\pgfpoint{164.160004pt}{66.050522pt}}
\pgflineto{\pgfpoint{173.087997pt}{66.050522pt}}
\pgflineto{\pgfpoint{173.087997pt}{59.873672pt}}
\pgfpathclose
\pgfusepath{fill,stroke}
\color[rgb]{0.251099,0.272573,0.532522}
\pgfpathmoveto{\pgfpoint{164.160004pt}{72.227356pt}}
\pgflineto{\pgfpoint{173.087997pt}{66.050522pt}}
\pgflineto{\pgfpoint{164.160004pt}{66.050522pt}}
\pgfpathclose
\pgfusepath{fill,stroke}
\pgfpathmoveto{\pgfpoint{164.160004pt}{72.227356pt}}
\pgflineto{\pgfpoint{173.087997pt}{72.227356pt}}
\pgflineto{\pgfpoint{173.087997pt}{66.050522pt}}
\pgfpathclose
\pgfusepath{fill,stroke}
\color[rgb]{0.243733,0.290620,0.538097}
\pgfpathmoveto{\pgfpoint{164.160004pt}{78.404205pt}}
\pgflineto{\pgfpoint{173.087997pt}{72.227356pt}}
\pgflineto{\pgfpoint{164.160004pt}{72.227356pt}}
\pgfpathclose
\pgfusepath{fill,stroke}
\pgfpathmoveto{\pgfpoint{164.160004pt}{78.404205pt}}
\pgflineto{\pgfpoint{173.087997pt}{78.404205pt}}
\pgflineto{\pgfpoint{173.087997pt}{72.227356pt}}
\pgfpathclose
\pgfusepath{fill,stroke}
\color[rgb]{0.251099,0.272573,0.532522}
\pgfpathmoveto{\pgfpoint{173.087997pt}{66.050522pt}}
\pgflineto{\pgfpoint{182.015991pt}{59.873672pt}}
\pgflineto{\pgfpoint{173.087997pt}{59.873672pt}}
\pgfpathclose
\pgfusepath{fill,stroke}
\color[rgb]{0.243733,0.290620,0.538097}
\pgfpathmoveto{\pgfpoint{173.087997pt}{72.227356pt}}
\pgflineto{\pgfpoint{182.015991pt}{72.227356pt}}
\pgflineto{\pgfpoint{182.015991pt}{66.050522pt}}
\pgfpathclose
\pgfusepath{fill,stroke}
\color[rgb]{0.236073,0.308291,0.542652}
\pgfpathmoveto{\pgfpoint{173.087997pt}{78.404205pt}}
\pgflineto{\pgfpoint{182.015991pt}{72.227356pt}}
\pgflineto{\pgfpoint{173.087997pt}{72.227356pt}}
\pgfpathclose
\pgfusepath{fill,stroke}
\color[rgb]{0.243733,0.290620,0.538097}
\pgfpathmoveto{\pgfpoint{182.015991pt}{66.050522pt}}
\pgflineto{\pgfpoint{190.943985pt}{66.050522pt}}
\pgflineto{\pgfpoint{190.943985pt}{59.873672pt}}
\pgfpathclose
\pgfusepath{fill,stroke}
\color[rgb]{0.236073,0.308291,0.542652}
\pgfpathmoveto{\pgfpoint{182.015991pt}{72.227356pt}}
\pgflineto{\pgfpoint{190.943985pt}{66.050522pt}}
\pgflineto{\pgfpoint{182.015991pt}{66.050522pt}}
\pgfpathclose
\pgfusepath{fill,stroke}
\pgfpathmoveto{\pgfpoint{182.015991pt}{72.227356pt}}
\pgflineto{\pgfpoint{190.943985pt}{72.227356pt}}
\pgflineto{\pgfpoint{190.943985pt}{66.050522pt}}
\pgfpathclose
\pgfusepath{fill,stroke}
\color[rgb]{0.243733,0.290620,0.538097}
\pgfpathmoveto{\pgfpoint{190.943985pt}{59.873672pt}}
\pgflineto{\pgfpoint{199.871979pt}{53.696838pt}}
\pgflineto{\pgfpoint{190.943985pt}{53.696838pt}}
\pgfpathclose
\pgfusepath{fill,stroke}
\pgfpathmoveto{\pgfpoint{190.943985pt}{59.873672pt}}
\pgflineto{\pgfpoint{199.871979pt}{59.873672pt}}
\pgflineto{\pgfpoint{199.871979pt}{53.696838pt}}
\pgfpathclose
\pgfusepath{fill,stroke}
\color[rgb]{0.236073,0.308291,0.542652}
\pgfpathmoveto{\pgfpoint{190.943985pt}{66.050522pt}}
\pgflineto{\pgfpoint{199.871979pt}{59.873672pt}}
\pgflineto{\pgfpoint{190.943985pt}{59.873672pt}}
\pgfpathclose
\pgfusepath{fill,stroke}
\pgfpathmoveto{\pgfpoint{190.943985pt}{66.050522pt}}
\pgflineto{\pgfpoint{199.871979pt}{66.050522pt}}
\pgflineto{\pgfpoint{199.871979pt}{59.873672pt}}
\pgfpathclose
\pgfusepath{fill,stroke}
\color[rgb]{0.228263,0.325586,0.546335}
\pgfpathmoveto{\pgfpoint{190.943985pt}{72.227356pt}}
\pgflineto{\pgfpoint{199.871979pt}{66.050522pt}}
\pgflineto{\pgfpoint{190.943985pt}{66.050522pt}}
\pgfpathclose
\pgfusepath{fill,stroke}
\color[rgb]{0.236073,0.308291,0.542652}
\pgfpathmoveto{\pgfpoint{199.871979pt}{59.873672pt}}
\pgflineto{\pgfpoint{208.799988pt}{59.873672pt}}
\pgflineto{\pgfpoint{208.799988pt}{53.696838pt}}
\pgfpathclose
\pgfusepath{fill,stroke}
\color[rgb]{0.228263,0.325586,0.546335}
\pgfpathmoveto{\pgfpoint{199.871979pt}{66.050522pt}}
\pgflineto{\pgfpoint{208.799988pt}{59.873672pt}}
\pgflineto{\pgfpoint{199.871979pt}{59.873672pt}}
\pgfpathclose
\pgfusepath{fill,stroke}
\pgfpathmoveto{\pgfpoint{199.871979pt}{66.050522pt}}
\pgflineto{\pgfpoint{208.799988pt}{66.050522pt}}
\pgflineto{\pgfpoint{208.799988pt}{59.873672pt}}
\pgfpathclose
\pgfusepath{fill,stroke}
\pgfpathmoveto{\pgfpoint{208.799988pt}{53.696838pt}}
\pgflineto{\pgfpoint{217.727982pt}{47.519989pt}}
\pgflineto{\pgfpoint{208.799988pt}{47.519989pt}}
\pgfpathclose
\pgfusepath{fill,stroke}
\pgfpathmoveto{\pgfpoint{208.799988pt}{53.696838pt}}
\pgflineto{\pgfpoint{217.727982pt}{53.696838pt}}
\pgflineto{\pgfpoint{217.727982pt}{47.519989pt}}
\pgfpathclose
\pgfusepath{fill,stroke}
\color[rgb]{0.220425,0.342517,0.549287}
\pgfpathmoveto{\pgfpoint{208.799988pt}{59.873672pt}}
\pgflineto{\pgfpoint{217.727982pt}{53.696838pt}}
\pgflineto{\pgfpoint{208.799988pt}{53.696838pt}}
\pgfpathclose
\pgfusepath{fill,stroke}
\pgfpathmoveto{\pgfpoint{208.799988pt}{59.873672pt}}
\pgflineto{\pgfpoint{217.727982pt}{59.873672pt}}
\pgflineto{\pgfpoint{217.727982pt}{53.696838pt}}
\pgfpathclose
\pgfusepath{fill,stroke}
\pgfpathmoveto{\pgfpoint{208.799988pt}{66.050522pt}}
\pgflineto{\pgfpoint{217.727982pt}{59.873672pt}}
\pgflineto{\pgfpoint{208.799988pt}{59.873672pt}}
\pgfpathclose
\pgfusepath{fill,stroke}
\pgfpathmoveto{\pgfpoint{217.727982pt}{53.696838pt}}
\pgflineto{\pgfpoint{226.655975pt}{53.696838pt}}
\pgflineto{\pgfpoint{226.655975pt}{47.519989pt}}
\pgfpathclose
\pgfusepath{fill,stroke}
\color[rgb]{0.212667,0.359102,0.551635}
\pgfpathmoveto{\pgfpoint{217.727982pt}{59.873672pt}}
\pgflineto{\pgfpoint{226.655975pt}{53.696838pt}}
\pgflineto{\pgfpoint{217.727982pt}{53.696838pt}}
\pgfpathclose
\pgfusepath{fill,stroke}
\pgfpathmoveto{\pgfpoint{226.655975pt}{53.696838pt}}
\pgflineto{\pgfpoint{235.583969pt}{47.519989pt}}
\pgflineto{\pgfpoint{226.655975pt}{47.519989pt}}
\pgfpathclose
\pgfusepath{fill,stroke}
\pgfpathmoveto{\pgfpoint{226.655975pt}{53.696838pt}}
\pgflineto{\pgfpoint{235.583969pt}{53.696838pt}}
\pgflineto{\pgfpoint{235.583969pt}{47.519989pt}}
\pgfpathclose
\pgfusepath{fill,stroke}
\color[rgb]{0.205079,0.375366,0.553493}
\pgfpathmoveto{\pgfpoint{235.583969pt}{53.696838pt}}
\pgflineto{\pgfpoint{244.511993pt}{47.519989pt}}
\pgflineto{\pgfpoint{235.583969pt}{47.519989pt}}
\pgfpathclose
\pgfusepath{fill,stroke}
\color[rgb]{0.197722,0.391341,0.554953}
\pgfpathmoveto{\pgfpoint{244.511993pt}{53.696838pt}}
\pgflineto{\pgfpoint{253.440002pt}{53.696838pt}}
\pgflineto{\pgfpoint{253.440002pt}{47.519989pt}}
\pgfpathclose
\pgfusepath{fill,stroke}
\color[rgb]{0.190631,0.407061,0.556089}
\pgfpathmoveto{\pgfpoint{244.511993pt}{59.873672pt}}
\pgflineto{\pgfpoint{253.440002pt}{53.696838pt}}
\pgflineto{\pgfpoint{244.511993pt}{53.696838pt}}
\pgfpathclose
\pgfusepath{fill,stroke}
\pgfpathmoveto{\pgfpoint{253.440002pt}{53.696838pt}}
\pgflineto{\pgfpoint{262.367981pt}{47.519989pt}}
\pgflineto{\pgfpoint{253.440002pt}{47.519989pt}}
\pgfpathclose
\pgfusepath{fill,stroke}
\color[rgb]{0.220425,0.342517,0.549287}
\pgfpathmoveto{\pgfpoint{208.799988pt}{66.050522pt}}
\pgflineto{\pgfpoint{217.727982pt}{66.050522pt}}
\pgflineto{\pgfpoint{217.727982pt}{59.873672pt}}
\pgfpathclose
\pgfusepath{fill,stroke}
\color[rgb]{0.212667,0.359102,0.551635}
\pgfpathmoveto{\pgfpoint{208.799988pt}{72.227356pt}}
\pgflineto{\pgfpoint{217.727982pt}{66.050522pt}}
\pgflineto{\pgfpoint{208.799988pt}{66.050522pt}}
\pgfpathclose
\pgfusepath{fill,stroke}
\color[rgb]{0.205079,0.375366,0.553493}
\pgfpathmoveto{\pgfpoint{226.655975pt}{59.873672pt}}
\pgflineto{\pgfpoint{235.583969pt}{59.873672pt}}
\pgflineto{\pgfpoint{235.583969pt}{53.696838pt}}
\pgfpathclose
\pgfusepath{fill,stroke}
\color[rgb]{0.197722,0.391341,0.554953}
\pgfpathmoveto{\pgfpoint{226.655975pt}{66.050522pt}}
\pgflineto{\pgfpoint{235.583969pt}{59.873672pt}}
\pgflineto{\pgfpoint{226.655975pt}{59.873672pt}}
\pgfpathclose
\pgfusepath{fill,stroke}
\color[rgb]{0.236073,0.308291,0.542652}
\pgfpathmoveto{\pgfpoint{164.160004pt}{84.581039pt}}
\pgflineto{\pgfpoint{173.087997pt}{84.581039pt}}
\pgflineto{\pgfpoint{173.087997pt}{78.404205pt}}
\pgfpathclose
\pgfusepath{fill,stroke}
\pgfpathmoveto{\pgfpoint{164.160004pt}{90.757896pt}}
\pgflineto{\pgfpoint{173.087997pt}{84.581039pt}}
\pgflineto{\pgfpoint{164.160004pt}{84.581039pt}}
\pgfpathclose
\pgfusepath{fill,stroke}
\color[rgb]{0.228263,0.325586,0.546335}
\pgfpathmoveto{\pgfpoint{182.015991pt}{78.404205pt}}
\pgflineto{\pgfpoint{190.943985pt}{78.404205pt}}
\pgflineto{\pgfpoint{190.943985pt}{72.227356pt}}
\pgfpathclose
\pgfusepath{fill,stroke}
\color[rgb]{0.220425,0.342517,0.549287}
\pgfpathmoveto{\pgfpoint{182.015991pt}{84.581039pt}}
\pgflineto{\pgfpoint{190.943985pt}{78.404205pt}}
\pgflineto{\pgfpoint{182.015991pt}{78.404205pt}}
\pgfpathclose
\pgfusepath{fill,stroke}
\color[rgb]{0.228263,0.325586,0.546335}
\pgfpathmoveto{\pgfpoint{190.943985pt}{72.227356pt}}
\pgflineto{\pgfpoint{199.871979pt}{72.227356pt}}
\pgflineto{\pgfpoint{199.871979pt}{66.050522pt}}
\pgfpathclose
\pgfusepath{fill,stroke}
\color[rgb]{0.220425,0.342517,0.549287}
\pgfpathmoveto{\pgfpoint{190.943985pt}{78.404205pt}}
\pgflineto{\pgfpoint{199.871979pt}{72.227356pt}}
\pgflineto{\pgfpoint{190.943985pt}{72.227356pt}}
\pgfpathclose
\pgfusepath{fill,stroke}
\color[rgb]{0.258026,0.254162,0.525780}
\pgfpathmoveto{\pgfpoint{128.447998pt}{96.934731pt}}
\pgflineto{\pgfpoint{137.376007pt}{96.934731pt}}
\pgflineto{\pgfpoint{137.376007pt}{90.757896pt}}
\pgfpathclose
\pgfusepath{fill,stroke}
\color[rgb]{0.251099,0.272573,0.532522}
\pgfpathmoveto{\pgfpoint{128.447998pt}{103.111580pt}}
\pgflineto{\pgfpoint{137.376007pt}{96.934731pt}}
\pgflineto{\pgfpoint{128.447998pt}{96.934731pt}}
\pgfpathclose
\pgfusepath{fill,stroke}
\pgfpathmoveto{\pgfpoint{146.303986pt}{90.757896pt}}
\pgflineto{\pgfpoint{155.231979pt}{90.757896pt}}
\pgflineto{\pgfpoint{155.231979pt}{84.581039pt}}
\pgfpathclose
\pgfusepath{fill,stroke}
\color[rgb]{0.243733,0.290620,0.538097}
\pgfpathmoveto{\pgfpoint{146.303986pt}{96.934731pt}}
\pgflineto{\pgfpoint{155.231979pt}{90.757896pt}}
\pgflineto{\pgfpoint{146.303986pt}{90.757896pt}}
\pgfpathclose
\pgfusepath{fill,stroke}
\color[rgb]{0.264369,0.235405,0.517732}
\pgfpathmoveto{\pgfpoint{110.591980pt}{109.288422pt}}
\pgflineto{\pgfpoint{119.519989pt}{109.288422pt}}
\pgflineto{\pgfpoint{119.519989pt}{103.111580pt}}
\pgfpathclose
\pgfusepath{fill,stroke}
\color[rgb]{0.258026,0.254162,0.525780}
\pgfpathmoveto{\pgfpoint{119.519989pt}{103.111580pt}}
\pgflineto{\pgfpoint{128.447998pt}{96.934731pt}}
\pgflineto{\pgfpoint{119.519989pt}{96.934731pt}}
\pgfpathclose
\pgfusepath{fill,stroke}
\pgfpathmoveto{\pgfpoint{119.519989pt}{103.111580pt}}
\pgflineto{\pgfpoint{128.447998pt}{103.111580pt}}
\pgflineto{\pgfpoint{128.447998pt}{96.934731pt}}
\pgfpathclose
\pgfusepath{fill,stroke}
\pgfpathmoveto{\pgfpoint{119.519989pt}{109.288422pt}}
\pgflineto{\pgfpoint{128.447998pt}{103.111580pt}}
\pgflineto{\pgfpoint{119.519989pt}{103.111580pt}}
\pgfpathclose
\pgfusepath{fill,stroke}
\pgfpathmoveto{\pgfpoint{119.519989pt}{109.288422pt}}
\pgflineto{\pgfpoint{128.447998pt}{109.288422pt}}
\pgflineto{\pgfpoint{128.447998pt}{103.111580pt}}
\pgfpathclose
\pgfusepath{fill,stroke}
\color[rgb]{0.251099,0.272573,0.532522}
\pgfpathmoveto{\pgfpoint{119.519989pt}{115.465263pt}}
\pgflineto{\pgfpoint{128.447998pt}{109.288422pt}}
\pgflineto{\pgfpoint{119.519989pt}{109.288422pt}}
\pgfpathclose
\pgfusepath{fill,stroke}
\pgfpathmoveto{\pgfpoint{128.447998pt}{103.111580pt}}
\pgflineto{\pgfpoint{137.376007pt}{103.111580pt}}
\pgflineto{\pgfpoint{137.376007pt}{96.934731pt}}
\pgfpathclose
\pgfusepath{fill,stroke}
\color[rgb]{0.243733,0.290620,0.538097}
\pgfpathmoveto{\pgfpoint{128.447998pt}{109.288422pt}}
\pgflineto{\pgfpoint{137.376007pt}{103.111580pt}}
\pgflineto{\pgfpoint{128.447998pt}{103.111580pt}}
\pgfpathclose
\pgfusepath{fill,stroke}
\pgfpathmoveto{\pgfpoint{128.447998pt}{109.288422pt}}
\pgflineto{\pgfpoint{137.376007pt}{109.288422pt}}
\pgflineto{\pgfpoint{137.376007pt}{103.111580pt}}
\pgfpathclose
\pgfusepath{fill,stroke}
\color[rgb]{0.251099,0.272573,0.532522}
\pgfpathmoveto{\pgfpoint{137.376007pt}{96.934731pt}}
\pgflineto{\pgfpoint{146.303986pt}{90.757896pt}}
\pgflineto{\pgfpoint{137.376007pt}{90.757896pt}}
\pgfpathclose
\pgfusepath{fill,stroke}
\pgfpathmoveto{\pgfpoint{137.376007pt}{96.934731pt}}
\pgflineto{\pgfpoint{146.303986pt}{96.934731pt}}
\pgflineto{\pgfpoint{146.303986pt}{90.757896pt}}
\pgfpathclose
\pgfusepath{fill,stroke}
\color[rgb]{0.243733,0.290620,0.538097}
\pgfpathmoveto{\pgfpoint{137.376007pt}{103.111580pt}}
\pgflineto{\pgfpoint{146.303986pt}{96.934731pt}}
\pgflineto{\pgfpoint{137.376007pt}{96.934731pt}}
\pgfpathclose
\pgfusepath{fill,stroke}
\pgfpathmoveto{\pgfpoint{137.376007pt}{103.111580pt}}
\pgflineto{\pgfpoint{146.303986pt}{103.111580pt}}
\pgflineto{\pgfpoint{146.303986pt}{96.934731pt}}
\pgfpathclose
\pgfusepath{fill,stroke}
\color[rgb]{0.236073,0.308291,0.542652}
\pgfpathmoveto{\pgfpoint{137.376007pt}{109.288422pt}}
\pgflineto{\pgfpoint{146.303986pt}{103.111580pt}}
\pgflineto{\pgfpoint{137.376007pt}{103.111580pt}}
\pgfpathclose
\pgfusepath{fill,stroke}
\color[rgb]{0.243733,0.290620,0.538097}
\pgfpathmoveto{\pgfpoint{146.303986pt}{96.934731pt}}
\pgflineto{\pgfpoint{155.231979pt}{96.934731pt}}
\pgflineto{\pgfpoint{155.231979pt}{90.757896pt}}
\pgfpathclose
\pgfusepath{fill,stroke}
\color[rgb]{0.236073,0.308291,0.542652}
\pgfpathmoveto{\pgfpoint{146.303986pt}{103.111580pt}}
\pgflineto{\pgfpoint{155.231979pt}{96.934731pt}}
\pgflineto{\pgfpoint{146.303986pt}{96.934731pt}}
\pgfpathclose
\pgfusepath{fill,stroke}
\pgfpathmoveto{\pgfpoint{146.303986pt}{103.111580pt}}
\pgflineto{\pgfpoint{155.231979pt}{103.111580pt}}
\pgflineto{\pgfpoint{155.231979pt}{96.934731pt}}
\pgfpathclose
\pgfusepath{fill,stroke}
\color[rgb]{0.251099,0.272573,0.532522}
\pgfpathmoveto{\pgfpoint{155.231979pt}{84.581039pt}}
\pgflineto{\pgfpoint{164.160004pt}{84.581039pt}}
\pgflineto{\pgfpoint{164.160004pt}{78.404205pt}}
\pgfpathclose
\pgfusepath{fill,stroke}
\color[rgb]{0.243733,0.290620,0.538097}
\pgfpathmoveto{\pgfpoint{155.231979pt}{90.757896pt}}
\pgflineto{\pgfpoint{164.160004pt}{84.581039pt}}
\pgflineto{\pgfpoint{155.231979pt}{84.581039pt}}
\pgfpathclose
\pgfusepath{fill,stroke}
\pgfpathmoveto{\pgfpoint{155.231979pt}{90.757896pt}}
\pgflineto{\pgfpoint{164.160004pt}{90.757896pt}}
\pgflineto{\pgfpoint{164.160004pt}{84.581039pt}}
\pgfpathclose
\pgfusepath{fill,stroke}
\color[rgb]{0.236073,0.308291,0.542652}
\pgfpathmoveto{\pgfpoint{155.231979pt}{96.934731pt}}
\pgflineto{\pgfpoint{164.160004pt}{90.757896pt}}
\pgflineto{\pgfpoint{155.231979pt}{90.757896pt}}
\pgfpathclose
\pgfusepath{fill,stroke}
\pgfpathmoveto{\pgfpoint{155.231979pt}{96.934731pt}}
\pgflineto{\pgfpoint{164.160004pt}{96.934731pt}}
\pgflineto{\pgfpoint{164.160004pt}{90.757896pt}}
\pgfpathclose
\pgfusepath{fill,stroke}
\color[rgb]{0.228263,0.325586,0.546335}
\pgfpathmoveto{\pgfpoint{155.231979pt}{103.111580pt}}
\pgflineto{\pgfpoint{164.160004pt}{96.934731pt}}
\pgflineto{\pgfpoint{155.231979pt}{96.934731pt}}
\pgfpathclose
\pgfusepath{fill,stroke}
\color[rgb]{0.236073,0.308291,0.542652}
\pgfpathmoveto{\pgfpoint{164.160004pt}{84.581039pt}}
\pgflineto{\pgfpoint{173.087997pt}{78.404205pt}}
\pgflineto{\pgfpoint{164.160004pt}{78.404205pt}}
\pgfpathclose
\pgfusepath{fill,stroke}
\pgfpathmoveto{\pgfpoint{164.160004pt}{90.757896pt}}
\pgflineto{\pgfpoint{173.087997pt}{90.757896pt}}
\pgflineto{\pgfpoint{173.087997pt}{84.581039pt}}
\pgfpathclose
\pgfusepath{fill,stroke}
\color[rgb]{0.228263,0.325586,0.546335}
\pgfpathmoveto{\pgfpoint{164.160004pt}{96.934731pt}}
\pgflineto{\pgfpoint{173.087997pt}{90.757896pt}}
\pgflineto{\pgfpoint{164.160004pt}{90.757896pt}}
\pgfpathclose
\pgfusepath{fill,stroke}
\pgfpathmoveto{\pgfpoint{164.160004pt}{96.934731pt}}
\pgflineto{\pgfpoint{173.087997pt}{96.934731pt}}
\pgflineto{\pgfpoint{173.087997pt}{90.757896pt}}
\pgfpathclose
\pgfusepath{fill,stroke}
\color[rgb]{0.236073,0.308291,0.542652}
\pgfpathmoveto{\pgfpoint{173.087997pt}{78.404205pt}}
\pgflineto{\pgfpoint{182.015991pt}{78.404205pt}}
\pgflineto{\pgfpoint{182.015991pt}{72.227356pt}}
\pgfpathclose
\pgfusepath{fill,stroke}
\color[rgb]{0.228263,0.325586,0.546335}
\pgfpathmoveto{\pgfpoint{173.087997pt}{84.581039pt}}
\pgflineto{\pgfpoint{182.015991pt}{78.404205pt}}
\pgflineto{\pgfpoint{173.087997pt}{78.404205pt}}
\pgfpathclose
\pgfusepath{fill,stroke}
\pgfpathmoveto{\pgfpoint{173.087997pt}{84.581039pt}}
\pgflineto{\pgfpoint{182.015991pt}{84.581039pt}}
\pgflineto{\pgfpoint{182.015991pt}{78.404205pt}}
\pgfpathclose
\pgfusepath{fill,stroke}
\pgfpathmoveto{\pgfpoint{173.087997pt}{90.757896pt}}
\pgflineto{\pgfpoint{182.015991pt}{84.581039pt}}
\pgflineto{\pgfpoint{173.087997pt}{84.581039pt}}
\pgfpathclose
\pgfusepath{fill,stroke}
\pgfpathmoveto{\pgfpoint{173.087997pt}{90.757896pt}}
\pgflineto{\pgfpoint{182.015991pt}{90.757896pt}}
\pgflineto{\pgfpoint{182.015991pt}{84.581039pt}}
\pgfpathclose
\pgfusepath{fill,stroke}
\color[rgb]{0.220425,0.342517,0.549287}
\pgfpathmoveto{\pgfpoint{173.087997pt}{96.934731pt}}
\pgflineto{\pgfpoint{182.015991pt}{90.757896pt}}
\pgflineto{\pgfpoint{173.087997pt}{90.757896pt}}
\pgfpathclose
\pgfusepath{fill,stroke}
\color[rgb]{0.228263,0.325586,0.546335}
\pgfpathmoveto{\pgfpoint{182.015991pt}{78.404205pt}}
\pgflineto{\pgfpoint{190.943985pt}{72.227356pt}}
\pgflineto{\pgfpoint{182.015991pt}{72.227356pt}}
\pgfpathclose
\pgfusepath{fill,stroke}
\color[rgb]{0.220425,0.342517,0.549287}
\pgfpathmoveto{\pgfpoint{182.015991pt}{84.581039pt}}
\pgflineto{\pgfpoint{190.943985pt}{84.581039pt}}
\pgflineto{\pgfpoint{190.943985pt}{78.404205pt}}
\pgfpathclose
\pgfusepath{fill,stroke}
\color[rgb]{0.212667,0.359102,0.551635}
\pgfpathmoveto{\pgfpoint{182.015991pt}{90.757896pt}}
\pgflineto{\pgfpoint{190.943985pt}{84.581039pt}}
\pgflineto{\pgfpoint{182.015991pt}{84.581039pt}}
\pgfpathclose
\pgfusepath{fill,stroke}
\color[rgb]{0.220425,0.342517,0.549287}
\pgfpathmoveto{\pgfpoint{190.943985pt}{78.404205pt}}
\pgflineto{\pgfpoint{199.871979pt}{78.404205pt}}
\pgflineto{\pgfpoint{199.871979pt}{72.227356pt}}
\pgfpathclose
\pgfusepath{fill,stroke}
\color[rgb]{0.212667,0.359102,0.551635}
\pgfpathmoveto{\pgfpoint{190.943985pt}{84.581039pt}}
\pgflineto{\pgfpoint{199.871979pt}{78.404205pt}}
\pgflineto{\pgfpoint{190.943985pt}{78.404205pt}}
\pgfpathclose
\pgfusepath{fill,stroke}
\pgfpathmoveto{\pgfpoint{190.943985pt}{84.581039pt}}
\pgflineto{\pgfpoint{199.871979pt}{84.581039pt}}
\pgflineto{\pgfpoint{199.871979pt}{78.404205pt}}
\pgfpathclose
\pgfusepath{fill,stroke}
\color[rgb]{0.220425,0.342517,0.549287}
\pgfpathmoveto{\pgfpoint{199.871979pt}{72.227356pt}}
\pgflineto{\pgfpoint{208.799988pt}{66.050522pt}}
\pgflineto{\pgfpoint{199.871979pt}{66.050522pt}}
\pgfpathclose
\pgfusepath{fill,stroke}
\pgfpathmoveto{\pgfpoint{199.871979pt}{72.227356pt}}
\pgflineto{\pgfpoint{208.799988pt}{72.227356pt}}
\pgflineto{\pgfpoint{208.799988pt}{66.050522pt}}
\pgfpathclose
\pgfusepath{fill,stroke}
\color[rgb]{0.212667,0.359102,0.551635}
\pgfpathmoveto{\pgfpoint{199.871979pt}{78.404205pt}}
\pgflineto{\pgfpoint{208.799988pt}{72.227356pt}}
\pgflineto{\pgfpoint{199.871979pt}{72.227356pt}}
\pgfpathclose
\pgfusepath{fill,stroke}
\pgfpathmoveto{\pgfpoint{199.871979pt}{78.404205pt}}
\pgflineto{\pgfpoint{208.799988pt}{78.404205pt}}
\pgflineto{\pgfpoint{208.799988pt}{72.227356pt}}
\pgfpathclose
\pgfusepath{fill,stroke}
\color[rgb]{0.205079,0.375366,0.553493}
\pgfpathmoveto{\pgfpoint{199.871979pt}{84.581039pt}}
\pgflineto{\pgfpoint{208.799988pt}{78.404205pt}}
\pgflineto{\pgfpoint{199.871979pt}{78.404205pt}}
\pgfpathclose
\pgfusepath{fill,stroke}
\color[rgb]{0.212667,0.359102,0.551635}
\pgfpathmoveto{\pgfpoint{208.799988pt}{72.227356pt}}
\pgflineto{\pgfpoint{217.727982pt}{72.227356pt}}
\pgflineto{\pgfpoint{217.727982pt}{66.050522pt}}
\pgfpathclose
\pgfusepath{fill,stroke}
\color[rgb]{0.205079,0.375366,0.553493}
\pgfpathmoveto{\pgfpoint{208.799988pt}{78.404205pt}}
\pgflineto{\pgfpoint{217.727982pt}{72.227356pt}}
\pgflineto{\pgfpoint{208.799988pt}{72.227356pt}}
\pgfpathclose
\pgfusepath{fill,stroke}
\pgfpathmoveto{\pgfpoint{208.799988pt}{78.404205pt}}
\pgflineto{\pgfpoint{217.727982pt}{78.404205pt}}
\pgflineto{\pgfpoint{217.727982pt}{72.227356pt}}
\pgfpathclose
\pgfusepath{fill,stroke}
\color[rgb]{0.212667,0.359102,0.551635}
\pgfpathmoveto{\pgfpoint{217.727982pt}{59.873672pt}}
\pgflineto{\pgfpoint{226.655975pt}{59.873672pt}}
\pgflineto{\pgfpoint{226.655975pt}{53.696838pt}}
\pgfpathclose
\pgfusepath{fill,stroke}
\pgfpathmoveto{\pgfpoint{217.727982pt}{66.050522pt}}
\pgflineto{\pgfpoint{226.655975pt}{59.873672pt}}
\pgflineto{\pgfpoint{217.727982pt}{59.873672pt}}
\pgfpathclose
\pgfusepath{fill,stroke}
\pgfpathmoveto{\pgfpoint{217.727982pt}{66.050522pt}}
\pgflineto{\pgfpoint{226.655975pt}{66.050522pt}}
\pgflineto{\pgfpoint{226.655975pt}{59.873672pt}}
\pgfpathclose
\pgfusepath{fill,stroke}
\color[rgb]{0.205079,0.375366,0.553493}
\pgfpathmoveto{\pgfpoint{217.727982pt}{72.227356pt}}
\pgflineto{\pgfpoint{226.655975pt}{66.050522pt}}
\pgflineto{\pgfpoint{217.727982pt}{66.050522pt}}
\pgfpathclose
\pgfusepath{fill,stroke}
\pgfpathmoveto{\pgfpoint{217.727982pt}{72.227356pt}}
\pgflineto{\pgfpoint{226.655975pt}{72.227356pt}}
\pgflineto{\pgfpoint{226.655975pt}{66.050522pt}}
\pgfpathclose
\pgfusepath{fill,stroke}
\color[rgb]{0.197722,0.391341,0.554953}
\pgfpathmoveto{\pgfpoint{217.727982pt}{78.404205pt}}
\pgflineto{\pgfpoint{226.655975pt}{72.227356pt}}
\pgflineto{\pgfpoint{217.727982pt}{72.227356pt}}
\pgfpathclose
\pgfusepath{fill,stroke}
\color[rgb]{0.205079,0.375366,0.553493}
\pgfpathmoveto{\pgfpoint{226.655975pt}{59.873672pt}}
\pgflineto{\pgfpoint{235.583969pt}{53.696838pt}}
\pgflineto{\pgfpoint{226.655975pt}{53.696838pt}}
\pgfpathclose
\pgfusepath{fill,stroke}
\color[rgb]{0.197722,0.391341,0.554953}
\pgfpathmoveto{\pgfpoint{226.655975pt}{66.050522pt}}
\pgflineto{\pgfpoint{235.583969pt}{66.050522pt}}
\pgflineto{\pgfpoint{235.583969pt}{59.873672pt}}
\pgfpathclose
\pgfusepath{fill,stroke}
\pgfpathmoveto{\pgfpoint{226.655975pt}{72.227356pt}}
\pgflineto{\pgfpoint{235.583969pt}{66.050522pt}}
\pgflineto{\pgfpoint{226.655975pt}{66.050522pt}}
\pgfpathclose
\pgfusepath{fill,stroke}
\pgfpathmoveto{\pgfpoint{226.655975pt}{72.227356pt}}
\pgflineto{\pgfpoint{235.583969pt}{72.227356pt}}
\pgflineto{\pgfpoint{235.583969pt}{66.050522pt}}
\pgfpathclose
\pgfusepath{fill,stroke}
\color[rgb]{0.205079,0.375366,0.553493}
\pgfpathmoveto{\pgfpoint{235.583969pt}{53.696838pt}}
\pgflineto{\pgfpoint{244.511993pt}{53.696838pt}}
\pgflineto{\pgfpoint{244.511993pt}{47.519989pt}}
\pgfpathclose
\pgfusepath{fill,stroke}
\color[rgb]{0.197722,0.391341,0.554953}
\pgfpathmoveto{\pgfpoint{235.583969pt}{59.873672pt}}
\pgflineto{\pgfpoint{244.511993pt}{53.696838pt}}
\pgflineto{\pgfpoint{235.583969pt}{53.696838pt}}
\pgfpathclose
\pgfusepath{fill,stroke}
\pgfpathmoveto{\pgfpoint{235.583969pt}{59.873672pt}}
\pgflineto{\pgfpoint{244.511993pt}{59.873672pt}}
\pgflineto{\pgfpoint{244.511993pt}{53.696838pt}}
\pgfpathclose
\pgfusepath{fill,stroke}
\color[rgb]{0.190631,0.407061,0.556089}
\pgfpathmoveto{\pgfpoint{235.583969pt}{66.050522pt}}
\pgflineto{\pgfpoint{244.511993pt}{59.873672pt}}
\pgflineto{\pgfpoint{235.583969pt}{59.873672pt}}
\pgfpathclose
\pgfusepath{fill,stroke}
\pgfpathmoveto{\pgfpoint{235.583969pt}{66.050522pt}}
\pgflineto{\pgfpoint{244.511993pt}{66.050522pt}}
\pgflineto{\pgfpoint{244.511993pt}{59.873672pt}}
\pgfpathclose
\pgfusepath{fill,stroke}
\pgfpathmoveto{\pgfpoint{235.583969pt}{72.227356pt}}
\pgflineto{\pgfpoint{244.511993pt}{66.050522pt}}
\pgflineto{\pgfpoint{235.583969pt}{66.050522pt}}
\pgfpathclose
\pgfusepath{fill,stroke}
\color[rgb]{0.197722,0.391341,0.554953}
\pgfpathmoveto{\pgfpoint{244.511993pt}{53.696838pt}}
\pgflineto{\pgfpoint{253.440002pt}{47.519989pt}}
\pgflineto{\pgfpoint{244.511993pt}{47.519989pt}}
\pgfpathclose
\pgfusepath{fill,stroke}
\color[rgb]{0.190631,0.407061,0.556089}
\pgfpathmoveto{\pgfpoint{244.511993pt}{59.873672pt}}
\pgflineto{\pgfpoint{253.440002pt}{59.873672pt}}
\pgflineto{\pgfpoint{253.440002pt}{53.696838pt}}
\pgfpathclose
\pgfusepath{fill,stroke}
\color[rgb]{0.183819,0.422564,0.556952}
\pgfpathmoveto{\pgfpoint{244.511993pt}{66.050522pt}}
\pgflineto{\pgfpoint{253.440002pt}{59.873672pt}}
\pgflineto{\pgfpoint{244.511993pt}{59.873672pt}}
\pgfpathclose
\pgfusepath{fill,stroke}
\pgfpathmoveto{\pgfpoint{244.511993pt}{66.050522pt}}
\pgflineto{\pgfpoint{253.440002pt}{66.050522pt}}
\pgflineto{\pgfpoint{253.440002pt}{59.873672pt}}
\pgfpathclose
\pgfusepath{fill,stroke}
\color[rgb]{0.190631,0.407061,0.556089}
\pgfpathmoveto{\pgfpoint{253.440002pt}{53.696838pt}}
\pgflineto{\pgfpoint{262.367981pt}{53.696838pt}}
\pgflineto{\pgfpoint{262.367981pt}{47.519989pt}}
\pgfpathclose
\pgfusepath{fill,stroke}
\color[rgb]{0.183819,0.422564,0.556952}
\pgfpathmoveto{\pgfpoint{253.440002pt}{59.873672pt}}
\pgflineto{\pgfpoint{262.367981pt}{53.696838pt}}
\pgflineto{\pgfpoint{253.440002pt}{53.696838pt}}
\pgfpathclose
\pgfusepath{fill,stroke}
\pgfpathmoveto{\pgfpoint{253.440002pt}{59.873672pt}}
\pgflineto{\pgfpoint{262.367981pt}{59.873672pt}}
\pgflineto{\pgfpoint{262.367981pt}{53.696838pt}}
\pgfpathclose
\pgfusepath{fill,stroke}
\color[rgb]{0.177272,0.437886,0.557576}
\pgfpathmoveto{\pgfpoint{253.440002pt}{66.050522pt}}
\pgflineto{\pgfpoint{262.367981pt}{59.873672pt}}
\pgflineto{\pgfpoint{253.440002pt}{59.873672pt}}
\pgfpathclose
\pgfusepath{fill,stroke}
\color[rgb]{0.183819,0.422564,0.556952}
\pgfpathmoveto{\pgfpoint{262.367981pt}{53.696838pt}}
\pgflineto{\pgfpoint{271.295990pt}{47.519989pt}}
\pgflineto{\pgfpoint{262.367981pt}{47.519989pt}}
\pgfpathclose
\pgfusepath{fill,stroke}
\pgfpathmoveto{\pgfpoint{262.367981pt}{53.696838pt}}
\pgflineto{\pgfpoint{271.295990pt}{53.696838pt}}
\pgflineto{\pgfpoint{271.295990pt}{47.519989pt}}
\pgfpathclose
\pgfusepath{fill,stroke}
\color[rgb]{0.177272,0.437886,0.557576}
\pgfpathmoveto{\pgfpoint{262.367981pt}{59.873672pt}}
\pgflineto{\pgfpoint{271.295990pt}{53.696838pt}}
\pgflineto{\pgfpoint{262.367981pt}{53.696838pt}}
\pgfpathclose
\pgfusepath{fill,stroke}
\pgfpathmoveto{\pgfpoint{271.295990pt}{53.696838pt}}
\pgflineto{\pgfpoint{280.223969pt}{47.519989pt}}
\pgflineto{\pgfpoint{271.295990pt}{47.519989pt}}
\pgfpathclose
\pgfusepath{fill,stroke}
\pgfpathmoveto{\pgfpoint{271.295990pt}{53.696838pt}}
\pgflineto{\pgfpoint{280.223969pt}{53.696838pt}}
\pgflineto{\pgfpoint{280.223969pt}{47.519989pt}}
\pgfpathclose
\pgfusepath{fill,stroke}
\color[rgb]{0.170958,0.453063,0.557974}
\pgfpathmoveto{\pgfpoint{280.223969pt}{53.696838pt}}
\pgflineto{\pgfpoint{289.151978pt}{47.519989pt}}
\pgflineto{\pgfpoint{280.223969pt}{47.519989pt}}
\pgfpathclose
\pgfusepath{fill,stroke}
\pgfpathmoveto{\pgfpoint{280.223969pt}{53.696838pt}}
\pgflineto{\pgfpoint{289.151978pt}{53.696838pt}}
\pgflineto{\pgfpoint{289.151978pt}{47.519989pt}}
\pgfpathclose
\pgfusepath{fill,stroke}
\color[rgb]{0.164833,0.468130,0.558143}
\pgfpathmoveto{\pgfpoint{280.223969pt}{59.873672pt}}
\pgflineto{\pgfpoint{289.151978pt}{53.696838pt}}
\pgflineto{\pgfpoint{280.223969pt}{53.696838pt}}
\pgfpathclose
\pgfusepath{fill,stroke}
\pgfpathmoveto{\pgfpoint{289.151978pt}{53.696838pt}}
\pgflineto{\pgfpoint{298.079987pt}{47.519989pt}}
\pgflineto{\pgfpoint{289.151978pt}{47.519989pt}}
\pgfpathclose
\pgfusepath{fill,stroke}
\pgfpathmoveto{\pgfpoint{289.151978pt}{53.696838pt}}
\pgflineto{\pgfpoint{298.079987pt}{53.696838pt}}
\pgflineto{\pgfpoint{298.079987pt}{47.519989pt}}
\pgfpathclose
\pgfusepath{fill,stroke}
\color[rgb]{0.152951,0.498053,0.557685}
\pgfpathmoveto{\pgfpoint{298.079987pt}{53.696838pt}}
\pgflineto{\pgfpoint{307.007965pt}{47.519989pt}}
\pgflineto{\pgfpoint{298.079987pt}{47.519989pt}}
\pgfpathclose
\pgfusepath{fill,stroke}
\color[rgb]{0.190631,0.407061,0.556089}
\pgfpathmoveto{\pgfpoint{235.583969pt}{72.227356pt}}
\pgflineto{\pgfpoint{244.511993pt}{72.227356pt}}
\pgflineto{\pgfpoint{244.511993pt}{66.050522pt}}
\pgfpathclose
\pgfusepath{fill,stroke}
\color[rgb]{0.183819,0.422564,0.556952}
\pgfpathmoveto{\pgfpoint{235.583969pt}{78.404205pt}}
\pgflineto{\pgfpoint{244.511993pt}{72.227356pt}}
\pgflineto{\pgfpoint{235.583969pt}{72.227356pt}}
\pgfpathclose
\pgfusepath{fill,stroke}
\color[rgb]{0.177272,0.437886,0.557576}
\pgfpathmoveto{\pgfpoint{253.440002pt}{66.050522pt}}
\pgflineto{\pgfpoint{262.367981pt}{66.050522pt}}
\pgflineto{\pgfpoint{262.367981pt}{59.873672pt}}
\pgfpathclose
\pgfusepath{fill,stroke}
\color[rgb]{0.170958,0.453063,0.557974}
\pgfpathmoveto{\pgfpoint{253.440002pt}{72.227356pt}}
\pgflineto{\pgfpoint{262.367981pt}{66.050522pt}}
\pgflineto{\pgfpoint{253.440002pt}{66.050522pt}}
\pgfpathclose
\pgfusepath{fill,stroke}
\color[rgb]{0.177272,0.437886,0.557576}
\pgfpathmoveto{\pgfpoint{262.367981pt}{59.873672pt}}
\pgflineto{\pgfpoint{271.295990pt}{59.873672pt}}
\pgflineto{\pgfpoint{271.295990pt}{53.696838pt}}
\pgfpathclose
\pgfusepath{fill,stroke}
\color[rgb]{0.170958,0.453063,0.557974}
\pgfpathmoveto{\pgfpoint{262.367981pt}{66.050522pt}}
\pgflineto{\pgfpoint{271.295990pt}{59.873672pt}}
\pgflineto{\pgfpoint{262.367981pt}{59.873672pt}}
\pgfpathclose
\pgfusepath{fill,stroke}
\color[rgb]{0.205079,0.375366,0.553493}
\pgfpathmoveto{\pgfpoint{199.871979pt}{84.581039pt}}
\pgflineto{\pgfpoint{208.799988pt}{84.581039pt}}
\pgflineto{\pgfpoint{208.799988pt}{78.404205pt}}
\pgfpathclose
\pgfusepath{fill,stroke}
\color[rgb]{0.197722,0.391341,0.554953}
\pgfpathmoveto{\pgfpoint{199.871979pt}{90.757896pt}}
\pgflineto{\pgfpoint{208.799988pt}{84.581039pt}}
\pgflineto{\pgfpoint{199.871979pt}{84.581039pt}}
\pgfpathclose
\pgfusepath{fill,stroke}
\pgfpathmoveto{\pgfpoint{217.727982pt}{78.404205pt}}
\pgflineto{\pgfpoint{226.655975pt}{78.404205pt}}
\pgflineto{\pgfpoint{226.655975pt}{72.227356pt}}
\pgfpathclose
\pgfusepath{fill,stroke}
\color[rgb]{0.190631,0.407061,0.556089}
\pgfpathmoveto{\pgfpoint{217.727982pt}{84.581039pt}}
\pgflineto{\pgfpoint{226.655975pt}{78.404205pt}}
\pgflineto{\pgfpoint{217.727982pt}{78.404205pt}}
\pgfpathclose
\pgfusepath{fill,stroke}
\color[rgb]{0.228263,0.325586,0.546335}
\pgfpathmoveto{\pgfpoint{155.231979pt}{103.111580pt}}
\pgflineto{\pgfpoint{164.160004pt}{103.111580pt}}
\pgflineto{\pgfpoint{164.160004pt}{96.934731pt}}
\pgfpathclose
\pgfusepath{fill,stroke}
\color[rgb]{0.220425,0.342517,0.549287}
\pgfpathmoveto{\pgfpoint{155.231979pt}{109.288422pt}}
\pgflineto{\pgfpoint{164.160004pt}{103.111580pt}}
\pgflineto{\pgfpoint{155.231979pt}{103.111580pt}}
\pgfpathclose
\pgfusepath{fill,stroke}
\pgfpathmoveto{\pgfpoint{173.087997pt}{96.934731pt}}
\pgflineto{\pgfpoint{182.015991pt}{96.934731pt}}
\pgflineto{\pgfpoint{182.015991pt}{90.757896pt}}
\pgfpathclose
\pgfusepath{fill,stroke}
\color[rgb]{0.212667,0.359102,0.551635}
\pgfpathmoveto{\pgfpoint{173.087997pt}{103.111580pt}}
\pgflineto{\pgfpoint{182.015991pt}{96.934731pt}}
\pgflineto{\pgfpoint{173.087997pt}{96.934731pt}}
\pgfpathclose
\pgfusepath{fill,stroke}
\pgfpathmoveto{\pgfpoint{182.015991pt}{90.757896pt}}
\pgflineto{\pgfpoint{190.943985pt}{90.757896pt}}
\pgflineto{\pgfpoint{190.943985pt}{84.581039pt}}
\pgfpathclose
\pgfusepath{fill,stroke}
\pgfpathmoveto{\pgfpoint{182.015991pt}{96.934731pt}}
\pgflineto{\pgfpoint{190.943985pt}{90.757896pt}}
\pgflineto{\pgfpoint{182.015991pt}{90.757896pt}}
\pgfpathclose
\pgfusepath{fill,stroke}
\color[rgb]{0.236073,0.308291,0.542652}
\pgfpathmoveto{\pgfpoint{137.376007pt}{115.465263pt}}
\pgflineto{\pgfpoint{146.303986pt}{115.465263pt}}
\pgflineto{\pgfpoint{146.303986pt}{109.288422pt}}
\pgfpathclose
\pgfusepath{fill,stroke}
\color[rgb]{0.228263,0.325586,0.546335}
\pgfpathmoveto{\pgfpoint{137.376007pt}{121.642097pt}}
\pgflineto{\pgfpoint{146.303986pt}{115.465263pt}}
\pgflineto{\pgfpoint{137.376007pt}{115.465263pt}}
\pgfpathclose
\pgfusepath{fill,stroke}
\pgfpathmoveto{\pgfpoint{137.376007pt}{121.642097pt}}
\pgflineto{\pgfpoint{146.303986pt}{121.642097pt}}
\pgflineto{\pgfpoint{146.303986pt}{115.465263pt}}
\pgfpathclose
\pgfusepath{fill,stroke}
\pgfpathmoveto{\pgfpoint{146.303986pt}{109.288422pt}}
\pgflineto{\pgfpoint{155.231979pt}{103.111580pt}}
\pgflineto{\pgfpoint{146.303986pt}{103.111580pt}}
\pgfpathclose
\pgfusepath{fill,stroke}
\pgfpathmoveto{\pgfpoint{146.303986pt}{109.288422pt}}
\pgflineto{\pgfpoint{155.231979pt}{109.288422pt}}
\pgflineto{\pgfpoint{155.231979pt}{103.111580pt}}
\pgfpathclose
\pgfusepath{fill,stroke}
\color[rgb]{0.220425,0.342517,0.549287}
\pgfpathmoveto{\pgfpoint{146.303986pt}{115.465263pt}}
\pgflineto{\pgfpoint{155.231979pt}{109.288422pt}}
\pgflineto{\pgfpoint{146.303986pt}{109.288422pt}}
\pgfpathclose
\pgfusepath{fill,stroke}
\pgfpathmoveto{\pgfpoint{146.303986pt}{115.465263pt}}
\pgflineto{\pgfpoint{155.231979pt}{115.465263pt}}
\pgflineto{\pgfpoint{155.231979pt}{109.288422pt}}
\pgfpathclose
\pgfusepath{fill,stroke}
\pgfpathmoveto{\pgfpoint{146.303986pt}{121.642097pt}}
\pgflineto{\pgfpoint{155.231979pt}{115.465263pt}}
\pgflineto{\pgfpoint{146.303986pt}{115.465263pt}}
\pgfpathclose
\pgfusepath{fill,stroke}
\pgfpathmoveto{\pgfpoint{155.231979pt}{109.288422pt}}
\pgflineto{\pgfpoint{164.160004pt}{109.288422pt}}
\pgflineto{\pgfpoint{164.160004pt}{103.111580pt}}
\pgfpathclose
\pgfusepath{fill,stroke}
\color[rgb]{0.212667,0.359102,0.551635}
\pgfpathmoveto{\pgfpoint{155.231979pt}{115.465263pt}}
\pgflineto{\pgfpoint{164.160004pt}{109.288422pt}}
\pgflineto{\pgfpoint{155.231979pt}{109.288422pt}}
\pgfpathclose
\pgfusepath{fill,stroke}
\color[rgb]{0.220425,0.342517,0.549287}
\pgfpathmoveto{\pgfpoint{164.160004pt}{103.111580pt}}
\pgflineto{\pgfpoint{173.087997pt}{96.934731pt}}
\pgflineto{\pgfpoint{164.160004pt}{96.934731pt}}
\pgfpathclose
\pgfusepath{fill,stroke}
\pgfpathmoveto{\pgfpoint{164.160004pt}{103.111580pt}}
\pgflineto{\pgfpoint{173.087997pt}{103.111580pt}}
\pgflineto{\pgfpoint{173.087997pt}{96.934731pt}}
\pgfpathclose
\pgfusepath{fill,stroke}
\color[rgb]{0.212667,0.359102,0.551635}
\pgfpathmoveto{\pgfpoint{164.160004pt}{109.288422pt}}
\pgflineto{\pgfpoint{173.087997pt}{103.111580pt}}
\pgflineto{\pgfpoint{164.160004pt}{103.111580pt}}
\pgfpathclose
\pgfusepath{fill,stroke}
\pgfpathmoveto{\pgfpoint{164.160004pt}{109.288422pt}}
\pgflineto{\pgfpoint{173.087997pt}{109.288422pt}}
\pgflineto{\pgfpoint{173.087997pt}{103.111580pt}}
\pgfpathclose
\pgfusepath{fill,stroke}
\pgfpathmoveto{\pgfpoint{173.087997pt}{103.111580pt}}
\pgflineto{\pgfpoint{182.015991pt}{103.111580pt}}
\pgflineto{\pgfpoint{182.015991pt}{96.934731pt}}
\pgfpathclose
\pgfusepath{fill,stroke}
\color[rgb]{0.205079,0.375366,0.553493}
\pgfpathmoveto{\pgfpoint{173.087997pt}{109.288422pt}}
\pgflineto{\pgfpoint{182.015991pt}{103.111580pt}}
\pgflineto{\pgfpoint{173.087997pt}{103.111580pt}}
\pgfpathclose
\pgfusepath{fill,stroke}
\color[rgb]{0.212667,0.359102,0.551635}
\pgfpathmoveto{\pgfpoint{182.015991pt}{96.934731pt}}
\pgflineto{\pgfpoint{190.943985pt}{96.934731pt}}
\pgflineto{\pgfpoint{190.943985pt}{90.757896pt}}
\pgfpathclose
\pgfusepath{fill,stroke}
\color[rgb]{0.205079,0.375366,0.553493}
\pgfpathmoveto{\pgfpoint{182.015991pt}{103.111580pt}}
\pgflineto{\pgfpoint{190.943985pt}{96.934731pt}}
\pgflineto{\pgfpoint{182.015991pt}{96.934731pt}}
\pgfpathclose
\pgfusepath{fill,stroke}
\pgfpathmoveto{\pgfpoint{182.015991pt}{103.111580pt}}
\pgflineto{\pgfpoint{190.943985pt}{103.111580pt}}
\pgflineto{\pgfpoint{190.943985pt}{96.934731pt}}
\pgfpathclose
\pgfusepath{fill,stroke}
\pgfpathmoveto{\pgfpoint{190.943985pt}{90.757896pt}}
\pgflineto{\pgfpoint{199.871979pt}{84.581039pt}}
\pgflineto{\pgfpoint{190.943985pt}{84.581039pt}}
\pgfpathclose
\pgfusepath{fill,stroke}
\pgfpathmoveto{\pgfpoint{190.943985pt}{90.757896pt}}
\pgflineto{\pgfpoint{199.871979pt}{90.757896pt}}
\pgflineto{\pgfpoint{199.871979pt}{84.581039pt}}
\pgfpathclose
\pgfusepath{fill,stroke}
\pgfpathmoveto{\pgfpoint{190.943985pt}{96.934731pt}}
\pgflineto{\pgfpoint{199.871979pt}{90.757896pt}}
\pgflineto{\pgfpoint{190.943985pt}{90.757896pt}}
\pgfpathclose
\pgfusepath{fill,stroke}
\pgfpathmoveto{\pgfpoint{190.943985pt}{96.934731pt}}
\pgflineto{\pgfpoint{199.871979pt}{96.934731pt}}
\pgflineto{\pgfpoint{199.871979pt}{90.757896pt}}
\pgfpathclose
\pgfusepath{fill,stroke}
\color[rgb]{0.197722,0.391341,0.554953}
\pgfpathmoveto{\pgfpoint{190.943985pt}{103.111580pt}}
\pgflineto{\pgfpoint{199.871979pt}{96.934731pt}}
\pgflineto{\pgfpoint{190.943985pt}{96.934731pt}}
\pgfpathclose
\pgfusepath{fill,stroke}
\pgfpathmoveto{\pgfpoint{199.871979pt}{90.757896pt}}
\pgflineto{\pgfpoint{208.799988pt}{90.757896pt}}
\pgflineto{\pgfpoint{208.799988pt}{84.581039pt}}
\pgfpathclose
\pgfusepath{fill,stroke}
\color[rgb]{0.190631,0.407061,0.556089}
\pgfpathmoveto{\pgfpoint{199.871979pt}{96.934731pt}}
\pgflineto{\pgfpoint{208.799988pt}{90.757896pt}}
\pgflineto{\pgfpoint{199.871979pt}{90.757896pt}}
\pgfpathclose
\pgfusepath{fill,stroke}
\pgfpathmoveto{\pgfpoint{199.871979pt}{96.934731pt}}
\pgflineto{\pgfpoint{208.799988pt}{96.934731pt}}
\pgflineto{\pgfpoint{208.799988pt}{90.757896pt}}
\pgfpathclose
\pgfusepath{fill,stroke}
\color[rgb]{0.197722,0.391341,0.554953}
\pgfpathmoveto{\pgfpoint{208.799988pt}{84.581039pt}}
\pgflineto{\pgfpoint{217.727982pt}{78.404205pt}}
\pgflineto{\pgfpoint{208.799988pt}{78.404205pt}}
\pgfpathclose
\pgfusepath{fill,stroke}
\pgfpathmoveto{\pgfpoint{208.799988pt}{84.581039pt}}
\pgflineto{\pgfpoint{217.727982pt}{84.581039pt}}
\pgflineto{\pgfpoint{217.727982pt}{78.404205pt}}
\pgfpathclose
\pgfusepath{fill,stroke}
\color[rgb]{0.190631,0.407061,0.556089}
\pgfpathmoveto{\pgfpoint{208.799988pt}{90.757896pt}}
\pgflineto{\pgfpoint{217.727982pt}{84.581039pt}}
\pgflineto{\pgfpoint{208.799988pt}{84.581039pt}}
\pgfpathclose
\pgfusepath{fill,stroke}
\pgfpathmoveto{\pgfpoint{208.799988pt}{90.757896pt}}
\pgflineto{\pgfpoint{217.727982pt}{90.757896pt}}
\pgflineto{\pgfpoint{217.727982pt}{84.581039pt}}
\pgfpathclose
\pgfusepath{fill,stroke}
\color[rgb]{0.183819,0.422564,0.556952}
\pgfpathmoveto{\pgfpoint{208.799988pt}{96.934731pt}}
\pgflineto{\pgfpoint{217.727982pt}{90.757896pt}}
\pgflineto{\pgfpoint{208.799988pt}{90.757896pt}}
\pgfpathclose
\pgfusepath{fill,stroke}
\color[rgb]{0.190631,0.407061,0.556089}
\pgfpathmoveto{\pgfpoint{217.727982pt}{84.581039pt}}
\pgflineto{\pgfpoint{226.655975pt}{84.581039pt}}
\pgflineto{\pgfpoint{226.655975pt}{78.404205pt}}
\pgfpathclose
\pgfusepath{fill,stroke}
\color[rgb]{0.183819,0.422564,0.556952}
\pgfpathmoveto{\pgfpoint{217.727982pt}{90.757896pt}}
\pgflineto{\pgfpoint{226.655975pt}{84.581039pt}}
\pgflineto{\pgfpoint{217.727982pt}{84.581039pt}}
\pgfpathclose
\pgfusepath{fill,stroke}
\color[rgb]{0.190631,0.407061,0.556089}
\pgfpathmoveto{\pgfpoint{226.655975pt}{78.404205pt}}
\pgflineto{\pgfpoint{235.583969pt}{72.227356pt}}
\pgflineto{\pgfpoint{226.655975pt}{72.227356pt}}
\pgfpathclose
\pgfusepath{fill,stroke}
\pgfpathmoveto{\pgfpoint{226.655975pt}{78.404205pt}}
\pgflineto{\pgfpoint{235.583969pt}{78.404205pt}}
\pgflineto{\pgfpoint{235.583969pt}{72.227356pt}}
\pgfpathclose
\pgfusepath{fill,stroke}
\color[rgb]{0.183819,0.422564,0.556952}
\pgfpathmoveto{\pgfpoint{226.655975pt}{84.581039pt}}
\pgflineto{\pgfpoint{235.583969pt}{78.404205pt}}
\pgflineto{\pgfpoint{226.655975pt}{78.404205pt}}
\pgfpathclose
\pgfusepath{fill,stroke}
\pgfpathmoveto{\pgfpoint{226.655975pt}{84.581039pt}}
\pgflineto{\pgfpoint{235.583969pt}{84.581039pt}}
\pgflineto{\pgfpoint{235.583969pt}{78.404205pt}}
\pgfpathclose
\pgfusepath{fill,stroke}
\pgfpathmoveto{\pgfpoint{235.583969pt}{78.404205pt}}
\pgflineto{\pgfpoint{244.511993pt}{78.404205pt}}
\pgflineto{\pgfpoint{244.511993pt}{72.227356pt}}
\pgfpathclose
\pgfusepath{fill,stroke}
\color[rgb]{0.177272,0.437886,0.557576}
\pgfpathmoveto{\pgfpoint{235.583969pt}{84.581039pt}}
\pgflineto{\pgfpoint{244.511993pt}{78.404205pt}}
\pgflineto{\pgfpoint{235.583969pt}{78.404205pt}}
\pgfpathclose
\pgfusepath{fill,stroke}
\pgfpathmoveto{\pgfpoint{244.511993pt}{72.227356pt}}
\pgflineto{\pgfpoint{253.440002pt}{66.050522pt}}
\pgflineto{\pgfpoint{244.511993pt}{66.050522pt}}
\pgfpathclose
\pgfusepath{fill,stroke}
\pgfpathmoveto{\pgfpoint{244.511993pt}{72.227356pt}}
\pgflineto{\pgfpoint{253.440002pt}{72.227356pt}}
\pgflineto{\pgfpoint{253.440002pt}{66.050522pt}}
\pgfpathclose
\pgfusepath{fill,stroke}
\pgfpathmoveto{\pgfpoint{244.511993pt}{78.404205pt}}
\pgflineto{\pgfpoint{253.440002pt}{72.227356pt}}
\pgflineto{\pgfpoint{244.511993pt}{72.227356pt}}
\pgfpathclose
\pgfusepath{fill,stroke}
\pgfpathmoveto{\pgfpoint{244.511993pt}{78.404205pt}}
\pgflineto{\pgfpoint{253.440002pt}{78.404205pt}}
\pgflineto{\pgfpoint{253.440002pt}{72.227356pt}}
\pgfpathclose
\pgfusepath{fill,stroke}
\color[rgb]{0.170958,0.453063,0.557974}
\pgfpathmoveto{\pgfpoint{253.440002pt}{72.227356pt}}
\pgflineto{\pgfpoint{262.367981pt}{72.227356pt}}
\pgflineto{\pgfpoint{262.367981pt}{66.050522pt}}
\pgfpathclose
\pgfusepath{fill,stroke}
\pgfpathmoveto{\pgfpoint{253.440002pt}{78.404205pt}}
\pgflineto{\pgfpoint{262.367981pt}{72.227356pt}}
\pgflineto{\pgfpoint{253.440002pt}{72.227356pt}}
\pgfpathclose
\pgfusepath{fill,stroke}
\pgfpathmoveto{\pgfpoint{262.367981pt}{66.050522pt}}
\pgflineto{\pgfpoint{271.295990pt}{66.050522pt}}
\pgflineto{\pgfpoint{271.295990pt}{59.873672pt}}
\pgfpathclose
\pgfusepath{fill,stroke}
\color[rgb]{0.164833,0.468130,0.558143}
\pgfpathmoveto{\pgfpoint{262.367981pt}{72.227356pt}}
\pgflineto{\pgfpoint{271.295990pt}{66.050522pt}}
\pgflineto{\pgfpoint{262.367981pt}{66.050522pt}}
\pgfpathclose
\pgfusepath{fill,stroke}
\pgfpathmoveto{\pgfpoint{262.367981pt}{72.227356pt}}
\pgflineto{\pgfpoint{271.295990pt}{72.227356pt}}
\pgflineto{\pgfpoint{271.295990pt}{66.050522pt}}
\pgfpathclose
\pgfusepath{fill,stroke}
\color[rgb]{0.170958,0.453063,0.557974}
\pgfpathmoveto{\pgfpoint{271.295990pt}{59.873672pt}}
\pgflineto{\pgfpoint{280.223969pt}{53.696838pt}}
\pgflineto{\pgfpoint{271.295990pt}{53.696838pt}}
\pgfpathclose
\pgfusepath{fill,stroke}
\pgfpathmoveto{\pgfpoint{271.295990pt}{59.873672pt}}
\pgflineto{\pgfpoint{280.223969pt}{59.873672pt}}
\pgflineto{\pgfpoint{280.223969pt}{53.696838pt}}
\pgfpathclose
\pgfusepath{fill,stroke}
\color[rgb]{0.164833,0.468130,0.558143}
\pgfpathmoveto{\pgfpoint{271.295990pt}{66.050522pt}}
\pgflineto{\pgfpoint{280.223969pt}{59.873672pt}}
\pgflineto{\pgfpoint{271.295990pt}{59.873672pt}}
\pgfpathclose
\pgfusepath{fill,stroke}
\pgfpathmoveto{\pgfpoint{271.295990pt}{66.050522pt}}
\pgflineto{\pgfpoint{280.223969pt}{66.050522pt}}
\pgflineto{\pgfpoint{280.223969pt}{59.873672pt}}
\pgfpathclose
\pgfusepath{fill,stroke}
\color[rgb]{0.158845,0.483117,0.558059}
\pgfpathmoveto{\pgfpoint{271.295990pt}{72.227356pt}}
\pgflineto{\pgfpoint{280.223969pt}{66.050522pt}}
\pgflineto{\pgfpoint{271.295990pt}{66.050522pt}}
\pgfpathclose
\pgfusepath{fill,stroke}
\color[rgb]{0.164833,0.468130,0.558143}
\pgfpathmoveto{\pgfpoint{280.223969pt}{59.873672pt}}
\pgflineto{\pgfpoint{289.151978pt}{59.873672pt}}
\pgflineto{\pgfpoint{289.151978pt}{53.696838pt}}
\pgfpathclose
\pgfusepath{fill,stroke}
\color[rgb]{0.158845,0.483117,0.558059}
\pgfpathmoveto{\pgfpoint{280.223969pt}{66.050522pt}}
\pgflineto{\pgfpoint{289.151978pt}{59.873672pt}}
\pgflineto{\pgfpoint{280.223969pt}{59.873672pt}}
\pgfpathclose
\pgfusepath{fill,stroke}
\pgfpathmoveto{\pgfpoint{280.223969pt}{66.050522pt}}
\pgflineto{\pgfpoint{289.151978pt}{66.050522pt}}
\pgflineto{\pgfpoint{289.151978pt}{59.873672pt}}
\pgfpathclose
\pgfusepath{fill,stroke}
\pgfpathmoveto{\pgfpoint{289.151978pt}{59.873672pt}}
\pgflineto{\pgfpoint{298.079987pt}{53.696838pt}}
\pgflineto{\pgfpoint{289.151978pt}{53.696838pt}}
\pgfpathclose
\pgfusepath{fill,stroke}
\pgfpathmoveto{\pgfpoint{289.151978pt}{59.873672pt}}
\pgflineto{\pgfpoint{298.079987pt}{59.873672pt}}
\pgflineto{\pgfpoint{298.079987pt}{53.696838pt}}
\pgfpathclose
\pgfusepath{fill,stroke}
\color[rgb]{0.152951,0.498053,0.557685}
\pgfpathmoveto{\pgfpoint{289.151978pt}{66.050522pt}}
\pgflineto{\pgfpoint{298.079987pt}{59.873672pt}}
\pgflineto{\pgfpoint{289.151978pt}{59.873672pt}}
\pgfpathclose
\pgfusepath{fill,stroke}
\pgfpathmoveto{\pgfpoint{298.079987pt}{53.696838pt}}
\pgflineto{\pgfpoint{307.007965pt}{53.696838pt}}
\pgflineto{\pgfpoint{307.007965pt}{47.519989pt}}
\pgfpathclose
\pgfusepath{fill,stroke}
\pgfpathmoveto{\pgfpoint{298.079987pt}{59.873672pt}}
\pgflineto{\pgfpoint{307.007965pt}{53.696838pt}}
\pgflineto{\pgfpoint{298.079987pt}{53.696838pt}}
\pgfpathclose
\pgfusepath{fill,stroke}
\color[rgb]{0.147132,0.512959,0.556973}
\pgfpathmoveto{\pgfpoint{307.007965pt}{53.696838pt}}
\pgflineto{\pgfpoint{315.935974pt}{47.519989pt}}
\pgflineto{\pgfpoint{307.007965pt}{47.519989pt}}
\pgfpathclose
\pgfusepath{fill,stroke}
\pgfpathmoveto{\pgfpoint{307.007965pt}{53.696838pt}}
\pgflineto{\pgfpoint{315.935974pt}{53.696838pt}}
\pgflineto{\pgfpoint{315.935974pt}{47.519989pt}}
\pgfpathclose
\pgfusepath{fill,stroke}
\color[rgb]{0.141402,0.527854,0.555864}
\pgfpathmoveto{\pgfpoint{315.935974pt}{53.696838pt}}
\pgflineto{\pgfpoint{324.863983pt}{47.519989pt}}
\pgflineto{\pgfpoint{315.935974pt}{47.519989pt}}
\pgfpathclose
\pgfusepath{fill,stroke}
\color[rgb]{0.147132,0.512959,0.556973}
\pgfpathmoveto{\pgfpoint{307.007965pt}{59.873672pt}}
\pgflineto{\pgfpoint{315.935974pt}{59.873672pt}}
\pgflineto{\pgfpoint{315.935974pt}{53.696838pt}}
\pgfpathclose
\pgfusepath{fill,stroke}
\color[rgb]{0.141402,0.527854,0.555864}
\pgfpathmoveto{\pgfpoint{307.007965pt}{66.050522pt}}
\pgflineto{\pgfpoint{315.935974pt}{59.873672pt}}
\pgflineto{\pgfpoint{307.007965pt}{59.873672pt}}
\pgfpathclose
\pgfusepath{fill,stroke}
\color[rgb]{0.135833,0.542750,0.554289}
\pgfpathmoveto{\pgfpoint{324.863983pt}{53.696838pt}}
\pgflineto{\pgfpoint{333.791992pt}{53.696838pt}}
\pgflineto{\pgfpoint{333.791992pt}{47.519989pt}}
\pgfpathclose
\pgfusepath{fill,stroke}
\color[rgb]{0.130582,0.557652,0.552176}
\pgfpathmoveto{\pgfpoint{324.863983pt}{59.873672pt}}
\pgflineto{\pgfpoint{333.791992pt}{53.696838pt}}
\pgflineto{\pgfpoint{324.863983pt}{53.696838pt}}
\pgfpathclose
\pgfusepath{fill,stroke}
\pgfpathmoveto{\pgfpoint{333.791992pt}{53.696838pt}}
\pgflineto{\pgfpoint{342.719971pt}{47.519989pt}}
\pgflineto{\pgfpoint{333.791992pt}{47.519989pt}}
\pgfpathclose
\pgfusepath{fill,stroke}
\color[rgb]{0.158845,0.483117,0.558059}
\pgfpathmoveto{\pgfpoint{271.295990pt}{72.227356pt}}
\pgflineto{\pgfpoint{280.223969pt}{72.227356pt}}
\pgflineto{\pgfpoint{280.223969pt}{66.050522pt}}
\pgfpathclose
\pgfusepath{fill,stroke}
\color[rgb]{0.152951,0.498053,0.557685}
\pgfpathmoveto{\pgfpoint{271.295990pt}{78.404205pt}}
\pgflineto{\pgfpoint{280.223969pt}{72.227356pt}}
\pgflineto{\pgfpoint{271.295990pt}{72.227356pt}}
\pgfpathclose
\pgfusepath{fill,stroke}
\pgfpathmoveto{\pgfpoint{289.151978pt}{66.050522pt}}
\pgflineto{\pgfpoint{298.079987pt}{66.050522pt}}
\pgflineto{\pgfpoint{298.079987pt}{59.873672pt}}
\pgfpathclose
\pgfusepath{fill,stroke}
\color[rgb]{0.147132,0.512959,0.556973}
\pgfpathmoveto{\pgfpoint{289.151978pt}{72.227356pt}}
\pgflineto{\pgfpoint{298.079987pt}{66.050522pt}}
\pgflineto{\pgfpoint{289.151978pt}{66.050522pt}}
\pgfpathclose
\pgfusepath{fill,stroke}
\color[rgb]{0.177272,0.437886,0.557576}
\pgfpathmoveto{\pgfpoint{226.655975pt}{90.757896pt}}
\pgflineto{\pgfpoint{235.583969pt}{90.757896pt}}
\pgflineto{\pgfpoint{235.583969pt}{84.581039pt}}
\pgfpathclose
\pgfusepath{fill,stroke}
\color[rgb]{0.170958,0.453063,0.557974}
\pgfpathmoveto{\pgfpoint{226.655975pt}{96.934731pt}}
\pgflineto{\pgfpoint{235.583969pt}{90.757896pt}}
\pgflineto{\pgfpoint{226.655975pt}{90.757896pt}}
\pgfpathclose
\pgfusepath{fill,stroke}
\pgfpathmoveto{\pgfpoint{244.511993pt}{84.581039pt}}
\pgflineto{\pgfpoint{253.440002pt}{84.581039pt}}
\pgflineto{\pgfpoint{253.440002pt}{78.404205pt}}
\pgfpathclose
\pgfusepath{fill,stroke}
\color[rgb]{0.164833,0.468130,0.558143}
\pgfpathmoveto{\pgfpoint{244.511993pt}{90.757896pt}}
\pgflineto{\pgfpoint{253.440002pt}{84.581039pt}}
\pgflineto{\pgfpoint{244.511993pt}{84.581039pt}}
\pgfpathclose
\pgfusepath{fill,stroke}
\color[rgb]{0.170958,0.453063,0.557974}
\pgfpathmoveto{\pgfpoint{253.440002pt}{78.404205pt}}
\pgflineto{\pgfpoint{262.367981pt}{78.404205pt}}
\pgflineto{\pgfpoint{262.367981pt}{72.227356pt}}
\pgfpathclose
\pgfusepath{fill,stroke}
\color[rgb]{0.164833,0.468130,0.558143}
\pgfpathmoveto{\pgfpoint{253.440002pt}{84.581039pt}}
\pgflineto{\pgfpoint{262.367981pt}{78.404205pt}}
\pgflineto{\pgfpoint{253.440002pt}{78.404205pt}}
\pgfpathclose
\pgfusepath{fill,stroke}
\color[rgb]{0.197722,0.391341,0.554953}
\pgfpathmoveto{\pgfpoint{182.015991pt}{109.288422pt}}
\pgflineto{\pgfpoint{190.943985pt}{109.288422pt}}
\pgflineto{\pgfpoint{190.943985pt}{103.111580pt}}
\pgfpathclose
\pgfusepath{fill,stroke}
\color[rgb]{0.190631,0.407061,0.556089}
\pgfpathmoveto{\pgfpoint{182.015991pt}{115.465263pt}}
\pgflineto{\pgfpoint{190.943985pt}{109.288422pt}}
\pgflineto{\pgfpoint{182.015991pt}{109.288422pt}}
\pgfpathclose
\pgfusepath{fill,stroke}
\color[rgb]{0.197722,0.391341,0.554953}
\pgfpathmoveto{\pgfpoint{190.943985pt}{103.111580pt}}
\pgflineto{\pgfpoint{199.871979pt}{103.111580pt}}
\pgflineto{\pgfpoint{199.871979pt}{96.934731pt}}
\pgfpathclose
\pgfusepath{fill,stroke}
\color[rgb]{0.190631,0.407061,0.556089}
\pgfpathmoveto{\pgfpoint{190.943985pt}{109.288422pt}}
\pgflineto{\pgfpoint{199.871979pt}{103.111580pt}}
\pgflineto{\pgfpoint{190.943985pt}{103.111580pt}}
\pgfpathclose
\pgfusepath{fill,stroke}
\color[rgb]{0.183819,0.422564,0.556952}
\pgfpathmoveto{\pgfpoint{208.799988pt}{96.934731pt}}
\pgflineto{\pgfpoint{217.727982pt}{96.934731pt}}
\pgflineto{\pgfpoint{217.727982pt}{90.757896pt}}
\pgfpathclose
\pgfusepath{fill,stroke}
\pgfpathmoveto{\pgfpoint{208.799988pt}{103.111580pt}}
\pgflineto{\pgfpoint{217.727982pt}{96.934731pt}}
\pgflineto{\pgfpoint{208.799988pt}{96.934731pt}}
\pgfpathclose
\pgfusepath{fill,stroke}
\color[rgb]{0.220425,0.342517,0.549287}
\pgfpathmoveto{\pgfpoint{146.303986pt}{121.642097pt}}
\pgflineto{\pgfpoint{155.231979pt}{121.642097pt}}
\pgflineto{\pgfpoint{155.231979pt}{115.465263pt}}
\pgfpathclose
\pgfusepath{fill,stroke}
\color[rgb]{0.212667,0.359102,0.551635}
\pgfpathmoveto{\pgfpoint{146.303986pt}{127.818947pt}}
\pgflineto{\pgfpoint{155.231979pt}{121.642097pt}}
\pgflineto{\pgfpoint{146.303986pt}{121.642097pt}}
\pgfpathclose
\pgfusepath{fill,stroke}
\color[rgb]{0.205079,0.375366,0.553493}
\pgfpathmoveto{\pgfpoint{164.160004pt}{115.465263pt}}
\pgflineto{\pgfpoint{173.087997pt}{115.465263pt}}
\pgflineto{\pgfpoint{173.087997pt}{109.288422pt}}
\pgfpathclose
\pgfusepath{fill,stroke}
\color[rgb]{0.197722,0.391341,0.554953}
\pgfpathmoveto{\pgfpoint{164.160004pt}{121.642097pt}}
\pgflineto{\pgfpoint{173.087997pt}{115.465263pt}}
\pgflineto{\pgfpoint{164.160004pt}{115.465263pt}}
\pgfpathclose
\pgfusepath{fill,stroke}
\color[rgb]{0.220425,0.342517,0.549287}
\pgfpathmoveto{\pgfpoint{128.447998pt}{133.995789pt}}
\pgflineto{\pgfpoint{137.376007pt}{133.995789pt}}
\pgflineto{\pgfpoint{137.376007pt}{127.818947pt}}
\pgfpathclose
\pgfusepath{fill,stroke}
\pgfpathmoveto{\pgfpoint{137.376007pt}{127.818947pt}}
\pgflineto{\pgfpoint{146.303986pt}{121.642097pt}}
\pgflineto{\pgfpoint{137.376007pt}{121.642097pt}}
\pgfpathclose
\pgfusepath{fill,stroke}
\pgfpathmoveto{\pgfpoint{137.376007pt}{127.818947pt}}
\pgflineto{\pgfpoint{146.303986pt}{127.818947pt}}
\pgflineto{\pgfpoint{146.303986pt}{121.642097pt}}
\pgfpathclose
\pgfusepath{fill,stroke}
\color[rgb]{0.212667,0.359102,0.551635}
\pgfpathmoveto{\pgfpoint{137.376007pt}{133.995789pt}}
\pgflineto{\pgfpoint{146.303986pt}{127.818947pt}}
\pgflineto{\pgfpoint{137.376007pt}{127.818947pt}}
\pgfpathclose
\pgfusepath{fill,stroke}
\pgfpathmoveto{\pgfpoint{137.376007pt}{133.995789pt}}
\pgflineto{\pgfpoint{146.303986pt}{133.995789pt}}
\pgflineto{\pgfpoint{146.303986pt}{127.818947pt}}
\pgfpathclose
\pgfusepath{fill,stroke}
\color[rgb]{0.205079,0.375366,0.553493}
\pgfpathmoveto{\pgfpoint{137.376007pt}{140.172638pt}}
\pgflineto{\pgfpoint{146.303986pt}{133.995789pt}}
\pgflineto{\pgfpoint{137.376007pt}{133.995789pt}}
\pgfpathclose
\pgfusepath{fill,stroke}
\color[rgb]{0.212667,0.359102,0.551635}
\pgfpathmoveto{\pgfpoint{146.303986pt}{127.818947pt}}
\pgflineto{\pgfpoint{155.231979pt}{127.818947pt}}
\pgflineto{\pgfpoint{155.231979pt}{121.642097pt}}
\pgfpathclose
\pgfusepath{fill,stroke}
\color[rgb]{0.205079,0.375366,0.553493}
\pgfpathmoveto{\pgfpoint{146.303986pt}{133.995789pt}}
\pgflineto{\pgfpoint{155.231979pt}{127.818947pt}}
\pgflineto{\pgfpoint{146.303986pt}{127.818947pt}}
\pgfpathclose
\pgfusepath{fill,stroke}
\pgfpathmoveto{\pgfpoint{146.303986pt}{133.995789pt}}
\pgflineto{\pgfpoint{155.231979pt}{133.995789pt}}
\pgflineto{\pgfpoint{155.231979pt}{127.818947pt}}
\pgfpathclose
\pgfusepath{fill,stroke}
\color[rgb]{0.212667,0.359102,0.551635}
\pgfpathmoveto{\pgfpoint{155.231979pt}{115.465263pt}}
\pgflineto{\pgfpoint{164.160004pt}{115.465263pt}}
\pgflineto{\pgfpoint{164.160004pt}{109.288422pt}}
\pgfpathclose
\pgfusepath{fill,stroke}
\pgfpathmoveto{\pgfpoint{155.231979pt}{121.642097pt}}
\pgflineto{\pgfpoint{164.160004pt}{115.465263pt}}
\pgflineto{\pgfpoint{155.231979pt}{115.465263pt}}
\pgfpathclose
\pgfusepath{fill,stroke}
\pgfpathmoveto{\pgfpoint{155.231979pt}{121.642097pt}}
\pgflineto{\pgfpoint{164.160004pt}{121.642097pt}}
\pgflineto{\pgfpoint{164.160004pt}{115.465263pt}}
\pgfpathclose
\pgfusepath{fill,stroke}
\color[rgb]{0.205079,0.375366,0.553493}
\pgfpathmoveto{\pgfpoint{155.231979pt}{127.818947pt}}
\pgflineto{\pgfpoint{164.160004pt}{121.642097pt}}
\pgflineto{\pgfpoint{155.231979pt}{121.642097pt}}
\pgfpathclose
\pgfusepath{fill,stroke}
\pgfpathmoveto{\pgfpoint{155.231979pt}{127.818947pt}}
\pgflineto{\pgfpoint{164.160004pt}{127.818947pt}}
\pgflineto{\pgfpoint{164.160004pt}{121.642097pt}}
\pgfpathclose
\pgfusepath{fill,stroke}
\color[rgb]{0.197722,0.391341,0.554953}
\pgfpathmoveto{\pgfpoint{155.231979pt}{133.995789pt}}
\pgflineto{\pgfpoint{164.160004pt}{127.818947pt}}
\pgflineto{\pgfpoint{155.231979pt}{127.818947pt}}
\pgfpathclose
\pgfusepath{fill,stroke}
\color[rgb]{0.205079,0.375366,0.553493}
\pgfpathmoveto{\pgfpoint{164.160004pt}{115.465263pt}}
\pgflineto{\pgfpoint{173.087997pt}{109.288422pt}}
\pgflineto{\pgfpoint{164.160004pt}{109.288422pt}}
\pgfpathclose
\pgfusepath{fill,stroke}
\color[rgb]{0.197722,0.391341,0.554953}
\pgfpathmoveto{\pgfpoint{164.160004pt}{121.642097pt}}
\pgflineto{\pgfpoint{173.087997pt}{121.642097pt}}
\pgflineto{\pgfpoint{173.087997pt}{115.465263pt}}
\pgfpathclose
\pgfusepath{fill,stroke}
\pgfpathmoveto{\pgfpoint{164.160004pt}{127.818947pt}}
\pgflineto{\pgfpoint{173.087997pt}{121.642097pt}}
\pgflineto{\pgfpoint{164.160004pt}{121.642097pt}}
\pgfpathclose
\pgfusepath{fill,stroke}
\pgfpathmoveto{\pgfpoint{164.160004pt}{127.818947pt}}
\pgflineto{\pgfpoint{173.087997pt}{127.818947pt}}
\pgflineto{\pgfpoint{173.087997pt}{121.642097pt}}
\pgfpathclose
\pgfusepath{fill,stroke}
\color[rgb]{0.205079,0.375366,0.553493}
\pgfpathmoveto{\pgfpoint{173.087997pt}{109.288422pt}}
\pgflineto{\pgfpoint{182.015991pt}{109.288422pt}}
\pgflineto{\pgfpoint{182.015991pt}{103.111580pt}}
\pgfpathclose
\pgfusepath{fill,stroke}
\color[rgb]{0.197722,0.391341,0.554953}
\pgfpathmoveto{\pgfpoint{173.087997pt}{115.465263pt}}
\pgflineto{\pgfpoint{182.015991pt}{109.288422pt}}
\pgflineto{\pgfpoint{173.087997pt}{109.288422pt}}
\pgfpathclose
\pgfusepath{fill,stroke}
\pgfpathmoveto{\pgfpoint{173.087997pt}{115.465263pt}}
\pgflineto{\pgfpoint{182.015991pt}{115.465263pt}}
\pgflineto{\pgfpoint{182.015991pt}{109.288422pt}}
\pgfpathclose
\pgfusepath{fill,stroke}
\color[rgb]{0.190631,0.407061,0.556089}
\pgfpathmoveto{\pgfpoint{173.087997pt}{121.642097pt}}
\pgflineto{\pgfpoint{182.015991pt}{115.465263pt}}
\pgflineto{\pgfpoint{173.087997pt}{115.465263pt}}
\pgfpathclose
\pgfusepath{fill,stroke}
\pgfpathmoveto{\pgfpoint{173.087997pt}{121.642097pt}}
\pgflineto{\pgfpoint{182.015991pt}{121.642097pt}}
\pgflineto{\pgfpoint{182.015991pt}{115.465263pt}}
\pgfpathclose
\pgfusepath{fill,stroke}
\pgfpathmoveto{\pgfpoint{173.087997pt}{127.818947pt}}
\pgflineto{\pgfpoint{182.015991pt}{121.642097pt}}
\pgflineto{\pgfpoint{173.087997pt}{121.642097pt}}
\pgfpathclose
\pgfusepath{fill,stroke}
\color[rgb]{0.197722,0.391341,0.554953}
\pgfpathmoveto{\pgfpoint{182.015991pt}{109.288422pt}}
\pgflineto{\pgfpoint{190.943985pt}{103.111580pt}}
\pgflineto{\pgfpoint{182.015991pt}{103.111580pt}}
\pgfpathclose
\pgfusepath{fill,stroke}
\color[rgb]{0.190631,0.407061,0.556089}
\pgfpathmoveto{\pgfpoint{182.015991pt}{115.465263pt}}
\pgflineto{\pgfpoint{190.943985pt}{115.465263pt}}
\pgflineto{\pgfpoint{190.943985pt}{109.288422pt}}
\pgfpathclose
\pgfusepath{fill,stroke}
\color[rgb]{0.183819,0.422564,0.556952}
\pgfpathmoveto{\pgfpoint{182.015991pt}{121.642097pt}}
\pgflineto{\pgfpoint{190.943985pt}{115.465263pt}}
\pgflineto{\pgfpoint{182.015991pt}{115.465263pt}}
\pgfpathclose
\pgfusepath{fill,stroke}
\pgfpathmoveto{\pgfpoint{182.015991pt}{121.642097pt}}
\pgflineto{\pgfpoint{190.943985pt}{121.642097pt}}
\pgflineto{\pgfpoint{190.943985pt}{115.465263pt}}
\pgfpathclose
\pgfusepath{fill,stroke}
\color[rgb]{0.190631,0.407061,0.556089}
\pgfpathmoveto{\pgfpoint{190.943985pt}{109.288422pt}}
\pgflineto{\pgfpoint{199.871979pt}{109.288422pt}}
\pgflineto{\pgfpoint{199.871979pt}{103.111580pt}}
\pgfpathclose
\pgfusepath{fill,stroke}
\color[rgb]{0.183819,0.422564,0.556952}
\pgfpathmoveto{\pgfpoint{190.943985pt}{115.465263pt}}
\pgflineto{\pgfpoint{199.871979pt}{109.288422pt}}
\pgflineto{\pgfpoint{190.943985pt}{109.288422pt}}
\pgfpathclose
\pgfusepath{fill,stroke}
\pgfpathmoveto{\pgfpoint{190.943985pt}{115.465263pt}}
\pgflineto{\pgfpoint{199.871979pt}{115.465263pt}}
\pgflineto{\pgfpoint{199.871979pt}{109.288422pt}}
\pgfpathclose
\pgfusepath{fill,stroke}
\color[rgb]{0.177272,0.437886,0.557576}
\pgfpathmoveto{\pgfpoint{190.943985pt}{121.642097pt}}
\pgflineto{\pgfpoint{199.871979pt}{115.465263pt}}
\pgflineto{\pgfpoint{190.943985pt}{115.465263pt}}
\pgfpathclose
\pgfusepath{fill,stroke}
\color[rgb]{0.190631,0.407061,0.556089}
\pgfpathmoveto{\pgfpoint{199.871979pt}{103.111580pt}}
\pgflineto{\pgfpoint{208.799988pt}{96.934731pt}}
\pgflineto{\pgfpoint{199.871979pt}{96.934731pt}}
\pgfpathclose
\pgfusepath{fill,stroke}
\pgfpathmoveto{\pgfpoint{199.871979pt}{103.111580pt}}
\pgflineto{\pgfpoint{208.799988pt}{103.111580pt}}
\pgflineto{\pgfpoint{208.799988pt}{96.934731pt}}
\pgfpathclose
\pgfusepath{fill,stroke}
\color[rgb]{0.183819,0.422564,0.556952}
\pgfpathmoveto{\pgfpoint{199.871979pt}{109.288422pt}}
\pgflineto{\pgfpoint{208.799988pt}{103.111580pt}}
\pgflineto{\pgfpoint{199.871979pt}{103.111580pt}}
\pgfpathclose
\pgfusepath{fill,stroke}
\pgfpathmoveto{\pgfpoint{199.871979pt}{109.288422pt}}
\pgflineto{\pgfpoint{208.799988pt}{109.288422pt}}
\pgflineto{\pgfpoint{208.799988pt}{103.111580pt}}
\pgfpathclose
\pgfusepath{fill,stroke}
\color[rgb]{0.177272,0.437886,0.557576}
\pgfpathmoveto{\pgfpoint{199.871979pt}{115.465263pt}}
\pgflineto{\pgfpoint{208.799988pt}{109.288422pt}}
\pgflineto{\pgfpoint{199.871979pt}{109.288422pt}}
\pgfpathclose
\pgfusepath{fill,stroke}
\color[rgb]{0.183819,0.422564,0.556952}
\pgfpathmoveto{\pgfpoint{208.799988pt}{103.111580pt}}
\pgflineto{\pgfpoint{217.727982pt}{103.111580pt}}
\pgflineto{\pgfpoint{217.727982pt}{96.934731pt}}
\pgfpathclose
\pgfusepath{fill,stroke}
\color[rgb]{0.177272,0.437886,0.557576}
\pgfpathmoveto{\pgfpoint{208.799988pt}{109.288422pt}}
\pgflineto{\pgfpoint{217.727982pt}{103.111580pt}}
\pgflineto{\pgfpoint{208.799988pt}{103.111580pt}}
\pgfpathclose
\pgfusepath{fill,stroke}
\pgfpathmoveto{\pgfpoint{208.799988pt}{109.288422pt}}
\pgflineto{\pgfpoint{217.727982pt}{109.288422pt}}
\pgflineto{\pgfpoint{217.727982pt}{103.111580pt}}
\pgfpathclose
\pgfusepath{fill,stroke}
\color[rgb]{0.183819,0.422564,0.556952}
\pgfpathmoveto{\pgfpoint{217.727982pt}{90.757896pt}}
\pgflineto{\pgfpoint{226.655975pt}{90.757896pt}}
\pgflineto{\pgfpoint{226.655975pt}{84.581039pt}}
\pgfpathclose
\pgfusepath{fill,stroke}
\color[rgb]{0.177272,0.437886,0.557576}
\pgfpathmoveto{\pgfpoint{217.727982pt}{96.934731pt}}
\pgflineto{\pgfpoint{226.655975pt}{90.757896pt}}
\pgflineto{\pgfpoint{217.727982pt}{90.757896pt}}
\pgfpathclose
\pgfusepath{fill,stroke}
\pgfpathmoveto{\pgfpoint{217.727982pt}{96.934731pt}}
\pgflineto{\pgfpoint{226.655975pt}{96.934731pt}}
\pgflineto{\pgfpoint{226.655975pt}{90.757896pt}}
\pgfpathclose
\pgfusepath{fill,stroke}
\color[rgb]{0.170958,0.453063,0.557974}
\pgfpathmoveto{\pgfpoint{217.727982pt}{103.111580pt}}
\pgflineto{\pgfpoint{226.655975pt}{96.934731pt}}
\pgflineto{\pgfpoint{217.727982pt}{96.934731pt}}
\pgfpathclose
\pgfusepath{fill,stroke}
\pgfpathmoveto{\pgfpoint{217.727982pt}{103.111580pt}}
\pgflineto{\pgfpoint{226.655975pt}{103.111580pt}}
\pgflineto{\pgfpoint{226.655975pt}{96.934731pt}}
\pgfpathclose
\pgfusepath{fill,stroke}
\pgfpathmoveto{\pgfpoint{217.727982pt}{109.288422pt}}
\pgflineto{\pgfpoint{226.655975pt}{103.111580pt}}
\pgflineto{\pgfpoint{217.727982pt}{103.111580pt}}
\pgfpathclose
\pgfusepath{fill,stroke}
\color[rgb]{0.177272,0.437886,0.557576}
\pgfpathmoveto{\pgfpoint{226.655975pt}{90.757896pt}}
\pgflineto{\pgfpoint{235.583969pt}{84.581039pt}}
\pgflineto{\pgfpoint{226.655975pt}{84.581039pt}}
\pgfpathclose
\pgfusepath{fill,stroke}
\color[rgb]{0.170958,0.453063,0.557974}
\pgfpathmoveto{\pgfpoint{226.655975pt}{96.934731pt}}
\pgflineto{\pgfpoint{235.583969pt}{96.934731pt}}
\pgflineto{\pgfpoint{235.583969pt}{90.757896pt}}
\pgfpathclose
\pgfusepath{fill,stroke}
\color[rgb]{0.164833,0.468130,0.558143}
\pgfpathmoveto{\pgfpoint{226.655975pt}{103.111580pt}}
\pgflineto{\pgfpoint{235.583969pt}{96.934731pt}}
\pgflineto{\pgfpoint{226.655975pt}{96.934731pt}}
\pgfpathclose
\pgfusepath{fill,stroke}
\pgfpathmoveto{\pgfpoint{226.655975pt}{103.111580pt}}
\pgflineto{\pgfpoint{235.583969pt}{103.111580pt}}
\pgflineto{\pgfpoint{235.583969pt}{96.934731pt}}
\pgfpathclose
\pgfusepath{fill,stroke}
\color[rgb]{0.177272,0.437886,0.557576}
\pgfpathmoveto{\pgfpoint{235.583969pt}{84.581039pt}}
\pgflineto{\pgfpoint{244.511993pt}{84.581039pt}}
\pgflineto{\pgfpoint{244.511993pt}{78.404205pt}}
\pgfpathclose
\pgfusepath{fill,stroke}
\color[rgb]{0.170958,0.453063,0.557974}
\pgfpathmoveto{\pgfpoint{235.583969pt}{90.757896pt}}
\pgflineto{\pgfpoint{244.511993pt}{84.581039pt}}
\pgflineto{\pgfpoint{235.583969pt}{84.581039pt}}
\pgfpathclose
\pgfusepath{fill,stroke}
\pgfpathmoveto{\pgfpoint{235.583969pt}{90.757896pt}}
\pgflineto{\pgfpoint{244.511993pt}{90.757896pt}}
\pgflineto{\pgfpoint{244.511993pt}{84.581039pt}}
\pgfpathclose
\pgfusepath{fill,stroke}
\color[rgb]{0.164833,0.468130,0.558143}
\pgfpathmoveto{\pgfpoint{235.583969pt}{96.934731pt}}
\pgflineto{\pgfpoint{244.511993pt}{90.757896pt}}
\pgflineto{\pgfpoint{235.583969pt}{90.757896pt}}
\pgfpathclose
\pgfusepath{fill,stroke}
\pgfpathmoveto{\pgfpoint{235.583969pt}{96.934731pt}}
\pgflineto{\pgfpoint{244.511993pt}{96.934731pt}}
\pgflineto{\pgfpoint{244.511993pt}{90.757896pt}}
\pgfpathclose
\pgfusepath{fill,stroke}
\color[rgb]{0.158845,0.483117,0.558059}
\pgfpathmoveto{\pgfpoint{235.583969pt}{103.111580pt}}
\pgflineto{\pgfpoint{244.511993pt}{96.934731pt}}
\pgflineto{\pgfpoint{235.583969pt}{96.934731pt}}
\pgfpathclose
\pgfusepath{fill,stroke}
\color[rgb]{0.170958,0.453063,0.557974}
\pgfpathmoveto{\pgfpoint{244.511993pt}{84.581039pt}}
\pgflineto{\pgfpoint{253.440002pt}{78.404205pt}}
\pgflineto{\pgfpoint{244.511993pt}{78.404205pt}}
\pgfpathclose
\pgfusepath{fill,stroke}
\color[rgb]{0.164833,0.468130,0.558143}
\pgfpathmoveto{\pgfpoint{244.511993pt}{90.757896pt}}
\pgflineto{\pgfpoint{253.440002pt}{90.757896pt}}
\pgflineto{\pgfpoint{253.440002pt}{84.581039pt}}
\pgfpathclose
\pgfusepath{fill,stroke}
\color[rgb]{0.158845,0.483117,0.558059}
\pgfpathmoveto{\pgfpoint{244.511993pt}{96.934731pt}}
\pgflineto{\pgfpoint{253.440002pt}{90.757896pt}}
\pgflineto{\pgfpoint{244.511993pt}{90.757896pt}}
\pgfpathclose
\pgfusepath{fill,stroke}
\pgfpathmoveto{\pgfpoint{244.511993pt}{96.934731pt}}
\pgflineto{\pgfpoint{253.440002pt}{96.934731pt}}
\pgflineto{\pgfpoint{253.440002pt}{90.757896pt}}
\pgfpathclose
\pgfusepath{fill,stroke}
\color[rgb]{0.164833,0.468130,0.558143}
\pgfpathmoveto{\pgfpoint{253.440002pt}{84.581039pt}}
\pgflineto{\pgfpoint{262.367981pt}{84.581039pt}}
\pgflineto{\pgfpoint{262.367981pt}{78.404205pt}}
\pgfpathclose
\pgfusepath{fill,stroke}
\color[rgb]{0.158845,0.483117,0.558059}
\pgfpathmoveto{\pgfpoint{253.440002pt}{90.757896pt}}
\pgflineto{\pgfpoint{262.367981pt}{84.581039pt}}
\pgflineto{\pgfpoint{253.440002pt}{84.581039pt}}
\pgfpathclose
\pgfusepath{fill,stroke}
\pgfpathmoveto{\pgfpoint{253.440002pt}{90.757896pt}}
\pgflineto{\pgfpoint{262.367981pt}{90.757896pt}}
\pgflineto{\pgfpoint{262.367981pt}{84.581039pt}}
\pgfpathclose
\pgfusepath{fill,stroke}
\color[rgb]{0.152951,0.498053,0.557685}
\pgfpathmoveto{\pgfpoint{253.440002pt}{96.934731pt}}
\pgflineto{\pgfpoint{262.367981pt}{90.757896pt}}
\pgflineto{\pgfpoint{253.440002pt}{90.757896pt}}
\pgfpathclose
\pgfusepath{fill,stroke}
\color[rgb]{0.158845,0.483117,0.558059}
\pgfpathmoveto{\pgfpoint{262.367981pt}{78.404205pt}}
\pgflineto{\pgfpoint{271.295990pt}{72.227356pt}}
\pgflineto{\pgfpoint{262.367981pt}{72.227356pt}}
\pgfpathclose
\pgfusepath{fill,stroke}
\pgfpathmoveto{\pgfpoint{262.367981pt}{78.404205pt}}
\pgflineto{\pgfpoint{271.295990pt}{78.404205pt}}
\pgflineto{\pgfpoint{271.295990pt}{72.227356pt}}
\pgfpathclose
\pgfusepath{fill,stroke}
\pgfpathmoveto{\pgfpoint{262.367981pt}{84.581039pt}}
\pgflineto{\pgfpoint{271.295990pt}{78.404205pt}}
\pgflineto{\pgfpoint{262.367981pt}{78.404205pt}}
\pgfpathclose
\pgfusepath{fill,stroke}
\pgfpathmoveto{\pgfpoint{262.367981pt}{84.581039pt}}
\pgflineto{\pgfpoint{271.295990pt}{84.581039pt}}
\pgflineto{\pgfpoint{271.295990pt}{78.404205pt}}
\pgfpathclose
\pgfusepath{fill,stroke}
\color[rgb]{0.152951,0.498053,0.557685}
\pgfpathmoveto{\pgfpoint{262.367981pt}{90.757896pt}}
\pgflineto{\pgfpoint{271.295990pt}{84.581039pt}}
\pgflineto{\pgfpoint{262.367981pt}{84.581039pt}}
\pgfpathclose
\pgfusepath{fill,stroke}
\pgfpathmoveto{\pgfpoint{271.295990pt}{78.404205pt}}
\pgflineto{\pgfpoint{280.223969pt}{78.404205pt}}
\pgflineto{\pgfpoint{280.223969pt}{72.227356pt}}
\pgfpathclose
\pgfusepath{fill,stroke}
\pgfpathmoveto{\pgfpoint{271.295990pt}{84.581039pt}}
\pgflineto{\pgfpoint{280.223969pt}{78.404205pt}}
\pgflineto{\pgfpoint{271.295990pt}{78.404205pt}}
\pgfpathclose
\pgfusepath{fill,stroke}
\pgfpathmoveto{\pgfpoint{271.295990pt}{84.581039pt}}
\pgflineto{\pgfpoint{280.223969pt}{84.581039pt}}
\pgflineto{\pgfpoint{280.223969pt}{78.404205pt}}
\pgfpathclose
\pgfusepath{fill,stroke}
\pgfpathmoveto{\pgfpoint{280.223969pt}{72.227356pt}}
\pgflineto{\pgfpoint{289.151978pt}{66.050522pt}}
\pgflineto{\pgfpoint{280.223969pt}{66.050522pt}}
\pgfpathclose
\pgfusepath{fill,stroke}
\pgfpathmoveto{\pgfpoint{280.223969pt}{72.227356pt}}
\pgflineto{\pgfpoint{289.151978pt}{72.227356pt}}
\pgflineto{\pgfpoint{289.151978pt}{66.050522pt}}
\pgfpathclose
\pgfusepath{fill,stroke}
\color[rgb]{0.147132,0.512959,0.556973}
\pgfpathmoveto{\pgfpoint{280.223969pt}{78.404205pt}}
\pgflineto{\pgfpoint{289.151978pt}{72.227356pt}}
\pgflineto{\pgfpoint{280.223969pt}{72.227356pt}}
\pgfpathclose
\pgfusepath{fill,stroke}
\pgfpathmoveto{\pgfpoint{280.223969pt}{78.404205pt}}
\pgflineto{\pgfpoint{289.151978pt}{78.404205pt}}
\pgflineto{\pgfpoint{289.151978pt}{72.227356pt}}
\pgfpathclose
\pgfusepath{fill,stroke}
\color[rgb]{0.141402,0.527854,0.555864}
\pgfpathmoveto{\pgfpoint{280.223969pt}{84.581039pt}}
\pgflineto{\pgfpoint{289.151978pt}{78.404205pt}}
\pgflineto{\pgfpoint{280.223969pt}{78.404205pt}}
\pgfpathclose
\pgfusepath{fill,stroke}
\color[rgb]{0.147132,0.512959,0.556973}
\pgfpathmoveto{\pgfpoint{289.151978pt}{72.227356pt}}
\pgflineto{\pgfpoint{298.079987pt}{72.227356pt}}
\pgflineto{\pgfpoint{298.079987pt}{66.050522pt}}
\pgfpathclose
\pgfusepath{fill,stroke}
\color[rgb]{0.141402,0.527854,0.555864}
\pgfpathmoveto{\pgfpoint{289.151978pt}{78.404205pt}}
\pgflineto{\pgfpoint{298.079987pt}{72.227356pt}}
\pgflineto{\pgfpoint{289.151978pt}{72.227356pt}}
\pgfpathclose
\pgfusepath{fill,stroke}
\pgfpathmoveto{\pgfpoint{289.151978pt}{78.404205pt}}
\pgflineto{\pgfpoint{298.079987pt}{78.404205pt}}
\pgflineto{\pgfpoint{298.079987pt}{72.227356pt}}
\pgfpathclose
\pgfusepath{fill,stroke}
\color[rgb]{0.152951,0.498053,0.557685}
\pgfpathmoveto{\pgfpoint{298.079987pt}{59.873672pt}}
\pgflineto{\pgfpoint{307.007965pt}{59.873672pt}}
\pgflineto{\pgfpoint{307.007965pt}{53.696838pt}}
\pgfpathclose
\pgfusepath{fill,stroke}
\color[rgb]{0.147132,0.512959,0.556973}
\pgfpathmoveto{\pgfpoint{298.079987pt}{66.050522pt}}
\pgflineto{\pgfpoint{307.007965pt}{59.873672pt}}
\pgflineto{\pgfpoint{298.079987pt}{59.873672pt}}
\pgfpathclose
\pgfusepath{fill,stroke}
\pgfpathmoveto{\pgfpoint{298.079987pt}{66.050522pt}}
\pgflineto{\pgfpoint{307.007965pt}{66.050522pt}}
\pgflineto{\pgfpoint{307.007965pt}{59.873672pt}}
\pgfpathclose
\pgfusepath{fill,stroke}
\color[rgb]{0.141402,0.527854,0.555864}
\pgfpathmoveto{\pgfpoint{298.079987pt}{72.227356pt}}
\pgflineto{\pgfpoint{307.007965pt}{66.050522pt}}
\pgflineto{\pgfpoint{298.079987pt}{66.050522pt}}
\pgfpathclose
\pgfusepath{fill,stroke}
\pgfpathmoveto{\pgfpoint{298.079987pt}{72.227356pt}}
\pgflineto{\pgfpoint{307.007965pt}{72.227356pt}}
\pgflineto{\pgfpoint{307.007965pt}{66.050522pt}}
\pgfpathclose
\pgfusepath{fill,stroke}
\color[rgb]{0.135833,0.542750,0.554289}
\pgfpathmoveto{\pgfpoint{298.079987pt}{78.404205pt}}
\pgflineto{\pgfpoint{307.007965pt}{72.227356pt}}
\pgflineto{\pgfpoint{298.079987pt}{72.227356pt}}
\pgfpathclose
\pgfusepath{fill,stroke}
\color[rgb]{0.147132,0.512959,0.556973}
\pgfpathmoveto{\pgfpoint{307.007965pt}{59.873672pt}}
\pgflineto{\pgfpoint{315.935974pt}{53.696838pt}}
\pgflineto{\pgfpoint{307.007965pt}{53.696838pt}}
\pgfpathclose
\pgfusepath{fill,stroke}
\color[rgb]{0.141402,0.527854,0.555864}
\pgfpathmoveto{\pgfpoint{307.007965pt}{66.050522pt}}
\pgflineto{\pgfpoint{315.935974pt}{66.050522pt}}
\pgflineto{\pgfpoint{315.935974pt}{59.873672pt}}
\pgfpathclose
\pgfusepath{fill,stroke}
\color[rgb]{0.135833,0.542750,0.554289}
\pgfpathmoveto{\pgfpoint{307.007965pt}{72.227356pt}}
\pgflineto{\pgfpoint{315.935974pt}{66.050522pt}}
\pgflineto{\pgfpoint{307.007965pt}{66.050522pt}}
\pgfpathclose
\pgfusepath{fill,stroke}
\pgfpathmoveto{\pgfpoint{307.007965pt}{72.227356pt}}
\pgflineto{\pgfpoint{315.935974pt}{72.227356pt}}
\pgflineto{\pgfpoint{315.935974pt}{66.050522pt}}
\pgfpathclose
\pgfusepath{fill,stroke}
\color[rgb]{0.141402,0.527854,0.555864}
\pgfpathmoveto{\pgfpoint{315.935974pt}{53.696838pt}}
\pgflineto{\pgfpoint{324.863983pt}{53.696838pt}}
\pgflineto{\pgfpoint{324.863983pt}{47.519989pt}}
\pgfpathclose
\pgfusepath{fill,stroke}
\color[rgb]{0.135833,0.542750,0.554289}
\pgfpathmoveto{\pgfpoint{315.935974pt}{59.873672pt}}
\pgflineto{\pgfpoint{324.863983pt}{53.696838pt}}
\pgflineto{\pgfpoint{315.935974pt}{53.696838pt}}
\pgfpathclose
\pgfusepath{fill,stroke}
\pgfpathmoveto{\pgfpoint{315.935974pt}{59.873672pt}}
\pgflineto{\pgfpoint{324.863983pt}{59.873672pt}}
\pgflineto{\pgfpoint{324.863983pt}{53.696838pt}}
\pgfpathclose
\pgfusepath{fill,stroke}
\pgfpathmoveto{\pgfpoint{315.935974pt}{66.050522pt}}
\pgflineto{\pgfpoint{324.863983pt}{59.873672pt}}
\pgflineto{\pgfpoint{315.935974pt}{59.873672pt}}
\pgfpathclose
\pgfusepath{fill,stroke}
\pgfpathmoveto{\pgfpoint{315.935974pt}{66.050522pt}}
\pgflineto{\pgfpoint{324.863983pt}{66.050522pt}}
\pgflineto{\pgfpoint{324.863983pt}{59.873672pt}}
\pgfpathclose
\pgfusepath{fill,stroke}
\color[rgb]{0.130582,0.557652,0.552176}
\pgfpathmoveto{\pgfpoint{315.935974pt}{72.227356pt}}
\pgflineto{\pgfpoint{324.863983pt}{66.050522pt}}
\pgflineto{\pgfpoint{315.935974pt}{66.050522pt}}
\pgfpathclose
\pgfusepath{fill,stroke}
\color[rgb]{0.135833,0.542750,0.554289}
\pgfpathmoveto{\pgfpoint{324.863983pt}{53.696838pt}}
\pgflineto{\pgfpoint{333.791992pt}{47.519989pt}}
\pgflineto{\pgfpoint{324.863983pt}{47.519989pt}}
\pgfpathclose
\pgfusepath{fill,stroke}
\color[rgb]{0.130582,0.557652,0.552176}
\pgfpathmoveto{\pgfpoint{324.863983pt}{59.873672pt}}
\pgflineto{\pgfpoint{333.791992pt}{59.873672pt}}
\pgflineto{\pgfpoint{333.791992pt}{53.696838pt}}
\pgfpathclose
\pgfusepath{fill,stroke}
\pgfpathmoveto{\pgfpoint{324.863983pt}{66.050522pt}}
\pgflineto{\pgfpoint{333.791992pt}{59.873672pt}}
\pgflineto{\pgfpoint{324.863983pt}{59.873672pt}}
\pgfpathclose
\pgfusepath{fill,stroke}
\pgfpathmoveto{\pgfpoint{324.863983pt}{66.050522pt}}
\pgflineto{\pgfpoint{333.791992pt}{66.050522pt}}
\pgflineto{\pgfpoint{333.791992pt}{59.873672pt}}
\pgfpathclose
\pgfusepath{fill,stroke}
\pgfpathmoveto{\pgfpoint{333.791992pt}{53.696838pt}}
\pgflineto{\pgfpoint{342.719971pt}{53.696838pt}}
\pgflineto{\pgfpoint{342.719971pt}{47.519989pt}}
\pgfpathclose
\pgfusepath{fill,stroke}
\color[rgb]{0.125898,0.572563,0.549445}
\pgfpathmoveto{\pgfpoint{333.791992pt}{59.873672pt}}
\pgflineto{\pgfpoint{342.719971pt}{53.696838pt}}
\pgflineto{\pgfpoint{333.791992pt}{53.696838pt}}
\pgfpathclose
\pgfusepath{fill,stroke}
\pgfpathmoveto{\pgfpoint{333.791992pt}{59.873672pt}}
\pgflineto{\pgfpoint{342.719971pt}{59.873672pt}}
\pgflineto{\pgfpoint{342.719971pt}{53.696838pt}}
\pgfpathclose
\pgfusepath{fill,stroke}
\color[rgb]{0.122163,0.587476,0.546023}
\pgfpathmoveto{\pgfpoint{333.791992pt}{66.050522pt}}
\pgflineto{\pgfpoint{342.719971pt}{59.873672pt}}
\pgflineto{\pgfpoint{333.791992pt}{59.873672pt}}
\pgfpathclose
\pgfusepath{fill,stroke}
\color[rgb]{0.125898,0.572563,0.549445}
\pgfpathmoveto{\pgfpoint{342.719971pt}{53.696838pt}}
\pgflineto{\pgfpoint{351.647980pt}{47.519989pt}}
\pgflineto{\pgfpoint{342.719971pt}{47.519989pt}}
\pgfpathclose
\pgfusepath{fill,stroke}
\pgfpathmoveto{\pgfpoint{342.719971pt}{53.696838pt}}
\pgflineto{\pgfpoint{351.647980pt}{53.696838pt}}
\pgflineto{\pgfpoint{351.647980pt}{47.519989pt}}
\pgfpathclose
\pgfusepath{fill,stroke}
\color[rgb]{0.122163,0.587476,0.546023}
\pgfpathmoveto{\pgfpoint{342.719971pt}{59.873672pt}}
\pgflineto{\pgfpoint{351.647980pt}{53.696838pt}}
\pgflineto{\pgfpoint{342.719971pt}{53.696838pt}}
\pgfpathclose
\pgfusepath{fill,stroke}
\pgfpathmoveto{\pgfpoint{351.647980pt}{53.696838pt}}
\pgflineto{\pgfpoint{360.575958pt}{47.519989pt}}
\pgflineto{\pgfpoint{351.647980pt}{47.519989pt}}
\pgfpathclose
\pgfusepath{fill,stroke}
\pgfpathmoveto{\pgfpoint{351.647980pt}{53.696838pt}}
\pgflineto{\pgfpoint{360.575958pt}{53.696838pt}}
\pgflineto{\pgfpoint{360.575958pt}{47.519989pt}}
\pgfpathclose
\pgfusepath{fill,stroke}
\color[rgb]{0.119872,0.602382,0.541831}
\pgfpathmoveto{\pgfpoint{360.575958pt}{53.696838pt}}
\pgflineto{\pgfpoint{369.503998pt}{47.519989pt}}
\pgflineto{\pgfpoint{360.575958pt}{47.519989pt}}
\pgfpathclose
\pgfusepath{fill,stroke}
\color[rgb]{0.190631,0.407061,0.556089}
\pgfpathmoveto{\pgfpoint{164.160004pt}{133.995789pt}}
\pgflineto{\pgfpoint{173.087997pt}{127.818947pt}}
\pgflineto{\pgfpoint{164.160004pt}{127.818947pt}}
\pgfpathclose
\pgfusepath{fill,stroke}
\pgfpathmoveto{\pgfpoint{164.160004pt}{133.995789pt}}
\pgflineto{\pgfpoint{173.087997pt}{133.995789pt}}
\pgflineto{\pgfpoint{173.087997pt}{127.818947pt}}
\pgfpathclose
\pgfusepath{fill,stroke}
\color[rgb]{0.183819,0.422564,0.556952}
\pgfpathmoveto{\pgfpoint{164.160004pt}{140.172638pt}}
\pgflineto{\pgfpoint{173.087997pt}{133.995789pt}}
\pgflineto{\pgfpoint{164.160004pt}{133.995789pt}}
\pgfpathclose
\pgfusepath{fill,stroke}
\pgfpathmoveto{\pgfpoint{164.160004pt}{140.172638pt}}
\pgflineto{\pgfpoint{173.087997pt}{140.172638pt}}
\pgflineto{\pgfpoint{173.087997pt}{133.995789pt}}
\pgfpathclose
\pgfusepath{fill,stroke}
\color[rgb]{0.177272,0.437886,0.557576}
\pgfpathmoveto{\pgfpoint{164.160004pt}{146.349472pt}}
\pgflineto{\pgfpoint{173.087997pt}{146.349472pt}}
\pgflineto{\pgfpoint{173.087997pt}{140.172638pt}}
\pgfpathclose
\pgfusepath{fill,stroke}
\color[rgb]{0.170958,0.453063,0.557974}
\pgfpathmoveto{\pgfpoint{164.160004pt}{152.526306pt}}
\pgflineto{\pgfpoint{173.087997pt}{146.349472pt}}
\pgflineto{\pgfpoint{164.160004pt}{146.349472pt}}
\pgfpathclose
\pgfusepath{fill,stroke}
\color[rgb]{0.190631,0.407061,0.556089}
\pgfpathmoveto{\pgfpoint{173.087997pt}{127.818947pt}}
\pgflineto{\pgfpoint{182.015991pt}{127.818947pt}}
\pgflineto{\pgfpoint{182.015991pt}{121.642097pt}}
\pgfpathclose
\pgfusepath{fill,stroke}
\color[rgb]{0.183819,0.422564,0.556952}
\pgfpathmoveto{\pgfpoint{173.087997pt}{133.995789pt}}
\pgflineto{\pgfpoint{182.015991pt}{127.818947pt}}
\pgflineto{\pgfpoint{173.087997pt}{127.818947pt}}
\pgfpathclose
\pgfusepath{fill,stroke}
\pgfpathmoveto{\pgfpoint{173.087997pt}{133.995789pt}}
\pgflineto{\pgfpoint{182.015991pt}{133.995789pt}}
\pgflineto{\pgfpoint{182.015991pt}{127.818947pt}}
\pgfpathclose
\pgfusepath{fill,stroke}
\color[rgb]{0.177272,0.437886,0.557576}
\pgfpathmoveto{\pgfpoint{173.087997pt}{140.172638pt}}
\pgflineto{\pgfpoint{182.015991pt}{133.995789pt}}
\pgflineto{\pgfpoint{173.087997pt}{133.995789pt}}
\pgfpathclose
\pgfusepath{fill,stroke}
\pgfpathmoveto{\pgfpoint{182.015991pt}{127.818947pt}}
\pgflineto{\pgfpoint{190.943985pt}{121.642097pt}}
\pgflineto{\pgfpoint{182.015991pt}{121.642097pt}}
\pgfpathclose
\pgfusepath{fill,stroke}
\pgfpathmoveto{\pgfpoint{182.015991pt}{127.818947pt}}
\pgflineto{\pgfpoint{190.943985pt}{127.818947pt}}
\pgflineto{\pgfpoint{190.943985pt}{121.642097pt}}
\pgfpathclose
\pgfusepath{fill,stroke}
\pgfpathmoveto{\pgfpoint{182.015991pt}{133.995789pt}}
\pgflineto{\pgfpoint{190.943985pt}{127.818947pt}}
\pgflineto{\pgfpoint{182.015991pt}{127.818947pt}}
\pgfpathclose
\pgfusepath{fill,stroke}
\pgfpathmoveto{\pgfpoint{182.015991pt}{133.995789pt}}
\pgflineto{\pgfpoint{190.943985pt}{133.995789pt}}
\pgflineto{\pgfpoint{190.943985pt}{127.818947pt}}
\pgfpathclose
\pgfusepath{fill,stroke}
\color[rgb]{0.170958,0.453063,0.557974}
\pgfpathmoveto{\pgfpoint{182.015991pt}{140.172638pt}}
\pgflineto{\pgfpoint{190.943985pt}{140.172638pt}}
\pgflineto{\pgfpoint{190.943985pt}{133.995789pt}}
\pgfpathclose
\pgfusepath{fill,stroke}
\color[rgb]{0.164833,0.468130,0.558143}
\pgfpathmoveto{\pgfpoint{182.015991pt}{146.349472pt}}
\pgflineto{\pgfpoint{190.943985pt}{140.172638pt}}
\pgflineto{\pgfpoint{182.015991pt}{140.172638pt}}
\pgfpathclose
\pgfusepath{fill,stroke}
\color[rgb]{0.119872,0.602382,0.541831}
\pgfpathmoveto{\pgfpoint{360.575958pt}{53.696838pt}}
\pgflineto{\pgfpoint{369.503998pt}{53.696838pt}}
\pgflineto{\pgfpoint{369.503998pt}{47.519989pt}}
\pgfpathclose
\pgfusepath{fill,stroke}
\color[rgb]{0.119627,0.617266,0.536796}
\pgfpathmoveto{\pgfpoint{360.575958pt}{59.873672pt}}
\pgflineto{\pgfpoint{369.503998pt}{53.696838pt}}
\pgflineto{\pgfpoint{360.575958pt}{53.696838pt}}
\pgfpathclose
\pgfusepath{fill,stroke}
\color[rgb]{0.122046,0.632107,0.530848}
\pgfpathmoveto{\pgfpoint{378.431976pt}{53.696838pt}}
\pgflineto{\pgfpoint{387.359985pt}{47.519989pt}}
\pgflineto{\pgfpoint{378.431976pt}{47.519989pt}}
\pgfpathclose
\pgfusepath{fill,stroke}
\color[rgb]{0.130582,0.557652,0.552176}
\pgfpathmoveto{\pgfpoint{315.935974pt}{72.227356pt}}
\pgflineto{\pgfpoint{324.863983pt}{72.227356pt}}
\pgflineto{\pgfpoint{324.863983pt}{66.050522pt}}
\pgfpathclose
\pgfusepath{fill,stroke}
\color[rgb]{0.125898,0.572563,0.549445}
\pgfpathmoveto{\pgfpoint{315.935974pt}{78.404205pt}}
\pgflineto{\pgfpoint{324.863983pt}{72.227356pt}}
\pgflineto{\pgfpoint{315.935974pt}{72.227356pt}}
\pgfpathclose
\pgfusepath{fill,stroke}
\color[rgb]{0.122163,0.587476,0.546023}
\pgfpathmoveto{\pgfpoint{333.791992pt}{66.050522pt}}
\pgflineto{\pgfpoint{342.719971pt}{66.050522pt}}
\pgflineto{\pgfpoint{342.719971pt}{59.873672pt}}
\pgfpathclose
\pgfusepath{fill,stroke}
\pgfpathmoveto{\pgfpoint{333.791992pt}{72.227356pt}}
\pgflineto{\pgfpoint{342.719971pt}{66.050522pt}}
\pgflineto{\pgfpoint{333.791992pt}{66.050522pt}}
\pgfpathclose
\pgfusepath{fill,stroke}
\pgfpathmoveto{\pgfpoint{342.719971pt}{59.873672pt}}
\pgflineto{\pgfpoint{351.647980pt}{59.873672pt}}
\pgflineto{\pgfpoint{351.647980pt}{53.696838pt}}
\pgfpathclose
\pgfusepath{fill,stroke}
\color[rgb]{0.119872,0.602382,0.541831}
\pgfpathmoveto{\pgfpoint{342.719971pt}{66.050522pt}}
\pgflineto{\pgfpoint{351.647980pt}{59.873672pt}}
\pgflineto{\pgfpoint{342.719971pt}{59.873672pt}}
\pgfpathclose
\pgfusepath{fill,stroke}
\color[rgb]{0.141402,0.527854,0.555864}
\pgfpathmoveto{\pgfpoint{280.223969pt}{84.581039pt}}
\pgflineto{\pgfpoint{289.151978pt}{84.581039pt}}
\pgflineto{\pgfpoint{289.151978pt}{78.404205pt}}
\pgfpathclose
\pgfusepath{fill,stroke}
\pgfpathmoveto{\pgfpoint{280.223969pt}{90.757896pt}}
\pgflineto{\pgfpoint{289.151978pt}{84.581039pt}}
\pgflineto{\pgfpoint{280.223969pt}{84.581039pt}}
\pgfpathclose
\pgfusepath{fill,stroke}
\color[rgb]{0.135833,0.542750,0.554289}
\pgfpathmoveto{\pgfpoint{298.079987pt}{78.404205pt}}
\pgflineto{\pgfpoint{307.007965pt}{78.404205pt}}
\pgflineto{\pgfpoint{307.007965pt}{72.227356pt}}
\pgfpathclose
\pgfusepath{fill,stroke}
\color[rgb]{0.130582,0.557652,0.552176}
\pgfpathmoveto{\pgfpoint{298.079987pt}{84.581039pt}}
\pgflineto{\pgfpoint{307.007965pt}{78.404205pt}}
\pgflineto{\pgfpoint{298.079987pt}{78.404205pt}}
\pgfpathclose
\pgfusepath{fill,stroke}
\color[rgb]{0.158845,0.483117,0.558059}
\pgfpathmoveto{\pgfpoint{235.583969pt}{103.111580pt}}
\pgflineto{\pgfpoint{244.511993pt}{103.111580pt}}
\pgflineto{\pgfpoint{244.511993pt}{96.934731pt}}
\pgfpathclose
\pgfusepath{fill,stroke}
\color[rgb]{0.152951,0.498053,0.557685}
\pgfpathmoveto{\pgfpoint{235.583969pt}{109.288422pt}}
\pgflineto{\pgfpoint{244.511993pt}{103.111580pt}}
\pgflineto{\pgfpoint{235.583969pt}{103.111580pt}}
\pgfpathclose
\pgfusepath{fill,stroke}
\pgfpathmoveto{\pgfpoint{253.440002pt}{96.934731pt}}
\pgflineto{\pgfpoint{262.367981pt}{96.934731pt}}
\pgflineto{\pgfpoint{262.367981pt}{90.757896pt}}
\pgfpathclose
\pgfusepath{fill,stroke}
\color[rgb]{0.147132,0.512959,0.556973}
\pgfpathmoveto{\pgfpoint{253.440002pt}{103.111580pt}}
\pgflineto{\pgfpoint{262.367981pt}{96.934731pt}}
\pgflineto{\pgfpoint{253.440002pt}{96.934731pt}}
\pgfpathclose
\pgfusepath{fill,stroke}
\color[rgb]{0.152951,0.498053,0.557685}
\pgfpathmoveto{\pgfpoint{262.367981pt}{90.757896pt}}
\pgflineto{\pgfpoint{271.295990pt}{90.757896pt}}
\pgflineto{\pgfpoint{271.295990pt}{84.581039pt}}
\pgfpathclose
\pgfusepath{fill,stroke}
\color[rgb]{0.147132,0.512959,0.556973}
\pgfpathmoveto{\pgfpoint{262.367981pt}{96.934731pt}}
\pgflineto{\pgfpoint{271.295990pt}{90.757896pt}}
\pgflineto{\pgfpoint{262.367981pt}{90.757896pt}}
\pgfpathclose
\pgfusepath{fill,stroke}
\color[rgb]{0.164833,0.468130,0.558143}
\pgfpathmoveto{\pgfpoint{217.727982pt}{115.465263pt}}
\pgflineto{\pgfpoint{226.655975pt}{115.465263pt}}
\pgflineto{\pgfpoint{226.655975pt}{109.288422pt}}
\pgfpathclose
\pgfusepath{fill,stroke}
\color[rgb]{0.158845,0.483117,0.558059}
\pgfpathmoveto{\pgfpoint{217.727982pt}{121.642097pt}}
\pgflineto{\pgfpoint{226.655975pt}{115.465263pt}}
\pgflineto{\pgfpoint{217.727982pt}{115.465263pt}}
\pgfpathclose
\pgfusepath{fill,stroke}
\pgfpathmoveto{\pgfpoint{217.727982pt}{121.642097pt}}
\pgflineto{\pgfpoint{226.655975pt}{121.642097pt}}
\pgflineto{\pgfpoint{226.655975pt}{115.465263pt}}
\pgfpathclose
\pgfusepath{fill,stroke}
\color[rgb]{0.164833,0.468130,0.558143}
\pgfpathmoveto{\pgfpoint{226.655975pt}{109.288422pt}}
\pgflineto{\pgfpoint{235.583969pt}{103.111580pt}}
\pgflineto{\pgfpoint{226.655975pt}{103.111580pt}}
\pgfpathclose
\pgfusepath{fill,stroke}
\pgfpathmoveto{\pgfpoint{226.655975pt}{109.288422pt}}
\pgflineto{\pgfpoint{235.583969pt}{109.288422pt}}
\pgflineto{\pgfpoint{235.583969pt}{103.111580pt}}
\pgfpathclose
\pgfusepath{fill,stroke}
\color[rgb]{0.158845,0.483117,0.558059}
\pgfpathmoveto{\pgfpoint{226.655975pt}{115.465263pt}}
\pgflineto{\pgfpoint{235.583969pt}{109.288422pt}}
\pgflineto{\pgfpoint{226.655975pt}{109.288422pt}}
\pgfpathclose
\pgfusepath{fill,stroke}
\pgfpathmoveto{\pgfpoint{226.655975pt}{115.465263pt}}
\pgflineto{\pgfpoint{235.583969pt}{115.465263pt}}
\pgflineto{\pgfpoint{235.583969pt}{109.288422pt}}
\pgfpathclose
\pgfusepath{fill,stroke}
\color[rgb]{0.152951,0.498053,0.557685}
\pgfpathmoveto{\pgfpoint{226.655975pt}{121.642097pt}}
\pgflineto{\pgfpoint{235.583969pt}{115.465263pt}}
\pgflineto{\pgfpoint{226.655975pt}{115.465263pt}}
\pgfpathclose
\pgfusepath{fill,stroke}
\pgfpathmoveto{\pgfpoint{235.583969pt}{109.288422pt}}
\pgflineto{\pgfpoint{244.511993pt}{109.288422pt}}
\pgflineto{\pgfpoint{244.511993pt}{103.111580pt}}
\pgfpathclose
\pgfusepath{fill,stroke}
\pgfpathmoveto{\pgfpoint{235.583969pt}{115.465263pt}}
\pgflineto{\pgfpoint{244.511993pt}{109.288422pt}}
\pgflineto{\pgfpoint{235.583969pt}{109.288422pt}}
\pgfpathclose
\pgfusepath{fill,stroke}
\pgfpathmoveto{\pgfpoint{244.511993pt}{103.111580pt}}
\pgflineto{\pgfpoint{253.440002pt}{96.934731pt}}
\pgflineto{\pgfpoint{244.511993pt}{96.934731pt}}
\pgfpathclose
\pgfusepath{fill,stroke}
\pgfpathmoveto{\pgfpoint{244.511993pt}{103.111580pt}}
\pgflineto{\pgfpoint{253.440002pt}{103.111580pt}}
\pgflineto{\pgfpoint{253.440002pt}{96.934731pt}}
\pgfpathclose
\pgfusepath{fill,stroke}
\color[rgb]{0.147132,0.512959,0.556973}
\pgfpathmoveto{\pgfpoint{244.511993pt}{109.288422pt}}
\pgflineto{\pgfpoint{253.440002pt}{103.111580pt}}
\pgflineto{\pgfpoint{244.511993pt}{103.111580pt}}
\pgfpathclose
\pgfusepath{fill,stroke}
\pgfpathmoveto{\pgfpoint{244.511993pt}{109.288422pt}}
\pgflineto{\pgfpoint{253.440002pt}{109.288422pt}}
\pgflineto{\pgfpoint{253.440002pt}{103.111580pt}}
\pgfpathclose
\pgfusepath{fill,stroke}
\pgfpathmoveto{\pgfpoint{253.440002pt}{103.111580pt}}
\pgflineto{\pgfpoint{262.367981pt}{103.111580pt}}
\pgflineto{\pgfpoint{262.367981pt}{96.934731pt}}
\pgfpathclose
\pgfusepath{fill,stroke}
\color[rgb]{0.141402,0.527854,0.555864}
\pgfpathmoveto{\pgfpoint{253.440002pt}{109.288422pt}}
\pgflineto{\pgfpoint{262.367981pt}{103.111580pt}}
\pgflineto{\pgfpoint{253.440002pt}{103.111580pt}}
\pgfpathclose
\pgfusepath{fill,stroke}
\color[rgb]{0.147132,0.512959,0.556973}
\pgfpathmoveto{\pgfpoint{262.367981pt}{96.934731pt}}
\pgflineto{\pgfpoint{271.295990pt}{96.934731pt}}
\pgflineto{\pgfpoint{271.295990pt}{90.757896pt}}
\pgfpathclose
\pgfusepath{fill,stroke}
\color[rgb]{0.141402,0.527854,0.555864}
\pgfpathmoveto{\pgfpoint{262.367981pt}{103.111580pt}}
\pgflineto{\pgfpoint{271.295990pt}{96.934731pt}}
\pgflineto{\pgfpoint{262.367981pt}{96.934731pt}}
\pgfpathclose
\pgfusepath{fill,stroke}
\pgfpathmoveto{\pgfpoint{262.367981pt}{103.111580pt}}
\pgflineto{\pgfpoint{271.295990pt}{103.111580pt}}
\pgflineto{\pgfpoint{271.295990pt}{96.934731pt}}
\pgfpathclose
\pgfusepath{fill,stroke}
\color[rgb]{0.147132,0.512959,0.556973}
\pgfpathmoveto{\pgfpoint{271.295990pt}{90.757896pt}}
\pgflineto{\pgfpoint{280.223969pt}{84.581039pt}}
\pgflineto{\pgfpoint{271.295990pt}{84.581039pt}}
\pgfpathclose
\pgfusepath{fill,stroke}
\pgfpathmoveto{\pgfpoint{271.295990pt}{90.757896pt}}
\pgflineto{\pgfpoint{280.223969pt}{90.757896pt}}
\pgflineto{\pgfpoint{280.223969pt}{84.581039pt}}
\pgfpathclose
\pgfusepath{fill,stroke}
\color[rgb]{0.141402,0.527854,0.555864}
\pgfpathmoveto{\pgfpoint{271.295990pt}{96.934731pt}}
\pgflineto{\pgfpoint{280.223969pt}{90.757896pt}}
\pgflineto{\pgfpoint{271.295990pt}{90.757896pt}}
\pgfpathclose
\pgfusepath{fill,stroke}
\pgfpathmoveto{\pgfpoint{271.295990pt}{96.934731pt}}
\pgflineto{\pgfpoint{280.223969pt}{96.934731pt}}
\pgflineto{\pgfpoint{280.223969pt}{90.757896pt}}
\pgfpathclose
\pgfusepath{fill,stroke}
\color[rgb]{0.135833,0.542750,0.554289}
\pgfpathmoveto{\pgfpoint{271.295990pt}{103.111580pt}}
\pgflineto{\pgfpoint{280.223969pt}{96.934731pt}}
\pgflineto{\pgfpoint{271.295990pt}{96.934731pt}}
\pgfpathclose
\pgfusepath{fill,stroke}
\color[rgb]{0.141402,0.527854,0.555864}
\pgfpathmoveto{\pgfpoint{280.223969pt}{90.757896pt}}
\pgflineto{\pgfpoint{289.151978pt}{90.757896pt}}
\pgflineto{\pgfpoint{289.151978pt}{84.581039pt}}
\pgfpathclose
\pgfusepath{fill,stroke}
\color[rgb]{0.135833,0.542750,0.554289}
\pgfpathmoveto{\pgfpoint{280.223969pt}{96.934731pt}}
\pgflineto{\pgfpoint{289.151978pt}{90.757896pt}}
\pgflineto{\pgfpoint{280.223969pt}{90.757896pt}}
\pgfpathclose
\pgfusepath{fill,stroke}
\pgfpathmoveto{\pgfpoint{280.223969pt}{96.934731pt}}
\pgflineto{\pgfpoint{289.151978pt}{96.934731pt}}
\pgflineto{\pgfpoint{289.151978pt}{90.757896pt}}
\pgfpathclose
\pgfusepath{fill,stroke}
\pgfpathmoveto{\pgfpoint{289.151978pt}{84.581039pt}}
\pgflineto{\pgfpoint{298.079987pt}{78.404205pt}}
\pgflineto{\pgfpoint{289.151978pt}{78.404205pt}}
\pgfpathclose
\pgfusepath{fill,stroke}
\pgfpathmoveto{\pgfpoint{289.151978pt}{84.581039pt}}
\pgflineto{\pgfpoint{298.079987pt}{84.581039pt}}
\pgflineto{\pgfpoint{298.079987pt}{78.404205pt}}
\pgfpathclose
\pgfusepath{fill,stroke}
\pgfpathmoveto{\pgfpoint{289.151978pt}{90.757896pt}}
\pgflineto{\pgfpoint{298.079987pt}{84.581039pt}}
\pgflineto{\pgfpoint{289.151978pt}{84.581039pt}}
\pgfpathclose
\pgfusepath{fill,stroke}
\pgfpathmoveto{\pgfpoint{289.151978pt}{90.757896pt}}
\pgflineto{\pgfpoint{298.079987pt}{90.757896pt}}
\pgflineto{\pgfpoint{298.079987pt}{84.581039pt}}
\pgfpathclose
\pgfusepath{fill,stroke}
\color[rgb]{0.130582,0.557652,0.552176}
\pgfpathmoveto{\pgfpoint{289.151978pt}{96.934731pt}}
\pgflineto{\pgfpoint{298.079987pt}{90.757896pt}}
\pgflineto{\pgfpoint{289.151978pt}{90.757896pt}}
\pgfpathclose
\pgfusepath{fill,stroke}
\pgfpathmoveto{\pgfpoint{298.079987pt}{84.581039pt}}
\pgflineto{\pgfpoint{307.007965pt}{84.581039pt}}
\pgflineto{\pgfpoint{307.007965pt}{78.404205pt}}
\pgfpathclose
\pgfusepath{fill,stroke}
\color[rgb]{0.125898,0.572563,0.549445}
\pgfpathmoveto{\pgfpoint{298.079987pt}{90.757896pt}}
\pgflineto{\pgfpoint{307.007965pt}{84.581039pt}}
\pgflineto{\pgfpoint{298.079987pt}{84.581039pt}}
\pgfpathclose
\pgfusepath{fill,stroke}
\color[rgb]{0.130582,0.557652,0.552176}
\pgfpathmoveto{\pgfpoint{307.007965pt}{78.404205pt}}
\pgflineto{\pgfpoint{315.935974pt}{72.227356pt}}
\pgflineto{\pgfpoint{307.007965pt}{72.227356pt}}
\pgfpathclose
\pgfusepath{fill,stroke}
\pgfpathmoveto{\pgfpoint{307.007965pt}{78.404205pt}}
\pgflineto{\pgfpoint{315.935974pt}{78.404205pt}}
\pgflineto{\pgfpoint{315.935974pt}{72.227356pt}}
\pgfpathclose
\pgfusepath{fill,stroke}
\color[rgb]{0.125898,0.572563,0.549445}
\pgfpathmoveto{\pgfpoint{307.007965pt}{84.581039pt}}
\pgflineto{\pgfpoint{315.935974pt}{78.404205pt}}
\pgflineto{\pgfpoint{307.007965pt}{78.404205pt}}
\pgfpathclose
\pgfusepath{fill,stroke}
\pgfpathmoveto{\pgfpoint{307.007965pt}{84.581039pt}}
\pgflineto{\pgfpoint{315.935974pt}{84.581039pt}}
\pgflineto{\pgfpoint{315.935974pt}{78.404205pt}}
\pgfpathclose
\pgfusepath{fill,stroke}
\pgfpathmoveto{\pgfpoint{315.935974pt}{78.404205pt}}
\pgflineto{\pgfpoint{324.863983pt}{78.404205pt}}
\pgflineto{\pgfpoint{324.863983pt}{72.227356pt}}
\pgfpathclose
\pgfusepath{fill,stroke}
\color[rgb]{0.122163,0.587476,0.546023}
\pgfpathmoveto{\pgfpoint{315.935974pt}{84.581039pt}}
\pgflineto{\pgfpoint{324.863983pt}{78.404205pt}}
\pgflineto{\pgfpoint{315.935974pt}{78.404205pt}}
\pgfpathclose
\pgfusepath{fill,stroke}
\color[rgb]{0.125898,0.572563,0.549445}
\pgfpathmoveto{\pgfpoint{324.863983pt}{72.227356pt}}
\pgflineto{\pgfpoint{333.791992pt}{66.050522pt}}
\pgflineto{\pgfpoint{324.863983pt}{66.050522pt}}
\pgfpathclose
\pgfusepath{fill,stroke}
\pgfpathmoveto{\pgfpoint{324.863983pt}{72.227356pt}}
\pgflineto{\pgfpoint{333.791992pt}{72.227356pt}}
\pgflineto{\pgfpoint{333.791992pt}{66.050522pt}}
\pgfpathclose
\pgfusepath{fill,stroke}
\color[rgb]{0.122163,0.587476,0.546023}
\pgfpathmoveto{\pgfpoint{324.863983pt}{78.404205pt}}
\pgflineto{\pgfpoint{333.791992pt}{72.227356pt}}
\pgflineto{\pgfpoint{324.863983pt}{72.227356pt}}
\pgfpathclose
\pgfusepath{fill,stroke}
\pgfpathmoveto{\pgfpoint{324.863983pt}{78.404205pt}}
\pgflineto{\pgfpoint{333.791992pt}{78.404205pt}}
\pgflineto{\pgfpoint{333.791992pt}{72.227356pt}}
\pgfpathclose
\pgfusepath{fill,stroke}
\pgfpathmoveto{\pgfpoint{333.791992pt}{72.227356pt}}
\pgflineto{\pgfpoint{342.719971pt}{72.227356pt}}
\pgflineto{\pgfpoint{342.719971pt}{66.050522pt}}
\pgfpathclose
\pgfusepath{fill,stroke}
\color[rgb]{0.119872,0.602382,0.541831}
\pgfpathmoveto{\pgfpoint{333.791992pt}{78.404205pt}}
\pgflineto{\pgfpoint{342.719971pt}{72.227356pt}}
\pgflineto{\pgfpoint{333.791992pt}{72.227356pt}}
\pgfpathclose
\pgfusepath{fill,stroke}
\pgfpathmoveto{\pgfpoint{342.719971pt}{66.050522pt}}
\pgflineto{\pgfpoint{351.647980pt}{66.050522pt}}
\pgflineto{\pgfpoint{351.647980pt}{59.873672pt}}
\pgfpathclose
\pgfusepath{fill,stroke}
\pgfpathmoveto{\pgfpoint{342.719971pt}{72.227356pt}}
\pgflineto{\pgfpoint{351.647980pt}{66.050522pt}}
\pgflineto{\pgfpoint{342.719971pt}{66.050522pt}}
\pgfpathclose
\pgfusepath{fill,stroke}
\pgfpathmoveto{\pgfpoint{342.719971pt}{72.227356pt}}
\pgflineto{\pgfpoint{351.647980pt}{72.227356pt}}
\pgflineto{\pgfpoint{351.647980pt}{66.050522pt}}
\pgfpathclose
\pgfusepath{fill,stroke}
\pgfpathmoveto{\pgfpoint{351.647980pt}{59.873672pt}}
\pgflineto{\pgfpoint{360.575958pt}{53.696838pt}}
\pgflineto{\pgfpoint{351.647980pt}{53.696838pt}}
\pgfpathclose
\pgfusepath{fill,stroke}
\pgfpathmoveto{\pgfpoint{351.647980pt}{59.873672pt}}
\pgflineto{\pgfpoint{360.575958pt}{59.873672pt}}
\pgflineto{\pgfpoint{360.575958pt}{53.696838pt}}
\pgfpathclose
\pgfusepath{fill,stroke}
\color[rgb]{0.119627,0.617266,0.536796}
\pgfpathmoveto{\pgfpoint{351.647980pt}{66.050522pt}}
\pgflineto{\pgfpoint{360.575958pt}{59.873672pt}}
\pgflineto{\pgfpoint{351.647980pt}{59.873672pt}}
\pgfpathclose
\pgfusepath{fill,stroke}
\pgfpathmoveto{\pgfpoint{351.647980pt}{66.050522pt}}
\pgflineto{\pgfpoint{360.575958pt}{66.050522pt}}
\pgflineto{\pgfpoint{360.575958pt}{59.873672pt}}
\pgfpathclose
\pgfusepath{fill,stroke}
\color[rgb]{0.122046,0.632107,0.530848}
\pgfpathmoveto{\pgfpoint{351.647980pt}{72.227356pt}}
\pgflineto{\pgfpoint{360.575958pt}{66.050522pt}}
\pgflineto{\pgfpoint{351.647980pt}{66.050522pt}}
\pgfpathclose
\pgfusepath{fill,stroke}
\color[rgb]{0.119627,0.617266,0.536796}
\pgfpathmoveto{\pgfpoint{360.575958pt}{59.873672pt}}
\pgflineto{\pgfpoint{369.503998pt}{59.873672pt}}
\pgflineto{\pgfpoint{369.503998pt}{53.696838pt}}
\pgfpathclose
\pgfusepath{fill,stroke}
\color[rgb]{0.122046,0.632107,0.530848}
\pgfpathmoveto{\pgfpoint{360.575958pt}{66.050522pt}}
\pgflineto{\pgfpoint{369.503998pt}{59.873672pt}}
\pgflineto{\pgfpoint{360.575958pt}{59.873672pt}}
\pgfpathclose
\pgfusepath{fill,stroke}
\pgfpathmoveto{\pgfpoint{360.575958pt}{66.050522pt}}
\pgflineto{\pgfpoint{369.503998pt}{66.050522pt}}
\pgflineto{\pgfpoint{369.503998pt}{59.873672pt}}
\pgfpathclose
\pgfusepath{fill,stroke}
\color[rgb]{0.119627,0.617266,0.536796}
\pgfpathmoveto{\pgfpoint{369.503998pt}{53.696838pt}}
\pgflineto{\pgfpoint{378.431976pt}{47.519989pt}}
\pgflineto{\pgfpoint{369.503998pt}{47.519989pt}}
\pgfpathclose
\pgfusepath{fill,stroke}
\pgfpathmoveto{\pgfpoint{369.503998pt}{53.696838pt}}
\pgflineto{\pgfpoint{378.431976pt}{53.696838pt}}
\pgflineto{\pgfpoint{378.431976pt}{47.519989pt}}
\pgfpathclose
\pgfusepath{fill,stroke}
\color[rgb]{0.122046,0.632107,0.530848}
\pgfpathmoveto{\pgfpoint{369.503998pt}{59.873672pt}}
\pgflineto{\pgfpoint{378.431976pt}{53.696838pt}}
\pgflineto{\pgfpoint{369.503998pt}{53.696838pt}}
\pgfpathclose
\pgfusepath{fill,stroke}
\pgfpathmoveto{\pgfpoint{369.503998pt}{59.873672pt}}
\pgflineto{\pgfpoint{378.431976pt}{59.873672pt}}
\pgflineto{\pgfpoint{378.431976pt}{53.696838pt}}
\pgfpathclose
\pgfusepath{fill,stroke}
\color[rgb]{0.127668,0.646882,0.523924}
\pgfpathmoveto{\pgfpoint{369.503998pt}{66.050522pt}}
\pgflineto{\pgfpoint{378.431976pt}{59.873672pt}}
\pgflineto{\pgfpoint{369.503998pt}{59.873672pt}}
\pgfpathclose
\pgfusepath{fill,stroke}
\color[rgb]{0.122046,0.632107,0.530848}
\pgfpathmoveto{\pgfpoint{378.431976pt}{53.696838pt}}
\pgflineto{\pgfpoint{387.359985pt}{53.696838pt}}
\pgflineto{\pgfpoint{387.359985pt}{47.519989pt}}
\pgfpathclose
\pgfusepath{fill,stroke}
\color[rgb]{0.127668,0.646882,0.523924}
\pgfpathmoveto{\pgfpoint{378.431976pt}{59.873672pt}}
\pgflineto{\pgfpoint{387.359985pt}{53.696838pt}}
\pgflineto{\pgfpoint{378.431976pt}{53.696838pt}}
\pgfpathclose
\pgfusepath{fill,stroke}
\pgfpathmoveto{\pgfpoint{387.359985pt}{53.696838pt}}
\pgflineto{\pgfpoint{396.287964pt}{47.519989pt}}
\pgflineto{\pgfpoint{387.359985pt}{47.519989pt}}
\pgfpathclose
\pgfusepath{fill,stroke}
\pgfpathmoveto{\pgfpoint{387.359985pt}{53.696838pt}}
\pgflineto{\pgfpoint{396.287964pt}{53.696838pt}}
\pgflineto{\pgfpoint{396.287964pt}{47.519989pt}}
\pgfpathclose
\pgfusepath{fill,stroke}
\color[rgb]{0.136835,0.661563,0.515967}
\pgfpathmoveto{\pgfpoint{396.287964pt}{53.696838pt}}
\pgflineto{\pgfpoint{405.216003pt}{47.519989pt}}
\pgflineto{\pgfpoint{396.287964pt}{47.519989pt}}
\pgfpathclose
\pgfusepath{fill,stroke}
\pgfpathmoveto{\pgfpoint{387.359985pt}{59.873672pt}}
\pgflineto{\pgfpoint{396.287964pt}{59.873672pt}}
\pgflineto{\pgfpoint{396.287964pt}{53.696838pt}}
\pgfpathclose
\pgfusepath{fill,stroke}
\color[rgb]{0.149643,0.676120,0.506924}
\pgfpathmoveto{\pgfpoint{387.359985pt}{66.050522pt}}
\pgflineto{\pgfpoint{396.287964pt}{59.873672pt}}
\pgflineto{\pgfpoint{387.359985pt}{59.873672pt}}
\pgfpathclose
\pgfusepath{fill,stroke}
\pgfpathmoveto{\pgfpoint{405.216003pt}{53.696838pt}}
\pgflineto{\pgfpoint{414.143982pt}{53.696838pt}}
\pgflineto{\pgfpoint{414.143982pt}{47.519989pt}}
\pgfpathclose
\pgfusepath{fill,stroke}
\color[rgb]{0.165967,0.690519,0.496752}
\pgfpathmoveto{\pgfpoint{405.216003pt}{59.873672pt}}
\pgflineto{\pgfpoint{414.143982pt}{53.696838pt}}
\pgflineto{\pgfpoint{405.216003pt}{53.696838pt}}
\pgfpathclose
\pgfusepath{fill,stroke}
\color[rgb]{0.185538,0.704725,0.485412}
\pgfpathmoveto{\pgfpoint{414.143982pt}{53.696838pt}}
\pgflineto{\pgfpoint{423.071960pt}{47.519989pt}}
\pgflineto{\pgfpoint{414.143982pt}{47.519989pt}}
\pgfpathclose
\pgfusepath{fill,stroke}
\color[rgb]{0.122046,0.632107,0.530848}
\pgfpathmoveto{\pgfpoint{351.647980pt}{72.227356pt}}
\pgflineto{\pgfpoint{360.575958pt}{72.227356pt}}
\pgflineto{\pgfpoint{360.575958pt}{66.050522pt}}
\pgfpathclose
\pgfusepath{fill,stroke}
\pgfpathmoveto{\pgfpoint{351.647980pt}{78.404205pt}}
\pgflineto{\pgfpoint{360.575958pt}{72.227356pt}}
\pgflineto{\pgfpoint{351.647980pt}{72.227356pt}}
\pgfpathclose
\pgfusepath{fill,stroke}
\color[rgb]{0.127668,0.646882,0.523924}
\pgfpathmoveto{\pgfpoint{369.503998pt}{66.050522pt}}
\pgflineto{\pgfpoint{378.431976pt}{66.050522pt}}
\pgflineto{\pgfpoint{378.431976pt}{59.873672pt}}
\pgfpathclose
\pgfusepath{fill,stroke}
\color[rgb]{0.136835,0.661563,0.515967}
\pgfpathmoveto{\pgfpoint{369.503998pt}{72.227356pt}}
\pgflineto{\pgfpoint{378.431976pt}{66.050522pt}}
\pgflineto{\pgfpoint{369.503998pt}{66.050522pt}}
\pgfpathclose
\pgfusepath{fill,stroke}
\color[rgb]{0.122163,0.587476,0.546023}
\pgfpathmoveto{\pgfpoint{307.007965pt}{90.757896pt}}
\pgflineto{\pgfpoint{315.935974pt}{90.757896pt}}
\pgflineto{\pgfpoint{315.935974pt}{84.581039pt}}
\pgfpathclose
\pgfusepath{fill,stroke}
\pgfpathmoveto{\pgfpoint{307.007965pt}{96.934731pt}}
\pgflineto{\pgfpoint{315.935974pt}{90.757896pt}}
\pgflineto{\pgfpoint{307.007965pt}{90.757896pt}}
\pgfpathclose
\pgfusepath{fill,stroke}
\color[rgb]{0.119872,0.602382,0.541831}
\pgfpathmoveto{\pgfpoint{324.863983pt}{84.581039pt}}
\pgflineto{\pgfpoint{333.791992pt}{84.581039pt}}
\pgflineto{\pgfpoint{333.791992pt}{78.404205pt}}
\pgfpathclose
\pgfusepath{fill,stroke}
\color[rgb]{0.119627,0.617266,0.536796}
\pgfpathmoveto{\pgfpoint{324.863983pt}{90.757896pt}}
\pgflineto{\pgfpoint{333.791992pt}{84.581039pt}}
\pgflineto{\pgfpoint{324.863983pt}{84.581039pt}}
\pgfpathclose
\pgfusepath{fill,stroke}
\color[rgb]{0.119872,0.602382,0.541831}
\pgfpathmoveto{\pgfpoint{333.791992pt}{78.404205pt}}
\pgflineto{\pgfpoint{342.719971pt}{78.404205pt}}
\pgflineto{\pgfpoint{342.719971pt}{72.227356pt}}
\pgfpathclose
\pgfusepath{fill,stroke}
\color[rgb]{0.119627,0.617266,0.536796}
\pgfpathmoveto{\pgfpoint{333.791992pt}{84.581039pt}}
\pgflineto{\pgfpoint{342.719971pt}{78.404205pt}}
\pgflineto{\pgfpoint{333.791992pt}{78.404205pt}}
\pgfpathclose
\pgfusepath{fill,stroke}
\color[rgb]{0.135833,0.542750,0.554289}
\pgfpathmoveto{\pgfpoint{262.367981pt}{109.288422pt}}
\pgflineto{\pgfpoint{271.295990pt}{109.288422pt}}
\pgflineto{\pgfpoint{271.295990pt}{103.111580pt}}
\pgfpathclose
\pgfusepath{fill,stroke}
\color[rgb]{0.130582,0.557652,0.552176}
\pgfpathmoveto{\pgfpoint{262.367981pt}{115.465263pt}}
\pgflineto{\pgfpoint{271.295990pt}{109.288422pt}}
\pgflineto{\pgfpoint{262.367981pt}{109.288422pt}}
\pgfpathclose
\pgfusepath{fill,stroke}
\color[rgb]{0.135833,0.542750,0.554289}
\pgfpathmoveto{\pgfpoint{271.295990pt}{103.111580pt}}
\pgflineto{\pgfpoint{280.223969pt}{103.111580pt}}
\pgflineto{\pgfpoint{280.223969pt}{96.934731pt}}
\pgfpathclose
\pgfusepath{fill,stroke}
\color[rgb]{0.130582,0.557652,0.552176}
\pgfpathmoveto{\pgfpoint{271.295990pt}{109.288422pt}}
\pgflineto{\pgfpoint{280.223969pt}{103.111580pt}}
\pgflineto{\pgfpoint{271.295990pt}{103.111580pt}}
\pgfpathclose
\pgfusepath{fill,stroke}
\pgfpathmoveto{\pgfpoint{289.151978pt}{96.934731pt}}
\pgflineto{\pgfpoint{298.079987pt}{96.934731pt}}
\pgflineto{\pgfpoint{298.079987pt}{90.757896pt}}
\pgfpathclose
\pgfusepath{fill,stroke}
\color[rgb]{0.125898,0.572563,0.549445}
\pgfpathmoveto{\pgfpoint{289.151978pt}{103.111580pt}}
\pgflineto{\pgfpoint{298.079987pt}{96.934731pt}}
\pgflineto{\pgfpoint{289.151978pt}{96.934731pt}}
\pgfpathclose
\pgfusepath{fill,stroke}
\color[rgb]{0.152951,0.498053,0.557685}
\pgfpathmoveto{\pgfpoint{226.655975pt}{121.642097pt}}
\pgflineto{\pgfpoint{235.583969pt}{121.642097pt}}
\pgflineto{\pgfpoint{235.583969pt}{115.465263pt}}
\pgfpathclose
\pgfusepath{fill,stroke}
\color[rgb]{0.147132,0.512959,0.556973}
\pgfpathmoveto{\pgfpoint{226.655975pt}{127.818947pt}}
\pgflineto{\pgfpoint{235.583969pt}{121.642097pt}}
\pgflineto{\pgfpoint{226.655975pt}{121.642097pt}}
\pgfpathclose
\pgfusepath{fill,stroke}
\pgfpathmoveto{\pgfpoint{244.511993pt}{115.465263pt}}
\pgflineto{\pgfpoint{253.440002pt}{115.465263pt}}
\pgflineto{\pgfpoint{253.440002pt}{109.288422pt}}
\pgfpathclose
\pgfusepath{fill,stroke}
\color[rgb]{0.141402,0.527854,0.555864}
\pgfpathmoveto{\pgfpoint{244.511993pt}{121.642097pt}}
\pgflineto{\pgfpoint{253.440002pt}{115.465263pt}}
\pgflineto{\pgfpoint{244.511993pt}{115.465263pt}}
\pgfpathclose
\pgfusepath{fill,stroke}
\color[rgb]{0.177272,0.437886,0.557576}
\pgfpathmoveto{\pgfpoint{190.943985pt}{121.642097pt}}
\pgflineto{\pgfpoint{199.871979pt}{121.642097pt}}
\pgflineto{\pgfpoint{199.871979pt}{115.465263pt}}
\pgfpathclose
\pgfusepath{fill,stroke}
\color[rgb]{0.170958,0.453063,0.557974}
\pgfpathmoveto{\pgfpoint{190.943985pt}{127.818947pt}}
\pgflineto{\pgfpoint{199.871979pt}{121.642097pt}}
\pgflineto{\pgfpoint{190.943985pt}{121.642097pt}}
\pgfpathclose
\pgfusepath{fill,stroke}
\pgfpathmoveto{\pgfpoint{190.943985pt}{127.818947pt}}
\pgflineto{\pgfpoint{199.871979pt}{127.818947pt}}
\pgflineto{\pgfpoint{199.871979pt}{121.642097pt}}
\pgfpathclose
\pgfusepath{fill,stroke}
\pgfpathmoveto{\pgfpoint{190.943985pt}{133.995789pt}}
\pgflineto{\pgfpoint{199.871979pt}{127.818947pt}}
\pgflineto{\pgfpoint{190.943985pt}{127.818947pt}}
\pgfpathclose
\pgfusepath{fill,stroke}
\pgfpathmoveto{\pgfpoint{190.943985pt}{133.995789pt}}
\pgflineto{\pgfpoint{199.871979pt}{133.995789pt}}
\pgflineto{\pgfpoint{199.871979pt}{127.818947pt}}
\pgfpathclose
\pgfusepath{fill,stroke}
\color[rgb]{0.164833,0.468130,0.558143}
\pgfpathmoveto{\pgfpoint{190.943985pt}{140.172638pt}}
\pgflineto{\pgfpoint{199.871979pt}{133.995789pt}}
\pgflineto{\pgfpoint{190.943985pt}{133.995789pt}}
\pgfpathclose
\pgfusepath{fill,stroke}
\color[rgb]{0.177272,0.437886,0.557576}
\pgfpathmoveto{\pgfpoint{199.871979pt}{115.465263pt}}
\pgflineto{\pgfpoint{208.799988pt}{115.465263pt}}
\pgflineto{\pgfpoint{208.799988pt}{109.288422pt}}
\pgfpathclose
\pgfusepath{fill,stroke}
\color[rgb]{0.170958,0.453063,0.557974}
\pgfpathmoveto{\pgfpoint{199.871979pt}{121.642097pt}}
\pgflineto{\pgfpoint{208.799988pt}{115.465263pt}}
\pgflineto{\pgfpoint{199.871979pt}{115.465263pt}}
\pgfpathclose
\pgfusepath{fill,stroke}
\pgfpathmoveto{\pgfpoint{199.871979pt}{121.642097pt}}
\pgflineto{\pgfpoint{208.799988pt}{121.642097pt}}
\pgflineto{\pgfpoint{208.799988pt}{115.465263pt}}
\pgfpathclose
\pgfusepath{fill,stroke}
\color[rgb]{0.164833,0.468130,0.558143}
\pgfpathmoveto{\pgfpoint{199.871979pt}{127.818947pt}}
\pgflineto{\pgfpoint{208.799988pt}{121.642097pt}}
\pgflineto{\pgfpoint{199.871979pt}{121.642097pt}}
\pgfpathclose
\pgfusepath{fill,stroke}
\pgfpathmoveto{\pgfpoint{199.871979pt}{127.818947pt}}
\pgflineto{\pgfpoint{208.799988pt}{127.818947pt}}
\pgflineto{\pgfpoint{208.799988pt}{121.642097pt}}
\pgfpathclose
\pgfusepath{fill,stroke}
\color[rgb]{0.170958,0.453063,0.557974}
\pgfpathmoveto{\pgfpoint{208.799988pt}{115.465263pt}}
\pgflineto{\pgfpoint{217.727982pt}{109.288422pt}}
\pgflineto{\pgfpoint{208.799988pt}{109.288422pt}}
\pgfpathclose
\pgfusepath{fill,stroke}
\pgfpathmoveto{\pgfpoint{208.799988pt}{115.465263pt}}
\pgflineto{\pgfpoint{217.727982pt}{115.465263pt}}
\pgflineto{\pgfpoint{217.727982pt}{109.288422pt}}
\pgfpathclose
\pgfusepath{fill,stroke}
\color[rgb]{0.164833,0.468130,0.558143}
\pgfpathmoveto{\pgfpoint{208.799988pt}{121.642097pt}}
\pgflineto{\pgfpoint{217.727982pt}{115.465263pt}}
\pgflineto{\pgfpoint{208.799988pt}{115.465263pt}}
\pgfpathclose
\pgfusepath{fill,stroke}
\pgfpathmoveto{\pgfpoint{208.799988pt}{121.642097pt}}
\pgflineto{\pgfpoint{217.727982pt}{121.642097pt}}
\pgflineto{\pgfpoint{217.727982pt}{115.465263pt}}
\pgfpathclose
\pgfusepath{fill,stroke}
\color[rgb]{0.158845,0.483117,0.558059}
\pgfpathmoveto{\pgfpoint{208.799988pt}{127.818947pt}}
\pgflineto{\pgfpoint{217.727982pt}{121.642097pt}}
\pgflineto{\pgfpoint{208.799988pt}{121.642097pt}}
\pgfpathclose
\pgfusepath{fill,stroke}
\pgfpathmoveto{\pgfpoint{208.799988pt}{127.818947pt}}
\pgflineto{\pgfpoint{217.727982pt}{127.818947pt}}
\pgflineto{\pgfpoint{217.727982pt}{121.642097pt}}
\pgfpathclose
\pgfusepath{fill,stroke}
\color[rgb]{0.152951,0.498053,0.557685}
\pgfpathmoveto{\pgfpoint{208.799988pt}{133.995789pt}}
\pgflineto{\pgfpoint{217.727982pt}{127.818947pt}}
\pgflineto{\pgfpoint{208.799988pt}{127.818947pt}}
\pgfpathclose
\pgfusepath{fill,stroke}
\color[rgb]{0.170958,0.453063,0.557974}
\pgfpathmoveto{\pgfpoint{217.727982pt}{109.288422pt}}
\pgflineto{\pgfpoint{226.655975pt}{109.288422pt}}
\pgflineto{\pgfpoint{226.655975pt}{103.111580pt}}
\pgfpathclose
\pgfusepath{fill,stroke}
\color[rgb]{0.164833,0.468130,0.558143}
\pgfpathmoveto{\pgfpoint{217.727982pt}{115.465263pt}}
\pgflineto{\pgfpoint{226.655975pt}{109.288422pt}}
\pgflineto{\pgfpoint{217.727982pt}{109.288422pt}}
\pgfpathclose
\pgfusepath{fill,stroke}
\color[rgb]{0.183819,0.422564,0.556952}
\pgfpathmoveto{\pgfpoint{146.303986pt}{158.703156pt}}
\pgflineto{\pgfpoint{155.231979pt}{158.703156pt}}
\pgflineto{\pgfpoint{155.231979pt}{152.526306pt}}
\pgfpathclose
\pgfusepath{fill,stroke}
\pgfpathmoveto{\pgfpoint{155.231979pt}{146.349472pt}}
\pgflineto{\pgfpoint{164.160004pt}{146.349472pt}}
\pgflineto{\pgfpoint{164.160004pt}{140.172638pt}}
\pgfpathclose
\pgfusepath{fill,stroke}
\color[rgb]{0.177272,0.437886,0.557576}
\pgfpathmoveto{\pgfpoint{155.231979pt}{152.526306pt}}
\pgflineto{\pgfpoint{164.160004pt}{146.349472pt}}
\pgflineto{\pgfpoint{155.231979pt}{146.349472pt}}
\pgfpathclose
\pgfusepath{fill,stroke}
\pgfpathmoveto{\pgfpoint{155.231979pt}{152.526306pt}}
\pgflineto{\pgfpoint{164.160004pt}{152.526306pt}}
\pgflineto{\pgfpoint{164.160004pt}{146.349472pt}}
\pgfpathclose
\pgfusepath{fill,stroke}
\color[rgb]{0.170958,0.453063,0.557974}
\pgfpathmoveto{\pgfpoint{155.231979pt}{158.703156pt}}
\pgflineto{\pgfpoint{164.160004pt}{152.526306pt}}
\pgflineto{\pgfpoint{155.231979pt}{152.526306pt}}
\pgfpathclose
\pgfusepath{fill,stroke}
\pgfpathmoveto{\pgfpoint{155.231979pt}{158.703156pt}}
\pgflineto{\pgfpoint{164.160004pt}{158.703156pt}}
\pgflineto{\pgfpoint{164.160004pt}{152.526306pt}}
\pgfpathclose
\pgfusepath{fill,stroke}
\pgfpathmoveto{\pgfpoint{155.231979pt}{164.880005pt}}
\pgflineto{\pgfpoint{164.160004pt}{158.703156pt}}
\pgflineto{\pgfpoint{155.231979pt}{158.703156pt}}
\pgfpathclose
\pgfusepath{fill,stroke}
\color[rgb]{0.177272,0.437886,0.557576}
\pgfpathmoveto{\pgfpoint{164.160004pt}{146.349472pt}}
\pgflineto{\pgfpoint{173.087997pt}{140.172638pt}}
\pgflineto{\pgfpoint{164.160004pt}{140.172638pt}}
\pgfpathclose
\pgfusepath{fill,stroke}
\color[rgb]{0.170958,0.453063,0.557974}
\pgfpathmoveto{\pgfpoint{164.160004pt}{152.526306pt}}
\pgflineto{\pgfpoint{173.087997pt}{152.526306pt}}
\pgflineto{\pgfpoint{173.087997pt}{146.349472pt}}
\pgfpathclose
\pgfusepath{fill,stroke}
\color[rgb]{0.164833,0.468130,0.558143}
\pgfpathmoveto{\pgfpoint{164.160004pt}{158.703156pt}}
\pgflineto{\pgfpoint{173.087997pt}{152.526306pt}}
\pgflineto{\pgfpoint{164.160004pt}{152.526306pt}}
\pgfpathclose
\pgfusepath{fill,stroke}
\pgfpathmoveto{\pgfpoint{164.160004pt}{158.703156pt}}
\pgflineto{\pgfpoint{173.087997pt}{158.703156pt}}
\pgflineto{\pgfpoint{173.087997pt}{152.526306pt}}
\pgfpathclose
\pgfusepath{fill,stroke}
\color[rgb]{0.177272,0.437886,0.557576}
\pgfpathmoveto{\pgfpoint{173.087997pt}{140.172638pt}}
\pgflineto{\pgfpoint{182.015991pt}{140.172638pt}}
\pgflineto{\pgfpoint{182.015991pt}{133.995789pt}}
\pgfpathclose
\pgfusepath{fill,stroke}
\color[rgb]{0.170958,0.453063,0.557974}
\pgfpathmoveto{\pgfpoint{173.087997pt}{146.349472pt}}
\pgflineto{\pgfpoint{182.015991pt}{140.172638pt}}
\pgflineto{\pgfpoint{173.087997pt}{140.172638pt}}
\pgfpathclose
\pgfusepath{fill,stroke}
\pgfpathmoveto{\pgfpoint{173.087997pt}{146.349472pt}}
\pgflineto{\pgfpoint{182.015991pt}{146.349472pt}}
\pgflineto{\pgfpoint{182.015991pt}{140.172638pt}}
\pgfpathclose
\pgfusepath{fill,stroke}
\color[rgb]{0.164833,0.468130,0.558143}
\pgfpathmoveto{\pgfpoint{173.087997pt}{152.526306pt}}
\pgflineto{\pgfpoint{182.015991pt}{146.349472pt}}
\pgflineto{\pgfpoint{173.087997pt}{146.349472pt}}
\pgfpathclose
\pgfusepath{fill,stroke}
\pgfpathmoveto{\pgfpoint{173.087997pt}{152.526306pt}}
\pgflineto{\pgfpoint{182.015991pt}{152.526306pt}}
\pgflineto{\pgfpoint{182.015991pt}{146.349472pt}}
\pgfpathclose
\pgfusepath{fill,stroke}
\color[rgb]{0.158845,0.483117,0.558059}
\pgfpathmoveto{\pgfpoint{173.087997pt}{158.703156pt}}
\pgflineto{\pgfpoint{182.015991pt}{152.526306pt}}
\pgflineto{\pgfpoint{173.087997pt}{152.526306pt}}
\pgfpathclose
\pgfusepath{fill,stroke}
\color[rgb]{0.170958,0.453063,0.557974}
\pgfpathmoveto{\pgfpoint{182.015991pt}{140.172638pt}}
\pgflineto{\pgfpoint{190.943985pt}{133.995789pt}}
\pgflineto{\pgfpoint{182.015991pt}{133.995789pt}}
\pgfpathclose
\pgfusepath{fill,stroke}
\color[rgb]{0.164833,0.468130,0.558143}
\pgfpathmoveto{\pgfpoint{182.015991pt}{146.349472pt}}
\pgflineto{\pgfpoint{190.943985pt}{146.349472pt}}
\pgflineto{\pgfpoint{190.943985pt}{140.172638pt}}
\pgfpathclose
\pgfusepath{fill,stroke}
\color[rgb]{0.158845,0.483117,0.558059}
\pgfpathmoveto{\pgfpoint{182.015991pt}{152.526306pt}}
\pgflineto{\pgfpoint{190.943985pt}{146.349472pt}}
\pgflineto{\pgfpoint{182.015991pt}{146.349472pt}}
\pgfpathclose
\pgfusepath{fill,stroke}
\pgfpathmoveto{\pgfpoint{182.015991pt}{152.526306pt}}
\pgflineto{\pgfpoint{190.943985pt}{152.526306pt}}
\pgflineto{\pgfpoint{190.943985pt}{146.349472pt}}
\pgfpathclose
\pgfusepath{fill,stroke}
\color[rgb]{0.164833,0.468130,0.558143}
\pgfpathmoveto{\pgfpoint{190.943985pt}{140.172638pt}}
\pgflineto{\pgfpoint{199.871979pt}{140.172638pt}}
\pgflineto{\pgfpoint{199.871979pt}{133.995789pt}}
\pgfpathclose
\pgfusepath{fill,stroke}
\color[rgb]{0.158845,0.483117,0.558059}
\pgfpathmoveto{\pgfpoint{190.943985pt}{146.349472pt}}
\pgflineto{\pgfpoint{199.871979pt}{140.172638pt}}
\pgflineto{\pgfpoint{190.943985pt}{140.172638pt}}
\pgfpathclose
\pgfusepath{fill,stroke}
\pgfpathmoveto{\pgfpoint{190.943985pt}{146.349472pt}}
\pgflineto{\pgfpoint{199.871979pt}{146.349472pt}}
\pgflineto{\pgfpoint{199.871979pt}{140.172638pt}}
\pgfpathclose
\pgfusepath{fill,stroke}
\color[rgb]{0.152951,0.498053,0.557685}
\pgfpathmoveto{\pgfpoint{190.943985pt}{152.526306pt}}
\pgflineto{\pgfpoint{199.871979pt}{146.349472pt}}
\pgflineto{\pgfpoint{190.943985pt}{146.349472pt}}
\pgfpathclose
\pgfusepath{fill,stroke}
\color[rgb]{0.158845,0.483117,0.558059}
\pgfpathmoveto{\pgfpoint{199.871979pt}{133.995789pt}}
\pgflineto{\pgfpoint{208.799988pt}{127.818947pt}}
\pgflineto{\pgfpoint{199.871979pt}{127.818947pt}}
\pgfpathclose
\pgfusepath{fill,stroke}
\pgfpathmoveto{\pgfpoint{199.871979pt}{133.995789pt}}
\pgflineto{\pgfpoint{208.799988pt}{133.995789pt}}
\pgflineto{\pgfpoint{208.799988pt}{127.818947pt}}
\pgfpathclose
\pgfusepath{fill,stroke}
\pgfpathmoveto{\pgfpoint{199.871979pt}{140.172638pt}}
\pgflineto{\pgfpoint{208.799988pt}{133.995789pt}}
\pgflineto{\pgfpoint{199.871979pt}{133.995789pt}}
\pgfpathclose
\pgfusepath{fill,stroke}
\pgfpathmoveto{\pgfpoint{199.871979pt}{140.172638pt}}
\pgflineto{\pgfpoint{208.799988pt}{140.172638pt}}
\pgflineto{\pgfpoint{208.799988pt}{133.995789pt}}
\pgfpathclose
\pgfusepath{fill,stroke}
\color[rgb]{0.152951,0.498053,0.557685}
\pgfpathmoveto{\pgfpoint{199.871979pt}{146.349472pt}}
\pgflineto{\pgfpoint{208.799988pt}{140.172638pt}}
\pgflineto{\pgfpoint{199.871979pt}{140.172638pt}}
\pgfpathclose
\pgfusepath{fill,stroke}
\pgfpathmoveto{\pgfpoint{208.799988pt}{133.995789pt}}
\pgflineto{\pgfpoint{217.727982pt}{133.995789pt}}
\pgflineto{\pgfpoint{217.727982pt}{127.818947pt}}
\pgfpathclose
\pgfusepath{fill,stroke}
\pgfpathmoveto{\pgfpoint{208.799988pt}{140.172638pt}}
\pgflineto{\pgfpoint{217.727982pt}{133.995789pt}}
\pgflineto{\pgfpoint{208.799988pt}{133.995789pt}}
\pgfpathclose
\pgfusepath{fill,stroke}
\pgfpathmoveto{\pgfpoint{208.799988pt}{140.172638pt}}
\pgflineto{\pgfpoint{217.727982pt}{140.172638pt}}
\pgflineto{\pgfpoint{217.727982pt}{133.995789pt}}
\pgfpathclose
\pgfusepath{fill,stroke}
\pgfpathmoveto{\pgfpoint{217.727982pt}{127.818947pt}}
\pgflineto{\pgfpoint{226.655975pt}{121.642097pt}}
\pgflineto{\pgfpoint{217.727982pt}{121.642097pt}}
\pgfpathclose
\pgfusepath{fill,stroke}
\pgfpathmoveto{\pgfpoint{217.727982pt}{127.818947pt}}
\pgflineto{\pgfpoint{226.655975pt}{127.818947pt}}
\pgflineto{\pgfpoint{226.655975pt}{121.642097pt}}
\pgfpathclose
\pgfusepath{fill,stroke}
\color[rgb]{0.147132,0.512959,0.556973}
\pgfpathmoveto{\pgfpoint{217.727982pt}{133.995789pt}}
\pgflineto{\pgfpoint{226.655975pt}{127.818947pt}}
\pgflineto{\pgfpoint{217.727982pt}{127.818947pt}}
\pgfpathclose
\pgfusepath{fill,stroke}
\pgfpathmoveto{\pgfpoint{217.727982pt}{133.995789pt}}
\pgflineto{\pgfpoint{226.655975pt}{133.995789pt}}
\pgflineto{\pgfpoint{226.655975pt}{127.818947pt}}
\pgfpathclose
\pgfusepath{fill,stroke}
\color[rgb]{0.141402,0.527854,0.555864}
\pgfpathmoveto{\pgfpoint{217.727982pt}{140.172638pt}}
\pgflineto{\pgfpoint{226.655975pt}{133.995789pt}}
\pgflineto{\pgfpoint{217.727982pt}{133.995789pt}}
\pgfpathclose
\pgfusepath{fill,stroke}
\color[rgb]{0.147132,0.512959,0.556973}
\pgfpathmoveto{\pgfpoint{226.655975pt}{127.818947pt}}
\pgflineto{\pgfpoint{235.583969pt}{127.818947pt}}
\pgflineto{\pgfpoint{235.583969pt}{121.642097pt}}
\pgfpathclose
\pgfusepath{fill,stroke}
\color[rgb]{0.141402,0.527854,0.555864}
\pgfpathmoveto{\pgfpoint{226.655975pt}{133.995789pt}}
\pgflineto{\pgfpoint{235.583969pt}{127.818947pt}}
\pgflineto{\pgfpoint{226.655975pt}{127.818947pt}}
\pgfpathclose
\pgfusepath{fill,stroke}
\pgfpathmoveto{\pgfpoint{226.655975pt}{133.995789pt}}
\pgflineto{\pgfpoint{235.583969pt}{133.995789pt}}
\pgflineto{\pgfpoint{235.583969pt}{127.818947pt}}
\pgfpathclose
\pgfusepath{fill,stroke}
\color[rgb]{0.152951,0.498053,0.557685}
\pgfpathmoveto{\pgfpoint{235.583969pt}{115.465263pt}}
\pgflineto{\pgfpoint{244.511993pt}{115.465263pt}}
\pgflineto{\pgfpoint{244.511993pt}{109.288422pt}}
\pgfpathclose
\pgfusepath{fill,stroke}
\color[rgb]{0.147132,0.512959,0.556973}
\pgfpathmoveto{\pgfpoint{235.583969pt}{121.642097pt}}
\pgflineto{\pgfpoint{244.511993pt}{115.465263pt}}
\pgflineto{\pgfpoint{235.583969pt}{115.465263pt}}
\pgfpathclose
\pgfusepath{fill,stroke}
\pgfpathmoveto{\pgfpoint{235.583969pt}{121.642097pt}}
\pgflineto{\pgfpoint{244.511993pt}{121.642097pt}}
\pgflineto{\pgfpoint{244.511993pt}{115.465263pt}}
\pgfpathclose
\pgfusepath{fill,stroke}
\color[rgb]{0.141402,0.527854,0.555864}
\pgfpathmoveto{\pgfpoint{235.583969pt}{127.818947pt}}
\pgflineto{\pgfpoint{244.511993pt}{121.642097pt}}
\pgflineto{\pgfpoint{235.583969pt}{121.642097pt}}
\pgfpathclose
\pgfusepath{fill,stroke}
\pgfpathmoveto{\pgfpoint{235.583969pt}{127.818947pt}}
\pgflineto{\pgfpoint{244.511993pt}{127.818947pt}}
\pgflineto{\pgfpoint{244.511993pt}{121.642097pt}}
\pgfpathclose
\pgfusepath{fill,stroke}
\color[rgb]{0.135833,0.542750,0.554289}
\pgfpathmoveto{\pgfpoint{235.583969pt}{133.995789pt}}
\pgflineto{\pgfpoint{244.511993pt}{127.818947pt}}
\pgflineto{\pgfpoint{235.583969pt}{127.818947pt}}
\pgfpathclose
\pgfusepath{fill,stroke}
\color[rgb]{0.147132,0.512959,0.556973}
\pgfpathmoveto{\pgfpoint{244.511993pt}{115.465263pt}}
\pgflineto{\pgfpoint{253.440002pt}{109.288422pt}}
\pgflineto{\pgfpoint{244.511993pt}{109.288422pt}}
\pgfpathclose
\pgfusepath{fill,stroke}
\color[rgb]{0.141402,0.527854,0.555864}
\pgfpathmoveto{\pgfpoint{244.511993pt}{121.642097pt}}
\pgflineto{\pgfpoint{253.440002pt}{121.642097pt}}
\pgflineto{\pgfpoint{253.440002pt}{115.465263pt}}
\pgfpathclose
\pgfusepath{fill,stroke}
\color[rgb]{0.135833,0.542750,0.554289}
\pgfpathmoveto{\pgfpoint{244.511993pt}{127.818947pt}}
\pgflineto{\pgfpoint{253.440002pt}{121.642097pt}}
\pgflineto{\pgfpoint{244.511993pt}{121.642097pt}}
\pgfpathclose
\pgfusepath{fill,stroke}
\pgfpathmoveto{\pgfpoint{244.511993pt}{127.818947pt}}
\pgflineto{\pgfpoint{253.440002pt}{127.818947pt}}
\pgflineto{\pgfpoint{253.440002pt}{121.642097pt}}
\pgfpathclose
\pgfusepath{fill,stroke}
\color[rgb]{0.141402,0.527854,0.555864}
\pgfpathmoveto{\pgfpoint{253.440002pt}{109.288422pt}}
\pgflineto{\pgfpoint{262.367981pt}{109.288422pt}}
\pgflineto{\pgfpoint{262.367981pt}{103.111580pt}}
\pgfpathclose
\pgfusepath{fill,stroke}
\color[rgb]{0.135833,0.542750,0.554289}
\pgfpathmoveto{\pgfpoint{253.440002pt}{115.465263pt}}
\pgflineto{\pgfpoint{262.367981pt}{109.288422pt}}
\pgflineto{\pgfpoint{253.440002pt}{109.288422pt}}
\pgfpathclose
\pgfusepath{fill,stroke}
\pgfpathmoveto{\pgfpoint{253.440002pt}{115.465263pt}}
\pgflineto{\pgfpoint{262.367981pt}{115.465263pt}}
\pgflineto{\pgfpoint{262.367981pt}{109.288422pt}}
\pgfpathclose
\pgfusepath{fill,stroke}
\pgfpathmoveto{\pgfpoint{253.440002pt}{121.642097pt}}
\pgflineto{\pgfpoint{262.367981pt}{115.465263pt}}
\pgflineto{\pgfpoint{253.440002pt}{115.465263pt}}
\pgfpathclose
\pgfusepath{fill,stroke}
\pgfpathmoveto{\pgfpoint{253.440002pt}{121.642097pt}}
\pgflineto{\pgfpoint{262.367981pt}{121.642097pt}}
\pgflineto{\pgfpoint{262.367981pt}{115.465263pt}}
\pgfpathclose
\pgfusepath{fill,stroke}
\color[rgb]{0.130582,0.557652,0.552176}
\pgfpathmoveto{\pgfpoint{253.440002pt}{127.818947pt}}
\pgflineto{\pgfpoint{262.367981pt}{121.642097pt}}
\pgflineto{\pgfpoint{253.440002pt}{121.642097pt}}
\pgfpathclose
\pgfusepath{fill,stroke}
\color[rgb]{0.135833,0.542750,0.554289}
\pgfpathmoveto{\pgfpoint{262.367981pt}{109.288422pt}}
\pgflineto{\pgfpoint{271.295990pt}{103.111580pt}}
\pgflineto{\pgfpoint{262.367981pt}{103.111580pt}}
\pgfpathclose
\pgfusepath{fill,stroke}
\color[rgb]{0.130582,0.557652,0.552176}
\pgfpathmoveto{\pgfpoint{262.367981pt}{115.465263pt}}
\pgflineto{\pgfpoint{271.295990pt}{115.465263pt}}
\pgflineto{\pgfpoint{271.295990pt}{109.288422pt}}
\pgfpathclose
\pgfusepath{fill,stroke}
\pgfpathmoveto{\pgfpoint{262.367981pt}{121.642097pt}}
\pgflineto{\pgfpoint{271.295990pt}{115.465263pt}}
\pgflineto{\pgfpoint{262.367981pt}{115.465263pt}}
\pgfpathclose
\pgfusepath{fill,stroke}
\pgfpathmoveto{\pgfpoint{262.367981pt}{121.642097pt}}
\pgflineto{\pgfpoint{271.295990pt}{121.642097pt}}
\pgflineto{\pgfpoint{271.295990pt}{115.465263pt}}
\pgfpathclose
\pgfusepath{fill,stroke}
\pgfpathmoveto{\pgfpoint{271.295990pt}{109.288422pt}}
\pgflineto{\pgfpoint{280.223969pt}{109.288422pt}}
\pgflineto{\pgfpoint{280.223969pt}{103.111580pt}}
\pgfpathclose
\pgfusepath{fill,stroke}
\color[rgb]{0.125898,0.572563,0.549445}
\pgfpathmoveto{\pgfpoint{271.295990pt}{115.465263pt}}
\pgflineto{\pgfpoint{280.223969pt}{109.288422pt}}
\pgflineto{\pgfpoint{271.295990pt}{109.288422pt}}
\pgfpathclose
\pgfusepath{fill,stroke}
\pgfpathmoveto{\pgfpoint{271.295990pt}{115.465263pt}}
\pgflineto{\pgfpoint{280.223969pt}{115.465263pt}}
\pgflineto{\pgfpoint{280.223969pt}{109.288422pt}}
\pgfpathclose
\pgfusepath{fill,stroke}
\color[rgb]{0.122163,0.587476,0.546023}
\pgfpathmoveto{\pgfpoint{271.295990pt}{121.642097pt}}
\pgflineto{\pgfpoint{280.223969pt}{115.465263pt}}
\pgflineto{\pgfpoint{271.295990pt}{115.465263pt}}
\pgfpathclose
\pgfusepath{fill,stroke}
\color[rgb]{0.130582,0.557652,0.552176}
\pgfpathmoveto{\pgfpoint{280.223969pt}{103.111580pt}}
\pgflineto{\pgfpoint{289.151978pt}{96.934731pt}}
\pgflineto{\pgfpoint{280.223969pt}{96.934731pt}}
\pgfpathclose
\pgfusepath{fill,stroke}
\pgfpathmoveto{\pgfpoint{280.223969pt}{103.111580pt}}
\pgflineto{\pgfpoint{289.151978pt}{103.111580pt}}
\pgflineto{\pgfpoint{289.151978pt}{96.934731pt}}
\pgfpathclose
\pgfusepath{fill,stroke}
\color[rgb]{0.125898,0.572563,0.549445}
\pgfpathmoveto{\pgfpoint{280.223969pt}{109.288422pt}}
\pgflineto{\pgfpoint{289.151978pt}{103.111580pt}}
\pgflineto{\pgfpoint{280.223969pt}{103.111580pt}}
\pgfpathclose
\pgfusepath{fill,stroke}
\pgfpathmoveto{\pgfpoint{280.223969pt}{109.288422pt}}
\pgflineto{\pgfpoint{289.151978pt}{109.288422pt}}
\pgflineto{\pgfpoint{289.151978pt}{103.111580pt}}
\pgfpathclose
\pgfusepath{fill,stroke}
\color[rgb]{0.122163,0.587476,0.546023}
\pgfpathmoveto{\pgfpoint{280.223969pt}{115.465263pt}}
\pgflineto{\pgfpoint{289.151978pt}{109.288422pt}}
\pgflineto{\pgfpoint{280.223969pt}{109.288422pt}}
\pgfpathclose
\pgfusepath{fill,stroke}
\color[rgb]{0.125898,0.572563,0.549445}
\pgfpathmoveto{\pgfpoint{289.151978pt}{103.111580pt}}
\pgflineto{\pgfpoint{298.079987pt}{103.111580pt}}
\pgflineto{\pgfpoint{298.079987pt}{96.934731pt}}
\pgfpathclose
\pgfusepath{fill,stroke}
\color[rgb]{0.122163,0.587476,0.546023}
\pgfpathmoveto{\pgfpoint{289.151978pt}{109.288422pt}}
\pgflineto{\pgfpoint{298.079987pt}{103.111580pt}}
\pgflineto{\pgfpoint{289.151978pt}{103.111580pt}}
\pgfpathclose
\pgfusepath{fill,stroke}
\pgfpathmoveto{\pgfpoint{289.151978pt}{109.288422pt}}
\pgflineto{\pgfpoint{298.079987pt}{109.288422pt}}
\pgflineto{\pgfpoint{298.079987pt}{103.111580pt}}
\pgfpathclose
\pgfusepath{fill,stroke}
\color[rgb]{0.125898,0.572563,0.549445}
\pgfpathmoveto{\pgfpoint{298.079987pt}{90.757896pt}}
\pgflineto{\pgfpoint{307.007965pt}{90.757896pt}}
\pgflineto{\pgfpoint{307.007965pt}{84.581039pt}}
\pgfpathclose
\pgfusepath{fill,stroke}
\pgfpathmoveto{\pgfpoint{298.079987pt}{96.934731pt}}
\pgflineto{\pgfpoint{307.007965pt}{90.757896pt}}
\pgflineto{\pgfpoint{298.079987pt}{90.757896pt}}
\pgfpathclose
\pgfusepath{fill,stroke}
\pgfpathmoveto{\pgfpoint{298.079987pt}{96.934731pt}}
\pgflineto{\pgfpoint{307.007965pt}{96.934731pt}}
\pgflineto{\pgfpoint{307.007965pt}{90.757896pt}}
\pgfpathclose
\pgfusepath{fill,stroke}
\color[rgb]{0.122163,0.587476,0.546023}
\pgfpathmoveto{\pgfpoint{298.079987pt}{103.111580pt}}
\pgflineto{\pgfpoint{307.007965pt}{96.934731pt}}
\pgflineto{\pgfpoint{298.079987pt}{96.934731pt}}
\pgfpathclose
\pgfusepath{fill,stroke}
\pgfpathmoveto{\pgfpoint{298.079987pt}{103.111580pt}}
\pgflineto{\pgfpoint{307.007965pt}{103.111580pt}}
\pgflineto{\pgfpoint{307.007965pt}{96.934731pt}}
\pgfpathclose
\pgfusepath{fill,stroke}
\color[rgb]{0.119872,0.602382,0.541831}
\pgfpathmoveto{\pgfpoint{298.079987pt}{109.288422pt}}
\pgflineto{\pgfpoint{307.007965pt}{103.111580pt}}
\pgflineto{\pgfpoint{298.079987pt}{103.111580pt}}
\pgfpathclose
\pgfusepath{fill,stroke}
\color[rgb]{0.122163,0.587476,0.546023}
\pgfpathmoveto{\pgfpoint{307.007965pt}{90.757896pt}}
\pgflineto{\pgfpoint{315.935974pt}{84.581039pt}}
\pgflineto{\pgfpoint{307.007965pt}{84.581039pt}}
\pgfpathclose
\pgfusepath{fill,stroke}
\pgfpathmoveto{\pgfpoint{307.007965pt}{96.934731pt}}
\pgflineto{\pgfpoint{315.935974pt}{96.934731pt}}
\pgflineto{\pgfpoint{315.935974pt}{90.757896pt}}
\pgfpathclose
\pgfusepath{fill,stroke}
\color[rgb]{0.119872,0.602382,0.541831}
\pgfpathmoveto{\pgfpoint{307.007965pt}{103.111580pt}}
\pgflineto{\pgfpoint{315.935974pt}{96.934731pt}}
\pgflineto{\pgfpoint{307.007965pt}{96.934731pt}}
\pgfpathclose
\pgfusepath{fill,stroke}
\pgfpathmoveto{\pgfpoint{307.007965pt}{103.111580pt}}
\pgflineto{\pgfpoint{315.935974pt}{103.111580pt}}
\pgflineto{\pgfpoint{315.935974pt}{96.934731pt}}
\pgfpathclose
\pgfusepath{fill,stroke}
\color[rgb]{0.122163,0.587476,0.546023}
\pgfpathmoveto{\pgfpoint{315.935974pt}{84.581039pt}}
\pgflineto{\pgfpoint{324.863983pt}{84.581039pt}}
\pgflineto{\pgfpoint{324.863983pt}{78.404205pt}}
\pgfpathclose
\pgfusepath{fill,stroke}
\color[rgb]{0.119872,0.602382,0.541831}
\pgfpathmoveto{\pgfpoint{315.935974pt}{90.757896pt}}
\pgflineto{\pgfpoint{324.863983pt}{84.581039pt}}
\pgflineto{\pgfpoint{315.935974pt}{84.581039pt}}
\pgfpathclose
\pgfusepath{fill,stroke}
\pgfpathmoveto{\pgfpoint{315.935974pt}{90.757896pt}}
\pgflineto{\pgfpoint{324.863983pt}{90.757896pt}}
\pgflineto{\pgfpoint{324.863983pt}{84.581039pt}}
\pgfpathclose
\pgfusepath{fill,stroke}
\color[rgb]{0.119627,0.617266,0.536796}
\pgfpathmoveto{\pgfpoint{315.935974pt}{96.934731pt}}
\pgflineto{\pgfpoint{324.863983pt}{90.757896pt}}
\pgflineto{\pgfpoint{315.935974pt}{90.757896pt}}
\pgfpathclose
\pgfusepath{fill,stroke}
\pgfpathmoveto{\pgfpoint{315.935974pt}{96.934731pt}}
\pgflineto{\pgfpoint{324.863983pt}{96.934731pt}}
\pgflineto{\pgfpoint{324.863983pt}{90.757896pt}}
\pgfpathclose
\pgfusepath{fill,stroke}
\pgfpathmoveto{\pgfpoint{315.935974pt}{103.111580pt}}
\pgflineto{\pgfpoint{324.863983pt}{96.934731pt}}
\pgflineto{\pgfpoint{315.935974pt}{96.934731pt}}
\pgfpathclose
\pgfusepath{fill,stroke}
\color[rgb]{0.119872,0.602382,0.541831}
\pgfpathmoveto{\pgfpoint{324.863983pt}{84.581039pt}}
\pgflineto{\pgfpoint{333.791992pt}{78.404205pt}}
\pgflineto{\pgfpoint{324.863983pt}{78.404205pt}}
\pgfpathclose
\pgfusepath{fill,stroke}
\color[rgb]{0.119627,0.617266,0.536796}
\pgfpathmoveto{\pgfpoint{324.863983pt}{90.757896pt}}
\pgflineto{\pgfpoint{333.791992pt}{90.757896pt}}
\pgflineto{\pgfpoint{333.791992pt}{84.581039pt}}
\pgfpathclose
\pgfusepath{fill,stroke}
\color[rgb]{0.122046,0.632107,0.530848}
\pgfpathmoveto{\pgfpoint{324.863983pt}{96.934731pt}}
\pgflineto{\pgfpoint{333.791992pt}{90.757896pt}}
\pgflineto{\pgfpoint{324.863983pt}{90.757896pt}}
\pgfpathclose
\pgfusepath{fill,stroke}
\pgfpathmoveto{\pgfpoint{324.863983pt}{96.934731pt}}
\pgflineto{\pgfpoint{333.791992pt}{96.934731pt}}
\pgflineto{\pgfpoint{333.791992pt}{90.757896pt}}
\pgfpathclose
\pgfusepath{fill,stroke}
\color[rgb]{0.119627,0.617266,0.536796}
\pgfpathmoveto{\pgfpoint{333.791992pt}{84.581039pt}}
\pgflineto{\pgfpoint{342.719971pt}{84.581039pt}}
\pgflineto{\pgfpoint{342.719971pt}{78.404205pt}}
\pgfpathclose
\pgfusepath{fill,stroke}
\color[rgb]{0.122046,0.632107,0.530848}
\pgfpathmoveto{\pgfpoint{333.791992pt}{90.757896pt}}
\pgflineto{\pgfpoint{342.719971pt}{84.581039pt}}
\pgflineto{\pgfpoint{333.791992pt}{84.581039pt}}
\pgfpathclose
\pgfusepath{fill,stroke}
\pgfpathmoveto{\pgfpoint{333.791992pt}{90.757896pt}}
\pgflineto{\pgfpoint{342.719971pt}{90.757896pt}}
\pgflineto{\pgfpoint{342.719971pt}{84.581039pt}}
\pgfpathclose
\pgfusepath{fill,stroke}
\color[rgb]{0.127668,0.646882,0.523924}
\pgfpathmoveto{\pgfpoint{333.791992pt}{96.934731pt}}
\pgflineto{\pgfpoint{342.719971pt}{90.757896pt}}
\pgflineto{\pgfpoint{333.791992pt}{90.757896pt}}
\pgfpathclose
\pgfusepath{fill,stroke}
\color[rgb]{0.119627,0.617266,0.536796}
\pgfpathmoveto{\pgfpoint{342.719971pt}{78.404205pt}}
\pgflineto{\pgfpoint{351.647980pt}{72.227356pt}}
\pgflineto{\pgfpoint{342.719971pt}{72.227356pt}}
\pgfpathclose
\pgfusepath{fill,stroke}
\pgfpathmoveto{\pgfpoint{342.719971pt}{78.404205pt}}
\pgflineto{\pgfpoint{351.647980pt}{78.404205pt}}
\pgflineto{\pgfpoint{351.647980pt}{72.227356pt}}
\pgfpathclose
\pgfusepath{fill,stroke}
\color[rgb]{0.122046,0.632107,0.530848}
\pgfpathmoveto{\pgfpoint{342.719971pt}{84.581039pt}}
\pgflineto{\pgfpoint{351.647980pt}{78.404205pt}}
\pgflineto{\pgfpoint{342.719971pt}{78.404205pt}}
\pgfpathclose
\pgfusepath{fill,stroke}
\pgfpathmoveto{\pgfpoint{342.719971pt}{84.581039pt}}
\pgflineto{\pgfpoint{351.647980pt}{84.581039pt}}
\pgflineto{\pgfpoint{351.647980pt}{78.404205pt}}
\pgfpathclose
\pgfusepath{fill,stroke}
\color[rgb]{0.127668,0.646882,0.523924}
\pgfpathmoveto{\pgfpoint{342.719971pt}{90.757896pt}}
\pgflineto{\pgfpoint{351.647980pt}{84.581039pt}}
\pgflineto{\pgfpoint{342.719971pt}{84.581039pt}}
\pgfpathclose
\pgfusepath{fill,stroke}
\color[rgb]{0.122046,0.632107,0.530848}
\pgfpathmoveto{\pgfpoint{351.647980pt}{78.404205pt}}
\pgflineto{\pgfpoint{360.575958pt}{78.404205pt}}
\pgflineto{\pgfpoint{360.575958pt}{72.227356pt}}
\pgfpathclose
\pgfusepath{fill,stroke}
\color[rgb]{0.127668,0.646882,0.523924}
\pgfpathmoveto{\pgfpoint{351.647980pt}{84.581039pt}}
\pgflineto{\pgfpoint{360.575958pt}{78.404205pt}}
\pgflineto{\pgfpoint{351.647980pt}{78.404205pt}}
\pgfpathclose
\pgfusepath{fill,stroke}
\pgfpathmoveto{\pgfpoint{351.647980pt}{84.581039pt}}
\pgflineto{\pgfpoint{360.575958pt}{84.581039pt}}
\pgflineto{\pgfpoint{360.575958pt}{78.404205pt}}
\pgfpathclose
\pgfusepath{fill,stroke}
\pgfpathmoveto{\pgfpoint{360.575958pt}{72.227356pt}}
\pgflineto{\pgfpoint{369.503998pt}{66.050522pt}}
\pgflineto{\pgfpoint{360.575958pt}{66.050522pt}}
\pgfpathclose
\pgfusepath{fill,stroke}
\pgfpathmoveto{\pgfpoint{360.575958pt}{72.227356pt}}
\pgflineto{\pgfpoint{369.503998pt}{72.227356pt}}
\pgflineto{\pgfpoint{369.503998pt}{66.050522pt}}
\pgfpathclose
\pgfusepath{fill,stroke}
\pgfpathmoveto{\pgfpoint{360.575958pt}{78.404205pt}}
\pgflineto{\pgfpoint{369.503998pt}{72.227356pt}}
\pgflineto{\pgfpoint{360.575958pt}{72.227356pt}}
\pgfpathclose
\pgfusepath{fill,stroke}
\pgfpathmoveto{\pgfpoint{360.575958pt}{78.404205pt}}
\pgflineto{\pgfpoint{369.503998pt}{78.404205pt}}
\pgflineto{\pgfpoint{369.503998pt}{72.227356pt}}
\pgfpathclose
\pgfusepath{fill,stroke}
\color[rgb]{0.136835,0.661563,0.515967}
\pgfpathmoveto{\pgfpoint{360.575958pt}{84.581039pt}}
\pgflineto{\pgfpoint{369.503998pt}{78.404205pt}}
\pgflineto{\pgfpoint{360.575958pt}{78.404205pt}}
\pgfpathclose
\pgfusepath{fill,stroke}
\pgfpathmoveto{\pgfpoint{369.503998pt}{72.227356pt}}
\pgflineto{\pgfpoint{378.431976pt}{72.227356pt}}
\pgflineto{\pgfpoint{378.431976pt}{66.050522pt}}
\pgfpathclose
\pgfusepath{fill,stroke}
\color[rgb]{0.149643,0.676120,0.506924}
\pgfpathmoveto{\pgfpoint{369.503998pt}{78.404205pt}}
\pgflineto{\pgfpoint{378.431976pt}{72.227356pt}}
\pgflineto{\pgfpoint{369.503998pt}{72.227356pt}}
\pgfpathclose
\pgfusepath{fill,stroke}
\pgfpathmoveto{\pgfpoint{369.503998pt}{78.404205pt}}
\pgflineto{\pgfpoint{378.431976pt}{78.404205pt}}
\pgflineto{\pgfpoint{378.431976pt}{72.227356pt}}
\pgfpathclose
\pgfusepath{fill,stroke}
\color[rgb]{0.127668,0.646882,0.523924}
\pgfpathmoveto{\pgfpoint{378.431976pt}{59.873672pt}}
\pgflineto{\pgfpoint{387.359985pt}{59.873672pt}}
\pgflineto{\pgfpoint{387.359985pt}{53.696838pt}}
\pgfpathclose
\pgfusepath{fill,stroke}
\color[rgb]{0.136835,0.661563,0.515967}
\pgfpathmoveto{\pgfpoint{378.431976pt}{66.050522pt}}
\pgflineto{\pgfpoint{387.359985pt}{59.873672pt}}
\pgflineto{\pgfpoint{378.431976pt}{59.873672pt}}
\pgfpathclose
\pgfusepath{fill,stroke}
\pgfpathmoveto{\pgfpoint{378.431976pt}{66.050522pt}}
\pgflineto{\pgfpoint{387.359985pt}{66.050522pt}}
\pgflineto{\pgfpoint{387.359985pt}{59.873672pt}}
\pgfpathclose
\pgfusepath{fill,stroke}
\color[rgb]{0.149643,0.676120,0.506924}
\pgfpathmoveto{\pgfpoint{378.431976pt}{72.227356pt}}
\pgflineto{\pgfpoint{387.359985pt}{66.050522pt}}
\pgflineto{\pgfpoint{378.431976pt}{66.050522pt}}
\pgfpathclose
\pgfusepath{fill,stroke}
\pgfpathmoveto{\pgfpoint{378.431976pt}{72.227356pt}}
\pgflineto{\pgfpoint{387.359985pt}{72.227356pt}}
\pgflineto{\pgfpoint{387.359985pt}{66.050522pt}}
\pgfpathclose
\pgfusepath{fill,stroke}
\color[rgb]{0.165967,0.690519,0.496752}
\pgfpathmoveto{\pgfpoint{378.431976pt}{78.404205pt}}
\pgflineto{\pgfpoint{387.359985pt}{72.227356pt}}
\pgflineto{\pgfpoint{378.431976pt}{72.227356pt}}
\pgfpathclose
\pgfusepath{fill,stroke}
\color[rgb]{0.136835,0.661563,0.515967}
\pgfpathmoveto{\pgfpoint{387.359985pt}{59.873672pt}}
\pgflineto{\pgfpoint{396.287964pt}{53.696838pt}}
\pgflineto{\pgfpoint{387.359985pt}{53.696838pt}}
\pgfpathclose
\pgfusepath{fill,stroke}
\color[rgb]{0.149643,0.676120,0.506924}
\pgfpathmoveto{\pgfpoint{387.359985pt}{66.050522pt}}
\pgflineto{\pgfpoint{396.287964pt}{66.050522pt}}
\pgflineto{\pgfpoint{396.287964pt}{59.873672pt}}
\pgfpathclose
\pgfusepath{fill,stroke}
\color[rgb]{0.165967,0.690519,0.496752}
\pgfpathmoveto{\pgfpoint{387.359985pt}{72.227356pt}}
\pgflineto{\pgfpoint{396.287964pt}{66.050522pt}}
\pgflineto{\pgfpoint{387.359985pt}{66.050522pt}}
\pgfpathclose
\pgfusepath{fill,stroke}
\pgfpathmoveto{\pgfpoint{387.359985pt}{72.227356pt}}
\pgflineto{\pgfpoint{396.287964pt}{72.227356pt}}
\pgflineto{\pgfpoint{396.287964pt}{66.050522pt}}
\pgfpathclose
\pgfusepath{fill,stroke}
\color[rgb]{0.136835,0.661563,0.515967}
\pgfpathmoveto{\pgfpoint{396.287964pt}{53.696838pt}}
\pgflineto{\pgfpoint{405.216003pt}{53.696838pt}}
\pgflineto{\pgfpoint{405.216003pt}{47.519989pt}}
\pgfpathclose
\pgfusepath{fill,stroke}
\color[rgb]{0.149643,0.676120,0.506924}
\pgfpathmoveto{\pgfpoint{396.287964pt}{59.873672pt}}
\pgflineto{\pgfpoint{405.216003pt}{53.696838pt}}
\pgflineto{\pgfpoint{396.287964pt}{53.696838pt}}
\pgfpathclose
\pgfusepath{fill,stroke}
\pgfpathmoveto{\pgfpoint{396.287964pt}{59.873672pt}}
\pgflineto{\pgfpoint{405.216003pt}{59.873672pt}}
\pgflineto{\pgfpoint{405.216003pt}{53.696838pt}}
\pgfpathclose
\pgfusepath{fill,stroke}
\color[rgb]{0.165967,0.690519,0.496752}
\pgfpathmoveto{\pgfpoint{396.287964pt}{66.050522pt}}
\pgflineto{\pgfpoint{405.216003pt}{59.873672pt}}
\pgflineto{\pgfpoint{396.287964pt}{59.873672pt}}
\pgfpathclose
\pgfusepath{fill,stroke}
\pgfpathmoveto{\pgfpoint{396.287964pt}{66.050522pt}}
\pgflineto{\pgfpoint{405.216003pt}{66.050522pt}}
\pgflineto{\pgfpoint{405.216003pt}{59.873672pt}}
\pgfpathclose
\pgfusepath{fill,stroke}
\color[rgb]{0.185538,0.704725,0.485412}
\pgfpathmoveto{\pgfpoint{396.287964pt}{72.227356pt}}
\pgflineto{\pgfpoint{405.216003pt}{66.050522pt}}
\pgflineto{\pgfpoint{396.287964pt}{66.050522pt}}
\pgfpathclose
\pgfusepath{fill,stroke}
\color[rgb]{0.149643,0.676120,0.506924}
\pgfpathmoveto{\pgfpoint{405.216003pt}{53.696838pt}}
\pgflineto{\pgfpoint{414.143982pt}{47.519989pt}}
\pgflineto{\pgfpoint{405.216003pt}{47.519989pt}}
\pgfpathclose
\pgfusepath{fill,stroke}
\color[rgb]{0.165967,0.690519,0.496752}
\pgfpathmoveto{\pgfpoint{405.216003pt}{59.873672pt}}
\pgflineto{\pgfpoint{414.143982pt}{59.873672pt}}
\pgflineto{\pgfpoint{414.143982pt}{53.696838pt}}
\pgfpathclose
\pgfusepath{fill,stroke}
\color[rgb]{0.185538,0.704725,0.485412}
\pgfpathmoveto{\pgfpoint{405.216003pt}{66.050522pt}}
\pgflineto{\pgfpoint{414.143982pt}{59.873672pt}}
\pgflineto{\pgfpoint{405.216003pt}{59.873672pt}}
\pgfpathclose
\pgfusepath{fill,stroke}
\pgfpathmoveto{\pgfpoint{405.216003pt}{66.050522pt}}
\pgflineto{\pgfpoint{414.143982pt}{66.050522pt}}
\pgflineto{\pgfpoint{414.143982pt}{59.873672pt}}
\pgfpathclose
\pgfusepath{fill,stroke}
\pgfpathmoveto{\pgfpoint{414.143982pt}{53.696838pt}}
\pgflineto{\pgfpoint{423.071960pt}{53.696838pt}}
\pgflineto{\pgfpoint{423.071960pt}{47.519989pt}}
\pgfpathclose
\pgfusepath{fill,stroke}
\pgfpathmoveto{\pgfpoint{414.143982pt}{59.873672pt}}
\pgflineto{\pgfpoint{423.071960pt}{53.696838pt}}
\pgflineto{\pgfpoint{414.143982pt}{53.696838pt}}
\pgfpathclose
\pgfusepath{fill,stroke}
\pgfpathmoveto{\pgfpoint{414.143982pt}{59.873672pt}}
\pgflineto{\pgfpoint{423.071960pt}{59.873672pt}}
\pgflineto{\pgfpoint{423.071960pt}{53.696838pt}}
\pgfpathclose
\pgfusepath{fill,stroke}
\color[rgb]{0.208030,0.718701,0.472873}
\pgfpathmoveto{\pgfpoint{414.143982pt}{66.050522pt}}
\pgflineto{\pgfpoint{423.071960pt}{59.873672pt}}
\pgflineto{\pgfpoint{414.143982pt}{59.873672pt}}
\pgfpathclose
\pgfusepath{fill,stroke}
\pgfpathmoveto{\pgfpoint{423.071960pt}{53.696838pt}}
\pgflineto{\pgfpoint{432.000000pt}{47.519989pt}}
\pgflineto{\pgfpoint{423.071960pt}{47.519989pt}}
\pgfpathclose
\pgfusepath{fill,stroke}
\pgfpathmoveto{\pgfpoint{423.071960pt}{53.696838pt}}
\pgflineto{\pgfpoint{432.000000pt}{53.696838pt}}
\pgflineto{\pgfpoint{432.000000pt}{47.519989pt}}
\pgfpathclose
\pgfusepath{fill,stroke}
\pgfpathmoveto{\pgfpoint{423.071960pt}{59.873672pt}}
\pgflineto{\pgfpoint{432.000000pt}{53.696838pt}}
\pgflineto{\pgfpoint{423.071960pt}{53.696838pt}}
\pgfpathclose
\pgfusepath{fill,stroke}
\color[rgb]{0.233127,0.732406,0.459106}
\pgfpathmoveto{\pgfpoint{432.000000pt}{53.696838pt}}
\pgflineto{\pgfpoint{440.927979pt}{47.519989pt}}
\pgflineto{\pgfpoint{432.000000pt}{47.519989pt}}
\pgfpathclose
\pgfusepath{fill,stroke}
\pgfpathmoveto{\pgfpoint{432.000000pt}{53.696838pt}}
\pgflineto{\pgfpoint{440.927979pt}{53.696838pt}}
\pgflineto{\pgfpoint{440.927979pt}{47.519989pt}}
\pgfpathclose
\pgfusepath{fill,stroke}
\color[rgb]{0.260531,0.745802,0.444096}
\pgfpathmoveto{\pgfpoint{440.927979pt}{53.696838pt}}
\pgflineto{\pgfpoint{449.855957pt}{47.519989pt}}
\pgflineto{\pgfpoint{440.927979pt}{47.519989pt}}
\pgfpathclose
\pgfusepath{fill,stroke}
\color[rgb]{0.152951,0.498053,0.557685}
\pgfpathmoveto{\pgfpoint{182.015991pt}{158.703156pt}}
\pgflineto{\pgfpoint{190.943985pt}{152.526306pt}}
\pgflineto{\pgfpoint{182.015991pt}{152.526306pt}}
\pgfpathclose
\pgfusepath{fill,stroke}
\pgfpathmoveto{\pgfpoint{182.015991pt}{158.703156pt}}
\pgflineto{\pgfpoint{190.943985pt}{158.703156pt}}
\pgflineto{\pgfpoint{190.943985pt}{152.526306pt}}
\pgfpathclose
\pgfusepath{fill,stroke}
\color[rgb]{0.147132,0.512959,0.556973}
\pgfpathmoveto{\pgfpoint{182.015991pt}{164.880005pt}}
\pgflineto{\pgfpoint{190.943985pt}{158.703156pt}}
\pgflineto{\pgfpoint{182.015991pt}{158.703156pt}}
\pgfpathclose
\pgfusepath{fill,stroke}
\pgfpathmoveto{\pgfpoint{182.015991pt}{164.880005pt}}
\pgflineto{\pgfpoint{190.943985pt}{164.880005pt}}
\pgflineto{\pgfpoint{190.943985pt}{158.703156pt}}
\pgfpathclose
\pgfusepath{fill,stroke}
\pgfpathmoveto{\pgfpoint{182.015991pt}{171.056854pt}}
\pgflineto{\pgfpoint{190.943985pt}{171.056854pt}}
\pgflineto{\pgfpoint{190.943985pt}{164.880005pt}}
\pgfpathclose
\pgfusepath{fill,stroke}
\color[rgb]{0.141402,0.527854,0.555864}
\pgfpathmoveto{\pgfpoint{182.015991pt}{177.233673pt}}
\pgflineto{\pgfpoint{190.943985pt}{171.056854pt}}
\pgflineto{\pgfpoint{182.015991pt}{171.056854pt}}
\pgfpathclose
\pgfusepath{fill,stroke}
\color[rgb]{0.152951,0.498053,0.557685}
\pgfpathmoveto{\pgfpoint{190.943985pt}{152.526306pt}}
\pgflineto{\pgfpoint{199.871979pt}{152.526306pt}}
\pgflineto{\pgfpoint{199.871979pt}{146.349472pt}}
\pgfpathclose
\pgfusepath{fill,stroke}
\color[rgb]{0.147132,0.512959,0.556973}
\pgfpathmoveto{\pgfpoint{190.943985pt}{158.703156pt}}
\pgflineto{\pgfpoint{199.871979pt}{152.526306pt}}
\pgflineto{\pgfpoint{190.943985pt}{152.526306pt}}
\pgfpathclose
\pgfusepath{fill,stroke}
\pgfpathmoveto{\pgfpoint{190.943985pt}{158.703156pt}}
\pgflineto{\pgfpoint{199.871979pt}{158.703156pt}}
\pgflineto{\pgfpoint{199.871979pt}{152.526306pt}}
\pgfpathclose
\pgfusepath{fill,stroke}
\color[rgb]{0.141402,0.527854,0.555864}
\pgfpathmoveto{\pgfpoint{190.943985pt}{164.880005pt}}
\pgflineto{\pgfpoint{199.871979pt}{158.703156pt}}
\pgflineto{\pgfpoint{190.943985pt}{158.703156pt}}
\pgfpathclose
\pgfusepath{fill,stroke}
\pgfpathmoveto{\pgfpoint{199.871979pt}{158.703156pt}}
\pgflineto{\pgfpoint{208.799988pt}{152.526306pt}}
\pgflineto{\pgfpoint{199.871979pt}{152.526306pt}}
\pgfpathclose
\pgfusepath{fill,stroke}
\pgfpathmoveto{\pgfpoint{199.871979pt}{158.703156pt}}
\pgflineto{\pgfpoint{208.799988pt}{158.703156pt}}
\pgflineto{\pgfpoint{208.799988pt}{152.526306pt}}
\pgfpathclose
\pgfusepath{fill,stroke}
\color[rgb]{0.135833,0.542750,0.554289}
\pgfpathmoveto{\pgfpoint{199.871979pt}{164.880005pt}}
\pgflineto{\pgfpoint{208.799988pt}{164.880005pt}}
\pgflineto{\pgfpoint{208.799988pt}{158.703156pt}}
\pgfpathclose
\pgfusepath{fill,stroke}
\color[rgb]{0.130582,0.557652,0.552176}
\pgfpathmoveto{\pgfpoint{199.871979pt}{171.056854pt}}
\pgflineto{\pgfpoint{208.799988pt}{164.880005pt}}
\pgflineto{\pgfpoint{199.871979pt}{164.880005pt}}
\pgfpathclose
\pgfusepath{fill,stroke}
\color[rgb]{0.119627,0.617266,0.536796}
\pgfpathmoveto{\pgfpoint{315.935974pt}{103.111580pt}}
\pgflineto{\pgfpoint{324.863983pt}{103.111580pt}}
\pgflineto{\pgfpoint{324.863983pt}{96.934731pt}}
\pgfpathclose
\pgfusepath{fill,stroke}
\color[rgb]{0.122046,0.632107,0.530848}
\pgfpathmoveto{\pgfpoint{315.935974pt}{109.288422pt}}
\pgflineto{\pgfpoint{324.863983pt}{103.111580pt}}
\pgflineto{\pgfpoint{315.935974pt}{103.111580pt}}
\pgfpathclose
\pgfusepath{fill,stroke}
\color[rgb]{0.127668,0.646882,0.523924}
\pgfpathmoveto{\pgfpoint{333.791992pt}{96.934731pt}}
\pgflineto{\pgfpoint{342.719971pt}{96.934731pt}}
\pgflineto{\pgfpoint{342.719971pt}{90.757896pt}}
\pgfpathclose
\pgfusepath{fill,stroke}
\color[rgb]{0.136835,0.661563,0.515967}
\pgfpathmoveto{\pgfpoint{333.791992pt}{103.111580pt}}
\pgflineto{\pgfpoint{342.719971pt}{96.934731pt}}
\pgflineto{\pgfpoint{333.791992pt}{96.934731pt}}
\pgfpathclose
\pgfusepath{fill,stroke}
\color[rgb]{0.122163,0.587476,0.546023}
\pgfpathmoveto{\pgfpoint{271.295990pt}{121.642097pt}}
\pgflineto{\pgfpoint{280.223969pt}{121.642097pt}}
\pgflineto{\pgfpoint{280.223969pt}{115.465263pt}}
\pgfpathclose
\pgfusepath{fill,stroke}
\pgfpathmoveto{\pgfpoint{271.295990pt}{127.818947pt}}
\pgflineto{\pgfpoint{280.223969pt}{121.642097pt}}
\pgflineto{\pgfpoint{271.295990pt}{121.642097pt}}
\pgfpathclose
\pgfusepath{fill,stroke}
\pgfpathmoveto{\pgfpoint{280.223969pt}{115.465263pt}}
\pgflineto{\pgfpoint{289.151978pt}{115.465263pt}}
\pgflineto{\pgfpoint{289.151978pt}{109.288422pt}}
\pgfpathclose
\pgfusepath{fill,stroke}
\color[rgb]{0.119872,0.602382,0.541831}
\pgfpathmoveto{\pgfpoint{280.223969pt}{121.642097pt}}
\pgflineto{\pgfpoint{289.151978pt}{115.465263pt}}
\pgflineto{\pgfpoint{280.223969pt}{115.465263pt}}
\pgfpathclose
\pgfusepath{fill,stroke}
\pgfpathmoveto{\pgfpoint{298.079987pt}{109.288422pt}}
\pgflineto{\pgfpoint{307.007965pt}{109.288422pt}}
\pgflineto{\pgfpoint{307.007965pt}{103.111580pt}}
\pgfpathclose
\pgfusepath{fill,stroke}
\color[rgb]{0.119627,0.617266,0.536796}
\pgfpathmoveto{\pgfpoint{298.079987pt}{115.465263pt}}
\pgflineto{\pgfpoint{307.007965pt}{109.288422pt}}
\pgflineto{\pgfpoint{298.079987pt}{109.288422pt}}
\pgfpathclose
\pgfusepath{fill,stroke}
\color[rgb]{0.135833,0.542750,0.554289}
\pgfpathmoveto{\pgfpoint{235.583969pt}{133.995789pt}}
\pgflineto{\pgfpoint{244.511993pt}{133.995789pt}}
\pgflineto{\pgfpoint{244.511993pt}{127.818947pt}}
\pgfpathclose
\pgfusepath{fill,stroke}
\color[rgb]{0.130582,0.557652,0.552176}
\pgfpathmoveto{\pgfpoint{235.583969pt}{140.172638pt}}
\pgflineto{\pgfpoint{244.511993pt}{133.995789pt}}
\pgflineto{\pgfpoint{235.583969pt}{133.995789pt}}
\pgfpathclose
\pgfusepath{fill,stroke}
\pgfpathmoveto{\pgfpoint{253.440002pt}{127.818947pt}}
\pgflineto{\pgfpoint{262.367981pt}{127.818947pt}}
\pgflineto{\pgfpoint{262.367981pt}{121.642097pt}}
\pgfpathclose
\pgfusepath{fill,stroke}
\color[rgb]{0.125898,0.572563,0.549445}
\pgfpathmoveto{\pgfpoint{253.440002pt}{133.995789pt}}
\pgflineto{\pgfpoint{262.367981pt}{127.818947pt}}
\pgflineto{\pgfpoint{253.440002pt}{127.818947pt}}
\pgfpathclose
\pgfusepath{fill,stroke}
\color[rgb]{0.141402,0.527854,0.555864}
\pgfpathmoveto{\pgfpoint{217.727982pt}{146.349472pt}}
\pgflineto{\pgfpoint{226.655975pt}{146.349472pt}}
\pgflineto{\pgfpoint{226.655975pt}{140.172638pt}}
\pgfpathclose
\pgfusepath{fill,stroke}
\color[rgb]{0.135833,0.542750,0.554289}
\pgfpathmoveto{\pgfpoint{217.727982pt}{152.526306pt}}
\pgflineto{\pgfpoint{226.655975pt}{146.349472pt}}
\pgflineto{\pgfpoint{217.727982pt}{146.349472pt}}
\pgfpathclose
\pgfusepath{fill,stroke}
\pgfpathmoveto{\pgfpoint{217.727982pt}{152.526306pt}}
\pgflineto{\pgfpoint{226.655975pt}{152.526306pt}}
\pgflineto{\pgfpoint{226.655975pt}{146.349472pt}}
\pgfpathclose
\pgfusepath{fill,stroke}
\pgfpathmoveto{\pgfpoint{226.655975pt}{140.172638pt}}
\pgflineto{\pgfpoint{235.583969pt}{133.995789pt}}
\pgflineto{\pgfpoint{226.655975pt}{133.995789pt}}
\pgfpathclose
\pgfusepath{fill,stroke}
\pgfpathmoveto{\pgfpoint{226.655975pt}{140.172638pt}}
\pgflineto{\pgfpoint{235.583969pt}{140.172638pt}}
\pgflineto{\pgfpoint{235.583969pt}{133.995789pt}}
\pgfpathclose
\pgfusepath{fill,stroke}
\pgfpathmoveto{\pgfpoint{226.655975pt}{146.349472pt}}
\pgflineto{\pgfpoint{235.583969pt}{140.172638pt}}
\pgflineto{\pgfpoint{226.655975pt}{140.172638pt}}
\pgfpathclose
\pgfusepath{fill,stroke}
\pgfpathmoveto{\pgfpoint{226.655975pt}{146.349472pt}}
\pgflineto{\pgfpoint{235.583969pt}{146.349472pt}}
\pgflineto{\pgfpoint{235.583969pt}{140.172638pt}}
\pgfpathclose
\pgfusepath{fill,stroke}
\color[rgb]{0.130582,0.557652,0.552176}
\pgfpathmoveto{\pgfpoint{226.655975pt}{152.526306pt}}
\pgflineto{\pgfpoint{235.583969pt}{146.349472pt}}
\pgflineto{\pgfpoint{226.655975pt}{146.349472pt}}
\pgfpathclose
\pgfusepath{fill,stroke}
\pgfpathmoveto{\pgfpoint{235.583969pt}{140.172638pt}}
\pgflineto{\pgfpoint{244.511993pt}{140.172638pt}}
\pgflineto{\pgfpoint{244.511993pt}{133.995789pt}}
\pgfpathclose
\pgfusepath{fill,stroke}
\color[rgb]{0.125898,0.572563,0.549445}
\pgfpathmoveto{\pgfpoint{235.583969pt}{146.349472pt}}
\pgflineto{\pgfpoint{244.511993pt}{140.172638pt}}
\pgflineto{\pgfpoint{235.583969pt}{140.172638pt}}
\pgfpathclose
\pgfusepath{fill,stroke}
\color[rgb]{0.130582,0.557652,0.552176}
\pgfpathmoveto{\pgfpoint{244.511993pt}{133.995789pt}}
\pgflineto{\pgfpoint{253.440002pt}{127.818947pt}}
\pgflineto{\pgfpoint{244.511993pt}{127.818947pt}}
\pgfpathclose
\pgfusepath{fill,stroke}
\pgfpathmoveto{\pgfpoint{244.511993pt}{133.995789pt}}
\pgflineto{\pgfpoint{253.440002pt}{133.995789pt}}
\pgflineto{\pgfpoint{253.440002pt}{127.818947pt}}
\pgfpathclose
\pgfusepath{fill,stroke}
\color[rgb]{0.125898,0.572563,0.549445}
\pgfpathmoveto{\pgfpoint{244.511993pt}{140.172638pt}}
\pgflineto{\pgfpoint{253.440002pt}{133.995789pt}}
\pgflineto{\pgfpoint{244.511993pt}{133.995789pt}}
\pgfpathclose
\pgfusepath{fill,stroke}
\pgfpathmoveto{\pgfpoint{244.511993pt}{140.172638pt}}
\pgflineto{\pgfpoint{253.440002pt}{140.172638pt}}
\pgflineto{\pgfpoint{253.440002pt}{133.995789pt}}
\pgfpathclose
\pgfusepath{fill,stroke}
\pgfpathmoveto{\pgfpoint{253.440002pt}{133.995789pt}}
\pgflineto{\pgfpoint{262.367981pt}{133.995789pt}}
\pgflineto{\pgfpoint{262.367981pt}{127.818947pt}}
\pgfpathclose
\pgfusepath{fill,stroke}
\color[rgb]{0.122163,0.587476,0.546023}
\pgfpathmoveto{\pgfpoint{253.440002pt}{140.172638pt}}
\pgflineto{\pgfpoint{262.367981pt}{133.995789pt}}
\pgflineto{\pgfpoint{253.440002pt}{133.995789pt}}
\pgfpathclose
\pgfusepath{fill,stroke}
\color[rgb]{0.125898,0.572563,0.549445}
\pgfpathmoveto{\pgfpoint{262.367981pt}{127.818947pt}}
\pgflineto{\pgfpoint{271.295990pt}{121.642097pt}}
\pgflineto{\pgfpoint{262.367981pt}{121.642097pt}}
\pgfpathclose
\pgfusepath{fill,stroke}
\pgfpathmoveto{\pgfpoint{262.367981pt}{127.818947pt}}
\pgflineto{\pgfpoint{271.295990pt}{127.818947pt}}
\pgflineto{\pgfpoint{271.295990pt}{121.642097pt}}
\pgfpathclose
\pgfusepath{fill,stroke}
\color[rgb]{0.122163,0.587476,0.546023}
\pgfpathmoveto{\pgfpoint{262.367981pt}{133.995789pt}}
\pgflineto{\pgfpoint{271.295990pt}{127.818947pt}}
\pgflineto{\pgfpoint{262.367981pt}{127.818947pt}}
\pgfpathclose
\pgfusepath{fill,stroke}
\pgfpathmoveto{\pgfpoint{262.367981pt}{133.995789pt}}
\pgflineto{\pgfpoint{271.295990pt}{133.995789pt}}
\pgflineto{\pgfpoint{271.295990pt}{127.818947pt}}
\pgfpathclose
\pgfusepath{fill,stroke}
\pgfpathmoveto{\pgfpoint{271.295990pt}{127.818947pt}}
\pgflineto{\pgfpoint{280.223969pt}{127.818947pt}}
\pgflineto{\pgfpoint{280.223969pt}{121.642097pt}}
\pgfpathclose
\pgfusepath{fill,stroke}
\color[rgb]{0.119872,0.602382,0.541831}
\pgfpathmoveto{\pgfpoint{271.295990pt}{133.995789pt}}
\pgflineto{\pgfpoint{280.223969pt}{127.818947pt}}
\pgflineto{\pgfpoint{271.295990pt}{127.818947pt}}
\pgfpathclose
\pgfusepath{fill,stroke}
\pgfpathmoveto{\pgfpoint{280.223969pt}{121.642097pt}}
\pgflineto{\pgfpoint{289.151978pt}{121.642097pt}}
\pgflineto{\pgfpoint{289.151978pt}{115.465263pt}}
\pgfpathclose
\pgfusepath{fill,stroke}
\pgfpathmoveto{\pgfpoint{280.223969pt}{127.818947pt}}
\pgflineto{\pgfpoint{289.151978pt}{121.642097pt}}
\pgflineto{\pgfpoint{280.223969pt}{121.642097pt}}
\pgfpathclose
\pgfusepath{fill,stroke}
\pgfpathmoveto{\pgfpoint{280.223969pt}{127.818947pt}}
\pgflineto{\pgfpoint{289.151978pt}{127.818947pt}}
\pgflineto{\pgfpoint{289.151978pt}{121.642097pt}}
\pgfpathclose
\pgfusepath{fill,stroke}
\pgfpathmoveto{\pgfpoint{289.151978pt}{115.465263pt}}
\pgflineto{\pgfpoint{298.079987pt}{109.288422pt}}
\pgflineto{\pgfpoint{289.151978pt}{109.288422pt}}
\pgfpathclose
\pgfusepath{fill,stroke}
\pgfpathmoveto{\pgfpoint{289.151978pt}{115.465263pt}}
\pgflineto{\pgfpoint{298.079987pt}{115.465263pt}}
\pgflineto{\pgfpoint{298.079987pt}{109.288422pt}}
\pgfpathclose
\pgfusepath{fill,stroke}
\color[rgb]{0.119627,0.617266,0.536796}
\pgfpathmoveto{\pgfpoint{289.151978pt}{121.642097pt}}
\pgflineto{\pgfpoint{298.079987pt}{115.465263pt}}
\pgflineto{\pgfpoint{289.151978pt}{115.465263pt}}
\pgfpathclose
\pgfusepath{fill,stroke}
\pgfpathmoveto{\pgfpoint{289.151978pt}{121.642097pt}}
\pgflineto{\pgfpoint{298.079987pt}{121.642097pt}}
\pgflineto{\pgfpoint{298.079987pt}{115.465263pt}}
\pgfpathclose
\pgfusepath{fill,stroke}
\color[rgb]{0.122046,0.632107,0.530848}
\pgfpathmoveto{\pgfpoint{289.151978pt}{127.818947pt}}
\pgflineto{\pgfpoint{298.079987pt}{121.642097pt}}
\pgflineto{\pgfpoint{289.151978pt}{121.642097pt}}
\pgfpathclose
\pgfusepath{fill,stroke}
\color[rgb]{0.119627,0.617266,0.536796}
\pgfpathmoveto{\pgfpoint{298.079987pt}{115.465263pt}}
\pgflineto{\pgfpoint{307.007965pt}{115.465263pt}}
\pgflineto{\pgfpoint{307.007965pt}{109.288422pt}}
\pgfpathclose
\pgfusepath{fill,stroke}
\color[rgb]{0.122046,0.632107,0.530848}
\pgfpathmoveto{\pgfpoint{298.079987pt}{121.642097pt}}
\pgflineto{\pgfpoint{307.007965pt}{115.465263pt}}
\pgflineto{\pgfpoint{298.079987pt}{115.465263pt}}
\pgfpathclose
\pgfusepath{fill,stroke}
\pgfpathmoveto{\pgfpoint{298.079987pt}{121.642097pt}}
\pgflineto{\pgfpoint{307.007965pt}{121.642097pt}}
\pgflineto{\pgfpoint{307.007965pt}{115.465263pt}}
\pgfpathclose
\pgfusepath{fill,stroke}
\color[rgb]{0.119627,0.617266,0.536796}
\pgfpathmoveto{\pgfpoint{307.007965pt}{109.288422pt}}
\pgflineto{\pgfpoint{315.935974pt}{103.111580pt}}
\pgflineto{\pgfpoint{307.007965pt}{103.111580pt}}
\pgfpathclose
\pgfusepath{fill,stroke}
\pgfpathmoveto{\pgfpoint{307.007965pt}{109.288422pt}}
\pgflineto{\pgfpoint{315.935974pt}{109.288422pt}}
\pgflineto{\pgfpoint{315.935974pt}{103.111580pt}}
\pgfpathclose
\pgfusepath{fill,stroke}
\color[rgb]{0.122046,0.632107,0.530848}
\pgfpathmoveto{\pgfpoint{307.007965pt}{115.465263pt}}
\pgflineto{\pgfpoint{315.935974pt}{109.288422pt}}
\pgflineto{\pgfpoint{307.007965pt}{109.288422pt}}
\pgfpathclose
\pgfusepath{fill,stroke}
\pgfpathmoveto{\pgfpoint{307.007965pt}{115.465263pt}}
\pgflineto{\pgfpoint{315.935974pt}{115.465263pt}}
\pgflineto{\pgfpoint{315.935974pt}{109.288422pt}}
\pgfpathclose
\pgfusepath{fill,stroke}
\color[rgb]{0.127668,0.646882,0.523924}
\pgfpathmoveto{\pgfpoint{307.007965pt}{121.642097pt}}
\pgflineto{\pgfpoint{315.935974pt}{115.465263pt}}
\pgflineto{\pgfpoint{307.007965pt}{115.465263pt}}
\pgfpathclose
\pgfusepath{fill,stroke}
\color[rgb]{0.122046,0.632107,0.530848}
\pgfpathmoveto{\pgfpoint{315.935974pt}{109.288422pt}}
\pgflineto{\pgfpoint{324.863983pt}{109.288422pt}}
\pgflineto{\pgfpoint{324.863983pt}{103.111580pt}}
\pgfpathclose
\pgfusepath{fill,stroke}
\color[rgb]{0.127668,0.646882,0.523924}
\pgfpathmoveto{\pgfpoint{315.935974pt}{115.465263pt}}
\pgflineto{\pgfpoint{324.863983pt}{109.288422pt}}
\pgflineto{\pgfpoint{315.935974pt}{109.288422pt}}
\pgfpathclose
\pgfusepath{fill,stroke}
\color[rgb]{0.122046,0.632107,0.530848}
\pgfpathmoveto{\pgfpoint{324.863983pt}{103.111580pt}}
\pgflineto{\pgfpoint{333.791992pt}{96.934731pt}}
\pgflineto{\pgfpoint{324.863983pt}{96.934731pt}}
\pgfpathclose
\pgfusepath{fill,stroke}
\pgfpathmoveto{\pgfpoint{324.863983pt}{103.111580pt}}
\pgflineto{\pgfpoint{333.791992pt}{103.111580pt}}
\pgflineto{\pgfpoint{333.791992pt}{96.934731pt}}
\pgfpathclose
\pgfusepath{fill,stroke}
\color[rgb]{0.127668,0.646882,0.523924}
\pgfpathmoveto{\pgfpoint{324.863983pt}{109.288422pt}}
\pgflineto{\pgfpoint{333.791992pt}{103.111580pt}}
\pgflineto{\pgfpoint{324.863983pt}{103.111580pt}}
\pgfpathclose
\pgfusepath{fill,stroke}
\pgfpathmoveto{\pgfpoint{324.863983pt}{109.288422pt}}
\pgflineto{\pgfpoint{333.791992pt}{109.288422pt}}
\pgflineto{\pgfpoint{333.791992pt}{103.111580pt}}
\pgfpathclose
\pgfusepath{fill,stroke}
\color[rgb]{0.136835,0.661563,0.515967}
\pgfpathmoveto{\pgfpoint{333.791992pt}{103.111580pt}}
\pgflineto{\pgfpoint{342.719971pt}{103.111580pt}}
\pgflineto{\pgfpoint{342.719971pt}{96.934731pt}}
\pgfpathclose
\pgfusepath{fill,stroke}
\pgfpathmoveto{\pgfpoint{333.791992pt}{109.288422pt}}
\pgflineto{\pgfpoint{342.719971pt}{103.111580pt}}
\pgflineto{\pgfpoint{333.791992pt}{103.111580pt}}
\pgfpathclose
\pgfusepath{fill,stroke}
\color[rgb]{0.127668,0.646882,0.523924}
\pgfpathmoveto{\pgfpoint{342.719971pt}{90.757896pt}}
\pgflineto{\pgfpoint{351.647980pt}{90.757896pt}}
\pgflineto{\pgfpoint{351.647980pt}{84.581039pt}}
\pgfpathclose
\pgfusepath{fill,stroke}
\color[rgb]{0.136835,0.661563,0.515967}
\pgfpathmoveto{\pgfpoint{342.719971pt}{96.934731pt}}
\pgflineto{\pgfpoint{351.647980pt}{90.757896pt}}
\pgflineto{\pgfpoint{342.719971pt}{90.757896pt}}
\pgfpathclose
\pgfusepath{fill,stroke}
\pgfpathmoveto{\pgfpoint{342.719971pt}{96.934731pt}}
\pgflineto{\pgfpoint{351.647980pt}{96.934731pt}}
\pgflineto{\pgfpoint{351.647980pt}{90.757896pt}}
\pgfpathclose
\pgfusepath{fill,stroke}
\color[rgb]{0.149643,0.676120,0.506924}
\pgfpathmoveto{\pgfpoint{342.719971pt}{103.111580pt}}
\pgflineto{\pgfpoint{351.647980pt}{96.934731pt}}
\pgflineto{\pgfpoint{342.719971pt}{96.934731pt}}
\pgfpathclose
\pgfusepath{fill,stroke}
\pgfpathmoveto{\pgfpoint{342.719971pt}{103.111580pt}}
\pgflineto{\pgfpoint{351.647980pt}{103.111580pt}}
\pgflineto{\pgfpoint{351.647980pt}{96.934731pt}}
\pgfpathclose
\pgfusepath{fill,stroke}
\color[rgb]{0.136835,0.661563,0.515967}
\pgfpathmoveto{\pgfpoint{351.647980pt}{90.757896pt}}
\pgflineto{\pgfpoint{360.575958pt}{84.581039pt}}
\pgflineto{\pgfpoint{351.647980pt}{84.581039pt}}
\pgfpathclose
\pgfusepath{fill,stroke}
\pgfpathmoveto{\pgfpoint{351.647980pt}{90.757896pt}}
\pgflineto{\pgfpoint{360.575958pt}{90.757896pt}}
\pgflineto{\pgfpoint{360.575958pt}{84.581039pt}}
\pgfpathclose
\pgfusepath{fill,stroke}
\color[rgb]{0.149643,0.676120,0.506924}
\pgfpathmoveto{\pgfpoint{351.647980pt}{96.934731pt}}
\pgflineto{\pgfpoint{360.575958pt}{90.757896pt}}
\pgflineto{\pgfpoint{351.647980pt}{90.757896pt}}
\pgfpathclose
\pgfusepath{fill,stroke}
\pgfpathmoveto{\pgfpoint{351.647980pt}{96.934731pt}}
\pgflineto{\pgfpoint{360.575958pt}{96.934731pt}}
\pgflineto{\pgfpoint{360.575958pt}{90.757896pt}}
\pgfpathclose
\pgfusepath{fill,stroke}
\color[rgb]{0.165967,0.690519,0.496752}
\pgfpathmoveto{\pgfpoint{351.647980pt}{103.111580pt}}
\pgflineto{\pgfpoint{360.575958pt}{96.934731pt}}
\pgflineto{\pgfpoint{351.647980pt}{96.934731pt}}
\pgfpathclose
\pgfusepath{fill,stroke}
\color[rgb]{0.136835,0.661563,0.515967}
\pgfpathmoveto{\pgfpoint{360.575958pt}{84.581039pt}}
\pgflineto{\pgfpoint{369.503998pt}{84.581039pt}}
\pgflineto{\pgfpoint{369.503998pt}{78.404205pt}}
\pgfpathclose
\pgfusepath{fill,stroke}
\color[rgb]{0.149643,0.676120,0.506924}
\pgfpathmoveto{\pgfpoint{360.575958pt}{90.757896pt}}
\pgflineto{\pgfpoint{369.503998pt}{84.581039pt}}
\pgflineto{\pgfpoint{360.575958pt}{84.581039pt}}
\pgfpathclose
\pgfusepath{fill,stroke}
\pgfpathmoveto{\pgfpoint{360.575958pt}{90.757896pt}}
\pgflineto{\pgfpoint{369.503998pt}{90.757896pt}}
\pgflineto{\pgfpoint{369.503998pt}{84.581039pt}}
\pgfpathclose
\pgfusepath{fill,stroke}
\color[rgb]{0.165967,0.690519,0.496752}
\pgfpathmoveto{\pgfpoint{360.575958pt}{96.934731pt}}
\pgflineto{\pgfpoint{369.503998pt}{90.757896pt}}
\pgflineto{\pgfpoint{360.575958pt}{90.757896pt}}
\pgfpathclose
\pgfusepath{fill,stroke}
\pgfpathmoveto{\pgfpoint{360.575958pt}{96.934731pt}}
\pgflineto{\pgfpoint{369.503998pt}{96.934731pt}}
\pgflineto{\pgfpoint{369.503998pt}{90.757896pt}}
\pgfpathclose
\pgfusepath{fill,stroke}
\color[rgb]{0.149643,0.676120,0.506924}
\pgfpathmoveto{\pgfpoint{369.503998pt}{84.581039pt}}
\pgflineto{\pgfpoint{378.431976pt}{78.404205pt}}
\pgflineto{\pgfpoint{369.503998pt}{78.404205pt}}
\pgfpathclose
\pgfusepath{fill,stroke}
\pgfpathmoveto{\pgfpoint{369.503998pt}{84.581039pt}}
\pgflineto{\pgfpoint{378.431976pt}{84.581039pt}}
\pgflineto{\pgfpoint{378.431976pt}{78.404205pt}}
\pgfpathclose
\pgfusepath{fill,stroke}
\color[rgb]{0.165967,0.690519,0.496752}
\pgfpathmoveto{\pgfpoint{369.503998pt}{90.757896pt}}
\pgflineto{\pgfpoint{378.431976pt}{84.581039pt}}
\pgflineto{\pgfpoint{369.503998pt}{84.581039pt}}
\pgfpathclose
\pgfusepath{fill,stroke}
\pgfpathmoveto{\pgfpoint{369.503998pt}{90.757896pt}}
\pgflineto{\pgfpoint{378.431976pt}{90.757896pt}}
\pgflineto{\pgfpoint{378.431976pt}{84.581039pt}}
\pgfpathclose
\pgfusepath{fill,stroke}
\color[rgb]{0.185538,0.704725,0.485412}
\pgfpathmoveto{\pgfpoint{369.503998pt}{96.934731pt}}
\pgflineto{\pgfpoint{378.431976pt}{90.757896pt}}
\pgflineto{\pgfpoint{369.503998pt}{90.757896pt}}
\pgfpathclose
\pgfusepath{fill,stroke}
\color[rgb]{0.165967,0.690519,0.496752}
\pgfpathmoveto{\pgfpoint{378.431976pt}{78.404205pt}}
\pgflineto{\pgfpoint{387.359985pt}{78.404205pt}}
\pgflineto{\pgfpoint{387.359985pt}{72.227356pt}}
\pgfpathclose
\pgfusepath{fill,stroke}
\pgfpathmoveto{\pgfpoint{378.431976pt}{84.581039pt}}
\pgflineto{\pgfpoint{387.359985pt}{78.404205pt}}
\pgflineto{\pgfpoint{378.431976pt}{78.404205pt}}
\pgfpathclose
\pgfusepath{fill,stroke}
\pgfpathmoveto{\pgfpoint{378.431976pt}{84.581039pt}}
\pgflineto{\pgfpoint{387.359985pt}{84.581039pt}}
\pgflineto{\pgfpoint{387.359985pt}{78.404205pt}}
\pgfpathclose
\pgfusepath{fill,stroke}
\color[rgb]{0.185538,0.704725,0.485412}
\pgfpathmoveto{\pgfpoint{378.431976pt}{90.757896pt}}
\pgflineto{\pgfpoint{387.359985pt}{84.581039pt}}
\pgflineto{\pgfpoint{378.431976pt}{84.581039pt}}
\pgfpathclose
\pgfusepath{fill,stroke}
\pgfpathmoveto{\pgfpoint{387.359985pt}{78.404205pt}}
\pgflineto{\pgfpoint{396.287964pt}{72.227356pt}}
\pgflineto{\pgfpoint{387.359985pt}{72.227356pt}}
\pgfpathclose
\pgfusepath{fill,stroke}
\pgfpathmoveto{\pgfpoint{387.359985pt}{78.404205pt}}
\pgflineto{\pgfpoint{396.287964pt}{78.404205pt}}
\pgflineto{\pgfpoint{396.287964pt}{72.227356pt}}
\pgfpathclose
\pgfusepath{fill,stroke}
\color[rgb]{0.208030,0.718701,0.472873}
\pgfpathmoveto{\pgfpoint{387.359985pt}{84.581039pt}}
\pgflineto{\pgfpoint{396.287964pt}{78.404205pt}}
\pgflineto{\pgfpoint{387.359985pt}{78.404205pt}}
\pgfpathclose
\pgfusepath{fill,stroke}
\pgfpathmoveto{\pgfpoint{387.359985pt}{84.581039pt}}
\pgflineto{\pgfpoint{396.287964pt}{84.581039pt}}
\pgflineto{\pgfpoint{396.287964pt}{78.404205pt}}
\pgfpathclose
\pgfusepath{fill,stroke}
\color[rgb]{0.185538,0.704725,0.485412}
\pgfpathmoveto{\pgfpoint{396.287964pt}{72.227356pt}}
\pgflineto{\pgfpoint{405.216003pt}{72.227356pt}}
\pgflineto{\pgfpoint{405.216003pt}{66.050522pt}}
\pgfpathclose
\pgfusepath{fill,stroke}
\color[rgb]{0.208030,0.718701,0.472873}
\pgfpathmoveto{\pgfpoint{396.287964pt}{78.404205pt}}
\pgflineto{\pgfpoint{405.216003pt}{72.227356pt}}
\pgflineto{\pgfpoint{396.287964pt}{72.227356pt}}
\pgfpathclose
\pgfusepath{fill,stroke}
\pgfpathmoveto{\pgfpoint{396.287964pt}{78.404205pt}}
\pgflineto{\pgfpoint{405.216003pt}{78.404205pt}}
\pgflineto{\pgfpoint{405.216003pt}{72.227356pt}}
\pgfpathclose
\pgfusepath{fill,stroke}
\color[rgb]{0.233127,0.732406,0.459106}
\pgfpathmoveto{\pgfpoint{396.287964pt}{84.581039pt}}
\pgflineto{\pgfpoint{405.216003pt}{78.404205pt}}
\pgflineto{\pgfpoint{396.287964pt}{78.404205pt}}
\pgfpathclose
\pgfusepath{fill,stroke}
\color[rgb]{0.208030,0.718701,0.472873}
\pgfpathmoveto{\pgfpoint{405.216003pt}{72.227356pt}}
\pgflineto{\pgfpoint{414.143982pt}{66.050522pt}}
\pgflineto{\pgfpoint{405.216003pt}{66.050522pt}}
\pgfpathclose
\pgfusepath{fill,stroke}
\pgfpathmoveto{\pgfpoint{405.216003pt}{72.227356pt}}
\pgflineto{\pgfpoint{414.143982pt}{72.227356pt}}
\pgflineto{\pgfpoint{414.143982pt}{66.050522pt}}
\pgfpathclose
\pgfusepath{fill,stroke}
\color[rgb]{0.233127,0.732406,0.459106}
\pgfpathmoveto{\pgfpoint{405.216003pt}{78.404205pt}}
\pgflineto{\pgfpoint{414.143982pt}{72.227356pt}}
\pgflineto{\pgfpoint{405.216003pt}{72.227356pt}}
\pgfpathclose
\pgfusepath{fill,stroke}
\pgfpathmoveto{\pgfpoint{405.216003pt}{78.404205pt}}
\pgflineto{\pgfpoint{414.143982pt}{78.404205pt}}
\pgflineto{\pgfpoint{414.143982pt}{72.227356pt}}
\pgfpathclose
\pgfusepath{fill,stroke}
\color[rgb]{0.208030,0.718701,0.472873}
\pgfpathmoveto{\pgfpoint{414.143982pt}{66.050522pt}}
\pgflineto{\pgfpoint{423.071960pt}{66.050522pt}}
\pgflineto{\pgfpoint{423.071960pt}{59.873672pt}}
\pgfpathclose
\pgfusepath{fill,stroke}
\color[rgb]{0.233127,0.732406,0.459106}
\pgfpathmoveto{\pgfpoint{414.143982pt}{72.227356pt}}
\pgflineto{\pgfpoint{423.071960pt}{66.050522pt}}
\pgflineto{\pgfpoint{414.143982pt}{66.050522pt}}
\pgfpathclose
\pgfusepath{fill,stroke}
\pgfpathmoveto{\pgfpoint{414.143982pt}{72.227356pt}}
\pgflineto{\pgfpoint{423.071960pt}{72.227356pt}}
\pgflineto{\pgfpoint{423.071960pt}{66.050522pt}}
\pgfpathclose
\pgfusepath{fill,stroke}
\color[rgb]{0.260531,0.745802,0.444096}
\pgfpathmoveto{\pgfpoint{414.143982pt}{78.404205pt}}
\pgflineto{\pgfpoint{423.071960pt}{72.227356pt}}
\pgflineto{\pgfpoint{414.143982pt}{72.227356pt}}
\pgfpathclose
\pgfusepath{fill,stroke}
\color[rgb]{0.208030,0.718701,0.472873}
\pgfpathmoveto{\pgfpoint{423.071960pt}{59.873672pt}}
\pgflineto{\pgfpoint{432.000000pt}{59.873672pt}}
\pgflineto{\pgfpoint{432.000000pt}{53.696838pt}}
\pgfpathclose
\pgfusepath{fill,stroke}
\color[rgb]{0.233127,0.732406,0.459106}
\pgfpathmoveto{\pgfpoint{423.071960pt}{66.050522pt}}
\pgflineto{\pgfpoint{432.000000pt}{59.873672pt}}
\pgflineto{\pgfpoint{423.071960pt}{59.873672pt}}
\pgfpathclose
\pgfusepath{fill,stroke}
\pgfpathmoveto{\pgfpoint{423.071960pt}{66.050522pt}}
\pgflineto{\pgfpoint{432.000000pt}{66.050522pt}}
\pgflineto{\pgfpoint{432.000000pt}{59.873672pt}}
\pgfpathclose
\pgfusepath{fill,stroke}
\color[rgb]{0.260531,0.745802,0.444096}
\pgfpathmoveto{\pgfpoint{423.071960pt}{72.227356pt}}
\pgflineto{\pgfpoint{432.000000pt}{66.050522pt}}
\pgflineto{\pgfpoint{423.071960pt}{66.050522pt}}
\pgfpathclose
\pgfusepath{fill,stroke}
\pgfpathmoveto{\pgfpoint{423.071960pt}{72.227356pt}}
\pgflineto{\pgfpoint{432.000000pt}{72.227356pt}}
\pgflineto{\pgfpoint{432.000000pt}{66.050522pt}}
\pgfpathclose
\pgfusepath{fill,stroke}
\pgfpathmoveto{\pgfpoint{432.000000pt}{59.873672pt}}
\pgflineto{\pgfpoint{440.927979pt}{53.696838pt}}
\pgflineto{\pgfpoint{432.000000pt}{53.696838pt}}
\pgfpathclose
\pgfusepath{fill,stroke}
\pgfpathmoveto{\pgfpoint{432.000000pt}{59.873672pt}}
\pgflineto{\pgfpoint{440.927979pt}{59.873672pt}}
\pgflineto{\pgfpoint{440.927979pt}{53.696838pt}}
\pgfpathclose
\pgfusepath{fill,stroke}
\pgfpathmoveto{\pgfpoint{432.000000pt}{66.050522pt}}
\pgflineto{\pgfpoint{440.927979pt}{59.873672pt}}
\pgflineto{\pgfpoint{432.000000pt}{59.873672pt}}
\pgfpathclose
\pgfusepath{fill,stroke}
\pgfpathmoveto{\pgfpoint{432.000000pt}{66.050522pt}}
\pgflineto{\pgfpoint{440.927979pt}{66.050522pt}}
\pgflineto{\pgfpoint{440.927979pt}{59.873672pt}}
\pgfpathclose
\pgfusepath{fill,stroke}
\color[rgb]{0.290001,0.758846,0.427826}
\pgfpathmoveto{\pgfpoint{432.000000pt}{72.227356pt}}
\pgflineto{\pgfpoint{440.927979pt}{66.050522pt}}
\pgflineto{\pgfpoint{432.000000pt}{66.050522pt}}
\pgfpathclose
\pgfusepath{fill,stroke}
\color[rgb]{0.260531,0.745802,0.444096}
\pgfpathmoveto{\pgfpoint{440.927979pt}{53.696838pt}}
\pgflineto{\pgfpoint{449.855957pt}{53.696838pt}}
\pgflineto{\pgfpoint{449.855957pt}{47.519989pt}}
\pgfpathclose
\pgfusepath{fill,stroke}
\color[rgb]{0.290001,0.758846,0.427826}
\pgfpathmoveto{\pgfpoint{440.927979pt}{59.873672pt}}
\pgflineto{\pgfpoint{449.855957pt}{53.696838pt}}
\pgflineto{\pgfpoint{440.927979pt}{53.696838pt}}
\pgfpathclose
\pgfusepath{fill,stroke}
\pgfpathmoveto{\pgfpoint{440.927979pt}{59.873672pt}}
\pgflineto{\pgfpoint{449.855957pt}{59.873672pt}}
\pgflineto{\pgfpoint{449.855957pt}{53.696838pt}}
\pgfpathclose
\pgfusepath{fill,stroke}
\pgfpathmoveto{\pgfpoint{440.927979pt}{66.050522pt}}
\pgflineto{\pgfpoint{449.855957pt}{59.873672pt}}
\pgflineto{\pgfpoint{440.927979pt}{59.873672pt}}
\pgfpathclose
\pgfusepath{fill,stroke}
\pgfpathmoveto{\pgfpoint{449.855957pt}{53.696838pt}}
\pgflineto{\pgfpoint{458.783936pt}{47.519989pt}}
\pgflineto{\pgfpoint{449.855957pt}{47.519989pt}}
\pgfpathclose
\pgfusepath{fill,stroke}
\pgfpathmoveto{\pgfpoint{449.855957pt}{53.696838pt}}
\pgflineto{\pgfpoint{458.783936pt}{53.696838pt}}
\pgflineto{\pgfpoint{458.783936pt}{47.519989pt}}
\pgfpathclose
\pgfusepath{fill,stroke}
\color[rgb]{0.321330,0.771498,0.410293}
\pgfpathmoveto{\pgfpoint{449.855957pt}{59.873672pt}}
\pgflineto{\pgfpoint{458.783936pt}{53.696838pt}}
\pgflineto{\pgfpoint{449.855957pt}{53.696838pt}}
\pgfpathclose
\pgfusepath{fill,stroke}
\pgfpathmoveto{\pgfpoint{449.855957pt}{59.873672pt}}
\pgflineto{\pgfpoint{458.783936pt}{59.873672pt}}
\pgflineto{\pgfpoint{458.783936pt}{53.696838pt}}
\pgfpathclose
\pgfusepath{fill,stroke}
\pgfpathmoveto{\pgfpoint{458.783936pt}{53.696838pt}}
\pgflineto{\pgfpoint{467.711975pt}{47.519989pt}}
\pgflineto{\pgfpoint{458.783936pt}{47.519989pt}}
\pgfpathclose
\pgfusepath{fill,stroke}
\pgfpathmoveto{\pgfpoint{458.783936pt}{53.696838pt}}
\pgflineto{\pgfpoint{467.711975pt}{53.696838pt}}
\pgflineto{\pgfpoint{467.711975pt}{47.519989pt}}
\pgfpathclose
\pgfusepath{fill,stroke}
\color[rgb]{0.354355,0.783714,0.391488}
\pgfpathmoveto{\pgfpoint{458.783936pt}{59.873672pt}}
\pgflineto{\pgfpoint{467.711975pt}{53.696838pt}}
\pgflineto{\pgfpoint{458.783936pt}{53.696838pt}}
\pgfpathclose
\pgfusepath{fill,stroke}
\pgfpathmoveto{\pgfpoint{467.711975pt}{53.696838pt}}
\pgflineto{\pgfpoint{476.639954pt}{47.519989pt}}
\pgflineto{\pgfpoint{467.711975pt}{47.519989pt}}
\pgfpathclose
\pgfusepath{fill,stroke}
\pgfpathmoveto{\pgfpoint{467.711975pt}{53.696838pt}}
\pgflineto{\pgfpoint{476.639954pt}{53.696838pt}}
\pgflineto{\pgfpoint{476.639954pt}{47.519989pt}}
\pgfpathclose
\pgfusepath{fill,stroke}
\color[rgb]{0.388930,0.795453,0.371421}
\pgfpathmoveto{\pgfpoint{476.639954pt}{53.696838pt}}
\pgflineto{\pgfpoint{485.567963pt}{47.519989pt}}
\pgflineto{\pgfpoint{476.639954pt}{47.519989pt}}
\pgfpathclose
\pgfusepath{fill,stroke}
\color[rgb]{0.208030,0.718701,0.472873}
\pgfpathmoveto{\pgfpoint{378.431976pt}{96.934731pt}}
\pgflineto{\pgfpoint{387.359985pt}{90.757896pt}}
\pgflineto{\pgfpoint{378.431976pt}{90.757896pt}}
\pgfpathclose
\pgfusepath{fill,stroke}
\pgfpathmoveto{\pgfpoint{378.431976pt}{96.934731pt}}
\pgflineto{\pgfpoint{387.359985pt}{96.934731pt}}
\pgflineto{\pgfpoint{387.359985pt}{90.757896pt}}
\pgfpathclose
\pgfusepath{fill,stroke}
\pgfpathmoveto{\pgfpoint{387.359985pt}{90.757896pt}}
\pgflineto{\pgfpoint{396.287964pt}{90.757896pt}}
\pgflineto{\pgfpoint{396.287964pt}{84.581039pt}}
\pgfpathclose
\pgfusepath{fill,stroke}
\color[rgb]{0.233127,0.732406,0.459106}
\pgfpathmoveto{\pgfpoint{387.359985pt}{96.934731pt}}
\pgflineto{\pgfpoint{396.287964pt}{90.757896pt}}
\pgflineto{\pgfpoint{387.359985pt}{90.757896pt}}
\pgfpathclose
\pgfusepath{fill,stroke}
\pgfpathmoveto{\pgfpoint{396.287964pt}{90.757896pt}}
\pgflineto{\pgfpoint{405.216003pt}{84.581039pt}}
\pgflineto{\pgfpoint{396.287964pt}{84.581039pt}}
\pgfpathclose
\pgfusepath{fill,stroke}
\pgfpathmoveto{\pgfpoint{396.287964pt}{90.757896pt}}
\pgflineto{\pgfpoint{405.216003pt}{90.757896pt}}
\pgflineto{\pgfpoint{405.216003pt}{84.581039pt}}
\pgfpathclose
\pgfusepath{fill,stroke}
\color[rgb]{0.321330,0.771498,0.410293}
\pgfpathmoveto{\pgfpoint{432.000000pt}{78.404205pt}}
\pgflineto{\pgfpoint{440.927979pt}{72.227356pt}}
\pgflineto{\pgfpoint{432.000000pt}{72.227356pt}}
\pgfpathclose
\pgfusepath{fill,stroke}
\color[rgb]{0.290001,0.758846,0.427826}
\pgfpathmoveto{\pgfpoint{440.927979pt}{66.050522pt}}
\pgflineto{\pgfpoint{449.855957pt}{66.050522pt}}
\pgflineto{\pgfpoint{449.855957pt}{59.873672pt}}
\pgfpathclose
\pgfusepath{fill,stroke}
\color[rgb]{0.354355,0.783714,0.391488}
\pgfpathmoveto{\pgfpoint{449.855957pt}{66.050522pt}}
\pgflineto{\pgfpoint{458.783936pt}{59.873672pt}}
\pgflineto{\pgfpoint{449.855957pt}{59.873672pt}}
\pgfpathclose
\pgfusepath{fill,stroke}
\pgfpathmoveto{\pgfpoint{458.783936pt}{59.873672pt}}
\pgflineto{\pgfpoint{467.711975pt}{59.873672pt}}
\pgflineto{\pgfpoint{467.711975pt}{53.696838pt}}
\pgfpathclose
\pgfusepath{fill,stroke}
\color[rgb]{0.388930,0.795453,0.371421}
\pgfpathmoveto{\pgfpoint{467.711975pt}{59.873672pt}}
\pgflineto{\pgfpoint{476.639954pt}{53.696838pt}}
\pgflineto{\pgfpoint{467.711975pt}{53.696838pt}}
\pgfpathclose
\pgfusepath{fill,stroke}
\pgfpathmoveto{\pgfpoint{467.711975pt}{59.873672pt}}
\pgflineto{\pgfpoint{476.639954pt}{59.873672pt}}
\pgflineto{\pgfpoint{476.639954pt}{53.696838pt}}
\pgfpathclose
\pgfusepath{fill,stroke}
\pgfpathmoveto{\pgfpoint{476.639954pt}{53.696838pt}}
\pgflineto{\pgfpoint{485.567963pt}{53.696838pt}}
\pgflineto{\pgfpoint{485.567963pt}{47.519989pt}}
\pgfpathclose
\pgfusepath{fill,stroke}
\color[rgb]{0.424933,0.806674,0.350099}
\pgfpathmoveto{\pgfpoint{476.639954pt}{59.873672pt}}
\pgflineto{\pgfpoint{485.567963pt}{53.696838pt}}
\pgflineto{\pgfpoint{476.639954pt}{53.696838pt}}
\pgfpathclose
\pgfusepath{fill,stroke}
\pgfpathmoveto{\pgfpoint{485.567963pt}{53.696838pt}}
\pgflineto{\pgfpoint{494.495972pt}{47.519989pt}}
\pgflineto{\pgfpoint{485.567963pt}{47.519989pt}}
\pgfpathclose
\pgfusepath{fill,stroke}
\pgfpathmoveto{\pgfpoint{485.567963pt}{53.696838pt}}
\pgflineto{\pgfpoint{494.495972pt}{53.696838pt}}
\pgflineto{\pgfpoint{494.495972pt}{47.519989pt}}
\pgfpathclose
\pgfusepath{fill,stroke}
\color[rgb]{0.462247,0.817338,0.327545}
\pgfpathmoveto{\pgfpoint{494.495972pt}{53.696838pt}}
\pgflineto{\pgfpoint{503.423981pt}{47.519989pt}}
\pgflineto{\pgfpoint{494.495972pt}{47.519989pt}}
\pgfpathclose
\pgfusepath{fill,stroke}
\color[rgb]{0.260531,0.745802,0.444096}
\pgfpathmoveto{\pgfpoint{405.216003pt}{84.581039pt}}
\pgflineto{\pgfpoint{414.143982pt}{84.581039pt}}
\pgflineto{\pgfpoint{414.143982pt}{78.404205pt}}
\pgfpathclose
\pgfusepath{fill,stroke}
\color[rgb]{0.290001,0.758846,0.427826}
\pgfpathmoveto{\pgfpoint{405.216003pt}{90.757896pt}}
\pgflineto{\pgfpoint{414.143982pt}{84.581039pt}}
\pgflineto{\pgfpoint{405.216003pt}{84.581039pt}}
\pgfpathclose
\pgfusepath{fill,stroke}
\pgfpathmoveto{\pgfpoint{414.143982pt}{84.581039pt}}
\pgflineto{\pgfpoint{423.071960pt}{78.404205pt}}
\pgflineto{\pgfpoint{414.143982pt}{78.404205pt}}
\pgfpathclose
\pgfusepath{fill,stroke}
\pgfpathmoveto{\pgfpoint{423.071960pt}{78.404205pt}}
\pgflineto{\pgfpoint{432.000000pt}{78.404205pt}}
\pgflineto{\pgfpoint{432.000000pt}{72.227356pt}}
\pgfpathclose
\pgfusepath{fill,stroke}
\color[rgb]{0.149643,0.676120,0.506924}
\pgfpathmoveto{\pgfpoint{342.719971pt}{109.288422pt}}
\pgflineto{\pgfpoint{351.647980pt}{109.288422pt}}
\pgflineto{\pgfpoint{351.647980pt}{103.111580pt}}
\pgfpathclose
\pgfusepath{fill,stroke}
\color[rgb]{0.165967,0.690519,0.496752}
\pgfpathmoveto{\pgfpoint{342.719971pt}{115.465263pt}}
\pgflineto{\pgfpoint{351.647980pt}{109.288422pt}}
\pgflineto{\pgfpoint{342.719971pt}{109.288422pt}}
\pgfpathclose
\pgfusepath{fill,stroke}
\pgfpathmoveto{\pgfpoint{351.647980pt}{103.111580pt}}
\pgflineto{\pgfpoint{360.575958pt}{103.111580pt}}
\pgflineto{\pgfpoint{360.575958pt}{96.934731pt}}
\pgfpathclose
\pgfusepath{fill,stroke}
\color[rgb]{0.185538,0.704725,0.485412}
\pgfpathmoveto{\pgfpoint{351.647980pt}{109.288422pt}}
\pgflineto{\pgfpoint{360.575958pt}{103.111580pt}}
\pgflineto{\pgfpoint{351.647980pt}{103.111580pt}}
\pgfpathclose
\pgfusepath{fill,stroke}
\pgfpathmoveto{\pgfpoint{369.503998pt}{96.934731pt}}
\pgflineto{\pgfpoint{378.431976pt}{96.934731pt}}
\pgflineto{\pgfpoint{378.431976pt}{90.757896pt}}
\pgfpathclose
\pgfusepath{fill,stroke}
\color[rgb]{0.208030,0.718701,0.472873}
\pgfpathmoveto{\pgfpoint{369.503998pt}{103.111580pt}}
\pgflineto{\pgfpoint{378.431976pt}{96.934731pt}}
\pgflineto{\pgfpoint{369.503998pt}{96.934731pt}}
\pgfpathclose
\pgfusepath{fill,stroke}
\color[rgb]{0.127668,0.646882,0.523924}
\pgfpathmoveto{\pgfpoint{307.007965pt}{121.642097pt}}
\pgflineto{\pgfpoint{315.935974pt}{121.642097pt}}
\pgflineto{\pgfpoint{315.935974pt}{115.465263pt}}
\pgfpathclose
\pgfusepath{fill,stroke}
\color[rgb]{0.136835,0.661563,0.515967}
\pgfpathmoveto{\pgfpoint{307.007965pt}{127.818947pt}}
\pgflineto{\pgfpoint{315.935974pt}{121.642097pt}}
\pgflineto{\pgfpoint{307.007965pt}{121.642097pt}}
\pgfpathclose
\pgfusepath{fill,stroke}
\pgfpathmoveto{\pgfpoint{324.863983pt}{115.465263pt}}
\pgflineto{\pgfpoint{333.791992pt}{115.465263pt}}
\pgflineto{\pgfpoint{333.791992pt}{109.288422pt}}
\pgfpathclose
\pgfusepath{fill,stroke}
\color[rgb]{0.149643,0.676120,0.506924}
\pgfpathmoveto{\pgfpoint{324.863983pt}{121.642097pt}}
\pgflineto{\pgfpoint{333.791992pt}{115.465263pt}}
\pgflineto{\pgfpoint{324.863983pt}{115.465263pt}}
\pgfpathclose
\pgfusepath{fill,stroke}
\color[rgb]{0.119872,0.602382,0.541831}
\pgfpathmoveto{\pgfpoint{262.367981pt}{140.172638pt}}
\pgflineto{\pgfpoint{271.295990pt}{140.172638pt}}
\pgflineto{\pgfpoint{271.295990pt}{133.995789pt}}
\pgfpathclose
\pgfusepath{fill,stroke}
\color[rgb]{0.119627,0.617266,0.536796}
\pgfpathmoveto{\pgfpoint{262.367981pt}{146.349472pt}}
\pgflineto{\pgfpoint{271.295990pt}{140.172638pt}}
\pgflineto{\pgfpoint{262.367981pt}{140.172638pt}}
\pgfpathclose
\pgfusepath{fill,stroke}
\color[rgb]{0.119872,0.602382,0.541831}
\pgfpathmoveto{\pgfpoint{271.295990pt}{133.995789pt}}
\pgflineto{\pgfpoint{280.223969pt}{133.995789pt}}
\pgflineto{\pgfpoint{280.223969pt}{127.818947pt}}
\pgfpathclose
\pgfusepath{fill,stroke}
\color[rgb]{0.119627,0.617266,0.536796}
\pgfpathmoveto{\pgfpoint{271.295990pt}{140.172638pt}}
\pgflineto{\pgfpoint{280.223969pt}{133.995789pt}}
\pgflineto{\pgfpoint{271.295990pt}{133.995789pt}}
\pgfpathclose
\pgfusepath{fill,stroke}
\color[rgb]{0.122046,0.632107,0.530848}
\pgfpathmoveto{\pgfpoint{289.151978pt}{127.818947pt}}
\pgflineto{\pgfpoint{298.079987pt}{127.818947pt}}
\pgflineto{\pgfpoint{298.079987pt}{121.642097pt}}
\pgfpathclose
\pgfusepath{fill,stroke}
\pgfpathmoveto{\pgfpoint{289.151978pt}{133.995789pt}}
\pgflineto{\pgfpoint{298.079987pt}{127.818947pt}}
\pgflineto{\pgfpoint{289.151978pt}{127.818947pt}}
\pgfpathclose
\pgfusepath{fill,stroke}
\color[rgb]{0.130582,0.557652,0.552176}
\pgfpathmoveto{\pgfpoint{226.655975pt}{152.526306pt}}
\pgflineto{\pgfpoint{235.583969pt}{152.526306pt}}
\pgflineto{\pgfpoint{235.583969pt}{146.349472pt}}
\pgfpathclose
\pgfusepath{fill,stroke}
\color[rgb]{0.125898,0.572563,0.549445}
\pgfpathmoveto{\pgfpoint{226.655975pt}{158.703156pt}}
\pgflineto{\pgfpoint{235.583969pt}{152.526306pt}}
\pgflineto{\pgfpoint{226.655975pt}{152.526306pt}}
\pgfpathclose
\pgfusepath{fill,stroke}
\color[rgb]{0.122163,0.587476,0.546023}
\pgfpathmoveto{\pgfpoint{244.511993pt}{146.349472pt}}
\pgflineto{\pgfpoint{253.440002pt}{146.349472pt}}
\pgflineto{\pgfpoint{253.440002pt}{140.172638pt}}
\pgfpathclose
\pgfusepath{fill,stroke}
\pgfpathmoveto{\pgfpoint{244.511993pt}{152.526306pt}}
\pgflineto{\pgfpoint{253.440002pt}{146.349472pt}}
\pgflineto{\pgfpoint{244.511993pt}{146.349472pt}}
\pgfpathclose
\pgfusepath{fill,stroke}
\color[rgb]{0.152951,0.498053,0.557685}
\pgfpathmoveto{\pgfpoint{199.871979pt}{146.349472pt}}
\pgflineto{\pgfpoint{208.799988pt}{146.349472pt}}
\pgflineto{\pgfpoint{208.799988pt}{140.172638pt}}
\pgfpathclose
\pgfusepath{fill,stroke}
\color[rgb]{0.147132,0.512959,0.556973}
\pgfpathmoveto{\pgfpoint{199.871979pt}{152.526306pt}}
\pgflineto{\pgfpoint{208.799988pt}{146.349472pt}}
\pgflineto{\pgfpoint{199.871979pt}{146.349472pt}}
\pgfpathclose
\pgfusepath{fill,stroke}
\pgfpathmoveto{\pgfpoint{199.871979pt}{152.526306pt}}
\pgflineto{\pgfpoint{208.799988pt}{152.526306pt}}
\pgflineto{\pgfpoint{208.799988pt}{146.349472pt}}
\pgfpathclose
\pgfusepath{fill,stroke}
\pgfpathmoveto{\pgfpoint{208.799988pt}{146.349472pt}}
\pgflineto{\pgfpoint{217.727982pt}{140.172638pt}}
\pgflineto{\pgfpoint{208.799988pt}{140.172638pt}}
\pgfpathclose
\pgfusepath{fill,stroke}
\pgfpathmoveto{\pgfpoint{208.799988pt}{146.349472pt}}
\pgflineto{\pgfpoint{217.727982pt}{146.349472pt}}
\pgflineto{\pgfpoint{217.727982pt}{140.172638pt}}
\pgfpathclose
\pgfusepath{fill,stroke}
\color[rgb]{0.141402,0.527854,0.555864}
\pgfpathmoveto{\pgfpoint{208.799988pt}{152.526306pt}}
\pgflineto{\pgfpoint{217.727982pt}{146.349472pt}}
\pgflineto{\pgfpoint{208.799988pt}{146.349472pt}}
\pgfpathclose
\pgfusepath{fill,stroke}
\pgfpathmoveto{\pgfpoint{208.799988pt}{152.526306pt}}
\pgflineto{\pgfpoint{217.727982pt}{152.526306pt}}
\pgflineto{\pgfpoint{217.727982pt}{146.349472pt}}
\pgfpathclose
\pgfusepath{fill,stroke}
\color[rgb]{0.135833,0.542750,0.554289}
\pgfpathmoveto{\pgfpoint{208.799988pt}{158.703156pt}}
\pgflineto{\pgfpoint{217.727982pt}{152.526306pt}}
\pgflineto{\pgfpoint{208.799988pt}{152.526306pt}}
\pgfpathclose
\pgfusepath{fill,stroke}
\pgfpathmoveto{\pgfpoint{208.799988pt}{158.703156pt}}
\pgflineto{\pgfpoint{217.727982pt}{158.703156pt}}
\pgflineto{\pgfpoint{217.727982pt}{152.526306pt}}
\pgfpathclose
\pgfusepath{fill,stroke}
\color[rgb]{0.130582,0.557652,0.552176}
\pgfpathmoveto{\pgfpoint{208.799988pt}{164.880005pt}}
\pgflineto{\pgfpoint{217.727982pt}{158.703156pt}}
\pgflineto{\pgfpoint{208.799988pt}{158.703156pt}}
\pgfpathclose
\pgfusepath{fill,stroke}
\color[rgb]{0.141402,0.527854,0.555864}
\pgfpathmoveto{\pgfpoint{217.727982pt}{140.172638pt}}
\pgflineto{\pgfpoint{226.655975pt}{140.172638pt}}
\pgflineto{\pgfpoint{226.655975pt}{133.995789pt}}
\pgfpathclose
\pgfusepath{fill,stroke}
\pgfpathmoveto{\pgfpoint{217.727982pt}{146.349472pt}}
\pgflineto{\pgfpoint{226.655975pt}{140.172638pt}}
\pgflineto{\pgfpoint{217.727982pt}{140.172638pt}}
\pgfpathclose
\pgfusepath{fill,stroke}
\color[rgb]{0.147132,0.512959,0.556973}
\pgfpathmoveto{\pgfpoint{164.160004pt}{183.410522pt}}
\pgflineto{\pgfpoint{173.087997pt}{183.410522pt}}
\pgflineto{\pgfpoint{173.087997pt}{177.233673pt}}
\pgfpathclose
\pgfusepath{fill,stroke}
\color[rgb]{0.152951,0.498053,0.557685}
\pgfpathmoveto{\pgfpoint{173.087997pt}{171.056854pt}}
\pgflineto{\pgfpoint{182.015991pt}{171.056854pt}}
\pgflineto{\pgfpoint{182.015991pt}{164.880005pt}}
\pgfpathclose
\pgfusepath{fill,stroke}
\color[rgb]{0.147132,0.512959,0.556973}
\pgfpathmoveto{\pgfpoint{173.087997pt}{177.233673pt}}
\pgflineto{\pgfpoint{182.015991pt}{171.056854pt}}
\pgflineto{\pgfpoint{173.087997pt}{171.056854pt}}
\pgfpathclose
\pgfusepath{fill,stroke}
\pgfpathmoveto{\pgfpoint{173.087997pt}{177.233673pt}}
\pgflineto{\pgfpoint{182.015991pt}{177.233673pt}}
\pgflineto{\pgfpoint{182.015991pt}{171.056854pt}}
\pgfpathclose
\pgfusepath{fill,stroke}
\color[rgb]{0.141402,0.527854,0.555864}
\pgfpathmoveto{\pgfpoint{173.087997pt}{183.410522pt}}
\pgflineto{\pgfpoint{182.015991pt}{177.233673pt}}
\pgflineto{\pgfpoint{173.087997pt}{177.233673pt}}
\pgfpathclose
\pgfusepath{fill,stroke}
\pgfpathmoveto{\pgfpoint{173.087997pt}{183.410522pt}}
\pgflineto{\pgfpoint{182.015991pt}{183.410522pt}}
\pgflineto{\pgfpoint{182.015991pt}{177.233673pt}}
\pgfpathclose
\pgfusepath{fill,stroke}
\color[rgb]{0.135833,0.542750,0.554289}
\pgfpathmoveto{\pgfpoint{173.087997pt}{189.587372pt}}
\pgflineto{\pgfpoint{182.015991pt}{183.410522pt}}
\pgflineto{\pgfpoint{173.087997pt}{183.410522pt}}
\pgfpathclose
\pgfusepath{fill,stroke}
\color[rgb]{0.147132,0.512959,0.556973}
\pgfpathmoveto{\pgfpoint{182.015991pt}{171.056854pt}}
\pgflineto{\pgfpoint{190.943985pt}{164.880005pt}}
\pgflineto{\pgfpoint{182.015991pt}{164.880005pt}}
\pgfpathclose
\pgfusepath{fill,stroke}
\color[rgb]{0.141402,0.527854,0.555864}
\pgfpathmoveto{\pgfpoint{182.015991pt}{177.233673pt}}
\pgflineto{\pgfpoint{190.943985pt}{177.233673pt}}
\pgflineto{\pgfpoint{190.943985pt}{171.056854pt}}
\pgfpathclose
\pgfusepath{fill,stroke}
\color[rgb]{0.135833,0.542750,0.554289}
\pgfpathmoveto{\pgfpoint{182.015991pt}{183.410522pt}}
\pgflineto{\pgfpoint{190.943985pt}{177.233673pt}}
\pgflineto{\pgfpoint{182.015991pt}{177.233673pt}}
\pgfpathclose
\pgfusepath{fill,stroke}
\pgfpathmoveto{\pgfpoint{182.015991pt}{183.410522pt}}
\pgflineto{\pgfpoint{190.943985pt}{183.410522pt}}
\pgflineto{\pgfpoint{190.943985pt}{177.233673pt}}
\pgfpathclose
\pgfusepath{fill,stroke}
\color[rgb]{0.141402,0.527854,0.555864}
\pgfpathmoveto{\pgfpoint{190.943985pt}{164.880005pt}}
\pgflineto{\pgfpoint{199.871979pt}{164.880005pt}}
\pgflineto{\pgfpoint{199.871979pt}{158.703156pt}}
\pgfpathclose
\pgfusepath{fill,stroke}
\color[rgb]{0.135833,0.542750,0.554289}
\pgfpathmoveto{\pgfpoint{190.943985pt}{171.056854pt}}
\pgflineto{\pgfpoint{199.871979pt}{164.880005pt}}
\pgflineto{\pgfpoint{190.943985pt}{164.880005pt}}
\pgfpathclose
\pgfusepath{fill,stroke}
\pgfpathmoveto{\pgfpoint{190.943985pt}{171.056854pt}}
\pgflineto{\pgfpoint{199.871979pt}{171.056854pt}}
\pgflineto{\pgfpoint{199.871979pt}{164.880005pt}}
\pgfpathclose
\pgfusepath{fill,stroke}
\pgfpathmoveto{\pgfpoint{190.943985pt}{177.233673pt}}
\pgflineto{\pgfpoint{199.871979pt}{171.056854pt}}
\pgflineto{\pgfpoint{190.943985pt}{171.056854pt}}
\pgfpathclose
\pgfusepath{fill,stroke}
\pgfpathmoveto{\pgfpoint{190.943985pt}{177.233673pt}}
\pgflineto{\pgfpoint{199.871979pt}{177.233673pt}}
\pgflineto{\pgfpoint{199.871979pt}{171.056854pt}}
\pgfpathclose
\pgfusepath{fill,stroke}
\color[rgb]{0.130582,0.557652,0.552176}
\pgfpathmoveto{\pgfpoint{190.943985pt}{183.410522pt}}
\pgflineto{\pgfpoint{199.871979pt}{177.233673pt}}
\pgflineto{\pgfpoint{190.943985pt}{177.233673pt}}
\pgfpathclose
\pgfusepath{fill,stroke}
\color[rgb]{0.135833,0.542750,0.554289}
\pgfpathmoveto{\pgfpoint{199.871979pt}{164.880005pt}}
\pgflineto{\pgfpoint{208.799988pt}{158.703156pt}}
\pgflineto{\pgfpoint{199.871979pt}{158.703156pt}}
\pgfpathclose
\pgfusepath{fill,stroke}
\color[rgb]{0.130582,0.557652,0.552176}
\pgfpathmoveto{\pgfpoint{199.871979pt}{171.056854pt}}
\pgflineto{\pgfpoint{208.799988pt}{171.056854pt}}
\pgflineto{\pgfpoint{208.799988pt}{164.880005pt}}
\pgfpathclose
\pgfusepath{fill,stroke}
\pgfpathmoveto{\pgfpoint{199.871979pt}{177.233673pt}}
\pgflineto{\pgfpoint{208.799988pt}{171.056854pt}}
\pgflineto{\pgfpoint{199.871979pt}{171.056854pt}}
\pgfpathclose
\pgfusepath{fill,stroke}
\pgfpathmoveto{\pgfpoint{199.871979pt}{177.233673pt}}
\pgflineto{\pgfpoint{208.799988pt}{177.233673pt}}
\pgflineto{\pgfpoint{208.799988pt}{171.056854pt}}
\pgfpathclose
\pgfusepath{fill,stroke}
\pgfpathmoveto{\pgfpoint{208.799988pt}{164.880005pt}}
\pgflineto{\pgfpoint{217.727982pt}{164.880005pt}}
\pgflineto{\pgfpoint{217.727982pt}{158.703156pt}}
\pgfpathclose
\pgfusepath{fill,stroke}
\color[rgb]{0.125898,0.572563,0.549445}
\pgfpathmoveto{\pgfpoint{208.799988pt}{171.056854pt}}
\pgflineto{\pgfpoint{217.727982pt}{164.880005pt}}
\pgflineto{\pgfpoint{208.799988pt}{164.880005pt}}
\pgfpathclose
\pgfusepath{fill,stroke}
\pgfpathmoveto{\pgfpoint{208.799988pt}{171.056854pt}}
\pgflineto{\pgfpoint{217.727982pt}{171.056854pt}}
\pgflineto{\pgfpoint{217.727982pt}{164.880005pt}}
\pgfpathclose
\pgfusepath{fill,stroke}
\color[rgb]{0.122163,0.587476,0.546023}
\pgfpathmoveto{\pgfpoint{208.799988pt}{177.233673pt}}
\pgflineto{\pgfpoint{217.727982pt}{171.056854pt}}
\pgflineto{\pgfpoint{208.799988pt}{171.056854pt}}
\pgfpathclose
\pgfusepath{fill,stroke}
\color[rgb]{0.130582,0.557652,0.552176}
\pgfpathmoveto{\pgfpoint{217.727982pt}{158.703156pt}}
\pgflineto{\pgfpoint{226.655975pt}{152.526306pt}}
\pgflineto{\pgfpoint{217.727982pt}{152.526306pt}}
\pgfpathclose
\pgfusepath{fill,stroke}
\pgfpathmoveto{\pgfpoint{217.727982pt}{158.703156pt}}
\pgflineto{\pgfpoint{226.655975pt}{158.703156pt}}
\pgflineto{\pgfpoint{226.655975pt}{152.526306pt}}
\pgfpathclose
\pgfusepath{fill,stroke}
\color[rgb]{0.125898,0.572563,0.549445}
\pgfpathmoveto{\pgfpoint{217.727982pt}{164.880005pt}}
\pgflineto{\pgfpoint{226.655975pt}{158.703156pt}}
\pgflineto{\pgfpoint{217.727982pt}{158.703156pt}}
\pgfpathclose
\pgfusepath{fill,stroke}
\pgfpathmoveto{\pgfpoint{217.727982pt}{164.880005pt}}
\pgflineto{\pgfpoint{226.655975pt}{164.880005pt}}
\pgflineto{\pgfpoint{226.655975pt}{158.703156pt}}
\pgfpathclose
\pgfusepath{fill,stroke}
\color[rgb]{0.122163,0.587476,0.546023}
\pgfpathmoveto{\pgfpoint{217.727982pt}{171.056854pt}}
\pgflineto{\pgfpoint{226.655975pt}{164.880005pt}}
\pgflineto{\pgfpoint{217.727982pt}{164.880005pt}}
\pgfpathclose
\pgfusepath{fill,stroke}
\color[rgb]{0.125898,0.572563,0.549445}
\pgfpathmoveto{\pgfpoint{226.655975pt}{158.703156pt}}
\pgflineto{\pgfpoint{235.583969pt}{158.703156pt}}
\pgflineto{\pgfpoint{235.583969pt}{152.526306pt}}
\pgfpathclose
\pgfusepath{fill,stroke}
\color[rgb]{0.122163,0.587476,0.546023}
\pgfpathmoveto{\pgfpoint{226.655975pt}{164.880005pt}}
\pgflineto{\pgfpoint{235.583969pt}{158.703156pt}}
\pgflineto{\pgfpoint{226.655975pt}{158.703156pt}}
\pgfpathclose
\pgfusepath{fill,stroke}
\pgfpathmoveto{\pgfpoint{226.655975pt}{164.880005pt}}
\pgflineto{\pgfpoint{235.583969pt}{164.880005pt}}
\pgflineto{\pgfpoint{235.583969pt}{158.703156pt}}
\pgfpathclose
\pgfusepath{fill,stroke}
\color[rgb]{0.125898,0.572563,0.549445}
\pgfpathmoveto{\pgfpoint{235.583969pt}{146.349472pt}}
\pgflineto{\pgfpoint{244.511993pt}{146.349472pt}}
\pgflineto{\pgfpoint{244.511993pt}{140.172638pt}}
\pgfpathclose
\pgfusepath{fill,stroke}
\pgfpathmoveto{\pgfpoint{235.583969pt}{152.526306pt}}
\pgflineto{\pgfpoint{244.511993pt}{146.349472pt}}
\pgflineto{\pgfpoint{235.583969pt}{146.349472pt}}
\pgfpathclose
\pgfusepath{fill,stroke}
\pgfpathmoveto{\pgfpoint{235.583969pt}{152.526306pt}}
\pgflineto{\pgfpoint{244.511993pt}{152.526306pt}}
\pgflineto{\pgfpoint{244.511993pt}{146.349472pt}}
\pgfpathclose
\pgfusepath{fill,stroke}
\color[rgb]{0.122163,0.587476,0.546023}
\pgfpathmoveto{\pgfpoint{235.583969pt}{158.703156pt}}
\pgflineto{\pgfpoint{244.511993pt}{152.526306pt}}
\pgflineto{\pgfpoint{235.583969pt}{152.526306pt}}
\pgfpathclose
\pgfusepath{fill,stroke}
\pgfpathmoveto{\pgfpoint{235.583969pt}{158.703156pt}}
\pgflineto{\pgfpoint{244.511993pt}{158.703156pt}}
\pgflineto{\pgfpoint{244.511993pt}{152.526306pt}}
\pgfpathclose
\pgfusepath{fill,stroke}
\color[rgb]{0.119872,0.602382,0.541831}
\pgfpathmoveto{\pgfpoint{235.583969pt}{164.880005pt}}
\pgflineto{\pgfpoint{244.511993pt}{158.703156pt}}
\pgflineto{\pgfpoint{235.583969pt}{158.703156pt}}
\pgfpathclose
\pgfusepath{fill,stroke}
\color[rgb]{0.122163,0.587476,0.546023}
\pgfpathmoveto{\pgfpoint{244.511993pt}{146.349472pt}}
\pgflineto{\pgfpoint{253.440002pt}{140.172638pt}}
\pgflineto{\pgfpoint{244.511993pt}{140.172638pt}}
\pgfpathclose
\pgfusepath{fill,stroke}
\pgfpathmoveto{\pgfpoint{244.511993pt}{152.526306pt}}
\pgflineto{\pgfpoint{253.440002pt}{152.526306pt}}
\pgflineto{\pgfpoint{253.440002pt}{146.349472pt}}
\pgfpathclose
\pgfusepath{fill,stroke}
\color[rgb]{0.119872,0.602382,0.541831}
\pgfpathmoveto{\pgfpoint{244.511993pt}{158.703156pt}}
\pgflineto{\pgfpoint{253.440002pt}{152.526306pt}}
\pgflineto{\pgfpoint{244.511993pt}{152.526306pt}}
\pgfpathclose
\pgfusepath{fill,stroke}
\pgfpathmoveto{\pgfpoint{244.511993pt}{158.703156pt}}
\pgflineto{\pgfpoint{253.440002pt}{158.703156pt}}
\pgflineto{\pgfpoint{253.440002pt}{152.526306pt}}
\pgfpathclose
\pgfusepath{fill,stroke}
\color[rgb]{0.122163,0.587476,0.546023}
\pgfpathmoveto{\pgfpoint{253.440002pt}{140.172638pt}}
\pgflineto{\pgfpoint{262.367981pt}{140.172638pt}}
\pgflineto{\pgfpoint{262.367981pt}{133.995789pt}}
\pgfpathclose
\pgfusepath{fill,stroke}
\color[rgb]{0.119872,0.602382,0.541831}
\pgfpathmoveto{\pgfpoint{253.440002pt}{146.349472pt}}
\pgflineto{\pgfpoint{262.367981pt}{140.172638pt}}
\pgflineto{\pgfpoint{253.440002pt}{140.172638pt}}
\pgfpathclose
\pgfusepath{fill,stroke}
\pgfpathmoveto{\pgfpoint{253.440002pt}{146.349472pt}}
\pgflineto{\pgfpoint{262.367981pt}{146.349472pt}}
\pgflineto{\pgfpoint{262.367981pt}{140.172638pt}}
\pgfpathclose
\pgfusepath{fill,stroke}
\color[rgb]{0.119627,0.617266,0.536796}
\pgfpathmoveto{\pgfpoint{253.440002pt}{152.526306pt}}
\pgflineto{\pgfpoint{262.367981pt}{146.349472pt}}
\pgflineto{\pgfpoint{253.440002pt}{146.349472pt}}
\pgfpathclose
\pgfusepath{fill,stroke}
\pgfpathmoveto{\pgfpoint{253.440002pt}{152.526306pt}}
\pgflineto{\pgfpoint{262.367981pt}{152.526306pt}}
\pgflineto{\pgfpoint{262.367981pt}{146.349472pt}}
\pgfpathclose
\pgfusepath{fill,stroke}
\pgfpathmoveto{\pgfpoint{253.440002pt}{158.703156pt}}
\pgflineto{\pgfpoint{262.367981pt}{152.526306pt}}
\pgflineto{\pgfpoint{253.440002pt}{152.526306pt}}
\pgfpathclose
\pgfusepath{fill,stroke}
\color[rgb]{0.119872,0.602382,0.541831}
\pgfpathmoveto{\pgfpoint{262.367981pt}{140.172638pt}}
\pgflineto{\pgfpoint{271.295990pt}{133.995789pt}}
\pgflineto{\pgfpoint{262.367981pt}{133.995789pt}}
\pgfpathclose
\pgfusepath{fill,stroke}
\color[rgb]{0.119627,0.617266,0.536796}
\pgfpathmoveto{\pgfpoint{262.367981pt}{146.349472pt}}
\pgflineto{\pgfpoint{271.295990pt}{146.349472pt}}
\pgflineto{\pgfpoint{271.295990pt}{140.172638pt}}
\pgfpathclose
\pgfusepath{fill,stroke}
\color[rgb]{0.122046,0.632107,0.530848}
\pgfpathmoveto{\pgfpoint{262.367981pt}{152.526306pt}}
\pgflineto{\pgfpoint{271.295990pt}{146.349472pt}}
\pgflineto{\pgfpoint{262.367981pt}{146.349472pt}}
\pgfpathclose
\pgfusepath{fill,stroke}
\pgfpathmoveto{\pgfpoint{262.367981pt}{152.526306pt}}
\pgflineto{\pgfpoint{271.295990pt}{152.526306pt}}
\pgflineto{\pgfpoint{271.295990pt}{146.349472pt}}
\pgfpathclose
\pgfusepath{fill,stroke}
\color[rgb]{0.119627,0.617266,0.536796}
\pgfpathmoveto{\pgfpoint{271.295990pt}{140.172638pt}}
\pgflineto{\pgfpoint{280.223969pt}{140.172638pt}}
\pgflineto{\pgfpoint{280.223969pt}{133.995789pt}}
\pgfpathclose
\pgfusepath{fill,stroke}
\color[rgb]{0.122046,0.632107,0.530848}
\pgfpathmoveto{\pgfpoint{271.295990pt}{146.349472pt}}
\pgflineto{\pgfpoint{280.223969pt}{140.172638pt}}
\pgflineto{\pgfpoint{271.295990pt}{140.172638pt}}
\pgfpathclose
\pgfusepath{fill,stroke}
\pgfpathmoveto{\pgfpoint{271.295990pt}{146.349472pt}}
\pgflineto{\pgfpoint{280.223969pt}{146.349472pt}}
\pgflineto{\pgfpoint{280.223969pt}{140.172638pt}}
\pgfpathclose
\pgfusepath{fill,stroke}
\color[rgb]{0.127668,0.646882,0.523924}
\pgfpathmoveto{\pgfpoint{271.295990pt}{152.526306pt}}
\pgflineto{\pgfpoint{280.223969pt}{146.349472pt}}
\pgflineto{\pgfpoint{271.295990pt}{146.349472pt}}
\pgfpathclose
\pgfusepath{fill,stroke}
\color[rgb]{0.119627,0.617266,0.536796}
\pgfpathmoveto{\pgfpoint{280.223969pt}{133.995789pt}}
\pgflineto{\pgfpoint{289.151978pt}{127.818947pt}}
\pgflineto{\pgfpoint{280.223969pt}{127.818947pt}}
\pgfpathclose
\pgfusepath{fill,stroke}
\pgfpathmoveto{\pgfpoint{280.223969pt}{133.995789pt}}
\pgflineto{\pgfpoint{289.151978pt}{133.995789pt}}
\pgflineto{\pgfpoint{289.151978pt}{127.818947pt}}
\pgfpathclose
\pgfusepath{fill,stroke}
\color[rgb]{0.122046,0.632107,0.530848}
\pgfpathmoveto{\pgfpoint{280.223969pt}{140.172638pt}}
\pgflineto{\pgfpoint{289.151978pt}{133.995789pt}}
\pgflineto{\pgfpoint{280.223969pt}{133.995789pt}}
\pgfpathclose
\pgfusepath{fill,stroke}
\pgfpathmoveto{\pgfpoint{280.223969pt}{140.172638pt}}
\pgflineto{\pgfpoint{289.151978pt}{140.172638pt}}
\pgflineto{\pgfpoint{289.151978pt}{133.995789pt}}
\pgfpathclose
\pgfusepath{fill,stroke}
\color[rgb]{0.127668,0.646882,0.523924}
\pgfpathmoveto{\pgfpoint{280.223969pt}{146.349472pt}}
\pgflineto{\pgfpoint{289.151978pt}{140.172638pt}}
\pgflineto{\pgfpoint{280.223969pt}{140.172638pt}}
\pgfpathclose
\pgfusepath{fill,stroke}
\color[rgb]{0.122046,0.632107,0.530848}
\pgfpathmoveto{\pgfpoint{289.151978pt}{133.995789pt}}
\pgflineto{\pgfpoint{298.079987pt}{133.995789pt}}
\pgflineto{\pgfpoint{298.079987pt}{127.818947pt}}
\pgfpathclose
\pgfusepath{fill,stroke}
\color[rgb]{0.127668,0.646882,0.523924}
\pgfpathmoveto{\pgfpoint{289.151978pt}{140.172638pt}}
\pgflineto{\pgfpoint{298.079987pt}{133.995789pt}}
\pgflineto{\pgfpoint{289.151978pt}{133.995789pt}}
\pgfpathclose
\pgfusepath{fill,stroke}
\pgfpathmoveto{\pgfpoint{289.151978pt}{140.172638pt}}
\pgflineto{\pgfpoint{298.079987pt}{140.172638pt}}
\pgflineto{\pgfpoint{298.079987pt}{133.995789pt}}
\pgfpathclose
\pgfusepath{fill,stroke}
\pgfpathmoveto{\pgfpoint{298.079987pt}{127.818947pt}}
\pgflineto{\pgfpoint{307.007965pt}{121.642097pt}}
\pgflineto{\pgfpoint{298.079987pt}{121.642097pt}}
\pgfpathclose
\pgfusepath{fill,stroke}
\pgfpathmoveto{\pgfpoint{298.079987pt}{127.818947pt}}
\pgflineto{\pgfpoint{307.007965pt}{127.818947pt}}
\pgflineto{\pgfpoint{307.007965pt}{121.642097pt}}
\pgfpathclose
\pgfusepath{fill,stroke}
\pgfpathmoveto{\pgfpoint{298.079987pt}{133.995789pt}}
\pgflineto{\pgfpoint{307.007965pt}{127.818947pt}}
\pgflineto{\pgfpoint{298.079987pt}{127.818947pt}}
\pgfpathclose
\pgfusepath{fill,stroke}
\pgfpathmoveto{\pgfpoint{298.079987pt}{133.995789pt}}
\pgflineto{\pgfpoint{307.007965pt}{133.995789pt}}
\pgflineto{\pgfpoint{307.007965pt}{127.818947pt}}
\pgfpathclose
\pgfusepath{fill,stroke}
\color[rgb]{0.136835,0.661563,0.515967}
\pgfpathmoveto{\pgfpoint{298.079987pt}{140.172638pt}}
\pgflineto{\pgfpoint{307.007965pt}{133.995789pt}}
\pgflineto{\pgfpoint{298.079987pt}{133.995789pt}}
\pgfpathclose
\pgfusepath{fill,stroke}
\pgfpathmoveto{\pgfpoint{307.007965pt}{127.818947pt}}
\pgflineto{\pgfpoint{315.935974pt}{127.818947pt}}
\pgflineto{\pgfpoint{315.935974pt}{121.642097pt}}
\pgfpathclose
\pgfusepath{fill,stroke}
\color[rgb]{0.149643,0.676120,0.506924}
\pgfpathmoveto{\pgfpoint{307.007965pt}{133.995789pt}}
\pgflineto{\pgfpoint{315.935974pt}{127.818947pt}}
\pgflineto{\pgfpoint{307.007965pt}{127.818947pt}}
\pgfpathclose
\pgfusepath{fill,stroke}
\pgfpathmoveto{\pgfpoint{307.007965pt}{133.995789pt}}
\pgflineto{\pgfpoint{315.935974pt}{133.995789pt}}
\pgflineto{\pgfpoint{315.935974pt}{127.818947pt}}
\pgfpathclose
\pgfusepath{fill,stroke}
\color[rgb]{0.127668,0.646882,0.523924}
\pgfpathmoveto{\pgfpoint{315.935974pt}{115.465263pt}}
\pgflineto{\pgfpoint{324.863983pt}{115.465263pt}}
\pgflineto{\pgfpoint{324.863983pt}{109.288422pt}}
\pgfpathclose
\pgfusepath{fill,stroke}
\color[rgb]{0.136835,0.661563,0.515967}
\pgfpathmoveto{\pgfpoint{315.935974pt}{121.642097pt}}
\pgflineto{\pgfpoint{324.863983pt}{115.465263pt}}
\pgflineto{\pgfpoint{315.935974pt}{115.465263pt}}
\pgfpathclose
\pgfusepath{fill,stroke}
\pgfpathmoveto{\pgfpoint{315.935974pt}{121.642097pt}}
\pgflineto{\pgfpoint{324.863983pt}{121.642097pt}}
\pgflineto{\pgfpoint{324.863983pt}{115.465263pt}}
\pgfpathclose
\pgfusepath{fill,stroke}
\color[rgb]{0.149643,0.676120,0.506924}
\pgfpathmoveto{\pgfpoint{315.935974pt}{127.818947pt}}
\pgflineto{\pgfpoint{324.863983pt}{121.642097pt}}
\pgflineto{\pgfpoint{315.935974pt}{121.642097pt}}
\pgfpathclose
\pgfusepath{fill,stroke}
\pgfpathmoveto{\pgfpoint{315.935974pt}{127.818947pt}}
\pgflineto{\pgfpoint{324.863983pt}{127.818947pt}}
\pgflineto{\pgfpoint{324.863983pt}{121.642097pt}}
\pgfpathclose
\pgfusepath{fill,stroke}
\color[rgb]{0.165967,0.690519,0.496752}
\pgfpathmoveto{\pgfpoint{315.935974pt}{133.995789pt}}
\pgflineto{\pgfpoint{324.863983pt}{127.818947pt}}
\pgflineto{\pgfpoint{315.935974pt}{127.818947pt}}
\pgfpathclose
\pgfusepath{fill,stroke}
\color[rgb]{0.136835,0.661563,0.515967}
\pgfpathmoveto{\pgfpoint{324.863983pt}{115.465263pt}}
\pgflineto{\pgfpoint{333.791992pt}{109.288422pt}}
\pgflineto{\pgfpoint{324.863983pt}{109.288422pt}}
\pgfpathclose
\pgfusepath{fill,stroke}
\color[rgb]{0.149643,0.676120,0.506924}
\pgfpathmoveto{\pgfpoint{324.863983pt}{121.642097pt}}
\pgflineto{\pgfpoint{333.791992pt}{121.642097pt}}
\pgflineto{\pgfpoint{333.791992pt}{115.465263pt}}
\pgfpathclose
\pgfusepath{fill,stroke}
\color[rgb]{0.165967,0.690519,0.496752}
\pgfpathmoveto{\pgfpoint{324.863983pt}{127.818947pt}}
\pgflineto{\pgfpoint{333.791992pt}{121.642097pt}}
\pgflineto{\pgfpoint{324.863983pt}{121.642097pt}}
\pgfpathclose
\pgfusepath{fill,stroke}
\pgfpathmoveto{\pgfpoint{324.863983pt}{127.818947pt}}
\pgflineto{\pgfpoint{333.791992pt}{127.818947pt}}
\pgflineto{\pgfpoint{333.791992pt}{121.642097pt}}
\pgfpathclose
\pgfusepath{fill,stroke}
\color[rgb]{0.136835,0.661563,0.515967}
\pgfpathmoveto{\pgfpoint{333.791992pt}{109.288422pt}}
\pgflineto{\pgfpoint{342.719971pt}{109.288422pt}}
\pgflineto{\pgfpoint{342.719971pt}{103.111580pt}}
\pgfpathclose
\pgfusepath{fill,stroke}
\color[rgb]{0.149643,0.676120,0.506924}
\pgfpathmoveto{\pgfpoint{333.791992pt}{115.465263pt}}
\pgflineto{\pgfpoint{342.719971pt}{109.288422pt}}
\pgflineto{\pgfpoint{333.791992pt}{109.288422pt}}
\pgfpathclose
\pgfusepath{fill,stroke}
\pgfpathmoveto{\pgfpoint{333.791992pt}{115.465263pt}}
\pgflineto{\pgfpoint{342.719971pt}{115.465263pt}}
\pgflineto{\pgfpoint{342.719971pt}{109.288422pt}}
\pgfpathclose
\pgfusepath{fill,stroke}
\color[rgb]{0.165967,0.690519,0.496752}
\pgfpathmoveto{\pgfpoint{333.791992pt}{121.642097pt}}
\pgflineto{\pgfpoint{342.719971pt}{115.465263pt}}
\pgflineto{\pgfpoint{333.791992pt}{115.465263pt}}
\pgfpathclose
\pgfusepath{fill,stroke}
\pgfpathmoveto{\pgfpoint{333.791992pt}{121.642097pt}}
\pgflineto{\pgfpoint{342.719971pt}{121.642097pt}}
\pgflineto{\pgfpoint{342.719971pt}{115.465263pt}}
\pgfpathclose
\pgfusepath{fill,stroke}
\color[rgb]{0.185538,0.704725,0.485412}
\pgfpathmoveto{\pgfpoint{333.791992pt}{127.818947pt}}
\pgflineto{\pgfpoint{342.719971pt}{121.642097pt}}
\pgflineto{\pgfpoint{333.791992pt}{121.642097pt}}
\pgfpathclose
\pgfusepath{fill,stroke}
\color[rgb]{0.149643,0.676120,0.506924}
\pgfpathmoveto{\pgfpoint{342.719971pt}{109.288422pt}}
\pgflineto{\pgfpoint{351.647980pt}{103.111580pt}}
\pgflineto{\pgfpoint{342.719971pt}{103.111580pt}}
\pgfpathclose
\pgfusepath{fill,stroke}
\color[rgb]{0.165967,0.690519,0.496752}
\pgfpathmoveto{\pgfpoint{342.719971pt}{115.465263pt}}
\pgflineto{\pgfpoint{351.647980pt}{115.465263pt}}
\pgflineto{\pgfpoint{351.647980pt}{109.288422pt}}
\pgfpathclose
\pgfusepath{fill,stroke}
\color[rgb]{0.185538,0.704725,0.485412}
\pgfpathmoveto{\pgfpoint{342.719971pt}{121.642097pt}}
\pgflineto{\pgfpoint{351.647980pt}{115.465263pt}}
\pgflineto{\pgfpoint{342.719971pt}{115.465263pt}}
\pgfpathclose
\pgfusepath{fill,stroke}
\pgfpathmoveto{\pgfpoint{342.719971pt}{121.642097pt}}
\pgflineto{\pgfpoint{351.647980pt}{121.642097pt}}
\pgflineto{\pgfpoint{351.647980pt}{115.465263pt}}
\pgfpathclose
\pgfusepath{fill,stroke}
\pgfpathmoveto{\pgfpoint{351.647980pt}{109.288422pt}}
\pgflineto{\pgfpoint{360.575958pt}{109.288422pt}}
\pgflineto{\pgfpoint{360.575958pt}{103.111580pt}}
\pgfpathclose
\pgfusepath{fill,stroke}
\pgfpathmoveto{\pgfpoint{351.647980pt}{115.465263pt}}
\pgflineto{\pgfpoint{360.575958pt}{109.288422pt}}
\pgflineto{\pgfpoint{351.647980pt}{109.288422pt}}
\pgfpathclose
\pgfusepath{fill,stroke}
\pgfpathmoveto{\pgfpoint{351.647980pt}{115.465263pt}}
\pgflineto{\pgfpoint{360.575958pt}{115.465263pt}}
\pgflineto{\pgfpoint{360.575958pt}{109.288422pt}}
\pgfpathclose
\pgfusepath{fill,stroke}
\color[rgb]{0.208030,0.718701,0.472873}
\pgfpathmoveto{\pgfpoint{351.647980pt}{121.642097pt}}
\pgflineto{\pgfpoint{360.575958pt}{115.465263pt}}
\pgflineto{\pgfpoint{351.647980pt}{115.465263pt}}
\pgfpathclose
\pgfusepath{fill,stroke}
\color[rgb]{0.185538,0.704725,0.485412}
\pgfpathmoveto{\pgfpoint{360.575958pt}{103.111580pt}}
\pgflineto{\pgfpoint{369.503998pt}{96.934731pt}}
\pgflineto{\pgfpoint{360.575958pt}{96.934731pt}}
\pgfpathclose
\pgfusepath{fill,stroke}
\pgfpathmoveto{\pgfpoint{360.575958pt}{103.111580pt}}
\pgflineto{\pgfpoint{369.503998pt}{103.111580pt}}
\pgflineto{\pgfpoint{369.503998pt}{96.934731pt}}
\pgfpathclose
\pgfusepath{fill,stroke}
\color[rgb]{0.208030,0.718701,0.472873}
\pgfpathmoveto{\pgfpoint{360.575958pt}{109.288422pt}}
\pgflineto{\pgfpoint{369.503998pt}{103.111580pt}}
\pgflineto{\pgfpoint{360.575958pt}{103.111580pt}}
\pgfpathclose
\pgfusepath{fill,stroke}
\pgfpathmoveto{\pgfpoint{360.575958pt}{109.288422pt}}
\pgflineto{\pgfpoint{369.503998pt}{109.288422pt}}
\pgflineto{\pgfpoint{369.503998pt}{103.111580pt}}
\pgfpathclose
\pgfusepath{fill,stroke}
\pgfpathmoveto{\pgfpoint{360.575958pt}{115.465263pt}}
\pgflineto{\pgfpoint{369.503998pt}{109.288422pt}}
\pgflineto{\pgfpoint{360.575958pt}{109.288422pt}}
\pgfpathclose
\pgfusepath{fill,stroke}
\pgfpathmoveto{\pgfpoint{369.503998pt}{103.111580pt}}
\pgflineto{\pgfpoint{378.431976pt}{103.111580pt}}
\pgflineto{\pgfpoint{378.431976pt}{96.934731pt}}
\pgfpathclose
\pgfusepath{fill,stroke}
\color[rgb]{0.233127,0.732406,0.459106}
\pgfpathmoveto{\pgfpoint{369.503998pt}{109.288422pt}}
\pgflineto{\pgfpoint{378.431976pt}{103.111580pt}}
\pgflineto{\pgfpoint{369.503998pt}{103.111580pt}}
\pgfpathclose
\pgfusepath{fill,stroke}
\pgfpathmoveto{\pgfpoint{369.503998pt}{109.288422pt}}
\pgflineto{\pgfpoint{378.431976pt}{109.288422pt}}
\pgflineto{\pgfpoint{378.431976pt}{103.111580pt}}
\pgfpathclose
\pgfusepath{fill,stroke}
\color[rgb]{0.185538,0.704725,0.485412}
\pgfpathmoveto{\pgfpoint{378.431976pt}{90.757896pt}}
\pgflineto{\pgfpoint{387.359985pt}{90.757896pt}}
\pgflineto{\pgfpoint{387.359985pt}{84.581039pt}}
\pgfpathclose
\pgfusepath{fill,stroke}
\color[rgb]{0.233127,0.732406,0.459106}
\pgfpathmoveto{\pgfpoint{378.431976pt}{103.111580pt}}
\pgflineto{\pgfpoint{387.359985pt}{96.934731pt}}
\pgflineto{\pgfpoint{378.431976pt}{96.934731pt}}
\pgfpathclose
\pgfusepath{fill,stroke}
\pgfpathmoveto{\pgfpoint{378.431976pt}{103.111580pt}}
\pgflineto{\pgfpoint{387.359985pt}{103.111580pt}}
\pgflineto{\pgfpoint{387.359985pt}{96.934731pt}}
\pgfpathclose
\pgfusepath{fill,stroke}
\color[rgb]{0.260531,0.745802,0.444096}
\pgfpathmoveto{\pgfpoint{378.431976pt}{109.288422pt}}
\pgflineto{\pgfpoint{387.359985pt}{103.111580pt}}
\pgflineto{\pgfpoint{378.431976pt}{103.111580pt}}
\pgfpathclose
\pgfusepath{fill,stroke}
\color[rgb]{0.208030,0.718701,0.472873}
\pgfpathmoveto{\pgfpoint{387.359985pt}{90.757896pt}}
\pgflineto{\pgfpoint{396.287964pt}{84.581039pt}}
\pgflineto{\pgfpoint{387.359985pt}{84.581039pt}}
\pgfpathclose
\pgfusepath{fill,stroke}
\color[rgb]{0.233127,0.732406,0.459106}
\pgfpathmoveto{\pgfpoint{387.359985pt}{96.934731pt}}
\pgflineto{\pgfpoint{396.287964pt}{96.934731pt}}
\pgflineto{\pgfpoint{396.287964pt}{90.757896pt}}
\pgfpathclose
\pgfusepath{fill,stroke}
\color[rgb]{0.260531,0.745802,0.444096}
\pgfpathmoveto{\pgfpoint{387.359985pt}{103.111580pt}}
\pgflineto{\pgfpoint{396.287964pt}{96.934731pt}}
\pgflineto{\pgfpoint{387.359985pt}{96.934731pt}}
\pgfpathclose
\pgfusepath{fill,stroke}
\color[rgb]{0.233127,0.732406,0.459106}
\pgfpathmoveto{\pgfpoint{396.287964pt}{84.581039pt}}
\pgflineto{\pgfpoint{405.216003pt}{84.581039pt}}
\pgflineto{\pgfpoint{405.216003pt}{78.404205pt}}
\pgfpathclose
\pgfusepath{fill,stroke}
\color[rgb]{0.260531,0.745802,0.444096}
\pgfpathmoveto{\pgfpoint{396.287964pt}{96.934731pt}}
\pgflineto{\pgfpoint{405.216003pt}{90.757896pt}}
\pgflineto{\pgfpoint{396.287964pt}{90.757896pt}}
\pgfpathclose
\pgfusepath{fill,stroke}
\pgfpathmoveto{\pgfpoint{396.287964pt}{96.934731pt}}
\pgflineto{\pgfpoint{405.216003pt}{96.934731pt}}
\pgflineto{\pgfpoint{405.216003pt}{90.757896pt}}
\pgfpathclose
\pgfusepath{fill,stroke}
\pgfpathmoveto{\pgfpoint{405.216003pt}{84.581039pt}}
\pgflineto{\pgfpoint{414.143982pt}{78.404205pt}}
\pgflineto{\pgfpoint{405.216003pt}{78.404205pt}}
\pgfpathclose
\pgfusepath{fill,stroke}
\color[rgb]{0.290001,0.758846,0.427826}
\pgfpathmoveto{\pgfpoint{405.216003pt}{90.757896pt}}
\pgflineto{\pgfpoint{414.143982pt}{90.757896pt}}
\pgflineto{\pgfpoint{414.143982pt}{84.581039pt}}
\pgfpathclose
\pgfusepath{fill,stroke}
\pgfpathmoveto{\pgfpoint{405.216003pt}{96.934731pt}}
\pgflineto{\pgfpoint{414.143982pt}{90.757896pt}}
\pgflineto{\pgfpoint{405.216003pt}{90.757896pt}}
\pgfpathclose
\pgfusepath{fill,stroke}
\color[rgb]{0.260531,0.745802,0.444096}
\pgfpathmoveto{\pgfpoint{414.143982pt}{78.404205pt}}
\pgflineto{\pgfpoint{423.071960pt}{78.404205pt}}
\pgflineto{\pgfpoint{423.071960pt}{72.227356pt}}
\pgfpathclose
\pgfusepath{fill,stroke}
\color[rgb]{0.290001,0.758846,0.427826}
\pgfpathmoveto{\pgfpoint{414.143982pt}{84.581039pt}}
\pgflineto{\pgfpoint{423.071960pt}{84.581039pt}}
\pgflineto{\pgfpoint{423.071960pt}{78.404205pt}}
\pgfpathclose
\pgfusepath{fill,stroke}
\color[rgb]{0.321330,0.771498,0.410293}
\pgfpathmoveto{\pgfpoint{414.143982pt}{90.757896pt}}
\pgflineto{\pgfpoint{423.071960pt}{84.581039pt}}
\pgflineto{\pgfpoint{414.143982pt}{84.581039pt}}
\pgfpathclose
\pgfusepath{fill,stroke}
\pgfpathmoveto{\pgfpoint{414.143982pt}{90.757896pt}}
\pgflineto{\pgfpoint{423.071960pt}{90.757896pt}}
\pgflineto{\pgfpoint{423.071960pt}{84.581039pt}}
\pgfpathclose
\pgfusepath{fill,stroke}
\color[rgb]{0.290001,0.758846,0.427826}
\pgfpathmoveto{\pgfpoint{423.071960pt}{78.404205pt}}
\pgflineto{\pgfpoint{432.000000pt}{72.227356pt}}
\pgflineto{\pgfpoint{423.071960pt}{72.227356pt}}
\pgfpathclose
\pgfusepath{fill,stroke}
\color[rgb]{0.321330,0.771498,0.410293}
\pgfpathmoveto{\pgfpoint{423.071960pt}{84.581039pt}}
\pgflineto{\pgfpoint{432.000000pt}{78.404205pt}}
\pgflineto{\pgfpoint{423.071960pt}{78.404205pt}}
\pgfpathclose
\pgfusepath{fill,stroke}
\pgfpathmoveto{\pgfpoint{423.071960pt}{84.581039pt}}
\pgflineto{\pgfpoint{432.000000pt}{84.581039pt}}
\pgflineto{\pgfpoint{432.000000pt}{78.404205pt}}
\pgfpathclose
\pgfusepath{fill,stroke}
\color[rgb]{0.354355,0.783714,0.391488}
\pgfpathmoveto{\pgfpoint{423.071960pt}{90.757896pt}}
\pgflineto{\pgfpoint{432.000000pt}{84.581039pt}}
\pgflineto{\pgfpoint{423.071960pt}{84.581039pt}}
\pgfpathclose
\pgfusepath{fill,stroke}
\color[rgb]{0.290001,0.758846,0.427826}
\pgfpathmoveto{\pgfpoint{432.000000pt}{72.227356pt}}
\pgflineto{\pgfpoint{440.927979pt}{72.227356pt}}
\pgflineto{\pgfpoint{440.927979pt}{66.050522pt}}
\pgfpathclose
\pgfusepath{fill,stroke}
\color[rgb]{0.321330,0.771498,0.410293}
\pgfpathmoveto{\pgfpoint{432.000000pt}{78.404205pt}}
\pgflineto{\pgfpoint{440.927979pt}{78.404205pt}}
\pgflineto{\pgfpoint{440.927979pt}{72.227356pt}}
\pgfpathclose
\pgfusepath{fill,stroke}
\color[rgb]{0.354355,0.783714,0.391488}
\pgfpathmoveto{\pgfpoint{432.000000pt}{84.581039pt}}
\pgflineto{\pgfpoint{440.927979pt}{78.404205pt}}
\pgflineto{\pgfpoint{432.000000pt}{78.404205pt}}
\pgfpathclose
\pgfusepath{fill,stroke}
\pgfpathmoveto{\pgfpoint{432.000000pt}{84.581039pt}}
\pgflineto{\pgfpoint{440.927979pt}{84.581039pt}}
\pgflineto{\pgfpoint{440.927979pt}{78.404205pt}}
\pgfpathclose
\pgfusepath{fill,stroke}
\color[rgb]{0.321330,0.771498,0.410293}
\pgfpathmoveto{\pgfpoint{440.927979pt}{72.227356pt}}
\pgflineto{\pgfpoint{449.855957pt}{66.050522pt}}
\pgflineto{\pgfpoint{440.927979pt}{66.050522pt}}
\pgfpathclose
\pgfusepath{fill,stroke}
\pgfpathmoveto{\pgfpoint{440.927979pt}{72.227356pt}}
\pgflineto{\pgfpoint{449.855957pt}{72.227356pt}}
\pgflineto{\pgfpoint{449.855957pt}{66.050522pt}}
\pgfpathclose
\pgfusepath{fill,stroke}
\color[rgb]{0.354355,0.783714,0.391488}
\pgfpathmoveto{\pgfpoint{440.927979pt}{78.404205pt}}
\pgflineto{\pgfpoint{449.855957pt}{72.227356pt}}
\pgflineto{\pgfpoint{440.927979pt}{72.227356pt}}
\pgfpathclose
\pgfusepath{fill,stroke}
\pgfpathmoveto{\pgfpoint{440.927979pt}{78.404205pt}}
\pgflineto{\pgfpoint{449.855957pt}{78.404205pt}}
\pgflineto{\pgfpoint{449.855957pt}{72.227356pt}}
\pgfpathclose
\pgfusepath{fill,stroke}
\color[rgb]{0.388930,0.795453,0.371421}
\pgfpathmoveto{\pgfpoint{440.927979pt}{84.581039pt}}
\pgflineto{\pgfpoint{449.855957pt}{78.404205pt}}
\pgflineto{\pgfpoint{440.927979pt}{78.404205pt}}
\pgfpathclose
\pgfusepath{fill,stroke}
\color[rgb]{0.354355,0.783714,0.391488}
\pgfpathmoveto{\pgfpoint{449.855957pt}{66.050522pt}}
\pgflineto{\pgfpoint{458.783936pt}{66.050522pt}}
\pgflineto{\pgfpoint{458.783936pt}{59.873672pt}}
\pgfpathclose
\pgfusepath{fill,stroke}
\pgfpathmoveto{\pgfpoint{449.855957pt}{72.227356pt}}
\pgflineto{\pgfpoint{458.783936pt}{66.050522pt}}
\pgflineto{\pgfpoint{449.855957pt}{66.050522pt}}
\pgfpathclose
\pgfusepath{fill,stroke}
\pgfpathmoveto{\pgfpoint{449.855957pt}{72.227356pt}}
\pgflineto{\pgfpoint{458.783936pt}{72.227356pt}}
\pgflineto{\pgfpoint{458.783936pt}{66.050522pt}}
\pgfpathclose
\pgfusepath{fill,stroke}
\color[rgb]{0.388930,0.795453,0.371421}
\pgfpathmoveto{\pgfpoint{449.855957pt}{78.404205pt}}
\pgflineto{\pgfpoint{458.783936pt}{72.227356pt}}
\pgflineto{\pgfpoint{449.855957pt}{72.227356pt}}
\pgfpathclose
\pgfusepath{fill,stroke}
\pgfpathmoveto{\pgfpoint{458.783936pt}{66.050522pt}}
\pgflineto{\pgfpoint{467.711975pt}{59.873672pt}}
\pgflineto{\pgfpoint{458.783936pt}{59.873672pt}}
\pgfpathclose
\pgfusepath{fill,stroke}
\pgfpathmoveto{\pgfpoint{458.783936pt}{66.050522pt}}
\pgflineto{\pgfpoint{467.711975pt}{66.050522pt}}
\pgflineto{\pgfpoint{467.711975pt}{59.873672pt}}
\pgfpathclose
\pgfusepath{fill,stroke}
\pgfpathmoveto{\pgfpoint{458.783936pt}{72.227356pt}}
\pgflineto{\pgfpoint{467.711975pt}{66.050522pt}}
\pgflineto{\pgfpoint{458.783936pt}{66.050522pt}}
\pgfpathclose
\pgfusepath{fill,stroke}
\pgfpathmoveto{\pgfpoint{458.783936pt}{72.227356pt}}
\pgflineto{\pgfpoint{467.711975pt}{72.227356pt}}
\pgflineto{\pgfpoint{467.711975pt}{66.050522pt}}
\pgfpathclose
\pgfusepath{fill,stroke}
\color[rgb]{0.424933,0.806674,0.350099}
\pgfpathmoveto{\pgfpoint{467.711975pt}{66.050522pt}}
\pgflineto{\pgfpoint{476.639954pt}{59.873672pt}}
\pgflineto{\pgfpoint{467.711975pt}{59.873672pt}}
\pgfpathclose
\pgfusepath{fill,stroke}
\pgfpathmoveto{\pgfpoint{467.711975pt}{66.050522pt}}
\pgflineto{\pgfpoint{476.639954pt}{66.050522pt}}
\pgflineto{\pgfpoint{476.639954pt}{59.873672pt}}
\pgfpathclose
\pgfusepath{fill,stroke}
\color[rgb]{0.462247,0.817338,0.327545}
\pgfpathmoveto{\pgfpoint{467.711975pt}{72.227356pt}}
\pgflineto{\pgfpoint{476.639954pt}{66.050522pt}}
\pgflineto{\pgfpoint{467.711975pt}{66.050522pt}}
\pgfpathclose
\pgfusepath{fill,stroke}
\color[rgb]{0.424933,0.806674,0.350099}
\pgfpathmoveto{\pgfpoint{476.639954pt}{59.873672pt}}
\pgflineto{\pgfpoint{485.567963pt}{59.873672pt}}
\pgflineto{\pgfpoint{485.567963pt}{53.696838pt}}
\pgfpathclose
\pgfusepath{fill,stroke}
\color[rgb]{0.462247,0.817338,0.327545}
\pgfpathmoveto{\pgfpoint{476.639954pt}{66.050522pt}}
\pgflineto{\pgfpoint{485.567963pt}{59.873672pt}}
\pgflineto{\pgfpoint{476.639954pt}{59.873672pt}}
\pgfpathclose
\pgfusepath{fill,stroke}
\pgfpathmoveto{\pgfpoint{476.639954pt}{66.050522pt}}
\pgflineto{\pgfpoint{485.567963pt}{66.050522pt}}
\pgflineto{\pgfpoint{485.567963pt}{59.873672pt}}
\pgfpathclose
\pgfusepath{fill,stroke}
\pgfpathmoveto{\pgfpoint{485.567963pt}{59.873672pt}}
\pgflineto{\pgfpoint{494.495972pt}{53.696838pt}}
\pgflineto{\pgfpoint{485.567963pt}{53.696838pt}}
\pgfpathclose
\pgfusepath{fill,stroke}
\pgfpathmoveto{\pgfpoint{485.567963pt}{59.873672pt}}
\pgflineto{\pgfpoint{494.495972pt}{59.873672pt}}
\pgflineto{\pgfpoint{494.495972pt}{53.696838pt}}
\pgfpathclose
\pgfusepath{fill,stroke}
\color[rgb]{0.500754,0.827409,0.303799}
\pgfpathmoveto{\pgfpoint{485.567963pt}{66.050522pt}}
\pgflineto{\pgfpoint{494.495972pt}{59.873672pt}}
\pgflineto{\pgfpoint{485.567963pt}{59.873672pt}}
\pgfpathclose
\pgfusepath{fill,stroke}
\color[rgb]{0.462247,0.817338,0.327545}
\pgfpathmoveto{\pgfpoint{494.495972pt}{53.696838pt}}
\pgflineto{\pgfpoint{503.423981pt}{53.696838pt}}
\pgflineto{\pgfpoint{503.423981pt}{47.519989pt}}
\pgfpathclose
\pgfusepath{fill,stroke}
\color[rgb]{0.500754,0.827409,0.303799}
\pgfpathmoveto{\pgfpoint{494.495972pt}{59.873672pt}}
\pgflineto{\pgfpoint{503.423981pt}{53.696838pt}}
\pgflineto{\pgfpoint{494.495972pt}{53.696838pt}}
\pgfpathclose
\pgfusepath{fill,stroke}
\pgfpathmoveto{\pgfpoint{494.495972pt}{59.873672pt}}
\pgflineto{\pgfpoint{503.423981pt}{59.873672pt}}
\pgflineto{\pgfpoint{503.423981pt}{53.696838pt}}
\pgfpathclose
\pgfusepath{fill,stroke}
\pgfpathmoveto{\pgfpoint{503.423981pt}{53.696838pt}}
\pgflineto{\pgfpoint{512.351990pt}{47.519989pt}}
\pgflineto{\pgfpoint{503.423981pt}{47.519989pt}}
\pgfpathclose
\pgfusepath{fill,stroke}
\pgfpathmoveto{\pgfpoint{503.423981pt}{53.696838pt}}
\pgflineto{\pgfpoint{512.351990pt}{53.696838pt}}
\pgflineto{\pgfpoint{512.351990pt}{47.519989pt}}
\pgfpathclose
\pgfusepath{fill,stroke}
\color[rgb]{0.540337,0.836858,0.278917}
\pgfpathmoveto{\pgfpoint{503.423981pt}{59.873672pt}}
\pgflineto{\pgfpoint{512.351990pt}{53.696838pt}}
\pgflineto{\pgfpoint{503.423981pt}{53.696838pt}}
\pgfpathclose
\pgfusepath{fill,stroke}
\pgfpathmoveto{\pgfpoint{512.351990pt}{53.696838pt}}
\pgflineto{\pgfpoint{521.279968pt}{47.519989pt}}
\pgflineto{\pgfpoint{512.351990pt}{47.519989pt}}
\pgfpathclose
\pgfusepath{fill,stroke}
\color[rgb]{0.125898,0.572563,0.549445}
\pgfpathmoveto{\pgfpoint{199.871979pt}{183.410522pt}}
\pgflineto{\pgfpoint{208.799988pt}{177.233673pt}}
\pgflineto{\pgfpoint{199.871979pt}{177.233673pt}}
\pgfpathclose
\pgfusepath{fill,stroke}
\pgfpathmoveto{\pgfpoint{199.871979pt}{183.410522pt}}
\pgflineto{\pgfpoint{208.799988pt}{183.410522pt}}
\pgflineto{\pgfpoint{208.799988pt}{177.233673pt}}
\pgfpathclose
\pgfusepath{fill,stroke}
\color[rgb]{0.122163,0.587476,0.546023}
\pgfpathmoveto{\pgfpoint{208.799988pt}{177.233673pt}}
\pgflineto{\pgfpoint{217.727982pt}{177.233673pt}}
\pgflineto{\pgfpoint{217.727982pt}{171.056854pt}}
\pgfpathclose
\pgfusepath{fill,stroke}
\pgfpathmoveto{\pgfpoint{208.799988pt}{183.410522pt}}
\pgflineto{\pgfpoint{217.727982pt}{177.233673pt}}
\pgflineto{\pgfpoint{208.799988pt}{177.233673pt}}
\pgfpathclose
\pgfusepath{fill,stroke}
\pgfpathmoveto{\pgfpoint{208.799988pt}{183.410522pt}}
\pgflineto{\pgfpoint{217.727982pt}{183.410522pt}}
\pgflineto{\pgfpoint{217.727982pt}{177.233673pt}}
\pgfpathclose
\pgfusepath{fill,stroke}
\color[rgb]{0.119872,0.602382,0.541831}
\pgfpathmoveto{\pgfpoint{208.799988pt}{189.587372pt}}
\pgflineto{\pgfpoint{217.727982pt}{183.410522pt}}
\pgflineto{\pgfpoint{208.799988pt}{183.410522pt}}
\pgfpathclose
\pgfusepath{fill,stroke}
\color[rgb]{0.119627,0.617266,0.536796}
\pgfpathmoveto{\pgfpoint{244.511993pt}{164.880005pt}}
\pgflineto{\pgfpoint{253.440002pt}{158.703156pt}}
\pgflineto{\pgfpoint{244.511993pt}{158.703156pt}}
\pgfpathclose
\pgfusepath{fill,stroke}
\pgfpathmoveto{\pgfpoint{244.511993pt}{164.880005pt}}
\pgflineto{\pgfpoint{253.440002pt}{164.880005pt}}
\pgflineto{\pgfpoint{253.440002pt}{158.703156pt}}
\pgfpathclose
\pgfusepath{fill,stroke}
\color[rgb]{0.122046,0.632107,0.530848}
\pgfpathmoveto{\pgfpoint{244.511993pt}{171.056854pt}}
\pgflineto{\pgfpoint{253.440002pt}{171.056854pt}}
\pgflineto{\pgfpoint{253.440002pt}{164.880005pt}}
\pgfpathclose
\pgfusepath{fill,stroke}
\color[rgb]{0.127668,0.646882,0.523924}
\pgfpathmoveto{\pgfpoint{244.511993pt}{177.233673pt}}
\pgflineto{\pgfpoint{253.440002pt}{171.056854pt}}
\pgflineto{\pgfpoint{244.511993pt}{171.056854pt}}
\pgfpathclose
\pgfusepath{fill,stroke}
\color[rgb]{0.119627,0.617266,0.536796}
\pgfpathmoveto{\pgfpoint{253.440002pt}{158.703156pt}}
\pgflineto{\pgfpoint{262.367981pt}{158.703156pt}}
\pgflineto{\pgfpoint{262.367981pt}{152.526306pt}}
\pgfpathclose
\pgfusepath{fill,stroke}
\color[rgb]{0.122046,0.632107,0.530848}
\pgfpathmoveto{\pgfpoint{253.440002pt}{164.880005pt}}
\pgflineto{\pgfpoint{262.367981pt}{158.703156pt}}
\pgflineto{\pgfpoint{253.440002pt}{158.703156pt}}
\pgfpathclose
\pgfusepath{fill,stroke}
\pgfpathmoveto{\pgfpoint{253.440002pt}{164.880005pt}}
\pgflineto{\pgfpoint{262.367981pt}{164.880005pt}}
\pgflineto{\pgfpoint{262.367981pt}{158.703156pt}}
\pgfpathclose
\pgfusepath{fill,stroke}
\color[rgb]{0.127668,0.646882,0.523924}
\pgfpathmoveto{\pgfpoint{253.440002pt}{171.056854pt}}
\pgflineto{\pgfpoint{262.367981pt}{164.880005pt}}
\pgflineto{\pgfpoint{253.440002pt}{164.880005pt}}
\pgfpathclose
\pgfusepath{fill,stroke}
\color[rgb]{0.122046,0.632107,0.530848}
\pgfpathmoveto{\pgfpoint{262.367981pt}{158.703156pt}}
\pgflineto{\pgfpoint{271.295990pt}{152.526306pt}}
\pgflineto{\pgfpoint{262.367981pt}{152.526306pt}}
\pgfpathclose
\pgfusepath{fill,stroke}
\pgfpathmoveto{\pgfpoint{262.367981pt}{158.703156pt}}
\pgflineto{\pgfpoint{271.295990pt}{158.703156pt}}
\pgflineto{\pgfpoint{271.295990pt}{152.526306pt}}
\pgfpathclose
\pgfusepath{fill,stroke}
\color[rgb]{0.149643,0.676120,0.506924}
\pgfpathmoveto{\pgfpoint{307.007965pt}{140.172638pt}}
\pgflineto{\pgfpoint{315.935974pt}{133.995789pt}}
\pgflineto{\pgfpoint{307.007965pt}{133.995789pt}}
\pgfpathclose
\pgfusepath{fill,stroke}
\pgfpathmoveto{\pgfpoint{307.007965pt}{140.172638pt}}
\pgflineto{\pgfpoint{315.935974pt}{140.172638pt}}
\pgflineto{\pgfpoint{315.935974pt}{133.995789pt}}
\pgfpathclose
\pgfusepath{fill,stroke}
\color[rgb]{0.165967,0.690519,0.496752}
\pgfpathmoveto{\pgfpoint{307.007965pt}{146.349472pt}}
\pgflineto{\pgfpoint{315.935974pt}{146.349472pt}}
\pgflineto{\pgfpoint{315.935974pt}{140.172638pt}}
\pgfpathclose
\pgfusepath{fill,stroke}
\color[rgb]{0.185538,0.704725,0.485412}
\pgfpathmoveto{\pgfpoint{307.007965pt}{152.526306pt}}
\pgflineto{\pgfpoint{315.935974pt}{146.349472pt}}
\pgflineto{\pgfpoint{307.007965pt}{146.349472pt}}
\pgfpathclose
\pgfusepath{fill,stroke}
\color[rgb]{0.165967,0.690519,0.496752}
\pgfpathmoveto{\pgfpoint{315.935974pt}{133.995789pt}}
\pgflineto{\pgfpoint{324.863983pt}{133.995789pt}}
\pgflineto{\pgfpoint{324.863983pt}{127.818947pt}}
\pgfpathclose
\pgfusepath{fill,stroke}
\pgfpathmoveto{\pgfpoint{315.935974pt}{140.172638pt}}
\pgflineto{\pgfpoint{324.863983pt}{133.995789pt}}
\pgflineto{\pgfpoint{315.935974pt}{133.995789pt}}
\pgfpathclose
\pgfusepath{fill,stroke}
\pgfpathmoveto{\pgfpoint{315.935974pt}{140.172638pt}}
\pgflineto{\pgfpoint{324.863983pt}{140.172638pt}}
\pgflineto{\pgfpoint{324.863983pt}{133.995789pt}}
\pgfpathclose
\pgfusepath{fill,stroke}
\color[rgb]{0.185538,0.704725,0.485412}
\pgfpathmoveto{\pgfpoint{315.935974pt}{146.349472pt}}
\pgflineto{\pgfpoint{324.863983pt}{140.172638pt}}
\pgflineto{\pgfpoint{315.935974pt}{140.172638pt}}
\pgfpathclose
\pgfusepath{fill,stroke}
\pgfpathmoveto{\pgfpoint{324.863983pt}{133.995789pt}}
\pgflineto{\pgfpoint{333.791992pt}{127.818947pt}}
\pgflineto{\pgfpoint{324.863983pt}{127.818947pt}}
\pgfpathclose
\pgfusepath{fill,stroke}
\pgfpathmoveto{\pgfpoint{324.863983pt}{133.995789pt}}
\pgflineto{\pgfpoint{333.791992pt}{133.995789pt}}
\pgflineto{\pgfpoint{333.791992pt}{127.818947pt}}
\pgfpathclose
\pgfusepath{fill,stroke}
\color[rgb]{0.260531,0.745802,0.444096}
\pgfpathmoveto{\pgfpoint{369.503998pt}{115.465263pt}}
\pgflineto{\pgfpoint{378.431976pt}{109.288422pt}}
\pgflineto{\pgfpoint{369.503998pt}{109.288422pt}}
\pgfpathclose
\pgfusepath{fill,stroke}
\pgfpathmoveto{\pgfpoint{369.503998pt}{115.465263pt}}
\pgflineto{\pgfpoint{378.431976pt}{115.465263pt}}
\pgflineto{\pgfpoint{378.431976pt}{109.288422pt}}
\pgfpathclose
\pgfusepath{fill,stroke}
\pgfpathmoveto{\pgfpoint{369.503998pt}{121.642097pt}}
\pgflineto{\pgfpoint{378.431976pt}{121.642097pt}}
\pgflineto{\pgfpoint{378.431976pt}{115.465263pt}}
\pgfpathclose
\pgfusepath{fill,stroke}
\color[rgb]{0.290001,0.758846,0.427826}
\pgfpathmoveto{\pgfpoint{369.503998pt}{127.818947pt}}
\pgflineto{\pgfpoint{378.431976pt}{121.642097pt}}
\pgflineto{\pgfpoint{369.503998pt}{121.642097pt}}
\pgfpathclose
\pgfusepath{fill,stroke}
\color[rgb]{0.260531,0.745802,0.444096}
\pgfpathmoveto{\pgfpoint{378.431976pt}{109.288422pt}}
\pgflineto{\pgfpoint{387.359985pt}{109.288422pt}}
\pgflineto{\pgfpoint{387.359985pt}{103.111580pt}}
\pgfpathclose
\pgfusepath{fill,stroke}
\color[rgb]{0.290001,0.758846,0.427826}
\pgfpathmoveto{\pgfpoint{378.431976pt}{115.465263pt}}
\pgflineto{\pgfpoint{387.359985pt}{109.288422pt}}
\pgflineto{\pgfpoint{378.431976pt}{109.288422pt}}
\pgfpathclose
\pgfusepath{fill,stroke}
\pgfpathmoveto{\pgfpoint{378.431976pt}{115.465263pt}}
\pgflineto{\pgfpoint{387.359985pt}{115.465263pt}}
\pgflineto{\pgfpoint{387.359985pt}{109.288422pt}}
\pgfpathclose
\pgfusepath{fill,stroke}
\pgfpathmoveto{\pgfpoint{378.431976pt}{121.642097pt}}
\pgflineto{\pgfpoint{387.359985pt}{115.465263pt}}
\pgflineto{\pgfpoint{378.431976pt}{115.465263pt}}
\pgfpathclose
\pgfusepath{fill,stroke}
\color[rgb]{0.321330,0.771498,0.410293}
\pgfpathmoveto{\pgfpoint{387.359985pt}{115.465263pt}}
\pgflineto{\pgfpoint{396.287964pt}{109.288422pt}}
\pgflineto{\pgfpoint{387.359985pt}{109.288422pt}}
\pgfpathclose
\pgfusepath{fill,stroke}
\pgfpathmoveto{\pgfpoint{387.359985pt}{115.465263pt}}
\pgflineto{\pgfpoint{396.287964pt}{115.465263pt}}
\pgflineto{\pgfpoint{396.287964pt}{109.288422pt}}
\pgfpathclose
\pgfusepath{fill,stroke}
\color[rgb]{0.354355,0.783714,0.391488}
\pgfpathmoveto{\pgfpoint{396.287964pt}{115.465263pt}}
\pgflineto{\pgfpoint{405.216003pt}{109.288422pt}}
\pgflineto{\pgfpoint{396.287964pt}{109.288422pt}}
\pgfpathclose
\pgfusepath{fill,stroke}
\color[rgb]{0.260531,0.745802,0.444096}
\pgfpathmoveto{\pgfpoint{387.359985pt}{103.111580pt}}
\pgflineto{\pgfpoint{396.287964pt}{103.111580pt}}
\pgflineto{\pgfpoint{396.287964pt}{96.934731pt}}
\pgfpathclose
\pgfusepath{fill,stroke}
\color[rgb]{0.290001,0.758846,0.427826}
\pgfpathmoveto{\pgfpoint{387.359985pt}{109.288422pt}}
\pgflineto{\pgfpoint{396.287964pt}{103.111580pt}}
\pgflineto{\pgfpoint{387.359985pt}{103.111580pt}}
\pgfpathclose
\pgfusepath{fill,stroke}
\pgfpathmoveto{\pgfpoint{387.359985pt}{109.288422pt}}
\pgflineto{\pgfpoint{396.287964pt}{109.288422pt}}
\pgflineto{\pgfpoint{396.287964pt}{103.111580pt}}
\pgfpathclose
\pgfusepath{fill,stroke}
\pgfpathmoveto{\pgfpoint{396.287964pt}{103.111580pt}}
\pgflineto{\pgfpoint{405.216003pt}{96.934731pt}}
\pgflineto{\pgfpoint{396.287964pt}{96.934731pt}}
\pgfpathclose
\pgfusepath{fill,stroke}
\pgfpathmoveto{\pgfpoint{396.287964pt}{103.111580pt}}
\pgflineto{\pgfpoint{405.216003pt}{103.111580pt}}
\pgflineto{\pgfpoint{405.216003pt}{96.934731pt}}
\pgfpathclose
\pgfusepath{fill,stroke}
\color[rgb]{0.321330,0.771498,0.410293}
\pgfpathmoveto{\pgfpoint{396.287964pt}{109.288422pt}}
\pgflineto{\pgfpoint{405.216003pt}{103.111580pt}}
\pgflineto{\pgfpoint{396.287964pt}{103.111580pt}}
\pgfpathclose
\pgfusepath{fill,stroke}
\color[rgb]{0.290001,0.758846,0.427826}
\pgfpathmoveto{\pgfpoint{405.216003pt}{96.934731pt}}
\pgflineto{\pgfpoint{414.143982pt}{96.934731pt}}
\pgflineto{\pgfpoint{414.143982pt}{90.757896pt}}
\pgfpathclose
\pgfusepath{fill,stroke}
\color[rgb]{0.321330,0.771498,0.410293}
\pgfpathmoveto{\pgfpoint{405.216003pt}{103.111580pt}}
\pgflineto{\pgfpoint{414.143982pt}{96.934731pt}}
\pgflineto{\pgfpoint{405.216003pt}{96.934731pt}}
\pgfpathclose
\pgfusepath{fill,stroke}
\pgfpathmoveto{\pgfpoint{405.216003pt}{103.111580pt}}
\pgflineto{\pgfpoint{414.143982pt}{103.111580pt}}
\pgflineto{\pgfpoint{414.143982pt}{96.934731pt}}
\pgfpathclose
\pgfusepath{fill,stroke}
\color[rgb]{0.354355,0.783714,0.391488}
\pgfpathmoveto{\pgfpoint{405.216003pt}{109.288422pt}}
\pgflineto{\pgfpoint{414.143982pt}{109.288422pt}}
\pgflineto{\pgfpoint{414.143982pt}{103.111580pt}}
\pgfpathclose
\pgfusepath{fill,stroke}
\color[rgb]{0.321330,0.771498,0.410293}
\pgfpathmoveto{\pgfpoint{414.143982pt}{96.934731pt}}
\pgflineto{\pgfpoint{423.071960pt}{90.757896pt}}
\pgflineto{\pgfpoint{414.143982pt}{90.757896pt}}
\pgfpathclose
\pgfusepath{fill,stroke}
\pgfpathmoveto{\pgfpoint{414.143982pt}{96.934731pt}}
\pgflineto{\pgfpoint{423.071960pt}{96.934731pt}}
\pgflineto{\pgfpoint{423.071960pt}{90.757896pt}}
\pgfpathclose
\pgfusepath{fill,stroke}
\color[rgb]{0.354355,0.783714,0.391488}
\pgfpathmoveto{\pgfpoint{414.143982pt}{103.111580pt}}
\pgflineto{\pgfpoint{423.071960pt}{96.934731pt}}
\pgflineto{\pgfpoint{414.143982pt}{96.934731pt}}
\pgfpathclose
\pgfusepath{fill,stroke}
\color[rgb]{0.388930,0.795453,0.371421}
\pgfpathmoveto{\pgfpoint{414.143982pt}{109.288422pt}}
\pgflineto{\pgfpoint{423.071960pt}{103.111580pt}}
\pgflineto{\pgfpoint{414.143982pt}{103.111580pt}}
\pgfpathclose
\pgfusepath{fill,stroke}
\pgfpathmoveto{\pgfpoint{423.071960pt}{96.934731pt}}
\pgflineto{\pgfpoint{432.000000pt}{90.757896pt}}
\pgflineto{\pgfpoint{423.071960pt}{90.757896pt}}
\pgfpathclose
\pgfusepath{fill,stroke}
\pgfpathmoveto{\pgfpoint{423.071960pt}{96.934731pt}}
\pgflineto{\pgfpoint{432.000000pt}{96.934731pt}}
\pgflineto{\pgfpoint{432.000000pt}{90.757896pt}}
\pgfpathclose
\pgfusepath{fill,stroke}
\color[rgb]{0.424933,0.806674,0.350099}
\pgfpathmoveto{\pgfpoint{432.000000pt}{96.934731pt}}
\pgflineto{\pgfpoint{440.927979pt}{90.757896pt}}
\pgflineto{\pgfpoint{432.000000pt}{90.757896pt}}
\pgfpathclose
\pgfusepath{fill,stroke}
\pgfpathmoveto{\pgfpoint{432.000000pt}{96.934731pt}}
\pgflineto{\pgfpoint{440.927979pt}{96.934731pt}}
\pgflineto{\pgfpoint{440.927979pt}{90.757896pt}}
\pgfpathclose
\pgfusepath{fill,stroke}
\color[rgb]{0.462247,0.817338,0.327545}
\pgfpathmoveto{\pgfpoint{440.927979pt}{96.934731pt}}
\pgflineto{\pgfpoint{449.855957pt}{90.757896pt}}
\pgflineto{\pgfpoint{440.927979pt}{90.757896pt}}
\pgfpathclose
\pgfusepath{fill,stroke}
\pgfpathmoveto{\pgfpoint{467.711975pt}{72.227356pt}}
\pgflineto{\pgfpoint{476.639954pt}{72.227356pt}}
\pgflineto{\pgfpoint{476.639954pt}{66.050522pt}}
\pgfpathclose
\pgfusepath{fill,stroke}
\color[rgb]{0.500754,0.827409,0.303799}
\pgfpathmoveto{\pgfpoint{476.639954pt}{72.227356pt}}
\pgflineto{\pgfpoint{485.567963pt}{66.050522pt}}
\pgflineto{\pgfpoint{476.639954pt}{66.050522pt}}
\pgfpathclose
\pgfusepath{fill,stroke}
\pgfpathmoveto{\pgfpoint{476.639954pt}{72.227356pt}}
\pgflineto{\pgfpoint{485.567963pt}{72.227356pt}}
\pgflineto{\pgfpoint{485.567963pt}{66.050522pt}}
\pgfpathclose
\pgfusepath{fill,stroke}
\pgfpathmoveto{\pgfpoint{485.567963pt}{66.050522pt}}
\pgflineto{\pgfpoint{494.495972pt}{66.050522pt}}
\pgflineto{\pgfpoint{494.495972pt}{59.873672pt}}
\pgfpathclose
\pgfusepath{fill,stroke}
\color[rgb]{0.540337,0.836858,0.278917}
\pgfpathmoveto{\pgfpoint{485.567963pt}{72.227356pt}}
\pgflineto{\pgfpoint{494.495972pt}{66.050522pt}}
\pgflineto{\pgfpoint{485.567963pt}{66.050522pt}}
\pgfpathclose
\pgfusepath{fill,stroke}
\pgfpathmoveto{\pgfpoint{485.567963pt}{72.227356pt}}
\pgflineto{\pgfpoint{494.495972pt}{72.227356pt}}
\pgflineto{\pgfpoint{494.495972pt}{66.050522pt}}
\pgfpathclose
\pgfusepath{fill,stroke}
\pgfpathmoveto{\pgfpoint{494.495972pt}{66.050522pt}}
\pgflineto{\pgfpoint{503.423981pt}{59.873672pt}}
\pgflineto{\pgfpoint{494.495972pt}{59.873672pt}}
\pgfpathclose
\pgfusepath{fill,stroke}
\pgfpathmoveto{\pgfpoint{494.495972pt}{66.050522pt}}
\pgflineto{\pgfpoint{503.423981pt}{66.050522pt}}
\pgflineto{\pgfpoint{503.423981pt}{59.873672pt}}
\pgfpathclose
\pgfusepath{fill,stroke}
\color[rgb]{0.580861,0.845663,0.253001}
\pgfpathmoveto{\pgfpoint{494.495972pt}{72.227356pt}}
\pgflineto{\pgfpoint{503.423981pt}{66.050522pt}}
\pgflineto{\pgfpoint{494.495972pt}{66.050522pt}}
\pgfpathclose
\pgfusepath{fill,stroke}
\color[rgb]{0.540337,0.836858,0.278917}
\pgfpathmoveto{\pgfpoint{503.423981pt}{59.873672pt}}
\pgflineto{\pgfpoint{512.351990pt}{59.873672pt}}
\pgflineto{\pgfpoint{512.351990pt}{53.696838pt}}
\pgfpathclose
\pgfusepath{fill,stroke}
\color[rgb]{0.580861,0.845663,0.253001}
\pgfpathmoveto{\pgfpoint{503.423981pt}{66.050522pt}}
\pgflineto{\pgfpoint{512.351990pt}{59.873672pt}}
\pgflineto{\pgfpoint{503.423981pt}{59.873672pt}}
\pgfpathclose
\pgfusepath{fill,stroke}
\pgfpathmoveto{\pgfpoint{503.423981pt}{66.050522pt}}
\pgflineto{\pgfpoint{512.351990pt}{66.050522pt}}
\pgflineto{\pgfpoint{512.351990pt}{59.873672pt}}
\pgfpathclose
\pgfusepath{fill,stroke}
\color[rgb]{0.540337,0.836858,0.278917}
\pgfpathmoveto{\pgfpoint{512.351990pt}{53.696838pt}}
\pgflineto{\pgfpoint{521.279968pt}{53.696838pt}}
\pgflineto{\pgfpoint{521.279968pt}{47.519989pt}}
\pgfpathclose
\pgfusepath{fill,stroke}
\color[rgb]{0.354355,0.783714,0.391488}
\pgfpathmoveto{\pgfpoint{423.071960pt}{90.757896pt}}
\pgflineto{\pgfpoint{432.000000pt}{90.757896pt}}
\pgflineto{\pgfpoint{432.000000pt}{84.581039pt}}
\pgfpathclose
\pgfusepath{fill,stroke}
\color[rgb]{0.388930,0.795453,0.371421}
\pgfpathmoveto{\pgfpoint{432.000000pt}{90.757896pt}}
\pgflineto{\pgfpoint{440.927979pt}{84.581039pt}}
\pgflineto{\pgfpoint{432.000000pt}{84.581039pt}}
\pgfpathclose
\pgfusepath{fill,stroke}
\pgfpathmoveto{\pgfpoint{432.000000pt}{90.757896pt}}
\pgflineto{\pgfpoint{440.927979pt}{90.757896pt}}
\pgflineto{\pgfpoint{440.927979pt}{84.581039pt}}
\pgfpathclose
\pgfusepath{fill,stroke}
\pgfpathmoveto{\pgfpoint{440.927979pt}{84.581039pt}}
\pgflineto{\pgfpoint{449.855957pt}{84.581039pt}}
\pgflineto{\pgfpoint{449.855957pt}{78.404205pt}}
\pgfpathclose
\pgfusepath{fill,stroke}
\color[rgb]{0.424933,0.806674,0.350099}
\pgfpathmoveto{\pgfpoint{440.927979pt}{90.757896pt}}
\pgflineto{\pgfpoint{449.855957pt}{84.581039pt}}
\pgflineto{\pgfpoint{440.927979pt}{84.581039pt}}
\pgfpathclose
\pgfusepath{fill,stroke}
\pgfpathmoveto{\pgfpoint{440.927979pt}{90.757896pt}}
\pgflineto{\pgfpoint{449.855957pt}{90.757896pt}}
\pgflineto{\pgfpoint{449.855957pt}{84.581039pt}}
\pgfpathclose
\pgfusepath{fill,stroke}
\color[rgb]{0.388930,0.795453,0.371421}
\pgfpathmoveto{\pgfpoint{449.855957pt}{78.404205pt}}
\pgflineto{\pgfpoint{458.783936pt}{78.404205pt}}
\pgflineto{\pgfpoint{458.783936pt}{72.227356pt}}
\pgfpathclose
\pgfusepath{fill,stroke}
\color[rgb]{0.424933,0.806674,0.350099}
\pgfpathmoveto{\pgfpoint{449.855957pt}{84.581039pt}}
\pgflineto{\pgfpoint{458.783936pt}{78.404205pt}}
\pgflineto{\pgfpoint{449.855957pt}{78.404205pt}}
\pgfpathclose
\pgfusepath{fill,stroke}
\pgfpathmoveto{\pgfpoint{449.855957pt}{84.581039pt}}
\pgflineto{\pgfpoint{458.783936pt}{84.581039pt}}
\pgflineto{\pgfpoint{458.783936pt}{78.404205pt}}
\pgfpathclose
\pgfusepath{fill,stroke}
\color[rgb]{0.462247,0.817338,0.327545}
\pgfpathmoveto{\pgfpoint{449.855957pt}{90.757896pt}}
\pgflineto{\pgfpoint{458.783936pt}{84.581039pt}}
\pgflineto{\pgfpoint{449.855957pt}{84.581039pt}}
\pgfpathclose
\pgfusepath{fill,stroke}
\pgfpathmoveto{\pgfpoint{449.855957pt}{90.757896pt}}
\pgflineto{\pgfpoint{458.783936pt}{90.757896pt}}
\pgflineto{\pgfpoint{458.783936pt}{84.581039pt}}
\pgfpathclose
\pgfusepath{fill,stroke}
\color[rgb]{0.424933,0.806674,0.350099}
\pgfpathmoveto{\pgfpoint{458.783936pt}{78.404205pt}}
\pgflineto{\pgfpoint{467.711975pt}{72.227356pt}}
\pgflineto{\pgfpoint{458.783936pt}{72.227356pt}}
\pgfpathclose
\pgfusepath{fill,stroke}
\pgfpathmoveto{\pgfpoint{458.783936pt}{78.404205pt}}
\pgflineto{\pgfpoint{467.711975pt}{78.404205pt}}
\pgflineto{\pgfpoint{467.711975pt}{72.227356pt}}
\pgfpathclose
\pgfusepath{fill,stroke}
\color[rgb]{0.462247,0.817338,0.327545}
\pgfpathmoveto{\pgfpoint{458.783936pt}{84.581039pt}}
\pgflineto{\pgfpoint{467.711975pt}{78.404205pt}}
\pgflineto{\pgfpoint{458.783936pt}{78.404205pt}}
\pgfpathclose
\pgfusepath{fill,stroke}
\pgfpathmoveto{\pgfpoint{458.783936pt}{84.581039pt}}
\pgflineto{\pgfpoint{467.711975pt}{84.581039pt}}
\pgflineto{\pgfpoint{467.711975pt}{78.404205pt}}
\pgfpathclose
\pgfusepath{fill,stroke}
\color[rgb]{0.500754,0.827409,0.303799}
\pgfpathmoveto{\pgfpoint{458.783936pt}{90.757896pt}}
\pgflineto{\pgfpoint{467.711975pt}{84.581039pt}}
\pgflineto{\pgfpoint{458.783936pt}{84.581039pt}}
\pgfpathclose
\pgfusepath{fill,stroke}
\color[rgb]{0.462247,0.817338,0.327545}
\pgfpathmoveto{\pgfpoint{467.711975pt}{78.404205pt}}
\pgflineto{\pgfpoint{476.639954pt}{72.227356pt}}
\pgflineto{\pgfpoint{467.711975pt}{72.227356pt}}
\pgfpathclose
\pgfusepath{fill,stroke}
\pgfpathmoveto{\pgfpoint{467.711975pt}{78.404205pt}}
\pgflineto{\pgfpoint{476.639954pt}{78.404205pt}}
\pgflineto{\pgfpoint{476.639954pt}{72.227356pt}}
\pgfpathclose
\pgfusepath{fill,stroke}
\color[rgb]{0.500754,0.827409,0.303799}
\pgfpathmoveto{\pgfpoint{467.711975pt}{84.581039pt}}
\pgflineto{\pgfpoint{476.639954pt}{78.404205pt}}
\pgflineto{\pgfpoint{467.711975pt}{78.404205pt}}
\pgfpathclose
\pgfusepath{fill,stroke}
\pgfpathmoveto{\pgfpoint{467.711975pt}{84.581039pt}}
\pgflineto{\pgfpoint{476.639954pt}{84.581039pt}}
\pgflineto{\pgfpoint{476.639954pt}{78.404205pt}}
\pgfpathclose
\pgfusepath{fill,stroke}
\pgfpathmoveto{\pgfpoint{476.639954pt}{78.404205pt}}
\pgflineto{\pgfpoint{485.567963pt}{72.227356pt}}
\pgflineto{\pgfpoint{476.639954pt}{72.227356pt}}
\pgfpathclose
\pgfusepath{fill,stroke}
\color[rgb]{0.540337,0.836858,0.278917}
\pgfpathmoveto{\pgfpoint{476.639954pt}{84.581039pt}}
\pgflineto{\pgfpoint{485.567963pt}{78.404205pt}}
\pgflineto{\pgfpoint{476.639954pt}{78.404205pt}}
\pgfpathclose
\pgfusepath{fill,stroke}
\color[rgb]{0.185538,0.704725,0.485412}
\pgfpathmoveto{\pgfpoint{333.791992pt}{127.818947pt}}
\pgflineto{\pgfpoint{342.719971pt}{127.818947pt}}
\pgflineto{\pgfpoint{342.719971pt}{121.642097pt}}
\pgfpathclose
\pgfusepath{fill,stroke}
\color[rgb]{0.208030,0.718701,0.472873}
\pgfpathmoveto{\pgfpoint{333.791992pt}{133.995789pt}}
\pgflineto{\pgfpoint{342.719971pt}{127.818947pt}}
\pgflineto{\pgfpoint{333.791992pt}{127.818947pt}}
\pgfpathclose
\pgfusepath{fill,stroke}
\pgfpathmoveto{\pgfpoint{333.791992pt}{133.995789pt}}
\pgflineto{\pgfpoint{342.719971pt}{133.995789pt}}
\pgflineto{\pgfpoint{342.719971pt}{127.818947pt}}
\pgfpathclose
\pgfusepath{fill,stroke}
\color[rgb]{0.233127,0.732406,0.459106}
\pgfpathmoveto{\pgfpoint{333.791992pt}{140.172638pt}}
\pgflineto{\pgfpoint{342.719971pt}{133.995789pt}}
\pgflineto{\pgfpoint{333.791992pt}{133.995789pt}}
\pgfpathclose
\pgfusepath{fill,stroke}
\color[rgb]{0.208030,0.718701,0.472873}
\pgfpathmoveto{\pgfpoint{342.719971pt}{127.818947pt}}
\pgflineto{\pgfpoint{351.647980pt}{121.642097pt}}
\pgflineto{\pgfpoint{342.719971pt}{121.642097pt}}
\pgfpathclose
\pgfusepath{fill,stroke}
\pgfpathmoveto{\pgfpoint{342.719971pt}{127.818947pt}}
\pgflineto{\pgfpoint{351.647980pt}{127.818947pt}}
\pgflineto{\pgfpoint{351.647980pt}{121.642097pt}}
\pgfpathclose
\pgfusepath{fill,stroke}
\pgfpathmoveto{\pgfpoint{351.647980pt}{121.642097pt}}
\pgflineto{\pgfpoint{360.575958pt}{121.642097pt}}
\pgflineto{\pgfpoint{360.575958pt}{115.465263pt}}
\pgfpathclose
\pgfusepath{fill,stroke}
\color[rgb]{0.233127,0.732406,0.459106}
\pgfpathmoveto{\pgfpoint{351.647980pt}{127.818947pt}}
\pgflineto{\pgfpoint{360.575958pt}{121.642097pt}}
\pgflineto{\pgfpoint{351.647980pt}{121.642097pt}}
\pgfpathclose
\pgfusepath{fill,stroke}
\pgfpathmoveto{\pgfpoint{351.647980pt}{127.818947pt}}
\pgflineto{\pgfpoint{360.575958pt}{127.818947pt}}
\pgflineto{\pgfpoint{360.575958pt}{121.642097pt}}
\pgfpathclose
\pgfusepath{fill,stroke}
\color[rgb]{0.260531,0.745802,0.444096}
\pgfpathmoveto{\pgfpoint{351.647980pt}{133.995789pt}}
\pgflineto{\pgfpoint{360.575958pt}{127.818947pt}}
\pgflineto{\pgfpoint{351.647980pt}{127.818947pt}}
\pgfpathclose
\pgfusepath{fill,stroke}
\color[rgb]{0.208030,0.718701,0.472873}
\pgfpathmoveto{\pgfpoint{360.575958pt}{115.465263pt}}
\pgflineto{\pgfpoint{369.503998pt}{115.465263pt}}
\pgflineto{\pgfpoint{369.503998pt}{109.288422pt}}
\pgfpathclose
\pgfusepath{fill,stroke}
\color[rgb]{0.233127,0.732406,0.459106}
\pgfpathmoveto{\pgfpoint{360.575958pt}{121.642097pt}}
\pgflineto{\pgfpoint{369.503998pt}{115.465263pt}}
\pgflineto{\pgfpoint{360.575958pt}{115.465263pt}}
\pgfpathclose
\pgfusepath{fill,stroke}
\color[rgb]{0.127668,0.646882,0.523924}
\pgfpathmoveto{\pgfpoint{271.295990pt}{152.526306pt}}
\pgflineto{\pgfpoint{280.223969pt}{152.526306pt}}
\pgflineto{\pgfpoint{280.223969pt}{146.349472pt}}
\pgfpathclose
\pgfusepath{fill,stroke}
\color[rgb]{0.136835,0.661563,0.515967}
\pgfpathmoveto{\pgfpoint{271.295990pt}{158.703156pt}}
\pgflineto{\pgfpoint{280.223969pt}{152.526306pt}}
\pgflineto{\pgfpoint{271.295990pt}{152.526306pt}}
\pgfpathclose
\pgfusepath{fill,stroke}
\pgfpathmoveto{\pgfpoint{271.295990pt}{158.703156pt}}
\pgflineto{\pgfpoint{280.223969pt}{158.703156pt}}
\pgflineto{\pgfpoint{280.223969pt}{152.526306pt}}
\pgfpathclose
\pgfusepath{fill,stroke}
\pgfpathmoveto{\pgfpoint{271.295990pt}{164.880005pt}}
\pgflineto{\pgfpoint{280.223969pt}{158.703156pt}}
\pgflineto{\pgfpoint{271.295990pt}{158.703156pt}}
\pgfpathclose
\pgfusepath{fill,stroke}
\color[rgb]{0.127668,0.646882,0.523924}
\pgfpathmoveto{\pgfpoint{280.223969pt}{146.349472pt}}
\pgflineto{\pgfpoint{289.151978pt}{146.349472pt}}
\pgflineto{\pgfpoint{289.151978pt}{140.172638pt}}
\pgfpathclose
\pgfusepath{fill,stroke}
\color[rgb]{0.136835,0.661563,0.515967}
\pgfpathmoveto{\pgfpoint{280.223969pt}{152.526306pt}}
\pgflineto{\pgfpoint{289.151978pt}{146.349472pt}}
\pgflineto{\pgfpoint{280.223969pt}{146.349472pt}}
\pgfpathclose
\pgfusepath{fill,stroke}
\pgfpathmoveto{\pgfpoint{289.151978pt}{146.349472pt}}
\pgflineto{\pgfpoint{298.079987pt}{140.172638pt}}
\pgflineto{\pgfpoint{289.151978pt}{140.172638pt}}
\pgfpathclose
\pgfusepath{fill,stroke}
\pgfpathmoveto{\pgfpoint{289.151978pt}{146.349472pt}}
\pgflineto{\pgfpoint{298.079987pt}{146.349472pt}}
\pgflineto{\pgfpoint{298.079987pt}{140.172638pt}}
\pgfpathclose
\pgfusepath{fill,stroke}
\color[rgb]{0.149643,0.676120,0.506924}
\pgfpathmoveto{\pgfpoint{289.151978pt}{152.526306pt}}
\pgflineto{\pgfpoint{298.079987pt}{152.526306pt}}
\pgflineto{\pgfpoint{298.079987pt}{146.349472pt}}
\pgfpathclose
\pgfusepath{fill,stroke}
\color[rgb]{0.165967,0.690519,0.496752}
\pgfpathmoveto{\pgfpoint{289.151978pt}{158.703156pt}}
\pgflineto{\pgfpoint{298.079987pt}{152.526306pt}}
\pgflineto{\pgfpoint{289.151978pt}{152.526306pt}}
\pgfpathclose
\pgfusepath{fill,stroke}
\color[rgb]{0.136835,0.661563,0.515967}
\pgfpathmoveto{\pgfpoint{298.079987pt}{140.172638pt}}
\pgflineto{\pgfpoint{307.007965pt}{140.172638pt}}
\pgflineto{\pgfpoint{307.007965pt}{133.995789pt}}
\pgfpathclose
\pgfusepath{fill,stroke}
\color[rgb]{0.149643,0.676120,0.506924}
\pgfpathmoveto{\pgfpoint{298.079987pt}{146.349472pt}}
\pgflineto{\pgfpoint{307.007965pt}{140.172638pt}}
\pgflineto{\pgfpoint{298.079987pt}{140.172638pt}}
\pgfpathclose
\pgfusepath{fill,stroke}
\color[rgb]{0.122163,0.587476,0.546023}
\pgfpathmoveto{\pgfpoint{217.727982pt}{171.056854pt}}
\pgflineto{\pgfpoint{226.655975pt}{171.056854pt}}
\pgflineto{\pgfpoint{226.655975pt}{164.880005pt}}
\pgfpathclose
\pgfusepath{fill,stroke}
\color[rgb]{0.119872,0.602382,0.541831}
\pgfpathmoveto{\pgfpoint{217.727982pt}{177.233673pt}}
\pgflineto{\pgfpoint{226.655975pt}{171.056854pt}}
\pgflineto{\pgfpoint{217.727982pt}{171.056854pt}}
\pgfpathclose
\pgfusepath{fill,stroke}
\pgfpathmoveto{\pgfpoint{226.655975pt}{171.056854pt}}
\pgflineto{\pgfpoint{235.583969pt}{164.880005pt}}
\pgflineto{\pgfpoint{226.655975pt}{164.880005pt}}
\pgfpathclose
\pgfusepath{fill,stroke}
\pgfpathmoveto{\pgfpoint{226.655975pt}{171.056854pt}}
\pgflineto{\pgfpoint{235.583969pt}{171.056854pt}}
\pgflineto{\pgfpoint{235.583969pt}{164.880005pt}}
\pgfpathclose
\pgfusepath{fill,stroke}
\color[rgb]{0.119627,0.617266,0.536796}
\pgfpathmoveto{\pgfpoint{226.655975pt}{177.233673pt}}
\pgflineto{\pgfpoint{235.583969pt}{177.233673pt}}
\pgflineto{\pgfpoint{235.583969pt}{171.056854pt}}
\pgfpathclose
\pgfusepath{fill,stroke}
\color[rgb]{0.122046,0.632107,0.530848}
\pgfpathmoveto{\pgfpoint{226.655975pt}{183.410522pt}}
\pgflineto{\pgfpoint{235.583969pt}{177.233673pt}}
\pgflineto{\pgfpoint{226.655975pt}{177.233673pt}}
\pgfpathclose
\pgfusepath{fill,stroke}
\color[rgb]{0.119872,0.602382,0.541831}
\pgfpathmoveto{\pgfpoint{235.583969pt}{164.880005pt}}
\pgflineto{\pgfpoint{244.511993pt}{164.880005pt}}
\pgflineto{\pgfpoint{244.511993pt}{158.703156pt}}
\pgfpathclose
\pgfusepath{fill,stroke}
\color[rgb]{0.119627,0.617266,0.536796}
\pgfpathmoveto{\pgfpoint{235.583969pt}{171.056854pt}}
\pgflineto{\pgfpoint{244.511993pt}{164.880005pt}}
\pgflineto{\pgfpoint{235.583969pt}{164.880005pt}}
\pgfpathclose
\pgfusepath{fill,stroke}
\color[rgb]{0.122163,0.587476,0.546023}
\pgfpathmoveto{\pgfpoint{190.943985pt}{195.764206pt}}
\pgflineto{\pgfpoint{199.871979pt}{195.764206pt}}
\pgflineto{\pgfpoint{199.871979pt}{189.587372pt}}
\pgfpathclose
\pgfusepath{fill,stroke}
\color[rgb]{0.119872,0.602382,0.541831}
\pgfpathmoveto{\pgfpoint{190.943985pt}{201.941055pt}}
\pgflineto{\pgfpoint{199.871979pt}{195.764206pt}}
\pgflineto{\pgfpoint{190.943985pt}{195.764206pt}}
\pgfpathclose
\pgfusepath{fill,stroke}
\pgfpathmoveto{\pgfpoint{190.943985pt}{201.941055pt}}
\pgflineto{\pgfpoint{199.871979pt}{201.941055pt}}
\pgflineto{\pgfpoint{199.871979pt}{195.764206pt}}
\pgfpathclose
\pgfusepath{fill,stroke}
\color[rgb]{0.122163,0.587476,0.546023}
\pgfpathmoveto{\pgfpoint{199.871979pt}{189.587372pt}}
\pgflineto{\pgfpoint{208.799988pt}{183.410522pt}}
\pgflineto{\pgfpoint{199.871979pt}{183.410522pt}}
\pgfpathclose
\pgfusepath{fill,stroke}
\pgfpathmoveto{\pgfpoint{199.871979pt}{189.587372pt}}
\pgflineto{\pgfpoint{208.799988pt}{189.587372pt}}
\pgflineto{\pgfpoint{208.799988pt}{183.410522pt}}
\pgfpathclose
\pgfusepath{fill,stroke}
\color[rgb]{0.119872,0.602382,0.541831}
\pgfpathmoveto{\pgfpoint{199.871979pt}{195.764206pt}}
\pgflineto{\pgfpoint{208.799988pt}{189.587372pt}}
\pgflineto{\pgfpoint{199.871979pt}{189.587372pt}}
\pgfpathclose
\pgfusepath{fill,stroke}
\pgfpathmoveto{\pgfpoint{199.871979pt}{195.764206pt}}
\pgflineto{\pgfpoint{208.799988pt}{195.764206pt}}
\pgflineto{\pgfpoint{208.799988pt}{189.587372pt}}
\pgfpathclose
\pgfusepath{fill,stroke}
\color[rgb]{0.119627,0.617266,0.536796}
\pgfpathmoveto{\pgfpoint{199.871979pt}{201.941055pt}}
\pgflineto{\pgfpoint{208.799988pt}{195.764206pt}}
\pgflineto{\pgfpoint{199.871979pt}{195.764206pt}}
\pgfpathclose
\pgfusepath{fill,stroke}
\color[rgb]{0.119872,0.602382,0.541831}
\pgfpathmoveto{\pgfpoint{208.799988pt}{189.587372pt}}
\pgflineto{\pgfpoint{217.727982pt}{189.587372pt}}
\pgflineto{\pgfpoint{217.727982pt}{183.410522pt}}
\pgfpathclose
\pgfusepath{fill,stroke}
\color[rgb]{0.119627,0.617266,0.536796}
\pgfpathmoveto{\pgfpoint{208.799988pt}{195.764206pt}}
\pgflineto{\pgfpoint{217.727982pt}{189.587372pt}}
\pgflineto{\pgfpoint{208.799988pt}{189.587372pt}}
\pgfpathclose
\pgfusepath{fill,stroke}
\color[rgb]{0.119872,0.602382,0.541831}
\pgfpathmoveto{\pgfpoint{217.727982pt}{177.233673pt}}
\pgflineto{\pgfpoint{226.655975pt}{177.233673pt}}
\pgflineto{\pgfpoint{226.655975pt}{171.056854pt}}
\pgfpathclose
\pgfusepath{fill,stroke}
\pgfpathmoveto{\pgfpoint{217.727982pt}{183.410522pt}}
\pgflineto{\pgfpoint{226.655975pt}{177.233673pt}}
\pgflineto{\pgfpoint{217.727982pt}{177.233673pt}}
\pgfpathclose
\pgfusepath{fill,stroke}
\pgfpathmoveto{\pgfpoint{217.727982pt}{183.410522pt}}
\pgflineto{\pgfpoint{226.655975pt}{183.410522pt}}
\pgflineto{\pgfpoint{226.655975pt}{177.233673pt}}
\pgfpathclose
\pgfusepath{fill,stroke}
\color[rgb]{0.119627,0.617266,0.536796}
\pgfpathmoveto{\pgfpoint{217.727982pt}{189.587372pt}}
\pgflineto{\pgfpoint{226.655975pt}{183.410522pt}}
\pgflineto{\pgfpoint{217.727982pt}{183.410522pt}}
\pgfpathclose
\pgfusepath{fill,stroke}
\pgfpathmoveto{\pgfpoint{217.727982pt}{189.587372pt}}
\pgflineto{\pgfpoint{226.655975pt}{189.587372pt}}
\pgflineto{\pgfpoint{226.655975pt}{183.410522pt}}
\pgfpathclose
\pgfusepath{fill,stroke}
\pgfpathmoveto{\pgfpoint{226.655975pt}{177.233673pt}}
\pgflineto{\pgfpoint{235.583969pt}{171.056854pt}}
\pgflineto{\pgfpoint{226.655975pt}{171.056854pt}}
\pgfpathclose
\pgfusepath{fill,stroke}
\color[rgb]{0.122046,0.632107,0.530848}
\pgfpathmoveto{\pgfpoint{226.655975pt}{183.410522pt}}
\pgflineto{\pgfpoint{235.583969pt}{183.410522pt}}
\pgflineto{\pgfpoint{235.583969pt}{177.233673pt}}
\pgfpathclose
\pgfusepath{fill,stroke}
\pgfpathmoveto{\pgfpoint{226.655975pt}{189.587372pt}}
\pgflineto{\pgfpoint{235.583969pt}{183.410522pt}}
\pgflineto{\pgfpoint{226.655975pt}{183.410522pt}}
\pgfpathclose
\pgfusepath{fill,stroke}
\color[rgb]{0.119627,0.617266,0.536796}
\pgfpathmoveto{\pgfpoint{235.583969pt}{171.056854pt}}
\pgflineto{\pgfpoint{244.511993pt}{171.056854pt}}
\pgflineto{\pgfpoint{244.511993pt}{164.880005pt}}
\pgfpathclose
\pgfusepath{fill,stroke}
\color[rgb]{0.122046,0.632107,0.530848}
\pgfpathmoveto{\pgfpoint{235.583969pt}{177.233673pt}}
\pgflineto{\pgfpoint{244.511993pt}{171.056854pt}}
\pgflineto{\pgfpoint{235.583969pt}{171.056854pt}}
\pgfpathclose
\pgfusepath{fill,stroke}
\pgfpathmoveto{\pgfpoint{235.583969pt}{177.233673pt}}
\pgflineto{\pgfpoint{244.511993pt}{177.233673pt}}
\pgflineto{\pgfpoint{244.511993pt}{171.056854pt}}
\pgfpathclose
\pgfusepath{fill,stroke}
\color[rgb]{0.127668,0.646882,0.523924}
\pgfpathmoveto{\pgfpoint{235.583969pt}{183.410522pt}}
\pgflineto{\pgfpoint{244.511993pt}{177.233673pt}}
\pgflineto{\pgfpoint{235.583969pt}{177.233673pt}}
\pgfpathclose
\pgfusepath{fill,stroke}
\pgfpathmoveto{\pgfpoint{235.583969pt}{183.410522pt}}
\pgflineto{\pgfpoint{244.511993pt}{183.410522pt}}
\pgflineto{\pgfpoint{244.511993pt}{177.233673pt}}
\pgfpathclose
\pgfusepath{fill,stroke}
\color[rgb]{0.122046,0.632107,0.530848}
\pgfpathmoveto{\pgfpoint{244.511993pt}{171.056854pt}}
\pgflineto{\pgfpoint{253.440002pt}{164.880005pt}}
\pgflineto{\pgfpoint{244.511993pt}{164.880005pt}}
\pgfpathclose
\pgfusepath{fill,stroke}
\color[rgb]{0.127668,0.646882,0.523924}
\pgfpathmoveto{\pgfpoint{244.511993pt}{177.233673pt}}
\pgflineto{\pgfpoint{253.440002pt}{177.233673pt}}
\pgflineto{\pgfpoint{253.440002pt}{171.056854pt}}
\pgfpathclose
\pgfusepath{fill,stroke}
\color[rgb]{0.136835,0.661563,0.515967}
\pgfpathmoveto{\pgfpoint{244.511993pt}{183.410522pt}}
\pgflineto{\pgfpoint{253.440002pt}{177.233673pt}}
\pgflineto{\pgfpoint{244.511993pt}{177.233673pt}}
\pgfpathclose
\pgfusepath{fill,stroke}
\color[rgb]{0.127668,0.646882,0.523924}
\pgfpathmoveto{\pgfpoint{253.440002pt}{171.056854pt}}
\pgflineto{\pgfpoint{262.367981pt}{171.056854pt}}
\pgflineto{\pgfpoint{262.367981pt}{164.880005pt}}
\pgfpathclose
\pgfusepath{fill,stroke}
\color[rgb]{0.136835,0.661563,0.515967}
\pgfpathmoveto{\pgfpoint{253.440002pt}{177.233673pt}}
\pgflineto{\pgfpoint{262.367981pt}{171.056854pt}}
\pgflineto{\pgfpoint{253.440002pt}{171.056854pt}}
\pgfpathclose
\pgfusepath{fill,stroke}
\pgfpathmoveto{\pgfpoint{253.440002pt}{177.233673pt}}
\pgflineto{\pgfpoint{262.367981pt}{177.233673pt}}
\pgflineto{\pgfpoint{262.367981pt}{171.056854pt}}
\pgfpathclose
\pgfusepath{fill,stroke}
\color[rgb]{0.127668,0.646882,0.523924}
\pgfpathmoveto{\pgfpoint{262.367981pt}{164.880005pt}}
\pgflineto{\pgfpoint{271.295990pt}{158.703156pt}}
\pgflineto{\pgfpoint{262.367981pt}{158.703156pt}}
\pgfpathclose
\pgfusepath{fill,stroke}
\pgfpathmoveto{\pgfpoint{262.367981pt}{164.880005pt}}
\pgflineto{\pgfpoint{271.295990pt}{164.880005pt}}
\pgflineto{\pgfpoint{271.295990pt}{158.703156pt}}
\pgfpathclose
\pgfusepath{fill,stroke}
\color[rgb]{0.136835,0.661563,0.515967}
\pgfpathmoveto{\pgfpoint{262.367981pt}{171.056854pt}}
\pgflineto{\pgfpoint{271.295990pt}{164.880005pt}}
\pgflineto{\pgfpoint{262.367981pt}{164.880005pt}}
\pgfpathclose
\pgfusepath{fill,stroke}
\pgfpathmoveto{\pgfpoint{262.367981pt}{171.056854pt}}
\pgflineto{\pgfpoint{271.295990pt}{171.056854pt}}
\pgflineto{\pgfpoint{271.295990pt}{164.880005pt}}
\pgfpathclose
\pgfusepath{fill,stroke}
\color[rgb]{0.149643,0.676120,0.506924}
\pgfpathmoveto{\pgfpoint{262.367981pt}{177.233673pt}}
\pgflineto{\pgfpoint{271.295990pt}{171.056854pt}}
\pgflineto{\pgfpoint{262.367981pt}{171.056854pt}}
\pgfpathclose
\pgfusepath{fill,stroke}
\color[rgb]{0.136835,0.661563,0.515967}
\pgfpathmoveto{\pgfpoint{271.295990pt}{164.880005pt}}
\pgflineto{\pgfpoint{280.223969pt}{164.880005pt}}
\pgflineto{\pgfpoint{280.223969pt}{158.703156pt}}
\pgfpathclose
\pgfusepath{fill,stroke}
\color[rgb]{0.149643,0.676120,0.506924}
\pgfpathmoveto{\pgfpoint{271.295990pt}{171.056854pt}}
\pgflineto{\pgfpoint{280.223969pt}{164.880005pt}}
\pgflineto{\pgfpoint{271.295990pt}{164.880005pt}}
\pgfpathclose
\pgfusepath{fill,stroke}
\pgfpathmoveto{\pgfpoint{271.295990pt}{171.056854pt}}
\pgflineto{\pgfpoint{280.223969pt}{171.056854pt}}
\pgflineto{\pgfpoint{280.223969pt}{164.880005pt}}
\pgfpathclose
\pgfusepath{fill,stroke}
\color[rgb]{0.136835,0.661563,0.515967}
\pgfpathmoveto{\pgfpoint{280.223969pt}{152.526306pt}}
\pgflineto{\pgfpoint{289.151978pt}{152.526306pt}}
\pgflineto{\pgfpoint{289.151978pt}{146.349472pt}}
\pgfpathclose
\pgfusepath{fill,stroke}
\color[rgb]{0.149643,0.676120,0.506924}
\pgfpathmoveto{\pgfpoint{280.223969pt}{158.703156pt}}
\pgflineto{\pgfpoint{289.151978pt}{152.526306pt}}
\pgflineto{\pgfpoint{280.223969pt}{152.526306pt}}
\pgfpathclose
\pgfusepath{fill,stroke}
\pgfpathmoveto{\pgfpoint{280.223969pt}{158.703156pt}}
\pgflineto{\pgfpoint{289.151978pt}{158.703156pt}}
\pgflineto{\pgfpoint{289.151978pt}{152.526306pt}}
\pgfpathclose
\pgfusepath{fill,stroke}
\pgfpathmoveto{\pgfpoint{280.223969pt}{164.880005pt}}
\pgflineto{\pgfpoint{289.151978pt}{158.703156pt}}
\pgflineto{\pgfpoint{280.223969pt}{158.703156pt}}
\pgfpathclose
\pgfusepath{fill,stroke}
\pgfpathmoveto{\pgfpoint{280.223969pt}{164.880005pt}}
\pgflineto{\pgfpoint{289.151978pt}{164.880005pt}}
\pgflineto{\pgfpoint{289.151978pt}{158.703156pt}}
\pgfpathclose
\pgfusepath{fill,stroke}
\color[rgb]{0.165967,0.690519,0.496752}
\pgfpathmoveto{\pgfpoint{280.223969pt}{171.056854pt}}
\pgflineto{\pgfpoint{289.151978pt}{164.880005pt}}
\pgflineto{\pgfpoint{280.223969pt}{164.880005pt}}
\pgfpathclose
\pgfusepath{fill,stroke}
\color[rgb]{0.149643,0.676120,0.506924}
\pgfpathmoveto{\pgfpoint{289.151978pt}{152.526306pt}}
\pgflineto{\pgfpoint{298.079987pt}{146.349472pt}}
\pgflineto{\pgfpoint{289.151978pt}{146.349472pt}}
\pgfpathclose
\pgfusepath{fill,stroke}
\color[rgb]{0.165967,0.690519,0.496752}
\pgfpathmoveto{\pgfpoint{289.151978pt}{158.703156pt}}
\pgflineto{\pgfpoint{298.079987pt}{158.703156pt}}
\pgflineto{\pgfpoint{298.079987pt}{152.526306pt}}
\pgfpathclose
\pgfusepath{fill,stroke}
\color[rgb]{0.185538,0.704725,0.485412}
\pgfpathmoveto{\pgfpoint{289.151978pt}{164.880005pt}}
\pgflineto{\pgfpoint{298.079987pt}{158.703156pt}}
\pgflineto{\pgfpoint{289.151978pt}{158.703156pt}}
\pgfpathclose
\pgfusepath{fill,stroke}
\color[rgb]{0.149643,0.676120,0.506924}
\pgfpathmoveto{\pgfpoint{298.079987pt}{146.349472pt}}
\pgflineto{\pgfpoint{307.007965pt}{146.349472pt}}
\pgflineto{\pgfpoint{307.007965pt}{140.172638pt}}
\pgfpathclose
\pgfusepath{fill,stroke}
\color[rgb]{0.165967,0.690519,0.496752}
\pgfpathmoveto{\pgfpoint{298.079987pt}{152.526306pt}}
\pgflineto{\pgfpoint{307.007965pt}{146.349472pt}}
\pgflineto{\pgfpoint{298.079987pt}{146.349472pt}}
\pgfpathclose
\pgfusepath{fill,stroke}
\pgfpathmoveto{\pgfpoint{298.079987pt}{152.526306pt}}
\pgflineto{\pgfpoint{307.007965pt}{152.526306pt}}
\pgflineto{\pgfpoint{307.007965pt}{146.349472pt}}
\pgfpathclose
\pgfusepath{fill,stroke}
\color[rgb]{0.185538,0.704725,0.485412}
\pgfpathmoveto{\pgfpoint{298.079987pt}{158.703156pt}}
\pgflineto{\pgfpoint{307.007965pt}{152.526306pt}}
\pgflineto{\pgfpoint{298.079987pt}{152.526306pt}}
\pgfpathclose
\pgfusepath{fill,stroke}
\pgfpathmoveto{\pgfpoint{298.079987pt}{158.703156pt}}
\pgflineto{\pgfpoint{307.007965pt}{158.703156pt}}
\pgflineto{\pgfpoint{307.007965pt}{152.526306pt}}
\pgfpathclose
\pgfusepath{fill,stroke}
\color[rgb]{0.165967,0.690519,0.496752}
\pgfpathmoveto{\pgfpoint{307.007965pt}{146.349472pt}}
\pgflineto{\pgfpoint{315.935974pt}{140.172638pt}}
\pgflineto{\pgfpoint{307.007965pt}{140.172638pt}}
\pgfpathclose
\pgfusepath{fill,stroke}
\color[rgb]{0.185538,0.704725,0.485412}
\pgfpathmoveto{\pgfpoint{307.007965pt}{152.526306pt}}
\pgflineto{\pgfpoint{315.935974pt}{152.526306pt}}
\pgflineto{\pgfpoint{315.935974pt}{146.349472pt}}
\pgfpathclose
\pgfusepath{fill,stroke}
\color[rgb]{0.208030,0.718701,0.472873}
\pgfpathmoveto{\pgfpoint{307.007965pt}{158.703156pt}}
\pgflineto{\pgfpoint{315.935974pt}{152.526306pt}}
\pgflineto{\pgfpoint{307.007965pt}{152.526306pt}}
\pgfpathclose
\pgfusepath{fill,stroke}
\color[rgb]{0.185538,0.704725,0.485412}
\pgfpathmoveto{\pgfpoint{315.935974pt}{146.349472pt}}
\pgflineto{\pgfpoint{324.863983pt}{146.349472pt}}
\pgflineto{\pgfpoint{324.863983pt}{140.172638pt}}
\pgfpathclose
\pgfusepath{fill,stroke}
\color[rgb]{0.208030,0.718701,0.472873}
\pgfpathmoveto{\pgfpoint{315.935974pt}{152.526306pt}}
\pgflineto{\pgfpoint{324.863983pt}{146.349472pt}}
\pgflineto{\pgfpoint{315.935974pt}{146.349472pt}}
\pgfpathclose
\pgfusepath{fill,stroke}
\pgfpathmoveto{\pgfpoint{315.935974pt}{152.526306pt}}
\pgflineto{\pgfpoint{324.863983pt}{152.526306pt}}
\pgflineto{\pgfpoint{324.863983pt}{146.349472pt}}
\pgfpathclose
\pgfusepath{fill,stroke}
\pgfpathmoveto{\pgfpoint{324.863983pt}{140.172638pt}}
\pgflineto{\pgfpoint{333.791992pt}{133.995789pt}}
\pgflineto{\pgfpoint{324.863983pt}{133.995789pt}}
\pgfpathclose
\pgfusepath{fill,stroke}
\pgfpathmoveto{\pgfpoint{324.863983pt}{140.172638pt}}
\pgflineto{\pgfpoint{333.791992pt}{140.172638pt}}
\pgflineto{\pgfpoint{333.791992pt}{133.995789pt}}
\pgfpathclose
\pgfusepath{fill,stroke}
\pgfpathmoveto{\pgfpoint{324.863983pt}{146.349472pt}}
\pgflineto{\pgfpoint{333.791992pt}{140.172638pt}}
\pgflineto{\pgfpoint{324.863983pt}{140.172638pt}}
\pgfpathclose
\pgfusepath{fill,stroke}
\pgfpathmoveto{\pgfpoint{324.863983pt}{146.349472pt}}
\pgflineto{\pgfpoint{333.791992pt}{146.349472pt}}
\pgflineto{\pgfpoint{333.791992pt}{140.172638pt}}
\pgfpathclose
\pgfusepath{fill,stroke}
\color[rgb]{0.233127,0.732406,0.459106}
\pgfpathmoveto{\pgfpoint{324.863983pt}{152.526306pt}}
\pgflineto{\pgfpoint{333.791992pt}{146.349472pt}}
\pgflineto{\pgfpoint{324.863983pt}{146.349472pt}}
\pgfpathclose
\pgfusepath{fill,stroke}
\pgfpathmoveto{\pgfpoint{333.791992pt}{140.172638pt}}
\pgflineto{\pgfpoint{342.719971pt}{140.172638pt}}
\pgflineto{\pgfpoint{342.719971pt}{133.995789pt}}
\pgfpathclose
\pgfusepath{fill,stroke}
\pgfpathmoveto{\pgfpoint{333.791992pt}{146.349472pt}}
\pgflineto{\pgfpoint{342.719971pt}{140.172638pt}}
\pgflineto{\pgfpoint{333.791992pt}{140.172638pt}}
\pgfpathclose
\pgfusepath{fill,stroke}
\pgfpathmoveto{\pgfpoint{333.791992pt}{146.349472pt}}
\pgflineto{\pgfpoint{342.719971pt}{146.349472pt}}
\pgflineto{\pgfpoint{342.719971pt}{140.172638pt}}
\pgfpathclose
\pgfusepath{fill,stroke}
\pgfpathmoveto{\pgfpoint{342.719971pt}{133.995789pt}}
\pgflineto{\pgfpoint{351.647980pt}{127.818947pt}}
\pgflineto{\pgfpoint{342.719971pt}{127.818947pt}}
\pgfpathclose
\pgfusepath{fill,stroke}
\pgfpathmoveto{\pgfpoint{342.719971pt}{133.995789pt}}
\pgflineto{\pgfpoint{351.647980pt}{133.995789pt}}
\pgflineto{\pgfpoint{351.647980pt}{127.818947pt}}
\pgfpathclose
\pgfusepath{fill,stroke}
\color[rgb]{0.260531,0.745802,0.444096}
\pgfpathmoveto{\pgfpoint{342.719971pt}{140.172638pt}}
\pgflineto{\pgfpoint{351.647980pt}{133.995789pt}}
\pgflineto{\pgfpoint{342.719971pt}{133.995789pt}}
\pgfpathclose
\pgfusepath{fill,stroke}
\pgfpathmoveto{\pgfpoint{342.719971pt}{140.172638pt}}
\pgflineto{\pgfpoint{351.647980pt}{140.172638pt}}
\pgflineto{\pgfpoint{351.647980pt}{133.995789pt}}
\pgfpathclose
\pgfusepath{fill,stroke}
\color[rgb]{0.290001,0.758846,0.427826}
\pgfpathmoveto{\pgfpoint{342.719971pt}{146.349472pt}}
\pgflineto{\pgfpoint{351.647980pt}{140.172638pt}}
\pgflineto{\pgfpoint{342.719971pt}{140.172638pt}}
\pgfpathclose
\pgfusepath{fill,stroke}
\color[rgb]{0.260531,0.745802,0.444096}
\pgfpathmoveto{\pgfpoint{351.647980pt}{133.995789pt}}
\pgflineto{\pgfpoint{360.575958pt}{133.995789pt}}
\pgflineto{\pgfpoint{360.575958pt}{127.818947pt}}
\pgfpathclose
\pgfusepath{fill,stroke}
\color[rgb]{0.290001,0.758846,0.427826}
\pgfpathmoveto{\pgfpoint{351.647980pt}{140.172638pt}}
\pgflineto{\pgfpoint{360.575958pt}{133.995789pt}}
\pgflineto{\pgfpoint{351.647980pt}{133.995789pt}}
\pgfpathclose
\pgfusepath{fill,stroke}
\color[rgb]{0.233127,0.732406,0.459106}
\pgfpathmoveto{\pgfpoint{360.575958pt}{121.642097pt}}
\pgflineto{\pgfpoint{369.503998pt}{121.642097pt}}
\pgflineto{\pgfpoint{369.503998pt}{115.465263pt}}
\pgfpathclose
\pgfusepath{fill,stroke}
\color[rgb]{0.260531,0.745802,0.444096}
\pgfpathmoveto{\pgfpoint{360.575958pt}{127.818947pt}}
\pgflineto{\pgfpoint{369.503998pt}{121.642097pt}}
\pgflineto{\pgfpoint{360.575958pt}{121.642097pt}}
\pgfpathclose
\pgfusepath{fill,stroke}
\pgfpathmoveto{\pgfpoint{360.575958pt}{127.818947pt}}
\pgflineto{\pgfpoint{369.503998pt}{127.818947pt}}
\pgflineto{\pgfpoint{369.503998pt}{121.642097pt}}
\pgfpathclose
\pgfusepath{fill,stroke}
\color[rgb]{0.290001,0.758846,0.427826}
\pgfpathmoveto{\pgfpoint{360.575958pt}{133.995789pt}}
\pgflineto{\pgfpoint{369.503998pt}{127.818947pt}}
\pgflineto{\pgfpoint{360.575958pt}{127.818947pt}}
\pgfpathclose
\pgfusepath{fill,stroke}
\pgfpathmoveto{\pgfpoint{360.575958pt}{133.995789pt}}
\pgflineto{\pgfpoint{369.503998pt}{133.995789pt}}
\pgflineto{\pgfpoint{369.503998pt}{127.818947pt}}
\pgfpathclose
\pgfusepath{fill,stroke}
\color[rgb]{0.260531,0.745802,0.444096}
\pgfpathmoveto{\pgfpoint{369.503998pt}{121.642097pt}}
\pgflineto{\pgfpoint{378.431976pt}{115.465263pt}}
\pgflineto{\pgfpoint{369.503998pt}{115.465263pt}}
\pgfpathclose
\pgfusepath{fill,stroke}
\color[rgb]{0.290001,0.758846,0.427826}
\pgfpathmoveto{\pgfpoint{369.503998pt}{127.818947pt}}
\pgflineto{\pgfpoint{378.431976pt}{127.818947pt}}
\pgflineto{\pgfpoint{378.431976pt}{121.642097pt}}
\pgfpathclose
\pgfusepath{fill,stroke}
\color[rgb]{0.321330,0.771498,0.410293}
\pgfpathmoveto{\pgfpoint{369.503998pt}{133.995789pt}}
\pgflineto{\pgfpoint{378.431976pt}{127.818947pt}}
\pgflineto{\pgfpoint{369.503998pt}{127.818947pt}}
\pgfpathclose
\pgfusepath{fill,stroke}
\color[rgb]{0.290001,0.758846,0.427826}
\pgfpathmoveto{\pgfpoint{378.431976pt}{121.642097pt}}
\pgflineto{\pgfpoint{387.359985pt}{121.642097pt}}
\pgflineto{\pgfpoint{387.359985pt}{115.465263pt}}
\pgfpathclose
\pgfusepath{fill,stroke}
\color[rgb]{0.321330,0.771498,0.410293}
\pgfpathmoveto{\pgfpoint{378.431976pt}{127.818947pt}}
\pgflineto{\pgfpoint{387.359985pt}{121.642097pt}}
\pgflineto{\pgfpoint{378.431976pt}{121.642097pt}}
\pgfpathclose
\pgfusepath{fill,stroke}
\pgfpathmoveto{\pgfpoint{378.431976pt}{127.818947pt}}
\pgflineto{\pgfpoint{387.359985pt}{127.818947pt}}
\pgflineto{\pgfpoint{387.359985pt}{121.642097pt}}
\pgfpathclose
\pgfusepath{fill,stroke}
\color[rgb]{0.354355,0.783714,0.391488}
\pgfpathmoveto{\pgfpoint{387.359985pt}{121.642097pt}}
\pgflineto{\pgfpoint{396.287964pt}{115.465263pt}}
\pgflineto{\pgfpoint{387.359985pt}{115.465263pt}}
\pgfpathclose
\pgfusepath{fill,stroke}
\pgfpathmoveto{\pgfpoint{387.359985pt}{121.642097pt}}
\pgflineto{\pgfpoint{396.287964pt}{121.642097pt}}
\pgflineto{\pgfpoint{396.287964pt}{115.465263pt}}
\pgfpathclose
\pgfusepath{fill,stroke}
\pgfpathmoveto{\pgfpoint{387.359985pt}{127.818947pt}}
\pgflineto{\pgfpoint{396.287964pt}{121.642097pt}}
\pgflineto{\pgfpoint{387.359985pt}{121.642097pt}}
\pgfpathclose
\pgfusepath{fill,stroke}
\color[rgb]{0.321330,0.771498,0.410293}
\pgfpathmoveto{\pgfpoint{396.287964pt}{109.288422pt}}
\pgflineto{\pgfpoint{405.216003pt}{109.288422pt}}
\pgflineto{\pgfpoint{405.216003pt}{103.111580pt}}
\pgfpathclose
\pgfusepath{fill,stroke}
\color[rgb]{0.354355,0.783714,0.391488}
\pgfpathmoveto{\pgfpoint{396.287964pt}{115.465263pt}}
\pgflineto{\pgfpoint{405.216003pt}{115.465263pt}}
\pgflineto{\pgfpoint{405.216003pt}{109.288422pt}}
\pgfpathclose
\pgfusepath{fill,stroke}
\color[rgb]{0.388930,0.795453,0.371421}
\pgfpathmoveto{\pgfpoint{396.287964pt}{121.642097pt}}
\pgflineto{\pgfpoint{405.216003pt}{115.465263pt}}
\pgflineto{\pgfpoint{396.287964pt}{115.465263pt}}
\pgfpathclose
\pgfusepath{fill,stroke}
\pgfpathmoveto{\pgfpoint{396.287964pt}{121.642097pt}}
\pgflineto{\pgfpoint{405.216003pt}{121.642097pt}}
\pgflineto{\pgfpoint{405.216003pt}{115.465263pt}}
\pgfpathclose
\pgfusepath{fill,stroke}
\color[rgb]{0.354355,0.783714,0.391488}
\pgfpathmoveto{\pgfpoint{405.216003pt}{109.288422pt}}
\pgflineto{\pgfpoint{414.143982pt}{103.111580pt}}
\pgflineto{\pgfpoint{405.216003pt}{103.111580pt}}
\pgfpathclose
\pgfusepath{fill,stroke}
\color[rgb]{0.388930,0.795453,0.371421}
\pgfpathmoveto{\pgfpoint{405.216003pt}{115.465263pt}}
\pgflineto{\pgfpoint{414.143982pt}{109.288422pt}}
\pgflineto{\pgfpoint{405.216003pt}{109.288422pt}}
\pgfpathclose
\pgfusepath{fill,stroke}
\pgfpathmoveto{\pgfpoint{405.216003pt}{115.465263pt}}
\pgflineto{\pgfpoint{414.143982pt}{115.465263pt}}
\pgflineto{\pgfpoint{414.143982pt}{109.288422pt}}
\pgfpathclose
\pgfusepath{fill,stroke}
\color[rgb]{0.424933,0.806674,0.350099}
\pgfpathmoveto{\pgfpoint{405.216003pt}{121.642097pt}}
\pgflineto{\pgfpoint{414.143982pt}{115.465263pt}}
\pgflineto{\pgfpoint{405.216003pt}{115.465263pt}}
\pgfpathclose
\pgfusepath{fill,stroke}
\color[rgb]{0.354355,0.783714,0.391488}
\pgfpathmoveto{\pgfpoint{414.143982pt}{103.111580pt}}
\pgflineto{\pgfpoint{423.071960pt}{103.111580pt}}
\pgflineto{\pgfpoint{423.071960pt}{96.934731pt}}
\pgfpathclose
\pgfusepath{fill,stroke}
\color[rgb]{0.388930,0.795453,0.371421}
\pgfpathmoveto{\pgfpoint{414.143982pt}{109.288422pt}}
\pgflineto{\pgfpoint{423.071960pt}{109.288422pt}}
\pgflineto{\pgfpoint{423.071960pt}{103.111580pt}}
\pgfpathclose
\pgfusepath{fill,stroke}
\color[rgb]{0.424933,0.806674,0.350099}
\pgfpathmoveto{\pgfpoint{414.143982pt}{115.465263pt}}
\pgflineto{\pgfpoint{423.071960pt}{109.288422pt}}
\pgflineto{\pgfpoint{414.143982pt}{109.288422pt}}
\pgfpathclose
\pgfusepath{fill,stroke}
\color[rgb]{0.388930,0.795453,0.371421}
\pgfpathmoveto{\pgfpoint{423.071960pt}{103.111580pt}}
\pgflineto{\pgfpoint{432.000000pt}{96.934731pt}}
\pgflineto{\pgfpoint{423.071960pt}{96.934731pt}}
\pgfpathclose
\pgfusepath{fill,stroke}
\pgfpathmoveto{\pgfpoint{423.071960pt}{103.111580pt}}
\pgflineto{\pgfpoint{432.000000pt}{103.111580pt}}
\pgflineto{\pgfpoint{432.000000pt}{96.934731pt}}
\pgfpathclose
\pgfusepath{fill,stroke}
\color[rgb]{0.424933,0.806674,0.350099}
\pgfpathmoveto{\pgfpoint{423.071960pt}{109.288422pt}}
\pgflineto{\pgfpoint{432.000000pt}{103.111580pt}}
\pgflineto{\pgfpoint{423.071960pt}{103.111580pt}}
\pgfpathclose
\pgfusepath{fill,stroke}
\pgfpathmoveto{\pgfpoint{423.071960pt}{109.288422pt}}
\pgflineto{\pgfpoint{432.000000pt}{109.288422pt}}
\pgflineto{\pgfpoint{432.000000pt}{103.111580pt}}
\pgfpathclose
\pgfusepath{fill,stroke}
\pgfpathmoveto{\pgfpoint{432.000000pt}{103.111580pt}}
\pgflineto{\pgfpoint{440.927979pt}{96.934731pt}}
\pgflineto{\pgfpoint{432.000000pt}{96.934731pt}}
\pgfpathclose
\pgfusepath{fill,stroke}
\pgfpathmoveto{\pgfpoint{432.000000pt}{103.111580pt}}
\pgflineto{\pgfpoint{440.927979pt}{103.111580pt}}
\pgflineto{\pgfpoint{440.927979pt}{96.934731pt}}
\pgfpathclose
\pgfusepath{fill,stroke}
\color[rgb]{0.462247,0.817338,0.327545}
\pgfpathmoveto{\pgfpoint{432.000000pt}{109.288422pt}}
\pgflineto{\pgfpoint{440.927979pt}{103.111580pt}}
\pgflineto{\pgfpoint{432.000000pt}{103.111580pt}}
\pgfpathclose
\pgfusepath{fill,stroke}
\pgfpathmoveto{\pgfpoint{440.927979pt}{96.934731pt}}
\pgflineto{\pgfpoint{449.855957pt}{96.934731pt}}
\pgflineto{\pgfpoint{449.855957pt}{90.757896pt}}
\pgfpathclose
\pgfusepath{fill,stroke}
\color[rgb]{0.500754,0.827409,0.303799}
\pgfpathmoveto{\pgfpoint{440.927979pt}{103.111580pt}}
\pgflineto{\pgfpoint{449.855957pt}{96.934731pt}}
\pgflineto{\pgfpoint{440.927979pt}{96.934731pt}}
\pgfpathclose
\pgfusepath{fill,stroke}
\pgfpathmoveto{\pgfpoint{440.927979pt}{103.111580pt}}
\pgflineto{\pgfpoint{449.855957pt}{103.111580pt}}
\pgflineto{\pgfpoint{449.855957pt}{96.934731pt}}
\pgfpathclose
\pgfusepath{fill,stroke}
\pgfpathmoveto{\pgfpoint{449.855957pt}{96.934731pt}}
\pgflineto{\pgfpoint{458.783936pt}{90.757896pt}}
\pgflineto{\pgfpoint{449.855957pt}{90.757896pt}}
\pgfpathclose
\pgfusepath{fill,stroke}
\pgfpathmoveto{\pgfpoint{449.855957pt}{96.934731pt}}
\pgflineto{\pgfpoint{458.783936pt}{96.934731pt}}
\pgflineto{\pgfpoint{458.783936pt}{90.757896pt}}
\pgfpathclose
\pgfusepath{fill,stroke}
\color[rgb]{0.540337,0.836858,0.278917}
\pgfpathmoveto{\pgfpoint{449.855957pt}{103.111580pt}}
\pgflineto{\pgfpoint{458.783936pt}{96.934731pt}}
\pgflineto{\pgfpoint{449.855957pt}{96.934731pt}}
\pgfpathclose
\pgfusepath{fill,stroke}
\color[rgb]{0.500754,0.827409,0.303799}
\pgfpathmoveto{\pgfpoint{458.783936pt}{90.757896pt}}
\pgflineto{\pgfpoint{467.711975pt}{90.757896pt}}
\pgflineto{\pgfpoint{467.711975pt}{84.581039pt}}
\pgfpathclose
\pgfusepath{fill,stroke}
\color[rgb]{0.540337,0.836858,0.278917}
\pgfpathmoveto{\pgfpoint{458.783936pt}{96.934731pt}}
\pgflineto{\pgfpoint{467.711975pt}{90.757896pt}}
\pgflineto{\pgfpoint{458.783936pt}{90.757896pt}}
\pgfpathclose
\pgfusepath{fill,stroke}
\pgfpathmoveto{\pgfpoint{458.783936pt}{96.934731pt}}
\pgflineto{\pgfpoint{467.711975pt}{96.934731pt}}
\pgflineto{\pgfpoint{467.711975pt}{90.757896pt}}
\pgfpathclose
\pgfusepath{fill,stroke}
\pgfpathmoveto{\pgfpoint{467.711975pt}{90.757896pt}}
\pgflineto{\pgfpoint{476.639954pt}{84.581039pt}}
\pgflineto{\pgfpoint{467.711975pt}{84.581039pt}}
\pgfpathclose
\pgfusepath{fill,stroke}
\pgfpathmoveto{\pgfpoint{467.711975pt}{90.757896pt}}
\pgflineto{\pgfpoint{476.639954pt}{90.757896pt}}
\pgflineto{\pgfpoint{476.639954pt}{84.581039pt}}
\pgfpathclose
\pgfusepath{fill,stroke}
\color[rgb]{0.580861,0.845663,0.253001}
\pgfpathmoveto{\pgfpoint{467.711975pt}{96.934731pt}}
\pgflineto{\pgfpoint{476.639954pt}{90.757896pt}}
\pgflineto{\pgfpoint{467.711975pt}{90.757896pt}}
\pgfpathclose
\pgfusepath{fill,stroke}
\color[rgb]{0.500754,0.827409,0.303799}
\pgfpathmoveto{\pgfpoint{476.639954pt}{78.404205pt}}
\pgflineto{\pgfpoint{485.567963pt}{78.404205pt}}
\pgflineto{\pgfpoint{485.567963pt}{72.227356pt}}
\pgfpathclose
\pgfusepath{fill,stroke}
\color[rgb]{0.540337,0.836858,0.278917}
\pgfpathmoveto{\pgfpoint{476.639954pt}{84.581039pt}}
\pgflineto{\pgfpoint{485.567963pt}{84.581039pt}}
\pgflineto{\pgfpoint{485.567963pt}{78.404205pt}}
\pgfpathclose
\pgfusepath{fill,stroke}
\color[rgb]{0.580861,0.845663,0.253001}
\pgfpathmoveto{\pgfpoint{476.639954pt}{90.757896pt}}
\pgflineto{\pgfpoint{485.567963pt}{84.581039pt}}
\pgflineto{\pgfpoint{476.639954pt}{84.581039pt}}
\pgfpathclose
\pgfusepath{fill,stroke}
\pgfpathmoveto{\pgfpoint{476.639954pt}{90.757896pt}}
\pgflineto{\pgfpoint{485.567963pt}{90.757896pt}}
\pgflineto{\pgfpoint{485.567963pt}{84.581039pt}}
\pgfpathclose
\pgfusepath{fill,stroke}
\pgfpathmoveto{\pgfpoint{485.567963pt}{78.404205pt}}
\pgflineto{\pgfpoint{494.495972pt}{72.227356pt}}
\pgflineto{\pgfpoint{485.567963pt}{72.227356pt}}
\pgfpathclose
\pgfusepath{fill,stroke}
\pgfpathmoveto{\pgfpoint{485.567963pt}{78.404205pt}}
\pgflineto{\pgfpoint{494.495972pt}{78.404205pt}}
\pgflineto{\pgfpoint{494.495972pt}{72.227356pt}}
\pgfpathclose
\pgfusepath{fill,stroke}
\pgfpathmoveto{\pgfpoint{485.567963pt}{84.581039pt}}
\pgflineto{\pgfpoint{494.495972pt}{78.404205pt}}
\pgflineto{\pgfpoint{485.567963pt}{78.404205pt}}
\pgfpathclose
\pgfusepath{fill,stroke}
\pgfpathmoveto{\pgfpoint{485.567963pt}{84.581039pt}}
\pgflineto{\pgfpoint{494.495972pt}{84.581039pt}}
\pgflineto{\pgfpoint{494.495972pt}{78.404205pt}}
\pgfpathclose
\pgfusepath{fill,stroke}
\pgfpathmoveto{\pgfpoint{494.495972pt}{72.227356pt}}
\pgflineto{\pgfpoint{503.423981pt}{72.227356pt}}
\pgflineto{\pgfpoint{503.423981pt}{66.050522pt}}
\pgfpathclose
\pgfusepath{fill,stroke}
\color[rgb]{0.622171,0.853816,0.226224}
\pgfpathmoveto{\pgfpoint{494.495972pt}{78.404205pt}}
\pgflineto{\pgfpoint{503.423981pt}{72.227356pt}}
\pgflineto{\pgfpoint{494.495972pt}{72.227356pt}}
\pgfpathclose
\pgfusepath{fill,stroke}
\pgfpathmoveto{\pgfpoint{494.495972pt}{78.404205pt}}
\pgflineto{\pgfpoint{503.423981pt}{78.404205pt}}
\pgflineto{\pgfpoint{503.423981pt}{72.227356pt}}
\pgfpathclose
\pgfusepath{fill,stroke}
\color[rgb]{0.290001,0.758846,0.427826}
\pgfpathmoveto{\pgfpoint{342.719971pt}{152.526306pt}}
\pgflineto{\pgfpoint{351.647980pt}{146.349472pt}}
\pgflineto{\pgfpoint{342.719971pt}{146.349472pt}}
\pgfpathclose
\pgfusepath{fill,stroke}
\pgfpathmoveto{\pgfpoint{351.647980pt}{140.172638pt}}
\pgflineto{\pgfpoint{360.575958pt}{140.172638pt}}
\pgflineto{\pgfpoint{360.575958pt}{133.995789pt}}
\pgfpathclose
\pgfusepath{fill,stroke}
\color[rgb]{0.321330,0.771498,0.410293}
\pgfpathmoveto{\pgfpoint{360.575958pt}{140.172638pt}}
\pgflineto{\pgfpoint{369.503998pt}{133.995789pt}}
\pgflineto{\pgfpoint{360.575958pt}{133.995789pt}}
\pgfpathclose
\pgfusepath{fill,stroke}
\pgfpathmoveto{\pgfpoint{360.575958pt}{140.172638pt}}
\pgflineto{\pgfpoint{369.503998pt}{140.172638pt}}
\pgflineto{\pgfpoint{369.503998pt}{133.995789pt}}
\pgfpathclose
\pgfusepath{fill,stroke}
\pgfpathmoveto{\pgfpoint{369.503998pt}{133.995789pt}}
\pgflineto{\pgfpoint{378.431976pt}{133.995789pt}}
\pgflineto{\pgfpoint{378.431976pt}{127.818947pt}}
\pgfpathclose
\pgfusepath{fill,stroke}
\color[rgb]{0.354355,0.783714,0.391488}
\pgfpathmoveto{\pgfpoint{369.503998pt}{140.172638pt}}
\pgflineto{\pgfpoint{378.431976pt}{133.995789pt}}
\pgflineto{\pgfpoint{369.503998pt}{133.995789pt}}
\pgfpathclose
\pgfusepath{fill,stroke}
\pgfpathmoveto{\pgfpoint{378.431976pt}{133.995789pt}}
\pgflineto{\pgfpoint{387.359985pt}{127.818947pt}}
\pgflineto{\pgfpoint{378.431976pt}{127.818947pt}}
\pgfpathclose
\pgfusepath{fill,stroke}
\pgfpathmoveto{\pgfpoint{378.431976pt}{133.995789pt}}
\pgflineto{\pgfpoint{387.359985pt}{133.995789pt}}
\pgflineto{\pgfpoint{387.359985pt}{127.818947pt}}
\pgfpathclose
\pgfusepath{fill,stroke}
\pgfpathmoveto{\pgfpoint{387.359985pt}{127.818947pt}}
\pgflineto{\pgfpoint{396.287964pt}{127.818947pt}}
\pgflineto{\pgfpoint{396.287964pt}{121.642097pt}}
\pgfpathclose
\pgfusepath{fill,stroke}
\color[rgb]{0.388930,0.795453,0.371421}
\pgfpathmoveto{\pgfpoint{387.359985pt}{133.995789pt}}
\pgflineto{\pgfpoint{396.287964pt}{127.818947pt}}
\pgflineto{\pgfpoint{387.359985pt}{127.818947pt}}
\pgfpathclose
\pgfusepath{fill,stroke}
\pgfpathmoveto{\pgfpoint{396.287964pt}{127.818947pt}}
\pgflineto{\pgfpoint{405.216003pt}{121.642097pt}}
\pgflineto{\pgfpoint{396.287964pt}{121.642097pt}}
\pgfpathclose
\pgfusepath{fill,stroke}
\pgfpathmoveto{\pgfpoint{396.287964pt}{127.818947pt}}
\pgflineto{\pgfpoint{405.216003pt}{127.818947pt}}
\pgflineto{\pgfpoint{405.216003pt}{121.642097pt}}
\pgfpathclose
\pgfusepath{fill,stroke}
\color[rgb]{0.462247,0.817338,0.327545}
\pgfpathmoveto{\pgfpoint{405.216003pt}{127.818947pt}}
\pgflineto{\pgfpoint{414.143982pt}{121.642097pt}}
\pgflineto{\pgfpoint{405.216003pt}{121.642097pt}}
\pgfpathclose
\pgfusepath{fill,stroke}
\color[rgb]{0.424933,0.806674,0.350099}
\pgfpathmoveto{\pgfpoint{414.143982pt}{115.465263pt}}
\pgflineto{\pgfpoint{423.071960pt}{115.465263pt}}
\pgflineto{\pgfpoint{423.071960pt}{109.288422pt}}
\pgfpathclose
\pgfusepath{fill,stroke}
\color[rgb]{0.462247,0.817338,0.327545}
\pgfpathmoveto{\pgfpoint{423.071960pt}{115.465263pt}}
\pgflineto{\pgfpoint{432.000000pt}{109.288422pt}}
\pgflineto{\pgfpoint{423.071960pt}{109.288422pt}}
\pgfpathclose
\pgfusepath{fill,stroke}
\pgfpathmoveto{\pgfpoint{432.000000pt}{109.288422pt}}
\pgflineto{\pgfpoint{440.927979pt}{109.288422pt}}
\pgflineto{\pgfpoint{440.927979pt}{103.111580pt}}
\pgfpathclose
\pgfusepath{fill,stroke}
\color[rgb]{0.500754,0.827409,0.303799}
\pgfpathmoveto{\pgfpoint{440.927979pt}{109.288422pt}}
\pgflineto{\pgfpoint{449.855957pt}{103.111580pt}}
\pgflineto{\pgfpoint{440.927979pt}{103.111580pt}}
\pgfpathclose
\pgfusepath{fill,stroke}
\pgfpathmoveto{\pgfpoint{440.927979pt}{109.288422pt}}
\pgflineto{\pgfpoint{449.855957pt}{109.288422pt}}
\pgflineto{\pgfpoint{449.855957pt}{103.111580pt}}
\pgfpathclose
\pgfusepath{fill,stroke}
\color[rgb]{0.540337,0.836858,0.278917}
\pgfpathmoveto{\pgfpoint{449.855957pt}{103.111580pt}}
\pgflineto{\pgfpoint{458.783936pt}{103.111580pt}}
\pgflineto{\pgfpoint{458.783936pt}{96.934731pt}}
\pgfpathclose
\pgfusepath{fill,stroke}
\pgfpathmoveto{\pgfpoint{449.855957pt}{109.288422pt}}
\pgflineto{\pgfpoint{458.783936pt}{103.111580pt}}
\pgflineto{\pgfpoint{449.855957pt}{103.111580pt}}
\pgfpathclose
\pgfusepath{fill,stroke}
\color[rgb]{0.580861,0.845663,0.253001}
\pgfpathmoveto{\pgfpoint{458.783936pt}{103.111580pt}}
\pgflineto{\pgfpoint{467.711975pt}{96.934731pt}}
\pgflineto{\pgfpoint{458.783936pt}{96.934731pt}}
\pgfpathclose
\pgfusepath{fill,stroke}
\pgfpathmoveto{\pgfpoint{458.783936pt}{103.111580pt}}
\pgflineto{\pgfpoint{467.711975pt}{103.111580pt}}
\pgflineto{\pgfpoint{467.711975pt}{96.934731pt}}
\pgfpathclose
\pgfusepath{fill,stroke}
\color[rgb]{0.622171,0.853816,0.226224}
\pgfpathmoveto{\pgfpoint{467.711975pt}{103.111580pt}}
\pgflineto{\pgfpoint{476.639954pt}{96.934731pt}}
\pgflineto{\pgfpoint{467.711975pt}{96.934731pt}}
\pgfpathclose
\pgfusepath{fill,stroke}
\pgfpathmoveto{\pgfpoint{476.639954pt}{96.934731pt}}
\pgflineto{\pgfpoint{485.567963pt}{96.934731pt}}
\pgflineto{\pgfpoint{485.567963pt}{90.757896pt}}
\pgfpathclose
\pgfusepath{fill,stroke}
\color[rgb]{0.233127,0.732406,0.459106}
\pgfpathmoveto{\pgfpoint{315.935974pt}{158.703156pt}}
\pgflineto{\pgfpoint{324.863983pt}{158.703156pt}}
\pgflineto{\pgfpoint{324.863983pt}{152.526306pt}}
\pgfpathclose
\pgfusepath{fill,stroke}
\color[rgb]{0.260531,0.745802,0.444096}
\pgfpathmoveto{\pgfpoint{315.935974pt}{164.880005pt}}
\pgflineto{\pgfpoint{324.863983pt}{158.703156pt}}
\pgflineto{\pgfpoint{315.935974pt}{158.703156pt}}
\pgfpathclose
\pgfusepath{fill,stroke}
\pgfpathmoveto{\pgfpoint{324.863983pt}{158.703156pt}}
\pgflineto{\pgfpoint{333.791992pt}{158.703156pt}}
\pgflineto{\pgfpoint{333.791992pt}{152.526306pt}}
\pgfpathclose
\pgfusepath{fill,stroke}
\pgfpathmoveto{\pgfpoint{333.791992pt}{152.526306pt}}
\pgflineto{\pgfpoint{342.719971pt}{152.526306pt}}
\pgflineto{\pgfpoint{342.719971pt}{146.349472pt}}
\pgfpathclose
\pgfusepath{fill,stroke}
\color[rgb]{0.290001,0.758846,0.427826}
\pgfpathmoveto{\pgfpoint{333.791992pt}{158.703156pt}}
\pgflineto{\pgfpoint{342.719971pt}{152.526306pt}}
\pgflineto{\pgfpoint{333.791992pt}{152.526306pt}}
\pgfpathclose
\pgfusepath{fill,stroke}
\color[rgb]{0.165967,0.690519,0.496752}
\pgfpathmoveto{\pgfpoint{280.223969pt}{171.056854pt}}
\pgflineto{\pgfpoint{289.151978pt}{171.056854pt}}
\pgflineto{\pgfpoint{289.151978pt}{164.880005pt}}
\pgfpathclose
\pgfusepath{fill,stroke}
\color[rgb]{0.185538,0.704725,0.485412}
\pgfpathmoveto{\pgfpoint{280.223969pt}{177.233673pt}}
\pgflineto{\pgfpoint{289.151978pt}{171.056854pt}}
\pgflineto{\pgfpoint{280.223969pt}{171.056854pt}}
\pgfpathclose
\pgfusepath{fill,stroke}
\color[rgb]{0.208030,0.718701,0.472873}
\pgfpathmoveto{\pgfpoint{298.079987pt}{164.880005pt}}
\pgflineto{\pgfpoint{307.007965pt}{164.880005pt}}
\pgflineto{\pgfpoint{307.007965pt}{158.703156pt}}
\pgfpathclose
\pgfusepath{fill,stroke}
\pgfpathmoveto{\pgfpoint{298.079987pt}{171.056854pt}}
\pgflineto{\pgfpoint{307.007965pt}{164.880005pt}}
\pgflineto{\pgfpoint{298.079987pt}{164.880005pt}}
\pgfpathclose
\pgfusepath{fill,stroke}
\color[rgb]{0.127668,0.646882,0.523924}
\pgfpathmoveto{\pgfpoint{235.583969pt}{189.587372pt}}
\pgflineto{\pgfpoint{244.511993pt}{189.587372pt}}
\pgflineto{\pgfpoint{244.511993pt}{183.410522pt}}
\pgfpathclose
\pgfusepath{fill,stroke}
\color[rgb]{0.136835,0.661563,0.515967}
\pgfpathmoveto{\pgfpoint{235.583969pt}{195.764206pt}}
\pgflineto{\pgfpoint{244.511993pt}{189.587372pt}}
\pgflineto{\pgfpoint{235.583969pt}{189.587372pt}}
\pgfpathclose
\pgfusepath{fill,stroke}
\pgfpathmoveto{\pgfpoint{244.511993pt}{183.410522pt}}
\pgflineto{\pgfpoint{253.440002pt}{183.410522pt}}
\pgflineto{\pgfpoint{253.440002pt}{177.233673pt}}
\pgfpathclose
\pgfusepath{fill,stroke}
\color[rgb]{0.149643,0.676120,0.506924}
\pgfpathmoveto{\pgfpoint{244.511993pt}{189.587372pt}}
\pgflineto{\pgfpoint{253.440002pt}{183.410522pt}}
\pgflineto{\pgfpoint{244.511993pt}{183.410522pt}}
\pgfpathclose
\pgfusepath{fill,stroke}
\pgfpathmoveto{\pgfpoint{262.367981pt}{177.233673pt}}
\pgflineto{\pgfpoint{271.295990pt}{177.233673pt}}
\pgflineto{\pgfpoint{271.295990pt}{171.056854pt}}
\pgfpathclose
\pgfusepath{fill,stroke}
\color[rgb]{0.165967,0.690519,0.496752}
\pgfpathmoveto{\pgfpoint{262.367981pt}{183.410522pt}}
\pgflineto{\pgfpoint{271.295990pt}{177.233673pt}}
\pgflineto{\pgfpoint{262.367981pt}{177.233673pt}}
\pgfpathclose
\pgfusepath{fill,stroke}
\color[rgb]{0.119627,0.617266,0.536796}
\pgfpathmoveto{\pgfpoint{199.871979pt}{201.941055pt}}
\pgflineto{\pgfpoint{208.799988pt}{201.941055pt}}
\pgflineto{\pgfpoint{208.799988pt}{195.764206pt}}
\pgfpathclose
\pgfusepath{fill,stroke}
\color[rgb]{0.122046,0.632107,0.530848}
\pgfpathmoveto{\pgfpoint{199.871979pt}{208.117905pt}}
\pgflineto{\pgfpoint{208.799988pt}{201.941055pt}}
\pgflineto{\pgfpoint{199.871979pt}{201.941055pt}}
\pgfpathclose
\pgfusepath{fill,stroke}
\pgfpathmoveto{\pgfpoint{217.727982pt}{195.764206pt}}
\pgflineto{\pgfpoint{226.655975pt}{195.764206pt}}
\pgflineto{\pgfpoint{226.655975pt}{189.587372pt}}
\pgfpathclose
\pgfusepath{fill,stroke}
\color[rgb]{0.127668,0.646882,0.523924}
\pgfpathmoveto{\pgfpoint{217.727982pt}{201.941055pt}}
\pgflineto{\pgfpoint{226.655975pt}{195.764206pt}}
\pgflineto{\pgfpoint{217.727982pt}{195.764206pt}}
\pgfpathclose
\pgfusepath{fill,stroke}
\color[rgb]{0.119872,0.602382,0.541831}
\pgfpathmoveto{\pgfpoint{182.015991pt}{214.294739pt}}
\pgflineto{\pgfpoint{190.943985pt}{214.294739pt}}
\pgflineto{\pgfpoint{190.943985pt}{208.117905pt}}
\pgfpathclose
\pgfusepath{fill,stroke}
\color[rgb]{0.119627,0.617266,0.536796}
\pgfpathmoveto{\pgfpoint{190.943985pt}{208.117905pt}}
\pgflineto{\pgfpoint{199.871979pt}{201.941055pt}}
\pgflineto{\pgfpoint{190.943985pt}{201.941055pt}}
\pgfpathclose
\pgfusepath{fill,stroke}
\pgfpathmoveto{\pgfpoint{190.943985pt}{208.117905pt}}
\pgflineto{\pgfpoint{199.871979pt}{208.117905pt}}
\pgflineto{\pgfpoint{199.871979pt}{201.941055pt}}
\pgfpathclose
\pgfusepath{fill,stroke}
\pgfpathmoveto{\pgfpoint{190.943985pt}{214.294739pt}}
\pgflineto{\pgfpoint{199.871979pt}{208.117905pt}}
\pgflineto{\pgfpoint{190.943985pt}{208.117905pt}}
\pgfpathclose
\pgfusepath{fill,stroke}
\pgfpathmoveto{\pgfpoint{190.943985pt}{214.294739pt}}
\pgflineto{\pgfpoint{199.871979pt}{214.294739pt}}
\pgflineto{\pgfpoint{199.871979pt}{208.117905pt}}
\pgfpathclose
\pgfusepath{fill,stroke}
\color[rgb]{0.122046,0.632107,0.530848}
\pgfpathmoveto{\pgfpoint{190.943985pt}{220.471588pt}}
\pgflineto{\pgfpoint{199.871979pt}{214.294739pt}}
\pgflineto{\pgfpoint{190.943985pt}{214.294739pt}}
\pgfpathclose
\pgfusepath{fill,stroke}
\pgfpathmoveto{\pgfpoint{199.871979pt}{208.117905pt}}
\pgflineto{\pgfpoint{208.799988pt}{208.117905pt}}
\pgflineto{\pgfpoint{208.799988pt}{201.941055pt}}
\pgfpathclose
\pgfusepath{fill,stroke}
\pgfpathmoveto{\pgfpoint{199.871979pt}{214.294739pt}}
\pgflineto{\pgfpoint{208.799988pt}{208.117905pt}}
\pgflineto{\pgfpoint{199.871979pt}{208.117905pt}}
\pgfpathclose
\pgfusepath{fill,stroke}
\pgfpathmoveto{\pgfpoint{199.871979pt}{214.294739pt}}
\pgflineto{\pgfpoint{208.799988pt}{214.294739pt}}
\pgflineto{\pgfpoint{208.799988pt}{208.117905pt}}
\pgfpathclose
\pgfusepath{fill,stroke}
\color[rgb]{0.119627,0.617266,0.536796}
\pgfpathmoveto{\pgfpoint{208.799988pt}{195.764206pt}}
\pgflineto{\pgfpoint{217.727982pt}{195.764206pt}}
\pgflineto{\pgfpoint{217.727982pt}{189.587372pt}}
\pgfpathclose
\pgfusepath{fill,stroke}
\color[rgb]{0.122046,0.632107,0.530848}
\pgfpathmoveto{\pgfpoint{208.799988pt}{201.941055pt}}
\pgflineto{\pgfpoint{217.727982pt}{195.764206pt}}
\pgflineto{\pgfpoint{208.799988pt}{195.764206pt}}
\pgfpathclose
\pgfusepath{fill,stroke}
\pgfpathmoveto{\pgfpoint{208.799988pt}{201.941055pt}}
\pgflineto{\pgfpoint{217.727982pt}{201.941055pt}}
\pgflineto{\pgfpoint{217.727982pt}{195.764206pt}}
\pgfpathclose
\pgfusepath{fill,stroke}
\color[rgb]{0.127668,0.646882,0.523924}
\pgfpathmoveto{\pgfpoint{208.799988pt}{208.117905pt}}
\pgflineto{\pgfpoint{217.727982pt}{201.941055pt}}
\pgflineto{\pgfpoint{208.799988pt}{201.941055pt}}
\pgfpathclose
\pgfusepath{fill,stroke}
\pgfpathmoveto{\pgfpoint{208.799988pt}{208.117905pt}}
\pgflineto{\pgfpoint{217.727982pt}{208.117905pt}}
\pgflineto{\pgfpoint{217.727982pt}{201.941055pt}}
\pgfpathclose
\pgfusepath{fill,stroke}
\color[rgb]{0.136835,0.661563,0.515967}
\pgfpathmoveto{\pgfpoint{208.799988pt}{214.294739pt}}
\pgflineto{\pgfpoint{217.727982pt}{208.117905pt}}
\pgflineto{\pgfpoint{208.799988pt}{208.117905pt}}
\pgfpathclose
\pgfusepath{fill,stroke}
\color[rgb]{0.122046,0.632107,0.530848}
\pgfpathmoveto{\pgfpoint{217.727982pt}{195.764206pt}}
\pgflineto{\pgfpoint{226.655975pt}{189.587372pt}}
\pgflineto{\pgfpoint{217.727982pt}{189.587372pt}}
\pgfpathclose
\pgfusepath{fill,stroke}
\color[rgb]{0.127668,0.646882,0.523924}
\pgfpathmoveto{\pgfpoint{217.727982pt}{201.941055pt}}
\pgflineto{\pgfpoint{226.655975pt}{201.941055pt}}
\pgflineto{\pgfpoint{226.655975pt}{195.764206pt}}
\pgfpathclose
\pgfusepath{fill,stroke}
\color[rgb]{0.136835,0.661563,0.515967}
\pgfpathmoveto{\pgfpoint{217.727982pt}{208.117905pt}}
\pgflineto{\pgfpoint{226.655975pt}{201.941055pt}}
\pgflineto{\pgfpoint{217.727982pt}{201.941055pt}}
\pgfpathclose
\pgfusepath{fill,stroke}
\pgfpathmoveto{\pgfpoint{217.727982pt}{208.117905pt}}
\pgflineto{\pgfpoint{226.655975pt}{208.117905pt}}
\pgflineto{\pgfpoint{226.655975pt}{201.941055pt}}
\pgfpathclose
\pgfusepath{fill,stroke}
\color[rgb]{0.122046,0.632107,0.530848}
\pgfpathmoveto{\pgfpoint{226.655975pt}{189.587372pt}}
\pgflineto{\pgfpoint{235.583969pt}{189.587372pt}}
\pgflineto{\pgfpoint{235.583969pt}{183.410522pt}}
\pgfpathclose
\pgfusepath{fill,stroke}
\color[rgb]{0.127668,0.646882,0.523924}
\pgfpathmoveto{\pgfpoint{226.655975pt}{195.764206pt}}
\pgflineto{\pgfpoint{235.583969pt}{189.587372pt}}
\pgflineto{\pgfpoint{226.655975pt}{189.587372pt}}
\pgfpathclose
\pgfusepath{fill,stroke}
\pgfpathmoveto{\pgfpoint{226.655975pt}{195.764206pt}}
\pgflineto{\pgfpoint{235.583969pt}{195.764206pt}}
\pgflineto{\pgfpoint{235.583969pt}{189.587372pt}}
\pgfpathclose
\pgfusepath{fill,stroke}
\color[rgb]{0.136835,0.661563,0.515967}
\pgfpathmoveto{\pgfpoint{226.655975pt}{201.941055pt}}
\pgflineto{\pgfpoint{235.583969pt}{195.764206pt}}
\pgflineto{\pgfpoint{226.655975pt}{195.764206pt}}
\pgfpathclose
\pgfusepath{fill,stroke}
\pgfpathmoveto{\pgfpoint{226.655975pt}{201.941055pt}}
\pgflineto{\pgfpoint{235.583969pt}{201.941055pt}}
\pgflineto{\pgfpoint{235.583969pt}{195.764206pt}}
\pgfpathclose
\pgfusepath{fill,stroke}
\color[rgb]{0.149643,0.676120,0.506924}
\pgfpathmoveto{\pgfpoint{226.655975pt}{208.117905pt}}
\pgflineto{\pgfpoint{235.583969pt}{201.941055pt}}
\pgflineto{\pgfpoint{226.655975pt}{201.941055pt}}
\pgfpathclose
\pgfusepath{fill,stroke}
\color[rgb]{0.127668,0.646882,0.523924}
\pgfpathmoveto{\pgfpoint{235.583969pt}{189.587372pt}}
\pgflineto{\pgfpoint{244.511993pt}{183.410522pt}}
\pgflineto{\pgfpoint{235.583969pt}{183.410522pt}}
\pgfpathclose
\pgfusepath{fill,stroke}
\color[rgb]{0.136835,0.661563,0.515967}
\pgfpathmoveto{\pgfpoint{235.583969pt}{195.764206pt}}
\pgflineto{\pgfpoint{244.511993pt}{195.764206pt}}
\pgflineto{\pgfpoint{244.511993pt}{189.587372pt}}
\pgfpathclose
\pgfusepath{fill,stroke}
\color[rgb]{0.149643,0.676120,0.506924}
\pgfpathmoveto{\pgfpoint{235.583969pt}{201.941055pt}}
\pgflineto{\pgfpoint{244.511993pt}{195.764206pt}}
\pgflineto{\pgfpoint{235.583969pt}{195.764206pt}}
\pgfpathclose
\pgfusepath{fill,stroke}
\pgfpathmoveto{\pgfpoint{235.583969pt}{201.941055pt}}
\pgflineto{\pgfpoint{244.511993pt}{201.941055pt}}
\pgflineto{\pgfpoint{244.511993pt}{195.764206pt}}
\pgfpathclose
\pgfusepath{fill,stroke}
\pgfpathmoveto{\pgfpoint{244.511993pt}{189.587372pt}}
\pgflineto{\pgfpoint{253.440002pt}{189.587372pt}}
\pgflineto{\pgfpoint{253.440002pt}{183.410522pt}}
\pgfpathclose
\pgfusepath{fill,stroke}
\pgfpathmoveto{\pgfpoint{244.511993pt}{195.764206pt}}
\pgflineto{\pgfpoint{253.440002pt}{189.587372pt}}
\pgflineto{\pgfpoint{244.511993pt}{189.587372pt}}
\pgfpathclose
\pgfusepath{fill,stroke}
\pgfpathmoveto{\pgfpoint{244.511993pt}{195.764206pt}}
\pgflineto{\pgfpoint{253.440002pt}{195.764206pt}}
\pgflineto{\pgfpoint{253.440002pt}{189.587372pt}}
\pgfpathclose
\pgfusepath{fill,stroke}
\color[rgb]{0.165967,0.690519,0.496752}
\pgfpathmoveto{\pgfpoint{244.511993pt}{201.941055pt}}
\pgflineto{\pgfpoint{253.440002pt}{195.764206pt}}
\pgflineto{\pgfpoint{244.511993pt}{195.764206pt}}
\pgfpathclose
\pgfusepath{fill,stroke}
\color[rgb]{0.149643,0.676120,0.506924}
\pgfpathmoveto{\pgfpoint{253.440002pt}{183.410522pt}}
\pgflineto{\pgfpoint{262.367981pt}{177.233673pt}}
\pgflineto{\pgfpoint{253.440002pt}{177.233673pt}}
\pgfpathclose
\pgfusepath{fill,stroke}
\pgfpathmoveto{\pgfpoint{253.440002pt}{183.410522pt}}
\pgflineto{\pgfpoint{262.367981pt}{183.410522pt}}
\pgflineto{\pgfpoint{262.367981pt}{177.233673pt}}
\pgfpathclose
\pgfusepath{fill,stroke}
\color[rgb]{0.165967,0.690519,0.496752}
\pgfpathmoveto{\pgfpoint{253.440002pt}{189.587372pt}}
\pgflineto{\pgfpoint{262.367981pt}{183.410522pt}}
\pgflineto{\pgfpoint{253.440002pt}{183.410522pt}}
\pgfpathclose
\pgfusepath{fill,stroke}
\pgfpathmoveto{\pgfpoint{253.440002pt}{189.587372pt}}
\pgflineto{\pgfpoint{262.367981pt}{189.587372pt}}
\pgflineto{\pgfpoint{262.367981pt}{183.410522pt}}
\pgfpathclose
\pgfusepath{fill,stroke}
\pgfpathmoveto{\pgfpoint{253.440002pt}{195.764206pt}}
\pgflineto{\pgfpoint{262.367981pt}{189.587372pt}}
\pgflineto{\pgfpoint{253.440002pt}{189.587372pt}}
\pgfpathclose
\pgfusepath{fill,stroke}
\pgfpathmoveto{\pgfpoint{262.367981pt}{183.410522pt}}
\pgflineto{\pgfpoint{271.295990pt}{183.410522pt}}
\pgflineto{\pgfpoint{271.295990pt}{177.233673pt}}
\pgfpathclose
\pgfusepath{fill,stroke}
\color[rgb]{0.185538,0.704725,0.485412}
\pgfpathmoveto{\pgfpoint{262.367981pt}{189.587372pt}}
\pgflineto{\pgfpoint{271.295990pt}{183.410522pt}}
\pgflineto{\pgfpoint{262.367981pt}{183.410522pt}}
\pgfpathclose
\pgfusepath{fill,stroke}
\pgfpathmoveto{\pgfpoint{262.367981pt}{189.587372pt}}
\pgflineto{\pgfpoint{271.295990pt}{189.587372pt}}
\pgflineto{\pgfpoint{271.295990pt}{183.410522pt}}
\pgfpathclose
\pgfusepath{fill,stroke}
\color[rgb]{0.165967,0.690519,0.496752}
\pgfpathmoveto{\pgfpoint{271.295990pt}{177.233673pt}}
\pgflineto{\pgfpoint{280.223969pt}{171.056854pt}}
\pgflineto{\pgfpoint{271.295990pt}{171.056854pt}}
\pgfpathclose
\pgfusepath{fill,stroke}
\pgfpathmoveto{\pgfpoint{271.295990pt}{177.233673pt}}
\pgflineto{\pgfpoint{280.223969pt}{177.233673pt}}
\pgflineto{\pgfpoint{280.223969pt}{171.056854pt}}
\pgfpathclose
\pgfusepath{fill,stroke}
\color[rgb]{0.185538,0.704725,0.485412}
\pgfpathmoveto{\pgfpoint{271.295990pt}{183.410522pt}}
\pgflineto{\pgfpoint{280.223969pt}{177.233673pt}}
\pgflineto{\pgfpoint{271.295990pt}{177.233673pt}}
\pgfpathclose
\pgfusepath{fill,stroke}
\pgfpathmoveto{\pgfpoint{271.295990pt}{183.410522pt}}
\pgflineto{\pgfpoint{280.223969pt}{183.410522pt}}
\pgflineto{\pgfpoint{280.223969pt}{177.233673pt}}
\pgfpathclose
\pgfusepath{fill,stroke}
\color[rgb]{0.208030,0.718701,0.472873}
\pgfpathmoveto{\pgfpoint{271.295990pt}{189.587372pt}}
\pgflineto{\pgfpoint{280.223969pt}{183.410522pt}}
\pgflineto{\pgfpoint{271.295990pt}{183.410522pt}}
\pgfpathclose
\pgfusepath{fill,stroke}
\color[rgb]{0.185538,0.704725,0.485412}
\pgfpathmoveto{\pgfpoint{280.223969pt}{177.233673pt}}
\pgflineto{\pgfpoint{289.151978pt}{177.233673pt}}
\pgflineto{\pgfpoint{289.151978pt}{171.056854pt}}
\pgfpathclose
\pgfusepath{fill,stroke}
\color[rgb]{0.208030,0.718701,0.472873}
\pgfpathmoveto{\pgfpoint{280.223969pt}{183.410522pt}}
\pgflineto{\pgfpoint{289.151978pt}{177.233673pt}}
\pgflineto{\pgfpoint{280.223969pt}{177.233673pt}}
\pgfpathclose
\pgfusepath{fill,stroke}
\pgfpathmoveto{\pgfpoint{280.223969pt}{183.410522pt}}
\pgflineto{\pgfpoint{289.151978pt}{183.410522pt}}
\pgflineto{\pgfpoint{289.151978pt}{177.233673pt}}
\pgfpathclose
\pgfusepath{fill,stroke}
\color[rgb]{0.185538,0.704725,0.485412}
\pgfpathmoveto{\pgfpoint{289.151978pt}{164.880005pt}}
\pgflineto{\pgfpoint{298.079987pt}{164.880005pt}}
\pgflineto{\pgfpoint{298.079987pt}{158.703156pt}}
\pgfpathclose
\pgfusepath{fill,stroke}
\pgfpathmoveto{\pgfpoint{289.151978pt}{171.056854pt}}
\pgflineto{\pgfpoint{298.079987pt}{164.880005pt}}
\pgflineto{\pgfpoint{289.151978pt}{164.880005pt}}
\pgfpathclose
\pgfusepath{fill,stroke}
\pgfpathmoveto{\pgfpoint{289.151978pt}{171.056854pt}}
\pgflineto{\pgfpoint{298.079987pt}{171.056854pt}}
\pgflineto{\pgfpoint{298.079987pt}{164.880005pt}}
\pgfpathclose
\pgfusepath{fill,stroke}
\color[rgb]{0.208030,0.718701,0.472873}
\pgfpathmoveto{\pgfpoint{289.151978pt}{177.233673pt}}
\pgflineto{\pgfpoint{298.079987pt}{171.056854pt}}
\pgflineto{\pgfpoint{289.151978pt}{171.056854pt}}
\pgfpathclose
\pgfusepath{fill,stroke}
\pgfpathmoveto{\pgfpoint{289.151978pt}{177.233673pt}}
\pgflineto{\pgfpoint{298.079987pt}{177.233673pt}}
\pgflineto{\pgfpoint{298.079987pt}{171.056854pt}}
\pgfpathclose
\pgfusepath{fill,stroke}
\color[rgb]{0.233127,0.732406,0.459106}
\pgfpathmoveto{\pgfpoint{289.151978pt}{183.410522pt}}
\pgflineto{\pgfpoint{298.079987pt}{177.233673pt}}
\pgflineto{\pgfpoint{289.151978pt}{177.233673pt}}
\pgfpathclose
\pgfusepath{fill,stroke}
\color[rgb]{0.208030,0.718701,0.472873}
\pgfpathmoveto{\pgfpoint{298.079987pt}{164.880005pt}}
\pgflineto{\pgfpoint{307.007965pt}{158.703156pt}}
\pgflineto{\pgfpoint{298.079987pt}{158.703156pt}}
\pgfpathclose
\pgfusepath{fill,stroke}
\pgfpathmoveto{\pgfpoint{298.079987pt}{171.056854pt}}
\pgflineto{\pgfpoint{307.007965pt}{171.056854pt}}
\pgflineto{\pgfpoint{307.007965pt}{164.880005pt}}
\pgfpathclose
\pgfusepath{fill,stroke}
\color[rgb]{0.233127,0.732406,0.459106}
\pgfpathmoveto{\pgfpoint{298.079987pt}{177.233673pt}}
\pgflineto{\pgfpoint{307.007965pt}{171.056854pt}}
\pgflineto{\pgfpoint{298.079987pt}{171.056854pt}}
\pgfpathclose
\pgfusepath{fill,stroke}
\pgfpathmoveto{\pgfpoint{298.079987pt}{177.233673pt}}
\pgflineto{\pgfpoint{307.007965pt}{177.233673pt}}
\pgflineto{\pgfpoint{307.007965pt}{171.056854pt}}
\pgfpathclose
\pgfusepath{fill,stroke}
\color[rgb]{0.208030,0.718701,0.472873}
\pgfpathmoveto{\pgfpoint{307.007965pt}{158.703156pt}}
\pgflineto{\pgfpoint{315.935974pt}{158.703156pt}}
\pgflineto{\pgfpoint{315.935974pt}{152.526306pt}}
\pgfpathclose
\pgfusepath{fill,stroke}
\color[rgb]{0.233127,0.732406,0.459106}
\pgfpathmoveto{\pgfpoint{307.007965pt}{164.880005pt}}
\pgflineto{\pgfpoint{315.935974pt}{158.703156pt}}
\pgflineto{\pgfpoint{307.007965pt}{158.703156pt}}
\pgfpathclose
\pgfusepath{fill,stroke}
\pgfpathmoveto{\pgfpoint{307.007965pt}{164.880005pt}}
\pgflineto{\pgfpoint{315.935974pt}{164.880005pt}}
\pgflineto{\pgfpoint{315.935974pt}{158.703156pt}}
\pgfpathclose
\pgfusepath{fill,stroke}
\color[rgb]{0.260531,0.745802,0.444096}
\pgfpathmoveto{\pgfpoint{307.007965pt}{171.056854pt}}
\pgflineto{\pgfpoint{315.935974pt}{164.880005pt}}
\pgflineto{\pgfpoint{307.007965pt}{164.880005pt}}
\pgfpathclose
\pgfusepath{fill,stroke}
\pgfpathmoveto{\pgfpoint{307.007965pt}{171.056854pt}}
\pgflineto{\pgfpoint{315.935974pt}{171.056854pt}}
\pgflineto{\pgfpoint{315.935974pt}{164.880005pt}}
\pgfpathclose
\pgfusepath{fill,stroke}
\pgfpathmoveto{\pgfpoint{307.007965pt}{177.233673pt}}
\pgflineto{\pgfpoint{315.935974pt}{171.056854pt}}
\pgflineto{\pgfpoint{307.007965pt}{171.056854pt}}
\pgfpathclose
\pgfusepath{fill,stroke}
\color[rgb]{0.233127,0.732406,0.459106}
\pgfpathmoveto{\pgfpoint{315.935974pt}{158.703156pt}}
\pgflineto{\pgfpoint{324.863983pt}{152.526306pt}}
\pgflineto{\pgfpoint{315.935974pt}{152.526306pt}}
\pgfpathclose
\pgfusepath{fill,stroke}
\color[rgb]{0.260531,0.745802,0.444096}
\pgfpathmoveto{\pgfpoint{315.935974pt}{164.880005pt}}
\pgflineto{\pgfpoint{324.863983pt}{164.880005pt}}
\pgflineto{\pgfpoint{324.863983pt}{158.703156pt}}
\pgfpathclose
\pgfusepath{fill,stroke}
\color[rgb]{0.290001,0.758846,0.427826}
\pgfpathmoveto{\pgfpoint{315.935974pt}{171.056854pt}}
\pgflineto{\pgfpoint{324.863983pt}{164.880005pt}}
\pgflineto{\pgfpoint{315.935974pt}{164.880005pt}}
\pgfpathclose
\pgfusepath{fill,stroke}
\color[rgb]{0.233127,0.732406,0.459106}
\pgfpathmoveto{\pgfpoint{324.863983pt}{152.526306pt}}
\pgflineto{\pgfpoint{333.791992pt}{152.526306pt}}
\pgflineto{\pgfpoint{333.791992pt}{146.349472pt}}
\pgfpathclose
\pgfusepath{fill,stroke}
\color[rgb]{0.260531,0.745802,0.444096}
\pgfpathmoveto{\pgfpoint{324.863983pt}{158.703156pt}}
\pgflineto{\pgfpoint{333.791992pt}{152.526306pt}}
\pgflineto{\pgfpoint{324.863983pt}{152.526306pt}}
\pgfpathclose
\pgfusepath{fill,stroke}
\color[rgb]{0.290001,0.758846,0.427826}
\pgfpathmoveto{\pgfpoint{324.863983pt}{164.880005pt}}
\pgflineto{\pgfpoint{333.791992pt}{158.703156pt}}
\pgflineto{\pgfpoint{324.863983pt}{158.703156pt}}
\pgfpathclose
\pgfusepath{fill,stroke}
\pgfpathmoveto{\pgfpoint{324.863983pt}{164.880005pt}}
\pgflineto{\pgfpoint{333.791992pt}{164.880005pt}}
\pgflineto{\pgfpoint{333.791992pt}{158.703156pt}}
\pgfpathclose
\pgfusepath{fill,stroke}
\color[rgb]{0.260531,0.745802,0.444096}
\pgfpathmoveto{\pgfpoint{333.791992pt}{152.526306pt}}
\pgflineto{\pgfpoint{342.719971pt}{146.349472pt}}
\pgflineto{\pgfpoint{333.791992pt}{146.349472pt}}
\pgfpathclose
\pgfusepath{fill,stroke}
\color[rgb]{0.290001,0.758846,0.427826}
\pgfpathmoveto{\pgfpoint{333.791992pt}{158.703156pt}}
\pgflineto{\pgfpoint{342.719971pt}{158.703156pt}}
\pgflineto{\pgfpoint{342.719971pt}{152.526306pt}}
\pgfpathclose
\pgfusepath{fill,stroke}
\color[rgb]{0.321330,0.771498,0.410293}
\pgfpathmoveto{\pgfpoint{333.791992pt}{164.880005pt}}
\pgflineto{\pgfpoint{342.719971pt}{158.703156pt}}
\pgflineto{\pgfpoint{333.791992pt}{158.703156pt}}
\pgfpathclose
\pgfusepath{fill,stroke}
\color[rgb]{0.290001,0.758846,0.427826}
\pgfpathmoveto{\pgfpoint{342.719971pt}{146.349472pt}}
\pgflineto{\pgfpoint{351.647980pt}{146.349472pt}}
\pgflineto{\pgfpoint{351.647980pt}{140.172638pt}}
\pgfpathclose
\pgfusepath{fill,stroke}
\pgfpathmoveto{\pgfpoint{342.719971pt}{152.526306pt}}
\pgflineto{\pgfpoint{351.647980pt}{152.526306pt}}
\pgflineto{\pgfpoint{351.647980pt}{146.349472pt}}
\pgfpathclose
\pgfusepath{fill,stroke}
\color[rgb]{0.321330,0.771498,0.410293}
\pgfpathmoveto{\pgfpoint{342.719971pt}{158.703156pt}}
\pgflineto{\pgfpoint{351.647980pt}{152.526306pt}}
\pgflineto{\pgfpoint{342.719971pt}{152.526306pt}}
\pgfpathclose
\pgfusepath{fill,stroke}
\pgfpathmoveto{\pgfpoint{342.719971pt}{158.703156pt}}
\pgflineto{\pgfpoint{351.647980pt}{158.703156pt}}
\pgflineto{\pgfpoint{351.647980pt}{152.526306pt}}
\pgfpathclose
\pgfusepath{fill,stroke}
\pgfpathmoveto{\pgfpoint{351.647980pt}{146.349472pt}}
\pgflineto{\pgfpoint{360.575958pt}{140.172638pt}}
\pgflineto{\pgfpoint{351.647980pt}{140.172638pt}}
\pgfpathclose
\pgfusepath{fill,stroke}
\pgfpathmoveto{\pgfpoint{351.647980pt}{146.349472pt}}
\pgflineto{\pgfpoint{360.575958pt}{146.349472pt}}
\pgflineto{\pgfpoint{360.575958pt}{140.172638pt}}
\pgfpathclose
\pgfusepath{fill,stroke}
\pgfpathmoveto{\pgfpoint{351.647980pt}{152.526306pt}}
\pgflineto{\pgfpoint{360.575958pt}{146.349472pt}}
\pgflineto{\pgfpoint{351.647980pt}{146.349472pt}}
\pgfpathclose
\pgfusepath{fill,stroke}
\pgfpathmoveto{\pgfpoint{351.647980pt}{152.526306pt}}
\pgflineto{\pgfpoint{360.575958pt}{152.526306pt}}
\pgflineto{\pgfpoint{360.575958pt}{146.349472pt}}
\pgfpathclose
\pgfusepath{fill,stroke}
\color[rgb]{0.354355,0.783714,0.391488}
\pgfpathmoveto{\pgfpoint{351.647980pt}{158.703156pt}}
\pgflineto{\pgfpoint{360.575958pt}{152.526306pt}}
\pgflineto{\pgfpoint{351.647980pt}{152.526306pt}}
\pgfpathclose
\pgfusepath{fill,stroke}
\pgfpathmoveto{\pgfpoint{360.575958pt}{146.349472pt}}
\pgflineto{\pgfpoint{369.503998pt}{140.172638pt}}
\pgflineto{\pgfpoint{360.575958pt}{140.172638pt}}
\pgfpathclose
\pgfusepath{fill,stroke}
\pgfpathmoveto{\pgfpoint{360.575958pt}{146.349472pt}}
\pgflineto{\pgfpoint{369.503998pt}{146.349472pt}}
\pgflineto{\pgfpoint{369.503998pt}{140.172638pt}}
\pgfpathclose
\pgfusepath{fill,stroke}
\color[rgb]{0.388930,0.795453,0.371421}
\pgfpathmoveto{\pgfpoint{360.575958pt}{152.526306pt}}
\pgflineto{\pgfpoint{369.503998pt}{146.349472pt}}
\pgflineto{\pgfpoint{360.575958pt}{146.349472pt}}
\pgfpathclose
\pgfusepath{fill,stroke}
\color[rgb]{0.354355,0.783714,0.391488}
\pgfpathmoveto{\pgfpoint{369.503998pt}{140.172638pt}}
\pgflineto{\pgfpoint{378.431976pt}{140.172638pt}}
\pgflineto{\pgfpoint{378.431976pt}{133.995789pt}}
\pgfpathclose
\pgfusepath{fill,stroke}
\color[rgb]{0.388930,0.795453,0.371421}
\pgfpathmoveto{\pgfpoint{369.503998pt}{146.349472pt}}
\pgflineto{\pgfpoint{378.431976pt}{140.172638pt}}
\pgflineto{\pgfpoint{369.503998pt}{140.172638pt}}
\pgfpathclose
\pgfusepath{fill,stroke}
\pgfpathmoveto{\pgfpoint{369.503998pt}{146.349472pt}}
\pgflineto{\pgfpoint{378.431976pt}{146.349472pt}}
\pgflineto{\pgfpoint{378.431976pt}{140.172638pt}}
\pgfpathclose
\pgfusepath{fill,stroke}
\pgfpathmoveto{\pgfpoint{378.431976pt}{140.172638pt}}
\pgflineto{\pgfpoint{387.359985pt}{133.995789pt}}
\pgflineto{\pgfpoint{378.431976pt}{133.995789pt}}
\pgfpathclose
\pgfusepath{fill,stroke}
\pgfpathmoveto{\pgfpoint{378.431976pt}{140.172638pt}}
\pgflineto{\pgfpoint{387.359985pt}{140.172638pt}}
\pgflineto{\pgfpoint{387.359985pt}{133.995789pt}}
\pgfpathclose
\pgfusepath{fill,stroke}
\color[rgb]{0.424933,0.806674,0.350099}
\pgfpathmoveto{\pgfpoint{378.431976pt}{146.349472pt}}
\pgflineto{\pgfpoint{387.359985pt}{140.172638pt}}
\pgflineto{\pgfpoint{378.431976pt}{140.172638pt}}
\pgfpathclose
\pgfusepath{fill,stroke}
\color[rgb]{0.388930,0.795453,0.371421}
\pgfpathmoveto{\pgfpoint{387.359985pt}{133.995789pt}}
\pgflineto{\pgfpoint{396.287964pt}{133.995789pt}}
\pgflineto{\pgfpoint{396.287964pt}{127.818947pt}}
\pgfpathclose
\pgfusepath{fill,stroke}
\color[rgb]{0.424933,0.806674,0.350099}
\pgfpathmoveto{\pgfpoint{387.359985pt}{140.172638pt}}
\pgflineto{\pgfpoint{396.287964pt}{133.995789pt}}
\pgflineto{\pgfpoint{387.359985pt}{133.995789pt}}
\pgfpathclose
\pgfusepath{fill,stroke}
\pgfpathmoveto{\pgfpoint{387.359985pt}{140.172638pt}}
\pgflineto{\pgfpoint{396.287964pt}{140.172638pt}}
\pgflineto{\pgfpoint{396.287964pt}{133.995789pt}}
\pgfpathclose
\pgfusepath{fill,stroke}
\pgfpathmoveto{\pgfpoint{396.287964pt}{133.995789pt}}
\pgflineto{\pgfpoint{405.216003pt}{127.818947pt}}
\pgflineto{\pgfpoint{396.287964pt}{127.818947pt}}
\pgfpathclose
\pgfusepath{fill,stroke}
\pgfpathmoveto{\pgfpoint{396.287964pt}{133.995789pt}}
\pgflineto{\pgfpoint{405.216003pt}{133.995789pt}}
\pgflineto{\pgfpoint{405.216003pt}{127.818947pt}}
\pgfpathclose
\pgfusepath{fill,stroke}
\color[rgb]{0.462247,0.817338,0.327545}
\pgfpathmoveto{\pgfpoint{396.287964pt}{140.172638pt}}
\pgflineto{\pgfpoint{405.216003pt}{133.995789pt}}
\pgflineto{\pgfpoint{396.287964pt}{133.995789pt}}
\pgfpathclose
\pgfusepath{fill,stroke}
\color[rgb]{0.424933,0.806674,0.350099}
\pgfpathmoveto{\pgfpoint{405.216003pt}{121.642097pt}}
\pgflineto{\pgfpoint{414.143982pt}{121.642097pt}}
\pgflineto{\pgfpoint{414.143982pt}{115.465263pt}}
\pgfpathclose
\pgfusepath{fill,stroke}
\color[rgb]{0.462247,0.817338,0.327545}
\pgfpathmoveto{\pgfpoint{405.216003pt}{127.818947pt}}
\pgflineto{\pgfpoint{414.143982pt}{127.818947pt}}
\pgflineto{\pgfpoint{414.143982pt}{121.642097pt}}
\pgfpathclose
\pgfusepath{fill,stroke}
\pgfpathmoveto{\pgfpoint{405.216003pt}{133.995789pt}}
\pgflineto{\pgfpoint{414.143982pt}{127.818947pt}}
\pgflineto{\pgfpoint{405.216003pt}{127.818947pt}}
\pgfpathclose
\pgfusepath{fill,stroke}
\pgfpathmoveto{\pgfpoint{405.216003pt}{133.995789pt}}
\pgflineto{\pgfpoint{414.143982pt}{133.995789pt}}
\pgflineto{\pgfpoint{414.143982pt}{127.818947pt}}
\pgfpathclose
\pgfusepath{fill,stroke}
\pgfpathmoveto{\pgfpoint{414.143982pt}{121.642097pt}}
\pgflineto{\pgfpoint{423.071960pt}{115.465263pt}}
\pgflineto{\pgfpoint{414.143982pt}{115.465263pt}}
\pgfpathclose
\pgfusepath{fill,stroke}
\pgfpathmoveto{\pgfpoint{414.143982pt}{121.642097pt}}
\pgflineto{\pgfpoint{423.071960pt}{121.642097pt}}
\pgflineto{\pgfpoint{423.071960pt}{115.465263pt}}
\pgfpathclose
\pgfusepath{fill,stroke}
\color[rgb]{0.500754,0.827409,0.303799}
\pgfpathmoveto{\pgfpoint{414.143982pt}{127.818947pt}}
\pgflineto{\pgfpoint{423.071960pt}{121.642097pt}}
\pgflineto{\pgfpoint{414.143982pt}{121.642097pt}}
\pgfpathclose
\pgfusepath{fill,stroke}
\pgfpathmoveto{\pgfpoint{414.143982pt}{127.818947pt}}
\pgflineto{\pgfpoint{423.071960pt}{127.818947pt}}
\pgflineto{\pgfpoint{423.071960pt}{121.642097pt}}
\pgfpathclose
\pgfusepath{fill,stroke}
\pgfpathmoveto{\pgfpoint{414.143982pt}{133.995789pt}}
\pgflineto{\pgfpoint{423.071960pt}{127.818947pt}}
\pgflineto{\pgfpoint{414.143982pt}{127.818947pt}}
\pgfpathclose
\pgfusepath{fill,stroke}
\color[rgb]{0.462247,0.817338,0.327545}
\pgfpathmoveto{\pgfpoint{423.071960pt}{115.465263pt}}
\pgflineto{\pgfpoint{432.000000pt}{115.465263pt}}
\pgflineto{\pgfpoint{432.000000pt}{109.288422pt}}
\pgfpathclose
\pgfusepath{fill,stroke}
\color[rgb]{0.500754,0.827409,0.303799}
\pgfpathmoveto{\pgfpoint{423.071960pt}{121.642097pt}}
\pgflineto{\pgfpoint{432.000000pt}{115.465263pt}}
\pgflineto{\pgfpoint{423.071960pt}{115.465263pt}}
\pgfpathclose
\pgfusepath{fill,stroke}
\pgfpathmoveto{\pgfpoint{423.071960pt}{121.642097pt}}
\pgflineto{\pgfpoint{432.000000pt}{121.642097pt}}
\pgflineto{\pgfpoint{432.000000pt}{115.465263pt}}
\pgfpathclose
\pgfusepath{fill,stroke}
\color[rgb]{0.540337,0.836858,0.278917}
\pgfpathmoveto{\pgfpoint{423.071960pt}{127.818947pt}}
\pgflineto{\pgfpoint{432.000000pt}{121.642097pt}}
\pgflineto{\pgfpoint{423.071960pt}{121.642097pt}}
\pgfpathclose
\pgfusepath{fill,stroke}
\color[rgb]{0.500754,0.827409,0.303799}
\pgfpathmoveto{\pgfpoint{432.000000pt}{115.465263pt}}
\pgflineto{\pgfpoint{440.927979pt}{109.288422pt}}
\pgflineto{\pgfpoint{432.000000pt}{109.288422pt}}
\pgfpathclose
\pgfusepath{fill,stroke}
\pgfpathmoveto{\pgfpoint{432.000000pt}{115.465263pt}}
\pgflineto{\pgfpoint{440.927979pt}{115.465263pt}}
\pgflineto{\pgfpoint{440.927979pt}{109.288422pt}}
\pgfpathclose
\pgfusepath{fill,stroke}
\color[rgb]{0.540337,0.836858,0.278917}
\pgfpathmoveto{\pgfpoint{432.000000pt}{121.642097pt}}
\pgflineto{\pgfpoint{440.927979pt}{115.465263pt}}
\pgflineto{\pgfpoint{432.000000pt}{115.465263pt}}
\pgfpathclose
\pgfusepath{fill,stroke}
\pgfpathmoveto{\pgfpoint{432.000000pt}{121.642097pt}}
\pgflineto{\pgfpoint{440.927979pt}{121.642097pt}}
\pgflineto{\pgfpoint{440.927979pt}{115.465263pt}}
\pgfpathclose
\pgfusepath{fill,stroke}
\pgfpathmoveto{\pgfpoint{440.927979pt}{115.465263pt}}
\pgflineto{\pgfpoint{449.855957pt}{109.288422pt}}
\pgflineto{\pgfpoint{440.927979pt}{109.288422pt}}
\pgfpathclose
\pgfusepath{fill,stroke}
\pgfpathmoveto{\pgfpoint{440.927979pt}{115.465263pt}}
\pgflineto{\pgfpoint{449.855957pt}{115.465263pt}}
\pgflineto{\pgfpoint{449.855957pt}{109.288422pt}}
\pgfpathclose
\pgfusepath{fill,stroke}
\color[rgb]{0.580861,0.845663,0.253001}
\pgfpathmoveto{\pgfpoint{440.927979pt}{121.642097pt}}
\pgflineto{\pgfpoint{449.855957pt}{115.465263pt}}
\pgflineto{\pgfpoint{440.927979pt}{115.465263pt}}
\pgfpathclose
\pgfusepath{fill,stroke}
\color[rgb]{0.540337,0.836858,0.278917}
\pgfpathmoveto{\pgfpoint{449.855957pt}{109.288422pt}}
\pgflineto{\pgfpoint{458.783936pt}{109.288422pt}}
\pgflineto{\pgfpoint{458.783936pt}{103.111580pt}}
\pgfpathclose
\pgfusepath{fill,stroke}
\color[rgb]{0.580861,0.845663,0.253001}
\pgfpathmoveto{\pgfpoint{449.855957pt}{115.465263pt}}
\pgflineto{\pgfpoint{458.783936pt}{109.288422pt}}
\pgflineto{\pgfpoint{449.855957pt}{109.288422pt}}
\pgfpathclose
\pgfusepath{fill,stroke}
\pgfpathmoveto{\pgfpoint{449.855957pt}{115.465263pt}}
\pgflineto{\pgfpoint{458.783936pt}{115.465263pt}}
\pgflineto{\pgfpoint{458.783936pt}{109.288422pt}}
\pgfpathclose
\pgfusepath{fill,stroke}
\color[rgb]{0.622171,0.853816,0.226224}
\pgfpathmoveto{\pgfpoint{458.783936pt}{109.288422pt}}
\pgflineto{\pgfpoint{467.711975pt}{103.111580pt}}
\pgflineto{\pgfpoint{458.783936pt}{103.111580pt}}
\pgfpathclose
\pgfusepath{fill,stroke}
\pgfpathmoveto{\pgfpoint{458.783936pt}{109.288422pt}}
\pgflineto{\pgfpoint{467.711975pt}{109.288422pt}}
\pgflineto{\pgfpoint{467.711975pt}{103.111580pt}}
\pgfpathclose
\pgfusepath{fill,stroke}
\pgfpathmoveto{\pgfpoint{458.783936pt}{115.465263pt}}
\pgflineto{\pgfpoint{467.711975pt}{109.288422pt}}
\pgflineto{\pgfpoint{458.783936pt}{109.288422pt}}
\pgfpathclose
\pgfusepath{fill,stroke}
\color[rgb]{0.580861,0.845663,0.253001}
\pgfpathmoveto{\pgfpoint{467.711975pt}{96.934731pt}}
\pgflineto{\pgfpoint{476.639954pt}{96.934731pt}}
\pgflineto{\pgfpoint{476.639954pt}{90.757896pt}}
\pgfpathclose
\pgfusepath{fill,stroke}
\color[rgb]{0.622171,0.853816,0.226224}
\pgfpathmoveto{\pgfpoint{467.711975pt}{103.111580pt}}
\pgflineto{\pgfpoint{476.639954pt}{103.111580pt}}
\pgflineto{\pgfpoint{476.639954pt}{96.934731pt}}
\pgfpathclose
\pgfusepath{fill,stroke}
\color[rgb]{0.664087,0.861321,0.198879}
\pgfpathmoveto{\pgfpoint{467.711975pt}{109.288422pt}}
\pgflineto{\pgfpoint{476.639954pt}{103.111580pt}}
\pgflineto{\pgfpoint{467.711975pt}{103.111580pt}}
\pgfpathclose
\pgfusepath{fill,stroke}
\pgfpathmoveto{\pgfpoint{467.711975pt}{109.288422pt}}
\pgflineto{\pgfpoint{476.639954pt}{109.288422pt}}
\pgflineto{\pgfpoint{476.639954pt}{103.111580pt}}
\pgfpathclose
\pgfusepath{fill,stroke}
\color[rgb]{0.622171,0.853816,0.226224}
\pgfpathmoveto{\pgfpoint{476.639954pt}{96.934731pt}}
\pgflineto{\pgfpoint{485.567963pt}{90.757896pt}}
\pgflineto{\pgfpoint{476.639954pt}{90.757896pt}}
\pgfpathclose
\pgfusepath{fill,stroke}
\color[rgb]{0.388930,0.795453,0.371421}
\pgfpathmoveto{\pgfpoint{360.575958pt}{152.526306pt}}
\pgflineto{\pgfpoint{369.503998pt}{152.526306pt}}
\pgflineto{\pgfpoint{369.503998pt}{146.349472pt}}
\pgfpathclose
\pgfusepath{fill,stroke}
\color[rgb]{0.424933,0.806674,0.350099}
\pgfpathmoveto{\pgfpoint{369.503998pt}{152.526306pt}}
\pgflineto{\pgfpoint{378.431976pt}{146.349472pt}}
\pgflineto{\pgfpoint{369.503998pt}{146.349472pt}}
\pgfpathclose
\pgfusepath{fill,stroke}
\pgfpathmoveto{\pgfpoint{378.431976pt}{146.349472pt}}
\pgflineto{\pgfpoint{387.359985pt}{146.349472pt}}
\pgflineto{\pgfpoint{387.359985pt}{140.172638pt}}
\pgfpathclose
\pgfusepath{fill,stroke}
\color[rgb]{0.462247,0.817338,0.327545}
\pgfpathmoveto{\pgfpoint{387.359985pt}{146.349472pt}}
\pgflineto{\pgfpoint{396.287964pt}{140.172638pt}}
\pgflineto{\pgfpoint{387.359985pt}{140.172638pt}}
\pgfpathclose
\pgfusepath{fill,stroke}
\pgfpathmoveto{\pgfpoint{396.287964pt}{140.172638pt}}
\pgflineto{\pgfpoint{405.216003pt}{140.172638pt}}
\pgflineto{\pgfpoint{405.216003pt}{133.995789pt}}
\pgfpathclose
\pgfusepath{fill,stroke}
\color[rgb]{0.500754,0.827409,0.303799}
\pgfpathmoveto{\pgfpoint{405.216003pt}{140.172638pt}}
\pgflineto{\pgfpoint{414.143982pt}{133.995789pt}}
\pgflineto{\pgfpoint{405.216003pt}{133.995789pt}}
\pgfpathclose
\pgfusepath{fill,stroke}
\color[rgb]{0.540337,0.836858,0.278917}
\pgfpathmoveto{\pgfpoint{423.071960pt}{127.818947pt}}
\pgflineto{\pgfpoint{432.000000pt}{127.818947pt}}
\pgflineto{\pgfpoint{432.000000pt}{121.642097pt}}
\pgfpathclose
\pgfusepath{fill,stroke}
\color[rgb]{0.580861,0.845663,0.253001}
\pgfpathmoveto{\pgfpoint{432.000000pt}{127.818947pt}}
\pgflineto{\pgfpoint{440.927979pt}{121.642097pt}}
\pgflineto{\pgfpoint{432.000000pt}{121.642097pt}}
\pgfpathclose
\pgfusepath{fill,stroke}
\pgfpathmoveto{\pgfpoint{440.927979pt}{121.642097pt}}
\pgflineto{\pgfpoint{449.855957pt}{121.642097pt}}
\pgflineto{\pgfpoint{449.855957pt}{115.465263pt}}
\pgfpathclose
\pgfusepath{fill,stroke}
\color[rgb]{0.622171,0.853816,0.226224}
\pgfpathmoveto{\pgfpoint{449.855957pt}{121.642097pt}}
\pgflineto{\pgfpoint{458.783936pt}{115.465263pt}}
\pgflineto{\pgfpoint{449.855957pt}{115.465263pt}}
\pgfpathclose
\pgfusepath{fill,stroke}
\pgfpathmoveto{\pgfpoint{458.783936pt}{115.465263pt}}
\pgflineto{\pgfpoint{467.711975pt}{115.465263pt}}
\pgflineto{\pgfpoint{467.711975pt}{109.288422pt}}
\pgfpathclose
\pgfusepath{fill,stroke}
\color[rgb]{0.260531,0.745802,0.444096}
\pgfpathmoveto{\pgfpoint{307.007965pt}{177.233673pt}}
\pgflineto{\pgfpoint{315.935974pt}{177.233673pt}}
\pgflineto{\pgfpoint{315.935974pt}{171.056854pt}}
\pgfpathclose
\pgfusepath{fill,stroke}
\color[rgb]{0.290001,0.758846,0.427826}
\pgfpathmoveto{\pgfpoint{307.007965pt}{183.410522pt}}
\pgflineto{\pgfpoint{315.935974pt}{177.233673pt}}
\pgflineto{\pgfpoint{307.007965pt}{177.233673pt}}
\pgfpathclose
\pgfusepath{fill,stroke}
\pgfpathmoveto{\pgfpoint{315.935974pt}{171.056854pt}}
\pgflineto{\pgfpoint{324.863983pt}{171.056854pt}}
\pgflineto{\pgfpoint{324.863983pt}{164.880005pt}}
\pgfpathclose
\pgfusepath{fill,stroke}
\color[rgb]{0.321330,0.771498,0.410293}
\pgfpathmoveto{\pgfpoint{324.863983pt}{171.056854pt}}
\pgflineto{\pgfpoint{333.791992pt}{164.880005pt}}
\pgflineto{\pgfpoint{324.863983pt}{164.880005pt}}
\pgfpathclose
\pgfusepath{fill,stroke}
\color[rgb]{0.208030,0.718701,0.472873}
\pgfpathmoveto{\pgfpoint{271.295990pt}{189.587372pt}}
\pgflineto{\pgfpoint{280.223969pt}{189.587372pt}}
\pgflineto{\pgfpoint{280.223969pt}{183.410522pt}}
\pgfpathclose
\pgfusepath{fill,stroke}
\color[rgb]{0.233127,0.732406,0.459106}
\pgfpathmoveto{\pgfpoint{271.295990pt}{195.764206pt}}
\pgflineto{\pgfpoint{280.223969pt}{189.587372pt}}
\pgflineto{\pgfpoint{271.295990pt}{189.587372pt}}
\pgfpathclose
\pgfusepath{fill,stroke}
\pgfpathmoveto{\pgfpoint{289.151978pt}{183.410522pt}}
\pgflineto{\pgfpoint{298.079987pt}{183.410522pt}}
\pgflineto{\pgfpoint{298.079987pt}{177.233673pt}}
\pgfpathclose
\pgfusepath{fill,stroke}
\color[rgb]{0.260531,0.745802,0.444096}
\pgfpathmoveto{\pgfpoint{289.151978pt}{189.587372pt}}
\pgflineto{\pgfpoint{298.079987pt}{183.410522pt}}
\pgflineto{\pgfpoint{289.151978pt}{183.410522pt}}
\pgfpathclose
\pgfusepath{fill,stroke}
\color[rgb]{0.149643,0.676120,0.506924}
\pgfpathmoveto{\pgfpoint{226.655975pt}{208.117905pt}}
\pgflineto{\pgfpoint{235.583969pt}{208.117905pt}}
\pgflineto{\pgfpoint{235.583969pt}{201.941055pt}}
\pgfpathclose
\pgfusepath{fill,stroke}
\color[rgb]{0.165967,0.690519,0.496752}
\pgfpathmoveto{\pgfpoint{226.655975pt}{214.294739pt}}
\pgflineto{\pgfpoint{235.583969pt}{208.117905pt}}
\pgflineto{\pgfpoint{226.655975pt}{208.117905pt}}
\pgfpathclose
\pgfusepath{fill,stroke}
\pgfpathmoveto{\pgfpoint{244.511993pt}{201.941055pt}}
\pgflineto{\pgfpoint{253.440002pt}{201.941055pt}}
\pgflineto{\pgfpoint{253.440002pt}{195.764206pt}}
\pgfpathclose
\pgfusepath{fill,stroke}
\color[rgb]{0.185538,0.704725,0.485412}
\pgfpathmoveto{\pgfpoint{244.511993pt}{208.117905pt}}
\pgflineto{\pgfpoint{253.440002pt}{201.941055pt}}
\pgflineto{\pgfpoint{244.511993pt}{201.941055pt}}
\pgfpathclose
\pgfusepath{fill,stroke}
\color[rgb]{0.165967,0.690519,0.496752}
\pgfpathmoveto{\pgfpoint{253.440002pt}{195.764206pt}}
\pgflineto{\pgfpoint{262.367981pt}{195.764206pt}}
\pgflineto{\pgfpoint{262.367981pt}{189.587372pt}}
\pgfpathclose
\pgfusepath{fill,stroke}
\color[rgb]{0.185538,0.704725,0.485412}
\pgfpathmoveto{\pgfpoint{253.440002pt}{201.941055pt}}
\pgflineto{\pgfpoint{262.367981pt}{195.764206pt}}
\pgflineto{\pgfpoint{253.440002pt}{195.764206pt}}
\pgfpathclose
\pgfusepath{fill,stroke}
\color[rgb]{0.136835,0.661563,0.515967}
\pgfpathmoveto{\pgfpoint{208.799988pt}{220.471588pt}}
\pgflineto{\pgfpoint{217.727982pt}{220.471588pt}}
\pgflineto{\pgfpoint{217.727982pt}{214.294739pt}}
\pgfpathclose
\pgfusepath{fill,stroke}
\color[rgb]{0.149643,0.676120,0.506924}
\pgfpathmoveto{\pgfpoint{208.799988pt}{226.648422pt}}
\pgflineto{\pgfpoint{217.727982pt}{220.471588pt}}
\pgflineto{\pgfpoint{208.799988pt}{220.471588pt}}
\pgfpathclose
\pgfusepath{fill,stroke}
\pgfpathmoveto{\pgfpoint{208.799988pt}{226.648422pt}}
\pgflineto{\pgfpoint{217.727982pt}{226.648422pt}}
\pgflineto{\pgfpoint{217.727982pt}{220.471588pt}}
\pgfpathclose
\pgfusepath{fill,stroke}
\pgfpathmoveto{\pgfpoint{217.727982pt}{214.294739pt}}
\pgflineto{\pgfpoint{226.655975pt}{208.117905pt}}
\pgflineto{\pgfpoint{217.727982pt}{208.117905pt}}
\pgfpathclose
\pgfusepath{fill,stroke}
\pgfpathmoveto{\pgfpoint{217.727982pt}{214.294739pt}}
\pgflineto{\pgfpoint{226.655975pt}{214.294739pt}}
\pgflineto{\pgfpoint{226.655975pt}{208.117905pt}}
\pgfpathclose
\pgfusepath{fill,stroke}
\pgfpathmoveto{\pgfpoint{217.727982pt}{220.471588pt}}
\pgflineto{\pgfpoint{226.655975pt}{214.294739pt}}
\pgflineto{\pgfpoint{217.727982pt}{214.294739pt}}
\pgfpathclose
\pgfusepath{fill,stroke}
\pgfpathmoveto{\pgfpoint{217.727982pt}{220.471588pt}}
\pgflineto{\pgfpoint{226.655975pt}{220.471588pt}}
\pgflineto{\pgfpoint{226.655975pt}{214.294739pt}}
\pgfpathclose
\pgfusepath{fill,stroke}
\color[rgb]{0.165967,0.690519,0.496752}
\pgfpathmoveto{\pgfpoint{217.727982pt}{226.648422pt}}
\pgflineto{\pgfpoint{226.655975pt}{220.471588pt}}
\pgflineto{\pgfpoint{217.727982pt}{220.471588pt}}
\pgfpathclose
\pgfusepath{fill,stroke}
\pgfpathmoveto{\pgfpoint{226.655975pt}{214.294739pt}}
\pgflineto{\pgfpoint{235.583969pt}{214.294739pt}}
\pgflineto{\pgfpoint{235.583969pt}{208.117905pt}}
\pgfpathclose
\pgfusepath{fill,stroke}
\color[rgb]{0.185538,0.704725,0.485412}
\pgfpathmoveto{\pgfpoint{226.655975pt}{220.471588pt}}
\pgflineto{\pgfpoint{235.583969pt}{214.294739pt}}
\pgflineto{\pgfpoint{226.655975pt}{214.294739pt}}
\pgfpathclose
\pgfusepath{fill,stroke}
\color[rgb]{0.165967,0.690519,0.496752}
\pgfpathmoveto{\pgfpoint{235.583969pt}{208.117905pt}}
\pgflineto{\pgfpoint{244.511993pt}{201.941055pt}}
\pgflineto{\pgfpoint{235.583969pt}{201.941055pt}}
\pgfpathclose
\pgfusepath{fill,stroke}
\pgfpathmoveto{\pgfpoint{235.583969pt}{208.117905pt}}
\pgflineto{\pgfpoint{244.511993pt}{208.117905pt}}
\pgflineto{\pgfpoint{244.511993pt}{201.941055pt}}
\pgfpathclose
\pgfusepath{fill,stroke}
\color[rgb]{0.185538,0.704725,0.485412}
\pgfpathmoveto{\pgfpoint{235.583969pt}{214.294739pt}}
\pgflineto{\pgfpoint{244.511993pt}{208.117905pt}}
\pgflineto{\pgfpoint{235.583969pt}{208.117905pt}}
\pgfpathclose
\pgfusepath{fill,stroke}
\pgfpathmoveto{\pgfpoint{235.583969pt}{214.294739pt}}
\pgflineto{\pgfpoint{244.511993pt}{214.294739pt}}
\pgflineto{\pgfpoint{244.511993pt}{208.117905pt}}
\pgfpathclose
\pgfusepath{fill,stroke}
\pgfpathmoveto{\pgfpoint{244.511993pt}{208.117905pt}}
\pgflineto{\pgfpoint{253.440002pt}{208.117905pt}}
\pgflineto{\pgfpoint{253.440002pt}{201.941055pt}}
\pgfpathclose
\pgfusepath{fill,stroke}
\color[rgb]{0.208030,0.718701,0.472873}
\pgfpathmoveto{\pgfpoint{244.511993pt}{214.294739pt}}
\pgflineto{\pgfpoint{253.440002pt}{208.117905pt}}
\pgflineto{\pgfpoint{244.511993pt}{208.117905pt}}
\pgfpathclose
\pgfusepath{fill,stroke}
\color[rgb]{0.185538,0.704725,0.485412}
\pgfpathmoveto{\pgfpoint{253.440002pt}{201.941055pt}}
\pgflineto{\pgfpoint{262.367981pt}{201.941055pt}}
\pgflineto{\pgfpoint{262.367981pt}{195.764206pt}}
\pgfpathclose
\pgfusepath{fill,stroke}
\color[rgb]{0.208030,0.718701,0.472873}
\pgfpathmoveto{\pgfpoint{253.440002pt}{208.117905pt}}
\pgflineto{\pgfpoint{262.367981pt}{201.941055pt}}
\pgflineto{\pgfpoint{253.440002pt}{201.941055pt}}
\pgfpathclose
\pgfusepath{fill,stroke}
\pgfpathmoveto{\pgfpoint{253.440002pt}{208.117905pt}}
\pgflineto{\pgfpoint{262.367981pt}{208.117905pt}}
\pgflineto{\pgfpoint{262.367981pt}{201.941055pt}}
\pgfpathclose
\pgfusepath{fill,stroke}
\pgfpathmoveto{\pgfpoint{262.367981pt}{195.764206pt}}
\pgflineto{\pgfpoint{271.295990pt}{189.587372pt}}
\pgflineto{\pgfpoint{262.367981pt}{189.587372pt}}
\pgfpathclose
\pgfusepath{fill,stroke}
\pgfpathmoveto{\pgfpoint{262.367981pt}{195.764206pt}}
\pgflineto{\pgfpoint{271.295990pt}{195.764206pt}}
\pgflineto{\pgfpoint{271.295990pt}{189.587372pt}}
\pgfpathclose
\pgfusepath{fill,stroke}
\pgfpathmoveto{\pgfpoint{262.367981pt}{201.941055pt}}
\pgflineto{\pgfpoint{271.295990pt}{195.764206pt}}
\pgflineto{\pgfpoint{262.367981pt}{195.764206pt}}
\pgfpathclose
\pgfusepath{fill,stroke}
\pgfpathmoveto{\pgfpoint{262.367981pt}{201.941055pt}}
\pgflineto{\pgfpoint{271.295990pt}{201.941055pt}}
\pgflineto{\pgfpoint{271.295990pt}{195.764206pt}}
\pgfpathclose
\pgfusepath{fill,stroke}
\color[rgb]{0.233127,0.732406,0.459106}
\pgfpathmoveto{\pgfpoint{262.367981pt}{208.117905pt}}
\pgflineto{\pgfpoint{271.295990pt}{201.941055pt}}
\pgflineto{\pgfpoint{262.367981pt}{201.941055pt}}
\pgfpathclose
\pgfusepath{fill,stroke}
\pgfpathmoveto{\pgfpoint{271.295990pt}{195.764206pt}}
\pgflineto{\pgfpoint{280.223969pt}{195.764206pt}}
\pgflineto{\pgfpoint{280.223969pt}{189.587372pt}}
\pgfpathclose
\pgfusepath{fill,stroke}
\pgfpathmoveto{\pgfpoint{271.295990pt}{201.941055pt}}
\pgflineto{\pgfpoint{280.223969pt}{195.764206pt}}
\pgflineto{\pgfpoint{271.295990pt}{195.764206pt}}
\pgfpathclose
\pgfusepath{fill,stroke}
\pgfpathmoveto{\pgfpoint{271.295990pt}{201.941055pt}}
\pgflineto{\pgfpoint{280.223969pt}{201.941055pt}}
\pgflineto{\pgfpoint{280.223969pt}{195.764206pt}}
\pgfpathclose
\pgfusepath{fill,stroke}
\pgfpathmoveto{\pgfpoint{280.223969pt}{189.587372pt}}
\pgflineto{\pgfpoint{289.151978pt}{183.410522pt}}
\pgflineto{\pgfpoint{280.223969pt}{183.410522pt}}
\pgfpathclose
\pgfusepath{fill,stroke}
\pgfpathmoveto{\pgfpoint{280.223969pt}{189.587372pt}}
\pgflineto{\pgfpoint{289.151978pt}{189.587372pt}}
\pgflineto{\pgfpoint{289.151978pt}{183.410522pt}}
\pgfpathclose
\pgfusepath{fill,stroke}
\color[rgb]{0.260531,0.745802,0.444096}
\pgfpathmoveto{\pgfpoint{280.223969pt}{195.764206pt}}
\pgflineto{\pgfpoint{289.151978pt}{189.587372pt}}
\pgflineto{\pgfpoint{280.223969pt}{189.587372pt}}
\pgfpathclose
\pgfusepath{fill,stroke}
\pgfpathmoveto{\pgfpoint{280.223969pt}{195.764206pt}}
\pgflineto{\pgfpoint{289.151978pt}{195.764206pt}}
\pgflineto{\pgfpoint{289.151978pt}{189.587372pt}}
\pgfpathclose
\pgfusepath{fill,stroke}
\color[rgb]{0.290001,0.758846,0.427826}
\pgfpathmoveto{\pgfpoint{280.223969pt}{201.941055pt}}
\pgflineto{\pgfpoint{289.151978pt}{195.764206pt}}
\pgflineto{\pgfpoint{280.223969pt}{195.764206pt}}
\pgfpathclose
\pgfusepath{fill,stroke}
\color[rgb]{0.260531,0.745802,0.444096}
\pgfpathmoveto{\pgfpoint{289.151978pt}{189.587372pt}}
\pgflineto{\pgfpoint{298.079987pt}{189.587372pt}}
\pgflineto{\pgfpoint{298.079987pt}{183.410522pt}}
\pgfpathclose
\pgfusepath{fill,stroke}
\color[rgb]{0.290001,0.758846,0.427826}
\pgfpathmoveto{\pgfpoint{289.151978pt}{195.764206pt}}
\pgflineto{\pgfpoint{298.079987pt}{189.587372pt}}
\pgflineto{\pgfpoint{289.151978pt}{189.587372pt}}
\pgfpathclose
\pgfusepath{fill,stroke}
\color[rgb]{0.260531,0.745802,0.444096}
\pgfpathmoveto{\pgfpoint{298.079987pt}{183.410522pt}}
\pgflineto{\pgfpoint{307.007965pt}{177.233673pt}}
\pgflineto{\pgfpoint{298.079987pt}{177.233673pt}}
\pgfpathclose
\pgfusepath{fill,stroke}
\pgfpathmoveto{\pgfpoint{298.079987pt}{183.410522pt}}
\pgflineto{\pgfpoint{307.007965pt}{183.410522pt}}
\pgflineto{\pgfpoint{307.007965pt}{177.233673pt}}
\pgfpathclose
\pgfusepath{fill,stroke}
\color[rgb]{0.290001,0.758846,0.427826}
\pgfpathmoveto{\pgfpoint{298.079987pt}{189.587372pt}}
\pgflineto{\pgfpoint{307.007965pt}{183.410522pt}}
\pgflineto{\pgfpoint{298.079987pt}{183.410522pt}}
\pgfpathclose
\pgfusepath{fill,stroke}
\pgfpathmoveto{\pgfpoint{298.079987pt}{189.587372pt}}
\pgflineto{\pgfpoint{307.007965pt}{189.587372pt}}
\pgflineto{\pgfpoint{307.007965pt}{183.410522pt}}
\pgfpathclose
\pgfusepath{fill,stroke}
\pgfpathmoveto{\pgfpoint{307.007965pt}{183.410522pt}}
\pgflineto{\pgfpoint{315.935974pt}{183.410522pt}}
\pgflineto{\pgfpoint{315.935974pt}{177.233673pt}}
\pgfpathclose
\pgfusepath{fill,stroke}
\color[rgb]{0.321330,0.771498,0.410293}
\pgfpathmoveto{\pgfpoint{307.007965pt}{189.587372pt}}
\pgflineto{\pgfpoint{315.935974pt}{183.410522pt}}
\pgflineto{\pgfpoint{307.007965pt}{183.410522pt}}
\pgfpathclose
\pgfusepath{fill,stroke}
\color[rgb]{0.290001,0.758846,0.427826}
\pgfpathmoveto{\pgfpoint{315.935974pt}{177.233673pt}}
\pgflineto{\pgfpoint{324.863983pt}{171.056854pt}}
\pgflineto{\pgfpoint{315.935974pt}{171.056854pt}}
\pgfpathclose
\pgfusepath{fill,stroke}
\pgfpathmoveto{\pgfpoint{315.935974pt}{177.233673pt}}
\pgflineto{\pgfpoint{324.863983pt}{177.233673pt}}
\pgflineto{\pgfpoint{324.863983pt}{171.056854pt}}
\pgfpathclose
\pgfusepath{fill,stroke}
\color[rgb]{0.321330,0.771498,0.410293}
\pgfpathmoveto{\pgfpoint{315.935974pt}{183.410522pt}}
\pgflineto{\pgfpoint{324.863983pt}{177.233673pt}}
\pgflineto{\pgfpoint{315.935974pt}{177.233673pt}}
\pgfpathclose
\pgfusepath{fill,stroke}
\pgfpathmoveto{\pgfpoint{315.935974pt}{183.410522pt}}
\pgflineto{\pgfpoint{324.863983pt}{183.410522pt}}
\pgflineto{\pgfpoint{324.863983pt}{177.233673pt}}
\pgfpathclose
\pgfusepath{fill,stroke}
\pgfpathmoveto{\pgfpoint{324.863983pt}{171.056854pt}}
\pgflineto{\pgfpoint{333.791992pt}{171.056854pt}}
\pgflineto{\pgfpoint{333.791992pt}{164.880005pt}}
\pgfpathclose
\pgfusepath{fill,stroke}
\color[rgb]{0.354355,0.783714,0.391488}
\pgfpathmoveto{\pgfpoint{324.863983pt}{177.233673pt}}
\pgflineto{\pgfpoint{333.791992pt}{171.056854pt}}
\pgflineto{\pgfpoint{324.863983pt}{171.056854pt}}
\pgfpathclose
\pgfusepath{fill,stroke}
\pgfpathmoveto{\pgfpoint{324.863983pt}{177.233673pt}}
\pgflineto{\pgfpoint{333.791992pt}{177.233673pt}}
\pgflineto{\pgfpoint{333.791992pt}{171.056854pt}}
\pgfpathclose
\pgfusepath{fill,stroke}
\pgfpathmoveto{\pgfpoint{324.863983pt}{183.410522pt}}
\pgflineto{\pgfpoint{333.791992pt}{177.233673pt}}
\pgflineto{\pgfpoint{324.863983pt}{177.233673pt}}
\pgfpathclose
\pgfusepath{fill,stroke}
\color[rgb]{0.321330,0.771498,0.410293}
\pgfpathmoveto{\pgfpoint{333.791992pt}{164.880005pt}}
\pgflineto{\pgfpoint{342.719971pt}{164.880005pt}}
\pgflineto{\pgfpoint{342.719971pt}{158.703156pt}}
\pgfpathclose
\pgfusepath{fill,stroke}
\color[rgb]{0.354355,0.783714,0.391488}
\pgfpathmoveto{\pgfpoint{333.791992pt}{171.056854pt}}
\pgflineto{\pgfpoint{342.719971pt}{164.880005pt}}
\pgflineto{\pgfpoint{333.791992pt}{164.880005pt}}
\pgfpathclose
\pgfusepath{fill,stroke}
\pgfpathmoveto{\pgfpoint{333.791992pt}{171.056854pt}}
\pgflineto{\pgfpoint{342.719971pt}{171.056854pt}}
\pgflineto{\pgfpoint{342.719971pt}{164.880005pt}}
\pgfpathclose
\pgfusepath{fill,stroke}
\color[rgb]{0.388930,0.795453,0.371421}
\pgfpathmoveto{\pgfpoint{333.791992pt}{177.233673pt}}
\pgflineto{\pgfpoint{342.719971pt}{171.056854pt}}
\pgflineto{\pgfpoint{333.791992pt}{171.056854pt}}
\pgfpathclose
\pgfusepath{fill,stroke}
\pgfpathmoveto{\pgfpoint{333.791992pt}{177.233673pt}}
\pgflineto{\pgfpoint{342.719971pt}{177.233673pt}}
\pgflineto{\pgfpoint{342.719971pt}{171.056854pt}}
\pgfpathclose
\pgfusepath{fill,stroke}
\color[rgb]{0.354355,0.783714,0.391488}
\pgfpathmoveto{\pgfpoint{342.719971pt}{164.880005pt}}
\pgflineto{\pgfpoint{351.647980pt}{158.703156pt}}
\pgflineto{\pgfpoint{342.719971pt}{158.703156pt}}
\pgfpathclose
\pgfusepath{fill,stroke}
\pgfpathmoveto{\pgfpoint{342.719971pt}{164.880005pt}}
\pgflineto{\pgfpoint{351.647980pt}{164.880005pt}}
\pgflineto{\pgfpoint{351.647980pt}{158.703156pt}}
\pgfpathclose
\pgfusepath{fill,stroke}
\color[rgb]{0.388930,0.795453,0.371421}
\pgfpathmoveto{\pgfpoint{342.719971pt}{171.056854pt}}
\pgflineto{\pgfpoint{351.647980pt}{164.880005pt}}
\pgflineto{\pgfpoint{342.719971pt}{164.880005pt}}
\pgfpathclose
\pgfusepath{fill,stroke}
\pgfpathmoveto{\pgfpoint{342.719971pt}{171.056854pt}}
\pgflineto{\pgfpoint{351.647980pt}{171.056854pt}}
\pgflineto{\pgfpoint{351.647980pt}{164.880005pt}}
\pgfpathclose
\pgfusepath{fill,stroke}
\color[rgb]{0.424933,0.806674,0.350099}
\pgfpathmoveto{\pgfpoint{342.719971pt}{177.233673pt}}
\pgflineto{\pgfpoint{351.647980pt}{171.056854pt}}
\pgflineto{\pgfpoint{342.719971pt}{171.056854pt}}
\pgfpathclose
\pgfusepath{fill,stroke}
\color[rgb]{0.354355,0.783714,0.391488}
\pgfpathmoveto{\pgfpoint{351.647980pt}{158.703156pt}}
\pgflineto{\pgfpoint{360.575958pt}{158.703156pt}}
\pgflineto{\pgfpoint{360.575958pt}{152.526306pt}}
\pgfpathclose
\pgfusepath{fill,stroke}
\color[rgb]{0.388930,0.795453,0.371421}
\pgfpathmoveto{\pgfpoint{351.647980pt}{164.880005pt}}
\pgflineto{\pgfpoint{360.575958pt}{158.703156pt}}
\pgflineto{\pgfpoint{351.647980pt}{158.703156pt}}
\pgfpathclose
\pgfusepath{fill,stroke}
\pgfpathmoveto{\pgfpoint{351.647980pt}{164.880005pt}}
\pgflineto{\pgfpoint{360.575958pt}{164.880005pt}}
\pgflineto{\pgfpoint{360.575958pt}{158.703156pt}}
\pgfpathclose
\pgfusepath{fill,stroke}
\color[rgb]{0.424933,0.806674,0.350099}
\pgfpathmoveto{\pgfpoint{351.647980pt}{171.056854pt}}
\pgflineto{\pgfpoint{360.575958pt}{164.880005pt}}
\pgflineto{\pgfpoint{351.647980pt}{164.880005pt}}
\pgfpathclose
\pgfusepath{fill,stroke}
\color[rgb]{0.388930,0.795453,0.371421}
\pgfpathmoveto{\pgfpoint{360.575958pt}{158.703156pt}}
\pgflineto{\pgfpoint{369.503998pt}{152.526306pt}}
\pgflineto{\pgfpoint{360.575958pt}{152.526306pt}}
\pgfpathclose
\pgfusepath{fill,stroke}
\pgfpathmoveto{\pgfpoint{360.575958pt}{158.703156pt}}
\pgflineto{\pgfpoint{369.503998pt}{158.703156pt}}
\pgflineto{\pgfpoint{369.503998pt}{152.526306pt}}
\pgfpathclose
\pgfusepath{fill,stroke}
\color[rgb]{0.424933,0.806674,0.350099}
\pgfpathmoveto{\pgfpoint{360.575958pt}{164.880005pt}}
\pgflineto{\pgfpoint{369.503998pt}{158.703156pt}}
\pgflineto{\pgfpoint{360.575958pt}{158.703156pt}}
\pgfpathclose
\pgfusepath{fill,stroke}
\pgfpathmoveto{\pgfpoint{360.575958pt}{164.880005pt}}
\pgflineto{\pgfpoint{369.503998pt}{164.880005pt}}
\pgflineto{\pgfpoint{369.503998pt}{158.703156pt}}
\pgfpathclose
\pgfusepath{fill,stroke}
\pgfpathmoveto{\pgfpoint{369.503998pt}{152.526306pt}}
\pgflineto{\pgfpoint{378.431976pt}{152.526306pt}}
\pgflineto{\pgfpoint{378.431976pt}{146.349472pt}}
\pgfpathclose
\pgfusepath{fill,stroke}
\pgfpathmoveto{\pgfpoint{369.503998pt}{158.703156pt}}
\pgflineto{\pgfpoint{378.431976pt}{152.526306pt}}
\pgflineto{\pgfpoint{369.503998pt}{152.526306pt}}
\pgfpathclose
\pgfusepath{fill,stroke}
\pgfpathmoveto{\pgfpoint{369.503998pt}{158.703156pt}}
\pgflineto{\pgfpoint{378.431976pt}{158.703156pt}}
\pgflineto{\pgfpoint{378.431976pt}{152.526306pt}}
\pgfpathclose
\pgfusepath{fill,stroke}
\color[rgb]{0.462247,0.817338,0.327545}
\pgfpathmoveto{\pgfpoint{369.503998pt}{164.880005pt}}
\pgflineto{\pgfpoint{378.431976pt}{158.703156pt}}
\pgflineto{\pgfpoint{369.503998pt}{158.703156pt}}
\pgfpathclose
\pgfusepath{fill,stroke}
\pgfpathmoveto{\pgfpoint{378.431976pt}{152.526306pt}}
\pgflineto{\pgfpoint{387.359985pt}{146.349472pt}}
\pgflineto{\pgfpoint{378.431976pt}{146.349472pt}}
\pgfpathclose
\pgfusepath{fill,stroke}
\pgfpathmoveto{\pgfpoint{378.431976pt}{152.526306pt}}
\pgflineto{\pgfpoint{387.359985pt}{152.526306pt}}
\pgflineto{\pgfpoint{387.359985pt}{146.349472pt}}
\pgfpathclose
\pgfusepath{fill,stroke}
\color[rgb]{0.500754,0.827409,0.303799}
\pgfpathmoveto{\pgfpoint{378.431976pt}{158.703156pt}}
\pgflineto{\pgfpoint{387.359985pt}{152.526306pt}}
\pgflineto{\pgfpoint{378.431976pt}{152.526306pt}}
\pgfpathclose
\pgfusepath{fill,stroke}
\pgfpathmoveto{\pgfpoint{378.431976pt}{158.703156pt}}
\pgflineto{\pgfpoint{387.359985pt}{158.703156pt}}
\pgflineto{\pgfpoint{387.359985pt}{152.526306pt}}
\pgfpathclose
\pgfusepath{fill,stroke}
\color[rgb]{0.462247,0.817338,0.327545}
\pgfpathmoveto{\pgfpoint{387.359985pt}{146.349472pt}}
\pgflineto{\pgfpoint{396.287964pt}{146.349472pt}}
\pgflineto{\pgfpoint{396.287964pt}{140.172638pt}}
\pgfpathclose
\pgfusepath{fill,stroke}
\color[rgb]{0.500754,0.827409,0.303799}
\pgfpathmoveto{\pgfpoint{387.359985pt}{152.526306pt}}
\pgflineto{\pgfpoint{396.287964pt}{146.349472pt}}
\pgflineto{\pgfpoint{387.359985pt}{146.349472pt}}
\pgfpathclose
\pgfusepath{fill,stroke}
\pgfpathmoveto{\pgfpoint{387.359985pt}{152.526306pt}}
\pgflineto{\pgfpoint{396.287964pt}{152.526306pt}}
\pgflineto{\pgfpoint{396.287964pt}{146.349472pt}}
\pgfpathclose
\pgfusepath{fill,stroke}
\color[rgb]{0.540337,0.836858,0.278917}
\pgfpathmoveto{\pgfpoint{387.359985pt}{158.703156pt}}
\pgflineto{\pgfpoint{396.287964pt}{152.526306pt}}
\pgflineto{\pgfpoint{387.359985pt}{152.526306pt}}
\pgfpathclose
\pgfusepath{fill,stroke}
\color[rgb]{0.500754,0.827409,0.303799}
\pgfpathmoveto{\pgfpoint{396.287964pt}{146.349472pt}}
\pgflineto{\pgfpoint{405.216003pt}{140.172638pt}}
\pgflineto{\pgfpoint{396.287964pt}{140.172638pt}}
\pgfpathclose
\pgfusepath{fill,stroke}
\pgfpathmoveto{\pgfpoint{396.287964pt}{146.349472pt}}
\pgflineto{\pgfpoint{405.216003pt}{146.349472pt}}
\pgflineto{\pgfpoint{405.216003pt}{140.172638pt}}
\pgfpathclose
\pgfusepath{fill,stroke}
\color[rgb]{0.540337,0.836858,0.278917}
\pgfpathmoveto{\pgfpoint{396.287964pt}{152.526306pt}}
\pgflineto{\pgfpoint{405.216003pt}{146.349472pt}}
\pgflineto{\pgfpoint{396.287964pt}{146.349472pt}}
\pgfpathclose
\pgfusepath{fill,stroke}
\pgfpathmoveto{\pgfpoint{396.287964pt}{152.526306pt}}
\pgflineto{\pgfpoint{405.216003pt}{152.526306pt}}
\pgflineto{\pgfpoint{405.216003pt}{146.349472pt}}
\pgfpathclose
\pgfusepath{fill,stroke}
\color[rgb]{0.500754,0.827409,0.303799}
\pgfpathmoveto{\pgfpoint{405.216003pt}{140.172638pt}}
\pgflineto{\pgfpoint{414.143982pt}{140.172638pt}}
\pgflineto{\pgfpoint{414.143982pt}{133.995789pt}}
\pgfpathclose
\pgfusepath{fill,stroke}
\color[rgb]{0.540337,0.836858,0.278917}
\pgfpathmoveto{\pgfpoint{405.216003pt}{146.349472pt}}
\pgflineto{\pgfpoint{414.143982pt}{140.172638pt}}
\pgflineto{\pgfpoint{405.216003pt}{140.172638pt}}
\pgfpathclose
\pgfusepath{fill,stroke}
\pgfpathmoveto{\pgfpoint{405.216003pt}{146.349472pt}}
\pgflineto{\pgfpoint{414.143982pt}{146.349472pt}}
\pgflineto{\pgfpoint{414.143982pt}{140.172638pt}}
\pgfpathclose
\pgfusepath{fill,stroke}
\color[rgb]{0.580861,0.845663,0.253001}
\pgfpathmoveto{\pgfpoint{405.216003pt}{152.526306pt}}
\pgflineto{\pgfpoint{414.143982pt}{146.349472pt}}
\pgflineto{\pgfpoint{405.216003pt}{146.349472pt}}
\pgfpathclose
\pgfusepath{fill,stroke}
\color[rgb]{0.500754,0.827409,0.303799}
\pgfpathmoveto{\pgfpoint{414.143982pt}{133.995789pt}}
\pgflineto{\pgfpoint{423.071960pt}{133.995789pt}}
\pgflineto{\pgfpoint{423.071960pt}{127.818947pt}}
\pgfpathclose
\pgfusepath{fill,stroke}
\color[rgb]{0.540337,0.836858,0.278917}
\pgfpathmoveto{\pgfpoint{414.143982pt}{140.172638pt}}
\pgflineto{\pgfpoint{423.071960pt}{133.995789pt}}
\pgflineto{\pgfpoint{414.143982pt}{133.995789pt}}
\pgfpathclose
\pgfusepath{fill,stroke}
\pgfpathmoveto{\pgfpoint{414.143982pt}{140.172638pt}}
\pgflineto{\pgfpoint{423.071960pt}{140.172638pt}}
\pgflineto{\pgfpoint{423.071960pt}{133.995789pt}}
\pgfpathclose
\pgfusepath{fill,stroke}
\color[rgb]{0.580861,0.845663,0.253001}
\pgfpathmoveto{\pgfpoint{414.143982pt}{146.349472pt}}
\pgflineto{\pgfpoint{423.071960pt}{140.172638pt}}
\pgflineto{\pgfpoint{414.143982pt}{140.172638pt}}
\pgfpathclose
\pgfusepath{fill,stroke}
\pgfpathmoveto{\pgfpoint{414.143982pt}{146.349472pt}}
\pgflineto{\pgfpoint{423.071960pt}{146.349472pt}}
\pgflineto{\pgfpoint{423.071960pt}{140.172638pt}}
\pgfpathclose
\pgfusepath{fill,stroke}
\pgfpathmoveto{\pgfpoint{423.071960pt}{133.995789pt}}
\pgflineto{\pgfpoint{432.000000pt}{127.818947pt}}
\pgflineto{\pgfpoint{423.071960pt}{127.818947pt}}
\pgfpathclose
\pgfusepath{fill,stroke}
\pgfpathmoveto{\pgfpoint{423.071960pt}{133.995789pt}}
\pgflineto{\pgfpoint{432.000000pt}{133.995789pt}}
\pgflineto{\pgfpoint{432.000000pt}{127.818947pt}}
\pgfpathclose
\pgfusepath{fill,stroke}
\pgfpathmoveto{\pgfpoint{423.071960pt}{140.172638pt}}
\pgflineto{\pgfpoint{432.000000pt}{133.995789pt}}
\pgflineto{\pgfpoint{423.071960pt}{133.995789pt}}
\pgfpathclose
\pgfusepath{fill,stroke}
\pgfpathmoveto{\pgfpoint{423.071960pt}{140.172638pt}}
\pgflineto{\pgfpoint{432.000000pt}{140.172638pt}}
\pgflineto{\pgfpoint{432.000000pt}{133.995789pt}}
\pgfpathclose
\pgfusepath{fill,stroke}
\color[rgb]{0.622171,0.853816,0.226224}
\pgfpathmoveto{\pgfpoint{423.071960pt}{146.349472pt}}
\pgflineto{\pgfpoint{432.000000pt}{140.172638pt}}
\pgflineto{\pgfpoint{423.071960pt}{140.172638pt}}
\pgfpathclose
\pgfusepath{fill,stroke}
\color[rgb]{0.580861,0.845663,0.253001}
\pgfpathmoveto{\pgfpoint{432.000000pt}{127.818947pt}}
\pgflineto{\pgfpoint{440.927979pt}{127.818947pt}}
\pgflineto{\pgfpoint{440.927979pt}{121.642097pt}}
\pgfpathclose
\pgfusepath{fill,stroke}
\color[rgb]{0.622171,0.853816,0.226224}
\pgfpathmoveto{\pgfpoint{432.000000pt}{133.995789pt}}
\pgflineto{\pgfpoint{440.927979pt}{127.818947pt}}
\pgflineto{\pgfpoint{432.000000pt}{127.818947pt}}
\pgfpathclose
\pgfusepath{fill,stroke}
\pgfpathmoveto{\pgfpoint{432.000000pt}{133.995789pt}}
\pgflineto{\pgfpoint{440.927979pt}{133.995789pt}}
\pgflineto{\pgfpoint{440.927979pt}{127.818947pt}}
\pgfpathclose
\pgfusepath{fill,stroke}
\pgfpathmoveto{\pgfpoint{432.000000pt}{140.172638pt}}
\pgflineto{\pgfpoint{440.927979pt}{133.995789pt}}
\pgflineto{\pgfpoint{432.000000pt}{133.995789pt}}
\pgfpathclose
\pgfusepath{fill,stroke}
\pgfpathmoveto{\pgfpoint{440.927979pt}{127.818947pt}}
\pgflineto{\pgfpoint{449.855957pt}{121.642097pt}}
\pgflineto{\pgfpoint{440.927979pt}{121.642097pt}}
\pgfpathclose
\pgfusepath{fill,stroke}
\pgfpathmoveto{\pgfpoint{440.927979pt}{127.818947pt}}
\pgflineto{\pgfpoint{449.855957pt}{127.818947pt}}
\pgflineto{\pgfpoint{449.855957pt}{121.642097pt}}
\pgfpathclose
\pgfusepath{fill,stroke}
\color[rgb]{0.664087,0.861321,0.198879}
\pgfpathmoveto{\pgfpoint{440.927979pt}{133.995789pt}}
\pgflineto{\pgfpoint{449.855957pt}{127.818947pt}}
\pgflineto{\pgfpoint{440.927979pt}{127.818947pt}}
\pgfpathclose
\pgfusepath{fill,stroke}
\pgfpathmoveto{\pgfpoint{440.927979pt}{133.995789pt}}
\pgflineto{\pgfpoint{449.855957pt}{133.995789pt}}
\pgflineto{\pgfpoint{449.855957pt}{127.818947pt}}
\pgfpathclose
\pgfusepath{fill,stroke}
\color[rgb]{0.622171,0.853816,0.226224}
\pgfpathmoveto{\pgfpoint{449.855957pt}{121.642097pt}}
\pgflineto{\pgfpoint{458.783936pt}{121.642097pt}}
\pgflineto{\pgfpoint{458.783936pt}{115.465263pt}}
\pgfpathclose
\pgfusepath{fill,stroke}
\color[rgb]{0.664087,0.861321,0.198879}
\pgfpathmoveto{\pgfpoint{449.855957pt}{127.818947pt}}
\pgflineto{\pgfpoint{458.783936pt}{121.642097pt}}
\pgflineto{\pgfpoint{449.855957pt}{121.642097pt}}
\pgfpathclose
\pgfusepath{fill,stroke}
\pgfpathmoveto{\pgfpoint{449.855957pt}{127.818947pt}}
\pgflineto{\pgfpoint{458.783936pt}{127.818947pt}}
\pgflineto{\pgfpoint{458.783936pt}{121.642097pt}}
\pgfpathclose
\pgfusepath{fill,stroke}
\color[rgb]{0.706404,0.868206,0.171495}
\pgfpathmoveto{\pgfpoint{449.855957pt}{133.995789pt}}
\pgflineto{\pgfpoint{458.783936pt}{127.818947pt}}
\pgflineto{\pgfpoint{449.855957pt}{127.818947pt}}
\pgfpathclose
\pgfusepath{fill,stroke}
\color[rgb]{0.664087,0.861321,0.198879}
\pgfpathmoveto{\pgfpoint{458.783936pt}{121.642097pt}}
\pgflineto{\pgfpoint{467.711975pt}{115.465263pt}}
\pgflineto{\pgfpoint{458.783936pt}{115.465263pt}}
\pgfpathclose
\pgfusepath{fill,stroke}
\pgfpathmoveto{\pgfpoint{458.783936pt}{121.642097pt}}
\pgflineto{\pgfpoint{467.711975pt}{121.642097pt}}
\pgflineto{\pgfpoint{467.711975pt}{115.465263pt}}
\pgfpathclose
\pgfusepath{fill,stroke}
\color[rgb]{0.290001,0.758846,0.427826}
\pgfpathmoveto{\pgfpoint{289.151978pt}{195.764206pt}}
\pgflineto{\pgfpoint{298.079987pt}{195.764206pt}}
\pgflineto{\pgfpoint{298.079987pt}{189.587372pt}}
\pgfpathclose
\pgfusepath{fill,stroke}
\color[rgb]{0.321330,0.771498,0.410293}
\pgfpathmoveto{\pgfpoint{298.079987pt}{195.764206pt}}
\pgflineto{\pgfpoint{307.007965pt}{189.587372pt}}
\pgflineto{\pgfpoint{298.079987pt}{189.587372pt}}
\pgfpathclose
\pgfusepath{fill,stroke}
\pgfpathmoveto{\pgfpoint{307.007965pt}{189.587372pt}}
\pgflineto{\pgfpoint{315.935974pt}{189.587372pt}}
\pgflineto{\pgfpoint{315.935974pt}{183.410522pt}}
\pgfpathclose
\pgfusepath{fill,stroke}
\color[rgb]{0.354355,0.783714,0.391488}
\pgfpathmoveto{\pgfpoint{315.935974pt}{189.587372pt}}
\pgflineto{\pgfpoint{324.863983pt}{183.410522pt}}
\pgflineto{\pgfpoint{315.935974pt}{183.410522pt}}
\pgfpathclose
\pgfusepath{fill,stroke}
\pgfpathmoveto{\pgfpoint{324.863983pt}{183.410522pt}}
\pgflineto{\pgfpoint{333.791992pt}{183.410522pt}}
\pgflineto{\pgfpoint{333.791992pt}{177.233673pt}}
\pgfpathclose
\pgfusepath{fill,stroke}
\color[rgb]{0.388930,0.795453,0.371421}
\pgfpathmoveto{\pgfpoint{333.791992pt}{183.410522pt}}
\pgflineto{\pgfpoint{342.719971pt}{177.233673pt}}
\pgflineto{\pgfpoint{333.791992pt}{177.233673pt}}
\pgfpathclose
\pgfusepath{fill,stroke}
\color[rgb]{0.424933,0.806674,0.350099}
\pgfpathmoveto{\pgfpoint{351.647980pt}{171.056854pt}}
\pgflineto{\pgfpoint{360.575958pt}{171.056854pt}}
\pgflineto{\pgfpoint{360.575958pt}{164.880005pt}}
\pgfpathclose
\pgfusepath{fill,stroke}
\color[rgb]{0.462247,0.817338,0.327545}
\pgfpathmoveto{\pgfpoint{360.575958pt}{171.056854pt}}
\pgflineto{\pgfpoint{369.503998pt}{164.880005pt}}
\pgflineto{\pgfpoint{360.575958pt}{164.880005pt}}
\pgfpathclose
\pgfusepath{fill,stroke}
\pgfpathmoveto{\pgfpoint{369.503998pt}{164.880005pt}}
\pgflineto{\pgfpoint{378.431976pt}{164.880005pt}}
\pgflineto{\pgfpoint{378.431976pt}{158.703156pt}}
\pgfpathclose
\pgfusepath{fill,stroke}
\color[rgb]{0.500754,0.827409,0.303799}
\pgfpathmoveto{\pgfpoint{378.431976pt}{164.880005pt}}
\pgflineto{\pgfpoint{387.359985pt}{158.703156pt}}
\pgflineto{\pgfpoint{378.431976pt}{158.703156pt}}
\pgfpathclose
\pgfusepath{fill,stroke}
\color[rgb]{0.540337,0.836858,0.278917}
\pgfpathmoveto{\pgfpoint{387.359985pt}{158.703156pt}}
\pgflineto{\pgfpoint{396.287964pt}{158.703156pt}}
\pgflineto{\pgfpoint{396.287964pt}{152.526306pt}}
\pgfpathclose
\pgfusepath{fill,stroke}
\color[rgb]{0.580861,0.845663,0.253001}
\pgfpathmoveto{\pgfpoint{396.287964pt}{158.703156pt}}
\pgflineto{\pgfpoint{405.216003pt}{152.526306pt}}
\pgflineto{\pgfpoint{396.287964pt}{152.526306pt}}
\pgfpathclose
\pgfusepath{fill,stroke}
\color[rgb]{0.622171,0.853816,0.226224}
\pgfpathmoveto{\pgfpoint{432.000000pt}{140.172638pt}}
\pgflineto{\pgfpoint{440.927979pt}{140.172638pt}}
\pgflineto{\pgfpoint{440.927979pt}{133.995789pt}}
\pgfpathclose
\pgfusepath{fill,stroke}
\color[rgb]{0.706404,0.868206,0.171495}
\pgfpathmoveto{\pgfpoint{440.927979pt}{140.172638pt}}
\pgflineto{\pgfpoint{449.855957pt}{133.995789pt}}
\pgflineto{\pgfpoint{440.927979pt}{133.995789pt}}
\pgfpathclose
\pgfusepath{fill,stroke}
\pgfpathmoveto{\pgfpoint{449.855957pt}{133.995789pt}}
\pgflineto{\pgfpoint{458.783936pt}{133.995789pt}}
\pgflineto{\pgfpoint{458.783936pt}{127.818947pt}}
\pgfpathclose
\pgfusepath{fill,stroke}
\color[rgb]{0.233127,0.732406,0.459106}
\pgfpathmoveto{\pgfpoint{253.440002pt}{214.294739pt}}
\pgflineto{\pgfpoint{262.367981pt}{214.294739pt}}
\pgflineto{\pgfpoint{262.367981pt}{208.117905pt}}
\pgfpathclose
\pgfusepath{fill,stroke}
\color[rgb]{0.260531,0.745802,0.444096}
\pgfpathmoveto{\pgfpoint{253.440002pt}{220.471588pt}}
\pgflineto{\pgfpoint{262.367981pt}{214.294739pt}}
\pgflineto{\pgfpoint{253.440002pt}{214.294739pt}}
\pgfpathclose
\pgfusepath{fill,stroke}
\pgfpathmoveto{\pgfpoint{262.367981pt}{214.294739pt}}
\pgflineto{\pgfpoint{271.295990pt}{214.294739pt}}
\pgflineto{\pgfpoint{271.295990pt}{208.117905pt}}
\pgfpathclose
\pgfusepath{fill,stroke}
\color[rgb]{0.290001,0.758846,0.427826}
\pgfpathmoveto{\pgfpoint{271.295990pt}{214.294739pt}}
\pgflineto{\pgfpoint{280.223969pt}{208.117905pt}}
\pgflineto{\pgfpoint{271.295990pt}{208.117905pt}}
\pgfpathclose
\pgfusepath{fill,stroke}
\color[rgb]{0.165967,0.690519,0.496752}
\pgfpathmoveto{\pgfpoint{217.727982pt}{226.648422pt}}
\pgflineto{\pgfpoint{226.655975pt}{226.648422pt}}
\pgflineto{\pgfpoint{226.655975pt}{220.471588pt}}
\pgfpathclose
\pgfusepath{fill,stroke}
\color[rgb]{0.185538,0.704725,0.485412}
\pgfpathmoveto{\pgfpoint{217.727982pt}{232.825272pt}}
\pgflineto{\pgfpoint{226.655975pt}{226.648422pt}}
\pgflineto{\pgfpoint{217.727982pt}{226.648422pt}}
\pgfpathclose
\pgfusepath{fill,stroke}
\color[rgb]{0.208030,0.718701,0.472873}
\pgfpathmoveto{\pgfpoint{235.583969pt}{220.471588pt}}
\pgflineto{\pgfpoint{244.511993pt}{220.471588pt}}
\pgflineto{\pgfpoint{244.511993pt}{214.294739pt}}
\pgfpathclose
\pgfusepath{fill,stroke}
\pgfpathmoveto{\pgfpoint{235.583969pt}{226.648422pt}}
\pgflineto{\pgfpoint{244.511993pt}{220.471588pt}}
\pgflineto{\pgfpoint{235.583969pt}{220.471588pt}}
\pgfpathclose
\pgfusepath{fill,stroke}
\color[rgb]{0.165967,0.690519,0.496752}
\pgfpathmoveto{\pgfpoint{199.871979pt}{239.002106pt}}
\pgflineto{\pgfpoint{208.799988pt}{239.002106pt}}
\pgflineto{\pgfpoint{208.799988pt}{232.825272pt}}
\pgfpathclose
\pgfusepath{fill,stroke}
\pgfpathmoveto{\pgfpoint{208.799988pt}{232.825272pt}}
\pgflineto{\pgfpoint{217.727982pt}{226.648422pt}}
\pgflineto{\pgfpoint{208.799988pt}{226.648422pt}}
\pgfpathclose
\pgfusepath{fill,stroke}
\pgfpathmoveto{\pgfpoint{208.799988pt}{232.825272pt}}
\pgflineto{\pgfpoint{217.727982pt}{232.825272pt}}
\pgflineto{\pgfpoint{217.727982pt}{226.648422pt}}
\pgfpathclose
\pgfusepath{fill,stroke}
\color[rgb]{0.185538,0.704725,0.485412}
\pgfpathmoveto{\pgfpoint{208.799988pt}{239.002106pt}}
\pgflineto{\pgfpoint{217.727982pt}{232.825272pt}}
\pgflineto{\pgfpoint{208.799988pt}{232.825272pt}}
\pgfpathclose
\pgfusepath{fill,stroke}
\pgfpathmoveto{\pgfpoint{208.799988pt}{239.002106pt}}
\pgflineto{\pgfpoint{217.727982pt}{239.002106pt}}
\pgflineto{\pgfpoint{217.727982pt}{232.825272pt}}
\pgfpathclose
\pgfusepath{fill,stroke}
\color[rgb]{0.208030,0.718701,0.472873}
\pgfpathmoveto{\pgfpoint{208.799988pt}{245.178955pt}}
\pgflineto{\pgfpoint{217.727982pt}{239.002106pt}}
\pgflineto{\pgfpoint{208.799988pt}{239.002106pt}}
\pgfpathclose
\pgfusepath{fill,stroke}
\color[rgb]{0.185538,0.704725,0.485412}
\pgfpathmoveto{\pgfpoint{217.727982pt}{232.825272pt}}
\pgflineto{\pgfpoint{226.655975pt}{232.825272pt}}
\pgflineto{\pgfpoint{226.655975pt}{226.648422pt}}
\pgfpathclose
\pgfusepath{fill,stroke}
\color[rgb]{0.208030,0.718701,0.472873}
\pgfpathmoveto{\pgfpoint{217.727982pt}{239.002106pt}}
\pgflineto{\pgfpoint{226.655975pt}{232.825272pt}}
\pgflineto{\pgfpoint{217.727982pt}{232.825272pt}}
\pgfpathclose
\pgfusepath{fill,stroke}
\pgfpathmoveto{\pgfpoint{217.727982pt}{239.002106pt}}
\pgflineto{\pgfpoint{226.655975pt}{239.002106pt}}
\pgflineto{\pgfpoint{226.655975pt}{232.825272pt}}
\pgfpathclose
\pgfusepath{fill,stroke}
\color[rgb]{0.185538,0.704725,0.485412}
\pgfpathmoveto{\pgfpoint{226.655975pt}{220.471588pt}}
\pgflineto{\pgfpoint{235.583969pt}{220.471588pt}}
\pgflineto{\pgfpoint{235.583969pt}{214.294739pt}}
\pgfpathclose
\pgfusepath{fill,stroke}
\pgfpathmoveto{\pgfpoint{226.655975pt}{226.648422pt}}
\pgflineto{\pgfpoint{235.583969pt}{220.471588pt}}
\pgflineto{\pgfpoint{226.655975pt}{220.471588pt}}
\pgfpathclose
\pgfusepath{fill,stroke}
\pgfpathmoveto{\pgfpoint{226.655975pt}{226.648422pt}}
\pgflineto{\pgfpoint{235.583969pt}{226.648422pt}}
\pgflineto{\pgfpoint{235.583969pt}{220.471588pt}}
\pgfpathclose
\pgfusepath{fill,stroke}
\color[rgb]{0.208030,0.718701,0.472873}
\pgfpathmoveto{\pgfpoint{226.655975pt}{232.825272pt}}
\pgflineto{\pgfpoint{235.583969pt}{226.648422pt}}
\pgflineto{\pgfpoint{226.655975pt}{226.648422pt}}
\pgfpathclose
\pgfusepath{fill,stroke}
\pgfpathmoveto{\pgfpoint{226.655975pt}{232.825272pt}}
\pgflineto{\pgfpoint{235.583969pt}{232.825272pt}}
\pgflineto{\pgfpoint{235.583969pt}{226.648422pt}}
\pgfpathclose
\pgfusepath{fill,stroke}
\color[rgb]{0.233127,0.732406,0.459106}
\pgfpathmoveto{\pgfpoint{226.655975pt}{239.002106pt}}
\pgflineto{\pgfpoint{235.583969pt}{232.825272pt}}
\pgflineto{\pgfpoint{226.655975pt}{232.825272pt}}
\pgfpathclose
\pgfusepath{fill,stroke}
\color[rgb]{0.208030,0.718701,0.472873}
\pgfpathmoveto{\pgfpoint{235.583969pt}{220.471588pt}}
\pgflineto{\pgfpoint{244.511993pt}{214.294739pt}}
\pgflineto{\pgfpoint{235.583969pt}{214.294739pt}}
\pgfpathclose
\pgfusepath{fill,stroke}
\pgfpathmoveto{\pgfpoint{235.583969pt}{226.648422pt}}
\pgflineto{\pgfpoint{244.511993pt}{226.648422pt}}
\pgflineto{\pgfpoint{244.511993pt}{220.471588pt}}
\pgfpathclose
\pgfusepath{fill,stroke}
\color[rgb]{0.233127,0.732406,0.459106}
\pgfpathmoveto{\pgfpoint{235.583969pt}{232.825272pt}}
\pgflineto{\pgfpoint{244.511993pt}{226.648422pt}}
\pgflineto{\pgfpoint{235.583969pt}{226.648422pt}}
\pgfpathclose
\pgfusepath{fill,stroke}
\pgfpathmoveto{\pgfpoint{235.583969pt}{232.825272pt}}
\pgflineto{\pgfpoint{244.511993pt}{232.825272pt}}
\pgflineto{\pgfpoint{244.511993pt}{226.648422pt}}
\pgfpathclose
\pgfusepath{fill,stroke}
\color[rgb]{0.208030,0.718701,0.472873}
\pgfpathmoveto{\pgfpoint{244.511993pt}{214.294739pt}}
\pgflineto{\pgfpoint{253.440002pt}{214.294739pt}}
\pgflineto{\pgfpoint{253.440002pt}{208.117905pt}}
\pgfpathclose
\pgfusepath{fill,stroke}
\color[rgb]{0.233127,0.732406,0.459106}
\pgfpathmoveto{\pgfpoint{244.511993pt}{220.471588pt}}
\pgflineto{\pgfpoint{253.440002pt}{214.294739pt}}
\pgflineto{\pgfpoint{244.511993pt}{214.294739pt}}
\pgfpathclose
\pgfusepath{fill,stroke}
\pgfpathmoveto{\pgfpoint{244.511993pt}{220.471588pt}}
\pgflineto{\pgfpoint{253.440002pt}{220.471588pt}}
\pgflineto{\pgfpoint{253.440002pt}{214.294739pt}}
\pgfpathclose
\pgfusepath{fill,stroke}
\color[rgb]{0.260531,0.745802,0.444096}
\pgfpathmoveto{\pgfpoint{244.511993pt}{226.648422pt}}
\pgflineto{\pgfpoint{253.440002pt}{220.471588pt}}
\pgflineto{\pgfpoint{244.511993pt}{220.471588pt}}
\pgfpathclose
\pgfusepath{fill,stroke}
\pgfpathmoveto{\pgfpoint{244.511993pt}{226.648422pt}}
\pgflineto{\pgfpoint{253.440002pt}{226.648422pt}}
\pgflineto{\pgfpoint{253.440002pt}{220.471588pt}}
\pgfpathclose
\pgfusepath{fill,stroke}
\pgfpathmoveto{\pgfpoint{244.511993pt}{232.825272pt}}
\pgflineto{\pgfpoint{253.440002pt}{226.648422pt}}
\pgflineto{\pgfpoint{244.511993pt}{226.648422pt}}
\pgfpathclose
\pgfusepath{fill,stroke}
\color[rgb]{0.233127,0.732406,0.459106}
\pgfpathmoveto{\pgfpoint{253.440002pt}{214.294739pt}}
\pgflineto{\pgfpoint{262.367981pt}{208.117905pt}}
\pgflineto{\pgfpoint{253.440002pt}{208.117905pt}}
\pgfpathclose
\pgfusepath{fill,stroke}
\color[rgb]{0.260531,0.745802,0.444096}
\pgfpathmoveto{\pgfpoint{253.440002pt}{220.471588pt}}
\pgflineto{\pgfpoint{262.367981pt}{220.471588pt}}
\pgflineto{\pgfpoint{262.367981pt}{214.294739pt}}
\pgfpathclose
\pgfusepath{fill,stroke}
\color[rgb]{0.290001,0.758846,0.427826}
\pgfpathmoveto{\pgfpoint{253.440002pt}{226.648422pt}}
\pgflineto{\pgfpoint{262.367981pt}{220.471588pt}}
\pgflineto{\pgfpoint{253.440002pt}{220.471588pt}}
\pgfpathclose
\pgfusepath{fill,stroke}
\color[rgb]{0.233127,0.732406,0.459106}
\pgfpathmoveto{\pgfpoint{262.367981pt}{208.117905pt}}
\pgflineto{\pgfpoint{271.295990pt}{208.117905pt}}
\pgflineto{\pgfpoint{271.295990pt}{201.941055pt}}
\pgfpathclose
\pgfusepath{fill,stroke}
\color[rgb]{0.260531,0.745802,0.444096}
\pgfpathmoveto{\pgfpoint{262.367981pt}{214.294739pt}}
\pgflineto{\pgfpoint{271.295990pt}{208.117905pt}}
\pgflineto{\pgfpoint{262.367981pt}{208.117905pt}}
\pgfpathclose
\pgfusepath{fill,stroke}
\color[rgb]{0.290001,0.758846,0.427826}
\pgfpathmoveto{\pgfpoint{262.367981pt}{220.471588pt}}
\pgflineto{\pgfpoint{271.295990pt}{214.294739pt}}
\pgflineto{\pgfpoint{262.367981pt}{214.294739pt}}
\pgfpathclose
\pgfusepath{fill,stroke}
\pgfpathmoveto{\pgfpoint{262.367981pt}{220.471588pt}}
\pgflineto{\pgfpoint{271.295990pt}{220.471588pt}}
\pgflineto{\pgfpoint{271.295990pt}{214.294739pt}}
\pgfpathclose
\pgfusepath{fill,stroke}
\color[rgb]{0.260531,0.745802,0.444096}
\pgfpathmoveto{\pgfpoint{271.295990pt}{208.117905pt}}
\pgflineto{\pgfpoint{280.223969pt}{201.941055pt}}
\pgflineto{\pgfpoint{271.295990pt}{201.941055pt}}
\pgfpathclose
\pgfusepath{fill,stroke}
\pgfpathmoveto{\pgfpoint{271.295990pt}{208.117905pt}}
\pgflineto{\pgfpoint{280.223969pt}{208.117905pt}}
\pgflineto{\pgfpoint{280.223969pt}{201.941055pt}}
\pgfpathclose
\pgfusepath{fill,stroke}
\color[rgb]{0.290001,0.758846,0.427826}
\pgfpathmoveto{\pgfpoint{271.295990pt}{214.294739pt}}
\pgflineto{\pgfpoint{280.223969pt}{214.294739pt}}
\pgflineto{\pgfpoint{280.223969pt}{208.117905pt}}
\pgfpathclose
\pgfusepath{fill,stroke}
\color[rgb]{0.321330,0.771498,0.410293}
\pgfpathmoveto{\pgfpoint{271.295990pt}{220.471588pt}}
\pgflineto{\pgfpoint{280.223969pt}{214.294739pt}}
\pgflineto{\pgfpoint{271.295990pt}{214.294739pt}}
\pgfpathclose
\pgfusepath{fill,stroke}
\color[rgb]{0.290001,0.758846,0.427826}
\pgfpathmoveto{\pgfpoint{280.223969pt}{201.941055pt}}
\pgflineto{\pgfpoint{289.151978pt}{201.941055pt}}
\pgflineto{\pgfpoint{289.151978pt}{195.764206pt}}
\pgfpathclose
\pgfusepath{fill,stroke}
\pgfpathmoveto{\pgfpoint{280.223969pt}{208.117905pt}}
\pgflineto{\pgfpoint{289.151978pt}{201.941055pt}}
\pgflineto{\pgfpoint{280.223969pt}{201.941055pt}}
\pgfpathclose
\pgfusepath{fill,stroke}
\pgfpathmoveto{\pgfpoint{280.223969pt}{208.117905pt}}
\pgflineto{\pgfpoint{289.151978pt}{208.117905pt}}
\pgflineto{\pgfpoint{289.151978pt}{201.941055pt}}
\pgfpathclose
\pgfusepath{fill,stroke}
\color[rgb]{0.321330,0.771498,0.410293}
\pgfpathmoveto{\pgfpoint{280.223969pt}{214.294739pt}}
\pgflineto{\pgfpoint{289.151978pt}{208.117905pt}}
\pgflineto{\pgfpoint{280.223969pt}{208.117905pt}}
\pgfpathclose
\pgfusepath{fill,stroke}
\pgfpathmoveto{\pgfpoint{280.223969pt}{214.294739pt}}
\pgflineto{\pgfpoint{289.151978pt}{214.294739pt}}
\pgflineto{\pgfpoint{289.151978pt}{208.117905pt}}
\pgfpathclose
\pgfusepath{fill,stroke}
\pgfpathmoveto{\pgfpoint{289.151978pt}{201.941055pt}}
\pgflineto{\pgfpoint{298.079987pt}{195.764206pt}}
\pgflineto{\pgfpoint{289.151978pt}{195.764206pt}}
\pgfpathclose
\pgfusepath{fill,stroke}
\pgfpathmoveto{\pgfpoint{289.151978pt}{201.941055pt}}
\pgflineto{\pgfpoint{298.079987pt}{201.941055pt}}
\pgflineto{\pgfpoint{298.079987pt}{195.764206pt}}
\pgfpathclose
\pgfusepath{fill,stroke}
\pgfpathmoveto{\pgfpoint{289.151978pt}{208.117905pt}}
\pgflineto{\pgfpoint{298.079987pt}{201.941055pt}}
\pgflineto{\pgfpoint{289.151978pt}{201.941055pt}}
\pgfpathclose
\pgfusepath{fill,stroke}
\pgfpathmoveto{\pgfpoint{289.151978pt}{208.117905pt}}
\pgflineto{\pgfpoint{298.079987pt}{208.117905pt}}
\pgflineto{\pgfpoint{298.079987pt}{201.941055pt}}
\pgfpathclose
\pgfusepath{fill,stroke}
\color[rgb]{0.354355,0.783714,0.391488}
\pgfpathmoveto{\pgfpoint{289.151978pt}{214.294739pt}}
\pgflineto{\pgfpoint{298.079987pt}{208.117905pt}}
\pgflineto{\pgfpoint{289.151978pt}{208.117905pt}}
\pgfpathclose
\pgfusepath{fill,stroke}
\color[rgb]{0.321330,0.771498,0.410293}
\pgfpathmoveto{\pgfpoint{298.079987pt}{195.764206pt}}
\pgflineto{\pgfpoint{307.007965pt}{195.764206pt}}
\pgflineto{\pgfpoint{307.007965pt}{189.587372pt}}
\pgfpathclose
\pgfusepath{fill,stroke}
\color[rgb]{0.354355,0.783714,0.391488}
\pgfpathmoveto{\pgfpoint{298.079987pt}{201.941055pt}}
\pgflineto{\pgfpoint{307.007965pt}{195.764206pt}}
\pgflineto{\pgfpoint{298.079987pt}{195.764206pt}}
\pgfpathclose
\pgfusepath{fill,stroke}
\pgfpathmoveto{\pgfpoint{298.079987pt}{201.941055pt}}
\pgflineto{\pgfpoint{307.007965pt}{201.941055pt}}
\pgflineto{\pgfpoint{307.007965pt}{195.764206pt}}
\pgfpathclose
\pgfusepath{fill,stroke}
\color[rgb]{0.388930,0.795453,0.371421}
\pgfpathmoveto{\pgfpoint{298.079987pt}{208.117905pt}}
\pgflineto{\pgfpoint{307.007965pt}{201.941055pt}}
\pgflineto{\pgfpoint{298.079987pt}{201.941055pt}}
\pgfpathclose
\pgfusepath{fill,stroke}
\color[rgb]{0.354355,0.783714,0.391488}
\pgfpathmoveto{\pgfpoint{307.007965pt}{195.764206pt}}
\pgflineto{\pgfpoint{315.935974pt}{189.587372pt}}
\pgflineto{\pgfpoint{307.007965pt}{189.587372pt}}
\pgfpathclose
\pgfusepath{fill,stroke}
\pgfpathmoveto{\pgfpoint{307.007965pt}{195.764206pt}}
\pgflineto{\pgfpoint{315.935974pt}{195.764206pt}}
\pgflineto{\pgfpoint{315.935974pt}{189.587372pt}}
\pgfpathclose
\pgfusepath{fill,stroke}
\color[rgb]{0.388930,0.795453,0.371421}
\pgfpathmoveto{\pgfpoint{307.007965pt}{201.941055pt}}
\pgflineto{\pgfpoint{315.935974pt}{195.764206pt}}
\pgflineto{\pgfpoint{307.007965pt}{195.764206pt}}
\pgfpathclose
\pgfusepath{fill,stroke}
\pgfpathmoveto{\pgfpoint{307.007965pt}{201.941055pt}}
\pgflineto{\pgfpoint{315.935974pt}{201.941055pt}}
\pgflineto{\pgfpoint{315.935974pt}{195.764206pt}}
\pgfpathclose
\pgfusepath{fill,stroke}
\color[rgb]{0.354355,0.783714,0.391488}
\pgfpathmoveto{\pgfpoint{315.935974pt}{189.587372pt}}
\pgflineto{\pgfpoint{324.863983pt}{189.587372pt}}
\pgflineto{\pgfpoint{324.863983pt}{183.410522pt}}
\pgfpathclose
\pgfusepath{fill,stroke}
\color[rgb]{0.388930,0.795453,0.371421}
\pgfpathmoveto{\pgfpoint{315.935974pt}{195.764206pt}}
\pgflineto{\pgfpoint{324.863983pt}{189.587372pt}}
\pgflineto{\pgfpoint{315.935974pt}{189.587372pt}}
\pgfpathclose
\pgfusepath{fill,stroke}
\pgfpathmoveto{\pgfpoint{315.935974pt}{195.764206pt}}
\pgflineto{\pgfpoint{324.863983pt}{195.764206pt}}
\pgflineto{\pgfpoint{324.863983pt}{189.587372pt}}
\pgfpathclose
\pgfusepath{fill,stroke}
\color[rgb]{0.424933,0.806674,0.350099}
\pgfpathmoveto{\pgfpoint{315.935974pt}{201.941055pt}}
\pgflineto{\pgfpoint{324.863983pt}{195.764206pt}}
\pgflineto{\pgfpoint{315.935974pt}{195.764206pt}}
\pgfpathclose
\pgfusepath{fill,stroke}
\color[rgb]{0.388930,0.795453,0.371421}
\pgfpathmoveto{\pgfpoint{324.863983pt}{189.587372pt}}
\pgflineto{\pgfpoint{333.791992pt}{183.410522pt}}
\pgflineto{\pgfpoint{324.863983pt}{183.410522pt}}
\pgfpathclose
\pgfusepath{fill,stroke}
\pgfpathmoveto{\pgfpoint{324.863983pt}{189.587372pt}}
\pgflineto{\pgfpoint{333.791992pt}{189.587372pt}}
\pgflineto{\pgfpoint{333.791992pt}{183.410522pt}}
\pgfpathclose
\pgfusepath{fill,stroke}
\color[rgb]{0.424933,0.806674,0.350099}
\pgfpathmoveto{\pgfpoint{324.863983pt}{195.764206pt}}
\pgflineto{\pgfpoint{333.791992pt}{189.587372pt}}
\pgflineto{\pgfpoint{324.863983pt}{189.587372pt}}
\pgfpathclose
\pgfusepath{fill,stroke}
\pgfpathmoveto{\pgfpoint{324.863983pt}{195.764206pt}}
\pgflineto{\pgfpoint{333.791992pt}{195.764206pt}}
\pgflineto{\pgfpoint{333.791992pt}{189.587372pt}}
\pgfpathclose
\pgfusepath{fill,stroke}
\color[rgb]{0.388930,0.795453,0.371421}
\pgfpathmoveto{\pgfpoint{333.791992pt}{183.410522pt}}
\pgflineto{\pgfpoint{342.719971pt}{183.410522pt}}
\pgflineto{\pgfpoint{342.719971pt}{177.233673pt}}
\pgfpathclose
\pgfusepath{fill,stroke}
\color[rgb]{0.424933,0.806674,0.350099}
\pgfpathmoveto{\pgfpoint{333.791992pt}{189.587372pt}}
\pgflineto{\pgfpoint{342.719971pt}{183.410522pt}}
\pgflineto{\pgfpoint{333.791992pt}{183.410522pt}}
\pgfpathclose
\pgfusepath{fill,stroke}
\pgfpathmoveto{\pgfpoint{333.791992pt}{189.587372pt}}
\pgflineto{\pgfpoint{342.719971pt}{189.587372pt}}
\pgflineto{\pgfpoint{342.719971pt}{183.410522pt}}
\pgfpathclose
\pgfusepath{fill,stroke}
\color[rgb]{0.462247,0.817338,0.327545}
\pgfpathmoveto{\pgfpoint{333.791992pt}{195.764206pt}}
\pgflineto{\pgfpoint{342.719971pt}{189.587372pt}}
\pgflineto{\pgfpoint{333.791992pt}{189.587372pt}}
\pgfpathclose
\pgfusepath{fill,stroke}
\color[rgb]{0.424933,0.806674,0.350099}
\pgfpathmoveto{\pgfpoint{342.719971pt}{177.233673pt}}
\pgflineto{\pgfpoint{351.647980pt}{177.233673pt}}
\pgflineto{\pgfpoint{351.647980pt}{171.056854pt}}
\pgfpathclose
\pgfusepath{fill,stroke}
\color[rgb]{0.462247,0.817338,0.327545}
\pgfpathmoveto{\pgfpoint{342.719971pt}{183.410522pt}}
\pgflineto{\pgfpoint{351.647980pt}{177.233673pt}}
\pgflineto{\pgfpoint{342.719971pt}{177.233673pt}}
\pgfpathclose
\pgfusepath{fill,stroke}
\pgfpathmoveto{\pgfpoint{342.719971pt}{183.410522pt}}
\pgflineto{\pgfpoint{351.647980pt}{183.410522pt}}
\pgflineto{\pgfpoint{351.647980pt}{177.233673pt}}
\pgfpathclose
\pgfusepath{fill,stroke}
\pgfpathmoveto{\pgfpoint{342.719971pt}{189.587372pt}}
\pgflineto{\pgfpoint{351.647980pt}{183.410522pt}}
\pgflineto{\pgfpoint{342.719971pt}{183.410522pt}}
\pgfpathclose
\pgfusepath{fill,stroke}
\pgfpathmoveto{\pgfpoint{342.719971pt}{189.587372pt}}
\pgflineto{\pgfpoint{351.647980pt}{189.587372pt}}
\pgflineto{\pgfpoint{351.647980pt}{183.410522pt}}
\pgfpathclose
\pgfusepath{fill,stroke}
\pgfpathmoveto{\pgfpoint{351.647980pt}{177.233673pt}}
\pgflineto{\pgfpoint{360.575958pt}{171.056854pt}}
\pgflineto{\pgfpoint{351.647980pt}{171.056854pt}}
\pgfpathclose
\pgfusepath{fill,stroke}
\pgfpathmoveto{\pgfpoint{351.647980pt}{177.233673pt}}
\pgflineto{\pgfpoint{360.575958pt}{177.233673pt}}
\pgflineto{\pgfpoint{360.575958pt}{171.056854pt}}
\pgfpathclose
\pgfusepath{fill,stroke}
\color[rgb]{0.500754,0.827409,0.303799}
\pgfpathmoveto{\pgfpoint{351.647980pt}{183.410522pt}}
\pgflineto{\pgfpoint{360.575958pt}{177.233673pt}}
\pgflineto{\pgfpoint{351.647980pt}{177.233673pt}}
\pgfpathclose
\pgfusepath{fill,stroke}
\pgfpathmoveto{\pgfpoint{351.647980pt}{183.410522pt}}
\pgflineto{\pgfpoint{360.575958pt}{183.410522pt}}
\pgflineto{\pgfpoint{360.575958pt}{177.233673pt}}
\pgfpathclose
\pgfusepath{fill,stroke}
\pgfpathmoveto{\pgfpoint{351.647980pt}{189.587372pt}}
\pgflineto{\pgfpoint{360.575958pt}{183.410522pt}}
\pgflineto{\pgfpoint{351.647980pt}{183.410522pt}}
\pgfpathclose
\pgfusepath{fill,stroke}
\color[rgb]{0.462247,0.817338,0.327545}
\pgfpathmoveto{\pgfpoint{360.575958pt}{171.056854pt}}
\pgflineto{\pgfpoint{369.503998pt}{171.056854pt}}
\pgflineto{\pgfpoint{369.503998pt}{164.880005pt}}
\pgfpathclose
\pgfusepath{fill,stroke}
\color[rgb]{0.500754,0.827409,0.303799}
\pgfpathmoveto{\pgfpoint{360.575958pt}{177.233673pt}}
\pgflineto{\pgfpoint{369.503998pt}{171.056854pt}}
\pgflineto{\pgfpoint{360.575958pt}{171.056854pt}}
\pgfpathclose
\pgfusepath{fill,stroke}
\pgfpathmoveto{\pgfpoint{360.575958pt}{177.233673pt}}
\pgflineto{\pgfpoint{369.503998pt}{177.233673pt}}
\pgflineto{\pgfpoint{369.503998pt}{171.056854pt}}
\pgfpathclose
\pgfusepath{fill,stroke}
\color[rgb]{0.540337,0.836858,0.278917}
\pgfpathmoveto{\pgfpoint{360.575958pt}{183.410522pt}}
\pgflineto{\pgfpoint{369.503998pt}{177.233673pt}}
\pgflineto{\pgfpoint{360.575958pt}{177.233673pt}}
\pgfpathclose
\pgfusepath{fill,stroke}
\color[rgb]{0.500754,0.827409,0.303799}
\pgfpathmoveto{\pgfpoint{369.503998pt}{171.056854pt}}
\pgflineto{\pgfpoint{378.431976pt}{164.880005pt}}
\pgflineto{\pgfpoint{369.503998pt}{164.880005pt}}
\pgfpathclose
\pgfusepath{fill,stroke}
\pgfpathmoveto{\pgfpoint{369.503998pt}{171.056854pt}}
\pgflineto{\pgfpoint{378.431976pt}{171.056854pt}}
\pgflineto{\pgfpoint{378.431976pt}{164.880005pt}}
\pgfpathclose
\pgfusepath{fill,stroke}
\color[rgb]{0.540337,0.836858,0.278917}
\pgfpathmoveto{\pgfpoint{369.503998pt}{177.233673pt}}
\pgflineto{\pgfpoint{378.431976pt}{171.056854pt}}
\pgflineto{\pgfpoint{369.503998pt}{171.056854pt}}
\pgfpathclose
\pgfusepath{fill,stroke}
\pgfpathmoveto{\pgfpoint{369.503998pt}{177.233673pt}}
\pgflineto{\pgfpoint{378.431976pt}{177.233673pt}}
\pgflineto{\pgfpoint{378.431976pt}{171.056854pt}}
\pgfpathclose
\pgfusepath{fill,stroke}
\color[rgb]{0.500754,0.827409,0.303799}
\pgfpathmoveto{\pgfpoint{378.431976pt}{164.880005pt}}
\pgflineto{\pgfpoint{387.359985pt}{164.880005pt}}
\pgflineto{\pgfpoint{387.359985pt}{158.703156pt}}
\pgfpathclose
\pgfusepath{fill,stroke}
\color[rgb]{0.540337,0.836858,0.278917}
\pgfpathmoveto{\pgfpoint{378.431976pt}{171.056854pt}}
\pgflineto{\pgfpoint{387.359985pt}{164.880005pt}}
\pgflineto{\pgfpoint{378.431976pt}{164.880005pt}}
\pgfpathclose
\pgfusepath{fill,stroke}
\pgfpathmoveto{\pgfpoint{378.431976pt}{171.056854pt}}
\pgflineto{\pgfpoint{387.359985pt}{171.056854pt}}
\pgflineto{\pgfpoint{387.359985pt}{164.880005pt}}
\pgfpathclose
\pgfusepath{fill,stroke}
\color[rgb]{0.580861,0.845663,0.253001}
\pgfpathmoveto{\pgfpoint{378.431976pt}{177.233673pt}}
\pgflineto{\pgfpoint{387.359985pt}{171.056854pt}}
\pgflineto{\pgfpoint{378.431976pt}{171.056854pt}}
\pgfpathclose
\pgfusepath{fill,stroke}
\color[rgb]{0.540337,0.836858,0.278917}
\pgfpathmoveto{\pgfpoint{387.359985pt}{164.880005pt}}
\pgflineto{\pgfpoint{396.287964pt}{158.703156pt}}
\pgflineto{\pgfpoint{387.359985pt}{158.703156pt}}
\pgfpathclose
\pgfusepath{fill,stroke}
\pgfpathmoveto{\pgfpoint{387.359985pt}{164.880005pt}}
\pgflineto{\pgfpoint{396.287964pt}{164.880005pt}}
\pgflineto{\pgfpoint{396.287964pt}{158.703156pt}}
\pgfpathclose
\pgfusepath{fill,stroke}
\color[rgb]{0.580861,0.845663,0.253001}
\pgfpathmoveto{\pgfpoint{387.359985pt}{171.056854pt}}
\pgflineto{\pgfpoint{396.287964pt}{164.880005pt}}
\pgflineto{\pgfpoint{387.359985pt}{164.880005pt}}
\pgfpathclose
\pgfusepath{fill,stroke}
\pgfpathmoveto{\pgfpoint{387.359985pt}{171.056854pt}}
\pgflineto{\pgfpoint{396.287964pt}{171.056854pt}}
\pgflineto{\pgfpoint{396.287964pt}{164.880005pt}}
\pgfpathclose
\pgfusepath{fill,stroke}
\pgfpathmoveto{\pgfpoint{396.287964pt}{158.703156pt}}
\pgflineto{\pgfpoint{405.216003pt}{158.703156pt}}
\pgflineto{\pgfpoint{405.216003pt}{152.526306pt}}
\pgfpathclose
\pgfusepath{fill,stroke}
\color[rgb]{0.622171,0.853816,0.226224}
\pgfpathmoveto{\pgfpoint{396.287964pt}{164.880005pt}}
\pgflineto{\pgfpoint{405.216003pt}{158.703156pt}}
\pgflineto{\pgfpoint{396.287964pt}{158.703156pt}}
\pgfpathclose
\pgfusepath{fill,stroke}
\pgfpathmoveto{\pgfpoint{396.287964pt}{164.880005pt}}
\pgflineto{\pgfpoint{405.216003pt}{164.880005pt}}
\pgflineto{\pgfpoint{405.216003pt}{158.703156pt}}
\pgfpathclose
\pgfusepath{fill,stroke}
\pgfpathmoveto{\pgfpoint{396.287964pt}{171.056854pt}}
\pgflineto{\pgfpoint{405.216003pt}{164.880005pt}}
\pgflineto{\pgfpoint{396.287964pt}{164.880005pt}}
\pgfpathclose
\pgfusepath{fill,stroke}
\color[rgb]{0.580861,0.845663,0.253001}
\pgfpathmoveto{\pgfpoint{405.216003pt}{152.526306pt}}
\pgflineto{\pgfpoint{414.143982pt}{152.526306pt}}
\pgflineto{\pgfpoint{414.143982pt}{146.349472pt}}
\pgfpathclose
\pgfusepath{fill,stroke}
\color[rgb]{0.622171,0.853816,0.226224}
\pgfpathmoveto{\pgfpoint{405.216003pt}{158.703156pt}}
\pgflineto{\pgfpoint{414.143982pt}{152.526306pt}}
\pgflineto{\pgfpoint{405.216003pt}{152.526306pt}}
\pgfpathclose
\pgfusepath{fill,stroke}
\pgfpathmoveto{\pgfpoint{405.216003pt}{158.703156pt}}
\pgflineto{\pgfpoint{414.143982pt}{158.703156pt}}
\pgflineto{\pgfpoint{414.143982pt}{152.526306pt}}
\pgfpathclose
\pgfusepath{fill,stroke}
\color[rgb]{0.664087,0.861321,0.198879}
\pgfpathmoveto{\pgfpoint{405.216003pt}{164.880005pt}}
\pgflineto{\pgfpoint{414.143982pt}{158.703156pt}}
\pgflineto{\pgfpoint{405.216003pt}{158.703156pt}}
\pgfpathclose
\pgfusepath{fill,stroke}
\pgfpathmoveto{\pgfpoint{405.216003pt}{164.880005pt}}
\pgflineto{\pgfpoint{414.143982pt}{164.880005pt}}
\pgflineto{\pgfpoint{414.143982pt}{158.703156pt}}
\pgfpathclose
\pgfusepath{fill,stroke}
\color[rgb]{0.622171,0.853816,0.226224}
\pgfpathmoveto{\pgfpoint{414.143982pt}{152.526306pt}}
\pgflineto{\pgfpoint{423.071960pt}{146.349472pt}}
\pgflineto{\pgfpoint{414.143982pt}{146.349472pt}}
\pgfpathclose
\pgfusepath{fill,stroke}
\pgfpathmoveto{\pgfpoint{414.143982pt}{152.526306pt}}
\pgflineto{\pgfpoint{423.071960pt}{152.526306pt}}
\pgflineto{\pgfpoint{423.071960pt}{146.349472pt}}
\pgfpathclose
\pgfusepath{fill,stroke}
\color[rgb]{0.664087,0.861321,0.198879}
\pgfpathmoveto{\pgfpoint{414.143982pt}{158.703156pt}}
\pgflineto{\pgfpoint{423.071960pt}{152.526306pt}}
\pgflineto{\pgfpoint{414.143982pt}{152.526306pt}}
\pgfpathclose
\pgfusepath{fill,stroke}
\pgfpathmoveto{\pgfpoint{414.143982pt}{158.703156pt}}
\pgflineto{\pgfpoint{423.071960pt}{158.703156pt}}
\pgflineto{\pgfpoint{423.071960pt}{152.526306pt}}
\pgfpathclose
\pgfusepath{fill,stroke}
\color[rgb]{0.706404,0.868206,0.171495}
\pgfpathmoveto{\pgfpoint{414.143982pt}{164.880005pt}}
\pgflineto{\pgfpoint{423.071960pt}{158.703156pt}}
\pgflineto{\pgfpoint{414.143982pt}{158.703156pt}}
\pgfpathclose
\pgfusepath{fill,stroke}
\color[rgb]{0.622171,0.853816,0.226224}
\pgfpathmoveto{\pgfpoint{423.071960pt}{146.349472pt}}
\pgflineto{\pgfpoint{432.000000pt}{146.349472pt}}
\pgflineto{\pgfpoint{432.000000pt}{140.172638pt}}
\pgfpathclose
\pgfusepath{fill,stroke}
\color[rgb]{0.664087,0.861321,0.198879}
\pgfpathmoveto{\pgfpoint{423.071960pt}{152.526306pt}}
\pgflineto{\pgfpoint{432.000000pt}{146.349472pt}}
\pgflineto{\pgfpoint{423.071960pt}{146.349472pt}}
\pgfpathclose
\pgfusepath{fill,stroke}
\pgfpathmoveto{\pgfpoint{423.071960pt}{152.526306pt}}
\pgflineto{\pgfpoint{432.000000pt}{152.526306pt}}
\pgflineto{\pgfpoint{432.000000pt}{146.349472pt}}
\pgfpathclose
\pgfusepath{fill,stroke}
\color[rgb]{0.706404,0.868206,0.171495}
\pgfpathmoveto{\pgfpoint{423.071960pt}{158.703156pt}}
\pgflineto{\pgfpoint{432.000000pt}{152.526306pt}}
\pgflineto{\pgfpoint{423.071960pt}{152.526306pt}}
\pgfpathclose
\pgfusepath{fill,stroke}
\pgfpathmoveto{\pgfpoint{423.071960pt}{158.703156pt}}
\pgflineto{\pgfpoint{432.000000pt}{158.703156pt}}
\pgflineto{\pgfpoint{432.000000pt}{152.526306pt}}
\pgfpathclose
\pgfusepath{fill,stroke}
\color[rgb]{0.664087,0.861321,0.198879}
\pgfpathmoveto{\pgfpoint{432.000000pt}{146.349472pt}}
\pgflineto{\pgfpoint{440.927979pt}{140.172638pt}}
\pgflineto{\pgfpoint{432.000000pt}{140.172638pt}}
\pgfpathclose
\pgfusepath{fill,stroke}
\pgfpathmoveto{\pgfpoint{432.000000pt}{146.349472pt}}
\pgflineto{\pgfpoint{440.927979pt}{146.349472pt}}
\pgflineto{\pgfpoint{440.927979pt}{140.172638pt}}
\pgfpathclose
\pgfusepath{fill,stroke}
\color[rgb]{0.706404,0.868206,0.171495}
\pgfpathmoveto{\pgfpoint{432.000000pt}{152.526306pt}}
\pgflineto{\pgfpoint{440.927979pt}{146.349472pt}}
\pgflineto{\pgfpoint{432.000000pt}{146.349472pt}}
\pgfpathclose
\pgfusepath{fill,stroke}
\pgfpathmoveto{\pgfpoint{432.000000pt}{152.526306pt}}
\pgflineto{\pgfpoint{440.927979pt}{152.526306pt}}
\pgflineto{\pgfpoint{440.927979pt}{146.349472pt}}
\pgfpathclose
\pgfusepath{fill,stroke}
\color[rgb]{0.748885,0.874522,0.145038}
\pgfpathmoveto{\pgfpoint{432.000000pt}{158.703156pt}}
\pgflineto{\pgfpoint{440.927979pt}{152.526306pt}}
\pgflineto{\pgfpoint{432.000000pt}{152.526306pt}}
\pgfpathclose
\pgfusepath{fill,stroke}
\color[rgb]{0.706404,0.868206,0.171495}
\pgfpathmoveto{\pgfpoint{440.927979pt}{140.172638pt}}
\pgflineto{\pgfpoint{449.855957pt}{140.172638pt}}
\pgflineto{\pgfpoint{449.855957pt}{133.995789pt}}
\pgfpathclose
\pgfusepath{fill,stroke}
\pgfpathmoveto{\pgfpoint{440.927979pt}{146.349472pt}}
\pgflineto{\pgfpoint{449.855957pt}{140.172638pt}}
\pgflineto{\pgfpoint{440.927979pt}{140.172638pt}}
\pgfpathclose
\pgfusepath{fill,stroke}
\pgfpathmoveto{\pgfpoint{440.927979pt}{146.349472pt}}
\pgflineto{\pgfpoint{449.855957pt}{146.349472pt}}
\pgflineto{\pgfpoint{449.855957pt}{140.172638pt}}
\pgfpathclose
\pgfusepath{fill,stroke}
\color[rgb]{0.260531,0.745802,0.444096}
\pgfpathmoveto{\pgfpoint{235.583969pt}{239.002106pt}}
\pgflineto{\pgfpoint{244.511993pt}{232.825272pt}}
\pgflineto{\pgfpoint{235.583969pt}{232.825272pt}}
\pgfpathclose
\pgfusepath{fill,stroke}
\pgfpathmoveto{\pgfpoint{235.583969pt}{239.002106pt}}
\pgflineto{\pgfpoint{244.511993pt}{239.002106pt}}
\pgflineto{\pgfpoint{244.511993pt}{232.825272pt}}
\pgfpathclose
\pgfusepath{fill,stroke}
\pgfpathmoveto{\pgfpoint{244.511993pt}{232.825272pt}}
\pgflineto{\pgfpoint{253.440002pt}{232.825272pt}}
\pgflineto{\pgfpoint{253.440002pt}{226.648422pt}}
\pgfpathclose
\pgfusepath{fill,stroke}
\color[rgb]{0.290001,0.758846,0.427826}
\pgfpathmoveto{\pgfpoint{244.511993pt}{239.002106pt}}
\pgflineto{\pgfpoint{253.440002pt}{232.825272pt}}
\pgflineto{\pgfpoint{244.511993pt}{232.825272pt}}
\pgfpathclose
\pgfusepath{fill,stroke}
\color[rgb]{0.874718,0.890945,0.095351}
\pgfpathmoveto{\pgfpoint{360.575958pt}{245.178955pt}}
\pgflineto{\pgfpoint{369.503998pt}{245.178955pt}}
\pgflineto{\pgfpoint{369.503998pt}{239.002106pt}}
\pgfpathclose
\pgfusepath{fill,stroke}
\color[rgb]{0.915296,0.895974,0.100470}
\pgfpathmoveto{\pgfpoint{360.575958pt}{251.355804pt}}
\pgflineto{\pgfpoint{369.503998pt}{245.178955pt}}
\pgflineto{\pgfpoint{360.575958pt}{245.178955pt}}
\pgfpathclose
\pgfusepath{fill,stroke}
\pgfpathmoveto{\pgfpoint{360.575958pt}{251.355804pt}}
\pgflineto{\pgfpoint{369.503998pt}{251.355804pt}}
\pgflineto{\pgfpoint{369.503998pt}{245.178955pt}}
\pgfpathclose
\pgfusepath{fill,stroke}
\color[rgb]{0.290001,0.758846,0.427826}
\pgfpathmoveto{\pgfpoint{253.440002pt}{226.648422pt}}
\pgflineto{\pgfpoint{262.367981pt}{226.648422pt}}
\pgflineto{\pgfpoint{262.367981pt}{220.471588pt}}
\pgfpathclose
\pgfusepath{fill,stroke}
\pgfpathmoveto{\pgfpoint{253.440002pt}{232.825272pt}}
\pgflineto{\pgfpoint{262.367981pt}{226.648422pt}}
\pgflineto{\pgfpoint{253.440002pt}{226.648422pt}}
\pgfpathclose
\pgfusepath{fill,stroke}
\pgfpathmoveto{\pgfpoint{253.440002pt}{232.825272pt}}
\pgflineto{\pgfpoint{262.367981pt}{232.825272pt}}
\pgflineto{\pgfpoint{262.367981pt}{226.648422pt}}
\pgfpathclose
\pgfusepath{fill,stroke}
\color[rgb]{0.321330,0.771498,0.410293}
\pgfpathmoveto{\pgfpoint{262.367981pt}{226.648422pt}}
\pgflineto{\pgfpoint{271.295990pt}{220.471588pt}}
\pgflineto{\pgfpoint{262.367981pt}{220.471588pt}}
\pgfpathclose
\pgfusepath{fill,stroke}
\pgfpathmoveto{\pgfpoint{262.367981pt}{226.648422pt}}
\pgflineto{\pgfpoint{271.295990pt}{226.648422pt}}
\pgflineto{\pgfpoint{271.295990pt}{220.471588pt}}
\pgfpathclose
\pgfusepath{fill,stroke}
\color[rgb]{0.354355,0.783714,0.391488}
\pgfpathmoveto{\pgfpoint{262.367981pt}{232.825272pt}}
\pgflineto{\pgfpoint{271.295990pt}{226.648422pt}}
\pgflineto{\pgfpoint{262.367981pt}{226.648422pt}}
\pgfpathclose
\pgfusepath{fill,stroke}
\pgfpathmoveto{\pgfpoint{262.367981pt}{232.825272pt}}
\pgflineto{\pgfpoint{271.295990pt}{232.825272pt}}
\pgflineto{\pgfpoint{271.295990pt}{226.648422pt}}
\pgfpathclose
\pgfusepath{fill,stroke}
\pgfpathmoveto{\pgfpoint{262.367981pt}{239.002106pt}}
\pgflineto{\pgfpoint{271.295990pt}{232.825272pt}}
\pgflineto{\pgfpoint{262.367981pt}{232.825272pt}}
\pgfpathclose
\pgfusepath{fill,stroke}
\pgfpathmoveto{\pgfpoint{262.367981pt}{239.002106pt}}
\pgflineto{\pgfpoint{271.295990pt}{239.002106pt}}
\pgflineto{\pgfpoint{271.295990pt}{232.825272pt}}
\pgfpathclose
\pgfusepath{fill,stroke}
\color[rgb]{0.321330,0.771498,0.410293}
\pgfpathmoveto{\pgfpoint{271.295990pt}{220.471588pt}}
\pgflineto{\pgfpoint{280.223969pt}{220.471588pt}}
\pgflineto{\pgfpoint{280.223969pt}{214.294739pt}}
\pgfpathclose
\pgfusepath{fill,stroke}
\color[rgb]{0.354355,0.783714,0.391488}
\pgfpathmoveto{\pgfpoint{271.295990pt}{226.648422pt}}
\pgflineto{\pgfpoint{280.223969pt}{220.471588pt}}
\pgflineto{\pgfpoint{271.295990pt}{220.471588pt}}
\pgfpathclose
\pgfusepath{fill,stroke}
\pgfpathmoveto{\pgfpoint{271.295990pt}{226.648422pt}}
\pgflineto{\pgfpoint{280.223969pt}{226.648422pt}}
\pgflineto{\pgfpoint{280.223969pt}{220.471588pt}}
\pgfpathclose
\pgfusepath{fill,stroke}
\color[rgb]{0.388930,0.795453,0.371421}
\pgfpathmoveto{\pgfpoint{271.295990pt}{232.825272pt}}
\pgflineto{\pgfpoint{280.223969pt}{226.648422pt}}
\pgflineto{\pgfpoint{271.295990pt}{226.648422pt}}
\pgfpathclose
\pgfusepath{fill,stroke}
\pgfpathmoveto{\pgfpoint{271.295990pt}{232.825272pt}}
\pgflineto{\pgfpoint{280.223969pt}{232.825272pt}}
\pgflineto{\pgfpoint{280.223969pt}{226.648422pt}}
\pgfpathclose
\pgfusepath{fill,stroke}
\pgfpathmoveto{\pgfpoint{271.295990pt}{239.002106pt}}
\pgflineto{\pgfpoint{280.223969pt}{232.825272pt}}
\pgflineto{\pgfpoint{271.295990pt}{232.825272pt}}
\pgfpathclose
\pgfusepath{fill,stroke}
\pgfpathmoveto{\pgfpoint{271.295990pt}{239.002106pt}}
\pgflineto{\pgfpoint{280.223969pt}{239.002106pt}}
\pgflineto{\pgfpoint{280.223969pt}{232.825272pt}}
\pgfpathclose
\pgfusepath{fill,stroke}
\color[rgb]{0.354355,0.783714,0.391488}
\pgfpathmoveto{\pgfpoint{280.223969pt}{220.471588pt}}
\pgflineto{\pgfpoint{289.151978pt}{214.294739pt}}
\pgflineto{\pgfpoint{280.223969pt}{214.294739pt}}
\pgfpathclose
\pgfusepath{fill,stroke}
\pgfpathmoveto{\pgfpoint{280.223969pt}{220.471588pt}}
\pgflineto{\pgfpoint{289.151978pt}{220.471588pt}}
\pgflineto{\pgfpoint{289.151978pt}{214.294739pt}}
\pgfpathclose
\pgfusepath{fill,stroke}
\color[rgb]{0.388930,0.795453,0.371421}
\pgfpathmoveto{\pgfpoint{280.223969pt}{226.648422pt}}
\pgflineto{\pgfpoint{289.151978pt}{220.471588pt}}
\pgflineto{\pgfpoint{280.223969pt}{220.471588pt}}
\pgfpathclose
\pgfusepath{fill,stroke}
\pgfpathmoveto{\pgfpoint{280.223969pt}{226.648422pt}}
\pgflineto{\pgfpoint{289.151978pt}{226.648422pt}}
\pgflineto{\pgfpoint{289.151978pt}{220.471588pt}}
\pgfpathclose
\pgfusepath{fill,stroke}
\color[rgb]{0.424933,0.806674,0.350099}
\pgfpathmoveto{\pgfpoint{280.223969pt}{232.825272pt}}
\pgflineto{\pgfpoint{289.151978pt}{226.648422pt}}
\pgflineto{\pgfpoint{280.223969pt}{226.648422pt}}
\pgfpathclose
\pgfusepath{fill,stroke}
\pgfpathmoveto{\pgfpoint{280.223969pt}{232.825272pt}}
\pgflineto{\pgfpoint{289.151978pt}{232.825272pt}}
\pgflineto{\pgfpoint{289.151978pt}{226.648422pt}}
\pgfpathclose
\pgfusepath{fill,stroke}
\color[rgb]{0.462247,0.817338,0.327545}
\pgfpathmoveto{\pgfpoint{280.223969pt}{239.002106pt}}
\pgflineto{\pgfpoint{289.151978pt}{232.825272pt}}
\pgflineto{\pgfpoint{280.223969pt}{232.825272pt}}
\pgfpathclose
\pgfusepath{fill,stroke}
\pgfpathmoveto{\pgfpoint{280.223969pt}{239.002106pt}}
\pgflineto{\pgfpoint{289.151978pt}{239.002106pt}}
\pgflineto{\pgfpoint{289.151978pt}{232.825272pt}}
\pgfpathclose
\pgfusepath{fill,stroke}
\color[rgb]{0.354355,0.783714,0.391488}
\pgfpathmoveto{\pgfpoint{289.151978pt}{214.294739pt}}
\pgflineto{\pgfpoint{298.079987pt}{214.294739pt}}
\pgflineto{\pgfpoint{298.079987pt}{208.117905pt}}
\pgfpathclose
\pgfusepath{fill,stroke}
\color[rgb]{0.388930,0.795453,0.371421}
\pgfpathmoveto{\pgfpoint{289.151978pt}{220.471588pt}}
\pgflineto{\pgfpoint{298.079987pt}{214.294739pt}}
\pgflineto{\pgfpoint{289.151978pt}{214.294739pt}}
\pgfpathclose
\pgfusepath{fill,stroke}
\pgfpathmoveto{\pgfpoint{289.151978pt}{220.471588pt}}
\pgflineto{\pgfpoint{298.079987pt}{220.471588pt}}
\pgflineto{\pgfpoint{298.079987pt}{214.294739pt}}
\pgfpathclose
\pgfusepath{fill,stroke}
\color[rgb]{0.424933,0.806674,0.350099}
\pgfpathmoveto{\pgfpoint{289.151978pt}{226.648422pt}}
\pgflineto{\pgfpoint{298.079987pt}{220.471588pt}}
\pgflineto{\pgfpoint{289.151978pt}{220.471588pt}}
\pgfpathclose
\pgfusepath{fill,stroke}
\pgfpathmoveto{\pgfpoint{289.151978pt}{226.648422pt}}
\pgflineto{\pgfpoint{298.079987pt}{226.648422pt}}
\pgflineto{\pgfpoint{298.079987pt}{220.471588pt}}
\pgfpathclose
\pgfusepath{fill,stroke}
\color[rgb]{0.462247,0.817338,0.327545}
\pgfpathmoveto{\pgfpoint{289.151978pt}{232.825272pt}}
\pgflineto{\pgfpoint{298.079987pt}{226.648422pt}}
\pgflineto{\pgfpoint{289.151978pt}{226.648422pt}}
\pgfpathclose
\pgfusepath{fill,stroke}
\pgfpathmoveto{\pgfpoint{289.151978pt}{232.825272pt}}
\pgflineto{\pgfpoint{298.079987pt}{232.825272pt}}
\pgflineto{\pgfpoint{298.079987pt}{226.648422pt}}
\pgfpathclose
\pgfusepath{fill,stroke}
\color[rgb]{0.500754,0.827409,0.303799}
\pgfpathmoveto{\pgfpoint{289.151978pt}{239.002106pt}}
\pgflineto{\pgfpoint{298.079987pt}{232.825272pt}}
\pgflineto{\pgfpoint{289.151978pt}{232.825272pt}}
\pgfpathclose
\pgfusepath{fill,stroke}
\pgfpathmoveto{\pgfpoint{289.151978pt}{239.002106pt}}
\pgflineto{\pgfpoint{298.079987pt}{239.002106pt}}
\pgflineto{\pgfpoint{298.079987pt}{232.825272pt}}
\pgfpathclose
\pgfusepath{fill,stroke}
\color[rgb]{0.388930,0.795453,0.371421}
\pgfpathmoveto{\pgfpoint{298.079987pt}{208.117905pt}}
\pgflineto{\pgfpoint{307.007965pt}{208.117905pt}}
\pgflineto{\pgfpoint{307.007965pt}{201.941055pt}}
\pgfpathclose
\pgfusepath{fill,stroke}
\pgfpathmoveto{\pgfpoint{298.079987pt}{214.294739pt}}
\pgflineto{\pgfpoint{307.007965pt}{208.117905pt}}
\pgflineto{\pgfpoint{298.079987pt}{208.117905pt}}
\pgfpathclose
\pgfusepath{fill,stroke}
\pgfpathmoveto{\pgfpoint{298.079987pt}{214.294739pt}}
\pgflineto{\pgfpoint{307.007965pt}{214.294739pt}}
\pgflineto{\pgfpoint{307.007965pt}{208.117905pt}}
\pgfpathclose
\pgfusepath{fill,stroke}
\color[rgb]{0.424933,0.806674,0.350099}
\pgfpathmoveto{\pgfpoint{298.079987pt}{220.471588pt}}
\pgflineto{\pgfpoint{307.007965pt}{214.294739pt}}
\pgflineto{\pgfpoint{298.079987pt}{214.294739pt}}
\pgfpathclose
\pgfusepath{fill,stroke}
\pgfpathmoveto{\pgfpoint{298.079987pt}{220.471588pt}}
\pgflineto{\pgfpoint{307.007965pt}{220.471588pt}}
\pgflineto{\pgfpoint{307.007965pt}{214.294739pt}}
\pgfpathclose
\pgfusepath{fill,stroke}
\color[rgb]{0.462247,0.817338,0.327545}
\pgfpathmoveto{\pgfpoint{298.079987pt}{226.648422pt}}
\pgflineto{\pgfpoint{307.007965pt}{220.471588pt}}
\pgflineto{\pgfpoint{298.079987pt}{220.471588pt}}
\pgfpathclose
\pgfusepath{fill,stroke}
\pgfpathmoveto{\pgfpoint{298.079987pt}{226.648422pt}}
\pgflineto{\pgfpoint{307.007965pt}{226.648422pt}}
\pgflineto{\pgfpoint{307.007965pt}{220.471588pt}}
\pgfpathclose
\pgfusepath{fill,stroke}
\color[rgb]{0.500754,0.827409,0.303799}
\pgfpathmoveto{\pgfpoint{298.079987pt}{232.825272pt}}
\pgflineto{\pgfpoint{307.007965pt}{226.648422pt}}
\pgflineto{\pgfpoint{298.079987pt}{226.648422pt}}
\pgfpathclose
\pgfusepath{fill,stroke}
\pgfpathmoveto{\pgfpoint{298.079987pt}{232.825272pt}}
\pgflineto{\pgfpoint{307.007965pt}{232.825272pt}}
\pgflineto{\pgfpoint{307.007965pt}{226.648422pt}}
\pgfpathclose
\pgfusepath{fill,stroke}
\color[rgb]{0.540337,0.836858,0.278917}
\pgfpathmoveto{\pgfpoint{298.079987pt}{239.002106pt}}
\pgflineto{\pgfpoint{307.007965pt}{232.825272pt}}
\pgflineto{\pgfpoint{298.079987pt}{232.825272pt}}
\pgfpathclose
\pgfusepath{fill,stroke}
\pgfpathmoveto{\pgfpoint{298.079987pt}{239.002106pt}}
\pgflineto{\pgfpoint{307.007965pt}{239.002106pt}}
\pgflineto{\pgfpoint{307.007965pt}{232.825272pt}}
\pgfpathclose
\pgfusepath{fill,stroke}
\color[rgb]{0.424933,0.806674,0.350099}
\pgfpathmoveto{\pgfpoint{307.007965pt}{208.117905pt}}
\pgflineto{\pgfpoint{315.935974pt}{201.941055pt}}
\pgflineto{\pgfpoint{307.007965pt}{201.941055pt}}
\pgfpathclose
\pgfusepath{fill,stroke}
\pgfpathmoveto{\pgfpoint{307.007965pt}{208.117905pt}}
\pgflineto{\pgfpoint{315.935974pt}{208.117905pt}}
\pgflineto{\pgfpoint{315.935974pt}{201.941055pt}}
\pgfpathclose
\pgfusepath{fill,stroke}
\pgfpathmoveto{\pgfpoint{307.007965pt}{214.294739pt}}
\pgflineto{\pgfpoint{315.935974pt}{208.117905pt}}
\pgflineto{\pgfpoint{307.007965pt}{208.117905pt}}
\pgfpathclose
\pgfusepath{fill,stroke}
\pgfpathmoveto{\pgfpoint{307.007965pt}{214.294739pt}}
\pgflineto{\pgfpoint{315.935974pt}{214.294739pt}}
\pgflineto{\pgfpoint{315.935974pt}{208.117905pt}}
\pgfpathclose
\pgfusepath{fill,stroke}
\color[rgb]{0.462247,0.817338,0.327545}
\pgfpathmoveto{\pgfpoint{307.007965pt}{220.471588pt}}
\pgflineto{\pgfpoint{315.935974pt}{214.294739pt}}
\pgflineto{\pgfpoint{307.007965pt}{214.294739pt}}
\pgfpathclose
\pgfusepath{fill,stroke}
\pgfpathmoveto{\pgfpoint{307.007965pt}{220.471588pt}}
\pgflineto{\pgfpoint{315.935974pt}{220.471588pt}}
\pgflineto{\pgfpoint{315.935974pt}{214.294739pt}}
\pgfpathclose
\pgfusepath{fill,stroke}
\color[rgb]{0.500754,0.827409,0.303799}
\pgfpathmoveto{\pgfpoint{307.007965pt}{226.648422pt}}
\pgflineto{\pgfpoint{315.935974pt}{220.471588pt}}
\pgflineto{\pgfpoint{307.007965pt}{220.471588pt}}
\pgfpathclose
\pgfusepath{fill,stroke}
\pgfpathmoveto{\pgfpoint{307.007965pt}{226.648422pt}}
\pgflineto{\pgfpoint{315.935974pt}{226.648422pt}}
\pgflineto{\pgfpoint{315.935974pt}{220.471588pt}}
\pgfpathclose
\pgfusepath{fill,stroke}
\color[rgb]{0.540337,0.836858,0.278917}
\pgfpathmoveto{\pgfpoint{307.007965pt}{232.825272pt}}
\pgflineto{\pgfpoint{315.935974pt}{226.648422pt}}
\pgflineto{\pgfpoint{307.007965pt}{226.648422pt}}
\pgfpathclose
\pgfusepath{fill,stroke}
\pgfpathmoveto{\pgfpoint{307.007965pt}{232.825272pt}}
\pgflineto{\pgfpoint{315.935974pt}{232.825272pt}}
\pgflineto{\pgfpoint{315.935974pt}{226.648422pt}}
\pgfpathclose
\pgfusepath{fill,stroke}
\color[rgb]{0.580861,0.845663,0.253001}
\pgfpathmoveto{\pgfpoint{307.007965pt}{239.002106pt}}
\pgflineto{\pgfpoint{315.935974pt}{232.825272pt}}
\pgflineto{\pgfpoint{307.007965pt}{232.825272pt}}
\pgfpathclose
\pgfusepath{fill,stroke}
\pgfpathmoveto{\pgfpoint{307.007965pt}{239.002106pt}}
\pgflineto{\pgfpoint{315.935974pt}{239.002106pt}}
\pgflineto{\pgfpoint{315.935974pt}{232.825272pt}}
\pgfpathclose
\pgfusepath{fill,stroke}
\color[rgb]{0.622171,0.853816,0.226224}
\pgfpathmoveto{\pgfpoint{307.007965pt}{245.178955pt}}
\pgflineto{\pgfpoint{315.935974pt}{239.002106pt}}
\pgflineto{\pgfpoint{307.007965pt}{239.002106pt}}
\pgfpathclose
\pgfusepath{fill,stroke}
\pgfpathmoveto{\pgfpoint{307.007965pt}{245.178955pt}}
\pgflineto{\pgfpoint{315.935974pt}{245.178955pt}}
\pgflineto{\pgfpoint{315.935974pt}{239.002106pt}}
\pgfpathclose
\pgfusepath{fill,stroke}
\color[rgb]{0.424933,0.806674,0.350099}
\pgfpathmoveto{\pgfpoint{315.935974pt}{201.941055pt}}
\pgflineto{\pgfpoint{324.863983pt}{201.941055pt}}
\pgflineto{\pgfpoint{324.863983pt}{195.764206pt}}
\pgfpathclose
\pgfusepath{fill,stroke}
\color[rgb]{0.462247,0.817338,0.327545}
\pgfpathmoveto{\pgfpoint{315.935974pt}{208.117905pt}}
\pgflineto{\pgfpoint{324.863983pt}{201.941055pt}}
\pgflineto{\pgfpoint{315.935974pt}{201.941055pt}}
\pgfpathclose
\pgfusepath{fill,stroke}
\pgfpathmoveto{\pgfpoint{315.935974pt}{208.117905pt}}
\pgflineto{\pgfpoint{324.863983pt}{208.117905pt}}
\pgflineto{\pgfpoint{324.863983pt}{201.941055pt}}
\pgfpathclose
\pgfusepath{fill,stroke}
\color[rgb]{0.500754,0.827409,0.303799}
\pgfpathmoveto{\pgfpoint{315.935974pt}{214.294739pt}}
\pgflineto{\pgfpoint{324.863983pt}{208.117905pt}}
\pgflineto{\pgfpoint{315.935974pt}{208.117905pt}}
\pgfpathclose
\pgfusepath{fill,stroke}
\pgfpathmoveto{\pgfpoint{315.935974pt}{214.294739pt}}
\pgflineto{\pgfpoint{324.863983pt}{214.294739pt}}
\pgflineto{\pgfpoint{324.863983pt}{208.117905pt}}
\pgfpathclose
\pgfusepath{fill,stroke}
\pgfpathmoveto{\pgfpoint{315.935974pt}{220.471588pt}}
\pgflineto{\pgfpoint{324.863983pt}{214.294739pt}}
\pgflineto{\pgfpoint{315.935974pt}{214.294739pt}}
\pgfpathclose
\pgfusepath{fill,stroke}
\pgfpathmoveto{\pgfpoint{315.935974pt}{220.471588pt}}
\pgflineto{\pgfpoint{324.863983pt}{220.471588pt}}
\pgflineto{\pgfpoint{324.863983pt}{214.294739pt}}
\pgfpathclose
\pgfusepath{fill,stroke}
\color[rgb]{0.540337,0.836858,0.278917}
\pgfpathmoveto{\pgfpoint{315.935974pt}{226.648422pt}}
\pgflineto{\pgfpoint{324.863983pt}{220.471588pt}}
\pgflineto{\pgfpoint{315.935974pt}{220.471588pt}}
\pgfpathclose
\pgfusepath{fill,stroke}
\pgfpathmoveto{\pgfpoint{315.935974pt}{226.648422pt}}
\pgflineto{\pgfpoint{324.863983pt}{226.648422pt}}
\pgflineto{\pgfpoint{324.863983pt}{220.471588pt}}
\pgfpathclose
\pgfusepath{fill,stroke}
\color[rgb]{0.580861,0.845663,0.253001}
\pgfpathmoveto{\pgfpoint{315.935974pt}{232.825272pt}}
\pgflineto{\pgfpoint{324.863983pt}{226.648422pt}}
\pgflineto{\pgfpoint{315.935974pt}{226.648422pt}}
\pgfpathclose
\pgfusepath{fill,stroke}
\pgfpathmoveto{\pgfpoint{315.935974pt}{232.825272pt}}
\pgflineto{\pgfpoint{324.863983pt}{232.825272pt}}
\pgflineto{\pgfpoint{324.863983pt}{226.648422pt}}
\pgfpathclose
\pgfusepath{fill,stroke}
\color[rgb]{0.622171,0.853816,0.226224}
\pgfpathmoveto{\pgfpoint{315.935974pt}{239.002106pt}}
\pgflineto{\pgfpoint{324.863983pt}{232.825272pt}}
\pgflineto{\pgfpoint{315.935974pt}{232.825272pt}}
\pgfpathclose
\pgfusepath{fill,stroke}
\pgfpathmoveto{\pgfpoint{315.935974pt}{239.002106pt}}
\pgflineto{\pgfpoint{324.863983pt}{239.002106pt}}
\pgflineto{\pgfpoint{324.863983pt}{232.825272pt}}
\pgfpathclose
\pgfusepath{fill,stroke}
\color[rgb]{0.664087,0.861321,0.198879}
\pgfpathmoveto{\pgfpoint{315.935974pt}{245.178955pt}}
\pgflineto{\pgfpoint{324.863983pt}{239.002106pt}}
\pgflineto{\pgfpoint{315.935974pt}{239.002106pt}}
\pgfpathclose
\pgfusepath{fill,stroke}
\pgfpathmoveto{\pgfpoint{315.935974pt}{245.178955pt}}
\pgflineto{\pgfpoint{324.863983pt}{245.178955pt}}
\pgflineto{\pgfpoint{324.863983pt}{239.002106pt}}
\pgfpathclose
\pgfusepath{fill,stroke}
\color[rgb]{0.462247,0.817338,0.327545}
\pgfpathmoveto{\pgfpoint{324.863983pt}{201.941055pt}}
\pgflineto{\pgfpoint{333.791992pt}{195.764206pt}}
\pgflineto{\pgfpoint{324.863983pt}{195.764206pt}}
\pgfpathclose
\pgfusepath{fill,stroke}
\pgfpathmoveto{\pgfpoint{324.863983pt}{201.941055pt}}
\pgflineto{\pgfpoint{333.791992pt}{201.941055pt}}
\pgflineto{\pgfpoint{333.791992pt}{195.764206pt}}
\pgfpathclose
\pgfusepath{fill,stroke}
\color[rgb]{0.500754,0.827409,0.303799}
\pgfpathmoveto{\pgfpoint{324.863983pt}{208.117905pt}}
\pgflineto{\pgfpoint{333.791992pt}{201.941055pt}}
\pgflineto{\pgfpoint{324.863983pt}{201.941055pt}}
\pgfpathclose
\pgfusepath{fill,stroke}
\pgfpathmoveto{\pgfpoint{324.863983pt}{208.117905pt}}
\pgflineto{\pgfpoint{333.791992pt}{208.117905pt}}
\pgflineto{\pgfpoint{333.791992pt}{201.941055pt}}
\pgfpathclose
\pgfusepath{fill,stroke}
\color[rgb]{0.540337,0.836858,0.278917}
\pgfpathmoveto{\pgfpoint{324.863983pt}{214.294739pt}}
\pgflineto{\pgfpoint{333.791992pt}{208.117905pt}}
\pgflineto{\pgfpoint{324.863983pt}{208.117905pt}}
\pgfpathclose
\pgfusepath{fill,stroke}
\pgfpathmoveto{\pgfpoint{324.863983pt}{214.294739pt}}
\pgflineto{\pgfpoint{333.791992pt}{214.294739pt}}
\pgflineto{\pgfpoint{333.791992pt}{208.117905pt}}
\pgfpathclose
\pgfusepath{fill,stroke}
\pgfpathmoveto{\pgfpoint{324.863983pt}{220.471588pt}}
\pgflineto{\pgfpoint{333.791992pt}{214.294739pt}}
\pgflineto{\pgfpoint{324.863983pt}{214.294739pt}}
\pgfpathclose
\pgfusepath{fill,stroke}
\pgfpathmoveto{\pgfpoint{324.863983pt}{220.471588pt}}
\pgflineto{\pgfpoint{333.791992pt}{220.471588pt}}
\pgflineto{\pgfpoint{333.791992pt}{214.294739pt}}
\pgfpathclose
\pgfusepath{fill,stroke}
\color[rgb]{0.580861,0.845663,0.253001}
\pgfpathmoveto{\pgfpoint{324.863983pt}{226.648422pt}}
\pgflineto{\pgfpoint{333.791992pt}{220.471588pt}}
\pgflineto{\pgfpoint{324.863983pt}{220.471588pt}}
\pgfpathclose
\pgfusepath{fill,stroke}
\pgfpathmoveto{\pgfpoint{324.863983pt}{226.648422pt}}
\pgflineto{\pgfpoint{333.791992pt}{226.648422pt}}
\pgflineto{\pgfpoint{333.791992pt}{220.471588pt}}
\pgfpathclose
\pgfusepath{fill,stroke}
\color[rgb]{0.622171,0.853816,0.226224}
\pgfpathmoveto{\pgfpoint{324.863983pt}{232.825272pt}}
\pgflineto{\pgfpoint{333.791992pt}{226.648422pt}}
\pgflineto{\pgfpoint{324.863983pt}{226.648422pt}}
\pgfpathclose
\pgfusepath{fill,stroke}
\pgfpathmoveto{\pgfpoint{324.863983pt}{232.825272pt}}
\pgflineto{\pgfpoint{333.791992pt}{232.825272pt}}
\pgflineto{\pgfpoint{333.791992pt}{226.648422pt}}
\pgfpathclose
\pgfusepath{fill,stroke}
\color[rgb]{0.664087,0.861321,0.198879}
\pgfpathmoveto{\pgfpoint{324.863983pt}{239.002106pt}}
\pgflineto{\pgfpoint{333.791992pt}{232.825272pt}}
\pgflineto{\pgfpoint{324.863983pt}{232.825272pt}}
\pgfpathclose
\pgfusepath{fill,stroke}
\pgfpathmoveto{\pgfpoint{324.863983pt}{239.002106pt}}
\pgflineto{\pgfpoint{333.791992pt}{239.002106pt}}
\pgflineto{\pgfpoint{333.791992pt}{232.825272pt}}
\pgfpathclose
\pgfusepath{fill,stroke}
\color[rgb]{0.706404,0.868206,0.171495}
\pgfpathmoveto{\pgfpoint{324.863983pt}{245.178955pt}}
\pgflineto{\pgfpoint{333.791992pt}{239.002106pt}}
\pgflineto{\pgfpoint{324.863983pt}{239.002106pt}}
\pgfpathclose
\pgfusepath{fill,stroke}
\pgfpathmoveto{\pgfpoint{324.863983pt}{245.178955pt}}
\pgflineto{\pgfpoint{333.791992pt}{245.178955pt}}
\pgflineto{\pgfpoint{333.791992pt}{239.002106pt}}
\pgfpathclose
\pgfusepath{fill,stroke}
\color[rgb]{0.462247,0.817338,0.327545}
\pgfpathmoveto{\pgfpoint{333.791992pt}{195.764206pt}}
\pgflineto{\pgfpoint{342.719971pt}{195.764206pt}}
\pgflineto{\pgfpoint{342.719971pt}{189.587372pt}}
\pgfpathclose
\pgfusepath{fill,stroke}
\color[rgb]{0.500754,0.827409,0.303799}
\pgfpathmoveto{\pgfpoint{333.791992pt}{201.941055pt}}
\pgflineto{\pgfpoint{342.719971pt}{195.764206pt}}
\pgflineto{\pgfpoint{333.791992pt}{195.764206pt}}
\pgfpathclose
\pgfusepath{fill,stroke}
\pgfpathmoveto{\pgfpoint{333.791992pt}{201.941055pt}}
\pgflineto{\pgfpoint{342.719971pt}{201.941055pt}}
\pgflineto{\pgfpoint{342.719971pt}{195.764206pt}}
\pgfpathclose
\pgfusepath{fill,stroke}
\color[rgb]{0.540337,0.836858,0.278917}
\pgfpathmoveto{\pgfpoint{333.791992pt}{208.117905pt}}
\pgflineto{\pgfpoint{342.719971pt}{201.941055pt}}
\pgflineto{\pgfpoint{333.791992pt}{201.941055pt}}
\pgfpathclose
\pgfusepath{fill,stroke}
\pgfpathmoveto{\pgfpoint{333.791992pt}{208.117905pt}}
\pgflineto{\pgfpoint{342.719971pt}{208.117905pt}}
\pgflineto{\pgfpoint{342.719971pt}{201.941055pt}}
\pgfpathclose
\pgfusepath{fill,stroke}
\color[rgb]{0.580861,0.845663,0.253001}
\pgfpathmoveto{\pgfpoint{333.791992pt}{214.294739pt}}
\pgflineto{\pgfpoint{342.719971pt}{208.117905pt}}
\pgflineto{\pgfpoint{333.791992pt}{208.117905pt}}
\pgfpathclose
\pgfusepath{fill,stroke}
\pgfpathmoveto{\pgfpoint{333.791992pt}{214.294739pt}}
\pgflineto{\pgfpoint{342.719971pt}{214.294739pt}}
\pgflineto{\pgfpoint{342.719971pt}{208.117905pt}}
\pgfpathclose
\pgfusepath{fill,stroke}
\color[rgb]{0.622171,0.853816,0.226224}
\pgfpathmoveto{\pgfpoint{333.791992pt}{220.471588pt}}
\pgflineto{\pgfpoint{342.719971pt}{214.294739pt}}
\pgflineto{\pgfpoint{333.791992pt}{214.294739pt}}
\pgfpathclose
\pgfusepath{fill,stroke}
\pgfpathmoveto{\pgfpoint{333.791992pt}{220.471588pt}}
\pgflineto{\pgfpoint{342.719971pt}{220.471588pt}}
\pgflineto{\pgfpoint{342.719971pt}{214.294739pt}}
\pgfpathclose
\pgfusepath{fill,stroke}
\pgfpathmoveto{\pgfpoint{333.791992pt}{226.648422pt}}
\pgflineto{\pgfpoint{342.719971pt}{220.471588pt}}
\pgflineto{\pgfpoint{333.791992pt}{220.471588pt}}
\pgfpathclose
\pgfusepath{fill,stroke}
\pgfpathmoveto{\pgfpoint{333.791992pt}{226.648422pt}}
\pgflineto{\pgfpoint{342.719971pt}{226.648422pt}}
\pgflineto{\pgfpoint{342.719971pt}{220.471588pt}}
\pgfpathclose
\pgfusepath{fill,stroke}
\color[rgb]{0.664087,0.861321,0.198879}
\pgfpathmoveto{\pgfpoint{333.791992pt}{232.825272pt}}
\pgflineto{\pgfpoint{342.719971pt}{226.648422pt}}
\pgflineto{\pgfpoint{333.791992pt}{226.648422pt}}
\pgfpathclose
\pgfusepath{fill,stroke}
\pgfpathmoveto{\pgfpoint{333.791992pt}{232.825272pt}}
\pgflineto{\pgfpoint{342.719971pt}{232.825272pt}}
\pgflineto{\pgfpoint{342.719971pt}{226.648422pt}}
\pgfpathclose
\pgfusepath{fill,stroke}
\color[rgb]{0.706404,0.868206,0.171495}
\pgfpathmoveto{\pgfpoint{333.791992pt}{239.002106pt}}
\pgflineto{\pgfpoint{342.719971pt}{232.825272pt}}
\pgflineto{\pgfpoint{333.791992pt}{232.825272pt}}
\pgfpathclose
\pgfusepath{fill,stroke}
\pgfpathmoveto{\pgfpoint{333.791992pt}{239.002106pt}}
\pgflineto{\pgfpoint{342.719971pt}{239.002106pt}}
\pgflineto{\pgfpoint{342.719971pt}{232.825272pt}}
\pgfpathclose
\pgfusepath{fill,stroke}
\color[rgb]{0.748885,0.874522,0.145038}
\pgfpathmoveto{\pgfpoint{333.791992pt}{245.178955pt}}
\pgflineto{\pgfpoint{342.719971pt}{239.002106pt}}
\pgflineto{\pgfpoint{333.791992pt}{239.002106pt}}
\pgfpathclose
\pgfusepath{fill,stroke}
\pgfpathmoveto{\pgfpoint{333.791992pt}{245.178955pt}}
\pgflineto{\pgfpoint{342.719971pt}{245.178955pt}}
\pgflineto{\pgfpoint{342.719971pt}{239.002106pt}}
\pgfpathclose
\pgfusepath{fill,stroke}
\color[rgb]{0.500754,0.827409,0.303799}
\pgfpathmoveto{\pgfpoint{342.719971pt}{195.764206pt}}
\pgflineto{\pgfpoint{351.647980pt}{189.587372pt}}
\pgflineto{\pgfpoint{342.719971pt}{189.587372pt}}
\pgfpathclose
\pgfusepath{fill,stroke}
\pgfpathmoveto{\pgfpoint{342.719971pt}{195.764206pt}}
\pgflineto{\pgfpoint{351.647980pt}{195.764206pt}}
\pgflineto{\pgfpoint{351.647980pt}{189.587372pt}}
\pgfpathclose
\pgfusepath{fill,stroke}
\color[rgb]{0.540337,0.836858,0.278917}
\pgfpathmoveto{\pgfpoint{342.719971pt}{201.941055pt}}
\pgflineto{\pgfpoint{351.647980pt}{195.764206pt}}
\pgflineto{\pgfpoint{342.719971pt}{195.764206pt}}
\pgfpathclose
\pgfusepath{fill,stroke}
\pgfpathmoveto{\pgfpoint{342.719971pt}{201.941055pt}}
\pgflineto{\pgfpoint{351.647980pt}{201.941055pt}}
\pgflineto{\pgfpoint{351.647980pt}{195.764206pt}}
\pgfpathclose
\pgfusepath{fill,stroke}
\color[rgb]{0.580861,0.845663,0.253001}
\pgfpathmoveto{\pgfpoint{342.719971pt}{208.117905pt}}
\pgflineto{\pgfpoint{351.647980pt}{201.941055pt}}
\pgflineto{\pgfpoint{342.719971pt}{201.941055pt}}
\pgfpathclose
\pgfusepath{fill,stroke}
\pgfpathmoveto{\pgfpoint{342.719971pt}{208.117905pt}}
\pgflineto{\pgfpoint{351.647980pt}{208.117905pt}}
\pgflineto{\pgfpoint{351.647980pt}{201.941055pt}}
\pgfpathclose
\pgfusepath{fill,stroke}
\color[rgb]{0.622171,0.853816,0.226224}
\pgfpathmoveto{\pgfpoint{342.719971pt}{214.294739pt}}
\pgflineto{\pgfpoint{351.647980pt}{208.117905pt}}
\pgflineto{\pgfpoint{342.719971pt}{208.117905pt}}
\pgfpathclose
\pgfusepath{fill,stroke}
\pgfpathmoveto{\pgfpoint{342.719971pt}{214.294739pt}}
\pgflineto{\pgfpoint{351.647980pt}{214.294739pt}}
\pgflineto{\pgfpoint{351.647980pt}{208.117905pt}}
\pgfpathclose
\pgfusepath{fill,stroke}
\color[rgb]{0.664087,0.861321,0.198879}
\pgfpathmoveto{\pgfpoint{342.719971pt}{220.471588pt}}
\pgflineto{\pgfpoint{351.647980pt}{214.294739pt}}
\pgflineto{\pgfpoint{342.719971pt}{214.294739pt}}
\pgfpathclose
\pgfusepath{fill,stroke}
\pgfpathmoveto{\pgfpoint{342.719971pt}{220.471588pt}}
\pgflineto{\pgfpoint{351.647980pt}{220.471588pt}}
\pgflineto{\pgfpoint{351.647980pt}{214.294739pt}}
\pgfpathclose
\pgfusepath{fill,stroke}
\pgfpathmoveto{\pgfpoint{342.719971pt}{226.648422pt}}
\pgflineto{\pgfpoint{351.647980pt}{220.471588pt}}
\pgflineto{\pgfpoint{342.719971pt}{220.471588pt}}
\pgfpathclose
\pgfusepath{fill,stroke}
\pgfpathmoveto{\pgfpoint{342.719971pt}{226.648422pt}}
\pgflineto{\pgfpoint{351.647980pt}{226.648422pt}}
\pgflineto{\pgfpoint{351.647980pt}{220.471588pt}}
\pgfpathclose
\pgfusepath{fill,stroke}
\color[rgb]{0.706404,0.868206,0.171495}
\pgfpathmoveto{\pgfpoint{342.719971pt}{232.825272pt}}
\pgflineto{\pgfpoint{351.647980pt}{226.648422pt}}
\pgflineto{\pgfpoint{342.719971pt}{226.648422pt}}
\pgfpathclose
\pgfusepath{fill,stroke}
\pgfpathmoveto{\pgfpoint{342.719971pt}{232.825272pt}}
\pgflineto{\pgfpoint{351.647980pt}{232.825272pt}}
\pgflineto{\pgfpoint{351.647980pt}{226.648422pt}}
\pgfpathclose
\pgfusepath{fill,stroke}
\color[rgb]{0.748885,0.874522,0.145038}
\pgfpathmoveto{\pgfpoint{342.719971pt}{239.002106pt}}
\pgflineto{\pgfpoint{351.647980pt}{232.825272pt}}
\pgflineto{\pgfpoint{342.719971pt}{232.825272pt}}
\pgfpathclose
\pgfusepath{fill,stroke}
\pgfpathmoveto{\pgfpoint{342.719971pt}{239.002106pt}}
\pgflineto{\pgfpoint{351.647980pt}{239.002106pt}}
\pgflineto{\pgfpoint{351.647980pt}{232.825272pt}}
\pgfpathclose
\pgfusepath{fill,stroke}
\color[rgb]{0.791273,0.880346,0.121291}
\pgfpathmoveto{\pgfpoint{342.719971pt}{245.178955pt}}
\pgflineto{\pgfpoint{351.647980pt}{239.002106pt}}
\pgflineto{\pgfpoint{342.719971pt}{239.002106pt}}
\pgfpathclose
\pgfusepath{fill,stroke}
\pgfpathmoveto{\pgfpoint{342.719971pt}{245.178955pt}}
\pgflineto{\pgfpoint{351.647980pt}{245.178955pt}}
\pgflineto{\pgfpoint{351.647980pt}{239.002106pt}}
\pgfpathclose
\pgfusepath{fill,stroke}
\color[rgb]{0.500754,0.827409,0.303799}
\pgfpathmoveto{\pgfpoint{351.647980pt}{189.587372pt}}
\pgflineto{\pgfpoint{360.575958pt}{189.587372pt}}
\pgflineto{\pgfpoint{360.575958pt}{183.410522pt}}
\pgfpathclose
\pgfusepath{fill,stroke}
\color[rgb]{0.540337,0.836858,0.278917}
\pgfpathmoveto{\pgfpoint{351.647980pt}{195.764206pt}}
\pgflineto{\pgfpoint{360.575958pt}{189.587372pt}}
\pgflineto{\pgfpoint{351.647980pt}{189.587372pt}}
\pgfpathclose
\pgfusepath{fill,stroke}
\pgfpathmoveto{\pgfpoint{351.647980pt}{195.764206pt}}
\pgflineto{\pgfpoint{360.575958pt}{195.764206pt}}
\pgflineto{\pgfpoint{360.575958pt}{189.587372pt}}
\pgfpathclose
\pgfusepath{fill,stroke}
\color[rgb]{0.580861,0.845663,0.253001}
\pgfpathmoveto{\pgfpoint{351.647980pt}{201.941055pt}}
\pgflineto{\pgfpoint{360.575958pt}{195.764206pt}}
\pgflineto{\pgfpoint{351.647980pt}{195.764206pt}}
\pgfpathclose
\pgfusepath{fill,stroke}
\pgfpathmoveto{\pgfpoint{351.647980pt}{201.941055pt}}
\pgflineto{\pgfpoint{360.575958pt}{201.941055pt}}
\pgflineto{\pgfpoint{360.575958pt}{195.764206pt}}
\pgfpathclose
\pgfusepath{fill,stroke}
\color[rgb]{0.622171,0.853816,0.226224}
\pgfpathmoveto{\pgfpoint{351.647980pt}{208.117905pt}}
\pgflineto{\pgfpoint{360.575958pt}{201.941055pt}}
\pgflineto{\pgfpoint{351.647980pt}{201.941055pt}}
\pgfpathclose
\pgfusepath{fill,stroke}
\pgfpathmoveto{\pgfpoint{351.647980pt}{208.117905pt}}
\pgflineto{\pgfpoint{360.575958pt}{208.117905pt}}
\pgflineto{\pgfpoint{360.575958pt}{201.941055pt}}
\pgfpathclose
\pgfusepath{fill,stroke}
\color[rgb]{0.664087,0.861321,0.198879}
\pgfpathmoveto{\pgfpoint{351.647980pt}{214.294739pt}}
\pgflineto{\pgfpoint{360.575958pt}{208.117905pt}}
\pgflineto{\pgfpoint{351.647980pt}{208.117905pt}}
\pgfpathclose
\pgfusepath{fill,stroke}
\pgfpathmoveto{\pgfpoint{351.647980pt}{214.294739pt}}
\pgflineto{\pgfpoint{360.575958pt}{214.294739pt}}
\pgflineto{\pgfpoint{360.575958pt}{208.117905pt}}
\pgfpathclose
\pgfusepath{fill,stroke}
\color[rgb]{0.706404,0.868206,0.171495}
\pgfpathmoveto{\pgfpoint{351.647980pt}{220.471588pt}}
\pgflineto{\pgfpoint{360.575958pt}{214.294739pt}}
\pgflineto{\pgfpoint{351.647980pt}{214.294739pt}}
\pgfpathclose
\pgfusepath{fill,stroke}
\pgfpathmoveto{\pgfpoint{351.647980pt}{220.471588pt}}
\pgflineto{\pgfpoint{360.575958pt}{220.471588pt}}
\pgflineto{\pgfpoint{360.575958pt}{214.294739pt}}
\pgfpathclose
\pgfusepath{fill,stroke}
\color[rgb]{0.748885,0.874522,0.145038}
\pgfpathmoveto{\pgfpoint{351.647980pt}{226.648422pt}}
\pgflineto{\pgfpoint{360.575958pt}{220.471588pt}}
\pgflineto{\pgfpoint{351.647980pt}{220.471588pt}}
\pgfpathclose
\pgfusepath{fill,stroke}
\pgfpathmoveto{\pgfpoint{351.647980pt}{226.648422pt}}
\pgflineto{\pgfpoint{360.575958pt}{226.648422pt}}
\pgflineto{\pgfpoint{360.575958pt}{220.471588pt}}
\pgfpathclose
\pgfusepath{fill,stroke}
\pgfpathmoveto{\pgfpoint{351.647980pt}{232.825272pt}}
\pgflineto{\pgfpoint{360.575958pt}{226.648422pt}}
\pgflineto{\pgfpoint{351.647980pt}{226.648422pt}}
\pgfpathclose
\pgfusepath{fill,stroke}
\pgfpathmoveto{\pgfpoint{351.647980pt}{232.825272pt}}
\pgflineto{\pgfpoint{360.575958pt}{232.825272pt}}
\pgflineto{\pgfpoint{360.575958pt}{226.648422pt}}
\pgfpathclose
\pgfusepath{fill,stroke}
\color[rgb]{0.791273,0.880346,0.121291}
\pgfpathmoveto{\pgfpoint{351.647980pt}{239.002106pt}}
\pgflineto{\pgfpoint{360.575958pt}{232.825272pt}}
\pgflineto{\pgfpoint{351.647980pt}{232.825272pt}}
\pgfpathclose
\pgfusepath{fill,stroke}
\pgfpathmoveto{\pgfpoint{351.647980pt}{239.002106pt}}
\pgflineto{\pgfpoint{360.575958pt}{239.002106pt}}
\pgflineto{\pgfpoint{360.575958pt}{232.825272pt}}
\pgfpathclose
\pgfusepath{fill,stroke}
\color[rgb]{0.833302,0.885780,0.103326}
\pgfpathmoveto{\pgfpoint{351.647980pt}{245.178955pt}}
\pgflineto{\pgfpoint{360.575958pt}{239.002106pt}}
\pgflineto{\pgfpoint{351.647980pt}{239.002106pt}}
\pgfpathclose
\pgfusepath{fill,stroke}
\pgfpathmoveto{\pgfpoint{351.647980pt}{245.178955pt}}
\pgflineto{\pgfpoint{360.575958pt}{245.178955pt}}
\pgflineto{\pgfpoint{360.575958pt}{239.002106pt}}
\pgfpathclose
\pgfusepath{fill,stroke}
\color[rgb]{0.540337,0.836858,0.278917}
\pgfpathmoveto{\pgfpoint{360.575958pt}{183.410522pt}}
\pgflineto{\pgfpoint{369.503998pt}{183.410522pt}}
\pgflineto{\pgfpoint{369.503998pt}{177.233673pt}}
\pgfpathclose
\pgfusepath{fill,stroke}
\color[rgb]{0.580861,0.845663,0.253001}
\pgfpathmoveto{\pgfpoint{360.575958pt}{189.587372pt}}
\pgflineto{\pgfpoint{369.503998pt}{183.410522pt}}
\pgflineto{\pgfpoint{360.575958pt}{183.410522pt}}
\pgfpathclose
\pgfusepath{fill,stroke}
\pgfpathmoveto{\pgfpoint{360.575958pt}{189.587372pt}}
\pgflineto{\pgfpoint{369.503998pt}{189.587372pt}}
\pgflineto{\pgfpoint{369.503998pt}{183.410522pt}}
\pgfpathclose
\pgfusepath{fill,stroke}
\pgfpathmoveto{\pgfpoint{360.575958pt}{195.764206pt}}
\pgflineto{\pgfpoint{369.503998pt}{189.587372pt}}
\pgflineto{\pgfpoint{360.575958pt}{189.587372pt}}
\pgfpathclose
\pgfusepath{fill,stroke}
\pgfpathmoveto{\pgfpoint{360.575958pt}{195.764206pt}}
\pgflineto{\pgfpoint{369.503998pt}{195.764206pt}}
\pgflineto{\pgfpoint{369.503998pt}{189.587372pt}}
\pgfpathclose
\pgfusepath{fill,stroke}
\color[rgb]{0.622171,0.853816,0.226224}
\pgfpathmoveto{\pgfpoint{360.575958pt}{201.941055pt}}
\pgflineto{\pgfpoint{369.503998pt}{195.764206pt}}
\pgflineto{\pgfpoint{360.575958pt}{195.764206pt}}
\pgfpathclose
\pgfusepath{fill,stroke}
\pgfpathmoveto{\pgfpoint{360.575958pt}{201.941055pt}}
\pgflineto{\pgfpoint{369.503998pt}{201.941055pt}}
\pgflineto{\pgfpoint{369.503998pt}{195.764206pt}}
\pgfpathclose
\pgfusepath{fill,stroke}
\color[rgb]{0.664087,0.861321,0.198879}
\pgfpathmoveto{\pgfpoint{360.575958pt}{208.117905pt}}
\pgflineto{\pgfpoint{369.503998pt}{201.941055pt}}
\pgflineto{\pgfpoint{360.575958pt}{201.941055pt}}
\pgfpathclose
\pgfusepath{fill,stroke}
\pgfpathmoveto{\pgfpoint{360.575958pt}{208.117905pt}}
\pgflineto{\pgfpoint{369.503998pt}{208.117905pt}}
\pgflineto{\pgfpoint{369.503998pt}{201.941055pt}}
\pgfpathclose
\pgfusepath{fill,stroke}
\color[rgb]{0.706404,0.868206,0.171495}
\pgfpathmoveto{\pgfpoint{360.575958pt}{214.294739pt}}
\pgflineto{\pgfpoint{369.503998pt}{208.117905pt}}
\pgflineto{\pgfpoint{360.575958pt}{208.117905pt}}
\pgfpathclose
\pgfusepath{fill,stroke}
\pgfpathmoveto{\pgfpoint{360.575958pt}{214.294739pt}}
\pgflineto{\pgfpoint{369.503998pt}{214.294739pt}}
\pgflineto{\pgfpoint{369.503998pt}{208.117905pt}}
\pgfpathclose
\pgfusepath{fill,stroke}
\color[rgb]{0.748885,0.874522,0.145038}
\pgfpathmoveto{\pgfpoint{360.575958pt}{220.471588pt}}
\pgflineto{\pgfpoint{369.503998pt}{214.294739pt}}
\pgflineto{\pgfpoint{360.575958pt}{214.294739pt}}
\pgfpathclose
\pgfusepath{fill,stroke}
\pgfpathmoveto{\pgfpoint{360.575958pt}{220.471588pt}}
\pgflineto{\pgfpoint{369.503998pt}{220.471588pt}}
\pgflineto{\pgfpoint{369.503998pt}{214.294739pt}}
\pgfpathclose
\pgfusepath{fill,stroke}
\color[rgb]{0.791273,0.880346,0.121291}
\pgfpathmoveto{\pgfpoint{360.575958pt}{226.648422pt}}
\pgflineto{\pgfpoint{369.503998pt}{220.471588pt}}
\pgflineto{\pgfpoint{360.575958pt}{220.471588pt}}
\pgfpathclose
\pgfusepath{fill,stroke}
\pgfpathmoveto{\pgfpoint{360.575958pt}{226.648422pt}}
\pgflineto{\pgfpoint{369.503998pt}{226.648422pt}}
\pgflineto{\pgfpoint{369.503998pt}{220.471588pt}}
\pgfpathclose
\pgfusepath{fill,stroke}
\pgfpathmoveto{\pgfpoint{360.575958pt}{232.825272pt}}
\pgflineto{\pgfpoint{369.503998pt}{226.648422pt}}
\pgflineto{\pgfpoint{360.575958pt}{226.648422pt}}
\pgfpathclose
\pgfusepath{fill,stroke}
\pgfpathmoveto{\pgfpoint{360.575958pt}{232.825272pt}}
\pgflineto{\pgfpoint{369.503998pt}{232.825272pt}}
\pgflineto{\pgfpoint{369.503998pt}{226.648422pt}}
\pgfpathclose
\pgfusepath{fill,stroke}
\color[rgb]{0.833302,0.885780,0.103326}
\pgfpathmoveto{\pgfpoint{360.575958pt}{239.002106pt}}
\pgflineto{\pgfpoint{369.503998pt}{232.825272pt}}
\pgflineto{\pgfpoint{360.575958pt}{232.825272pt}}
\pgfpathclose
\pgfusepath{fill,stroke}
\pgfpathmoveto{\pgfpoint{360.575958pt}{239.002106pt}}
\pgflineto{\pgfpoint{369.503998pt}{239.002106pt}}
\pgflineto{\pgfpoint{369.503998pt}{232.825272pt}}
\pgfpathclose
\pgfusepath{fill,stroke}
\color[rgb]{0.874718,0.890945,0.095351}
\pgfpathmoveto{\pgfpoint{360.575958pt}{245.178955pt}}
\pgflineto{\pgfpoint{369.503998pt}{239.002106pt}}
\pgflineto{\pgfpoint{360.575958pt}{239.002106pt}}
\pgfpathclose
\pgfusepath{fill,stroke}
\color[rgb]{0.580861,0.845663,0.253001}
\pgfpathmoveto{\pgfpoint{369.503998pt}{183.410522pt}}
\pgflineto{\pgfpoint{378.431976pt}{177.233673pt}}
\pgflineto{\pgfpoint{369.503998pt}{177.233673pt}}
\pgfpathclose
\pgfusepath{fill,stroke}
\pgfpathmoveto{\pgfpoint{369.503998pt}{183.410522pt}}
\pgflineto{\pgfpoint{378.431976pt}{183.410522pt}}
\pgflineto{\pgfpoint{378.431976pt}{177.233673pt}}
\pgfpathclose
\pgfusepath{fill,stroke}
\color[rgb]{0.622171,0.853816,0.226224}
\pgfpathmoveto{\pgfpoint{369.503998pt}{189.587372pt}}
\pgflineto{\pgfpoint{378.431976pt}{183.410522pt}}
\pgflineto{\pgfpoint{369.503998pt}{183.410522pt}}
\pgfpathclose
\pgfusepath{fill,stroke}
\pgfpathmoveto{\pgfpoint{369.503998pt}{189.587372pt}}
\pgflineto{\pgfpoint{378.431976pt}{189.587372pt}}
\pgflineto{\pgfpoint{378.431976pt}{183.410522pt}}
\pgfpathclose
\pgfusepath{fill,stroke}
\pgfpathmoveto{\pgfpoint{369.503998pt}{195.764206pt}}
\pgflineto{\pgfpoint{378.431976pt}{189.587372pt}}
\pgflineto{\pgfpoint{369.503998pt}{189.587372pt}}
\pgfpathclose
\pgfusepath{fill,stroke}
\pgfpathmoveto{\pgfpoint{369.503998pt}{195.764206pt}}
\pgflineto{\pgfpoint{378.431976pt}{195.764206pt}}
\pgflineto{\pgfpoint{378.431976pt}{189.587372pt}}
\pgfpathclose
\pgfusepath{fill,stroke}
\color[rgb]{0.664087,0.861321,0.198879}
\pgfpathmoveto{\pgfpoint{369.503998pt}{201.941055pt}}
\pgflineto{\pgfpoint{378.431976pt}{195.764206pt}}
\pgflineto{\pgfpoint{369.503998pt}{195.764206pt}}
\pgfpathclose
\pgfusepath{fill,stroke}
\pgfpathmoveto{\pgfpoint{369.503998pt}{201.941055pt}}
\pgflineto{\pgfpoint{378.431976pt}{201.941055pt}}
\pgflineto{\pgfpoint{378.431976pt}{195.764206pt}}
\pgfpathclose
\pgfusepath{fill,stroke}
\color[rgb]{0.706404,0.868206,0.171495}
\pgfpathmoveto{\pgfpoint{369.503998pt}{208.117905pt}}
\pgflineto{\pgfpoint{378.431976pt}{201.941055pt}}
\pgflineto{\pgfpoint{369.503998pt}{201.941055pt}}
\pgfpathclose
\pgfusepath{fill,stroke}
\pgfpathmoveto{\pgfpoint{369.503998pt}{208.117905pt}}
\pgflineto{\pgfpoint{378.431976pt}{208.117905pt}}
\pgflineto{\pgfpoint{378.431976pt}{201.941055pt}}
\pgfpathclose
\pgfusepath{fill,stroke}
\color[rgb]{0.748885,0.874522,0.145038}
\pgfpathmoveto{\pgfpoint{369.503998pt}{214.294739pt}}
\pgflineto{\pgfpoint{378.431976pt}{208.117905pt}}
\pgflineto{\pgfpoint{369.503998pt}{208.117905pt}}
\pgfpathclose
\pgfusepath{fill,stroke}
\pgfpathmoveto{\pgfpoint{369.503998pt}{214.294739pt}}
\pgflineto{\pgfpoint{378.431976pt}{214.294739pt}}
\pgflineto{\pgfpoint{378.431976pt}{208.117905pt}}
\pgfpathclose
\pgfusepath{fill,stroke}
\color[rgb]{0.791273,0.880346,0.121291}
\pgfpathmoveto{\pgfpoint{369.503998pt}{220.471588pt}}
\pgflineto{\pgfpoint{378.431976pt}{214.294739pt}}
\pgflineto{\pgfpoint{369.503998pt}{214.294739pt}}
\pgfpathclose
\pgfusepath{fill,stroke}
\pgfpathmoveto{\pgfpoint{369.503998pt}{220.471588pt}}
\pgflineto{\pgfpoint{378.431976pt}{220.471588pt}}
\pgflineto{\pgfpoint{378.431976pt}{214.294739pt}}
\pgfpathclose
\pgfusepath{fill,stroke}
\color[rgb]{0.833302,0.885780,0.103326}
\pgfpathmoveto{\pgfpoint{369.503998pt}{226.648422pt}}
\pgflineto{\pgfpoint{378.431976pt}{220.471588pt}}
\pgflineto{\pgfpoint{369.503998pt}{220.471588pt}}
\pgfpathclose
\pgfusepath{fill,stroke}
\pgfpathmoveto{\pgfpoint{369.503998pt}{226.648422pt}}
\pgflineto{\pgfpoint{378.431976pt}{226.648422pt}}
\pgflineto{\pgfpoint{378.431976pt}{220.471588pt}}
\pgfpathclose
\pgfusepath{fill,stroke}
\color[rgb]{0.874718,0.890945,0.095351}
\pgfpathmoveto{\pgfpoint{369.503998pt}{232.825272pt}}
\pgflineto{\pgfpoint{378.431976pt}{226.648422pt}}
\pgflineto{\pgfpoint{369.503998pt}{226.648422pt}}
\pgfpathclose
\pgfusepath{fill,stroke}
\pgfpathmoveto{\pgfpoint{369.503998pt}{232.825272pt}}
\pgflineto{\pgfpoint{378.431976pt}{232.825272pt}}
\pgflineto{\pgfpoint{378.431976pt}{226.648422pt}}
\pgfpathclose
\pgfusepath{fill,stroke}
\pgfpathmoveto{\pgfpoint{369.503998pt}{239.002106pt}}
\pgflineto{\pgfpoint{378.431976pt}{232.825272pt}}
\pgflineto{\pgfpoint{369.503998pt}{232.825272pt}}
\pgfpathclose
\pgfusepath{fill,stroke}
\pgfpathmoveto{\pgfpoint{369.503998pt}{239.002106pt}}
\pgflineto{\pgfpoint{378.431976pt}{239.002106pt}}
\pgflineto{\pgfpoint{378.431976pt}{232.825272pt}}
\pgfpathclose
\pgfusepath{fill,stroke}
\color[rgb]{0.580861,0.845663,0.253001}
\pgfpathmoveto{\pgfpoint{378.431976pt}{177.233673pt}}
\pgflineto{\pgfpoint{387.359985pt}{177.233673pt}}
\pgflineto{\pgfpoint{387.359985pt}{171.056854pt}}
\pgfpathclose
\pgfusepath{fill,stroke}
\color[rgb]{0.622171,0.853816,0.226224}
\pgfpathmoveto{\pgfpoint{378.431976pt}{183.410522pt}}
\pgflineto{\pgfpoint{387.359985pt}{177.233673pt}}
\pgflineto{\pgfpoint{378.431976pt}{177.233673pt}}
\pgfpathclose
\pgfusepath{fill,stroke}
\pgfpathmoveto{\pgfpoint{378.431976pt}{183.410522pt}}
\pgflineto{\pgfpoint{387.359985pt}{183.410522pt}}
\pgflineto{\pgfpoint{387.359985pt}{177.233673pt}}
\pgfpathclose
\pgfusepath{fill,stroke}
\color[rgb]{0.664087,0.861321,0.198879}
\pgfpathmoveto{\pgfpoint{378.431976pt}{189.587372pt}}
\pgflineto{\pgfpoint{387.359985pt}{183.410522pt}}
\pgflineto{\pgfpoint{378.431976pt}{183.410522pt}}
\pgfpathclose
\pgfusepath{fill,stroke}
\pgfpathmoveto{\pgfpoint{378.431976pt}{189.587372pt}}
\pgflineto{\pgfpoint{387.359985pt}{189.587372pt}}
\pgflineto{\pgfpoint{387.359985pt}{183.410522pt}}
\pgfpathclose
\pgfusepath{fill,stroke}
\color[rgb]{0.706404,0.868206,0.171495}
\pgfpathmoveto{\pgfpoint{378.431976pt}{195.764206pt}}
\pgflineto{\pgfpoint{387.359985pt}{189.587372pt}}
\pgflineto{\pgfpoint{378.431976pt}{189.587372pt}}
\pgfpathclose
\pgfusepath{fill,stroke}
\pgfpathmoveto{\pgfpoint{378.431976pt}{195.764206pt}}
\pgflineto{\pgfpoint{387.359985pt}{195.764206pt}}
\pgflineto{\pgfpoint{387.359985pt}{189.587372pt}}
\pgfpathclose
\pgfusepath{fill,stroke}
\pgfpathmoveto{\pgfpoint{378.431976pt}{201.941055pt}}
\pgflineto{\pgfpoint{387.359985pt}{195.764206pt}}
\pgflineto{\pgfpoint{378.431976pt}{195.764206pt}}
\pgfpathclose
\pgfusepath{fill,stroke}
\pgfpathmoveto{\pgfpoint{378.431976pt}{201.941055pt}}
\pgflineto{\pgfpoint{387.359985pt}{201.941055pt}}
\pgflineto{\pgfpoint{387.359985pt}{195.764206pt}}
\pgfpathclose
\pgfusepath{fill,stroke}
\color[rgb]{0.748885,0.874522,0.145038}
\pgfpathmoveto{\pgfpoint{378.431976pt}{208.117905pt}}
\pgflineto{\pgfpoint{387.359985pt}{201.941055pt}}
\pgflineto{\pgfpoint{378.431976pt}{201.941055pt}}
\pgfpathclose
\pgfusepath{fill,stroke}
\pgfpathmoveto{\pgfpoint{378.431976pt}{208.117905pt}}
\pgflineto{\pgfpoint{387.359985pt}{208.117905pt}}
\pgflineto{\pgfpoint{387.359985pt}{201.941055pt}}
\pgfpathclose
\pgfusepath{fill,stroke}
\color[rgb]{0.791273,0.880346,0.121291}
\pgfpathmoveto{\pgfpoint{378.431976pt}{214.294739pt}}
\pgflineto{\pgfpoint{387.359985pt}{208.117905pt}}
\pgflineto{\pgfpoint{378.431976pt}{208.117905pt}}
\pgfpathclose
\pgfusepath{fill,stroke}
\pgfpathmoveto{\pgfpoint{378.431976pt}{214.294739pt}}
\pgflineto{\pgfpoint{387.359985pt}{214.294739pt}}
\pgflineto{\pgfpoint{387.359985pt}{208.117905pt}}
\pgfpathclose
\pgfusepath{fill,stroke}
\color[rgb]{0.833302,0.885780,0.103326}
\pgfpathmoveto{\pgfpoint{378.431976pt}{220.471588pt}}
\pgflineto{\pgfpoint{387.359985pt}{214.294739pt}}
\pgflineto{\pgfpoint{378.431976pt}{214.294739pt}}
\pgfpathclose
\pgfusepath{fill,stroke}
\pgfpathmoveto{\pgfpoint{378.431976pt}{220.471588pt}}
\pgflineto{\pgfpoint{387.359985pt}{220.471588pt}}
\pgflineto{\pgfpoint{387.359985pt}{214.294739pt}}
\pgfpathclose
\pgfusepath{fill,stroke}
\color[rgb]{0.874718,0.890945,0.095351}
\pgfpathmoveto{\pgfpoint{378.431976pt}{226.648422pt}}
\pgflineto{\pgfpoint{387.359985pt}{220.471588pt}}
\pgflineto{\pgfpoint{378.431976pt}{220.471588pt}}
\pgfpathclose
\pgfusepath{fill,stroke}
\pgfpathmoveto{\pgfpoint{378.431976pt}{226.648422pt}}
\pgflineto{\pgfpoint{387.359985pt}{226.648422pt}}
\pgflineto{\pgfpoint{387.359985pt}{220.471588pt}}
\pgfpathclose
\pgfusepath{fill,stroke}
\color[rgb]{0.622171,0.853816,0.226224}
\pgfpathmoveto{\pgfpoint{387.359985pt}{177.233673pt}}
\pgflineto{\pgfpoint{396.287964pt}{171.056854pt}}
\pgflineto{\pgfpoint{387.359985pt}{171.056854pt}}
\pgfpathclose
\pgfusepath{fill,stroke}
\pgfpathmoveto{\pgfpoint{387.359985pt}{177.233673pt}}
\pgflineto{\pgfpoint{396.287964pt}{177.233673pt}}
\pgflineto{\pgfpoint{396.287964pt}{171.056854pt}}
\pgfpathclose
\pgfusepath{fill,stroke}
\color[rgb]{0.664087,0.861321,0.198879}
\pgfpathmoveto{\pgfpoint{387.359985pt}{183.410522pt}}
\pgflineto{\pgfpoint{396.287964pt}{177.233673pt}}
\pgflineto{\pgfpoint{387.359985pt}{177.233673pt}}
\pgfpathclose
\pgfusepath{fill,stroke}
\pgfpathmoveto{\pgfpoint{387.359985pt}{183.410522pt}}
\pgflineto{\pgfpoint{396.287964pt}{183.410522pt}}
\pgflineto{\pgfpoint{396.287964pt}{177.233673pt}}
\pgfpathclose
\pgfusepath{fill,stroke}
\color[rgb]{0.706404,0.868206,0.171495}
\pgfpathmoveto{\pgfpoint{387.359985pt}{189.587372pt}}
\pgflineto{\pgfpoint{396.287964pt}{183.410522pt}}
\pgflineto{\pgfpoint{387.359985pt}{183.410522pt}}
\pgfpathclose
\pgfusepath{fill,stroke}
\pgfpathmoveto{\pgfpoint{387.359985pt}{189.587372pt}}
\pgflineto{\pgfpoint{396.287964pt}{189.587372pt}}
\pgflineto{\pgfpoint{396.287964pt}{183.410522pt}}
\pgfpathclose
\pgfusepath{fill,stroke}
\color[rgb]{0.748885,0.874522,0.145038}
\pgfpathmoveto{\pgfpoint{387.359985pt}{195.764206pt}}
\pgflineto{\pgfpoint{396.287964pt}{189.587372pt}}
\pgflineto{\pgfpoint{387.359985pt}{189.587372pt}}
\pgfpathclose
\pgfusepath{fill,stroke}
\pgfpathmoveto{\pgfpoint{387.359985pt}{195.764206pt}}
\pgflineto{\pgfpoint{396.287964pt}{195.764206pt}}
\pgflineto{\pgfpoint{396.287964pt}{189.587372pt}}
\pgfpathclose
\pgfusepath{fill,stroke}
\pgfpathmoveto{\pgfpoint{387.359985pt}{201.941055pt}}
\pgflineto{\pgfpoint{396.287964pt}{195.764206pt}}
\pgflineto{\pgfpoint{387.359985pt}{195.764206pt}}
\pgfpathclose
\pgfusepath{fill,stroke}
\pgfpathmoveto{\pgfpoint{387.359985pt}{201.941055pt}}
\pgflineto{\pgfpoint{396.287964pt}{201.941055pt}}
\pgflineto{\pgfpoint{396.287964pt}{195.764206pt}}
\pgfpathclose
\pgfusepath{fill,stroke}
\color[rgb]{0.791273,0.880346,0.121291}
\pgfpathmoveto{\pgfpoint{387.359985pt}{208.117905pt}}
\pgflineto{\pgfpoint{396.287964pt}{201.941055pt}}
\pgflineto{\pgfpoint{387.359985pt}{201.941055pt}}
\pgfpathclose
\pgfusepath{fill,stroke}
\pgfpathmoveto{\pgfpoint{387.359985pt}{208.117905pt}}
\pgflineto{\pgfpoint{396.287964pt}{208.117905pt}}
\pgflineto{\pgfpoint{396.287964pt}{201.941055pt}}
\pgfpathclose
\pgfusepath{fill,stroke}
\color[rgb]{0.833302,0.885780,0.103326}
\pgfpathmoveto{\pgfpoint{387.359985pt}{214.294739pt}}
\pgflineto{\pgfpoint{396.287964pt}{208.117905pt}}
\pgflineto{\pgfpoint{387.359985pt}{208.117905pt}}
\pgfpathclose
\pgfusepath{fill,stroke}
\pgfpathmoveto{\pgfpoint{387.359985pt}{214.294739pt}}
\pgflineto{\pgfpoint{396.287964pt}{214.294739pt}}
\pgflineto{\pgfpoint{396.287964pt}{208.117905pt}}
\pgfpathclose
\pgfusepath{fill,stroke}
\color[rgb]{0.622171,0.853816,0.226224}
\pgfpathmoveto{\pgfpoint{396.287964pt}{171.056854pt}}
\pgflineto{\pgfpoint{405.216003pt}{171.056854pt}}
\pgflineto{\pgfpoint{405.216003pt}{164.880005pt}}
\pgfpathclose
\pgfusepath{fill,stroke}
\color[rgb]{0.664087,0.861321,0.198879}
\pgfpathmoveto{\pgfpoint{396.287964pt}{177.233673pt}}
\pgflineto{\pgfpoint{405.216003pt}{171.056854pt}}
\pgflineto{\pgfpoint{396.287964pt}{171.056854pt}}
\pgfpathclose
\pgfusepath{fill,stroke}
\pgfpathmoveto{\pgfpoint{396.287964pt}{177.233673pt}}
\pgflineto{\pgfpoint{405.216003pt}{177.233673pt}}
\pgflineto{\pgfpoint{405.216003pt}{171.056854pt}}
\pgfpathclose
\pgfusepath{fill,stroke}
\color[rgb]{0.706404,0.868206,0.171495}
\pgfpathmoveto{\pgfpoint{396.287964pt}{183.410522pt}}
\pgflineto{\pgfpoint{405.216003pt}{177.233673pt}}
\pgflineto{\pgfpoint{396.287964pt}{177.233673pt}}
\pgfpathclose
\pgfusepath{fill,stroke}
\pgfpathmoveto{\pgfpoint{396.287964pt}{183.410522pt}}
\pgflineto{\pgfpoint{405.216003pt}{183.410522pt}}
\pgflineto{\pgfpoint{405.216003pt}{177.233673pt}}
\pgfpathclose
\pgfusepath{fill,stroke}
\color[rgb]{0.748885,0.874522,0.145038}
\pgfpathmoveto{\pgfpoint{396.287964pt}{189.587372pt}}
\pgflineto{\pgfpoint{405.216003pt}{183.410522pt}}
\pgflineto{\pgfpoint{396.287964pt}{183.410522pt}}
\pgfpathclose
\pgfusepath{fill,stroke}
\pgfpathmoveto{\pgfpoint{396.287964pt}{189.587372pt}}
\pgflineto{\pgfpoint{405.216003pt}{189.587372pt}}
\pgflineto{\pgfpoint{405.216003pt}{183.410522pt}}
\pgfpathclose
\pgfusepath{fill,stroke}
\color[rgb]{0.791273,0.880346,0.121291}
\pgfpathmoveto{\pgfpoint{396.287964pt}{195.764206pt}}
\pgflineto{\pgfpoint{405.216003pt}{189.587372pt}}
\pgflineto{\pgfpoint{396.287964pt}{189.587372pt}}
\pgfpathclose
\pgfusepath{fill,stroke}
\pgfpathmoveto{\pgfpoint{396.287964pt}{195.764206pt}}
\pgflineto{\pgfpoint{405.216003pt}{195.764206pt}}
\pgflineto{\pgfpoint{405.216003pt}{189.587372pt}}
\pgfpathclose
\pgfusepath{fill,stroke}
\color[rgb]{0.833302,0.885780,0.103326}
\pgfpathmoveto{\pgfpoint{396.287964pt}{201.941055pt}}
\pgflineto{\pgfpoint{405.216003pt}{195.764206pt}}
\pgflineto{\pgfpoint{396.287964pt}{195.764206pt}}
\pgfpathclose
\pgfusepath{fill,stroke}
\pgfpathmoveto{\pgfpoint{396.287964pt}{201.941055pt}}
\pgflineto{\pgfpoint{405.216003pt}{201.941055pt}}
\pgflineto{\pgfpoint{405.216003pt}{195.764206pt}}
\pgfpathclose
\pgfusepath{fill,stroke}
\color[rgb]{0.664087,0.861321,0.198879}
\pgfpathmoveto{\pgfpoint{405.216003pt}{171.056854pt}}
\pgflineto{\pgfpoint{414.143982pt}{164.880005pt}}
\pgflineto{\pgfpoint{405.216003pt}{164.880005pt}}
\pgfpathclose
\pgfusepath{fill,stroke}
\pgfpathmoveto{\pgfpoint{405.216003pt}{171.056854pt}}
\pgflineto{\pgfpoint{414.143982pt}{171.056854pt}}
\pgflineto{\pgfpoint{414.143982pt}{164.880005pt}}
\pgfpathclose
\pgfusepath{fill,stroke}
\color[rgb]{0.706404,0.868206,0.171495}
\pgfpathmoveto{\pgfpoint{405.216003pt}{177.233673pt}}
\pgflineto{\pgfpoint{414.143982pt}{171.056854pt}}
\pgflineto{\pgfpoint{405.216003pt}{171.056854pt}}
\pgfpathclose
\pgfusepath{fill,stroke}
\pgfpathmoveto{\pgfpoint{405.216003pt}{177.233673pt}}
\pgflineto{\pgfpoint{414.143982pt}{177.233673pt}}
\pgflineto{\pgfpoint{414.143982pt}{171.056854pt}}
\pgfpathclose
\pgfusepath{fill,stroke}
\color[rgb]{0.748885,0.874522,0.145038}
\pgfpathmoveto{\pgfpoint{405.216003pt}{183.410522pt}}
\pgflineto{\pgfpoint{414.143982pt}{177.233673pt}}
\pgflineto{\pgfpoint{405.216003pt}{177.233673pt}}
\pgfpathclose
\pgfusepath{fill,stroke}
\pgfpathmoveto{\pgfpoint{405.216003pt}{183.410522pt}}
\pgflineto{\pgfpoint{414.143982pt}{183.410522pt}}
\pgflineto{\pgfpoint{414.143982pt}{177.233673pt}}
\pgfpathclose
\pgfusepath{fill,stroke}
\color[rgb]{0.791273,0.880346,0.121291}
\pgfpathmoveto{\pgfpoint{405.216003pt}{189.587372pt}}
\pgflineto{\pgfpoint{414.143982pt}{183.410522pt}}
\pgflineto{\pgfpoint{405.216003pt}{183.410522pt}}
\pgfpathclose
\pgfusepath{fill,stroke}
\pgfpathmoveto{\pgfpoint{405.216003pt}{189.587372pt}}
\pgflineto{\pgfpoint{414.143982pt}{189.587372pt}}
\pgflineto{\pgfpoint{414.143982pt}{183.410522pt}}
\pgfpathclose
\pgfusepath{fill,stroke}
\color[rgb]{0.748885,0.874522,0.145038}
\pgfpathmoveto{\pgfpoint{414.143982pt}{171.056854pt}}
\pgflineto{\pgfpoint{423.071960pt}{164.880005pt}}
\pgflineto{\pgfpoint{414.143982pt}{164.880005pt}}
\pgfpathclose
\pgfusepath{fill,stroke}
\pgfpathmoveto{\pgfpoint{414.143982pt}{171.056854pt}}
\pgflineto{\pgfpoint{423.071960pt}{171.056854pt}}
\pgflineto{\pgfpoint{423.071960pt}{164.880005pt}}
\pgfpathclose
\pgfusepath{fill,stroke}
\pgfpathmoveto{\pgfpoint{414.143982pt}{177.233673pt}}
\pgflineto{\pgfpoint{423.071960pt}{171.056854pt}}
\pgflineto{\pgfpoint{414.143982pt}{171.056854pt}}
\pgfpathclose
\pgfusepath{fill,stroke}
\pgfpathmoveto{\pgfpoint{414.143982pt}{177.233673pt}}
\pgflineto{\pgfpoint{423.071960pt}{177.233673pt}}
\pgflineto{\pgfpoint{423.071960pt}{171.056854pt}}
\pgfpathclose
\pgfusepath{fill,stroke}
\pgfpathmoveto{\pgfpoint{423.071960pt}{164.880005pt}}
\pgflineto{\pgfpoint{432.000000pt}{164.880005pt}}
\pgflineto{\pgfpoint{432.000000pt}{158.703156pt}}
\pgfpathclose
\pgfusepath{fill,stroke}
\color[rgb]{0.791273,0.880346,0.121291}
\pgfpathmoveto{\pgfpoint{423.071960pt}{171.056854pt}}
\pgflineto{\pgfpoint{432.000000pt}{164.880005pt}}
\pgflineto{\pgfpoint{423.071960pt}{164.880005pt}}
\pgfpathclose
\pgfusepath{fill,stroke}
\pgfpathmoveto{\pgfpoint{423.071960pt}{171.056854pt}}
\pgflineto{\pgfpoint{432.000000pt}{171.056854pt}}
\pgflineto{\pgfpoint{432.000000pt}{164.880005pt}}
\pgfpathclose
\pgfusepath{fill,stroke}
\color[rgb]{0.706404,0.868206,0.171495}
\pgfpathmoveto{\pgfpoint{414.143982pt}{164.880005pt}}
\pgflineto{\pgfpoint{423.071960pt}{164.880005pt}}
\pgflineto{\pgfpoint{423.071960pt}{158.703156pt}}
\pgfpathclose
\pgfusepath{fill,stroke}
\color[rgb]{0.748885,0.874522,0.145038}
\pgfpathmoveto{\pgfpoint{423.071960pt}{164.880005pt}}
\pgflineto{\pgfpoint{432.000000pt}{158.703156pt}}
\pgflineto{\pgfpoint{423.071960pt}{158.703156pt}}
\pgfpathclose
\pgfusepath{fill,stroke}
\pgfpathmoveto{\pgfpoint{432.000000pt}{158.703156pt}}
\pgflineto{\pgfpoint{440.927979pt}{158.703156pt}}
\pgflineto{\pgfpoint{440.927979pt}{152.526306pt}}
\pgfpathclose
\pgfusepath{fill,stroke}
\color[rgb]{0.290001,0.758846,0.427826}
\pgfpathmoveto{\pgfpoint{217.727982pt}{257.532623pt}}
\pgflineto{\pgfpoint{226.655975pt}{257.532623pt}}
\pgflineto{\pgfpoint{226.655975pt}{251.355804pt}}
\pgfpathclose
\pgfusepath{fill,stroke}
\pgfpathmoveto{\pgfpoint{217.727982pt}{263.709473pt}}
\pgflineto{\pgfpoint{226.655975pt}{257.532623pt}}
\pgflineto{\pgfpoint{217.727982pt}{257.532623pt}}
\pgfpathclose
\pgfusepath{fill,stroke}
\pgfpathmoveto{\pgfpoint{217.727982pt}{263.709473pt}}
\pgflineto{\pgfpoint{226.655975pt}{263.709473pt}}
\pgflineto{\pgfpoint{226.655975pt}{257.532623pt}}
\pgfpathclose
\pgfusepath{fill,stroke}
\color[rgb]{0.321330,0.771498,0.410293}
\pgfpathmoveto{\pgfpoint{217.727982pt}{269.886322pt}}
\pgflineto{\pgfpoint{226.655975pt}{263.709473pt}}
\pgflineto{\pgfpoint{217.727982pt}{263.709473pt}}
\pgfpathclose
\pgfusepath{fill,stroke}
\pgfpathmoveto{\pgfpoint{217.727982pt}{269.886322pt}}
\pgflineto{\pgfpoint{226.655975pt}{269.886322pt}}
\pgflineto{\pgfpoint{226.655975pt}{263.709473pt}}
\pgfpathclose
\pgfusepath{fill,stroke}
\color[rgb]{0.260531,0.745802,0.444096}
\pgfpathmoveto{\pgfpoint{226.655975pt}{245.178955pt}}
\pgflineto{\pgfpoint{235.583969pt}{245.178955pt}}
\pgflineto{\pgfpoint{235.583969pt}{239.002106pt}}
\pgfpathclose
\pgfusepath{fill,stroke}
\color[rgb]{0.290001,0.758846,0.427826}
\pgfpathmoveto{\pgfpoint{226.655975pt}{251.355804pt}}
\pgflineto{\pgfpoint{235.583969pt}{245.178955pt}}
\pgflineto{\pgfpoint{226.655975pt}{245.178955pt}}
\pgfpathclose
\pgfusepath{fill,stroke}
\pgfpathmoveto{\pgfpoint{226.655975pt}{251.355804pt}}
\pgflineto{\pgfpoint{235.583969pt}{251.355804pt}}
\pgflineto{\pgfpoint{235.583969pt}{245.178955pt}}
\pgfpathclose
\pgfusepath{fill,stroke}
\color[rgb]{0.321330,0.771498,0.410293}
\pgfpathmoveto{\pgfpoint{226.655975pt}{257.532623pt}}
\pgflineto{\pgfpoint{235.583969pt}{251.355804pt}}
\pgflineto{\pgfpoint{226.655975pt}{251.355804pt}}
\pgfpathclose
\pgfusepath{fill,stroke}
\pgfpathmoveto{\pgfpoint{226.655975pt}{257.532623pt}}
\pgflineto{\pgfpoint{235.583969pt}{257.532623pt}}
\pgflineto{\pgfpoint{235.583969pt}{251.355804pt}}
\pgfpathclose
\pgfusepath{fill,stroke}
\pgfpathmoveto{\pgfpoint{226.655975pt}{263.709473pt}}
\pgflineto{\pgfpoint{235.583969pt}{257.532623pt}}
\pgflineto{\pgfpoint{226.655975pt}{257.532623pt}}
\pgfpathclose
\pgfusepath{fill,stroke}
\pgfpathmoveto{\pgfpoint{226.655975pt}{263.709473pt}}
\pgflineto{\pgfpoint{235.583969pt}{263.709473pt}}
\pgflineto{\pgfpoint{235.583969pt}{257.532623pt}}
\pgfpathclose
\pgfusepath{fill,stroke}
\color[rgb]{0.354355,0.783714,0.391488}
\pgfpathmoveto{\pgfpoint{226.655975pt}{269.886322pt}}
\pgflineto{\pgfpoint{235.583969pt}{263.709473pt}}
\pgflineto{\pgfpoint{226.655975pt}{263.709473pt}}
\pgfpathclose
\pgfusepath{fill,stroke}
\pgfpathmoveto{\pgfpoint{226.655975pt}{269.886322pt}}
\pgflineto{\pgfpoint{235.583969pt}{269.886322pt}}
\pgflineto{\pgfpoint{235.583969pt}{263.709473pt}}
\pgfpathclose
\pgfusepath{fill,stroke}
\color[rgb]{0.388930,0.795453,0.371421}
\pgfpathmoveto{\pgfpoint{226.655975pt}{276.063141pt}}
\pgflineto{\pgfpoint{235.583969pt}{269.886322pt}}
\pgflineto{\pgfpoint{226.655975pt}{269.886322pt}}
\pgfpathclose
\pgfusepath{fill,stroke}
\pgfpathmoveto{\pgfpoint{226.655975pt}{276.063141pt}}
\pgflineto{\pgfpoint{235.583969pt}{276.063141pt}}
\pgflineto{\pgfpoint{235.583969pt}{269.886322pt}}
\pgfpathclose
\pgfusepath{fill,stroke}
\color[rgb]{0.290001,0.758846,0.427826}
\pgfpathmoveto{\pgfpoint{235.583969pt}{245.178955pt}}
\pgflineto{\pgfpoint{244.511993pt}{239.002106pt}}
\pgflineto{\pgfpoint{235.583969pt}{239.002106pt}}
\pgfpathclose
\pgfusepath{fill,stroke}
\pgfpathmoveto{\pgfpoint{235.583969pt}{245.178955pt}}
\pgflineto{\pgfpoint{244.511993pt}{245.178955pt}}
\pgflineto{\pgfpoint{244.511993pt}{239.002106pt}}
\pgfpathclose
\pgfusepath{fill,stroke}
\color[rgb]{0.321330,0.771498,0.410293}
\pgfpathmoveto{\pgfpoint{235.583969pt}{251.355804pt}}
\pgflineto{\pgfpoint{244.511993pt}{245.178955pt}}
\pgflineto{\pgfpoint{235.583969pt}{245.178955pt}}
\pgfpathclose
\pgfusepath{fill,stroke}
\pgfpathmoveto{\pgfpoint{235.583969pt}{251.355804pt}}
\pgflineto{\pgfpoint{244.511993pt}{251.355804pt}}
\pgflineto{\pgfpoint{244.511993pt}{245.178955pt}}
\pgfpathclose
\pgfusepath{fill,stroke}
\color[rgb]{0.354355,0.783714,0.391488}
\pgfpathmoveto{\pgfpoint{235.583969pt}{257.532623pt}}
\pgflineto{\pgfpoint{244.511993pt}{251.355804pt}}
\pgflineto{\pgfpoint{235.583969pt}{251.355804pt}}
\pgfpathclose
\pgfusepath{fill,stroke}
\pgfpathmoveto{\pgfpoint{235.583969pt}{257.532623pt}}
\pgflineto{\pgfpoint{244.511993pt}{257.532623pt}}
\pgflineto{\pgfpoint{244.511993pt}{251.355804pt}}
\pgfpathclose
\pgfusepath{fill,stroke}
\color[rgb]{0.388930,0.795453,0.371421}
\pgfpathmoveto{\pgfpoint{235.583969pt}{263.709473pt}}
\pgflineto{\pgfpoint{244.511993pt}{257.532623pt}}
\pgflineto{\pgfpoint{235.583969pt}{257.532623pt}}
\pgfpathclose
\pgfusepath{fill,stroke}
\pgfpathmoveto{\pgfpoint{235.583969pt}{263.709473pt}}
\pgflineto{\pgfpoint{244.511993pt}{263.709473pt}}
\pgflineto{\pgfpoint{244.511993pt}{257.532623pt}}
\pgfpathclose
\pgfusepath{fill,stroke}
\pgfpathmoveto{\pgfpoint{235.583969pt}{269.886322pt}}
\pgflineto{\pgfpoint{244.511993pt}{263.709473pt}}
\pgflineto{\pgfpoint{235.583969pt}{263.709473pt}}
\pgfpathclose
\pgfusepath{fill,stroke}
\pgfpathmoveto{\pgfpoint{235.583969pt}{269.886322pt}}
\pgflineto{\pgfpoint{244.511993pt}{269.886322pt}}
\pgflineto{\pgfpoint{244.511993pt}{263.709473pt}}
\pgfpathclose
\pgfusepath{fill,stroke}
\color[rgb]{0.424933,0.806674,0.350099}
\pgfpathmoveto{\pgfpoint{235.583969pt}{276.063141pt}}
\pgflineto{\pgfpoint{244.511993pt}{269.886322pt}}
\pgflineto{\pgfpoint{235.583969pt}{269.886322pt}}
\pgfpathclose
\pgfusepath{fill,stroke}
\pgfpathmoveto{\pgfpoint{235.583969pt}{276.063141pt}}
\pgflineto{\pgfpoint{244.511993pt}{276.063141pt}}
\pgflineto{\pgfpoint{244.511993pt}{269.886322pt}}
\pgfpathclose
\pgfusepath{fill,stroke}
\color[rgb]{0.290001,0.758846,0.427826}
\pgfpathmoveto{\pgfpoint{244.511993pt}{239.002106pt}}
\pgflineto{\pgfpoint{253.440002pt}{239.002106pt}}
\pgflineto{\pgfpoint{253.440002pt}{232.825272pt}}
\pgfpathclose
\pgfusepath{fill,stroke}
\color[rgb]{0.321330,0.771498,0.410293}
\pgfpathmoveto{\pgfpoint{244.511993pt}{245.178955pt}}
\pgflineto{\pgfpoint{253.440002pt}{239.002106pt}}
\pgflineto{\pgfpoint{244.511993pt}{239.002106pt}}
\pgfpathclose
\pgfusepath{fill,stroke}
\pgfpathmoveto{\pgfpoint{244.511993pt}{245.178955pt}}
\pgflineto{\pgfpoint{253.440002pt}{245.178955pt}}
\pgflineto{\pgfpoint{253.440002pt}{239.002106pt}}
\pgfpathclose
\pgfusepath{fill,stroke}
\color[rgb]{0.354355,0.783714,0.391488}
\pgfpathmoveto{\pgfpoint{244.511993pt}{251.355804pt}}
\pgflineto{\pgfpoint{253.440002pt}{245.178955pt}}
\pgflineto{\pgfpoint{244.511993pt}{245.178955pt}}
\pgfpathclose
\pgfusepath{fill,stroke}
\pgfpathmoveto{\pgfpoint{244.511993pt}{251.355804pt}}
\pgflineto{\pgfpoint{253.440002pt}{251.355804pt}}
\pgflineto{\pgfpoint{253.440002pt}{245.178955pt}}
\pgfpathclose
\pgfusepath{fill,stroke}
\color[rgb]{0.388930,0.795453,0.371421}
\pgfpathmoveto{\pgfpoint{244.511993pt}{257.532623pt}}
\pgflineto{\pgfpoint{253.440002pt}{251.355804pt}}
\pgflineto{\pgfpoint{244.511993pt}{251.355804pt}}
\pgfpathclose
\pgfusepath{fill,stroke}
\pgfpathmoveto{\pgfpoint{244.511993pt}{257.532623pt}}
\pgflineto{\pgfpoint{253.440002pt}{257.532623pt}}
\pgflineto{\pgfpoint{253.440002pt}{251.355804pt}}
\pgfpathclose
\pgfusepath{fill,stroke}
\color[rgb]{0.424933,0.806674,0.350099}
\pgfpathmoveto{\pgfpoint{244.511993pt}{263.709473pt}}
\pgflineto{\pgfpoint{253.440002pt}{257.532623pt}}
\pgflineto{\pgfpoint{244.511993pt}{257.532623pt}}
\pgfpathclose
\pgfusepath{fill,stroke}
\pgfpathmoveto{\pgfpoint{244.511993pt}{263.709473pt}}
\pgflineto{\pgfpoint{253.440002pt}{263.709473pt}}
\pgflineto{\pgfpoint{253.440002pt}{257.532623pt}}
\pgfpathclose
\pgfusepath{fill,stroke}
\pgfpathmoveto{\pgfpoint{244.511993pt}{269.886322pt}}
\pgflineto{\pgfpoint{253.440002pt}{263.709473pt}}
\pgflineto{\pgfpoint{244.511993pt}{263.709473pt}}
\pgfpathclose
\pgfusepath{fill,stroke}
\pgfpathmoveto{\pgfpoint{244.511993pt}{269.886322pt}}
\pgflineto{\pgfpoint{253.440002pt}{269.886322pt}}
\pgflineto{\pgfpoint{253.440002pt}{263.709473pt}}
\pgfpathclose
\pgfusepath{fill,stroke}
\color[rgb]{0.462247,0.817338,0.327545}
\pgfpathmoveto{\pgfpoint{244.511993pt}{276.063141pt}}
\pgflineto{\pgfpoint{253.440002pt}{269.886322pt}}
\pgflineto{\pgfpoint{244.511993pt}{269.886322pt}}
\pgfpathclose
\pgfusepath{fill,stroke}
\pgfpathmoveto{\pgfpoint{244.511993pt}{276.063141pt}}
\pgflineto{\pgfpoint{253.440002pt}{276.063141pt}}
\pgflineto{\pgfpoint{253.440002pt}{269.886322pt}}
\pgfpathclose
\pgfusepath{fill,stroke}
\color[rgb]{0.500754,0.827409,0.303799}
\pgfpathmoveto{\pgfpoint{244.511993pt}{282.239990pt}}
\pgflineto{\pgfpoint{253.440002pt}{276.063141pt}}
\pgflineto{\pgfpoint{244.511993pt}{276.063141pt}}
\pgfpathclose
\pgfusepath{fill,stroke}
\pgfpathmoveto{\pgfpoint{244.511993pt}{282.239990pt}}
\pgflineto{\pgfpoint{253.440002pt}{282.239990pt}}
\pgflineto{\pgfpoint{253.440002pt}{276.063141pt}}
\pgfpathclose
\pgfusepath{fill,stroke}
\color[rgb]{0.321330,0.771498,0.410293}
\pgfpathmoveto{\pgfpoint{253.440002pt}{239.002106pt}}
\pgflineto{\pgfpoint{262.367981pt}{232.825272pt}}
\pgflineto{\pgfpoint{253.440002pt}{232.825272pt}}
\pgfpathclose
\pgfusepath{fill,stroke}
\pgfpathmoveto{\pgfpoint{253.440002pt}{239.002106pt}}
\pgflineto{\pgfpoint{262.367981pt}{239.002106pt}}
\pgflineto{\pgfpoint{262.367981pt}{232.825272pt}}
\pgfpathclose
\pgfusepath{fill,stroke}
\color[rgb]{0.354355,0.783714,0.391488}
\pgfpathmoveto{\pgfpoint{253.440002pt}{245.178955pt}}
\pgflineto{\pgfpoint{262.367981pt}{239.002106pt}}
\pgflineto{\pgfpoint{253.440002pt}{239.002106pt}}
\pgfpathclose
\pgfusepath{fill,stroke}
\pgfpathmoveto{\pgfpoint{253.440002pt}{245.178955pt}}
\pgflineto{\pgfpoint{262.367981pt}{245.178955pt}}
\pgflineto{\pgfpoint{262.367981pt}{239.002106pt}}
\pgfpathclose
\pgfusepath{fill,stroke}
\color[rgb]{0.388930,0.795453,0.371421}
\pgfpathmoveto{\pgfpoint{253.440002pt}{251.355804pt}}
\pgflineto{\pgfpoint{262.367981pt}{245.178955pt}}
\pgflineto{\pgfpoint{253.440002pt}{245.178955pt}}
\pgfpathclose
\pgfusepath{fill,stroke}
\pgfpathmoveto{\pgfpoint{253.440002pt}{251.355804pt}}
\pgflineto{\pgfpoint{262.367981pt}{251.355804pt}}
\pgflineto{\pgfpoint{262.367981pt}{245.178955pt}}
\pgfpathclose
\pgfusepath{fill,stroke}
\color[rgb]{0.424933,0.806674,0.350099}
\pgfpathmoveto{\pgfpoint{253.440002pt}{257.532623pt}}
\pgflineto{\pgfpoint{262.367981pt}{251.355804pt}}
\pgflineto{\pgfpoint{253.440002pt}{251.355804pt}}
\pgfpathclose
\pgfusepath{fill,stroke}
\pgfpathmoveto{\pgfpoint{253.440002pt}{257.532623pt}}
\pgflineto{\pgfpoint{262.367981pt}{257.532623pt}}
\pgflineto{\pgfpoint{262.367981pt}{251.355804pt}}
\pgfpathclose
\pgfusepath{fill,stroke}
\color[rgb]{0.462247,0.817338,0.327545}
\pgfpathmoveto{\pgfpoint{253.440002pt}{263.709473pt}}
\pgflineto{\pgfpoint{262.367981pt}{257.532623pt}}
\pgflineto{\pgfpoint{253.440002pt}{257.532623pt}}
\pgfpathclose
\pgfusepath{fill,stroke}
\pgfpathmoveto{\pgfpoint{253.440002pt}{263.709473pt}}
\pgflineto{\pgfpoint{262.367981pt}{263.709473pt}}
\pgflineto{\pgfpoint{262.367981pt}{257.532623pt}}
\pgfpathclose
\pgfusepath{fill,stroke}
\color[rgb]{0.500754,0.827409,0.303799}
\pgfpathmoveto{\pgfpoint{253.440002pt}{269.886322pt}}
\pgflineto{\pgfpoint{262.367981pt}{263.709473pt}}
\pgflineto{\pgfpoint{253.440002pt}{263.709473pt}}
\pgfpathclose
\pgfusepath{fill,stroke}
\pgfpathmoveto{\pgfpoint{253.440002pt}{269.886322pt}}
\pgflineto{\pgfpoint{262.367981pt}{269.886322pt}}
\pgflineto{\pgfpoint{262.367981pt}{263.709473pt}}
\pgfpathclose
\pgfusepath{fill,stroke}
\pgfpathmoveto{\pgfpoint{253.440002pt}{276.063141pt}}
\pgflineto{\pgfpoint{262.367981pt}{269.886322pt}}
\pgflineto{\pgfpoint{253.440002pt}{269.886322pt}}
\pgfpathclose
\pgfusepath{fill,stroke}
\pgfpathmoveto{\pgfpoint{253.440002pt}{276.063141pt}}
\pgflineto{\pgfpoint{262.367981pt}{276.063141pt}}
\pgflineto{\pgfpoint{262.367981pt}{269.886322pt}}
\pgfpathclose
\pgfusepath{fill,stroke}
\color[rgb]{0.540337,0.836858,0.278917}
\pgfpathmoveto{\pgfpoint{253.440002pt}{282.239990pt}}
\pgflineto{\pgfpoint{262.367981pt}{276.063141pt}}
\pgflineto{\pgfpoint{253.440002pt}{276.063141pt}}
\pgfpathclose
\pgfusepath{fill,stroke}
\pgfpathmoveto{\pgfpoint{253.440002pt}{282.239990pt}}
\pgflineto{\pgfpoint{262.367981pt}{282.239990pt}}
\pgflineto{\pgfpoint{262.367981pt}{276.063141pt}}
\pgfpathclose
\pgfusepath{fill,stroke}
\color[rgb]{0.388930,0.795453,0.371421}
\pgfpathmoveto{\pgfpoint{262.367981pt}{245.178955pt}}
\pgflineto{\pgfpoint{271.295990pt}{239.002106pt}}
\pgflineto{\pgfpoint{262.367981pt}{239.002106pt}}
\pgfpathclose
\pgfusepath{fill,stroke}
\pgfpathmoveto{\pgfpoint{262.367981pt}{245.178955pt}}
\pgflineto{\pgfpoint{271.295990pt}{245.178955pt}}
\pgflineto{\pgfpoint{271.295990pt}{239.002106pt}}
\pgfpathclose
\pgfusepath{fill,stroke}
\color[rgb]{0.424933,0.806674,0.350099}
\pgfpathmoveto{\pgfpoint{262.367981pt}{251.355804pt}}
\pgflineto{\pgfpoint{271.295990pt}{245.178955pt}}
\pgflineto{\pgfpoint{262.367981pt}{245.178955pt}}
\pgfpathclose
\pgfusepath{fill,stroke}
\pgfpathmoveto{\pgfpoint{262.367981pt}{251.355804pt}}
\pgflineto{\pgfpoint{271.295990pt}{251.355804pt}}
\pgflineto{\pgfpoint{271.295990pt}{245.178955pt}}
\pgfpathclose
\pgfusepath{fill,stroke}
\color[rgb]{0.462247,0.817338,0.327545}
\pgfpathmoveto{\pgfpoint{262.367981pt}{257.532623pt}}
\pgflineto{\pgfpoint{271.295990pt}{251.355804pt}}
\pgflineto{\pgfpoint{262.367981pt}{251.355804pt}}
\pgfpathclose
\pgfusepath{fill,stroke}
\pgfpathmoveto{\pgfpoint{262.367981pt}{257.532623pt}}
\pgflineto{\pgfpoint{271.295990pt}{257.532623pt}}
\pgflineto{\pgfpoint{271.295990pt}{251.355804pt}}
\pgfpathclose
\pgfusepath{fill,stroke}
\color[rgb]{0.500754,0.827409,0.303799}
\pgfpathmoveto{\pgfpoint{262.367981pt}{263.709473pt}}
\pgflineto{\pgfpoint{271.295990pt}{257.532623pt}}
\pgflineto{\pgfpoint{262.367981pt}{257.532623pt}}
\pgfpathclose
\pgfusepath{fill,stroke}
\pgfpathmoveto{\pgfpoint{262.367981pt}{263.709473pt}}
\pgflineto{\pgfpoint{271.295990pt}{263.709473pt}}
\pgflineto{\pgfpoint{271.295990pt}{257.532623pt}}
\pgfpathclose
\pgfusepath{fill,stroke}
\color[rgb]{0.540337,0.836858,0.278917}
\pgfpathmoveto{\pgfpoint{262.367981pt}{269.886322pt}}
\pgflineto{\pgfpoint{271.295990pt}{263.709473pt}}
\pgflineto{\pgfpoint{262.367981pt}{263.709473pt}}
\pgfpathclose
\pgfusepath{fill,stroke}
\pgfpathmoveto{\pgfpoint{262.367981pt}{269.886322pt}}
\pgflineto{\pgfpoint{271.295990pt}{269.886322pt}}
\pgflineto{\pgfpoint{271.295990pt}{263.709473pt}}
\pgfpathclose
\pgfusepath{fill,stroke}
\pgfpathmoveto{\pgfpoint{262.367981pt}{276.063141pt}}
\pgflineto{\pgfpoint{271.295990pt}{269.886322pt}}
\pgflineto{\pgfpoint{262.367981pt}{269.886322pt}}
\pgfpathclose
\pgfusepath{fill,stroke}
\pgfpathmoveto{\pgfpoint{262.367981pt}{276.063141pt}}
\pgflineto{\pgfpoint{271.295990pt}{276.063141pt}}
\pgflineto{\pgfpoint{271.295990pt}{269.886322pt}}
\pgfpathclose
\pgfusepath{fill,stroke}
\color[rgb]{0.580861,0.845663,0.253001}
\pgfpathmoveto{\pgfpoint{262.367981pt}{282.239990pt}}
\pgflineto{\pgfpoint{271.295990pt}{276.063141pt}}
\pgflineto{\pgfpoint{262.367981pt}{276.063141pt}}
\pgfpathclose
\pgfusepath{fill,stroke}
\pgfpathmoveto{\pgfpoint{262.367981pt}{282.239990pt}}
\pgflineto{\pgfpoint{271.295990pt}{282.239990pt}}
\pgflineto{\pgfpoint{271.295990pt}{276.063141pt}}
\pgfpathclose
\pgfusepath{fill,stroke}
\color[rgb]{0.424933,0.806674,0.350099}
\pgfpathmoveto{\pgfpoint{271.295990pt}{245.178955pt}}
\pgflineto{\pgfpoint{280.223969pt}{239.002106pt}}
\pgflineto{\pgfpoint{271.295990pt}{239.002106pt}}
\pgfpathclose
\pgfusepath{fill,stroke}
\pgfpathmoveto{\pgfpoint{271.295990pt}{245.178955pt}}
\pgflineto{\pgfpoint{280.223969pt}{245.178955pt}}
\pgflineto{\pgfpoint{280.223969pt}{239.002106pt}}
\pgfpathclose
\pgfusepath{fill,stroke}
\color[rgb]{0.462247,0.817338,0.327545}
\pgfpathmoveto{\pgfpoint{271.295990pt}{251.355804pt}}
\pgflineto{\pgfpoint{280.223969pt}{245.178955pt}}
\pgflineto{\pgfpoint{271.295990pt}{245.178955pt}}
\pgfpathclose
\pgfusepath{fill,stroke}
\pgfpathmoveto{\pgfpoint{271.295990pt}{251.355804pt}}
\pgflineto{\pgfpoint{280.223969pt}{251.355804pt}}
\pgflineto{\pgfpoint{280.223969pt}{245.178955pt}}
\pgfpathclose
\pgfusepath{fill,stroke}
\color[rgb]{0.500754,0.827409,0.303799}
\pgfpathmoveto{\pgfpoint{271.295990pt}{257.532623pt}}
\pgflineto{\pgfpoint{280.223969pt}{251.355804pt}}
\pgflineto{\pgfpoint{271.295990pt}{251.355804pt}}
\pgfpathclose
\pgfusepath{fill,stroke}
\pgfpathmoveto{\pgfpoint{271.295990pt}{257.532623pt}}
\pgflineto{\pgfpoint{280.223969pt}{257.532623pt}}
\pgflineto{\pgfpoint{280.223969pt}{251.355804pt}}
\pgfpathclose
\pgfusepath{fill,stroke}
\color[rgb]{0.540337,0.836858,0.278917}
\pgfpathmoveto{\pgfpoint{271.295990pt}{263.709473pt}}
\pgflineto{\pgfpoint{280.223969pt}{257.532623pt}}
\pgflineto{\pgfpoint{271.295990pt}{257.532623pt}}
\pgfpathclose
\pgfusepath{fill,stroke}
\pgfpathmoveto{\pgfpoint{271.295990pt}{263.709473pt}}
\pgflineto{\pgfpoint{280.223969pt}{263.709473pt}}
\pgflineto{\pgfpoint{280.223969pt}{257.532623pt}}
\pgfpathclose
\pgfusepath{fill,stroke}
\color[rgb]{0.580861,0.845663,0.253001}
\pgfpathmoveto{\pgfpoint{271.295990pt}{269.886322pt}}
\pgflineto{\pgfpoint{280.223969pt}{263.709473pt}}
\pgflineto{\pgfpoint{271.295990pt}{263.709473pt}}
\pgfpathclose
\pgfusepath{fill,stroke}
\pgfpathmoveto{\pgfpoint{271.295990pt}{269.886322pt}}
\pgflineto{\pgfpoint{280.223969pt}{269.886322pt}}
\pgflineto{\pgfpoint{280.223969pt}{263.709473pt}}
\pgfpathclose
\pgfusepath{fill,stroke}
\color[rgb]{0.622171,0.853816,0.226224}
\pgfpathmoveto{\pgfpoint{271.295990pt}{276.063141pt}}
\pgflineto{\pgfpoint{280.223969pt}{269.886322pt}}
\pgflineto{\pgfpoint{271.295990pt}{269.886322pt}}
\pgfpathclose
\pgfusepath{fill,stroke}
\pgfpathmoveto{\pgfpoint{271.295990pt}{276.063141pt}}
\pgflineto{\pgfpoint{280.223969pt}{276.063141pt}}
\pgflineto{\pgfpoint{280.223969pt}{269.886322pt}}
\pgfpathclose
\pgfusepath{fill,stroke}
\pgfpathmoveto{\pgfpoint{271.295990pt}{282.239990pt}}
\pgflineto{\pgfpoint{280.223969pt}{276.063141pt}}
\pgflineto{\pgfpoint{271.295990pt}{276.063141pt}}
\pgfpathclose
\pgfusepath{fill,stroke}
\pgfpathmoveto{\pgfpoint{271.295990pt}{282.239990pt}}
\pgflineto{\pgfpoint{280.223969pt}{282.239990pt}}
\pgflineto{\pgfpoint{280.223969pt}{276.063141pt}}
\pgfpathclose
\pgfusepath{fill,stroke}
\color[rgb]{0.462247,0.817338,0.327545}
\pgfpathmoveto{\pgfpoint{280.223969pt}{245.178955pt}}
\pgflineto{\pgfpoint{289.151978pt}{239.002106pt}}
\pgflineto{\pgfpoint{280.223969pt}{239.002106pt}}
\pgfpathclose
\pgfusepath{fill,stroke}
\pgfpathmoveto{\pgfpoint{280.223969pt}{245.178955pt}}
\pgflineto{\pgfpoint{289.151978pt}{245.178955pt}}
\pgflineto{\pgfpoint{289.151978pt}{239.002106pt}}
\pgfpathclose
\pgfusepath{fill,stroke}
\color[rgb]{0.500754,0.827409,0.303799}
\pgfpathmoveto{\pgfpoint{280.223969pt}{251.355804pt}}
\pgflineto{\pgfpoint{289.151978pt}{245.178955pt}}
\pgflineto{\pgfpoint{280.223969pt}{245.178955pt}}
\pgfpathclose
\pgfusepath{fill,stroke}
\pgfpathmoveto{\pgfpoint{280.223969pt}{251.355804pt}}
\pgflineto{\pgfpoint{289.151978pt}{251.355804pt}}
\pgflineto{\pgfpoint{289.151978pt}{245.178955pt}}
\pgfpathclose
\pgfusepath{fill,stroke}
\color[rgb]{0.540337,0.836858,0.278917}
\pgfpathmoveto{\pgfpoint{280.223969pt}{257.532623pt}}
\pgflineto{\pgfpoint{289.151978pt}{251.355804pt}}
\pgflineto{\pgfpoint{280.223969pt}{251.355804pt}}
\pgfpathclose
\pgfusepath{fill,stroke}
\pgfpathmoveto{\pgfpoint{280.223969pt}{257.532623pt}}
\pgflineto{\pgfpoint{289.151978pt}{257.532623pt}}
\pgflineto{\pgfpoint{289.151978pt}{251.355804pt}}
\pgfpathclose
\pgfusepath{fill,stroke}
\color[rgb]{0.580861,0.845663,0.253001}
\pgfpathmoveto{\pgfpoint{280.223969pt}{263.709473pt}}
\pgflineto{\pgfpoint{289.151978pt}{257.532623pt}}
\pgflineto{\pgfpoint{280.223969pt}{257.532623pt}}
\pgfpathclose
\pgfusepath{fill,stroke}
\pgfpathmoveto{\pgfpoint{280.223969pt}{263.709473pt}}
\pgflineto{\pgfpoint{289.151978pt}{263.709473pt}}
\pgflineto{\pgfpoint{289.151978pt}{257.532623pt}}
\pgfpathclose
\pgfusepath{fill,stroke}
\color[rgb]{0.622171,0.853816,0.226224}
\pgfpathmoveto{\pgfpoint{280.223969pt}{269.886322pt}}
\pgflineto{\pgfpoint{289.151978pt}{263.709473pt}}
\pgflineto{\pgfpoint{280.223969pt}{263.709473pt}}
\pgfpathclose
\pgfusepath{fill,stroke}
\pgfpathmoveto{\pgfpoint{280.223969pt}{269.886322pt}}
\pgflineto{\pgfpoint{289.151978pt}{269.886322pt}}
\pgflineto{\pgfpoint{289.151978pt}{263.709473pt}}
\pgfpathclose
\pgfusepath{fill,stroke}
\color[rgb]{0.664087,0.861321,0.198879}
\pgfpathmoveto{\pgfpoint{280.223969pt}{276.063141pt}}
\pgflineto{\pgfpoint{289.151978pt}{269.886322pt}}
\pgflineto{\pgfpoint{280.223969pt}{269.886322pt}}
\pgfpathclose
\pgfusepath{fill,stroke}
\pgfpathmoveto{\pgfpoint{280.223969pt}{276.063141pt}}
\pgflineto{\pgfpoint{289.151978pt}{276.063141pt}}
\pgflineto{\pgfpoint{289.151978pt}{269.886322pt}}
\pgfpathclose
\pgfusepath{fill,stroke}
\pgfpathmoveto{\pgfpoint{280.223969pt}{282.239990pt}}
\pgflineto{\pgfpoint{289.151978pt}{276.063141pt}}
\pgflineto{\pgfpoint{280.223969pt}{276.063141pt}}
\pgfpathclose
\pgfusepath{fill,stroke}
\pgfpathmoveto{\pgfpoint{280.223969pt}{282.239990pt}}
\pgflineto{\pgfpoint{289.151978pt}{282.239990pt}}
\pgflineto{\pgfpoint{289.151978pt}{276.063141pt}}
\pgfpathclose
\pgfusepath{fill,stroke}
\color[rgb]{0.500754,0.827409,0.303799}
\pgfpathmoveto{\pgfpoint{289.151978pt}{245.178955pt}}
\pgflineto{\pgfpoint{298.079987pt}{239.002106pt}}
\pgflineto{\pgfpoint{289.151978pt}{239.002106pt}}
\pgfpathclose
\pgfusepath{fill,stroke}
\pgfpathmoveto{\pgfpoint{289.151978pt}{245.178955pt}}
\pgflineto{\pgfpoint{298.079987pt}{245.178955pt}}
\pgflineto{\pgfpoint{298.079987pt}{239.002106pt}}
\pgfpathclose
\pgfusepath{fill,stroke}
\color[rgb]{0.540337,0.836858,0.278917}
\pgfpathmoveto{\pgfpoint{289.151978pt}{251.355804pt}}
\pgflineto{\pgfpoint{298.079987pt}{245.178955pt}}
\pgflineto{\pgfpoint{289.151978pt}{245.178955pt}}
\pgfpathclose
\pgfusepath{fill,stroke}
\pgfpathmoveto{\pgfpoint{289.151978pt}{251.355804pt}}
\pgflineto{\pgfpoint{298.079987pt}{251.355804pt}}
\pgflineto{\pgfpoint{298.079987pt}{245.178955pt}}
\pgfpathclose
\pgfusepath{fill,stroke}
\color[rgb]{0.580861,0.845663,0.253001}
\pgfpathmoveto{\pgfpoint{289.151978pt}{257.532623pt}}
\pgflineto{\pgfpoint{298.079987pt}{251.355804pt}}
\pgflineto{\pgfpoint{289.151978pt}{251.355804pt}}
\pgfpathclose
\pgfusepath{fill,stroke}
\pgfpathmoveto{\pgfpoint{289.151978pt}{257.532623pt}}
\pgflineto{\pgfpoint{298.079987pt}{257.532623pt}}
\pgflineto{\pgfpoint{298.079987pt}{251.355804pt}}
\pgfpathclose
\pgfusepath{fill,stroke}
\color[rgb]{0.622171,0.853816,0.226224}
\pgfpathmoveto{\pgfpoint{289.151978pt}{263.709473pt}}
\pgflineto{\pgfpoint{298.079987pt}{257.532623pt}}
\pgflineto{\pgfpoint{289.151978pt}{257.532623pt}}
\pgfpathclose
\pgfusepath{fill,stroke}
\pgfpathmoveto{\pgfpoint{289.151978pt}{263.709473pt}}
\pgflineto{\pgfpoint{298.079987pt}{263.709473pt}}
\pgflineto{\pgfpoint{298.079987pt}{257.532623pt}}
\pgfpathclose
\pgfusepath{fill,stroke}
\color[rgb]{0.664087,0.861321,0.198879}
\pgfpathmoveto{\pgfpoint{289.151978pt}{269.886322pt}}
\pgflineto{\pgfpoint{298.079987pt}{263.709473pt}}
\pgflineto{\pgfpoint{289.151978pt}{263.709473pt}}
\pgfpathclose
\pgfusepath{fill,stroke}
\pgfpathmoveto{\pgfpoint{289.151978pt}{269.886322pt}}
\pgflineto{\pgfpoint{298.079987pt}{269.886322pt}}
\pgflineto{\pgfpoint{298.079987pt}{263.709473pt}}
\pgfpathclose
\pgfusepath{fill,stroke}
\color[rgb]{0.706404,0.868206,0.171495}
\pgfpathmoveto{\pgfpoint{289.151978pt}{276.063141pt}}
\pgflineto{\pgfpoint{298.079987pt}{269.886322pt}}
\pgflineto{\pgfpoint{289.151978pt}{269.886322pt}}
\pgfpathclose
\pgfusepath{fill,stroke}
\pgfpathmoveto{\pgfpoint{289.151978pt}{276.063141pt}}
\pgflineto{\pgfpoint{298.079987pt}{276.063141pt}}
\pgflineto{\pgfpoint{298.079987pt}{269.886322pt}}
\pgfpathclose
\pgfusepath{fill,stroke}
\color[rgb]{0.748885,0.874522,0.145038}
\pgfpathmoveto{\pgfpoint{289.151978pt}{282.239990pt}}
\pgflineto{\pgfpoint{298.079987pt}{276.063141pt}}
\pgflineto{\pgfpoint{289.151978pt}{276.063141pt}}
\pgfpathclose
\pgfusepath{fill,stroke}
\pgfpathmoveto{\pgfpoint{289.151978pt}{282.239990pt}}
\pgflineto{\pgfpoint{298.079987pt}{282.239990pt}}
\pgflineto{\pgfpoint{298.079987pt}{276.063141pt}}
\pgfpathclose
\pgfusepath{fill,stroke}
\pgfpathmoveto{\pgfpoint{289.151978pt}{288.416840pt}}
\pgflineto{\pgfpoint{298.079987pt}{282.239990pt}}
\pgflineto{\pgfpoint{289.151978pt}{282.239990pt}}
\pgfpathclose
\pgfusepath{fill,stroke}
\pgfpathmoveto{\pgfpoint{289.151978pt}{288.416840pt}}
\pgflineto{\pgfpoint{298.079987pt}{288.416840pt}}
\pgflineto{\pgfpoint{298.079987pt}{282.239990pt}}
\pgfpathclose
\pgfusepath{fill,stroke}
\color[rgb]{0.580861,0.845663,0.253001}
\pgfpathmoveto{\pgfpoint{298.079987pt}{245.178955pt}}
\pgflineto{\pgfpoint{307.007965pt}{239.002106pt}}
\pgflineto{\pgfpoint{298.079987pt}{239.002106pt}}
\pgfpathclose
\pgfusepath{fill,stroke}
\pgfpathmoveto{\pgfpoint{298.079987pt}{245.178955pt}}
\pgflineto{\pgfpoint{307.007965pt}{245.178955pt}}
\pgflineto{\pgfpoint{307.007965pt}{239.002106pt}}
\pgfpathclose
\pgfusepath{fill,stroke}
\pgfpathmoveto{\pgfpoint{298.079987pt}{251.355804pt}}
\pgflineto{\pgfpoint{307.007965pt}{245.178955pt}}
\pgflineto{\pgfpoint{298.079987pt}{245.178955pt}}
\pgfpathclose
\pgfusepath{fill,stroke}
\pgfpathmoveto{\pgfpoint{298.079987pt}{251.355804pt}}
\pgflineto{\pgfpoint{307.007965pt}{251.355804pt}}
\pgflineto{\pgfpoint{307.007965pt}{245.178955pt}}
\pgfpathclose
\pgfusepath{fill,stroke}
\color[rgb]{0.622171,0.853816,0.226224}
\pgfpathmoveto{\pgfpoint{298.079987pt}{257.532623pt}}
\pgflineto{\pgfpoint{307.007965pt}{251.355804pt}}
\pgflineto{\pgfpoint{298.079987pt}{251.355804pt}}
\pgfpathclose
\pgfusepath{fill,stroke}
\pgfpathmoveto{\pgfpoint{298.079987pt}{257.532623pt}}
\pgflineto{\pgfpoint{307.007965pt}{257.532623pt}}
\pgflineto{\pgfpoint{307.007965pt}{251.355804pt}}
\pgfpathclose
\pgfusepath{fill,stroke}
\color[rgb]{0.664087,0.861321,0.198879}
\pgfpathmoveto{\pgfpoint{298.079987pt}{263.709473pt}}
\pgflineto{\pgfpoint{307.007965pt}{257.532623pt}}
\pgflineto{\pgfpoint{298.079987pt}{257.532623pt}}
\pgfpathclose
\pgfusepath{fill,stroke}
\pgfpathmoveto{\pgfpoint{298.079987pt}{263.709473pt}}
\pgflineto{\pgfpoint{307.007965pt}{263.709473pt}}
\pgflineto{\pgfpoint{307.007965pt}{257.532623pt}}
\pgfpathclose
\pgfusepath{fill,stroke}
\color[rgb]{0.706404,0.868206,0.171495}
\pgfpathmoveto{\pgfpoint{298.079987pt}{269.886322pt}}
\pgflineto{\pgfpoint{307.007965pt}{263.709473pt}}
\pgflineto{\pgfpoint{298.079987pt}{263.709473pt}}
\pgfpathclose
\pgfusepath{fill,stroke}
\pgfpathmoveto{\pgfpoint{298.079987pt}{269.886322pt}}
\pgflineto{\pgfpoint{307.007965pt}{269.886322pt}}
\pgflineto{\pgfpoint{307.007965pt}{263.709473pt}}
\pgfpathclose
\pgfusepath{fill,stroke}
\color[rgb]{0.748885,0.874522,0.145038}
\pgfpathmoveto{\pgfpoint{298.079987pt}{276.063141pt}}
\pgflineto{\pgfpoint{307.007965pt}{269.886322pt}}
\pgflineto{\pgfpoint{298.079987pt}{269.886322pt}}
\pgfpathclose
\pgfusepath{fill,stroke}
\pgfpathmoveto{\pgfpoint{298.079987pt}{276.063141pt}}
\pgflineto{\pgfpoint{307.007965pt}{276.063141pt}}
\pgflineto{\pgfpoint{307.007965pt}{269.886322pt}}
\pgfpathclose
\pgfusepath{fill,stroke}
\color[rgb]{0.791273,0.880346,0.121291}
\pgfpathmoveto{\pgfpoint{298.079987pt}{282.239990pt}}
\pgflineto{\pgfpoint{307.007965pt}{276.063141pt}}
\pgflineto{\pgfpoint{298.079987pt}{276.063141pt}}
\pgfpathclose
\pgfusepath{fill,stroke}
\pgfpathmoveto{\pgfpoint{298.079987pt}{282.239990pt}}
\pgflineto{\pgfpoint{307.007965pt}{282.239990pt}}
\pgflineto{\pgfpoint{307.007965pt}{276.063141pt}}
\pgfpathclose
\pgfusepath{fill,stroke}
\pgfpathmoveto{\pgfpoint{298.079987pt}{288.416840pt}}
\pgflineto{\pgfpoint{307.007965pt}{282.239990pt}}
\pgflineto{\pgfpoint{298.079987pt}{282.239990pt}}
\pgfpathclose
\pgfusepath{fill,stroke}
\pgfpathmoveto{\pgfpoint{298.079987pt}{288.416840pt}}
\pgflineto{\pgfpoint{307.007965pt}{288.416840pt}}
\pgflineto{\pgfpoint{307.007965pt}{282.239990pt}}
\pgfpathclose
\pgfusepath{fill,stroke}
\color[rgb]{0.622171,0.853816,0.226224}
\pgfpathmoveto{\pgfpoint{307.007965pt}{251.355804pt}}
\pgflineto{\pgfpoint{315.935974pt}{245.178955pt}}
\pgflineto{\pgfpoint{307.007965pt}{245.178955pt}}
\pgfpathclose
\pgfusepath{fill,stroke}
\pgfpathmoveto{\pgfpoint{307.007965pt}{251.355804pt}}
\pgflineto{\pgfpoint{315.935974pt}{251.355804pt}}
\pgflineto{\pgfpoint{315.935974pt}{245.178955pt}}
\pgfpathclose
\pgfusepath{fill,stroke}
\color[rgb]{0.664087,0.861321,0.198879}
\pgfpathmoveto{\pgfpoint{307.007965pt}{257.532623pt}}
\pgflineto{\pgfpoint{315.935974pt}{251.355804pt}}
\pgflineto{\pgfpoint{307.007965pt}{251.355804pt}}
\pgfpathclose
\pgfusepath{fill,stroke}
\pgfpathmoveto{\pgfpoint{307.007965pt}{257.532623pt}}
\pgflineto{\pgfpoint{315.935974pt}{257.532623pt}}
\pgflineto{\pgfpoint{315.935974pt}{251.355804pt}}
\pgfpathclose
\pgfusepath{fill,stroke}
\color[rgb]{0.706404,0.868206,0.171495}
\pgfpathmoveto{\pgfpoint{307.007965pt}{263.709473pt}}
\pgflineto{\pgfpoint{315.935974pt}{257.532623pt}}
\pgflineto{\pgfpoint{307.007965pt}{257.532623pt}}
\pgfpathclose
\pgfusepath{fill,stroke}
\pgfpathmoveto{\pgfpoint{307.007965pt}{263.709473pt}}
\pgflineto{\pgfpoint{315.935974pt}{263.709473pt}}
\pgflineto{\pgfpoint{315.935974pt}{257.532623pt}}
\pgfpathclose
\pgfusepath{fill,stroke}
\color[rgb]{0.748885,0.874522,0.145038}
\pgfpathmoveto{\pgfpoint{307.007965pt}{269.886322pt}}
\pgflineto{\pgfpoint{315.935974pt}{263.709473pt}}
\pgflineto{\pgfpoint{307.007965pt}{263.709473pt}}
\pgfpathclose
\pgfusepath{fill,stroke}
\pgfpathmoveto{\pgfpoint{307.007965pt}{269.886322pt}}
\pgflineto{\pgfpoint{315.935974pt}{269.886322pt}}
\pgflineto{\pgfpoint{315.935974pt}{263.709473pt}}
\pgfpathclose
\pgfusepath{fill,stroke}
\color[rgb]{0.791273,0.880346,0.121291}
\pgfpathmoveto{\pgfpoint{307.007965pt}{276.063141pt}}
\pgflineto{\pgfpoint{315.935974pt}{269.886322pt}}
\pgflineto{\pgfpoint{307.007965pt}{269.886322pt}}
\pgfpathclose
\pgfusepath{fill,stroke}
\pgfpathmoveto{\pgfpoint{307.007965pt}{276.063141pt}}
\pgflineto{\pgfpoint{315.935974pt}{276.063141pt}}
\pgflineto{\pgfpoint{315.935974pt}{269.886322pt}}
\pgfpathclose
\pgfusepath{fill,stroke}
\color[rgb]{0.833302,0.885780,0.103326}
\pgfpathmoveto{\pgfpoint{307.007965pt}{282.239990pt}}
\pgflineto{\pgfpoint{315.935974pt}{276.063141pt}}
\pgflineto{\pgfpoint{307.007965pt}{276.063141pt}}
\pgfpathclose
\pgfusepath{fill,stroke}
\pgfpathmoveto{\pgfpoint{307.007965pt}{282.239990pt}}
\pgflineto{\pgfpoint{315.935974pt}{282.239990pt}}
\pgflineto{\pgfpoint{315.935974pt}{276.063141pt}}
\pgfpathclose
\pgfusepath{fill,stroke}
\color[rgb]{0.874718,0.890945,0.095351}
\pgfpathmoveto{\pgfpoint{307.007965pt}{288.416840pt}}
\pgflineto{\pgfpoint{315.935974pt}{282.239990pt}}
\pgflineto{\pgfpoint{307.007965pt}{282.239990pt}}
\pgfpathclose
\pgfusepath{fill,stroke}
\pgfpathmoveto{\pgfpoint{307.007965pt}{288.416840pt}}
\pgflineto{\pgfpoint{315.935974pt}{288.416840pt}}
\pgflineto{\pgfpoint{315.935974pt}{282.239990pt}}
\pgfpathclose
\pgfusepath{fill,stroke}
\color[rgb]{0.706404,0.868206,0.171495}
\pgfpathmoveto{\pgfpoint{315.935974pt}{251.355804pt}}
\pgflineto{\pgfpoint{324.863983pt}{245.178955pt}}
\pgflineto{\pgfpoint{315.935974pt}{245.178955pt}}
\pgfpathclose
\pgfusepath{fill,stroke}
\pgfpathmoveto{\pgfpoint{315.935974pt}{251.355804pt}}
\pgflineto{\pgfpoint{324.863983pt}{251.355804pt}}
\pgflineto{\pgfpoint{324.863983pt}{245.178955pt}}
\pgfpathclose
\pgfusepath{fill,stroke}
\pgfpathmoveto{\pgfpoint{315.935974pt}{257.532623pt}}
\pgflineto{\pgfpoint{324.863983pt}{251.355804pt}}
\pgflineto{\pgfpoint{315.935974pt}{251.355804pt}}
\pgfpathclose
\pgfusepath{fill,stroke}
\pgfpathmoveto{\pgfpoint{315.935974pt}{257.532623pt}}
\pgflineto{\pgfpoint{324.863983pt}{257.532623pt}}
\pgflineto{\pgfpoint{324.863983pt}{251.355804pt}}
\pgfpathclose
\pgfusepath{fill,stroke}
\color[rgb]{0.748885,0.874522,0.145038}
\pgfpathmoveto{\pgfpoint{315.935974pt}{263.709473pt}}
\pgflineto{\pgfpoint{324.863983pt}{257.532623pt}}
\pgflineto{\pgfpoint{315.935974pt}{257.532623pt}}
\pgfpathclose
\pgfusepath{fill,stroke}
\pgfpathmoveto{\pgfpoint{315.935974pt}{263.709473pt}}
\pgflineto{\pgfpoint{324.863983pt}{263.709473pt}}
\pgflineto{\pgfpoint{324.863983pt}{257.532623pt}}
\pgfpathclose
\pgfusepath{fill,stroke}
\color[rgb]{0.791273,0.880346,0.121291}
\pgfpathmoveto{\pgfpoint{315.935974pt}{269.886322pt}}
\pgflineto{\pgfpoint{324.863983pt}{263.709473pt}}
\pgflineto{\pgfpoint{315.935974pt}{263.709473pt}}
\pgfpathclose
\pgfusepath{fill,stroke}
\pgfpathmoveto{\pgfpoint{315.935974pt}{269.886322pt}}
\pgflineto{\pgfpoint{324.863983pt}{269.886322pt}}
\pgflineto{\pgfpoint{324.863983pt}{263.709473pt}}
\pgfpathclose
\pgfusepath{fill,stroke}
\color[rgb]{0.833302,0.885780,0.103326}
\pgfpathmoveto{\pgfpoint{315.935974pt}{276.063141pt}}
\pgflineto{\pgfpoint{324.863983pt}{269.886322pt}}
\pgflineto{\pgfpoint{315.935974pt}{269.886322pt}}
\pgfpathclose
\pgfusepath{fill,stroke}
\pgfpathmoveto{\pgfpoint{315.935974pt}{276.063141pt}}
\pgflineto{\pgfpoint{324.863983pt}{276.063141pt}}
\pgflineto{\pgfpoint{324.863983pt}{269.886322pt}}
\pgfpathclose
\pgfusepath{fill,stroke}
\color[rgb]{0.874718,0.890945,0.095351}
\pgfpathmoveto{\pgfpoint{315.935974pt}{282.239990pt}}
\pgflineto{\pgfpoint{324.863983pt}{276.063141pt}}
\pgflineto{\pgfpoint{315.935974pt}{276.063141pt}}
\pgfpathclose
\pgfusepath{fill,stroke}
\pgfpathmoveto{\pgfpoint{315.935974pt}{282.239990pt}}
\pgflineto{\pgfpoint{324.863983pt}{282.239990pt}}
\pgflineto{\pgfpoint{324.863983pt}{276.063141pt}}
\pgfpathclose
\pgfusepath{fill,stroke}
\color[rgb]{0.915296,0.895974,0.100470}
\pgfpathmoveto{\pgfpoint{315.935974pt}{288.416840pt}}
\pgflineto{\pgfpoint{324.863983pt}{282.239990pt}}
\pgflineto{\pgfpoint{315.935974pt}{282.239990pt}}
\pgfpathclose
\pgfusepath{fill,stroke}
\pgfpathmoveto{\pgfpoint{315.935974pt}{288.416840pt}}
\pgflineto{\pgfpoint{324.863983pt}{288.416840pt}}
\pgflineto{\pgfpoint{324.863983pt}{282.239990pt}}
\pgfpathclose
\pgfusepath{fill,stroke}
\color[rgb]{0.748885,0.874522,0.145038}
\pgfpathmoveto{\pgfpoint{324.863983pt}{251.355804pt}}
\pgflineto{\pgfpoint{333.791992pt}{245.178955pt}}
\pgflineto{\pgfpoint{324.863983pt}{245.178955pt}}
\pgfpathclose
\pgfusepath{fill,stroke}
\pgfpathmoveto{\pgfpoint{324.863983pt}{251.355804pt}}
\pgflineto{\pgfpoint{333.791992pt}{251.355804pt}}
\pgflineto{\pgfpoint{333.791992pt}{245.178955pt}}
\pgfpathclose
\pgfusepath{fill,stroke}
\pgfpathmoveto{\pgfpoint{324.863983pt}{257.532623pt}}
\pgflineto{\pgfpoint{333.791992pt}{251.355804pt}}
\pgflineto{\pgfpoint{324.863983pt}{251.355804pt}}
\pgfpathclose
\pgfusepath{fill,stroke}
\pgfpathmoveto{\pgfpoint{324.863983pt}{257.532623pt}}
\pgflineto{\pgfpoint{333.791992pt}{257.532623pt}}
\pgflineto{\pgfpoint{333.791992pt}{251.355804pt}}
\pgfpathclose
\pgfusepath{fill,stroke}
\color[rgb]{0.791273,0.880346,0.121291}
\pgfpathmoveto{\pgfpoint{324.863983pt}{263.709473pt}}
\pgflineto{\pgfpoint{333.791992pt}{257.532623pt}}
\pgflineto{\pgfpoint{324.863983pt}{257.532623pt}}
\pgfpathclose
\pgfusepath{fill,stroke}
\pgfpathmoveto{\pgfpoint{324.863983pt}{263.709473pt}}
\pgflineto{\pgfpoint{333.791992pt}{263.709473pt}}
\pgflineto{\pgfpoint{333.791992pt}{257.532623pt}}
\pgfpathclose
\pgfusepath{fill,stroke}
\color[rgb]{0.833302,0.885780,0.103326}
\pgfpathmoveto{\pgfpoint{324.863983pt}{269.886322pt}}
\pgflineto{\pgfpoint{333.791992pt}{263.709473pt}}
\pgflineto{\pgfpoint{324.863983pt}{263.709473pt}}
\pgfpathclose
\pgfusepath{fill,stroke}
\pgfpathmoveto{\pgfpoint{324.863983pt}{269.886322pt}}
\pgflineto{\pgfpoint{333.791992pt}{269.886322pt}}
\pgflineto{\pgfpoint{333.791992pt}{263.709473pt}}
\pgfpathclose
\pgfusepath{fill,stroke}
\color[rgb]{0.874718,0.890945,0.095351}
\pgfpathmoveto{\pgfpoint{324.863983pt}{276.063141pt}}
\pgflineto{\pgfpoint{333.791992pt}{269.886322pt}}
\pgflineto{\pgfpoint{324.863983pt}{269.886322pt}}
\pgfpathclose
\pgfusepath{fill,stroke}
\pgfpathmoveto{\pgfpoint{324.863983pt}{276.063141pt}}
\pgflineto{\pgfpoint{333.791992pt}{276.063141pt}}
\pgflineto{\pgfpoint{333.791992pt}{269.886322pt}}
\pgfpathclose
\pgfusepath{fill,stroke}
\color[rgb]{0.915296,0.895974,0.100470}
\pgfpathmoveto{\pgfpoint{324.863983pt}{282.239990pt}}
\pgflineto{\pgfpoint{333.791992pt}{276.063141pt}}
\pgflineto{\pgfpoint{324.863983pt}{276.063141pt}}
\pgfpathclose
\pgfusepath{fill,stroke}
\pgfpathmoveto{\pgfpoint{324.863983pt}{282.239990pt}}
\pgflineto{\pgfpoint{333.791992pt}{282.239990pt}}
\pgflineto{\pgfpoint{333.791992pt}{276.063141pt}}
\pgfpathclose
\pgfusepath{fill,stroke}
\color[rgb]{0.791273,0.880346,0.121291}
\pgfpathmoveto{\pgfpoint{333.791992pt}{251.355804pt}}
\pgflineto{\pgfpoint{342.719971pt}{245.178955pt}}
\pgflineto{\pgfpoint{333.791992pt}{245.178955pt}}
\pgfpathclose
\pgfusepath{fill,stroke}
\pgfpathmoveto{\pgfpoint{333.791992pt}{251.355804pt}}
\pgflineto{\pgfpoint{342.719971pt}{251.355804pt}}
\pgflineto{\pgfpoint{342.719971pt}{245.178955pt}}
\pgfpathclose
\pgfusepath{fill,stroke}
\color[rgb]{0.833302,0.885780,0.103326}
\pgfpathmoveto{\pgfpoint{333.791992pt}{257.532623pt}}
\pgflineto{\pgfpoint{342.719971pt}{251.355804pt}}
\pgflineto{\pgfpoint{333.791992pt}{251.355804pt}}
\pgfpathclose
\pgfusepath{fill,stroke}
\pgfpathmoveto{\pgfpoint{333.791992pt}{257.532623pt}}
\pgflineto{\pgfpoint{342.719971pt}{257.532623pt}}
\pgflineto{\pgfpoint{342.719971pt}{251.355804pt}}
\pgfpathclose
\pgfusepath{fill,stroke}
\pgfpathmoveto{\pgfpoint{333.791992pt}{263.709473pt}}
\pgflineto{\pgfpoint{342.719971pt}{257.532623pt}}
\pgflineto{\pgfpoint{333.791992pt}{257.532623pt}}
\pgfpathclose
\pgfusepath{fill,stroke}
\pgfpathmoveto{\pgfpoint{333.791992pt}{263.709473pt}}
\pgflineto{\pgfpoint{342.719971pt}{263.709473pt}}
\pgflineto{\pgfpoint{342.719971pt}{257.532623pt}}
\pgfpathclose
\pgfusepath{fill,stroke}
\color[rgb]{0.874718,0.890945,0.095351}
\pgfpathmoveto{\pgfpoint{333.791992pt}{269.886322pt}}
\pgflineto{\pgfpoint{342.719971pt}{263.709473pt}}
\pgflineto{\pgfpoint{333.791992pt}{263.709473pt}}
\pgfpathclose
\pgfusepath{fill,stroke}
\pgfpathmoveto{\pgfpoint{333.791992pt}{269.886322pt}}
\pgflineto{\pgfpoint{342.719971pt}{269.886322pt}}
\pgflineto{\pgfpoint{342.719971pt}{263.709473pt}}
\pgfpathclose
\pgfusepath{fill,stroke}
\color[rgb]{0.915296,0.895974,0.100470}
\pgfpathmoveto{\pgfpoint{333.791992pt}{276.063141pt}}
\pgflineto{\pgfpoint{342.719971pt}{269.886322pt}}
\pgflineto{\pgfpoint{333.791992pt}{269.886322pt}}
\pgfpathclose
\pgfusepath{fill,stroke}
\pgfpathmoveto{\pgfpoint{333.791992pt}{276.063141pt}}
\pgflineto{\pgfpoint{342.719971pt}{276.063141pt}}
\pgflineto{\pgfpoint{342.719971pt}{269.886322pt}}
\pgfpathclose
\pgfusepath{fill,stroke}
\color[rgb]{0.833302,0.885780,0.103326}
\pgfpathmoveto{\pgfpoint{342.719971pt}{251.355804pt}}
\pgflineto{\pgfpoint{351.647980pt}{245.178955pt}}
\pgflineto{\pgfpoint{342.719971pt}{245.178955pt}}
\pgfpathclose
\pgfusepath{fill,stroke}
\pgfpathmoveto{\pgfpoint{342.719971pt}{251.355804pt}}
\pgflineto{\pgfpoint{351.647980pt}{251.355804pt}}
\pgflineto{\pgfpoint{351.647980pt}{245.178955pt}}
\pgfpathclose
\pgfusepath{fill,stroke}
\color[rgb]{0.874718,0.890945,0.095351}
\pgfpathmoveto{\pgfpoint{342.719971pt}{257.532623pt}}
\pgflineto{\pgfpoint{351.647980pt}{251.355804pt}}
\pgflineto{\pgfpoint{342.719971pt}{251.355804pt}}
\pgfpathclose
\pgfusepath{fill,stroke}
\pgfpathmoveto{\pgfpoint{342.719971pt}{257.532623pt}}
\pgflineto{\pgfpoint{351.647980pt}{257.532623pt}}
\pgflineto{\pgfpoint{351.647980pt}{251.355804pt}}
\pgfpathclose
\pgfusepath{fill,stroke}
\pgfpathmoveto{\pgfpoint{342.719971pt}{263.709473pt}}
\pgflineto{\pgfpoint{351.647980pt}{257.532623pt}}
\pgflineto{\pgfpoint{342.719971pt}{257.532623pt}}
\pgfpathclose
\pgfusepath{fill,stroke}
\pgfpathmoveto{\pgfpoint{342.719971pt}{263.709473pt}}
\pgflineto{\pgfpoint{351.647980pt}{263.709473pt}}
\pgflineto{\pgfpoint{351.647980pt}{257.532623pt}}
\pgfpathclose
\pgfusepath{fill,stroke}
\color[rgb]{0.915296,0.895974,0.100470}
\pgfpathmoveto{\pgfpoint{342.719971pt}{269.886322pt}}
\pgflineto{\pgfpoint{351.647980pt}{263.709473pt}}
\pgflineto{\pgfpoint{342.719971pt}{263.709473pt}}
\pgfpathclose
\pgfusepath{fill,stroke}
\pgfpathmoveto{\pgfpoint{342.719971pt}{269.886322pt}}
\pgflineto{\pgfpoint{351.647980pt}{269.886322pt}}
\pgflineto{\pgfpoint{351.647980pt}{263.709473pt}}
\pgfpathclose
\pgfusepath{fill,stroke}
\color[rgb]{0.874718,0.890945,0.095351}
\pgfpathmoveto{\pgfpoint{351.647980pt}{251.355804pt}}
\pgflineto{\pgfpoint{360.575958pt}{245.178955pt}}
\pgflineto{\pgfpoint{351.647980pt}{245.178955pt}}
\pgfpathclose
\pgfusepath{fill,stroke}
\pgfpathmoveto{\pgfpoint{351.647980pt}{251.355804pt}}
\pgflineto{\pgfpoint{360.575958pt}{251.355804pt}}
\pgflineto{\pgfpoint{360.575958pt}{245.178955pt}}
\pgfpathclose
\pgfusepath{fill,stroke}
\color[rgb]{0.915296,0.895974,0.100470}
\pgfpathmoveto{\pgfpoint{351.647980pt}{257.532623pt}}
\pgflineto{\pgfpoint{360.575958pt}{251.355804pt}}
\pgflineto{\pgfpoint{351.647980pt}{251.355804pt}}
\pgfpathclose
\pgfusepath{fill,stroke}
\pgfpathmoveto{\pgfpoint{351.647980pt}{257.532623pt}}
\pgflineto{\pgfpoint{360.575958pt}{257.532623pt}}
\pgflineto{\pgfpoint{360.575958pt}{251.355804pt}}
\pgfpathclose
\pgfusepath{fill,stroke}
\color[rgb]{0.954840,0.901006,0.117876}
\pgfpathmoveto{\pgfpoint{351.647980pt}{263.709473pt}}
\pgflineto{\pgfpoint{360.575958pt}{257.532623pt}}
\pgflineto{\pgfpoint{351.647980pt}{257.532623pt}}
\pgfpathclose
\pgfusepath{fill,stroke}
\pgfpathmoveto{\pgfpoint{351.647980pt}{263.709473pt}}
\pgflineto{\pgfpoint{360.575958pt}{263.709473pt}}
\pgflineto{\pgfpoint{360.575958pt}{257.532623pt}}
\pgfpathclose
\pgfusepath{fill,stroke}
\color[rgb]{0.664087,0.861321,0.198879}
\pgfpathmoveto{\pgfpoint{244.511993pt}{313.124207pt}}
\pgflineto{\pgfpoint{253.440002pt}{306.947388pt}}
\pgflineto{\pgfpoint{244.511993pt}{306.947388pt}}
\pgfpathclose
\pgfusepath{fill,stroke}
\pgfpathmoveto{\pgfpoint{244.511993pt}{313.124207pt}}
\pgflineto{\pgfpoint{253.440002pt}{313.124207pt}}
\pgflineto{\pgfpoint{253.440002pt}{306.947388pt}}
\pgfpathclose
\pgfusepath{fill,stroke}
\color[rgb]{0.706404,0.868206,0.171495}
\pgfpathmoveto{\pgfpoint{253.440002pt}{313.124207pt}}
\pgflineto{\pgfpoint{262.367981pt}{306.947388pt}}
\pgflineto{\pgfpoint{253.440002pt}{306.947388pt}}
\pgfpathclose
\pgfusepath{fill,stroke}
\pgfpathmoveto{\pgfpoint{253.440002pt}{313.124207pt}}
\pgflineto{\pgfpoint{262.367981pt}{313.124207pt}}
\pgflineto{\pgfpoint{262.367981pt}{306.947388pt}}
\pgfpathclose
\pgfusepath{fill,stroke}
\color[rgb]{0.748885,0.874522,0.145038}
\pgfpathmoveto{\pgfpoint{253.440002pt}{319.301056pt}}
\pgflineto{\pgfpoint{262.367981pt}{313.124207pt}}
\pgflineto{\pgfpoint{253.440002pt}{313.124207pt}}
\pgfpathclose
\pgfusepath{fill,stroke}
\pgfpathmoveto{\pgfpoint{253.440002pt}{319.301056pt}}
\pgflineto{\pgfpoint{262.367981pt}{319.301056pt}}
\pgflineto{\pgfpoint{262.367981pt}{313.124207pt}}
\pgfpathclose
\pgfusepath{fill,stroke}
\color[rgb]{0.791273,0.880346,0.121291}
\pgfpathmoveto{\pgfpoint{253.440002pt}{325.477905pt}}
\pgflineto{\pgfpoint{262.367981pt}{319.301056pt}}
\pgflineto{\pgfpoint{253.440002pt}{319.301056pt}}
\pgfpathclose
\pgfusepath{fill,stroke}
\pgfpathmoveto{\pgfpoint{253.440002pt}{325.477905pt}}
\pgflineto{\pgfpoint{262.367981pt}{325.477905pt}}
\pgflineto{\pgfpoint{262.367981pt}{319.301056pt}}
\pgfpathclose
\pgfusepath{fill,stroke}
\color[rgb]{0.748885,0.874522,0.145038}
\pgfpathmoveto{\pgfpoint{262.367981pt}{306.947388pt}}
\pgflineto{\pgfpoint{271.295990pt}{306.947388pt}}
\pgflineto{\pgfpoint{271.295990pt}{300.770538pt}}
\pgfpathclose
\pgfusepath{fill,stroke}
\pgfpathmoveto{\pgfpoint{262.367981pt}{313.124207pt}}
\pgflineto{\pgfpoint{271.295990pt}{306.947388pt}}
\pgflineto{\pgfpoint{262.367981pt}{306.947388pt}}
\pgfpathclose
\pgfusepath{fill,stroke}
\pgfpathmoveto{\pgfpoint{262.367981pt}{313.124207pt}}
\pgflineto{\pgfpoint{271.295990pt}{313.124207pt}}
\pgflineto{\pgfpoint{271.295990pt}{306.947388pt}}
\pgfpathclose
\pgfusepath{fill,stroke}
\color[rgb]{0.791273,0.880346,0.121291}
\pgfpathmoveto{\pgfpoint{262.367981pt}{319.301056pt}}
\pgflineto{\pgfpoint{271.295990pt}{313.124207pt}}
\pgflineto{\pgfpoint{262.367981pt}{313.124207pt}}
\pgfpathclose
\pgfusepath{fill,stroke}
\pgfpathmoveto{\pgfpoint{262.367981pt}{319.301056pt}}
\pgflineto{\pgfpoint{271.295990pt}{319.301056pt}}
\pgflineto{\pgfpoint{271.295990pt}{313.124207pt}}
\pgfpathclose
\pgfusepath{fill,stroke}
\color[rgb]{0.833302,0.885780,0.103326}
\pgfpathmoveto{\pgfpoint{262.367981pt}{325.477905pt}}
\pgflineto{\pgfpoint{271.295990pt}{319.301056pt}}
\pgflineto{\pgfpoint{262.367981pt}{319.301056pt}}
\pgfpathclose
\pgfusepath{fill,stroke}
\pgfpathmoveto{\pgfpoint{262.367981pt}{325.477905pt}}
\pgflineto{\pgfpoint{271.295990pt}{325.477905pt}}
\pgflineto{\pgfpoint{271.295990pt}{319.301056pt}}
\pgfpathclose
\pgfusepath{fill,stroke}
\color[rgb]{0.791273,0.880346,0.121291}
\pgfpathmoveto{\pgfpoint{271.295990pt}{306.947388pt}}
\pgflineto{\pgfpoint{280.223969pt}{300.770538pt}}
\pgflineto{\pgfpoint{271.295990pt}{300.770538pt}}
\pgfpathclose
\pgfusepath{fill,stroke}
\pgfpathmoveto{\pgfpoint{271.295990pt}{306.947388pt}}
\pgflineto{\pgfpoint{280.223969pt}{306.947388pt}}
\pgflineto{\pgfpoint{280.223969pt}{300.770538pt}}
\pgfpathclose
\pgfusepath{fill,stroke}
\color[rgb]{0.833302,0.885780,0.103326}
\pgfpathmoveto{\pgfpoint{271.295990pt}{313.124207pt}}
\pgflineto{\pgfpoint{280.223969pt}{306.947388pt}}
\pgflineto{\pgfpoint{271.295990pt}{306.947388pt}}
\pgfpathclose
\pgfusepath{fill,stroke}
\pgfpathmoveto{\pgfpoint{271.295990pt}{313.124207pt}}
\pgflineto{\pgfpoint{280.223969pt}{313.124207pt}}
\pgflineto{\pgfpoint{280.223969pt}{306.947388pt}}
\pgfpathclose
\pgfusepath{fill,stroke}
\pgfpathmoveto{\pgfpoint{271.295990pt}{319.301056pt}}
\pgflineto{\pgfpoint{280.223969pt}{313.124207pt}}
\pgflineto{\pgfpoint{271.295990pt}{313.124207pt}}
\pgfpathclose
\pgfusepath{fill,stroke}
\pgfpathmoveto{\pgfpoint{271.295990pt}{319.301056pt}}
\pgflineto{\pgfpoint{280.223969pt}{319.301056pt}}
\pgflineto{\pgfpoint{280.223969pt}{313.124207pt}}
\pgfpathclose
\pgfusepath{fill,stroke}
\pgfpathmoveto{\pgfpoint{280.223969pt}{306.947388pt}}
\pgflineto{\pgfpoint{289.151978pt}{300.770538pt}}
\pgflineto{\pgfpoint{280.223969pt}{300.770538pt}}
\pgfpathclose
\pgfusepath{fill,stroke}
\pgfpathmoveto{\pgfpoint{280.223969pt}{306.947388pt}}
\pgflineto{\pgfpoint{289.151978pt}{306.947388pt}}
\pgflineto{\pgfpoint{289.151978pt}{300.770538pt}}
\pgfpathclose
\pgfusepath{fill,stroke}
\color[rgb]{0.874718,0.890945,0.095351}
\pgfpathmoveto{\pgfpoint{280.223969pt}{313.124207pt}}
\pgflineto{\pgfpoint{289.151978pt}{306.947388pt}}
\pgflineto{\pgfpoint{280.223969pt}{306.947388pt}}
\pgfpathclose
\pgfusepath{fill,stroke}
\pgfpathmoveto{\pgfpoint{280.223969pt}{313.124207pt}}
\pgflineto{\pgfpoint{289.151978pt}{313.124207pt}}
\pgflineto{\pgfpoint{289.151978pt}{306.947388pt}}
\pgfpathclose
\pgfusepath{fill,stroke}
\pgfpathmoveto{\pgfpoint{289.151978pt}{306.947388pt}}
\pgflineto{\pgfpoint{298.079987pt}{300.770538pt}}
\pgflineto{\pgfpoint{289.151978pt}{300.770538pt}}
\pgfpathclose
\pgfusepath{fill,stroke}
\pgfpathmoveto{\pgfpoint{289.151978pt}{306.947388pt}}
\pgflineto{\pgfpoint{298.079987pt}{306.947388pt}}
\pgflineto{\pgfpoint{298.079987pt}{300.770538pt}}
\pgfpathclose
\pgfusepath{fill,stroke}
\pgfpathmoveto{\pgfpoint{298.079987pt}{300.770538pt}}
\pgflineto{\pgfpoint{307.007965pt}{300.770538pt}}
\pgflineto{\pgfpoint{307.007965pt}{294.593689pt}}
\pgfpathclose
\pgfusepath{fill,stroke}
\color[rgb]{0.424933,0.806674,0.350099}
\pgfpathmoveto{\pgfpoint{226.655975pt}{282.239990pt}}
\pgflineto{\pgfpoint{235.583969pt}{276.063141pt}}
\pgflineto{\pgfpoint{226.655975pt}{276.063141pt}}
\pgfpathclose
\pgfusepath{fill,stroke}
\pgfpathmoveto{\pgfpoint{226.655975pt}{282.239990pt}}
\pgflineto{\pgfpoint{235.583969pt}{282.239990pt}}
\pgflineto{\pgfpoint{235.583969pt}{276.063141pt}}
\pgfpathclose
\pgfusepath{fill,stroke}
\color[rgb]{0.462247,0.817338,0.327545}
\pgfpathmoveto{\pgfpoint{226.655975pt}{288.416840pt}}
\pgflineto{\pgfpoint{235.583969pt}{282.239990pt}}
\pgflineto{\pgfpoint{226.655975pt}{282.239990pt}}
\pgfpathclose
\pgfusepath{fill,stroke}
\pgfpathmoveto{\pgfpoint{226.655975pt}{288.416840pt}}
\pgflineto{\pgfpoint{235.583969pt}{288.416840pt}}
\pgflineto{\pgfpoint{235.583969pt}{282.239990pt}}
\pgfpathclose
\pgfusepath{fill,stroke}
\pgfpathmoveto{\pgfpoint{235.583969pt}{282.239990pt}}
\pgflineto{\pgfpoint{244.511993pt}{276.063141pt}}
\pgflineto{\pgfpoint{235.583969pt}{276.063141pt}}
\pgfpathclose
\pgfusepath{fill,stroke}
\pgfpathmoveto{\pgfpoint{235.583969pt}{282.239990pt}}
\pgflineto{\pgfpoint{244.511993pt}{282.239990pt}}
\pgflineto{\pgfpoint{244.511993pt}{276.063141pt}}
\pgfpathclose
\pgfusepath{fill,stroke}
\color[rgb]{0.500754,0.827409,0.303799}
\pgfpathmoveto{\pgfpoint{235.583969pt}{288.416840pt}}
\pgflineto{\pgfpoint{244.511993pt}{282.239990pt}}
\pgflineto{\pgfpoint{235.583969pt}{282.239990pt}}
\pgfpathclose
\pgfusepath{fill,stroke}
\pgfpathmoveto{\pgfpoint{235.583969pt}{288.416840pt}}
\pgflineto{\pgfpoint{244.511993pt}{288.416840pt}}
\pgflineto{\pgfpoint{244.511993pt}{282.239990pt}}
\pgfpathclose
\pgfusepath{fill,stroke}
\color[rgb]{0.540337,0.836858,0.278917}
\pgfpathmoveto{\pgfpoint{235.583969pt}{294.593689pt}}
\pgflineto{\pgfpoint{244.511993pt}{288.416840pt}}
\pgflineto{\pgfpoint{235.583969pt}{288.416840pt}}
\pgfpathclose
\pgfusepath{fill,stroke}
\pgfpathmoveto{\pgfpoint{235.583969pt}{294.593689pt}}
\pgflineto{\pgfpoint{244.511993pt}{294.593689pt}}
\pgflineto{\pgfpoint{244.511993pt}{288.416840pt}}
\pgfpathclose
\pgfusepath{fill,stroke}
\color[rgb]{0.580861,0.845663,0.253001}
\pgfpathmoveto{\pgfpoint{235.583969pt}{300.770538pt}}
\pgflineto{\pgfpoint{244.511993pt}{294.593689pt}}
\pgflineto{\pgfpoint{235.583969pt}{294.593689pt}}
\pgfpathclose
\pgfusepath{fill,stroke}
\pgfpathmoveto{\pgfpoint{235.583969pt}{300.770538pt}}
\pgflineto{\pgfpoint{244.511993pt}{300.770538pt}}
\pgflineto{\pgfpoint{244.511993pt}{294.593689pt}}
\pgfpathclose
\pgfusepath{fill,stroke}
\color[rgb]{0.540337,0.836858,0.278917}
\pgfpathmoveto{\pgfpoint{244.511993pt}{288.416840pt}}
\pgflineto{\pgfpoint{253.440002pt}{282.239990pt}}
\pgflineto{\pgfpoint{244.511993pt}{282.239990pt}}
\pgfpathclose
\pgfusepath{fill,stroke}
\pgfpathmoveto{\pgfpoint{244.511993pt}{288.416840pt}}
\pgflineto{\pgfpoint{253.440002pt}{288.416840pt}}
\pgflineto{\pgfpoint{253.440002pt}{282.239990pt}}
\pgfpathclose
\pgfusepath{fill,stroke}
\color[rgb]{0.580861,0.845663,0.253001}
\pgfpathmoveto{\pgfpoint{244.511993pt}{294.593689pt}}
\pgflineto{\pgfpoint{253.440002pt}{288.416840pt}}
\pgflineto{\pgfpoint{244.511993pt}{288.416840pt}}
\pgfpathclose
\pgfusepath{fill,stroke}
\pgfpathmoveto{\pgfpoint{244.511993pt}{294.593689pt}}
\pgflineto{\pgfpoint{253.440002pt}{294.593689pt}}
\pgflineto{\pgfpoint{253.440002pt}{288.416840pt}}
\pgfpathclose
\pgfusepath{fill,stroke}
\color[rgb]{0.622171,0.853816,0.226224}
\pgfpathmoveto{\pgfpoint{244.511993pt}{300.770538pt}}
\pgflineto{\pgfpoint{253.440002pt}{294.593689pt}}
\pgflineto{\pgfpoint{244.511993pt}{294.593689pt}}
\pgfpathclose
\pgfusepath{fill,stroke}
\pgfpathmoveto{\pgfpoint{244.511993pt}{300.770538pt}}
\pgflineto{\pgfpoint{253.440002pt}{300.770538pt}}
\pgflineto{\pgfpoint{253.440002pt}{294.593689pt}}
\pgfpathclose
\pgfusepath{fill,stroke}
\pgfpathmoveto{\pgfpoint{244.511993pt}{306.947388pt}}
\pgflineto{\pgfpoint{253.440002pt}{300.770538pt}}
\pgflineto{\pgfpoint{244.511993pt}{300.770538pt}}
\pgfpathclose
\pgfusepath{fill,stroke}
\pgfpathmoveto{\pgfpoint{244.511993pt}{306.947388pt}}
\pgflineto{\pgfpoint{253.440002pt}{306.947388pt}}
\pgflineto{\pgfpoint{253.440002pt}{300.770538pt}}
\pgfpathclose
\pgfusepath{fill,stroke}
\color[rgb]{0.580861,0.845663,0.253001}
\pgfpathmoveto{\pgfpoint{253.440002pt}{288.416840pt}}
\pgflineto{\pgfpoint{262.367981pt}{282.239990pt}}
\pgflineto{\pgfpoint{253.440002pt}{282.239990pt}}
\pgfpathclose
\pgfusepath{fill,stroke}
\pgfpathmoveto{\pgfpoint{253.440002pt}{288.416840pt}}
\pgflineto{\pgfpoint{262.367981pt}{288.416840pt}}
\pgflineto{\pgfpoint{262.367981pt}{282.239990pt}}
\pgfpathclose
\pgfusepath{fill,stroke}
\color[rgb]{0.622171,0.853816,0.226224}
\pgfpathmoveto{\pgfpoint{253.440002pt}{294.593689pt}}
\pgflineto{\pgfpoint{262.367981pt}{288.416840pt}}
\pgflineto{\pgfpoint{253.440002pt}{288.416840pt}}
\pgfpathclose
\pgfusepath{fill,stroke}
\pgfpathmoveto{\pgfpoint{253.440002pt}{294.593689pt}}
\pgflineto{\pgfpoint{262.367981pt}{294.593689pt}}
\pgflineto{\pgfpoint{262.367981pt}{288.416840pt}}
\pgfpathclose
\pgfusepath{fill,stroke}
\color[rgb]{0.664087,0.861321,0.198879}
\pgfpathmoveto{\pgfpoint{253.440002pt}{300.770538pt}}
\pgflineto{\pgfpoint{262.367981pt}{294.593689pt}}
\pgflineto{\pgfpoint{253.440002pt}{294.593689pt}}
\pgfpathclose
\pgfusepath{fill,stroke}
\pgfpathmoveto{\pgfpoint{253.440002pt}{300.770538pt}}
\pgflineto{\pgfpoint{262.367981pt}{300.770538pt}}
\pgflineto{\pgfpoint{262.367981pt}{294.593689pt}}
\pgfpathclose
\pgfusepath{fill,stroke}
\color[rgb]{0.706404,0.868206,0.171495}
\pgfpathmoveto{\pgfpoint{253.440002pt}{306.947388pt}}
\pgflineto{\pgfpoint{262.367981pt}{300.770538pt}}
\pgflineto{\pgfpoint{253.440002pt}{300.770538pt}}
\pgfpathclose
\pgfusepath{fill,stroke}
\pgfpathmoveto{\pgfpoint{253.440002pt}{306.947388pt}}
\pgflineto{\pgfpoint{262.367981pt}{306.947388pt}}
\pgflineto{\pgfpoint{262.367981pt}{300.770538pt}}
\pgfpathclose
\pgfusepath{fill,stroke}
\color[rgb]{0.622171,0.853816,0.226224}
\pgfpathmoveto{\pgfpoint{262.367981pt}{288.416840pt}}
\pgflineto{\pgfpoint{271.295990pt}{282.239990pt}}
\pgflineto{\pgfpoint{262.367981pt}{282.239990pt}}
\pgfpathclose
\pgfusepath{fill,stroke}
\pgfpathmoveto{\pgfpoint{262.367981pt}{288.416840pt}}
\pgflineto{\pgfpoint{271.295990pt}{288.416840pt}}
\pgflineto{\pgfpoint{271.295990pt}{282.239990pt}}
\pgfpathclose
\pgfusepath{fill,stroke}
\color[rgb]{0.664087,0.861321,0.198879}
\pgfpathmoveto{\pgfpoint{262.367981pt}{294.593689pt}}
\pgflineto{\pgfpoint{271.295990pt}{288.416840pt}}
\pgflineto{\pgfpoint{262.367981pt}{288.416840pt}}
\pgfpathclose
\pgfusepath{fill,stroke}
\pgfpathmoveto{\pgfpoint{262.367981pt}{294.593689pt}}
\pgflineto{\pgfpoint{271.295990pt}{294.593689pt}}
\pgflineto{\pgfpoint{271.295990pt}{288.416840pt}}
\pgfpathclose
\pgfusepath{fill,stroke}
\color[rgb]{0.706404,0.868206,0.171495}
\pgfpathmoveto{\pgfpoint{262.367981pt}{300.770538pt}}
\pgflineto{\pgfpoint{271.295990pt}{294.593689pt}}
\pgflineto{\pgfpoint{262.367981pt}{294.593689pt}}
\pgfpathclose
\pgfusepath{fill,stroke}
\pgfpathmoveto{\pgfpoint{262.367981pt}{300.770538pt}}
\pgflineto{\pgfpoint{271.295990pt}{300.770538pt}}
\pgflineto{\pgfpoint{271.295990pt}{294.593689pt}}
\pgfpathclose
\pgfusepath{fill,stroke}
\color[rgb]{0.748885,0.874522,0.145038}
\pgfpathmoveto{\pgfpoint{262.367981pt}{306.947388pt}}
\pgflineto{\pgfpoint{271.295990pt}{300.770538pt}}
\pgflineto{\pgfpoint{262.367981pt}{300.770538pt}}
\pgfpathclose
\pgfusepath{fill,stroke}
\color[rgb]{0.664087,0.861321,0.198879}
\pgfpathmoveto{\pgfpoint{271.295990pt}{288.416840pt}}
\pgflineto{\pgfpoint{280.223969pt}{282.239990pt}}
\pgflineto{\pgfpoint{271.295990pt}{282.239990pt}}
\pgfpathclose
\pgfusepath{fill,stroke}
\pgfpathmoveto{\pgfpoint{271.295990pt}{288.416840pt}}
\pgflineto{\pgfpoint{280.223969pt}{288.416840pt}}
\pgflineto{\pgfpoint{280.223969pt}{282.239990pt}}
\pgfpathclose
\pgfusepath{fill,stroke}
\color[rgb]{0.706404,0.868206,0.171495}
\pgfpathmoveto{\pgfpoint{271.295990pt}{294.593689pt}}
\pgflineto{\pgfpoint{280.223969pt}{288.416840pt}}
\pgflineto{\pgfpoint{271.295990pt}{288.416840pt}}
\pgfpathclose
\pgfusepath{fill,stroke}
\pgfpathmoveto{\pgfpoint{271.295990pt}{294.593689pt}}
\pgflineto{\pgfpoint{280.223969pt}{294.593689pt}}
\pgflineto{\pgfpoint{280.223969pt}{288.416840pt}}
\pgfpathclose
\pgfusepath{fill,stroke}
\color[rgb]{0.748885,0.874522,0.145038}
\pgfpathmoveto{\pgfpoint{271.295990pt}{300.770538pt}}
\pgflineto{\pgfpoint{280.223969pt}{294.593689pt}}
\pgflineto{\pgfpoint{271.295990pt}{294.593689pt}}
\pgfpathclose
\pgfusepath{fill,stroke}
\pgfpathmoveto{\pgfpoint{271.295990pt}{300.770538pt}}
\pgflineto{\pgfpoint{280.223969pt}{300.770538pt}}
\pgflineto{\pgfpoint{280.223969pt}{294.593689pt}}
\pgfpathclose
\pgfusepath{fill,stroke}
\color[rgb]{0.706404,0.868206,0.171495}
\pgfpathmoveto{\pgfpoint{280.223969pt}{288.416840pt}}
\pgflineto{\pgfpoint{289.151978pt}{282.239990pt}}
\pgflineto{\pgfpoint{280.223969pt}{282.239990pt}}
\pgfpathclose
\pgfusepath{fill,stroke}
\pgfpathmoveto{\pgfpoint{280.223969pt}{288.416840pt}}
\pgflineto{\pgfpoint{289.151978pt}{288.416840pt}}
\pgflineto{\pgfpoint{289.151978pt}{282.239990pt}}
\pgfpathclose
\pgfusepath{fill,stroke}
\color[rgb]{0.748885,0.874522,0.145038}
\pgfpathmoveto{\pgfpoint{280.223969pt}{294.593689pt}}
\pgflineto{\pgfpoint{289.151978pt}{288.416840pt}}
\pgflineto{\pgfpoint{280.223969pt}{288.416840pt}}
\pgfpathclose
\pgfusepath{fill,stroke}
\pgfpathmoveto{\pgfpoint{280.223969pt}{294.593689pt}}
\pgflineto{\pgfpoint{289.151978pt}{294.593689pt}}
\pgflineto{\pgfpoint{289.151978pt}{288.416840pt}}
\pgfpathclose
\pgfusepath{fill,stroke}
\color[rgb]{0.791273,0.880346,0.121291}
\pgfpathmoveto{\pgfpoint{280.223969pt}{300.770538pt}}
\pgflineto{\pgfpoint{289.151978pt}{294.593689pt}}
\pgflineto{\pgfpoint{280.223969pt}{294.593689pt}}
\pgfpathclose
\pgfusepath{fill,stroke}
\pgfpathmoveto{\pgfpoint{280.223969pt}{300.770538pt}}
\pgflineto{\pgfpoint{289.151978pt}{300.770538pt}}
\pgflineto{\pgfpoint{289.151978pt}{294.593689pt}}
\pgfpathclose
\pgfusepath{fill,stroke}
\pgfpathmoveto{\pgfpoint{289.151978pt}{294.593689pt}}
\pgflineto{\pgfpoint{298.079987pt}{288.416840pt}}
\pgflineto{\pgfpoint{289.151978pt}{288.416840pt}}
\pgfpathclose
\pgfusepath{fill,stroke}
\pgfpathmoveto{\pgfpoint{289.151978pt}{294.593689pt}}
\pgflineto{\pgfpoint{298.079987pt}{294.593689pt}}
\pgflineto{\pgfpoint{298.079987pt}{288.416840pt}}
\pgfpathclose
\pgfusepath{fill,stroke}
\color[rgb]{0.833302,0.885780,0.103326}
\pgfpathmoveto{\pgfpoint{289.151978pt}{300.770538pt}}
\pgflineto{\pgfpoint{298.079987pt}{294.593689pt}}
\pgflineto{\pgfpoint{289.151978pt}{294.593689pt}}
\pgfpathclose
\pgfusepath{fill,stroke}
\pgfpathmoveto{\pgfpoint{289.151978pt}{300.770538pt}}
\pgflineto{\pgfpoint{298.079987pt}{300.770538pt}}
\pgflineto{\pgfpoint{298.079987pt}{294.593689pt}}
\pgfpathclose
\pgfusepath{fill,stroke}
\pgfpathmoveto{\pgfpoint{298.079987pt}{294.593689pt}}
\pgflineto{\pgfpoint{307.007965pt}{288.416840pt}}
\pgflineto{\pgfpoint{298.079987pt}{288.416840pt}}
\pgfpathclose
\pgfusepath{fill,stroke}
\pgfpathmoveto{\pgfpoint{298.079987pt}{294.593689pt}}
\pgflineto{\pgfpoint{307.007965pt}{294.593689pt}}
\pgflineto{\pgfpoint{307.007965pt}{288.416840pt}}
\pgfpathclose
\pgfusepath{fill,stroke}
\color[rgb]{0.874718,0.890945,0.095351}
\pgfpathmoveto{\pgfpoint{298.079987pt}{300.770538pt}}
\pgflineto{\pgfpoint{307.007965pt}{294.593689pt}}
\pgflineto{\pgfpoint{298.079987pt}{294.593689pt}}
\pgfpathclose
\pgfusepath{fill,stroke}
\pgfpathmoveto{\pgfpoint{307.007965pt}{294.593689pt}}
\pgflineto{\pgfpoint{315.935974pt}{288.416840pt}}
\pgflineto{\pgfpoint{307.007965pt}{288.416840pt}}
\pgfpathclose
\pgfusepath{fill,stroke}
\pgfpathmoveto{\pgfpoint{307.007965pt}{294.593689pt}}
\pgflineto{\pgfpoint{315.935974pt}{294.593689pt}}
\pgflineto{\pgfpoint{315.935974pt}{288.416840pt}}
\pgfpathclose
\pgfusepath{fill,stroke}
\color[rgb]{0.208030,0.718701,0.472873}
\pgfpathmoveto{\pgfpoint{208.799988pt}{245.178955pt}}
\pgflineto{\pgfpoint{217.727982pt}{245.178955pt}}
\pgflineto{\pgfpoint{217.727982pt}{239.002106pt}}
\pgfpathclose
\pgfusepath{fill,stroke}
\color[rgb]{0.233127,0.732406,0.459106}
\pgfpathmoveto{\pgfpoint{208.799988pt}{251.355804pt}}
\pgflineto{\pgfpoint{217.727982pt}{245.178955pt}}
\pgflineto{\pgfpoint{208.799988pt}{245.178955pt}}
\pgfpathclose
\pgfusepath{fill,stroke}
\pgfpathmoveto{\pgfpoint{208.799988pt}{251.355804pt}}
\pgflineto{\pgfpoint{217.727982pt}{251.355804pt}}
\pgflineto{\pgfpoint{217.727982pt}{245.178955pt}}
\pgfpathclose
\pgfusepath{fill,stroke}
\pgfpathmoveto{\pgfpoint{217.727982pt}{245.178955pt}}
\pgflineto{\pgfpoint{226.655975pt}{239.002106pt}}
\pgflineto{\pgfpoint{217.727982pt}{239.002106pt}}
\pgfpathclose
\pgfusepath{fill,stroke}
\pgfpathmoveto{\pgfpoint{217.727982pt}{245.178955pt}}
\pgflineto{\pgfpoint{226.655975pt}{245.178955pt}}
\pgflineto{\pgfpoint{226.655975pt}{239.002106pt}}
\pgfpathclose
\pgfusepath{fill,stroke}
\color[rgb]{0.260531,0.745802,0.444096}
\pgfpathmoveto{\pgfpoint{217.727982pt}{251.355804pt}}
\pgflineto{\pgfpoint{226.655975pt}{245.178955pt}}
\pgflineto{\pgfpoint{217.727982pt}{245.178955pt}}
\pgfpathclose
\pgfusepath{fill,stroke}
\pgfpathmoveto{\pgfpoint{217.727982pt}{251.355804pt}}
\pgflineto{\pgfpoint{226.655975pt}{251.355804pt}}
\pgflineto{\pgfpoint{226.655975pt}{245.178955pt}}
\pgfpathclose
\pgfusepath{fill,stroke}
\color[rgb]{0.290001,0.758846,0.427826}
\pgfpathmoveto{\pgfpoint{217.727982pt}{257.532623pt}}
\pgflineto{\pgfpoint{226.655975pt}{251.355804pt}}
\pgflineto{\pgfpoint{217.727982pt}{251.355804pt}}
\pgfpathclose
\pgfusepath{fill,stroke}
\color[rgb]{0.233127,0.732406,0.459106}
\pgfpathmoveto{\pgfpoint{226.655975pt}{239.002106pt}}
\pgflineto{\pgfpoint{235.583969pt}{239.002106pt}}
\pgflineto{\pgfpoint{235.583969pt}{232.825272pt}}
\pgfpathclose
\pgfusepath{fill,stroke}
\color[rgb]{0.260531,0.745802,0.444096}
\pgfpathmoveto{\pgfpoint{226.655975pt}{245.178955pt}}
\pgflineto{\pgfpoint{235.583969pt}{239.002106pt}}
\pgflineto{\pgfpoint{226.655975pt}{239.002106pt}}
\pgfpathclose
\pgfusepath{fill,stroke}
\color[rgb]{0.122046,0.632107,0.530848}
\pgfpathmoveto{\pgfpoint{190.943985pt}{220.471588pt}}
\pgflineto{\pgfpoint{199.871979pt}{220.471588pt}}
\pgflineto{\pgfpoint{199.871979pt}{214.294739pt}}
\pgfpathclose
\pgfusepath{fill,stroke}
\color[rgb]{0.127668,0.646882,0.523924}
\pgfpathmoveto{\pgfpoint{190.943985pt}{226.648422pt}}
\pgflineto{\pgfpoint{199.871979pt}{220.471588pt}}
\pgflineto{\pgfpoint{190.943985pt}{220.471588pt}}
\pgfpathclose
\pgfusepath{fill,stroke}
\pgfpathmoveto{\pgfpoint{190.943985pt}{226.648422pt}}
\pgflineto{\pgfpoint{199.871979pt}{226.648422pt}}
\pgflineto{\pgfpoint{199.871979pt}{220.471588pt}}
\pgfpathclose
\pgfusepath{fill,stroke}
\pgfpathmoveto{\pgfpoint{199.871979pt}{220.471588pt}}
\pgflineto{\pgfpoint{208.799988pt}{214.294739pt}}
\pgflineto{\pgfpoint{199.871979pt}{214.294739pt}}
\pgfpathclose
\pgfusepath{fill,stroke}
\pgfpathmoveto{\pgfpoint{199.871979pt}{220.471588pt}}
\pgflineto{\pgfpoint{208.799988pt}{220.471588pt}}
\pgflineto{\pgfpoint{208.799988pt}{214.294739pt}}
\pgfpathclose
\pgfusepath{fill,stroke}
\color[rgb]{0.136835,0.661563,0.515967}
\pgfpathmoveto{\pgfpoint{199.871979pt}{226.648422pt}}
\pgflineto{\pgfpoint{208.799988pt}{220.471588pt}}
\pgflineto{\pgfpoint{199.871979pt}{220.471588pt}}
\pgfpathclose
\pgfusepath{fill,stroke}
\pgfpathmoveto{\pgfpoint{199.871979pt}{226.648422pt}}
\pgflineto{\pgfpoint{208.799988pt}{226.648422pt}}
\pgflineto{\pgfpoint{208.799988pt}{220.471588pt}}
\pgfpathclose
\pgfusepath{fill,stroke}
\color[rgb]{0.149643,0.676120,0.506924}
\pgfpathmoveto{\pgfpoint{199.871979pt}{232.825272pt}}
\pgflineto{\pgfpoint{208.799988pt}{226.648422pt}}
\pgflineto{\pgfpoint{199.871979pt}{226.648422pt}}
\pgfpathclose
\pgfusepath{fill,stroke}
\pgfpathmoveto{\pgfpoint{199.871979pt}{232.825272pt}}
\pgflineto{\pgfpoint{208.799988pt}{232.825272pt}}
\pgflineto{\pgfpoint{208.799988pt}{226.648422pt}}
\pgfpathclose
\pgfusepath{fill,stroke}
\color[rgb]{0.165967,0.690519,0.496752}
\pgfpathmoveto{\pgfpoint{199.871979pt}{239.002106pt}}
\pgflineto{\pgfpoint{208.799988pt}{232.825272pt}}
\pgflineto{\pgfpoint{199.871979pt}{232.825272pt}}
\pgfpathclose
\pgfusepath{fill,stroke}
\color[rgb]{0.136835,0.661563,0.515967}
\pgfpathmoveto{\pgfpoint{208.799988pt}{214.294739pt}}
\pgflineto{\pgfpoint{217.727982pt}{214.294739pt}}
\pgflineto{\pgfpoint{217.727982pt}{208.117905pt}}
\pgfpathclose
\pgfusepath{fill,stroke}
\pgfpathmoveto{\pgfpoint{208.799988pt}{220.471588pt}}
\pgflineto{\pgfpoint{217.727982pt}{214.294739pt}}
\pgflineto{\pgfpoint{208.799988pt}{214.294739pt}}
\pgfpathclose
\pgfusepath{fill,stroke}
\color[rgb]{0.135833,0.542750,0.554289}
\pgfpathmoveto{\pgfpoint{173.087997pt}{189.587372pt}}
\pgflineto{\pgfpoint{182.015991pt}{189.587372pt}}
\pgflineto{\pgfpoint{182.015991pt}{183.410522pt}}
\pgfpathclose
\pgfusepath{fill,stroke}
\color[rgb]{0.130582,0.557652,0.552176}
\pgfpathmoveto{\pgfpoint{173.087997pt}{195.764206pt}}
\pgflineto{\pgfpoint{182.015991pt}{189.587372pt}}
\pgflineto{\pgfpoint{173.087997pt}{189.587372pt}}
\pgfpathclose
\pgfusepath{fill,stroke}
\pgfpathmoveto{\pgfpoint{173.087997pt}{195.764206pt}}
\pgflineto{\pgfpoint{182.015991pt}{195.764206pt}}
\pgflineto{\pgfpoint{182.015991pt}{189.587372pt}}
\pgfpathclose
\pgfusepath{fill,stroke}
\color[rgb]{0.125898,0.572563,0.549445}
\pgfpathmoveto{\pgfpoint{173.087997pt}{201.941055pt}}
\pgflineto{\pgfpoint{182.015991pt}{195.764206pt}}
\pgflineto{\pgfpoint{173.087997pt}{195.764206pt}}
\pgfpathclose
\pgfusepath{fill,stroke}
\pgfpathmoveto{\pgfpoint{173.087997pt}{201.941055pt}}
\pgflineto{\pgfpoint{182.015991pt}{201.941055pt}}
\pgflineto{\pgfpoint{182.015991pt}{195.764206pt}}
\pgfpathclose
\pgfusepath{fill,stroke}
\color[rgb]{0.130582,0.557652,0.552176}
\pgfpathmoveto{\pgfpoint{182.015991pt}{189.587372pt}}
\pgflineto{\pgfpoint{190.943985pt}{183.410522pt}}
\pgflineto{\pgfpoint{182.015991pt}{183.410522pt}}
\pgfpathclose
\pgfusepath{fill,stroke}
\pgfpathmoveto{\pgfpoint{182.015991pt}{189.587372pt}}
\pgflineto{\pgfpoint{190.943985pt}{189.587372pt}}
\pgflineto{\pgfpoint{190.943985pt}{183.410522pt}}
\pgfpathclose
\pgfusepath{fill,stroke}
\color[rgb]{0.125898,0.572563,0.549445}
\pgfpathmoveto{\pgfpoint{182.015991pt}{195.764206pt}}
\pgflineto{\pgfpoint{190.943985pt}{189.587372pt}}
\pgflineto{\pgfpoint{182.015991pt}{189.587372pt}}
\pgfpathclose
\pgfusepath{fill,stroke}
\pgfpathmoveto{\pgfpoint{182.015991pt}{195.764206pt}}
\pgflineto{\pgfpoint{190.943985pt}{195.764206pt}}
\pgflineto{\pgfpoint{190.943985pt}{189.587372pt}}
\pgfpathclose
\pgfusepath{fill,stroke}
\color[rgb]{0.122163,0.587476,0.546023}
\pgfpathmoveto{\pgfpoint{182.015991pt}{201.941055pt}}
\pgflineto{\pgfpoint{190.943985pt}{195.764206pt}}
\pgflineto{\pgfpoint{182.015991pt}{195.764206pt}}
\pgfpathclose
\pgfusepath{fill,stroke}
\pgfpathmoveto{\pgfpoint{182.015991pt}{201.941055pt}}
\pgflineto{\pgfpoint{190.943985pt}{201.941055pt}}
\pgflineto{\pgfpoint{190.943985pt}{195.764206pt}}
\pgfpathclose
\pgfusepath{fill,stroke}
\pgfpathmoveto{\pgfpoint{182.015991pt}{208.117905pt}}
\pgflineto{\pgfpoint{190.943985pt}{201.941055pt}}
\pgflineto{\pgfpoint{182.015991pt}{201.941055pt}}
\pgfpathclose
\pgfusepath{fill,stroke}
\pgfpathmoveto{\pgfpoint{182.015991pt}{208.117905pt}}
\pgflineto{\pgfpoint{190.943985pt}{208.117905pt}}
\pgflineto{\pgfpoint{190.943985pt}{201.941055pt}}
\pgfpathclose
\pgfusepath{fill,stroke}
\color[rgb]{0.119872,0.602382,0.541831}
\pgfpathmoveto{\pgfpoint{182.015991pt}{214.294739pt}}
\pgflineto{\pgfpoint{190.943985pt}{208.117905pt}}
\pgflineto{\pgfpoint{182.015991pt}{208.117905pt}}
\pgfpathclose
\pgfusepath{fill,stroke}
\color[rgb]{0.130582,0.557652,0.552176}
\pgfpathmoveto{\pgfpoint{190.943985pt}{183.410522pt}}
\pgflineto{\pgfpoint{199.871979pt}{183.410522pt}}
\pgflineto{\pgfpoint{199.871979pt}{177.233673pt}}
\pgfpathclose
\pgfusepath{fill,stroke}
\color[rgb]{0.125898,0.572563,0.549445}
\pgfpathmoveto{\pgfpoint{190.943985pt}{189.587372pt}}
\pgflineto{\pgfpoint{199.871979pt}{183.410522pt}}
\pgflineto{\pgfpoint{190.943985pt}{183.410522pt}}
\pgfpathclose
\pgfusepath{fill,stroke}
\pgfpathmoveto{\pgfpoint{190.943985pt}{189.587372pt}}
\pgflineto{\pgfpoint{199.871979pt}{189.587372pt}}
\pgflineto{\pgfpoint{199.871979pt}{183.410522pt}}
\pgfpathclose
\pgfusepath{fill,stroke}
\color[rgb]{0.122163,0.587476,0.546023}
\pgfpathmoveto{\pgfpoint{190.943985pt}{195.764206pt}}
\pgflineto{\pgfpoint{199.871979pt}{189.587372pt}}
\pgflineto{\pgfpoint{190.943985pt}{189.587372pt}}
\pgfpathclose
\pgfusepath{fill,stroke}
\color[rgb]{0.170958,0.453063,0.557974}
\pgfpathmoveto{\pgfpoint{155.231979pt}{164.880005pt}}
\pgflineto{\pgfpoint{164.160004pt}{164.880005pt}}
\pgflineto{\pgfpoint{164.160004pt}{158.703156pt}}
\pgfpathclose
\pgfusepath{fill,stroke}
\color[rgb]{0.164833,0.468130,0.558143}
\pgfpathmoveto{\pgfpoint{155.231979pt}{171.056854pt}}
\pgflineto{\pgfpoint{164.160004pt}{164.880005pt}}
\pgflineto{\pgfpoint{155.231979pt}{164.880005pt}}
\pgfpathclose
\pgfusepath{fill,stroke}
\pgfpathmoveto{\pgfpoint{155.231979pt}{171.056854pt}}
\pgflineto{\pgfpoint{164.160004pt}{171.056854pt}}
\pgflineto{\pgfpoint{164.160004pt}{164.880005pt}}
\pgfpathclose
\pgfusepath{fill,stroke}
\pgfpathmoveto{\pgfpoint{164.160004pt}{164.880005pt}}
\pgflineto{\pgfpoint{173.087997pt}{158.703156pt}}
\pgflineto{\pgfpoint{164.160004pt}{158.703156pt}}
\pgfpathclose
\pgfusepath{fill,stroke}
\pgfpathmoveto{\pgfpoint{164.160004pt}{164.880005pt}}
\pgflineto{\pgfpoint{173.087997pt}{164.880005pt}}
\pgflineto{\pgfpoint{173.087997pt}{158.703156pt}}
\pgfpathclose
\pgfusepath{fill,stroke}
\color[rgb]{0.158845,0.483117,0.558059}
\pgfpathmoveto{\pgfpoint{164.160004pt}{171.056854pt}}
\pgflineto{\pgfpoint{173.087997pt}{164.880005pt}}
\pgflineto{\pgfpoint{164.160004pt}{164.880005pt}}
\pgfpathclose
\pgfusepath{fill,stroke}
\pgfpathmoveto{\pgfpoint{164.160004pt}{171.056854pt}}
\pgflineto{\pgfpoint{173.087997pt}{171.056854pt}}
\pgflineto{\pgfpoint{173.087997pt}{164.880005pt}}
\pgfpathclose
\pgfusepath{fill,stroke}
\color[rgb]{0.152951,0.498053,0.557685}
\pgfpathmoveto{\pgfpoint{164.160004pt}{177.233673pt}}
\pgflineto{\pgfpoint{173.087997pt}{171.056854pt}}
\pgflineto{\pgfpoint{164.160004pt}{171.056854pt}}
\pgfpathclose
\pgfusepath{fill,stroke}
\pgfpathmoveto{\pgfpoint{164.160004pt}{177.233673pt}}
\pgflineto{\pgfpoint{173.087997pt}{177.233673pt}}
\pgflineto{\pgfpoint{173.087997pt}{171.056854pt}}
\pgfpathclose
\pgfusepath{fill,stroke}
\color[rgb]{0.147132,0.512959,0.556973}
\pgfpathmoveto{\pgfpoint{164.160004pt}{183.410522pt}}
\pgflineto{\pgfpoint{173.087997pt}{177.233673pt}}
\pgflineto{\pgfpoint{164.160004pt}{177.233673pt}}
\pgfpathclose
\pgfusepath{fill,stroke}
\color[rgb]{0.158845,0.483117,0.558059}
\pgfpathmoveto{\pgfpoint{173.087997pt}{158.703156pt}}
\pgflineto{\pgfpoint{182.015991pt}{158.703156pt}}
\pgflineto{\pgfpoint{182.015991pt}{152.526306pt}}
\pgfpathclose
\pgfusepath{fill,stroke}
\color[rgb]{0.152951,0.498053,0.557685}
\pgfpathmoveto{\pgfpoint{173.087997pt}{164.880005pt}}
\pgflineto{\pgfpoint{182.015991pt}{158.703156pt}}
\pgflineto{\pgfpoint{173.087997pt}{158.703156pt}}
\pgfpathclose
\pgfusepath{fill,stroke}
\pgfpathmoveto{\pgfpoint{173.087997pt}{164.880005pt}}
\pgflineto{\pgfpoint{182.015991pt}{164.880005pt}}
\pgflineto{\pgfpoint{182.015991pt}{158.703156pt}}
\pgfpathclose
\pgfusepath{fill,stroke}
\pgfpathmoveto{\pgfpoint{173.087997pt}{171.056854pt}}
\pgflineto{\pgfpoint{182.015991pt}{164.880005pt}}
\pgflineto{\pgfpoint{173.087997pt}{164.880005pt}}
\pgfpathclose
\pgfusepath{fill,stroke}
\color[rgb]{0.205079,0.375366,0.553493}
\pgfpathmoveto{\pgfpoint{137.376007pt}{140.172638pt}}
\pgflineto{\pgfpoint{146.303986pt}{140.172638pt}}
\pgflineto{\pgfpoint{146.303986pt}{133.995789pt}}
\pgfpathclose
\pgfusepath{fill,stroke}
\color[rgb]{0.197722,0.391341,0.554953}
\pgfpathmoveto{\pgfpoint{137.376007pt}{146.349472pt}}
\pgflineto{\pgfpoint{146.303986pt}{140.172638pt}}
\pgflineto{\pgfpoint{137.376007pt}{140.172638pt}}
\pgfpathclose
\pgfusepath{fill,stroke}
\pgfpathmoveto{\pgfpoint{137.376007pt}{146.349472pt}}
\pgflineto{\pgfpoint{146.303986pt}{146.349472pt}}
\pgflineto{\pgfpoint{146.303986pt}{140.172638pt}}
\pgfpathclose
\pgfusepath{fill,stroke}
\pgfpathmoveto{\pgfpoint{146.303986pt}{140.172638pt}}
\pgflineto{\pgfpoint{155.231979pt}{133.995789pt}}
\pgflineto{\pgfpoint{146.303986pt}{133.995789pt}}
\pgfpathclose
\pgfusepath{fill,stroke}
\pgfpathmoveto{\pgfpoint{146.303986pt}{140.172638pt}}
\pgflineto{\pgfpoint{155.231979pt}{140.172638pt}}
\pgflineto{\pgfpoint{155.231979pt}{133.995789pt}}
\pgfpathclose
\pgfusepath{fill,stroke}
\color[rgb]{0.190631,0.407061,0.556089}
\pgfpathmoveto{\pgfpoint{146.303986pt}{146.349472pt}}
\pgflineto{\pgfpoint{155.231979pt}{140.172638pt}}
\pgflineto{\pgfpoint{146.303986pt}{140.172638pt}}
\pgfpathclose
\pgfusepath{fill,stroke}
\pgfpathmoveto{\pgfpoint{146.303986pt}{146.349472pt}}
\pgflineto{\pgfpoint{155.231979pt}{146.349472pt}}
\pgflineto{\pgfpoint{155.231979pt}{140.172638pt}}
\pgfpathclose
\pgfusepath{fill,stroke}
\color[rgb]{0.183819,0.422564,0.556952}
\pgfpathmoveto{\pgfpoint{146.303986pt}{152.526306pt}}
\pgflineto{\pgfpoint{155.231979pt}{146.349472pt}}
\pgflineto{\pgfpoint{146.303986pt}{146.349472pt}}
\pgfpathclose
\pgfusepath{fill,stroke}
\pgfpathmoveto{\pgfpoint{146.303986pt}{152.526306pt}}
\pgflineto{\pgfpoint{155.231979pt}{152.526306pt}}
\pgflineto{\pgfpoint{155.231979pt}{146.349472pt}}
\pgfpathclose
\pgfusepath{fill,stroke}
\pgfpathmoveto{\pgfpoint{146.303986pt}{158.703156pt}}
\pgflineto{\pgfpoint{155.231979pt}{152.526306pt}}
\pgflineto{\pgfpoint{146.303986pt}{152.526306pt}}
\pgfpathclose
\pgfusepath{fill,stroke}
\color[rgb]{0.197722,0.391341,0.554953}
\pgfpathmoveto{\pgfpoint{155.231979pt}{133.995789pt}}
\pgflineto{\pgfpoint{164.160004pt}{133.995789pt}}
\pgflineto{\pgfpoint{164.160004pt}{127.818947pt}}
\pgfpathclose
\pgfusepath{fill,stroke}
\color[rgb]{0.190631,0.407061,0.556089}
\pgfpathmoveto{\pgfpoint{155.231979pt}{140.172638pt}}
\pgflineto{\pgfpoint{164.160004pt}{133.995789pt}}
\pgflineto{\pgfpoint{155.231979pt}{133.995789pt}}
\pgfpathclose
\pgfusepath{fill,stroke}
\pgfpathmoveto{\pgfpoint{155.231979pt}{140.172638pt}}
\pgflineto{\pgfpoint{164.160004pt}{140.172638pt}}
\pgflineto{\pgfpoint{164.160004pt}{133.995789pt}}
\pgfpathclose
\pgfusepath{fill,stroke}
\color[rgb]{0.183819,0.422564,0.556952}
\pgfpathmoveto{\pgfpoint{155.231979pt}{146.349472pt}}
\pgflineto{\pgfpoint{164.160004pt}{140.172638pt}}
\pgflineto{\pgfpoint{155.231979pt}{140.172638pt}}
\pgfpathclose
\pgfusepath{fill,stroke}
\color[rgb]{0.251099,0.272573,0.532522}
\pgfpathmoveto{\pgfpoint{119.519989pt}{115.465263pt}}
\pgflineto{\pgfpoint{128.447998pt}{115.465263pt}}
\pgflineto{\pgfpoint{128.447998pt}{109.288422pt}}
\pgfpathclose
\pgfusepath{fill,stroke}
\color[rgb]{0.243733,0.290620,0.538097}
\pgfpathmoveto{\pgfpoint{119.519989pt}{121.642097pt}}
\pgflineto{\pgfpoint{128.447998pt}{115.465263pt}}
\pgflineto{\pgfpoint{119.519989pt}{115.465263pt}}
\pgfpathclose
\pgfusepath{fill,stroke}
\pgfpathmoveto{\pgfpoint{119.519989pt}{121.642097pt}}
\pgflineto{\pgfpoint{128.447998pt}{121.642097pt}}
\pgflineto{\pgfpoint{128.447998pt}{115.465263pt}}
\pgfpathclose
\pgfusepath{fill,stroke}
\pgfpathmoveto{\pgfpoint{128.447998pt}{115.465263pt}}
\pgflineto{\pgfpoint{137.376007pt}{109.288422pt}}
\pgflineto{\pgfpoint{128.447998pt}{109.288422pt}}
\pgfpathclose
\pgfusepath{fill,stroke}
\pgfpathmoveto{\pgfpoint{128.447998pt}{115.465263pt}}
\pgflineto{\pgfpoint{137.376007pt}{115.465263pt}}
\pgflineto{\pgfpoint{137.376007pt}{109.288422pt}}
\pgfpathclose
\pgfusepath{fill,stroke}
\color[rgb]{0.236073,0.308291,0.542652}
\pgfpathmoveto{\pgfpoint{128.447998pt}{121.642097pt}}
\pgflineto{\pgfpoint{137.376007pt}{115.465263pt}}
\pgflineto{\pgfpoint{128.447998pt}{115.465263pt}}
\pgfpathclose
\pgfusepath{fill,stroke}
\pgfpathmoveto{\pgfpoint{128.447998pt}{121.642097pt}}
\pgflineto{\pgfpoint{137.376007pt}{121.642097pt}}
\pgflineto{\pgfpoint{137.376007pt}{115.465263pt}}
\pgfpathclose
\pgfusepath{fill,stroke}
\color[rgb]{0.228263,0.325586,0.546335}
\pgfpathmoveto{\pgfpoint{128.447998pt}{127.818947pt}}
\pgflineto{\pgfpoint{137.376007pt}{121.642097pt}}
\pgflineto{\pgfpoint{128.447998pt}{121.642097pt}}
\pgfpathclose
\pgfusepath{fill,stroke}
\pgfpathmoveto{\pgfpoint{128.447998pt}{127.818947pt}}
\pgflineto{\pgfpoint{137.376007pt}{127.818947pt}}
\pgflineto{\pgfpoint{137.376007pt}{121.642097pt}}
\pgfpathclose
\pgfusepath{fill,stroke}
\color[rgb]{0.220425,0.342517,0.549287}
\pgfpathmoveto{\pgfpoint{128.447998pt}{133.995789pt}}
\pgflineto{\pgfpoint{137.376007pt}{127.818947pt}}
\pgflineto{\pgfpoint{128.447998pt}{127.818947pt}}
\pgfpathclose
\pgfusepath{fill,stroke}
\color[rgb]{0.236073,0.308291,0.542652}
\pgfpathmoveto{\pgfpoint{137.376007pt}{109.288422pt}}
\pgflineto{\pgfpoint{146.303986pt}{109.288422pt}}
\pgflineto{\pgfpoint{146.303986pt}{103.111580pt}}
\pgfpathclose
\pgfusepath{fill,stroke}
\pgfpathmoveto{\pgfpoint{137.376007pt}{115.465263pt}}
\pgflineto{\pgfpoint{146.303986pt}{109.288422pt}}
\pgflineto{\pgfpoint{137.376007pt}{109.288422pt}}
\pgfpathclose
\pgfusepath{fill,stroke}
\color[rgb]{0.281231,0.157480,0.470434}
\pgfpathmoveto{\pgfpoint{101.664001pt}{84.581039pt}}
\pgflineto{\pgfpoint{110.591980pt}{84.581039pt}}
\pgflineto{\pgfpoint{110.591980pt}{78.404205pt}}
\pgfpathclose
\pgfusepath{fill,stroke}
\color[rgb]{0.278516,0.177348,0.484654}
\pgfpathmoveto{\pgfpoint{101.664001pt}{90.757896pt}}
\pgflineto{\pgfpoint{110.591980pt}{84.581039pt}}
\pgflineto{\pgfpoint{101.664001pt}{84.581039pt}}
\pgfpathclose
\pgfusepath{fill,stroke}
\pgfpathmoveto{\pgfpoint{101.664001pt}{90.757896pt}}
\pgflineto{\pgfpoint{110.591980pt}{90.757896pt}}
\pgflineto{\pgfpoint{110.591980pt}{84.581039pt}}
\pgfpathclose
\pgfusepath{fill,stroke}
\color[rgb]{0.274736,0.196969,0.497250}
\pgfpathmoveto{\pgfpoint{101.664001pt}{96.934731pt}}
\pgflineto{\pgfpoint{110.591980pt}{90.757896pt}}
\pgflineto{\pgfpoint{101.664001pt}{90.757896pt}}
\pgfpathclose
\pgfusepath{fill,stroke}
\pgfpathmoveto{\pgfpoint{101.664001pt}{96.934731pt}}
\pgflineto{\pgfpoint{110.591980pt}{96.934731pt}}
\pgflineto{\pgfpoint{110.591980pt}{90.757896pt}}
\pgfpathclose
\pgfusepath{fill,stroke}
\color[rgb]{0.278516,0.177348,0.484654}
\pgfpathmoveto{\pgfpoint{110.591980pt}{84.581039pt}}
\pgflineto{\pgfpoint{119.519989pt}{78.404205pt}}
\pgflineto{\pgfpoint{110.591980pt}{78.404205pt}}
\pgfpathclose
\pgfusepath{fill,stroke}
\pgfpathmoveto{\pgfpoint{110.591980pt}{84.581039pt}}
\pgflineto{\pgfpoint{119.519989pt}{84.581039pt}}
\pgflineto{\pgfpoint{119.519989pt}{78.404205pt}}
\pgfpathclose
\pgfusepath{fill,stroke}
\color[rgb]{0.274736,0.196969,0.497250}
\pgfpathmoveto{\pgfpoint{110.591980pt}{90.757896pt}}
\pgflineto{\pgfpoint{119.519989pt}{84.581039pt}}
\pgflineto{\pgfpoint{110.591980pt}{84.581039pt}}
\pgfpathclose
\pgfusepath{fill,stroke}
\pgfpathmoveto{\pgfpoint{110.591980pt}{90.757896pt}}
\pgflineto{\pgfpoint{119.519989pt}{90.757896pt}}
\pgflineto{\pgfpoint{119.519989pt}{84.581039pt}}
\pgfpathclose
\pgfusepath{fill,stroke}
\color[rgb]{0.269982,0.216330,0.508255}
\pgfpathmoveto{\pgfpoint{110.591980pt}{96.934731pt}}
\pgflineto{\pgfpoint{119.519989pt}{90.757896pt}}
\pgflineto{\pgfpoint{110.591980pt}{90.757896pt}}
\pgfpathclose
\pgfusepath{fill,stroke}
\pgfpathmoveto{\pgfpoint{110.591980pt}{96.934731pt}}
\pgflineto{\pgfpoint{119.519989pt}{96.934731pt}}
\pgflineto{\pgfpoint{119.519989pt}{90.757896pt}}
\pgfpathclose
\pgfusepath{fill,stroke}
\color[rgb]{0.264369,0.235405,0.517732}
\pgfpathmoveto{\pgfpoint{110.591980pt}{103.111580pt}}
\pgflineto{\pgfpoint{119.519989pt}{96.934731pt}}
\pgflineto{\pgfpoint{110.591980pt}{96.934731pt}}
\pgfpathclose
\pgfusepath{fill,stroke}
\pgfpathmoveto{\pgfpoint{110.591980pt}{103.111580pt}}
\pgflineto{\pgfpoint{119.519989pt}{103.111580pt}}
\pgflineto{\pgfpoint{119.519989pt}{96.934731pt}}
\pgfpathclose
\pgfusepath{fill,stroke}
\pgfpathmoveto{\pgfpoint{110.591980pt}{109.288422pt}}
\pgflineto{\pgfpoint{119.519989pt}{103.111580pt}}
\pgflineto{\pgfpoint{110.591980pt}{103.111580pt}}
\pgfpathclose
\pgfusepath{fill,stroke}
\color[rgb]{0.278516,0.177348,0.484654}
\pgfpathmoveto{\pgfpoint{119.519989pt}{78.404205pt}}
\pgflineto{\pgfpoint{128.447998pt}{78.404205pt}}
\pgflineto{\pgfpoint{128.447998pt}{72.227356pt}}
\pgfpathclose
\pgfusepath{fill,stroke}
\color[rgb]{0.274736,0.196969,0.497250}
\pgfpathmoveto{\pgfpoint{119.519989pt}{84.581039pt}}
\pgflineto{\pgfpoint{128.447998pt}{78.404205pt}}
\pgflineto{\pgfpoint{119.519989pt}{78.404205pt}}
\pgfpathclose
\pgfusepath{fill,stroke}
\pgfpathmoveto{\pgfpoint{119.519989pt}{84.581039pt}}
\pgflineto{\pgfpoint{128.447998pt}{84.581039pt}}
\pgflineto{\pgfpoint{128.447998pt}{78.404205pt}}
\pgfpathclose
\pgfusepath{fill,stroke}
\color[rgb]{0.269982,0.216330,0.508255}
\pgfpathmoveto{\pgfpoint{119.519989pt}{90.757896pt}}
\pgflineto{\pgfpoint{128.447998pt}{84.581039pt}}
\pgflineto{\pgfpoint{119.519989pt}{84.581039pt}}
\pgfpathclose
\pgfusepath{fill,stroke}
\color[rgb]{0.267004,0.004874,0.329415}
\pgfpathmoveto{\pgfpoint{74.880005pt}{53.696838pt}}
\pgflineto{\pgfpoint{83.807999pt}{47.519989pt}}
\pgflineto{\pgfpoint{74.880005pt}{47.519989pt}}
\pgfpathclose
\pgfusepath{fill,stroke}
\pgfpathmoveto{\pgfpoint{74.880005pt}{53.696838pt}}
\pgflineto{\pgfpoint{83.807999pt}{53.696838pt}}
\pgflineto{\pgfpoint{83.807999pt}{47.519989pt}}
\pgfpathclose
\pgfusepath{fill,stroke}
\pgfpathmoveto{\pgfpoint{74.880005pt}{59.873672pt}}
\pgflineto{\pgfpoint{83.807999pt}{53.696838pt}}
\pgflineto{\pgfpoint{74.880005pt}{53.696838pt}}
\pgfpathclose
\pgfusepath{fill,stroke}
\pgfpathmoveto{\pgfpoint{74.880005pt}{59.873672pt}}
\pgflineto{\pgfpoint{83.807999pt}{59.873672pt}}
\pgflineto{\pgfpoint{83.807999pt}{53.696838pt}}
\pgfpathclose
\pgfusepath{fill,stroke}
\color[rgb]{0.272652,0.025846,0.353367}
\pgfpathmoveto{\pgfpoint{74.880005pt}{66.050522pt}}
\pgflineto{\pgfpoint{83.807999pt}{59.873672pt}}
\pgflineto{\pgfpoint{74.880005pt}{59.873672pt}}
\pgfpathclose
\pgfusepath{fill,stroke}
\pgfpathmoveto{\pgfpoint{74.880005pt}{66.050522pt}}
\pgflineto{\pgfpoint{83.807999pt}{66.050522pt}}
\pgflineto{\pgfpoint{83.807999pt}{59.873672pt}}
\pgfpathclose
\pgfusepath{fill,stroke}
\color[rgb]{0.277106,0.050914,0.376236}
\pgfpathmoveto{\pgfpoint{74.880005pt}{72.227356pt}}
\pgflineto{\pgfpoint{83.807999pt}{66.050522pt}}
\pgflineto{\pgfpoint{74.880005pt}{66.050522pt}}
\pgfpathclose
\pgfusepath{fill,stroke}
\pgfpathmoveto{\pgfpoint{74.880005pt}{72.227356pt}}
\pgflineto{\pgfpoint{83.807999pt}{72.227356pt}}
\pgflineto{\pgfpoint{83.807999pt}{66.050522pt}}
\pgfpathclose
\pgfusepath{fill,stroke}
\color[rgb]{0.280356,0.074201,0.397901}
\pgfpathmoveto{\pgfpoint{74.880005pt}{78.404205pt}}
\pgflineto{\pgfpoint{83.807999pt}{72.227356pt}}
\pgflineto{\pgfpoint{74.880005pt}{72.227356pt}}
\pgfpathclose
\pgfusepath{fill,stroke}
\pgfpathmoveto{\pgfpoint{74.880005pt}{78.404205pt}}
\pgflineto{\pgfpoint{83.807999pt}{78.404205pt}}
\pgflineto{\pgfpoint{83.807999pt}{72.227356pt}}
\pgfpathclose
\pgfusepath{fill,stroke}
\color[rgb]{0.282390,0.095954,0.418251}
\pgfpathmoveto{\pgfpoint{74.880005pt}{84.581039pt}}
\pgflineto{\pgfpoint{83.807999pt}{78.404205pt}}
\pgflineto{\pgfpoint{74.880005pt}{78.404205pt}}
\pgfpathclose
\pgfusepath{fill,stroke}
\color[rgb]{0.272652,0.025846,0.353367}
\pgfpathmoveto{\pgfpoint{83.807999pt}{53.696838pt}}
\pgflineto{\pgfpoint{92.735992pt}{47.519989pt}}
\pgflineto{\pgfpoint{83.807999pt}{47.519989pt}}
\pgfpathclose
\pgfusepath{fill,stroke}
\pgfpathmoveto{\pgfpoint{83.807999pt}{53.696838pt}}
\pgflineto{\pgfpoint{92.735992pt}{53.696838pt}}
\pgflineto{\pgfpoint{92.735992pt}{47.519989pt}}
\pgfpathclose
\pgfusepath{fill,stroke}
\pgfpathmoveto{\pgfpoint{83.807999pt}{59.873672pt}}
\pgflineto{\pgfpoint{92.735992pt}{53.696838pt}}
\pgflineto{\pgfpoint{83.807999pt}{53.696838pt}}
\pgfpathclose
\pgfusepath{fill,stroke}
\pgfpathmoveto{\pgfpoint{83.807999pt}{59.873672pt}}
\pgflineto{\pgfpoint{92.735992pt}{59.873672pt}}
\pgflineto{\pgfpoint{92.735992pt}{53.696838pt}}
\pgfpathclose
\pgfusepath{fill,stroke}
\color[rgb]{0.277106,0.050914,0.376236}
\pgfpathmoveto{\pgfpoint{83.807999pt}{66.050522pt}}
\pgflineto{\pgfpoint{92.735992pt}{59.873672pt}}
\pgflineto{\pgfpoint{83.807999pt}{59.873672pt}}
\pgfpathclose
\pgfusepath{fill,stroke}
\pgfpathmoveto{\pgfpoint{92.735992pt}{53.696838pt}}
\pgflineto{\pgfpoint{101.664001pt}{47.519989pt}}
\pgflineto{\pgfpoint{92.735992pt}{47.519989pt}}
\pgfpathclose
\pgfusepath{fill,stroke}
\color[rgb]{0.150000,0.150000,0.150000}
\pgfsetlinewidth{0.500000pt}
\pgfsetbuttcap
\pgfsetdash{}{0pt}
\pgfpathmoveto{\pgfpoint{74.880005pt}{47.519989pt}}
\pgflineto{\pgfpoint{74.880005pt}{51.210175pt}}
\pgfusepath{stroke}
\pgfpathmoveto{\pgfpoint{74.880005pt}{47.519989pt}}
\pgflineto{\pgfpoint{79.027298pt}{47.519989pt}}
\pgfusepath{stroke}
\pgfsetrectcap
\pgfsetdash{{16pt}{0pt}}{0pt}
\pgfpathmoveto{\pgfpoint{74.880005pt}{47.519989pt}}
\pgflineto{\pgfpoint{79.027298pt}{47.519989pt}}
\pgfusepath{stroke}
\pgfpathmoveto{\pgfpoint{74.880005pt}{47.519989pt}}
\pgflineto{\pgfpoint{74.880005pt}{51.210175pt}}
\pgfusepath{stroke}
\color[rgb]{0.000000,0.000000,0.000000}
\pgfsetbuttcap
\pgfsetdash{}{0pt}
\pgfpathmoveto{\pgfpoint{74.880005pt}{53.696838pt}}
\pgflineto{\pgfpoint{74.880005pt}{47.519989pt}}
\pgfusepath{stroke}
\pgfpathmoveto{\pgfpoint{74.880005pt}{59.873672pt}}
\pgflineto{\pgfpoint{74.880005pt}{53.696838pt}}
\pgfusepath{stroke}
{
\pgftransformshift{\pgfpoint{69.875519pt}{47.519989pt}}
\pgfnode{rectangle}{east}{\fontsize{10}{0}\selectfont\textcolor[rgb]{0.15,0.15,0.15}{{0}}}{}{\pgfusepath{discard}}}
\pgfpathmoveto{\pgfpoint{74.880005pt}{47.519989pt}}
\pgflineto{\pgfpoint{74.906982pt}{47.519989pt}}
\pgfusepath{stroke}
\pgfpathmoveto{\pgfpoint{74.880005pt}{53.696838pt}}
\pgflineto{\pgfpoint{74.897972pt}{53.696838pt}}
\pgfusepath{stroke}
\pgfpathmoveto{\pgfpoint{74.880005pt}{72.227356pt}}
\pgflineto{\pgfpoint{74.880005pt}{66.050522pt}}
\pgfusepath{stroke}
\pgfpathmoveto{\pgfpoint{74.880005pt}{66.050522pt}}
\pgflineto{\pgfpoint{74.880005pt}{59.873672pt}}
\pgfusepath{stroke}
\pgfpathmoveto{\pgfpoint{74.880005pt}{78.404205pt}}
\pgflineto{\pgfpoint{74.880005pt}{72.227356pt}}
\pgfusepath{stroke}
\pgfpathmoveto{\pgfpoint{74.880005pt}{84.581039pt}}
\pgflineto{\pgfpoint{74.880005pt}{78.404205pt}}
\pgfusepath{stroke}
\pgfpathmoveto{\pgfpoint{74.880005pt}{90.757896pt}}
\pgflineto{\pgfpoint{74.880005pt}{84.581039pt}}
\pgfusepath{stroke}
\pgfpathmoveto{\pgfpoint{74.906982pt}{47.519989pt}}
\pgflineto{\pgfpoint{74.943222pt}{47.519989pt}}
\pgfusepath{stroke}
\pgfpathmoveto{\pgfpoint{74.897972pt}{53.696838pt}}
\pgflineto{\pgfpoint{74.934250pt}{53.696838pt}}
\pgfusepath{stroke}
\pgfpathmoveto{\pgfpoint{74.880005pt}{59.873672pt}}
\pgflineto{\pgfpoint{74.925209pt}{59.873672pt}}
\pgfusepath{stroke}
\pgfpathmoveto{\pgfpoint{74.880005pt}{66.050522pt}}
\pgflineto{\pgfpoint{74.916168pt}{66.050522pt}}
\pgfusepath{stroke}
\pgfpathmoveto{\pgfpoint{74.880005pt}{72.227356pt}}
\pgflineto{\pgfpoint{74.916130pt}{72.227356pt}}
\pgfusepath{stroke}
\pgfpathmoveto{\pgfpoint{74.880005pt}{78.404205pt}}
\pgflineto{\pgfpoint{74.907104pt}{78.404205pt}}
\pgfusepath{stroke}
\pgfpathmoveto{\pgfpoint{74.880005pt}{84.581039pt}}
\pgflineto{\pgfpoint{74.898033pt}{84.581039pt}}
\pgfusepath{stroke}
\pgfpathmoveto{\pgfpoint{74.880005pt}{103.111580pt}}
\pgflineto{\pgfpoint{74.880005pt}{96.934731pt}}
\pgfusepath{stroke}
\pgfpathmoveto{\pgfpoint{74.880005pt}{96.934731pt}}
\pgflineto{\pgfpoint{74.880005pt}{90.757896pt}}
\pgfusepath{stroke}
\pgfpathmoveto{\pgfpoint{74.880005pt}{109.288422pt}}
\pgflineto{\pgfpoint{74.880005pt}{103.111580pt}}
\pgfusepath{stroke}
\pgfpathmoveto{\pgfpoint{74.880005pt}{115.465263pt}}
\pgflineto{\pgfpoint{74.880005pt}{109.288422pt}}
\pgfusepath{stroke}
\pgfpathmoveto{\pgfpoint{74.943222pt}{47.519989pt}}
\pgflineto{\pgfpoint{74.970375pt}{47.519989pt}}
\pgfusepath{stroke}
\pgfpathmoveto{\pgfpoint{74.934250pt}{53.696838pt}}
\pgflineto{\pgfpoint{74.961433pt}{53.696838pt}}
\pgfusepath{stroke}
\pgfpathmoveto{\pgfpoint{74.925209pt}{59.873672pt}}
\pgflineto{\pgfpoint{74.952393pt}{59.873672pt}}
\pgfusepath{stroke}
\pgfpathmoveto{\pgfpoint{74.916168pt}{66.050522pt}}
\pgflineto{\pgfpoint{74.943352pt}{66.050522pt}}
\pgfusepath{stroke}
\pgfpathmoveto{\pgfpoint{74.916130pt}{72.227356pt}}
\pgflineto{\pgfpoint{74.943230pt}{72.227356pt}}
\pgfusepath{stroke}
\pgfpathmoveto{\pgfpoint{74.907104pt}{78.404205pt}}
\pgflineto{\pgfpoint{74.934258pt}{78.404205pt}}
\pgfusepath{stroke}
\pgfpathmoveto{\pgfpoint{74.898033pt}{84.581039pt}}
\pgflineto{\pgfpoint{74.925217pt}{84.581039pt}}
\pgfusepath{stroke}
\pgfpathmoveto{\pgfpoint{74.880005pt}{90.757896pt}}
\pgflineto{\pgfpoint{74.916229pt}{90.757896pt}}
\pgfusepath{stroke}
\pgfpathmoveto{\pgfpoint{74.880005pt}{96.934731pt}}
\pgflineto{\pgfpoint{74.907188pt}{96.934731pt}}
\pgfusepath{stroke}
\pgfpathmoveto{\pgfpoint{74.880005pt}{103.111580pt}}
\pgflineto{\pgfpoint{74.907158pt}{103.111580pt}}
\pgfusepath{stroke}
\pgfpathmoveto{\pgfpoint{74.880005pt}{109.288422pt}}
\pgflineto{\pgfpoint{74.898071pt}{109.288422pt}}
\pgfusepath{stroke}
\pgfpathmoveto{\pgfpoint{74.880005pt}{127.818947pt}}
\pgflineto{\pgfpoint{74.880005pt}{121.642097pt}}
\pgfusepath{stroke}
\pgfpathmoveto{\pgfpoint{74.880005pt}{121.642097pt}}
\pgflineto{\pgfpoint{74.880005pt}{115.465263pt}}
\pgfusepath{stroke}
\pgfpathmoveto{\pgfpoint{74.880005pt}{133.995789pt}}
\pgflineto{\pgfpoint{74.880005pt}{127.818947pt}}
\pgfusepath{stroke}
\pgfpathmoveto{\pgfpoint{74.880005pt}{140.172638pt}}
\pgflineto{\pgfpoint{74.880005pt}{133.995789pt}}
\pgfusepath{stroke}
\pgfpathmoveto{\pgfpoint{74.880005pt}{146.349472pt}}
\pgflineto{\pgfpoint{74.880005pt}{140.172638pt}}
\pgfusepath{stroke}
\pgfpathmoveto{\pgfpoint{74.970375pt}{47.519989pt}}
\pgflineto{\pgfpoint{75.006500pt}{47.519989pt}}
\pgfusepath{stroke}
\pgfpathmoveto{\pgfpoint{74.961433pt}{53.696838pt}}
\pgflineto{\pgfpoint{74.997597pt}{53.696838pt}}
\pgfusepath{stroke}
\pgfpathmoveto{\pgfpoint{74.952393pt}{59.873672pt}}
\pgflineto{\pgfpoint{74.988556pt}{59.873672pt}}
\pgfusepath{stroke}
\pgfpathmoveto{\pgfpoint{74.943352pt}{66.050522pt}}
\pgflineto{\pgfpoint{74.979515pt}{66.050522pt}}
\pgfusepath{stroke}
\pgfpathmoveto{\pgfpoint{74.943230pt}{72.227356pt}}
\pgflineto{\pgfpoint{74.979416pt}{72.227356pt}}
\pgfusepath{stroke}
\pgfpathmoveto{\pgfpoint{74.934258pt}{78.404205pt}}
\pgflineto{\pgfpoint{74.970444pt}{78.404205pt}}
\pgfusepath{stroke}
\pgfpathmoveto{\pgfpoint{74.925217pt}{84.581039pt}}
\pgflineto{\pgfpoint{74.961380pt}{84.581039pt}}
\pgfusepath{stroke}
\pgfpathmoveto{\pgfpoint{74.916229pt}{90.757896pt}}
\pgflineto{\pgfpoint{74.952393pt}{90.757896pt}}
\pgfusepath{stroke}
\pgfpathmoveto{\pgfpoint{74.907188pt}{96.934731pt}}
\pgflineto{\pgfpoint{74.943352pt}{96.934731pt}}
\pgfusepath{stroke}
\pgfpathmoveto{\pgfpoint{74.907158pt}{103.111580pt}}
\pgflineto{\pgfpoint{74.943283pt}{103.111580pt}}
\pgfusepath{stroke}
\pgfpathmoveto{\pgfpoint{74.898071pt}{109.288422pt}}
\pgflineto{\pgfpoint{74.934196pt}{109.288422pt}}
\pgfusepath{stroke}
\pgfpathmoveto{\pgfpoint{74.880005pt}{115.465263pt}}
\pgflineto{\pgfpoint{74.925270pt}{115.465263pt}}
\pgfusepath{stroke}
\pgfpathmoveto{\pgfpoint{74.880005pt}{121.642097pt}}
\pgflineto{\pgfpoint{74.916168pt}{121.642097pt}}
\pgfusepath{stroke}
\pgfpathmoveto{\pgfpoint{74.880005pt}{127.818947pt}}
\pgflineto{\pgfpoint{74.916130pt}{127.818947pt}}
\pgfusepath{stroke}
\pgfpathmoveto{\pgfpoint{74.880005pt}{133.995789pt}}
\pgflineto{\pgfpoint{74.907104pt}{133.995789pt}}
\pgfusepath{stroke}
\pgfpathmoveto{\pgfpoint{74.880005pt}{140.172638pt}}
\pgflineto{\pgfpoint{74.898087pt}{140.172638pt}}
\pgfusepath{stroke}
\pgfpathmoveto{\pgfpoint{74.880005pt}{158.703156pt}}
\pgflineto{\pgfpoint{74.880005pt}{152.526306pt}}
\pgfusepath{stroke}
\pgfpathmoveto{\pgfpoint{74.880005pt}{152.526306pt}}
\pgflineto{\pgfpoint{74.880005pt}{146.349472pt}}
\pgfusepath{stroke}
\pgfpathmoveto{\pgfpoint{74.880005pt}{164.880005pt}}
\pgflineto{\pgfpoint{74.880005pt}{158.703156pt}}
\pgfusepath{stroke}
\pgfpathmoveto{\pgfpoint{74.880005pt}{171.056854pt}}
\pgflineto{\pgfpoint{74.880005pt}{164.880005pt}}
\pgfusepath{stroke}
\pgfpathmoveto{\pgfpoint{74.880005pt}{177.233673pt}}
\pgflineto{\pgfpoint{74.880005pt}{171.056854pt}}
\pgfusepath{stroke}
\pgfpathmoveto{\pgfpoint{75.006500pt}{47.519989pt}}
\pgflineto{\pgfpoint{75.042686pt}{47.519989pt}}
\pgfusepath{stroke}
\pgfpathmoveto{\pgfpoint{74.997597pt}{53.696838pt}}
\pgflineto{\pgfpoint{75.033821pt}{53.696838pt}}
\pgfusepath{stroke}
\pgfpathmoveto{\pgfpoint{74.988556pt}{59.873672pt}}
\pgflineto{\pgfpoint{75.024780pt}{59.873672pt}}
\pgfusepath{stroke}
\pgfpathmoveto{\pgfpoint{74.979515pt}{66.050522pt}}
\pgflineto{\pgfpoint{75.015739pt}{66.050522pt}}
\pgfusepath{stroke}
\pgfpathmoveto{\pgfpoint{74.979416pt}{72.227356pt}}
\pgflineto{\pgfpoint{75.015602pt}{72.227356pt}}
\pgfusepath{stroke}
\pgfpathmoveto{\pgfpoint{74.970444pt}{78.404205pt}}
\pgflineto{\pgfpoint{75.006569pt}{78.404205pt}}
\pgfusepath{stroke}
\pgfpathmoveto{\pgfpoint{74.961380pt}{84.581039pt}}
\pgflineto{\pgfpoint{74.997604pt}{84.581039pt}}
\pgfusepath{stroke}
\pgfpathmoveto{\pgfpoint{74.952393pt}{90.757896pt}}
\pgflineto{\pgfpoint{74.988556pt}{90.757896pt}}
\pgfusepath{stroke}
\pgfpathmoveto{\pgfpoint{74.943352pt}{96.934731pt}}
\pgflineto{\pgfpoint{74.979576pt}{96.934731pt}}
\pgfusepath{stroke}
\pgfpathmoveto{\pgfpoint{74.943283pt}{103.111580pt}}
\pgflineto{\pgfpoint{74.979469pt}{103.111580pt}}
\pgfusepath{stroke}
\pgfpathmoveto{\pgfpoint{74.934196pt}{109.288422pt}}
\pgflineto{\pgfpoint{74.970383pt}{109.288422pt}}
\pgfusepath{stroke}
\pgfpathmoveto{\pgfpoint{74.925270pt}{115.465263pt}}
\pgflineto{\pgfpoint{74.961494pt}{115.465263pt}}
\pgfusepath{stroke}
\pgfpathmoveto{\pgfpoint{74.916168pt}{121.642097pt}}
\pgflineto{\pgfpoint{74.952393pt}{121.642097pt}}
\pgfusepath{stroke}
\pgfpathmoveto{\pgfpoint{74.916130pt}{127.818947pt}}
\pgflineto{\pgfpoint{74.952316pt}{127.818947pt}}
\pgfusepath{stroke}
\pgfpathmoveto{\pgfpoint{74.907104pt}{133.995789pt}}
\pgflineto{\pgfpoint{74.943291pt}{133.995789pt}}
\pgfusepath{stroke}
\pgfpathmoveto{\pgfpoint{74.898087pt}{140.172638pt}}
\pgflineto{\pgfpoint{74.934311pt}{140.172638pt}}
\pgfusepath{stroke}
\pgfpathmoveto{\pgfpoint{74.880005pt}{146.349472pt}}
\pgflineto{\pgfpoint{74.925323pt}{146.349472pt}}
\pgfusepath{stroke}
\pgfpathmoveto{\pgfpoint{74.880005pt}{152.526306pt}}
\pgflineto{\pgfpoint{74.916229pt}{152.526306pt}}
\pgfusepath{stroke}
\pgfpathmoveto{\pgfpoint{74.880005pt}{158.703156pt}}
\pgflineto{\pgfpoint{74.916130pt}{158.703156pt}}
\pgfusepath{stroke}
\pgfpathmoveto{\pgfpoint{74.880005pt}{164.880005pt}}
\pgflineto{\pgfpoint{74.907158pt}{164.880005pt}}
\pgfusepath{stroke}
\pgfpathmoveto{\pgfpoint{74.880005pt}{171.056854pt}}
\pgflineto{\pgfpoint{74.898148pt}{171.056854pt}}
\pgfusepath{stroke}
\pgfpathmoveto{\pgfpoint{74.880005pt}{189.587372pt}}
\pgflineto{\pgfpoint{74.880005pt}{183.410522pt}}
\pgfusepath{stroke}
\pgfpathmoveto{\pgfpoint{74.880005pt}{183.410522pt}}
\pgflineto{\pgfpoint{74.880005pt}{177.233673pt}}
\pgfusepath{stroke}
\pgfpathmoveto{\pgfpoint{74.880005pt}{195.764206pt}}
\pgflineto{\pgfpoint{74.880005pt}{189.587372pt}}
\pgfusepath{stroke}
\pgfpathmoveto{\pgfpoint{74.880005pt}{306.947388pt}}
\pgflineto{\pgfpoint{74.880005pt}{300.770538pt}}
\pgfusepath{stroke}
\pgfpathmoveto{\pgfpoint{74.880005pt}{313.124207pt}}
\pgflineto{\pgfpoint{74.880005pt}{306.947388pt}}
\pgfusepath{stroke}
\pgfpathmoveto{\pgfpoint{74.880005pt}{319.301056pt}}
\pgflineto{\pgfpoint{74.880005pt}{313.124207pt}}
\pgfusepath{stroke}
\pgfpathmoveto{\pgfpoint{74.880005pt}{325.477905pt}}
\pgflineto{\pgfpoint{74.880005pt}{319.301056pt}}
\pgfusepath{stroke}
\pgfpathmoveto{\pgfpoint{74.880005pt}{306.947388pt}}
\pgflineto{\pgfpoint{74.898132pt}{306.947388pt}}
\pgfusepath{stroke}
\pgfpathmoveto{\pgfpoint{74.880005pt}{313.124207pt}}
\pgflineto{\pgfpoint{74.898132pt}{313.124207pt}}
\pgfusepath{stroke}
\pgfpathmoveto{\pgfpoint{74.880005pt}{319.301056pt}}
\pgflineto{\pgfpoint{74.898155pt}{319.301056pt}}
\pgfusepath{stroke}
\pgfpathmoveto{\pgfpoint{74.880005pt}{337.831604pt}}
\pgflineto{\pgfpoint{74.880005pt}{331.654724pt}}
\pgfusepath{stroke}
\pgfpathmoveto{\pgfpoint{74.880005pt}{331.654724pt}}
\pgflineto{\pgfpoint{74.880005pt}{325.477905pt}}
\pgfusepath{stroke}
\pgfpathmoveto{\pgfpoint{74.880005pt}{344.008423pt}}
\pgflineto{\pgfpoint{74.880005pt}{337.831604pt}}
\pgfusepath{stroke}
\pgfpathmoveto{\pgfpoint{74.880005pt}{325.477905pt}}
\pgflineto{\pgfpoint{74.898209pt}{325.477905pt}}
\pgfusepath{stroke}
\pgfpathmoveto{\pgfpoint{74.880005pt}{350.185242pt}}
\pgflineto{\pgfpoint{74.880005pt}{344.008423pt}}
\pgfusepath{stroke}
\pgfpathmoveto{\pgfpoint{74.880005pt}{356.362122pt}}
\pgflineto{\pgfpoint{74.880005pt}{350.185242pt}}
\pgfusepath{stroke}
\pgfpathmoveto{\pgfpoint{74.880005pt}{362.538940pt}}
\pgflineto{\pgfpoint{74.880005pt}{356.362122pt}}
\pgfusepath{stroke}
\pgfpathmoveto{\pgfpoint{74.880005pt}{368.715820pt}}
\pgflineto{\pgfpoint{74.880005pt}{362.538940pt}}
\pgfusepath{stroke}
\pgfpathmoveto{\pgfpoint{74.880005pt}{374.892639pt}}
\pgflineto{\pgfpoint{74.880005pt}{368.715820pt}}
\pgfusepath{stroke}
\pgfpathmoveto{\pgfpoint{74.880005pt}{381.069458pt}}
\pgflineto{\pgfpoint{74.880005pt}{374.892639pt}}
\pgfusepath{stroke}
\pgfpathmoveto{\pgfpoint{74.880005pt}{387.246338pt}}
\pgflineto{\pgfpoint{74.880005pt}{381.069458pt}}
\pgfusepath{stroke}
\pgfpathmoveto{\pgfpoint{74.880005pt}{393.423157pt}}
\pgflineto{\pgfpoint{74.880005pt}{387.246338pt}}
\pgfusepath{stroke}
\pgfpathmoveto{\pgfpoint{74.880005pt}{399.600037pt}}
\pgflineto{\pgfpoint{74.880005pt}{393.423157pt}}
\pgfusepath{stroke}
\pgfpathmoveto{\pgfpoint{74.880005pt}{331.654724pt}}
\pgflineto{\pgfpoint{74.898155pt}{331.654724pt}}
\pgfusepath{stroke}
\pgfpathmoveto{\pgfpoint{74.880005pt}{337.831604pt}}
\pgflineto{\pgfpoint{74.907227pt}{337.831604pt}}
\pgfusepath{stroke}
\pgfpathmoveto{\pgfpoint{74.880005pt}{344.008423pt}}
\pgflineto{\pgfpoint{74.907227pt}{344.008423pt}}
\pgfusepath{stroke}
\pgfpathmoveto{\pgfpoint{74.880005pt}{350.185242pt}}
\pgflineto{\pgfpoint{74.907257pt}{350.185242pt}}
\pgfusepath{stroke}
\pgfpathmoveto{\pgfpoint{74.880005pt}{356.362122pt}}
\pgflineto{\pgfpoint{74.907310pt}{356.362122pt}}
\pgfusepath{stroke}
\pgfpathmoveto{\pgfpoint{74.880005pt}{362.538940pt}}
\pgflineto{\pgfpoint{74.907310pt}{362.538940pt}}
\pgfusepath{stroke}
\pgfpathmoveto{\pgfpoint{74.880005pt}{368.715820pt}}
\pgflineto{\pgfpoint{74.916321pt}{368.715820pt}}
\pgfusepath{stroke}
\pgfpathmoveto{\pgfpoint{74.880005pt}{374.892639pt}}
\pgflineto{\pgfpoint{74.916321pt}{374.892639pt}}
\pgfusepath{stroke}
\pgfpathmoveto{\pgfpoint{74.880005pt}{381.069458pt}}
\pgflineto{\pgfpoint{74.916321pt}{381.069458pt}}
\pgfusepath{stroke}
\pgfpathmoveto{\pgfpoint{74.880005pt}{387.246338pt}}
\pgflineto{\pgfpoint{74.916397pt}{387.246338pt}}
\pgfusepath{stroke}
\pgfpathmoveto{\pgfpoint{74.880005pt}{393.423157pt}}
\pgflineto{\pgfpoint{74.916397pt}{393.423157pt}}
\pgfusepath{stroke}
\pgfpathmoveto{\pgfpoint{74.880005pt}{399.600037pt}}
\pgflineto{\pgfpoint{74.916420pt}{399.600037pt}}
\pgfusepath{stroke}
\pgfpathmoveto{\pgfpoint{74.880005pt}{257.532623pt}}
\pgflineto{\pgfpoint{74.880005pt}{251.355804pt}}
\pgfusepath{stroke}
\pgfpathmoveto{\pgfpoint{74.880005pt}{226.648422pt}}
\pgflineto{\pgfpoint{74.880005pt}{220.471588pt}}
\pgfusepath{stroke}
\pgfpathmoveto{\pgfpoint{74.880005pt}{232.825272pt}}
\pgflineto{\pgfpoint{74.880005pt}{226.648422pt}}
\pgfusepath{stroke}
\pgfpathmoveto{\pgfpoint{74.880005pt}{239.002106pt}}
\pgflineto{\pgfpoint{74.880005pt}{232.825272pt}}
\pgfusepath{stroke}
\pgfpathmoveto{\pgfpoint{74.880005pt}{245.178955pt}}
\pgflineto{\pgfpoint{74.880005pt}{239.002106pt}}
\pgfusepath{stroke}
\pgfpathmoveto{\pgfpoint{74.880005pt}{251.355804pt}}
\pgflineto{\pgfpoint{74.880005pt}{245.178955pt}}
\pgfusepath{stroke}
\pgfpathmoveto{\pgfpoint{74.880005pt}{263.709473pt}}
\pgflineto{\pgfpoint{74.880005pt}{257.532623pt}}
\pgfusepath{stroke}
\pgfpathmoveto{\pgfpoint{74.880005pt}{269.886322pt}}
\pgflineto{\pgfpoint{74.880005pt}{263.709473pt}}
\pgfusepath{stroke}
\pgfpathmoveto{\pgfpoint{74.880005pt}{276.063141pt}}
\pgflineto{\pgfpoint{74.880005pt}{269.886322pt}}
\pgfusepath{stroke}
\pgfpathmoveto{\pgfpoint{74.880005pt}{282.239990pt}}
\pgflineto{\pgfpoint{74.880005pt}{276.063141pt}}
\pgfusepath{stroke}
\pgfpathmoveto{\pgfpoint{74.880005pt}{288.416840pt}}
\pgflineto{\pgfpoint{74.880005pt}{282.239990pt}}
\pgfusepath{stroke}
\pgfpathmoveto{\pgfpoint{74.880005pt}{294.593689pt}}
\pgflineto{\pgfpoint{74.880005pt}{288.416840pt}}
\pgfusepath{stroke}
\pgfpathmoveto{\pgfpoint{74.880005pt}{300.770538pt}}
\pgflineto{\pgfpoint{74.880005pt}{294.593689pt}}
\pgfusepath{stroke}
\pgfpathmoveto{\pgfpoint{74.880005pt}{251.355804pt}}
\pgflineto{\pgfpoint{74.898079pt}{251.355804pt}}
\pgfusepath{stroke}
\pgfpathmoveto{\pgfpoint{74.880005pt}{257.532623pt}}
\pgflineto{\pgfpoint{74.898079pt}{257.532623pt}}
\pgfusepath{stroke}
\pgfpathmoveto{\pgfpoint{74.880005pt}{263.709473pt}}
\pgflineto{\pgfpoint{74.898041pt}{263.709473pt}}
\pgfusepath{stroke}
\pgfpathmoveto{\pgfpoint{74.880005pt}{269.886322pt}}
\pgflineto{\pgfpoint{74.898041pt}{269.886322pt}}
\pgfusepath{stroke}
\pgfpathmoveto{\pgfpoint{74.880005pt}{276.063141pt}}
\pgflineto{\pgfpoint{74.907059pt}{276.063141pt}}
\pgfusepath{stroke}
\pgfpathmoveto{\pgfpoint{74.880005pt}{282.239990pt}}
\pgflineto{\pgfpoint{74.907112pt}{282.239990pt}}
\pgfusepath{stroke}
\pgfpathmoveto{\pgfpoint{74.880005pt}{288.416840pt}}
\pgflineto{\pgfpoint{74.907143pt}{288.416840pt}}
\pgfusepath{stroke}
\pgfpathmoveto{\pgfpoint{74.880005pt}{294.593689pt}}
\pgflineto{\pgfpoint{74.907143pt}{294.593689pt}}
\pgfusepath{stroke}
\pgfpathmoveto{\pgfpoint{74.880005pt}{300.770538pt}}
\pgflineto{\pgfpoint{74.907143pt}{300.770538pt}}
\pgfusepath{stroke}
\pgfpathmoveto{\pgfpoint{74.898132pt}{306.947388pt}}
\pgflineto{\pgfpoint{74.916145pt}{306.947388pt}}
\pgfusepath{stroke}
\pgfpathmoveto{\pgfpoint{74.898132pt}{313.124207pt}}
\pgflineto{\pgfpoint{74.916092pt}{313.124207pt}}
\pgfusepath{stroke}
\pgfpathmoveto{\pgfpoint{74.898155pt}{319.301056pt}}
\pgflineto{\pgfpoint{74.916191pt}{319.301056pt}}
\pgfusepath{stroke}
\pgfpathmoveto{\pgfpoint{74.898209pt}{325.477905pt}}
\pgflineto{\pgfpoint{74.916130pt}{325.477905pt}}
\pgfusepath{stroke}
\pgfpathmoveto{\pgfpoint{74.898155pt}{331.654724pt}}
\pgflineto{\pgfpoint{74.916130pt}{331.654724pt}}
\pgfusepath{stroke}
\pgfpathmoveto{\pgfpoint{74.907227pt}{337.831604pt}}
\pgflineto{\pgfpoint{74.925125pt}{337.831604pt}}
\pgfusepath{stroke}
\pgfpathmoveto{\pgfpoint{74.907227pt}{344.008423pt}}
\pgflineto{\pgfpoint{74.925125pt}{344.008423pt}}
\pgfusepath{stroke}
\pgfpathmoveto{\pgfpoint{74.907257pt}{350.185242pt}}
\pgflineto{\pgfpoint{74.925179pt}{350.185242pt}}
\pgfusepath{stroke}
\pgfpathmoveto{\pgfpoint{74.907310pt}{356.362122pt}}
\pgflineto{\pgfpoint{74.925232pt}{356.362122pt}}
\pgfusepath{stroke}
\pgfpathmoveto{\pgfpoint{74.907310pt}{362.538940pt}}
\pgflineto{\pgfpoint{74.925171pt}{362.538940pt}}
\pgfusepath{stroke}
\pgfpathmoveto{\pgfpoint{74.916321pt}{368.715820pt}}
\pgflineto{\pgfpoint{74.934219pt}{368.715820pt}}
\pgfusepath{stroke}
\pgfpathmoveto{\pgfpoint{74.916321pt}{374.892639pt}}
\pgflineto{\pgfpoint{74.934219pt}{374.892639pt}}
\pgfusepath{stroke}
\pgfpathmoveto{\pgfpoint{74.916321pt}{381.069458pt}}
\pgflineto{\pgfpoint{74.934105pt}{381.069458pt}}
\pgfusepath{stroke}
\pgfpathmoveto{\pgfpoint{74.916397pt}{387.246338pt}}
\pgflineto{\pgfpoint{74.934250pt}{387.246338pt}}
\pgfusepath{stroke}
\pgfpathmoveto{\pgfpoint{74.916397pt}{393.423157pt}}
\pgflineto{\pgfpoint{74.934189pt}{393.423157pt}}
\pgfusepath{stroke}
\pgfpathmoveto{\pgfpoint{74.916420pt}{399.600037pt}}
\pgflineto{\pgfpoint{74.934280pt}{399.600037pt}}
\pgfusepath{stroke}
\pgfpathmoveto{\pgfpoint{74.880005pt}{201.941055pt}}
\pgflineto{\pgfpoint{74.880005pt}{195.764206pt}}
\pgfusepath{stroke}
\pgfpathmoveto{\pgfpoint{74.880005pt}{208.117905pt}}
\pgflineto{\pgfpoint{74.880005pt}{201.941055pt}}
\pgfusepath{stroke}
\pgfpathmoveto{\pgfpoint{74.880005pt}{214.294739pt}}
\pgflineto{\pgfpoint{74.880005pt}{208.117905pt}}
\pgfusepath{stroke}
\pgfpathmoveto{\pgfpoint{74.880005pt}{220.471588pt}}
\pgflineto{\pgfpoint{74.880005pt}{214.294739pt}}
\pgfusepath{stroke}
\pgfpathmoveto{\pgfpoint{75.042686pt}{47.519989pt}}
\pgflineto{\pgfpoint{83.790169pt}{47.519989pt}}
\pgfusepath{stroke}
{
\pgftransformshift{\pgfpoint{74.880005pt}{40.018295pt}}
\pgfnode{rectangle}{north}{\fontsize{10}{0}\selectfont\textcolor[rgb]{0.15,0.15,0.15}{{0}}}{}{\pgfusepath{discard}}}
\pgfpathmoveto{\pgfpoint{83.807999pt}{53.696838pt}}
\pgflineto{\pgfpoint{75.033821pt}{53.696838pt}}
\pgfusepath{stroke}
\pgfpathmoveto{\pgfpoint{83.807999pt}{59.873672pt}}
\pgflineto{\pgfpoint{75.024780pt}{59.873672pt}}
\pgfusepath{stroke}
\pgfpathmoveto{\pgfpoint{83.807999pt}{66.050522pt}}
\pgflineto{\pgfpoint{75.015739pt}{66.050522pt}}
\pgfusepath{stroke}
\pgfpathmoveto{\pgfpoint{83.807999pt}{72.227356pt}}
\pgflineto{\pgfpoint{75.015602pt}{72.227356pt}}
\pgfusepath{stroke}
\pgfpathmoveto{\pgfpoint{83.807999pt}{78.404205pt}}
\pgflineto{\pgfpoint{75.006569pt}{78.404205pt}}
\pgfusepath{stroke}
\pgfpathmoveto{\pgfpoint{83.807999pt}{84.581039pt}}
\pgflineto{\pgfpoint{74.997604pt}{84.581039pt}}
\pgfusepath{stroke}
\pgfpathmoveto{\pgfpoint{83.807999pt}{90.757896pt}}
\pgflineto{\pgfpoint{74.988556pt}{90.757896pt}}
\pgfusepath{stroke}
\pgfpathmoveto{\pgfpoint{83.807999pt}{96.934731pt}}
\pgflineto{\pgfpoint{74.979576pt}{96.934731pt}}
\pgfusepath{stroke}
\pgfpathmoveto{\pgfpoint{83.807999pt}{103.111580pt}}
\pgflineto{\pgfpoint{74.979469pt}{103.111580pt}}
\pgfusepath{stroke}
\pgfpathmoveto{\pgfpoint{83.807999pt}{109.288422pt}}
\pgflineto{\pgfpoint{74.970383pt}{109.288422pt}}
\pgfusepath{stroke}
\pgfpathmoveto{\pgfpoint{83.807999pt}{115.465263pt}}
\pgflineto{\pgfpoint{74.961494pt}{115.465263pt}}
\pgfusepath{stroke}
\pgfpathmoveto{\pgfpoint{83.807999pt}{121.642097pt}}
\pgflineto{\pgfpoint{74.952393pt}{121.642097pt}}
\pgfusepath{stroke}
\pgfpathmoveto{\pgfpoint{83.807999pt}{127.818947pt}}
\pgflineto{\pgfpoint{74.952316pt}{127.818947pt}}
\pgfusepath{stroke}
\pgfpathmoveto{\pgfpoint{83.807999pt}{133.995789pt}}
\pgflineto{\pgfpoint{74.943291pt}{133.995789pt}}
\pgfusepath{stroke}
\pgfpathmoveto{\pgfpoint{83.807999pt}{140.172638pt}}
\pgflineto{\pgfpoint{74.934311pt}{140.172638pt}}
\pgfusepath{stroke}
\pgfpathmoveto{\pgfpoint{83.807999pt}{146.349472pt}}
\pgflineto{\pgfpoint{74.925323pt}{146.349472pt}}
\pgfusepath{stroke}
\pgfpathmoveto{\pgfpoint{83.807999pt}{152.526306pt}}
\pgflineto{\pgfpoint{74.916229pt}{152.526306pt}}
\pgfusepath{stroke}
\pgfpathmoveto{\pgfpoint{83.807999pt}{158.703156pt}}
\pgflineto{\pgfpoint{74.916130pt}{158.703156pt}}
\pgfusepath{stroke}
\pgfpathmoveto{\pgfpoint{83.807999pt}{164.880005pt}}
\pgflineto{\pgfpoint{74.907158pt}{164.880005pt}}
\pgfusepath{stroke}
\pgfpathmoveto{\pgfpoint{83.807999pt}{171.056854pt}}
\pgflineto{\pgfpoint{74.898148pt}{171.056854pt}}
\pgfusepath{stroke}
\pgfpathmoveto{\pgfpoint{83.807999pt}{177.233673pt}}
\pgflineto{\pgfpoint{74.880005pt}{177.233673pt}}
\pgfusepath{stroke}
\pgfpathmoveto{\pgfpoint{74.880005pt}{183.410522pt}}
\pgflineto{\pgfpoint{83.789864pt}{183.410522pt}}
\pgfusepath{stroke}
\pgfpathmoveto{\pgfpoint{74.880005pt}{189.587372pt}}
\pgflineto{\pgfpoint{83.780846pt}{189.587372pt}}
\pgfusepath{stroke}
\pgfpathmoveto{\pgfpoint{74.880005pt}{195.764206pt}}
\pgflineto{\pgfpoint{83.771812pt}{195.764206pt}}
\pgfusepath{stroke}
\pgfpathmoveto{\pgfpoint{74.880005pt}{201.941055pt}}
\pgflineto{\pgfpoint{83.771782pt}{201.941055pt}}
\pgfusepath{stroke}
\pgfpathmoveto{\pgfpoint{74.880005pt}{208.117905pt}}
\pgflineto{\pgfpoint{83.762741pt}{208.117905pt}}
\pgfusepath{stroke}
\pgfpathmoveto{\pgfpoint{74.880005pt}{214.294739pt}}
\pgflineto{\pgfpoint{83.753677pt}{214.294739pt}}
\pgfusepath{stroke}
\pgfpathmoveto{\pgfpoint{74.880005pt}{220.471588pt}}
\pgflineto{\pgfpoint{83.744720pt}{220.471588pt}}
\pgfusepath{stroke}
\pgfpathmoveto{\pgfpoint{74.880005pt}{226.648422pt}}
\pgflineto{\pgfpoint{83.735687pt}{226.648422pt}}
\pgfusepath{stroke}
\pgfpathmoveto{\pgfpoint{74.880005pt}{232.825272pt}}
\pgflineto{\pgfpoint{83.735596pt}{232.825272pt}}
\pgfusepath{stroke}
\pgfpathmoveto{\pgfpoint{74.880005pt}{239.002106pt}}
\pgflineto{\pgfpoint{83.726555pt}{239.002106pt}}
\pgfusepath{stroke}
\pgfpathmoveto{\pgfpoint{74.880005pt}{245.178955pt}}
\pgflineto{\pgfpoint{83.717537pt}{245.178955pt}}
\pgfusepath{stroke}
\pgfpathmoveto{\pgfpoint{74.898079pt}{251.355804pt}}
\pgflineto{\pgfpoint{83.708542pt}{251.355804pt}}
\pgfusepath{stroke}
\pgfpathmoveto{\pgfpoint{74.898079pt}{257.532623pt}}
\pgflineto{\pgfpoint{83.699570pt}{257.532623pt}}
\pgfusepath{stroke}
\pgfpathmoveto{\pgfpoint{74.898041pt}{263.709473pt}}
\pgflineto{\pgfpoint{83.699471pt}{263.709473pt}}
\pgfusepath{stroke}
\pgfpathmoveto{\pgfpoint{74.898041pt}{269.886322pt}}
\pgflineto{\pgfpoint{83.690376pt}{269.886322pt}}
\pgfusepath{stroke}
\pgfpathmoveto{\pgfpoint{74.907059pt}{276.063141pt}}
\pgflineto{\pgfpoint{83.681404pt}{276.063141pt}}
\pgfusepath{stroke}
\pgfpathmoveto{\pgfpoint{74.907112pt}{282.239990pt}}
\pgflineto{\pgfpoint{83.672417pt}{282.239990pt}}
\pgfusepath{stroke}
\pgfpathmoveto{\pgfpoint{74.907143pt}{288.416840pt}}
\pgflineto{\pgfpoint{83.672287pt}{288.416840pt}}
\pgfusepath{stroke}
\pgfpathmoveto{\pgfpoint{74.907143pt}{294.593689pt}}
\pgflineto{\pgfpoint{83.663223pt}{294.593689pt}}
\pgfusepath{stroke}
\pgfpathmoveto{\pgfpoint{74.907143pt}{300.770538pt}}
\pgflineto{\pgfpoint{83.654175pt}{300.770538pt}}
\pgfusepath{stroke}
\pgfpathmoveto{\pgfpoint{74.916145pt}{306.947388pt}}
\pgflineto{\pgfpoint{83.645348pt}{306.947388pt}}
\pgfusepath{stroke}
\pgfpathmoveto{\pgfpoint{74.916092pt}{313.124207pt}}
\pgflineto{\pgfpoint{83.636246pt}{313.124207pt}}
\pgfusepath{stroke}
\pgfpathmoveto{\pgfpoint{74.916191pt}{319.301056pt}}
\pgflineto{\pgfpoint{83.636063pt}{319.301056pt}}
\pgfusepath{stroke}
\pgfpathmoveto{\pgfpoint{74.916130pt}{325.477905pt}}
\pgflineto{\pgfpoint{83.627037pt}{325.477905pt}}
\pgfusepath{stroke}
\pgfpathmoveto{\pgfpoint{74.916130pt}{331.654724pt}}
\pgflineto{\pgfpoint{83.617996pt}{331.654724pt}}
\pgfusepath{stroke}
\pgfpathmoveto{\pgfpoint{74.925125pt}{337.831604pt}}
\pgflineto{\pgfpoint{83.609169pt}{337.831604pt}}
\pgfusepath{stroke}
\pgfpathmoveto{\pgfpoint{74.925125pt}{344.008423pt}}
\pgflineto{\pgfpoint{83.600113pt}{344.008423pt}}
\pgfusepath{stroke}
\pgfpathmoveto{\pgfpoint{74.925179pt}{350.185242pt}}
\pgflineto{\pgfpoint{83.599892pt}{350.185242pt}}
\pgfusepath{stroke}
\pgfpathmoveto{\pgfpoint{74.925232pt}{356.362122pt}}
\pgflineto{\pgfpoint{83.590851pt}{356.362122pt}}
\pgfusepath{stroke}
\pgfpathmoveto{\pgfpoint{74.925171pt}{362.538940pt}}
\pgflineto{\pgfpoint{83.581810pt}{362.538940pt}}
\pgfusepath{stroke}
\pgfpathmoveto{\pgfpoint{74.934219pt}{368.715820pt}}
\pgflineto{\pgfpoint{83.573021pt}{368.715820pt}}
\pgfusepath{stroke}
\pgfpathmoveto{\pgfpoint{74.934219pt}{374.892639pt}}
\pgflineto{\pgfpoint{83.563980pt}{374.892639pt}}
\pgfusepath{stroke}
\pgfpathmoveto{\pgfpoint{74.934105pt}{381.069458pt}}
\pgflineto{\pgfpoint{83.554955pt}{381.069458pt}}
\pgfusepath{stroke}
\pgfpathmoveto{\pgfpoint{74.934250pt}{387.246338pt}}
\pgflineto{\pgfpoint{83.550301pt}{387.246338pt}}
\pgfusepath{stroke}
\pgfpathmoveto{\pgfpoint{74.934189pt}{393.423157pt}}
\pgflineto{\pgfpoint{83.541206pt}{393.423157pt}}
\pgfusepath{stroke}
\pgfpathmoveto{\pgfpoint{74.934280pt}{399.600037pt}}
\pgflineto{\pgfpoint{83.536606pt}{399.600037pt}}
\pgfusepath{stroke}
\pgfpathmoveto{\pgfpoint{83.807999pt}{146.349472pt}}
\pgflineto{\pgfpoint{83.807999pt}{140.172638pt}}
\pgfusepath{stroke}
\pgfpathmoveto{\pgfpoint{83.807999pt}{140.172638pt}}
\pgflineto{\pgfpoint{83.807999pt}{133.995789pt}}
\pgfusepath{stroke}
\pgfpathmoveto{\pgfpoint{83.807999pt}{152.526306pt}}
\pgflineto{\pgfpoint{83.807999pt}{146.349472pt}}
\pgfusepath{stroke}
\pgfpathmoveto{\pgfpoint{83.807999pt}{158.703156pt}}
\pgflineto{\pgfpoint{83.807999pt}{152.526306pt}}
\pgfusepath{stroke}
\pgfpathmoveto{\pgfpoint{83.807999pt}{140.172638pt}}
\pgflineto{\pgfpoint{83.825783pt}{140.172638pt}}
\pgfusepath{stroke}
\pgfpathmoveto{\pgfpoint{83.807999pt}{146.349472pt}}
\pgflineto{\pgfpoint{83.825783pt}{146.349472pt}}
\pgfusepath{stroke}
\pgfpathmoveto{\pgfpoint{83.807999pt}{152.526306pt}}
\pgflineto{\pgfpoint{83.825729pt}{152.526306pt}}
\pgfusepath{stroke}
\pgfpathmoveto{\pgfpoint{83.807999pt}{171.056854pt}}
\pgflineto{\pgfpoint{83.807999pt}{164.880005pt}}
\pgfusepath{stroke}
\pgfpathmoveto{\pgfpoint{83.807999pt}{164.880005pt}}
\pgflineto{\pgfpoint{83.807999pt}{158.703156pt}}
\pgfusepath{stroke}
\pgfpathmoveto{\pgfpoint{83.807999pt}{177.233673pt}}
\pgflineto{\pgfpoint{83.807999pt}{171.056854pt}}
\pgfusepath{stroke}
\pgfpathmoveto{\pgfpoint{83.807999pt}{158.703156pt}}
\pgflineto{\pgfpoint{83.825684pt}{158.703156pt}}
\pgfusepath{stroke}
\pgfpathmoveto{\pgfpoint{83.771812pt}{195.764206pt}}
\pgflineto{\pgfpoint{83.789825pt}{195.764206pt}}
\pgfusepath{stroke}
\pgfpathmoveto{\pgfpoint{83.807999pt}{201.941055pt}}
\pgflineto{\pgfpoint{83.771782pt}{201.941055pt}}
\pgfusepath{stroke}
\pgfpathmoveto{\pgfpoint{83.807999pt}{208.117905pt}}
\pgflineto{\pgfpoint{83.762741pt}{208.117905pt}}
\pgfusepath{stroke}
\pgfpathmoveto{\pgfpoint{83.807999pt}{214.294739pt}}
\pgflineto{\pgfpoint{83.753677pt}{214.294739pt}}
\pgfusepath{stroke}
\pgfpathmoveto{\pgfpoint{83.807999pt}{220.471588pt}}
\pgflineto{\pgfpoint{83.744720pt}{220.471588pt}}
\pgfusepath{stroke}
\pgfpathmoveto{\pgfpoint{83.807999pt}{226.648422pt}}
\pgflineto{\pgfpoint{83.735687pt}{226.648422pt}}
\pgfusepath{stroke}
\pgfpathmoveto{\pgfpoint{83.807999pt}{232.825272pt}}
\pgflineto{\pgfpoint{83.735596pt}{232.825272pt}}
\pgfusepath{stroke}
\pgfpathmoveto{\pgfpoint{83.807999pt}{239.002106pt}}
\pgflineto{\pgfpoint{83.726555pt}{239.002106pt}}
\pgfusepath{stroke}
\pgfpathmoveto{\pgfpoint{83.807999pt}{245.178955pt}}
\pgflineto{\pgfpoint{83.717537pt}{245.178955pt}}
\pgfusepath{stroke}
\pgfpathmoveto{\pgfpoint{83.807999pt}{251.355804pt}}
\pgflineto{\pgfpoint{83.708542pt}{251.355804pt}}
\pgfusepath{stroke}
\pgfpathmoveto{\pgfpoint{83.807999pt}{257.532623pt}}
\pgflineto{\pgfpoint{83.699570pt}{257.532623pt}}
\pgfusepath{stroke}
\pgfpathmoveto{\pgfpoint{83.807999pt}{263.709473pt}}
\pgflineto{\pgfpoint{83.699471pt}{263.709473pt}}
\pgfusepath{stroke}
\pgfpathmoveto{\pgfpoint{83.807999pt}{269.886322pt}}
\pgflineto{\pgfpoint{83.690376pt}{269.886322pt}}
\pgfusepath{stroke}
\pgfpathmoveto{\pgfpoint{83.807999pt}{276.063141pt}}
\pgflineto{\pgfpoint{83.681404pt}{276.063141pt}}
\pgfusepath{stroke}
\pgfpathmoveto{\pgfpoint{83.807999pt}{282.239990pt}}
\pgflineto{\pgfpoint{83.672417pt}{282.239990pt}}
\pgfusepath{stroke}
\pgfpathmoveto{\pgfpoint{83.807999pt}{288.416840pt}}
\pgflineto{\pgfpoint{83.672287pt}{288.416840pt}}
\pgfusepath{stroke}
\pgfpathmoveto{\pgfpoint{83.807999pt}{294.593689pt}}
\pgflineto{\pgfpoint{83.663223pt}{294.593689pt}}
\pgfusepath{stroke}
\pgfpathmoveto{\pgfpoint{83.807999pt}{300.770538pt}}
\pgflineto{\pgfpoint{83.654175pt}{300.770538pt}}
\pgfusepath{stroke}
\pgfpathmoveto{\pgfpoint{83.807999pt}{306.947388pt}}
\pgflineto{\pgfpoint{83.645348pt}{306.947388pt}}
\pgfusepath{stroke}
\pgfpathmoveto{\pgfpoint{83.807999pt}{313.124207pt}}
\pgflineto{\pgfpoint{83.636246pt}{313.124207pt}}
\pgfusepath{stroke}
\pgfpathmoveto{\pgfpoint{83.807999pt}{319.301056pt}}
\pgflineto{\pgfpoint{83.636063pt}{319.301056pt}}
\pgfusepath{stroke}
\pgfpathmoveto{\pgfpoint{83.807999pt}{325.477905pt}}
\pgflineto{\pgfpoint{83.627037pt}{325.477905pt}}
\pgfusepath{stroke}
\pgfpathmoveto{\pgfpoint{83.617996pt}{331.654724pt}}
\pgflineto{\pgfpoint{83.789841pt}{331.654724pt}}
\pgfusepath{stroke}
\pgfpathmoveto{\pgfpoint{83.609169pt}{337.831604pt}}
\pgflineto{\pgfpoint{83.780952pt}{337.831604pt}}
\pgfusepath{stroke}
\pgfpathmoveto{\pgfpoint{83.600113pt}{344.008423pt}}
\pgflineto{\pgfpoint{83.771835pt}{344.008423pt}}
\pgfusepath{stroke}
\pgfpathmoveto{\pgfpoint{83.599892pt}{350.185242pt}}
\pgflineto{\pgfpoint{83.771790pt}{350.185242pt}}
\pgfusepath{stroke}
\pgfpathmoveto{\pgfpoint{83.590851pt}{356.362122pt}}
\pgflineto{\pgfpoint{83.762741pt}{356.362122pt}}
\pgfusepath{stroke}
\pgfpathmoveto{\pgfpoint{83.581810pt}{362.538940pt}}
\pgflineto{\pgfpoint{83.753708pt}{362.538940pt}}
\pgfusepath{stroke}
\pgfpathmoveto{\pgfpoint{83.573021pt}{368.715820pt}}
\pgflineto{\pgfpoint{83.744797pt}{368.715820pt}}
\pgfusepath{stroke}
\pgfpathmoveto{\pgfpoint{83.563980pt}{374.892639pt}}
\pgflineto{\pgfpoint{83.735703pt}{374.892639pt}}
\pgfusepath{stroke}
\pgfpathmoveto{\pgfpoint{83.554955pt}{381.069458pt}}
\pgflineto{\pgfpoint{83.726624pt}{381.069458pt}}
\pgfusepath{stroke}
\pgfpathmoveto{\pgfpoint{83.550301pt}{387.246338pt}}
\pgflineto{\pgfpoint{83.722107pt}{387.246338pt}}
\pgfusepath{stroke}
\pgfpathmoveto{\pgfpoint{83.541206pt}{393.423157pt}}
\pgflineto{\pgfpoint{83.713074pt}{393.423157pt}}
\pgfusepath{stroke}
\pgfpathmoveto{\pgfpoint{83.536606pt}{399.600037pt}}
\pgflineto{\pgfpoint{83.708504pt}{399.600037pt}}
\pgfusepath{stroke}
\pgfpathmoveto{\pgfpoint{83.807999pt}{294.593689pt}}
\pgflineto{\pgfpoint{83.807999pt}{288.416840pt}}
\pgfusepath{stroke}
\pgfpathmoveto{\pgfpoint{83.807999pt}{288.416840pt}}
\pgflineto{\pgfpoint{83.807999pt}{282.239990pt}}
\pgfusepath{stroke}
\pgfpathmoveto{\pgfpoint{83.807999pt}{300.770538pt}}
\pgflineto{\pgfpoint{83.807999pt}{294.593689pt}}
\pgfusepath{stroke}
\pgfpathmoveto{\pgfpoint{83.807999pt}{306.947388pt}}
\pgflineto{\pgfpoint{83.807999pt}{300.770538pt}}
\pgfusepath{stroke}
\pgfpathmoveto{\pgfpoint{83.807999pt}{288.416840pt}}
\pgflineto{\pgfpoint{83.826241pt}{288.416840pt}}
\pgfusepath{stroke}
\pgfpathmoveto{\pgfpoint{83.807999pt}{294.593689pt}}
\pgflineto{\pgfpoint{83.826241pt}{294.593689pt}}
\pgfusepath{stroke}
\pgfpathmoveto{\pgfpoint{83.807999pt}{300.770538pt}}
\pgflineto{\pgfpoint{83.826241pt}{300.770538pt}}
\pgfusepath{stroke}
\pgfpathmoveto{\pgfpoint{83.807999pt}{319.301056pt}}
\pgflineto{\pgfpoint{83.807999pt}{313.124207pt}}
\pgfusepath{stroke}
\pgfpathmoveto{\pgfpoint{83.807999pt}{313.124207pt}}
\pgflineto{\pgfpoint{83.807999pt}{306.947388pt}}
\pgfusepath{stroke}
\pgfpathmoveto{\pgfpoint{83.807999pt}{325.477905pt}}
\pgflineto{\pgfpoint{83.807999pt}{319.301056pt}}
\pgfusepath{stroke}
\pgfpathmoveto{\pgfpoint{83.807999pt}{306.947388pt}}
\pgflineto{\pgfpoint{83.826317pt}{306.947388pt}}
\pgfusepath{stroke}
\pgfpathmoveto{\pgfpoint{83.807999pt}{337.831604pt}}
\pgflineto{\pgfpoint{83.780952pt}{337.831604pt}}
\pgfusepath{stroke}
\pgfpathmoveto{\pgfpoint{83.807999pt}{344.008423pt}}
\pgflineto{\pgfpoint{83.771835pt}{344.008423pt}}
\pgfusepath{stroke}
\pgfpathmoveto{\pgfpoint{83.807999pt}{350.185242pt}}
\pgflineto{\pgfpoint{83.771790pt}{350.185242pt}}
\pgfusepath{stroke}
\pgfpathmoveto{\pgfpoint{83.807999pt}{356.362122pt}}
\pgflineto{\pgfpoint{83.762741pt}{356.362122pt}}
\pgfusepath{stroke}
\pgfpathmoveto{\pgfpoint{83.807999pt}{362.538940pt}}
\pgflineto{\pgfpoint{83.753708pt}{362.538940pt}}
\pgfusepath{stroke}
\pgfpathmoveto{\pgfpoint{83.807999pt}{368.715820pt}}
\pgflineto{\pgfpoint{83.744797pt}{368.715820pt}}
\pgfusepath{stroke}
\pgfpathmoveto{\pgfpoint{83.807999pt}{374.892639pt}}
\pgflineto{\pgfpoint{83.735703pt}{374.892639pt}}
\pgfusepath{stroke}
\pgfpathmoveto{\pgfpoint{83.807999pt}{381.069458pt}}
\pgflineto{\pgfpoint{83.726624pt}{381.069458pt}}
\pgfusepath{stroke}
\pgfpathmoveto{\pgfpoint{83.807999pt}{387.246338pt}}
\pgflineto{\pgfpoint{83.722107pt}{387.246338pt}}
\pgfusepath{stroke}
\pgfpathmoveto{\pgfpoint{83.807999pt}{393.423157pt}}
\pgflineto{\pgfpoint{83.713074pt}{393.423157pt}}
\pgfusepath{stroke}
\pgfpathmoveto{\pgfpoint{83.807999pt}{399.600037pt}}
\pgflineto{\pgfpoint{83.708504pt}{399.600037pt}}
\pgfusepath{stroke}
\pgfpathmoveto{\pgfpoint{83.807999pt}{356.362122pt}}
\pgflineto{\pgfpoint{83.807999pt}{350.185242pt}}
\pgfusepath{stroke}
\pgfpathmoveto{\pgfpoint{83.807999pt}{331.654724pt}}
\pgflineto{\pgfpoint{83.807999pt}{325.477905pt}}
\pgfusepath{stroke}
\pgfpathmoveto{\pgfpoint{83.807999pt}{337.831604pt}}
\pgflineto{\pgfpoint{83.807999pt}{331.654724pt}}
\pgfusepath{stroke}
\pgfpathmoveto{\pgfpoint{83.807999pt}{344.008423pt}}
\pgflineto{\pgfpoint{83.807999pt}{337.831604pt}}
\pgfusepath{stroke}
\pgfpathmoveto{\pgfpoint{83.807999pt}{350.185242pt}}
\pgflineto{\pgfpoint{83.807999pt}{344.008423pt}}
\pgfusepath{stroke}
\pgfpathmoveto{\pgfpoint{83.807999pt}{362.538940pt}}
\pgflineto{\pgfpoint{83.807999pt}{356.362122pt}}
\pgfusepath{stroke}
\pgfpathmoveto{\pgfpoint{83.807999pt}{368.715820pt}}
\pgflineto{\pgfpoint{83.807999pt}{362.538940pt}}
\pgfusepath{stroke}
\pgfpathmoveto{\pgfpoint{83.807999pt}{374.892639pt}}
\pgflineto{\pgfpoint{83.807999pt}{368.715820pt}}
\pgfusepath{stroke}
\pgfpathmoveto{\pgfpoint{83.807999pt}{381.069458pt}}
\pgflineto{\pgfpoint{83.807999pt}{374.892639pt}}
\pgfusepath{stroke}
\pgfpathmoveto{\pgfpoint{83.807999pt}{387.246338pt}}
\pgflineto{\pgfpoint{83.807999pt}{381.069458pt}}
\pgfusepath{stroke}
\pgfpathmoveto{\pgfpoint{83.807999pt}{393.423157pt}}
\pgflineto{\pgfpoint{83.807999pt}{387.246338pt}}
\pgfusepath{stroke}
\pgfpathmoveto{\pgfpoint{83.807999pt}{399.600037pt}}
\pgflineto{\pgfpoint{83.807999pt}{393.423157pt}}
\pgfusepath{stroke}
\pgfpathmoveto{\pgfpoint{83.807999pt}{331.654724pt}}
\pgflineto{\pgfpoint{83.789841pt}{331.654724pt}}
\pgfusepath{stroke}
\pgfpathmoveto{\pgfpoint{83.807999pt}{313.124207pt}}
\pgflineto{\pgfpoint{83.826263pt}{313.124207pt}}
\pgfusepath{stroke}
\pgfpathmoveto{\pgfpoint{83.807999pt}{319.301056pt}}
\pgflineto{\pgfpoint{83.835335pt}{319.301056pt}}
\pgfusepath{stroke}
\pgfpathmoveto{\pgfpoint{83.807999pt}{325.477905pt}}
\pgflineto{\pgfpoint{83.835335pt}{325.477905pt}}
\pgfusepath{stroke}
\pgfpathmoveto{\pgfpoint{83.807999pt}{331.654724pt}}
\pgflineto{\pgfpoint{83.835365pt}{331.654724pt}}
\pgfusepath{stroke}
\pgfpathmoveto{\pgfpoint{83.807999pt}{337.831604pt}}
\pgflineto{\pgfpoint{83.835426pt}{337.831604pt}}
\pgfusepath{stroke}
\pgfpathmoveto{\pgfpoint{83.807999pt}{344.008423pt}}
\pgflineto{\pgfpoint{83.835426pt}{344.008423pt}}
\pgfusepath{stroke}
\pgfpathmoveto{\pgfpoint{83.807999pt}{350.185242pt}}
\pgflineto{\pgfpoint{83.844376pt}{350.185242pt}}
\pgfusepath{stroke}
\pgfpathmoveto{\pgfpoint{83.807999pt}{356.362122pt}}
\pgflineto{\pgfpoint{83.844490pt}{356.362122pt}}
\pgfusepath{stroke}
\pgfpathmoveto{\pgfpoint{83.807999pt}{362.538940pt}}
\pgflineto{\pgfpoint{83.844467pt}{362.538940pt}}
\pgfusepath{stroke}
\pgfpathmoveto{\pgfpoint{83.807999pt}{368.715820pt}}
\pgflineto{\pgfpoint{83.844566pt}{368.715820pt}}
\pgfusepath{stroke}
\pgfpathmoveto{\pgfpoint{83.807999pt}{374.892639pt}}
\pgflineto{\pgfpoint{83.844505pt}{374.892639pt}}
\pgfusepath{stroke}
\pgfpathmoveto{\pgfpoint{83.807999pt}{381.069458pt}}
\pgflineto{\pgfpoint{83.844513pt}{381.069458pt}}
\pgfusepath{stroke}
\pgfpathmoveto{\pgfpoint{83.807999pt}{387.246338pt}}
\pgflineto{\pgfpoint{83.849113pt}{387.246338pt}}
\pgfusepath{stroke}
\pgfpathmoveto{\pgfpoint{83.807999pt}{393.423157pt}}
\pgflineto{\pgfpoint{83.849052pt}{393.423157pt}}
\pgfusepath{stroke}
\pgfpathmoveto{\pgfpoint{83.807999pt}{399.600037pt}}
\pgflineto{\pgfpoint{83.853668pt}{399.600037pt}}
\pgfusepath{stroke}
\pgfpathmoveto{\pgfpoint{83.807999pt}{239.002106pt}}
\pgflineto{\pgfpoint{83.807999pt}{232.825272pt}}
\pgfusepath{stroke}
\pgfpathmoveto{\pgfpoint{83.807999pt}{208.117905pt}}
\pgflineto{\pgfpoint{83.807999pt}{201.941055pt}}
\pgfusepath{stroke}
\pgfpathmoveto{\pgfpoint{83.807999pt}{214.294739pt}}
\pgflineto{\pgfpoint{83.807999pt}{208.117905pt}}
\pgfusepath{stroke}
\pgfpathmoveto{\pgfpoint{83.807999pt}{220.471588pt}}
\pgflineto{\pgfpoint{83.807999pt}{214.294739pt}}
\pgfusepath{stroke}
\pgfpathmoveto{\pgfpoint{83.807999pt}{226.648422pt}}
\pgflineto{\pgfpoint{83.807999pt}{220.471588pt}}
\pgfusepath{stroke}
\pgfpathmoveto{\pgfpoint{83.807999pt}{232.825272pt}}
\pgflineto{\pgfpoint{83.807999pt}{226.648422pt}}
\pgfusepath{stroke}
\pgfpathmoveto{\pgfpoint{83.807999pt}{245.178955pt}}
\pgflineto{\pgfpoint{83.807999pt}{239.002106pt}}
\pgfusepath{stroke}
\pgfpathmoveto{\pgfpoint{83.807999pt}{251.355804pt}}
\pgflineto{\pgfpoint{83.807999pt}{245.178955pt}}
\pgfusepath{stroke}
\pgfpathmoveto{\pgfpoint{83.807999pt}{257.532623pt}}
\pgflineto{\pgfpoint{83.807999pt}{251.355804pt}}
\pgfusepath{stroke}
\pgfpathmoveto{\pgfpoint{83.807999pt}{263.709473pt}}
\pgflineto{\pgfpoint{83.807999pt}{257.532623pt}}
\pgfusepath{stroke}
\pgfpathmoveto{\pgfpoint{83.807999pt}{269.886322pt}}
\pgflineto{\pgfpoint{83.807999pt}{263.709473pt}}
\pgfusepath{stroke}
\pgfpathmoveto{\pgfpoint{83.807999pt}{276.063141pt}}
\pgflineto{\pgfpoint{83.807999pt}{269.886322pt}}
\pgfusepath{stroke}
\pgfpathmoveto{\pgfpoint{83.807999pt}{282.239990pt}}
\pgflineto{\pgfpoint{83.807999pt}{276.063141pt}}
\pgfusepath{stroke}
\pgfpathmoveto{\pgfpoint{83.807999pt}{183.410522pt}}
\pgflineto{\pgfpoint{83.789864pt}{183.410522pt}}
\pgfusepath{stroke}
\pgfpathmoveto{\pgfpoint{83.807999pt}{189.587372pt}}
\pgflineto{\pgfpoint{83.780846pt}{189.587372pt}}
\pgfusepath{stroke}
\pgfpathmoveto{\pgfpoint{83.807999pt}{195.764206pt}}
\pgflineto{\pgfpoint{83.789825pt}{195.764206pt}}
\pgfusepath{stroke}
\pgfpathmoveto{\pgfpoint{83.807999pt}{183.410522pt}}
\pgflineto{\pgfpoint{83.807999pt}{177.233673pt}}
\pgfusepath{stroke}
\pgfpathmoveto{\pgfpoint{83.807999pt}{189.587372pt}}
\pgflineto{\pgfpoint{83.807999pt}{183.410522pt}}
\pgfusepath{stroke}
\pgfpathmoveto{\pgfpoint{83.807999pt}{164.880005pt}}
\pgflineto{\pgfpoint{83.825684pt}{164.880005pt}}
\pgfusepath{stroke}
\pgfpathmoveto{\pgfpoint{83.807999pt}{171.056854pt}}
\pgflineto{\pgfpoint{83.834648pt}{171.056854pt}}
\pgfusepath{stroke}
\pgfpathmoveto{\pgfpoint{83.807999pt}{177.233673pt}}
\pgflineto{\pgfpoint{83.834587pt}{177.233673pt}}
\pgfusepath{stroke}
\pgfpathmoveto{\pgfpoint{83.807999pt}{183.410522pt}}
\pgflineto{\pgfpoint{83.834618pt}{183.410522pt}}
\pgfusepath{stroke}
\pgfpathmoveto{\pgfpoint{83.807999pt}{201.941055pt}}
\pgflineto{\pgfpoint{83.807999pt}{195.764206pt}}
\pgfusepath{stroke}
\pgfpathmoveto{\pgfpoint{83.807999pt}{195.764206pt}}
\pgflineto{\pgfpoint{83.807999pt}{189.587372pt}}
\pgfusepath{stroke}
\pgfpathmoveto{\pgfpoint{83.807999pt}{189.587372pt}}
\pgflineto{\pgfpoint{83.834618pt}{189.587372pt}}
\pgfusepath{stroke}
\pgfpathmoveto{\pgfpoint{83.807999pt}{195.764206pt}}
\pgflineto{\pgfpoint{83.834557pt}{195.764206pt}}
\pgfusepath{stroke}
\pgfpathmoveto{\pgfpoint{83.807999pt}{201.941055pt}}
\pgflineto{\pgfpoint{83.843513pt}{201.941055pt}}
\pgfusepath{stroke}
\pgfpathmoveto{\pgfpoint{83.807999pt}{208.117905pt}}
\pgflineto{\pgfpoint{83.843513pt}{208.117905pt}}
\pgfusepath{stroke}
\pgfpathmoveto{\pgfpoint{83.807999pt}{214.294739pt}}
\pgflineto{\pgfpoint{83.843491pt}{214.294739pt}}
\pgfusepath{stroke}
\pgfpathmoveto{\pgfpoint{83.807999pt}{220.471588pt}}
\pgflineto{\pgfpoint{83.843491pt}{220.471588pt}}
\pgfusepath{stroke}
\pgfpathmoveto{\pgfpoint{83.807999pt}{226.648422pt}}
\pgflineto{\pgfpoint{83.843430pt}{226.648422pt}}
\pgfusepath{stroke}
\pgfpathmoveto{\pgfpoint{83.807999pt}{232.825272pt}}
\pgflineto{\pgfpoint{83.852371pt}{232.825272pt}}
\pgfusepath{stroke}
\pgfpathmoveto{\pgfpoint{83.807999pt}{239.002106pt}}
\pgflineto{\pgfpoint{83.852371pt}{239.002106pt}}
\pgfusepath{stroke}
\pgfpathmoveto{\pgfpoint{83.807999pt}{245.178955pt}}
\pgflineto{\pgfpoint{83.852417pt}{245.178955pt}}
\pgfusepath{stroke}
\pgfpathmoveto{\pgfpoint{83.807999pt}{251.355804pt}}
\pgflineto{\pgfpoint{83.852303pt}{251.355804pt}}
\pgfusepath{stroke}
\pgfpathmoveto{\pgfpoint{83.807999pt}{257.532623pt}}
\pgflineto{\pgfpoint{83.852303pt}{257.532623pt}}
\pgfusepath{stroke}
\pgfpathmoveto{\pgfpoint{83.807999pt}{263.709473pt}}
\pgflineto{\pgfpoint{83.861298pt}{263.709473pt}}
\pgfusepath{stroke}
\pgfpathmoveto{\pgfpoint{83.807999pt}{269.886322pt}}
\pgflineto{\pgfpoint{83.861237pt}{269.886322pt}}
\pgfusepath{stroke}
\pgfpathmoveto{\pgfpoint{83.807999pt}{276.063141pt}}
\pgflineto{\pgfpoint{83.861237pt}{276.063141pt}}
\pgfusepath{stroke}
\pgfpathmoveto{\pgfpoint{83.807999pt}{282.239990pt}}
\pgflineto{\pgfpoint{83.861237pt}{282.239990pt}}
\pgfusepath{stroke}
\pgfpathmoveto{\pgfpoint{83.870216pt}{288.416840pt}}
\pgflineto{\pgfpoint{83.826241pt}{288.416840pt}}
\pgfusepath{stroke}
\pgfpathmoveto{\pgfpoint{83.870102pt}{294.593689pt}}
\pgflineto{\pgfpoint{83.826241pt}{294.593689pt}}
\pgfusepath{stroke}
\pgfpathmoveto{\pgfpoint{83.870102pt}{300.770538pt}}
\pgflineto{\pgfpoint{83.826241pt}{300.770538pt}}
\pgfusepath{stroke}
\pgfpathmoveto{\pgfpoint{83.870163pt}{306.947388pt}}
\pgflineto{\pgfpoint{83.826317pt}{306.947388pt}}
\pgfusepath{stroke}
\pgfpathmoveto{\pgfpoint{83.870110pt}{313.124207pt}}
\pgflineto{\pgfpoint{83.826263pt}{313.124207pt}}
\pgfusepath{stroke}
\pgfpathmoveto{\pgfpoint{83.879021pt}{319.301056pt}}
\pgflineto{\pgfpoint{83.835335pt}{319.301056pt}}
\pgfusepath{stroke}
\pgfpathmoveto{\pgfpoint{83.879021pt}{325.477905pt}}
\pgflineto{\pgfpoint{83.835335pt}{325.477905pt}}
\pgfusepath{stroke}
\pgfpathmoveto{\pgfpoint{83.879036pt}{331.654724pt}}
\pgflineto{\pgfpoint{83.835365pt}{331.654724pt}}
\pgfusepath{stroke}
\pgfpathmoveto{\pgfpoint{83.879036pt}{337.831604pt}}
\pgflineto{\pgfpoint{83.835426pt}{337.831604pt}}
\pgfusepath{stroke}
\pgfpathmoveto{\pgfpoint{83.878975pt}{344.008423pt}}
\pgflineto{\pgfpoint{83.835426pt}{344.008423pt}}
\pgfusepath{stroke}
\pgfpathmoveto{\pgfpoint{83.887886pt}{350.185242pt}}
\pgflineto{\pgfpoint{83.844376pt}{350.185242pt}}
\pgfusepath{stroke}
\pgfpathmoveto{\pgfpoint{83.887886pt}{356.362122pt}}
\pgflineto{\pgfpoint{83.844490pt}{356.362122pt}}
\pgfusepath{stroke}
\pgfpathmoveto{\pgfpoint{83.887909pt}{362.538940pt}}
\pgflineto{\pgfpoint{83.844467pt}{362.538940pt}}
\pgfusepath{stroke}
\pgfpathmoveto{\pgfpoint{83.887871pt}{368.715820pt}}
\pgflineto{\pgfpoint{83.844566pt}{368.715820pt}}
\pgfusepath{stroke}
\pgfpathmoveto{\pgfpoint{83.887810pt}{374.892639pt}}
\pgflineto{\pgfpoint{83.844505pt}{374.892639pt}}
\pgfusepath{stroke}
\pgfpathmoveto{\pgfpoint{83.887756pt}{381.069458pt}}
\pgflineto{\pgfpoint{83.844513pt}{381.069458pt}}
\pgfusepath{stroke}
\pgfpathmoveto{\pgfpoint{83.892303pt}{387.246338pt}}
\pgflineto{\pgfpoint{83.849113pt}{387.246338pt}}
\pgfusepath{stroke}
\pgfpathmoveto{\pgfpoint{83.892242pt}{393.423157pt}}
\pgflineto{\pgfpoint{83.849052pt}{393.423157pt}}
\pgfusepath{stroke}
\pgfpathmoveto{\pgfpoint{83.896797pt}{399.600037pt}}
\pgflineto{\pgfpoint{83.853668pt}{399.600037pt}}
\pgfusepath{stroke}
\pgfpathmoveto{\pgfpoint{83.807999pt}{90.757896pt}}
\pgflineto{\pgfpoint{83.807999pt}{84.581039pt}}
\pgfusepath{stroke}
\pgfpathmoveto{\pgfpoint{83.807999pt}{59.873672pt}}
\pgflineto{\pgfpoint{83.807999pt}{53.696838pt}}
\pgfusepath{stroke}
\pgfpathmoveto{\pgfpoint{83.807999pt}{66.050522pt}}
\pgflineto{\pgfpoint{83.807999pt}{59.873672pt}}
\pgfusepath{stroke}
\pgfpathmoveto{\pgfpoint{83.807999pt}{72.227356pt}}
\pgflineto{\pgfpoint{83.807999pt}{66.050522pt}}
\pgfusepath{stroke}
\pgfpathmoveto{\pgfpoint{83.807999pt}{78.404205pt}}
\pgflineto{\pgfpoint{83.807999pt}{72.227356pt}}
\pgfusepath{stroke}
\pgfpathmoveto{\pgfpoint{83.807999pt}{84.581039pt}}
\pgflineto{\pgfpoint{83.807999pt}{78.404205pt}}
\pgfusepath{stroke}
\pgfpathmoveto{\pgfpoint{83.807999pt}{96.934731pt}}
\pgflineto{\pgfpoint{83.807999pt}{90.757896pt}}
\pgfusepath{stroke}
\pgfpathmoveto{\pgfpoint{83.807999pt}{103.111580pt}}
\pgflineto{\pgfpoint{83.807999pt}{96.934731pt}}
\pgfusepath{stroke}
\pgfpathmoveto{\pgfpoint{83.807999pt}{109.288422pt}}
\pgflineto{\pgfpoint{83.807999pt}{103.111580pt}}
\pgfusepath{stroke}
\pgfpathmoveto{\pgfpoint{83.807999pt}{115.465263pt}}
\pgflineto{\pgfpoint{83.807999pt}{109.288422pt}}
\pgfusepath{stroke}
\pgfpathmoveto{\pgfpoint{83.807999pt}{121.642097pt}}
\pgflineto{\pgfpoint{83.807999pt}{115.465263pt}}
\pgfusepath{stroke}
\pgfpathmoveto{\pgfpoint{83.807999pt}{127.818947pt}}
\pgflineto{\pgfpoint{83.807999pt}{121.642097pt}}
\pgfusepath{stroke}
\pgfpathmoveto{\pgfpoint{83.807999pt}{133.995789pt}}
\pgflineto{\pgfpoint{83.807999pt}{127.818947pt}}
\pgfusepath{stroke}
\pgfpathmoveto{\pgfpoint{83.807999pt}{47.519989pt}}
\pgflineto{\pgfpoint{83.790169pt}{47.519989pt}}
\pgfusepath{stroke}
\pgfpathmoveto{\pgfpoint{83.807999pt}{53.696838pt}}
\pgflineto{\pgfpoint{83.807999pt}{47.519989pt}}
\pgfusepath{stroke}
\pgfpathmoveto{\pgfpoint{83.807999pt}{47.519989pt}}
\pgflineto{\pgfpoint{92.718140pt}{47.519989pt}}
\pgfusepath{stroke}
\pgfpathmoveto{\pgfpoint{83.807999pt}{53.696838pt}}
\pgflineto{\pgfpoint{92.718124pt}{53.696838pt}}
\pgfusepath{stroke}
\pgfpathmoveto{\pgfpoint{83.807999pt}{59.873672pt}}
\pgflineto{\pgfpoint{92.718063pt}{59.873672pt}}
\pgfusepath{stroke}
\pgfpathmoveto{\pgfpoint{92.735992pt}{66.050522pt}}
\pgflineto{\pgfpoint{83.807999pt}{66.050522pt}}
\pgfusepath{stroke}
\pgfpathmoveto{\pgfpoint{92.735992pt}{72.227356pt}}
\pgflineto{\pgfpoint{83.807999pt}{72.227356pt}}
\pgfusepath{stroke}
\pgfpathmoveto{\pgfpoint{92.735992pt}{78.404205pt}}
\pgflineto{\pgfpoint{83.807999pt}{78.404205pt}}
\pgfusepath{stroke}
\pgfpathmoveto{\pgfpoint{92.735992pt}{84.581039pt}}
\pgflineto{\pgfpoint{83.807999pt}{84.581039pt}}
\pgfusepath{stroke}
\pgfpathmoveto{\pgfpoint{92.735992pt}{90.757896pt}}
\pgflineto{\pgfpoint{83.807999pt}{90.757896pt}}
\pgfusepath{stroke}
\pgfpathmoveto{\pgfpoint{92.735992pt}{96.934731pt}}
\pgflineto{\pgfpoint{83.807999pt}{96.934731pt}}
\pgfusepath{stroke}
\pgfpathmoveto{\pgfpoint{92.735992pt}{103.111580pt}}
\pgflineto{\pgfpoint{83.807999pt}{103.111580pt}}
\pgfusepath{stroke}
\pgfpathmoveto{\pgfpoint{92.735992pt}{109.288422pt}}
\pgflineto{\pgfpoint{83.807999pt}{109.288422pt}}
\pgfusepath{stroke}
\pgfpathmoveto{\pgfpoint{92.735992pt}{115.465263pt}}
\pgflineto{\pgfpoint{83.807999pt}{115.465263pt}}
\pgfusepath{stroke}
\pgfpathmoveto{\pgfpoint{92.735992pt}{121.642097pt}}
\pgflineto{\pgfpoint{83.807999pt}{121.642097pt}}
\pgfusepath{stroke}
\pgfpathmoveto{\pgfpoint{92.735992pt}{127.818947pt}}
\pgflineto{\pgfpoint{83.807999pt}{127.818947pt}}
\pgfusepath{stroke}
\pgfpathmoveto{\pgfpoint{92.735992pt}{133.995789pt}}
\pgflineto{\pgfpoint{83.807999pt}{133.995789pt}}
\pgfusepath{stroke}
\pgfpathmoveto{\pgfpoint{92.735992pt}{140.172638pt}}
\pgflineto{\pgfpoint{83.825783pt}{140.172638pt}}
\pgfusepath{stroke}
\pgfpathmoveto{\pgfpoint{92.735992pt}{146.349472pt}}
\pgflineto{\pgfpoint{83.825783pt}{146.349472pt}}
\pgfusepath{stroke}
\pgfpathmoveto{\pgfpoint{92.735992pt}{152.526306pt}}
\pgflineto{\pgfpoint{83.825729pt}{152.526306pt}}
\pgfusepath{stroke}
\pgfpathmoveto{\pgfpoint{92.735992pt}{158.703156pt}}
\pgflineto{\pgfpoint{83.825684pt}{158.703156pt}}
\pgfusepath{stroke}
\pgfpathmoveto{\pgfpoint{92.735992pt}{164.880005pt}}
\pgflineto{\pgfpoint{83.825684pt}{164.880005pt}}
\pgfusepath{stroke}
\pgfpathmoveto{\pgfpoint{92.735992pt}{171.056854pt}}
\pgflineto{\pgfpoint{83.834648pt}{171.056854pt}}
\pgfusepath{stroke}
\pgfpathmoveto{\pgfpoint{92.735992pt}{177.233673pt}}
\pgflineto{\pgfpoint{83.834587pt}{177.233673pt}}
\pgfusepath{stroke}
\pgfpathmoveto{\pgfpoint{92.735992pt}{183.410522pt}}
\pgflineto{\pgfpoint{83.834618pt}{183.410522pt}}
\pgfusepath{stroke}
\pgfpathmoveto{\pgfpoint{92.735992pt}{189.587372pt}}
\pgflineto{\pgfpoint{83.834618pt}{189.587372pt}}
\pgfusepath{stroke}
\pgfpathmoveto{\pgfpoint{92.735992pt}{195.764206pt}}
\pgflineto{\pgfpoint{83.834557pt}{195.764206pt}}
\pgfusepath{stroke}
\pgfpathmoveto{\pgfpoint{92.735992pt}{201.941055pt}}
\pgflineto{\pgfpoint{83.843513pt}{201.941055pt}}
\pgfusepath{stroke}
\pgfpathmoveto{\pgfpoint{92.735992pt}{208.117905pt}}
\pgflineto{\pgfpoint{83.843513pt}{208.117905pt}}
\pgfusepath{stroke}
\pgfpathmoveto{\pgfpoint{92.735992pt}{214.294739pt}}
\pgflineto{\pgfpoint{83.843491pt}{214.294739pt}}
\pgfusepath{stroke}
\pgfpathmoveto{\pgfpoint{92.735992pt}{220.471588pt}}
\pgflineto{\pgfpoint{83.843491pt}{220.471588pt}}
\pgfusepath{stroke}
\pgfpathmoveto{\pgfpoint{83.843430pt}{226.648422pt}}
\pgflineto{\pgfpoint{92.717949pt}{226.648422pt}}
\pgfusepath{stroke}
\pgfpathmoveto{\pgfpoint{83.852371pt}{232.825272pt}}
\pgflineto{\pgfpoint{92.708923pt}{232.825272pt}}
\pgfusepath{stroke}
\pgfpathmoveto{\pgfpoint{83.852371pt}{239.002106pt}}
\pgflineto{\pgfpoint{92.699890pt}{239.002106pt}}
\pgfusepath{stroke}
\pgfpathmoveto{\pgfpoint{83.852417pt}{245.178955pt}}
\pgflineto{\pgfpoint{92.699837pt}{245.178955pt}}
\pgfusepath{stroke}
\pgfpathmoveto{\pgfpoint{83.852303pt}{251.355804pt}}
\pgflineto{\pgfpoint{92.690872pt}{251.355804pt}}
\pgfusepath{stroke}
\pgfpathmoveto{\pgfpoint{83.852303pt}{257.532623pt}}
\pgflineto{\pgfpoint{92.681755pt}{257.532623pt}}
\pgfusepath{stroke}
\pgfpathmoveto{\pgfpoint{83.861298pt}{263.709473pt}}
\pgflineto{\pgfpoint{92.672836pt}{263.709473pt}}
\pgfusepath{stroke}
\pgfpathmoveto{\pgfpoint{83.861237pt}{269.886322pt}}
\pgflineto{\pgfpoint{92.663834pt}{269.886322pt}}
\pgfusepath{stroke}
\pgfpathmoveto{\pgfpoint{83.861237pt}{276.063141pt}}
\pgflineto{\pgfpoint{92.663757pt}{276.063141pt}}
\pgfusepath{stroke}
\pgfpathmoveto{\pgfpoint{83.861237pt}{282.239990pt}}
\pgflineto{\pgfpoint{92.654747pt}{282.239990pt}}
\pgfusepath{stroke}
\pgfpathmoveto{\pgfpoint{83.870216pt}{288.416840pt}}
\pgflineto{\pgfpoint{92.645760pt}{288.416840pt}}
\pgfusepath{stroke}
\pgfpathmoveto{\pgfpoint{83.870102pt}{294.593689pt}}
\pgflineto{\pgfpoint{92.636787pt}{294.593689pt}}
\pgfusepath{stroke}
\pgfpathmoveto{\pgfpoint{83.870102pt}{300.770538pt}}
\pgflineto{\pgfpoint{92.627815pt}{300.770538pt}}
\pgfusepath{stroke}
\pgfpathmoveto{\pgfpoint{83.870163pt}{306.947388pt}}
\pgflineto{\pgfpoint{92.627686pt}{306.947388pt}}
\pgfusepath{stroke}
\pgfpathmoveto{\pgfpoint{83.870110pt}{313.124207pt}}
\pgflineto{\pgfpoint{92.618622pt}{313.124207pt}}
\pgfusepath{stroke}
\pgfpathmoveto{\pgfpoint{83.879021pt}{319.301056pt}}
\pgflineto{\pgfpoint{92.609756pt}{319.301056pt}}
\pgfusepath{stroke}
\pgfpathmoveto{\pgfpoint{83.879021pt}{325.477905pt}}
\pgflineto{\pgfpoint{92.600700pt}{325.477905pt}}
\pgfusepath{stroke}
\pgfpathmoveto{\pgfpoint{83.879036pt}{331.654724pt}}
\pgflineto{\pgfpoint{92.600586pt}{331.654724pt}}
\pgfusepath{stroke}
\pgfpathmoveto{\pgfpoint{83.879036pt}{337.831604pt}}
\pgflineto{\pgfpoint{92.591606pt}{337.831604pt}}
\pgfusepath{stroke}
\pgfpathmoveto{\pgfpoint{83.878975pt}{344.008423pt}}
\pgflineto{\pgfpoint{92.582565pt}{344.008423pt}}
\pgfusepath{stroke}
\pgfpathmoveto{\pgfpoint{83.887886pt}{350.185242pt}}
\pgflineto{\pgfpoint{92.573639pt}{350.185242pt}}
\pgfusepath{stroke}
\pgfpathmoveto{\pgfpoint{83.887886pt}{356.362122pt}}
\pgflineto{\pgfpoint{92.564667pt}{356.362122pt}}
\pgfusepath{stroke}
\pgfpathmoveto{\pgfpoint{83.887909pt}{362.538940pt}}
\pgflineto{\pgfpoint{92.564484pt}{362.538940pt}}
\pgfusepath{stroke}
\pgfpathmoveto{\pgfpoint{83.887871pt}{368.715820pt}}
\pgflineto{\pgfpoint{92.551071pt}{368.715820pt}}
\pgfusepath{stroke}
\pgfpathmoveto{\pgfpoint{83.887810pt}{374.892639pt}}
\pgflineto{\pgfpoint{92.541985pt}{374.892639pt}}
\pgfusepath{stroke}
\pgfpathmoveto{\pgfpoint{83.887756pt}{381.069458pt}}
\pgflineto{\pgfpoint{92.533012pt}{381.069458pt}}
\pgfusepath{stroke}
\pgfpathmoveto{\pgfpoint{83.892303pt}{387.246338pt}}
\pgflineto{\pgfpoint{92.528519pt}{387.246338pt}}
\pgfusepath{stroke}
\pgfpathmoveto{\pgfpoint{83.892242pt}{393.423157pt}}
\pgflineto{\pgfpoint{92.519478pt}{393.423157pt}}
\pgfusepath{stroke}
\pgfpathmoveto{\pgfpoint{83.896797pt}{399.600037pt}}
\pgflineto{\pgfpoint{92.514984pt}{399.600037pt}}
\pgfusepath{stroke}
\pgfpathmoveto{\pgfpoint{92.735992pt}{164.880005pt}}
\pgflineto{\pgfpoint{92.735992pt}{158.703156pt}}
\pgfusepath{stroke}
\pgfpathmoveto{\pgfpoint{92.735992pt}{158.703156pt}}
\pgflineto{\pgfpoint{92.735992pt}{152.526306pt}}
\pgfusepath{stroke}
\pgfpathmoveto{\pgfpoint{92.735992pt}{171.056854pt}}
\pgflineto{\pgfpoint{92.735992pt}{164.880005pt}}
\pgfusepath{stroke}
\pgfpathmoveto{\pgfpoint{92.735992pt}{177.233673pt}}
\pgflineto{\pgfpoint{92.735992pt}{171.056854pt}}
\pgfusepath{stroke}
\pgfpathmoveto{\pgfpoint{92.735992pt}{183.410522pt}}
\pgflineto{\pgfpoint{92.735992pt}{177.233673pt}}
\pgfusepath{stroke}
\pgfpathmoveto{\pgfpoint{92.735992pt}{189.587372pt}}
\pgflineto{\pgfpoint{92.735992pt}{183.410522pt}}
\pgfusepath{stroke}
\pgfpathmoveto{\pgfpoint{92.735992pt}{195.764206pt}}
\pgflineto{\pgfpoint{92.735992pt}{189.587372pt}}
\pgfusepath{stroke}
\pgfpathmoveto{\pgfpoint{92.735992pt}{201.941055pt}}
\pgflineto{\pgfpoint{92.735992pt}{195.764206pt}}
\pgfusepath{stroke}
\pgfpathmoveto{\pgfpoint{92.735992pt}{158.703156pt}}
\pgflineto{\pgfpoint{92.753838pt}{158.703156pt}}
\pgfusepath{stroke}
\pgfpathmoveto{\pgfpoint{92.735992pt}{164.880005pt}}
\pgflineto{\pgfpoint{92.753838pt}{164.880005pt}}
\pgfusepath{stroke}
\pgfpathmoveto{\pgfpoint{92.735992pt}{171.056854pt}}
\pgflineto{\pgfpoint{92.753853pt}{171.056854pt}}
\pgfusepath{stroke}
\pgfpathmoveto{\pgfpoint{92.735992pt}{177.233673pt}}
\pgflineto{\pgfpoint{92.753792pt}{177.233673pt}}
\pgfusepath{stroke}
\pgfpathmoveto{\pgfpoint{92.735992pt}{183.410522pt}}
\pgflineto{\pgfpoint{92.762810pt}{183.410522pt}}
\pgfusepath{stroke}
\pgfpathmoveto{\pgfpoint{92.735992pt}{189.587372pt}}
\pgflineto{\pgfpoint{92.762810pt}{189.587372pt}}
\pgfusepath{stroke}
\pgfpathmoveto{\pgfpoint{92.735992pt}{195.764206pt}}
\pgflineto{\pgfpoint{92.762756pt}{195.764206pt}}
\pgfusepath{stroke}
\pgfpathmoveto{\pgfpoint{92.735992pt}{214.294739pt}}
\pgflineto{\pgfpoint{92.735992pt}{208.117905pt}}
\pgfusepath{stroke}
\pgfpathmoveto{\pgfpoint{92.735992pt}{208.117905pt}}
\pgflineto{\pgfpoint{92.735992pt}{201.941055pt}}
\pgfusepath{stroke}
\pgfpathmoveto{\pgfpoint{92.735992pt}{220.471588pt}}
\pgflineto{\pgfpoint{92.735992pt}{214.294739pt}}
\pgfusepath{stroke}
\pgfpathmoveto{\pgfpoint{92.699890pt}{239.002106pt}}
\pgflineto{\pgfpoint{92.717911pt}{239.002106pt}}
\pgfusepath{stroke}
\pgfpathmoveto{\pgfpoint{92.735992pt}{245.178955pt}}
\pgflineto{\pgfpoint{92.699837pt}{245.178955pt}}
\pgfusepath{stroke}
\pgfpathmoveto{\pgfpoint{92.735992pt}{251.355804pt}}
\pgflineto{\pgfpoint{92.690872pt}{251.355804pt}}
\pgfusepath{stroke}
\pgfpathmoveto{\pgfpoint{92.735992pt}{257.532623pt}}
\pgflineto{\pgfpoint{92.681755pt}{257.532623pt}}
\pgfusepath{stroke}
\pgfpathmoveto{\pgfpoint{92.735992pt}{263.709473pt}}
\pgflineto{\pgfpoint{92.672836pt}{263.709473pt}}
\pgfusepath{stroke}
\pgfpathmoveto{\pgfpoint{92.735992pt}{269.886322pt}}
\pgflineto{\pgfpoint{92.663834pt}{269.886322pt}}
\pgfusepath{stroke}
\pgfpathmoveto{\pgfpoint{92.735992pt}{276.063141pt}}
\pgflineto{\pgfpoint{92.663757pt}{276.063141pt}}
\pgfusepath{stroke}
\pgfpathmoveto{\pgfpoint{92.735992pt}{282.239990pt}}
\pgflineto{\pgfpoint{92.654747pt}{282.239990pt}}
\pgfusepath{stroke}
\pgfpathmoveto{\pgfpoint{92.735992pt}{288.416840pt}}
\pgflineto{\pgfpoint{92.645760pt}{288.416840pt}}
\pgfusepath{stroke}
\pgfpathmoveto{\pgfpoint{92.735992pt}{294.593689pt}}
\pgflineto{\pgfpoint{92.636787pt}{294.593689pt}}
\pgfusepath{stroke}
\pgfpathmoveto{\pgfpoint{92.735992pt}{300.770538pt}}
\pgflineto{\pgfpoint{92.627815pt}{300.770538pt}}
\pgfusepath{stroke}
\pgfpathmoveto{\pgfpoint{92.735992pt}{306.947388pt}}
\pgflineto{\pgfpoint{92.627686pt}{306.947388pt}}
\pgfusepath{stroke}
\pgfpathmoveto{\pgfpoint{92.735992pt}{313.124207pt}}
\pgflineto{\pgfpoint{92.618622pt}{313.124207pt}}
\pgfusepath{stroke}
\pgfpathmoveto{\pgfpoint{92.735992pt}{319.301056pt}}
\pgflineto{\pgfpoint{92.609756pt}{319.301056pt}}
\pgfusepath{stroke}
\pgfpathmoveto{\pgfpoint{92.735992pt}{325.477905pt}}
\pgflineto{\pgfpoint{92.600700pt}{325.477905pt}}
\pgfusepath{stroke}
\pgfpathmoveto{\pgfpoint{92.735992pt}{331.654724pt}}
\pgflineto{\pgfpoint{92.600586pt}{331.654724pt}}
\pgfusepath{stroke}
\pgfpathmoveto{\pgfpoint{92.735992pt}{337.831604pt}}
\pgflineto{\pgfpoint{92.591606pt}{337.831604pt}}
\pgfusepath{stroke}
\pgfpathmoveto{\pgfpoint{92.735992pt}{344.008423pt}}
\pgflineto{\pgfpoint{92.582565pt}{344.008423pt}}
\pgfusepath{stroke}
\pgfpathmoveto{\pgfpoint{92.735992pt}{350.185242pt}}
\pgflineto{\pgfpoint{92.573639pt}{350.185242pt}}
\pgfusepath{stroke}
\pgfpathmoveto{\pgfpoint{92.735992pt}{356.362122pt}}
\pgflineto{\pgfpoint{92.564667pt}{356.362122pt}}
\pgfusepath{stroke}
\pgfpathmoveto{\pgfpoint{92.735992pt}{362.538940pt}}
\pgflineto{\pgfpoint{92.564484pt}{362.538940pt}}
\pgfusepath{stroke}
\pgfpathmoveto{\pgfpoint{92.735992pt}{368.715820pt}}
\pgflineto{\pgfpoint{92.551071pt}{368.715820pt}}
\pgfusepath{stroke}
\pgfpathmoveto{\pgfpoint{92.735992pt}{374.892639pt}}
\pgflineto{\pgfpoint{92.541985pt}{374.892639pt}}
\pgfusepath{stroke}
\pgfpathmoveto{\pgfpoint{92.735992pt}{381.069458pt}}
\pgflineto{\pgfpoint{92.533012pt}{381.069458pt}}
\pgfusepath{stroke}
\pgfpathmoveto{\pgfpoint{92.735992pt}{387.246338pt}}
\pgflineto{\pgfpoint{92.528519pt}{387.246338pt}}
\pgfusepath{stroke}
\pgfpathmoveto{\pgfpoint{92.735992pt}{393.423157pt}}
\pgflineto{\pgfpoint{92.519478pt}{393.423157pt}}
\pgfusepath{stroke}
\pgfpathmoveto{\pgfpoint{92.735992pt}{399.600037pt}}
\pgflineto{\pgfpoint{92.514984pt}{399.600037pt}}
\pgfusepath{stroke}
\pgfpathmoveto{\pgfpoint{92.735992pt}{337.831604pt}}
\pgflineto{\pgfpoint{92.735992pt}{331.654724pt}}
\pgfusepath{stroke}
\pgfpathmoveto{\pgfpoint{92.735992pt}{331.654724pt}}
\pgflineto{\pgfpoint{92.735992pt}{325.477905pt}}
\pgfusepath{stroke}
\pgfpathmoveto{\pgfpoint{92.735992pt}{344.008423pt}}
\pgflineto{\pgfpoint{92.735992pt}{337.831604pt}}
\pgfusepath{stroke}
\pgfpathmoveto{\pgfpoint{92.735992pt}{350.185242pt}}
\pgflineto{\pgfpoint{92.735992pt}{344.008423pt}}
\pgfusepath{stroke}
\pgfpathmoveto{\pgfpoint{92.735992pt}{356.362122pt}}
\pgflineto{\pgfpoint{92.735992pt}{350.185242pt}}
\pgfusepath{stroke}
\pgfpathmoveto{\pgfpoint{92.735992pt}{362.538940pt}}
\pgflineto{\pgfpoint{92.735992pt}{356.362122pt}}
\pgfusepath{stroke}
\pgfpathmoveto{\pgfpoint{92.735992pt}{368.715820pt}}
\pgflineto{\pgfpoint{92.735992pt}{362.538940pt}}
\pgfusepath{stroke}
\pgfpathmoveto{\pgfpoint{92.735992pt}{331.654724pt}}
\pgflineto{\pgfpoint{92.754005pt}{331.654724pt}}
\pgfusepath{stroke}
\pgfpathmoveto{\pgfpoint{92.735992pt}{337.831604pt}}
\pgflineto{\pgfpoint{92.754059pt}{337.831604pt}}
\pgfusepath{stroke}
\pgfpathmoveto{\pgfpoint{92.735992pt}{344.008423pt}}
\pgflineto{\pgfpoint{92.754082pt}{344.008423pt}}
\pgfusepath{stroke}
\pgfpathmoveto{\pgfpoint{92.735992pt}{350.185242pt}}
\pgflineto{\pgfpoint{92.754021pt}{350.185242pt}}
\pgfusepath{stroke}
\pgfpathmoveto{\pgfpoint{92.735992pt}{356.362122pt}}
\pgflineto{\pgfpoint{92.754082pt}{356.362122pt}}
\pgfusepath{stroke}
\pgfpathmoveto{\pgfpoint{92.735992pt}{362.538940pt}}
\pgflineto{\pgfpoint{92.763023pt}{362.538940pt}}
\pgfusepath{stroke}
\pgfpathmoveto{\pgfpoint{92.735992pt}{387.246338pt}}
\pgflineto{\pgfpoint{92.735992pt}{381.069458pt}}
\pgfusepath{stroke}
\pgfpathmoveto{\pgfpoint{92.735992pt}{374.892639pt}}
\pgflineto{\pgfpoint{92.735992pt}{368.715820pt}}
\pgfusepath{stroke}
\pgfpathmoveto{\pgfpoint{92.735992pt}{381.069458pt}}
\pgflineto{\pgfpoint{92.735992pt}{374.892639pt}}
\pgfusepath{stroke}
\pgfpathmoveto{\pgfpoint{92.735992pt}{393.423157pt}}
\pgflineto{\pgfpoint{92.735992pt}{387.246338pt}}
\pgfusepath{stroke}
\pgfpathmoveto{\pgfpoint{92.735992pt}{399.600037pt}}
\pgflineto{\pgfpoint{92.735992pt}{393.423157pt}}
\pgfusepath{stroke}
\pgfpathmoveto{\pgfpoint{92.735992pt}{368.715820pt}}
\pgflineto{\pgfpoint{92.758621pt}{368.715820pt}}
\pgfusepath{stroke}
\pgfpathmoveto{\pgfpoint{92.735992pt}{374.892639pt}}
\pgflineto{\pgfpoint{92.758568pt}{374.892639pt}}
\pgfusepath{stroke}
\pgfpathmoveto{\pgfpoint{92.735992pt}{381.069458pt}}
\pgflineto{\pgfpoint{92.758575pt}{381.069458pt}}
\pgfusepath{stroke}
\pgfpathmoveto{\pgfpoint{92.735992pt}{387.246338pt}}
\pgflineto{\pgfpoint{92.763115pt}{387.246338pt}}
\pgfusepath{stroke}
\pgfpathmoveto{\pgfpoint{92.735992pt}{393.423157pt}}
\pgflineto{\pgfpoint{92.763062pt}{393.423157pt}}
\pgfusepath{stroke}
\pgfpathmoveto{\pgfpoint{92.735992pt}{399.600037pt}}
\pgflineto{\pgfpoint{92.767662pt}{399.600037pt}}
\pgfusepath{stroke}
\pgfpathmoveto{\pgfpoint{92.735992pt}{282.239990pt}}
\pgflineto{\pgfpoint{92.735992pt}{276.063141pt}}
\pgfusepath{stroke}
\pgfpathmoveto{\pgfpoint{92.735992pt}{251.355804pt}}
\pgflineto{\pgfpoint{92.735992pt}{245.178955pt}}
\pgfusepath{stroke}
\pgfpathmoveto{\pgfpoint{92.735992pt}{257.532623pt}}
\pgflineto{\pgfpoint{92.735992pt}{251.355804pt}}
\pgfusepath{stroke}
\pgfpathmoveto{\pgfpoint{92.735992pt}{263.709473pt}}
\pgflineto{\pgfpoint{92.735992pt}{257.532623pt}}
\pgfusepath{stroke}
\pgfpathmoveto{\pgfpoint{92.735992pt}{269.886322pt}}
\pgflineto{\pgfpoint{92.735992pt}{263.709473pt}}
\pgfusepath{stroke}
\pgfpathmoveto{\pgfpoint{92.735992pt}{276.063141pt}}
\pgflineto{\pgfpoint{92.735992pt}{269.886322pt}}
\pgfusepath{stroke}
\pgfpathmoveto{\pgfpoint{92.735992pt}{288.416840pt}}
\pgflineto{\pgfpoint{92.735992pt}{282.239990pt}}
\pgfusepath{stroke}
\pgfpathmoveto{\pgfpoint{92.735992pt}{294.593689pt}}
\pgflineto{\pgfpoint{92.735992pt}{288.416840pt}}
\pgfusepath{stroke}
\pgfpathmoveto{\pgfpoint{92.735992pt}{300.770538pt}}
\pgflineto{\pgfpoint{92.735992pt}{294.593689pt}}
\pgfusepath{stroke}
\pgfpathmoveto{\pgfpoint{92.735992pt}{306.947388pt}}
\pgflineto{\pgfpoint{92.735992pt}{300.770538pt}}
\pgfusepath{stroke}
\pgfpathmoveto{\pgfpoint{92.735992pt}{313.124207pt}}
\pgflineto{\pgfpoint{92.735992pt}{306.947388pt}}
\pgfusepath{stroke}
\pgfpathmoveto{\pgfpoint{92.735992pt}{319.301056pt}}
\pgflineto{\pgfpoint{92.735992pt}{313.124207pt}}
\pgfusepath{stroke}
\pgfpathmoveto{\pgfpoint{92.735992pt}{325.477905pt}}
\pgflineto{\pgfpoint{92.735992pt}{319.301056pt}}
\pgfusepath{stroke}
\pgfpathmoveto{\pgfpoint{92.735992pt}{201.941055pt}}
\pgflineto{\pgfpoint{92.762726pt}{201.941055pt}}
\pgfusepath{stroke}
\pgfpathmoveto{\pgfpoint{92.735992pt}{208.117905pt}}
\pgflineto{\pgfpoint{92.762779pt}{208.117905pt}}
\pgfusepath{stroke}
\pgfpathmoveto{\pgfpoint{92.735992pt}{214.294739pt}}
\pgflineto{\pgfpoint{92.771736pt}{214.294739pt}}
\pgfusepath{stroke}
\pgfpathmoveto{\pgfpoint{92.735992pt}{220.471588pt}}
\pgflineto{\pgfpoint{92.771675pt}{220.471588pt}}
\pgfusepath{stroke}
\pgfpathmoveto{\pgfpoint{92.735992pt}{226.648422pt}}
\pgflineto{\pgfpoint{92.735992pt}{220.471588pt}}
\pgfusepath{stroke}
\pgfpathmoveto{\pgfpoint{92.735992pt}{226.648422pt}}
\pgflineto{\pgfpoint{92.717949pt}{226.648422pt}}
\pgfusepath{stroke}
\pgfpathmoveto{\pgfpoint{92.735992pt}{226.648422pt}}
\pgflineto{\pgfpoint{92.771713pt}{226.648422pt}}
\pgfusepath{stroke}
\pgfpathmoveto{\pgfpoint{92.735992pt}{232.825272pt}}
\pgflineto{\pgfpoint{92.708923pt}{232.825272pt}}
\pgfusepath{stroke}
\pgfpathmoveto{\pgfpoint{92.735992pt}{239.002106pt}}
\pgflineto{\pgfpoint{92.717911pt}{239.002106pt}}
\pgfusepath{stroke}
\pgfpathmoveto{\pgfpoint{92.735992pt}{232.825272pt}}
\pgflineto{\pgfpoint{92.735992pt}{226.648422pt}}
\pgfusepath{stroke}
\pgfpathmoveto{\pgfpoint{92.735992pt}{245.178955pt}}
\pgflineto{\pgfpoint{92.735992pt}{239.002106pt}}
\pgfusepath{stroke}
\pgfpathmoveto{\pgfpoint{92.735992pt}{239.002106pt}}
\pgflineto{\pgfpoint{92.735992pt}{232.825272pt}}
\pgfusepath{stroke}
\pgfpathmoveto{\pgfpoint{92.735992pt}{232.825272pt}}
\pgflineto{\pgfpoint{92.771713pt}{232.825272pt}}
\pgfusepath{stroke}
\pgfpathmoveto{\pgfpoint{92.735992pt}{239.002106pt}}
\pgflineto{\pgfpoint{92.771713pt}{239.002106pt}}
\pgfusepath{stroke}
\pgfpathmoveto{\pgfpoint{92.735992pt}{245.178955pt}}
\pgflineto{\pgfpoint{92.780655pt}{245.178955pt}}
\pgfusepath{stroke}
\pgfpathmoveto{\pgfpoint{92.735992pt}{251.355804pt}}
\pgflineto{\pgfpoint{92.780655pt}{251.355804pt}}
\pgfusepath{stroke}
\pgfpathmoveto{\pgfpoint{92.735992pt}{257.532623pt}}
\pgflineto{\pgfpoint{92.780586pt}{257.532623pt}}
\pgfusepath{stroke}
\pgfpathmoveto{\pgfpoint{92.735992pt}{263.709473pt}}
\pgflineto{\pgfpoint{92.780586pt}{263.709473pt}}
\pgfusepath{stroke}
\pgfpathmoveto{\pgfpoint{92.735992pt}{269.886322pt}}
\pgflineto{\pgfpoint{92.780586pt}{269.886322pt}}
\pgfusepath{stroke}
\pgfpathmoveto{\pgfpoint{92.735992pt}{276.063141pt}}
\pgflineto{\pgfpoint{92.789520pt}{276.063141pt}}
\pgfusepath{stroke}
\pgfpathmoveto{\pgfpoint{92.735992pt}{282.239990pt}}
\pgflineto{\pgfpoint{92.789520pt}{282.239990pt}}
\pgfusepath{stroke}
\pgfpathmoveto{\pgfpoint{92.735992pt}{288.416840pt}}
\pgflineto{\pgfpoint{92.789574pt}{288.416840pt}}
\pgfusepath{stroke}
\pgfpathmoveto{\pgfpoint{92.735992pt}{294.593689pt}}
\pgflineto{\pgfpoint{92.789513pt}{294.593689pt}}
\pgfusepath{stroke}
\pgfpathmoveto{\pgfpoint{92.735992pt}{300.770538pt}}
\pgflineto{\pgfpoint{92.789513pt}{300.770538pt}}
\pgfusepath{stroke}
\pgfpathmoveto{\pgfpoint{92.735992pt}{306.947388pt}}
\pgflineto{\pgfpoint{92.798500pt}{306.947388pt}}
\pgfusepath{stroke}
\pgfpathmoveto{\pgfpoint{92.735992pt}{313.124207pt}}
\pgflineto{\pgfpoint{92.798439pt}{313.124207pt}}
\pgfusepath{stroke}
\pgfpathmoveto{\pgfpoint{92.735992pt}{319.301056pt}}
\pgflineto{\pgfpoint{92.798500pt}{319.301056pt}}
\pgfusepath{stroke}
\pgfpathmoveto{\pgfpoint{92.735992pt}{325.477905pt}}
\pgflineto{\pgfpoint{92.798447pt}{325.477905pt}}
\pgfusepath{stroke}
\pgfpathmoveto{\pgfpoint{92.807358pt}{331.654724pt}}
\pgflineto{\pgfpoint{92.754005pt}{331.654724pt}}
\pgfusepath{stroke}
\pgfpathmoveto{\pgfpoint{92.807419pt}{337.831604pt}}
\pgflineto{\pgfpoint{92.754059pt}{337.831604pt}}
\pgfusepath{stroke}
\pgfpathmoveto{\pgfpoint{92.807434pt}{344.008423pt}}
\pgflineto{\pgfpoint{92.754082pt}{344.008423pt}}
\pgfusepath{stroke}
\pgfpathmoveto{\pgfpoint{92.807373pt}{350.185242pt}}
\pgflineto{\pgfpoint{92.754021pt}{350.185242pt}}
\pgfusepath{stroke}
\pgfpathmoveto{\pgfpoint{92.807434pt}{356.362122pt}}
\pgflineto{\pgfpoint{92.754082pt}{356.362122pt}}
\pgfusepath{stroke}
\pgfpathmoveto{\pgfpoint{92.816299pt}{362.538940pt}}
\pgflineto{\pgfpoint{92.763023pt}{362.538940pt}}
\pgfusepath{stroke}
\pgfpathmoveto{\pgfpoint{92.811829pt}{368.715820pt}}
\pgflineto{\pgfpoint{92.758621pt}{368.715820pt}}
\pgfusepath{stroke}
\pgfpathmoveto{\pgfpoint{92.811829pt}{374.892639pt}}
\pgflineto{\pgfpoint{92.758568pt}{374.892639pt}}
\pgfusepath{stroke}
\pgfpathmoveto{\pgfpoint{92.811813pt}{381.069458pt}}
\pgflineto{\pgfpoint{92.758575pt}{381.069458pt}}
\pgfusepath{stroke}
\pgfpathmoveto{\pgfpoint{92.816261pt}{387.246338pt}}
\pgflineto{\pgfpoint{92.763115pt}{387.246338pt}}
\pgfusepath{stroke}
\pgfpathmoveto{\pgfpoint{92.816261pt}{393.423157pt}}
\pgflineto{\pgfpoint{92.763062pt}{393.423157pt}}
\pgfusepath{stroke}
\pgfpathmoveto{\pgfpoint{92.820755pt}{399.600037pt}}
\pgflineto{\pgfpoint{92.767662pt}{399.600037pt}}
\pgfusepath{stroke}
\pgfpathmoveto{\pgfpoint{92.735992pt}{103.111580pt}}
\pgflineto{\pgfpoint{92.735992pt}{96.934731pt}}
\pgfusepath{stroke}
\pgfpathmoveto{\pgfpoint{92.735992pt}{72.227356pt}}
\pgflineto{\pgfpoint{92.735992pt}{66.050522pt}}
\pgfusepath{stroke}
\pgfpathmoveto{\pgfpoint{92.735992pt}{78.404205pt}}
\pgflineto{\pgfpoint{92.735992pt}{72.227356pt}}
\pgfusepath{stroke}
\pgfpathmoveto{\pgfpoint{92.735992pt}{84.581039pt}}
\pgflineto{\pgfpoint{92.735992pt}{78.404205pt}}
\pgfusepath{stroke}
\pgfpathmoveto{\pgfpoint{92.735992pt}{90.757896pt}}
\pgflineto{\pgfpoint{92.735992pt}{84.581039pt}}
\pgfusepath{stroke}
\pgfpathmoveto{\pgfpoint{92.735992pt}{96.934731pt}}
\pgflineto{\pgfpoint{92.735992pt}{90.757896pt}}
\pgfusepath{stroke}
\pgfpathmoveto{\pgfpoint{92.735992pt}{109.288422pt}}
\pgflineto{\pgfpoint{92.735992pt}{103.111580pt}}
\pgfusepath{stroke}
\pgfpathmoveto{\pgfpoint{92.735992pt}{115.465263pt}}
\pgflineto{\pgfpoint{92.735992pt}{109.288422pt}}
\pgfusepath{stroke}
\pgfpathmoveto{\pgfpoint{92.735992pt}{121.642097pt}}
\pgflineto{\pgfpoint{92.735992pt}{115.465263pt}}
\pgfusepath{stroke}
\pgfpathmoveto{\pgfpoint{92.735992pt}{127.818947pt}}
\pgflineto{\pgfpoint{92.735992pt}{121.642097pt}}
\pgfusepath{stroke}
\pgfpathmoveto{\pgfpoint{92.735992pt}{133.995789pt}}
\pgflineto{\pgfpoint{92.735992pt}{127.818947pt}}
\pgfusepath{stroke}
\pgfpathmoveto{\pgfpoint{92.735992pt}{140.172638pt}}
\pgflineto{\pgfpoint{92.735992pt}{133.995789pt}}
\pgfusepath{stroke}
\pgfpathmoveto{\pgfpoint{92.735992pt}{146.349472pt}}
\pgflineto{\pgfpoint{92.735992pt}{140.172638pt}}
\pgfusepath{stroke}
\pgfpathmoveto{\pgfpoint{92.735992pt}{152.526306pt}}
\pgflineto{\pgfpoint{92.735992pt}{146.349472pt}}
\pgfusepath{stroke}
\pgfpathmoveto{\pgfpoint{92.735992pt}{47.519989pt}}
\pgflineto{\pgfpoint{92.718140pt}{47.519989pt}}
\pgfusepath{stroke}
\pgfpathmoveto{\pgfpoint{92.735992pt}{53.696838pt}}
\pgflineto{\pgfpoint{92.718124pt}{53.696838pt}}
\pgfusepath{stroke}
\pgfpathmoveto{\pgfpoint{92.735992pt}{59.873672pt}}
\pgflineto{\pgfpoint{92.718063pt}{59.873672pt}}
\pgfusepath{stroke}
\pgfpathmoveto{\pgfpoint{92.735992pt}{53.696838pt}}
\pgflineto{\pgfpoint{92.735992pt}{47.519989pt}}
\pgfusepath{stroke}
\pgfpathmoveto{\pgfpoint{92.735992pt}{47.519989pt}}
\pgflineto{\pgfpoint{92.754112pt}{47.519989pt}}
\pgfusepath{stroke}
\pgfpathmoveto{\pgfpoint{92.735992pt}{66.050522pt}}
\pgflineto{\pgfpoint{92.735992pt}{59.873672pt}}
\pgfusepath{stroke}
\pgfpathmoveto{\pgfpoint{92.735992pt}{59.873672pt}}
\pgflineto{\pgfpoint{92.735992pt}{53.696838pt}}
\pgfusepath{stroke}
\pgfpathmoveto{\pgfpoint{92.754112pt}{47.519989pt}}
\pgflineto{\pgfpoint{101.645958pt}{47.519989pt}}
\pgfusepath{stroke}
\pgfpathmoveto{\pgfpoint{101.664001pt}{53.696838pt}}
\pgflineto{\pgfpoint{92.735992pt}{53.696838pt}}
\pgfusepath{stroke}
\pgfpathmoveto{\pgfpoint{101.664001pt}{59.873672pt}}
\pgflineto{\pgfpoint{92.735992pt}{59.873672pt}}
\pgfusepath{stroke}
\pgfpathmoveto{\pgfpoint{101.664001pt}{66.050522pt}}
\pgflineto{\pgfpoint{92.735992pt}{66.050522pt}}
\pgfusepath{stroke}
\pgfpathmoveto{\pgfpoint{101.664001pt}{72.227356pt}}
\pgflineto{\pgfpoint{92.735992pt}{72.227356pt}}
\pgfusepath{stroke}
\pgfpathmoveto{\pgfpoint{101.664001pt}{78.404205pt}}
\pgflineto{\pgfpoint{92.735992pt}{78.404205pt}}
\pgfusepath{stroke}
\pgfpathmoveto{\pgfpoint{101.664001pt}{84.581039pt}}
\pgflineto{\pgfpoint{92.735992pt}{84.581039pt}}
\pgfusepath{stroke}
\pgfpathmoveto{\pgfpoint{101.664001pt}{90.757896pt}}
\pgflineto{\pgfpoint{92.735992pt}{90.757896pt}}
\pgfusepath{stroke}
\pgfpathmoveto{\pgfpoint{101.664001pt}{96.934731pt}}
\pgflineto{\pgfpoint{92.735992pt}{96.934731pt}}
\pgfusepath{stroke}
\pgfpathmoveto{\pgfpoint{101.664001pt}{103.111580pt}}
\pgflineto{\pgfpoint{92.735992pt}{103.111580pt}}
\pgfusepath{stroke}
\pgfpathmoveto{\pgfpoint{101.664001pt}{109.288422pt}}
\pgflineto{\pgfpoint{92.735992pt}{109.288422pt}}
\pgfusepath{stroke}
\pgfpathmoveto{\pgfpoint{101.664001pt}{115.465263pt}}
\pgflineto{\pgfpoint{92.735992pt}{115.465263pt}}
\pgfusepath{stroke}
\pgfpathmoveto{\pgfpoint{101.664001pt}{121.642097pt}}
\pgflineto{\pgfpoint{92.735992pt}{121.642097pt}}
\pgfusepath{stroke}
\pgfpathmoveto{\pgfpoint{101.664001pt}{127.818947pt}}
\pgflineto{\pgfpoint{92.735992pt}{127.818947pt}}
\pgfusepath{stroke}
\pgfpathmoveto{\pgfpoint{101.664001pt}{133.995789pt}}
\pgflineto{\pgfpoint{92.735992pt}{133.995789pt}}
\pgfusepath{stroke}
\pgfpathmoveto{\pgfpoint{101.664001pt}{140.172638pt}}
\pgflineto{\pgfpoint{92.735992pt}{140.172638pt}}
\pgfusepath{stroke}
\pgfpathmoveto{\pgfpoint{101.664001pt}{146.349472pt}}
\pgflineto{\pgfpoint{92.735992pt}{146.349472pt}}
\pgfusepath{stroke}
\pgfpathmoveto{\pgfpoint{101.664001pt}{152.526306pt}}
\pgflineto{\pgfpoint{92.735992pt}{152.526306pt}}
\pgfusepath{stroke}
\pgfpathmoveto{\pgfpoint{101.664001pt}{158.703156pt}}
\pgflineto{\pgfpoint{92.753838pt}{158.703156pt}}
\pgfusepath{stroke}
\pgfpathmoveto{\pgfpoint{101.664001pt}{164.880005pt}}
\pgflineto{\pgfpoint{92.753838pt}{164.880005pt}}
\pgfusepath{stroke}
\pgfpathmoveto{\pgfpoint{101.664001pt}{171.056854pt}}
\pgflineto{\pgfpoint{92.753853pt}{171.056854pt}}
\pgfusepath{stroke}
\pgfpathmoveto{\pgfpoint{101.664001pt}{177.233673pt}}
\pgflineto{\pgfpoint{92.753792pt}{177.233673pt}}
\pgfusepath{stroke}
\pgfpathmoveto{\pgfpoint{92.762810pt}{183.410522pt}}
\pgflineto{\pgfpoint{101.645905pt}{183.410522pt}}
\pgfusepath{stroke}
\pgfpathmoveto{\pgfpoint{92.762810pt}{189.587372pt}}
\pgflineto{\pgfpoint{101.636871pt}{189.587372pt}}
\pgfusepath{stroke}
\pgfpathmoveto{\pgfpoint{92.762756pt}{195.764206pt}}
\pgflineto{\pgfpoint{101.627831pt}{195.764206pt}}
\pgfusepath{stroke}
\pgfpathmoveto{\pgfpoint{92.762726pt}{201.941055pt}}
\pgflineto{\pgfpoint{101.627808pt}{201.941055pt}}
\pgfusepath{stroke}
\pgfpathmoveto{\pgfpoint{92.762779pt}{208.117905pt}}
\pgflineto{\pgfpoint{101.618729pt}{208.117905pt}}
\pgfusepath{stroke}
\pgfpathmoveto{\pgfpoint{92.771736pt}{214.294739pt}}
\pgflineto{\pgfpoint{101.609779pt}{214.294739pt}}
\pgfusepath{stroke}
\pgfpathmoveto{\pgfpoint{92.771675pt}{220.471588pt}}
\pgflineto{\pgfpoint{101.600739pt}{220.471588pt}}
\pgfusepath{stroke}
\pgfpathmoveto{\pgfpoint{92.771713pt}{226.648422pt}}
\pgflineto{\pgfpoint{101.600655pt}{226.648422pt}}
\pgfusepath{stroke}
\pgfpathmoveto{\pgfpoint{92.771713pt}{232.825272pt}}
\pgflineto{\pgfpoint{101.591614pt}{232.825272pt}}
\pgfusepath{stroke}
\pgfpathmoveto{\pgfpoint{92.771713pt}{239.002106pt}}
\pgflineto{\pgfpoint{101.582535pt}{239.002106pt}}
\pgfusepath{stroke}
\pgfpathmoveto{\pgfpoint{92.780655pt}{245.178955pt}}
\pgflineto{\pgfpoint{101.573601pt}{245.178955pt}}
\pgfusepath{stroke}
\pgfpathmoveto{\pgfpoint{92.780655pt}{251.355804pt}}
\pgflineto{\pgfpoint{101.564621pt}{251.355804pt}}
\pgfusepath{stroke}
\pgfpathmoveto{\pgfpoint{92.780586pt}{257.532623pt}}
\pgflineto{\pgfpoint{101.564468pt}{257.532623pt}}
\pgfusepath{stroke}
\pgfpathmoveto{\pgfpoint{92.780586pt}{263.709473pt}}
\pgflineto{\pgfpoint{101.555428pt}{263.709473pt}}
\pgfusepath{stroke}
\pgfpathmoveto{\pgfpoint{92.780586pt}{269.886322pt}}
\pgflineto{\pgfpoint{101.546364pt}{269.886322pt}}
\pgfusepath{stroke}
\pgfpathmoveto{\pgfpoint{92.789520pt}{276.063141pt}}
\pgflineto{\pgfpoint{101.537460pt}{276.063141pt}}
\pgfusepath{stroke}
\pgfpathmoveto{\pgfpoint{92.789520pt}{282.239990pt}}
\pgflineto{\pgfpoint{101.528374pt}{282.239990pt}}
\pgfusepath{stroke}
\pgfpathmoveto{\pgfpoint{92.789574pt}{288.416840pt}}
\pgflineto{\pgfpoint{101.528275pt}{288.416840pt}}
\pgfusepath{stroke}
\pgfpathmoveto{\pgfpoint{92.789513pt}{294.593689pt}}
\pgflineto{\pgfpoint{101.519241pt}{294.593689pt}}
\pgfusepath{stroke}
\pgfpathmoveto{\pgfpoint{92.789513pt}{300.770538pt}}
\pgflineto{\pgfpoint{101.510223pt}{300.770538pt}}
\pgfusepath{stroke}
\pgfpathmoveto{\pgfpoint{92.798500pt}{306.947388pt}}
\pgflineto{\pgfpoint{101.501350pt}{306.947388pt}}
\pgfusepath{stroke}
\pgfpathmoveto{\pgfpoint{92.798439pt}{313.124207pt}}
\pgflineto{\pgfpoint{101.492302pt}{313.124207pt}}
\pgfusepath{stroke}
\pgfpathmoveto{\pgfpoint{92.798500pt}{319.301056pt}}
\pgflineto{\pgfpoint{101.492081pt}{319.301056pt}}
\pgfusepath{stroke}
\pgfpathmoveto{\pgfpoint{92.798447pt}{325.477905pt}}
\pgflineto{\pgfpoint{101.483047pt}{325.477905pt}}
\pgfusepath{stroke}
\pgfpathmoveto{\pgfpoint{92.807358pt}{331.654724pt}}
\pgflineto{\pgfpoint{101.474182pt}{331.654724pt}}
\pgfusepath{stroke}
\pgfpathmoveto{\pgfpoint{92.807419pt}{337.831604pt}}
\pgflineto{\pgfpoint{101.465202pt}{337.831604pt}}
\pgfusepath{stroke}
\pgfpathmoveto{\pgfpoint{92.807434pt}{344.008423pt}}
\pgflineto{\pgfpoint{101.464943pt}{344.008423pt}}
\pgfusepath{stroke}
\pgfpathmoveto{\pgfpoint{92.807373pt}{350.185242pt}}
\pgflineto{\pgfpoint{101.455902pt}{350.185242pt}}
\pgfusepath{stroke}
\pgfpathmoveto{\pgfpoint{92.807434pt}{356.362122pt}}
\pgflineto{\pgfpoint{101.446861pt}{356.362122pt}}
\pgfusepath{stroke}
\pgfpathmoveto{\pgfpoint{92.816299pt}{362.538940pt}}
\pgflineto{\pgfpoint{101.433662pt}{362.538940pt}}
\pgfusepath{stroke}
\pgfpathmoveto{\pgfpoint{92.811829pt}{368.715820pt}}
\pgflineto{\pgfpoint{101.428932pt}{368.715820pt}}
\pgfusepath{stroke}
\pgfpathmoveto{\pgfpoint{92.811829pt}{374.892639pt}}
\pgflineto{\pgfpoint{101.419838pt}{374.892639pt}}
\pgfusepath{stroke}
\pgfpathmoveto{\pgfpoint{92.811813pt}{381.069458pt}}
\pgflineto{\pgfpoint{101.415184pt}{381.069458pt}}
\pgfusepath{stroke}
\pgfpathmoveto{\pgfpoint{92.816261pt}{387.246338pt}}
\pgflineto{\pgfpoint{101.406319pt}{387.246338pt}}
\pgfusepath{stroke}
\pgfpathmoveto{\pgfpoint{92.816261pt}{393.423157pt}}
\pgflineto{\pgfpoint{101.397224pt}{393.423157pt}}
\pgfusepath{stroke}
\pgfpathmoveto{\pgfpoint{92.820755pt}{399.600037pt}}
\pgflineto{\pgfpoint{101.392731pt}{399.600037pt}}
\pgfusepath{stroke}
\pgfpathmoveto{\pgfpoint{101.664001pt}{146.349472pt}}
\pgflineto{\pgfpoint{101.664001pt}{140.172638pt}}
\pgfusepath{stroke}
\pgfpathmoveto{\pgfpoint{101.664001pt}{140.172638pt}}
\pgflineto{\pgfpoint{101.664001pt}{133.995789pt}}
\pgfusepath{stroke}
\pgfpathmoveto{\pgfpoint{101.664001pt}{152.526306pt}}
\pgflineto{\pgfpoint{101.664001pt}{146.349472pt}}
\pgfusepath{stroke}
\pgfpathmoveto{\pgfpoint{101.664001pt}{158.703156pt}}
\pgflineto{\pgfpoint{101.664001pt}{152.526306pt}}
\pgfusepath{stroke}
\pgfpathmoveto{\pgfpoint{101.664001pt}{140.172638pt}}
\pgflineto{\pgfpoint{101.682014pt}{140.172638pt}}
\pgfusepath{stroke}
\pgfpathmoveto{\pgfpoint{101.664001pt}{146.349472pt}}
\pgflineto{\pgfpoint{101.682014pt}{146.349472pt}}
\pgfusepath{stroke}
\pgfpathmoveto{\pgfpoint{101.664001pt}{152.526306pt}}
\pgflineto{\pgfpoint{101.682037pt}{152.526306pt}}
\pgfusepath{stroke}
\pgfpathmoveto{\pgfpoint{101.664001pt}{171.056854pt}}
\pgflineto{\pgfpoint{101.664001pt}{164.880005pt}}
\pgfusepath{stroke}
\pgfpathmoveto{\pgfpoint{101.664001pt}{164.880005pt}}
\pgflineto{\pgfpoint{101.664001pt}{158.703156pt}}
\pgfusepath{stroke}
\pgfpathmoveto{\pgfpoint{101.664001pt}{177.233673pt}}
\pgflineto{\pgfpoint{101.664001pt}{171.056854pt}}
\pgfusepath{stroke}
\pgfpathmoveto{\pgfpoint{101.664001pt}{158.703156pt}}
\pgflineto{\pgfpoint{101.682037pt}{158.703156pt}}
\pgfusepath{stroke}
\pgfpathmoveto{\pgfpoint{101.627831pt}{195.764206pt}}
\pgflineto{\pgfpoint{101.645966pt}{195.764206pt}}
\pgfusepath{stroke}
\pgfpathmoveto{\pgfpoint{101.664001pt}{201.941055pt}}
\pgflineto{\pgfpoint{101.627808pt}{201.941055pt}}
\pgfusepath{stroke}
\pgfpathmoveto{\pgfpoint{101.664001pt}{208.117905pt}}
\pgflineto{\pgfpoint{101.618729pt}{208.117905pt}}
\pgfusepath{stroke}
\pgfpathmoveto{\pgfpoint{101.664001pt}{214.294739pt}}
\pgflineto{\pgfpoint{101.609779pt}{214.294739pt}}
\pgfusepath{stroke}
\pgfpathmoveto{\pgfpoint{101.664001pt}{220.471588pt}}
\pgflineto{\pgfpoint{101.600739pt}{220.471588pt}}
\pgfusepath{stroke}
\pgfpathmoveto{\pgfpoint{101.664001pt}{226.648422pt}}
\pgflineto{\pgfpoint{101.600655pt}{226.648422pt}}
\pgfusepath{stroke}
\pgfpathmoveto{\pgfpoint{101.664001pt}{232.825272pt}}
\pgflineto{\pgfpoint{101.591614pt}{232.825272pt}}
\pgfusepath{stroke}
\pgfpathmoveto{\pgfpoint{101.664001pt}{239.002106pt}}
\pgflineto{\pgfpoint{101.582535pt}{239.002106pt}}
\pgfusepath{stroke}
\pgfpathmoveto{\pgfpoint{101.664001pt}{245.178955pt}}
\pgflineto{\pgfpoint{101.573601pt}{245.178955pt}}
\pgfusepath{stroke}
\pgfpathmoveto{\pgfpoint{101.664001pt}{251.355804pt}}
\pgflineto{\pgfpoint{101.564621pt}{251.355804pt}}
\pgfusepath{stroke}
\pgfpathmoveto{\pgfpoint{101.664001pt}{257.532623pt}}
\pgflineto{\pgfpoint{101.564468pt}{257.532623pt}}
\pgfusepath{stroke}
\pgfpathmoveto{\pgfpoint{101.664001pt}{263.709473pt}}
\pgflineto{\pgfpoint{101.555428pt}{263.709473pt}}
\pgfusepath{stroke}
\pgfpathmoveto{\pgfpoint{101.664001pt}{269.886322pt}}
\pgflineto{\pgfpoint{101.546364pt}{269.886322pt}}
\pgfusepath{stroke}
\pgfpathmoveto{\pgfpoint{101.664001pt}{276.063141pt}}
\pgflineto{\pgfpoint{101.537460pt}{276.063141pt}}
\pgfusepath{stroke}
\pgfpathmoveto{\pgfpoint{101.664001pt}{282.239990pt}}
\pgflineto{\pgfpoint{101.528374pt}{282.239990pt}}
\pgfusepath{stroke}
\pgfpathmoveto{\pgfpoint{101.664001pt}{288.416840pt}}
\pgflineto{\pgfpoint{101.528275pt}{288.416840pt}}
\pgfusepath{stroke}
\pgfpathmoveto{\pgfpoint{101.664001pt}{294.593689pt}}
\pgflineto{\pgfpoint{101.519241pt}{294.593689pt}}
\pgfusepath{stroke}
\pgfpathmoveto{\pgfpoint{101.664001pt}{300.770538pt}}
\pgflineto{\pgfpoint{101.510223pt}{300.770538pt}}
\pgfusepath{stroke}
\pgfpathmoveto{\pgfpoint{101.664001pt}{306.947388pt}}
\pgflineto{\pgfpoint{101.501350pt}{306.947388pt}}
\pgfusepath{stroke}
\pgfpathmoveto{\pgfpoint{101.664001pt}{313.124207pt}}
\pgflineto{\pgfpoint{101.492302pt}{313.124207pt}}
\pgfusepath{stroke}
\pgfpathmoveto{\pgfpoint{101.664001pt}{319.301056pt}}
\pgflineto{\pgfpoint{101.492081pt}{319.301056pt}}
\pgfusepath{stroke}
\pgfpathmoveto{\pgfpoint{101.664001pt}{325.477905pt}}
\pgflineto{\pgfpoint{101.483047pt}{325.477905pt}}
\pgfusepath{stroke}
\pgfpathmoveto{\pgfpoint{101.664001pt}{331.654724pt}}
\pgflineto{\pgfpoint{101.474182pt}{331.654724pt}}
\pgfusepath{stroke}
\pgfpathmoveto{\pgfpoint{101.664001pt}{337.831604pt}}
\pgflineto{\pgfpoint{101.465202pt}{337.831604pt}}
\pgfusepath{stroke}
\pgfpathmoveto{\pgfpoint{101.664001pt}{344.008423pt}}
\pgflineto{\pgfpoint{101.464943pt}{344.008423pt}}
\pgfusepath{stroke}
\pgfpathmoveto{\pgfpoint{101.664001pt}{350.185242pt}}
\pgflineto{\pgfpoint{101.455902pt}{350.185242pt}}
\pgfusepath{stroke}
\pgfpathmoveto{\pgfpoint{101.446861pt}{356.362122pt}}
\pgflineto{\pgfpoint{101.645996pt}{356.362122pt}}
\pgfusepath{stroke}
\pgfpathmoveto{\pgfpoint{101.433662pt}{362.538940pt}}
\pgflineto{\pgfpoint{101.632492pt}{362.538940pt}}
\pgfusepath{stroke}
\pgfpathmoveto{\pgfpoint{101.428932pt}{368.715820pt}}
\pgflineto{\pgfpoint{101.628029pt}{368.715820pt}}
\pgfusepath{stroke}
\pgfpathmoveto{\pgfpoint{101.419838pt}{374.892639pt}}
\pgflineto{\pgfpoint{101.618988pt}{374.892639pt}}
\pgfusepath{stroke}
\pgfpathmoveto{\pgfpoint{101.415184pt}{381.069458pt}}
\pgflineto{\pgfpoint{101.614441pt}{381.069458pt}}
\pgfusepath{stroke}
\pgfpathmoveto{\pgfpoint{101.406319pt}{387.246338pt}}
\pgflineto{\pgfpoint{101.605530pt}{387.246338pt}}
\pgfusepath{stroke}
\pgfpathmoveto{\pgfpoint{101.397224pt}{393.423157pt}}
\pgflineto{\pgfpoint{101.596489pt}{393.423157pt}}
\pgfusepath{stroke}
\pgfpathmoveto{\pgfpoint{101.392731pt}{399.600037pt}}
\pgflineto{\pgfpoint{101.592056pt}{399.600037pt}}
\pgfusepath{stroke}
\pgfpathmoveto{\pgfpoint{101.664001pt}{294.593689pt}}
\pgflineto{\pgfpoint{101.664001pt}{288.416840pt}}
\pgfusepath{stroke}
\pgfpathmoveto{\pgfpoint{101.664001pt}{288.416840pt}}
\pgflineto{\pgfpoint{101.664001pt}{282.239990pt}}
\pgfusepath{stroke}
\pgfpathmoveto{\pgfpoint{101.664001pt}{300.770538pt}}
\pgflineto{\pgfpoint{101.664001pt}{294.593689pt}}
\pgfusepath{stroke}
\pgfpathmoveto{\pgfpoint{101.664001pt}{306.947388pt}}
\pgflineto{\pgfpoint{101.664001pt}{300.770538pt}}
\pgfusepath{stroke}
\pgfpathmoveto{\pgfpoint{101.664001pt}{313.124207pt}}
\pgflineto{\pgfpoint{101.664001pt}{306.947388pt}}
\pgfusepath{stroke}
\pgfpathmoveto{\pgfpoint{101.664001pt}{319.301056pt}}
\pgflineto{\pgfpoint{101.664001pt}{313.124207pt}}
\pgfusepath{stroke}
\pgfpathmoveto{\pgfpoint{101.664001pt}{325.477905pt}}
\pgflineto{\pgfpoint{101.664001pt}{319.301056pt}}
\pgfusepath{stroke}
\pgfpathmoveto{\pgfpoint{101.664001pt}{331.654724pt}}
\pgflineto{\pgfpoint{101.664001pt}{325.477905pt}}
\pgfusepath{stroke}
\pgfpathmoveto{\pgfpoint{101.664001pt}{288.416840pt}}
\pgflineto{\pgfpoint{101.682129pt}{288.416840pt}}
\pgfusepath{stroke}
\pgfpathmoveto{\pgfpoint{101.664001pt}{294.593689pt}}
\pgflineto{\pgfpoint{101.682076pt}{294.593689pt}}
\pgfusepath{stroke}
\pgfpathmoveto{\pgfpoint{101.664001pt}{300.770538pt}}
\pgflineto{\pgfpoint{101.682152pt}{300.770538pt}}
\pgfusepath{stroke}
\pgfpathmoveto{\pgfpoint{101.664001pt}{306.947388pt}}
\pgflineto{\pgfpoint{101.682152pt}{306.947388pt}}
\pgfusepath{stroke}
\pgfpathmoveto{\pgfpoint{101.664001pt}{313.124207pt}}
\pgflineto{\pgfpoint{101.682152pt}{313.124207pt}}
\pgfusepath{stroke}
\pgfpathmoveto{\pgfpoint{101.664001pt}{319.301056pt}}
\pgflineto{\pgfpoint{101.691170pt}{319.301056pt}}
\pgfusepath{stroke}
\pgfpathmoveto{\pgfpoint{101.664001pt}{325.477905pt}}
\pgflineto{\pgfpoint{101.691109pt}{325.477905pt}}
\pgfusepath{stroke}
\pgfpathmoveto{\pgfpoint{101.664001pt}{344.008423pt}}
\pgflineto{\pgfpoint{101.664001pt}{337.831604pt}}
\pgfusepath{stroke}
\pgfpathmoveto{\pgfpoint{101.664001pt}{337.831604pt}}
\pgflineto{\pgfpoint{101.664001pt}{331.654724pt}}
\pgfusepath{stroke}
\pgfpathmoveto{\pgfpoint{101.664001pt}{350.185242pt}}
\pgflineto{\pgfpoint{101.664001pt}{344.008423pt}}
\pgfusepath{stroke}
\pgfpathmoveto{\pgfpoint{101.664001pt}{356.362122pt}}
\pgflineto{\pgfpoint{101.645996pt}{356.362122pt}}
\pgfusepath{stroke}
\pgfpathmoveto{\pgfpoint{101.664001pt}{362.538940pt}}
\pgflineto{\pgfpoint{101.632492pt}{362.538940pt}}
\pgfusepath{stroke}
\pgfpathmoveto{\pgfpoint{101.664001pt}{368.715820pt}}
\pgflineto{\pgfpoint{101.628029pt}{368.715820pt}}
\pgfusepath{stroke}
\pgfpathmoveto{\pgfpoint{101.664001pt}{374.892639pt}}
\pgflineto{\pgfpoint{101.618988pt}{374.892639pt}}
\pgfusepath{stroke}
\pgfpathmoveto{\pgfpoint{101.664001pt}{381.069458pt}}
\pgflineto{\pgfpoint{101.614441pt}{381.069458pt}}
\pgfusepath{stroke}
\pgfpathmoveto{\pgfpoint{101.664001pt}{387.246338pt}}
\pgflineto{\pgfpoint{101.605530pt}{387.246338pt}}
\pgfusepath{stroke}
\pgfpathmoveto{\pgfpoint{101.664001pt}{393.423157pt}}
\pgflineto{\pgfpoint{101.596489pt}{393.423157pt}}
\pgfusepath{stroke}
\pgfpathmoveto{\pgfpoint{101.664001pt}{399.600037pt}}
\pgflineto{\pgfpoint{101.592056pt}{399.600037pt}}
\pgfusepath{stroke}
\pgfpathmoveto{\pgfpoint{101.664001pt}{356.362122pt}}
\pgflineto{\pgfpoint{101.664001pt}{350.185242pt}}
\pgfusepath{stroke}
\pgfpathmoveto{\pgfpoint{101.664001pt}{362.538940pt}}
\pgflineto{\pgfpoint{101.664001pt}{356.362122pt}}
\pgfusepath{stroke}
\pgfpathmoveto{\pgfpoint{101.664001pt}{368.715820pt}}
\pgflineto{\pgfpoint{101.664001pt}{362.538940pt}}
\pgfusepath{stroke}
\pgfpathmoveto{\pgfpoint{101.664001pt}{374.892639pt}}
\pgflineto{\pgfpoint{101.664001pt}{368.715820pt}}
\pgfusepath{stroke}
\pgfpathmoveto{\pgfpoint{101.664001pt}{381.069458pt}}
\pgflineto{\pgfpoint{101.664001pt}{374.892639pt}}
\pgfusepath{stroke}
\pgfpathmoveto{\pgfpoint{101.664001pt}{387.246338pt}}
\pgflineto{\pgfpoint{101.664001pt}{381.069458pt}}
\pgfusepath{stroke}
\pgfpathmoveto{\pgfpoint{101.664001pt}{393.423157pt}}
\pgflineto{\pgfpoint{101.664001pt}{387.246338pt}}
\pgfusepath{stroke}
\pgfpathmoveto{\pgfpoint{101.664001pt}{399.600037pt}}
\pgflineto{\pgfpoint{101.664001pt}{393.423157pt}}
\pgfusepath{stroke}
\pgfpathmoveto{\pgfpoint{101.664001pt}{331.654724pt}}
\pgflineto{\pgfpoint{101.691139pt}{331.654724pt}}
\pgfusepath{stroke}
\pgfpathmoveto{\pgfpoint{101.664001pt}{337.831604pt}}
\pgflineto{\pgfpoint{101.691193pt}{337.831604pt}}
\pgfusepath{stroke}
\pgfpathmoveto{\pgfpoint{101.664001pt}{344.008423pt}}
\pgflineto{\pgfpoint{101.700142pt}{344.008423pt}}
\pgfusepath{stroke}
\pgfpathmoveto{\pgfpoint{101.664001pt}{350.185242pt}}
\pgflineto{\pgfpoint{101.700089pt}{350.185242pt}}
\pgfusepath{stroke}
\pgfpathmoveto{\pgfpoint{101.664001pt}{356.362122pt}}
\pgflineto{\pgfpoint{101.700203pt}{356.362122pt}}
\pgfusepath{stroke}
\pgfpathmoveto{\pgfpoint{101.664001pt}{362.538940pt}}
\pgflineto{\pgfpoint{101.695618pt}{362.538940pt}}
\pgfusepath{stroke}
\pgfpathmoveto{\pgfpoint{101.664001pt}{368.715820pt}}
\pgflineto{\pgfpoint{101.700218pt}{368.715820pt}}
\pgfusepath{stroke}
\pgfpathmoveto{\pgfpoint{101.664001pt}{374.892639pt}}
\pgflineto{\pgfpoint{101.700218pt}{374.892639pt}}
\pgfusepath{stroke}
\pgfpathmoveto{\pgfpoint{101.664001pt}{381.069458pt}}
\pgflineto{\pgfpoint{101.704659pt}{381.069458pt}}
\pgfusepath{stroke}
\pgfpathmoveto{\pgfpoint{101.664001pt}{387.246338pt}}
\pgflineto{\pgfpoint{101.704712pt}{387.246338pt}}
\pgfusepath{stroke}
\pgfpathmoveto{\pgfpoint{101.664001pt}{393.423157pt}}
\pgflineto{\pgfpoint{101.704712pt}{393.423157pt}}
\pgfusepath{stroke}
\pgfpathmoveto{\pgfpoint{101.664001pt}{399.600037pt}}
\pgflineto{\pgfpoint{101.709259pt}{399.600037pt}}
\pgfusepath{stroke}
\pgfpathmoveto{\pgfpoint{101.664001pt}{232.825272pt}}
\pgflineto{\pgfpoint{101.664001pt}{226.648422pt}}
\pgfusepath{stroke}
\pgfpathmoveto{\pgfpoint{101.664001pt}{208.117905pt}}
\pgflineto{\pgfpoint{101.664001pt}{201.941055pt}}
\pgfusepath{stroke}
\pgfpathmoveto{\pgfpoint{101.664001pt}{214.294739pt}}
\pgflineto{\pgfpoint{101.664001pt}{208.117905pt}}
\pgfusepath{stroke}
\pgfpathmoveto{\pgfpoint{101.664001pt}{220.471588pt}}
\pgflineto{\pgfpoint{101.664001pt}{214.294739pt}}
\pgfusepath{stroke}
\pgfpathmoveto{\pgfpoint{101.664001pt}{226.648422pt}}
\pgflineto{\pgfpoint{101.664001pt}{220.471588pt}}
\pgfusepath{stroke}
\pgfpathmoveto{\pgfpoint{101.664001pt}{239.002106pt}}
\pgflineto{\pgfpoint{101.664001pt}{232.825272pt}}
\pgfusepath{stroke}
\pgfpathmoveto{\pgfpoint{101.664001pt}{245.178955pt}}
\pgflineto{\pgfpoint{101.664001pt}{239.002106pt}}
\pgfusepath{stroke}
\pgfpathmoveto{\pgfpoint{101.664001pt}{251.355804pt}}
\pgflineto{\pgfpoint{101.664001pt}{245.178955pt}}
\pgfusepath{stroke}
\pgfpathmoveto{\pgfpoint{101.664001pt}{257.532623pt}}
\pgflineto{\pgfpoint{101.664001pt}{251.355804pt}}
\pgfusepath{stroke}
\pgfpathmoveto{\pgfpoint{101.664001pt}{263.709473pt}}
\pgflineto{\pgfpoint{101.664001pt}{257.532623pt}}
\pgfusepath{stroke}
\pgfpathmoveto{\pgfpoint{101.664001pt}{269.886322pt}}
\pgflineto{\pgfpoint{101.664001pt}{263.709473pt}}
\pgfusepath{stroke}
\pgfpathmoveto{\pgfpoint{101.664001pt}{276.063141pt}}
\pgflineto{\pgfpoint{101.664001pt}{269.886322pt}}
\pgfusepath{stroke}
\pgfpathmoveto{\pgfpoint{101.664001pt}{282.239990pt}}
\pgflineto{\pgfpoint{101.664001pt}{276.063141pt}}
\pgfusepath{stroke}
\pgfpathmoveto{\pgfpoint{101.664001pt}{183.410522pt}}
\pgflineto{\pgfpoint{101.645905pt}{183.410522pt}}
\pgfusepath{stroke}
\pgfpathmoveto{\pgfpoint{101.664001pt}{189.587372pt}}
\pgflineto{\pgfpoint{101.636871pt}{189.587372pt}}
\pgfusepath{stroke}
\pgfpathmoveto{\pgfpoint{101.664001pt}{195.764206pt}}
\pgflineto{\pgfpoint{101.645966pt}{195.764206pt}}
\pgfusepath{stroke}
\pgfpathmoveto{\pgfpoint{101.664001pt}{183.410522pt}}
\pgflineto{\pgfpoint{101.664001pt}{177.233673pt}}
\pgfusepath{stroke}
\pgfpathmoveto{\pgfpoint{101.664001pt}{189.587372pt}}
\pgflineto{\pgfpoint{101.664001pt}{183.410522pt}}
\pgfusepath{stroke}
\pgfpathmoveto{\pgfpoint{101.664001pt}{164.880005pt}}
\pgflineto{\pgfpoint{101.682037pt}{164.880005pt}}
\pgfusepath{stroke}
\pgfpathmoveto{\pgfpoint{101.664001pt}{171.056854pt}}
\pgflineto{\pgfpoint{101.690994pt}{171.056854pt}}
\pgfusepath{stroke}
\pgfpathmoveto{\pgfpoint{101.664001pt}{177.233673pt}}
\pgflineto{\pgfpoint{101.690994pt}{177.233673pt}}
\pgfusepath{stroke}
\pgfpathmoveto{\pgfpoint{101.664001pt}{183.410522pt}}
\pgflineto{\pgfpoint{101.691025pt}{183.410522pt}}
\pgfusepath{stroke}
\pgfpathmoveto{\pgfpoint{101.664001pt}{201.941055pt}}
\pgflineto{\pgfpoint{101.664001pt}{195.764206pt}}
\pgfusepath{stroke}
\pgfpathmoveto{\pgfpoint{101.664001pt}{195.764206pt}}
\pgflineto{\pgfpoint{101.664001pt}{189.587372pt}}
\pgfusepath{stroke}
\pgfpathmoveto{\pgfpoint{101.664001pt}{189.587372pt}}
\pgflineto{\pgfpoint{101.691025pt}{189.587372pt}}
\pgfusepath{stroke}
\pgfpathmoveto{\pgfpoint{101.664001pt}{195.764206pt}}
\pgflineto{\pgfpoint{101.691025pt}{195.764206pt}}
\pgfusepath{stroke}
\pgfpathmoveto{\pgfpoint{101.664001pt}{201.941055pt}}
\pgflineto{\pgfpoint{101.700027pt}{201.941055pt}}
\pgfusepath{stroke}
\pgfpathmoveto{\pgfpoint{101.664001pt}{208.117905pt}}
\pgflineto{\pgfpoint{101.700027pt}{208.117905pt}}
\pgfusepath{stroke}
\pgfpathmoveto{\pgfpoint{101.664001pt}{214.294739pt}}
\pgflineto{\pgfpoint{101.700066pt}{214.294739pt}}
\pgfusepath{stroke}
\pgfpathmoveto{\pgfpoint{101.664001pt}{220.471588pt}}
\pgflineto{\pgfpoint{101.700066pt}{220.471588pt}}
\pgfusepath{stroke}
\pgfpathmoveto{\pgfpoint{101.664001pt}{226.648422pt}}
\pgflineto{\pgfpoint{101.709068pt}{226.648422pt}}
\pgfusepath{stroke}
\pgfpathmoveto{\pgfpoint{101.664001pt}{232.825272pt}}
\pgflineto{\pgfpoint{101.709007pt}{232.825272pt}}
\pgfusepath{stroke}
\pgfpathmoveto{\pgfpoint{101.664001pt}{239.002106pt}}
\pgflineto{\pgfpoint{101.709114pt}{239.002106pt}}
\pgfusepath{stroke}
\pgfpathmoveto{\pgfpoint{101.664001pt}{245.178955pt}}
\pgflineto{\pgfpoint{101.709053pt}{245.178955pt}}
\pgfusepath{stroke}
\pgfpathmoveto{\pgfpoint{101.664001pt}{251.355804pt}}
\pgflineto{\pgfpoint{101.709053pt}{251.355804pt}}
\pgfusepath{stroke}
\pgfpathmoveto{\pgfpoint{101.664001pt}{257.532623pt}}
\pgflineto{\pgfpoint{101.718048pt}{257.532623pt}}
\pgfusepath{stroke}
\pgfpathmoveto{\pgfpoint{101.664001pt}{263.709473pt}}
\pgflineto{\pgfpoint{101.718048pt}{263.709473pt}}
\pgfusepath{stroke}
\pgfpathmoveto{\pgfpoint{101.664001pt}{269.886322pt}}
\pgflineto{\pgfpoint{101.718102pt}{269.886322pt}}
\pgfusepath{stroke}
\pgfpathmoveto{\pgfpoint{101.664001pt}{276.063141pt}}
\pgflineto{\pgfpoint{101.718040pt}{276.063141pt}}
\pgfusepath{stroke}
\pgfpathmoveto{\pgfpoint{101.664001pt}{282.239990pt}}
\pgflineto{\pgfpoint{101.718040pt}{282.239990pt}}
\pgfusepath{stroke}
\pgfpathmoveto{\pgfpoint{101.727028pt}{288.416840pt}}
\pgflineto{\pgfpoint{101.682129pt}{288.416840pt}}
\pgfusepath{stroke}
\pgfpathmoveto{\pgfpoint{101.727028pt}{294.593689pt}}
\pgflineto{\pgfpoint{101.682076pt}{294.593689pt}}
\pgfusepath{stroke}
\pgfpathmoveto{\pgfpoint{101.727089pt}{300.770538pt}}
\pgflineto{\pgfpoint{101.682152pt}{300.770538pt}}
\pgfusepath{stroke}
\pgfpathmoveto{\pgfpoint{101.727089pt}{306.947388pt}}
\pgflineto{\pgfpoint{101.682152pt}{306.947388pt}}
\pgfusepath{stroke}
\pgfpathmoveto{\pgfpoint{101.727089pt}{313.124207pt}}
\pgflineto{\pgfpoint{101.682152pt}{313.124207pt}}
\pgfusepath{stroke}
\pgfpathmoveto{\pgfpoint{101.736061pt}{319.301056pt}}
\pgflineto{\pgfpoint{101.691170pt}{319.301056pt}}
\pgfusepath{stroke}
\pgfpathmoveto{\pgfpoint{101.736000pt}{325.477905pt}}
\pgflineto{\pgfpoint{101.691109pt}{325.477905pt}}
\pgfusepath{stroke}
\pgfpathmoveto{\pgfpoint{101.736076pt}{331.654724pt}}
\pgflineto{\pgfpoint{101.691139pt}{331.654724pt}}
\pgfusepath{stroke}
\pgfpathmoveto{\pgfpoint{101.736076pt}{337.831604pt}}
\pgflineto{\pgfpoint{101.691193pt}{337.831604pt}}
\pgfusepath{stroke}
\pgfpathmoveto{\pgfpoint{101.744980pt}{344.008423pt}}
\pgflineto{\pgfpoint{101.700142pt}{344.008423pt}}
\pgfusepath{stroke}
\pgfpathmoveto{\pgfpoint{101.744980pt}{350.185242pt}}
\pgflineto{\pgfpoint{101.700089pt}{350.185242pt}}
\pgfusepath{stroke}
\pgfpathmoveto{\pgfpoint{101.745041pt}{356.362122pt}}
\pgflineto{\pgfpoint{101.700203pt}{356.362122pt}}
\pgfusepath{stroke}
\pgfpathmoveto{\pgfpoint{101.740532pt}{362.538940pt}}
\pgflineto{\pgfpoint{101.695618pt}{362.538940pt}}
\pgfusepath{stroke}
\pgfpathmoveto{\pgfpoint{101.745079pt}{368.715820pt}}
\pgflineto{\pgfpoint{101.700218pt}{368.715820pt}}
\pgfusepath{stroke}
\pgfpathmoveto{\pgfpoint{101.745026pt}{374.892639pt}}
\pgflineto{\pgfpoint{101.700218pt}{374.892639pt}}
\pgfusepath{stroke}
\pgfpathmoveto{\pgfpoint{101.749512pt}{381.069458pt}}
\pgflineto{\pgfpoint{101.704659pt}{381.069458pt}}
\pgfusepath{stroke}
\pgfpathmoveto{\pgfpoint{101.749573pt}{387.246338pt}}
\pgflineto{\pgfpoint{101.704712pt}{387.246338pt}}
\pgfusepath{stroke}
\pgfpathmoveto{\pgfpoint{101.749512pt}{393.423157pt}}
\pgflineto{\pgfpoint{101.704712pt}{393.423157pt}}
\pgfusepath{stroke}
\pgfpathmoveto{\pgfpoint{101.754066pt}{399.600037pt}}
\pgflineto{\pgfpoint{101.709259pt}{399.600037pt}}
\pgfusepath{stroke}
\pgfpathmoveto{\pgfpoint{101.664001pt}{84.581039pt}}
\pgflineto{\pgfpoint{101.664001pt}{78.404205pt}}
\pgfusepath{stroke}
\pgfpathmoveto{\pgfpoint{101.664001pt}{59.873672pt}}
\pgflineto{\pgfpoint{101.664001pt}{53.696838pt}}
\pgfusepath{stroke}
\pgfpathmoveto{\pgfpoint{101.664001pt}{66.050522pt}}
\pgflineto{\pgfpoint{101.664001pt}{59.873672pt}}
\pgfusepath{stroke}
\pgfpathmoveto{\pgfpoint{101.664001pt}{72.227356pt}}
\pgflineto{\pgfpoint{101.664001pt}{66.050522pt}}
\pgfusepath{stroke}
\pgfpathmoveto{\pgfpoint{101.664001pt}{78.404205pt}}
\pgflineto{\pgfpoint{101.664001pt}{72.227356pt}}
\pgfusepath{stroke}
\pgfpathmoveto{\pgfpoint{101.664001pt}{90.757896pt}}
\pgflineto{\pgfpoint{101.664001pt}{84.581039pt}}
\pgfusepath{stroke}
\pgfpathmoveto{\pgfpoint{101.664001pt}{96.934731pt}}
\pgflineto{\pgfpoint{101.664001pt}{90.757896pt}}
\pgfusepath{stroke}
\pgfpathmoveto{\pgfpoint{101.664001pt}{103.111580pt}}
\pgflineto{\pgfpoint{101.664001pt}{96.934731pt}}
\pgfusepath{stroke}
\pgfpathmoveto{\pgfpoint{101.664001pt}{109.288422pt}}
\pgflineto{\pgfpoint{101.664001pt}{103.111580pt}}
\pgfusepath{stroke}
\pgfpathmoveto{\pgfpoint{101.664001pt}{115.465263pt}}
\pgflineto{\pgfpoint{101.664001pt}{109.288422pt}}
\pgfusepath{stroke}
\pgfpathmoveto{\pgfpoint{101.664001pt}{121.642097pt}}
\pgflineto{\pgfpoint{101.664001pt}{115.465263pt}}
\pgfusepath{stroke}
\pgfpathmoveto{\pgfpoint{101.664001pt}{127.818947pt}}
\pgflineto{\pgfpoint{101.664001pt}{121.642097pt}}
\pgfusepath{stroke}
\pgfpathmoveto{\pgfpoint{101.664001pt}{133.995789pt}}
\pgflineto{\pgfpoint{101.664001pt}{127.818947pt}}
\pgfusepath{stroke}
\pgfpathmoveto{\pgfpoint{101.664001pt}{47.519989pt}}
\pgflineto{\pgfpoint{101.645958pt}{47.519989pt}}
\pgfusepath{stroke}
\pgfpathmoveto{\pgfpoint{101.664001pt}{53.696838pt}}
\pgflineto{\pgfpoint{101.664001pt}{47.519989pt}}
\pgfusepath{stroke}
\pgfpathmoveto{\pgfpoint{101.664001pt}{47.519989pt}}
\pgflineto{\pgfpoint{110.574127pt}{47.519989pt}}
\pgfusepath{stroke}
\pgfpathmoveto{\pgfpoint{101.664001pt}{53.696838pt}}
\pgflineto{\pgfpoint{110.574112pt}{53.696838pt}}
\pgfusepath{stroke}
\pgfpathmoveto{\pgfpoint{101.664001pt}{59.873672pt}}
\pgflineto{\pgfpoint{110.574112pt}{59.873672pt}}
\pgfusepath{stroke}
\pgfpathmoveto{\pgfpoint{110.591980pt}{66.050522pt}}
\pgflineto{\pgfpoint{101.664001pt}{66.050522pt}}
\pgfusepath{stroke}
\pgfpathmoveto{\pgfpoint{110.591980pt}{72.227356pt}}
\pgflineto{\pgfpoint{101.664001pt}{72.227356pt}}
\pgfusepath{stroke}
\pgfpathmoveto{\pgfpoint{110.591980pt}{78.404205pt}}
\pgflineto{\pgfpoint{101.664001pt}{78.404205pt}}
\pgfusepath{stroke}
\pgfpathmoveto{\pgfpoint{110.591980pt}{84.581039pt}}
\pgflineto{\pgfpoint{101.664001pt}{84.581039pt}}
\pgfusepath{stroke}
\pgfpathmoveto{\pgfpoint{110.591980pt}{90.757896pt}}
\pgflineto{\pgfpoint{101.664001pt}{90.757896pt}}
\pgfusepath{stroke}
\pgfpathmoveto{\pgfpoint{110.591980pt}{96.934731pt}}
\pgflineto{\pgfpoint{101.664001pt}{96.934731pt}}
\pgfusepath{stroke}
\pgfpathmoveto{\pgfpoint{110.591980pt}{103.111580pt}}
\pgflineto{\pgfpoint{101.664001pt}{103.111580pt}}
\pgfusepath{stroke}
\pgfpathmoveto{\pgfpoint{110.591980pt}{109.288422pt}}
\pgflineto{\pgfpoint{101.664001pt}{109.288422pt}}
\pgfusepath{stroke}
\pgfpathmoveto{\pgfpoint{110.591980pt}{115.465263pt}}
\pgflineto{\pgfpoint{101.664001pt}{115.465263pt}}
\pgfusepath{stroke}
\pgfpathmoveto{\pgfpoint{110.591980pt}{121.642097pt}}
\pgflineto{\pgfpoint{101.664001pt}{121.642097pt}}
\pgfusepath{stroke}
\pgfpathmoveto{\pgfpoint{110.591980pt}{127.818947pt}}
\pgflineto{\pgfpoint{101.664001pt}{127.818947pt}}
\pgfusepath{stroke}
\pgfpathmoveto{\pgfpoint{110.591980pt}{133.995789pt}}
\pgflineto{\pgfpoint{101.664001pt}{133.995789pt}}
\pgfusepath{stroke}
\pgfpathmoveto{\pgfpoint{110.591980pt}{140.172638pt}}
\pgflineto{\pgfpoint{101.682014pt}{140.172638pt}}
\pgfusepath{stroke}
\pgfpathmoveto{\pgfpoint{110.591980pt}{146.349472pt}}
\pgflineto{\pgfpoint{101.682014pt}{146.349472pt}}
\pgfusepath{stroke}
\pgfpathmoveto{\pgfpoint{110.591980pt}{152.526306pt}}
\pgflineto{\pgfpoint{101.682037pt}{152.526306pt}}
\pgfusepath{stroke}
\pgfpathmoveto{\pgfpoint{110.591980pt}{158.703156pt}}
\pgflineto{\pgfpoint{101.682037pt}{158.703156pt}}
\pgfusepath{stroke}
\pgfpathmoveto{\pgfpoint{110.591980pt}{164.880005pt}}
\pgflineto{\pgfpoint{101.682037pt}{164.880005pt}}
\pgfusepath{stroke}
\pgfpathmoveto{\pgfpoint{110.591980pt}{171.056854pt}}
\pgflineto{\pgfpoint{101.690994pt}{171.056854pt}}
\pgfusepath{stroke}
\pgfpathmoveto{\pgfpoint{110.591980pt}{177.233673pt}}
\pgflineto{\pgfpoint{101.690994pt}{177.233673pt}}
\pgfusepath{stroke}
\pgfpathmoveto{\pgfpoint{110.591980pt}{183.410522pt}}
\pgflineto{\pgfpoint{101.691025pt}{183.410522pt}}
\pgfusepath{stroke}
\pgfpathmoveto{\pgfpoint{110.591980pt}{189.587372pt}}
\pgflineto{\pgfpoint{101.691025pt}{189.587372pt}}
\pgfusepath{stroke}
\pgfpathmoveto{\pgfpoint{101.691025pt}{195.764206pt}}
\pgflineto{\pgfpoint{110.573891pt}{195.764206pt}}
\pgfusepath{stroke}
\pgfpathmoveto{\pgfpoint{101.700027pt}{201.941055pt}}
\pgflineto{\pgfpoint{110.564903pt}{201.941055pt}}
\pgfusepath{stroke}
\pgfpathmoveto{\pgfpoint{101.700027pt}{208.117905pt}}
\pgflineto{\pgfpoint{110.555870pt}{208.117905pt}}
\pgfusepath{stroke}
\pgfpathmoveto{\pgfpoint{101.700066pt}{214.294739pt}}
\pgflineto{\pgfpoint{110.555817pt}{214.294739pt}}
\pgfusepath{stroke}
\pgfpathmoveto{\pgfpoint{101.700066pt}{220.471588pt}}
\pgflineto{\pgfpoint{110.546799pt}{220.471588pt}}
\pgfusepath{stroke}
\pgfpathmoveto{\pgfpoint{101.709068pt}{226.648422pt}}
\pgflineto{\pgfpoint{110.537750pt}{226.648422pt}}
\pgfusepath{stroke}
\pgfpathmoveto{\pgfpoint{101.709007pt}{232.825272pt}}
\pgflineto{\pgfpoint{110.528732pt}{232.825272pt}}
\pgfusepath{stroke}
\pgfpathmoveto{\pgfpoint{101.709114pt}{239.002106pt}}
\pgflineto{\pgfpoint{110.528641pt}{239.002106pt}}
\pgfusepath{stroke}
\pgfpathmoveto{\pgfpoint{101.709053pt}{245.178955pt}}
\pgflineto{\pgfpoint{110.519630pt}{245.178955pt}}
\pgfusepath{stroke}
\pgfpathmoveto{\pgfpoint{101.709053pt}{251.355804pt}}
\pgflineto{\pgfpoint{110.510620pt}{251.355804pt}}
\pgfusepath{stroke}
\pgfpathmoveto{\pgfpoint{101.718048pt}{257.532623pt}}
\pgflineto{\pgfpoint{110.501617pt}{257.532623pt}}
\pgfusepath{stroke}
\pgfpathmoveto{\pgfpoint{101.718048pt}{263.709473pt}}
\pgflineto{\pgfpoint{110.492607pt}{263.709473pt}}
\pgfusepath{stroke}
\pgfpathmoveto{\pgfpoint{101.718102pt}{269.886322pt}}
\pgflineto{\pgfpoint{110.492508pt}{269.886322pt}}
\pgfusepath{stroke}
\pgfpathmoveto{\pgfpoint{101.718040pt}{276.063141pt}}
\pgflineto{\pgfpoint{110.483437pt}{276.063141pt}}
\pgfusepath{stroke}
\pgfpathmoveto{\pgfpoint{101.718040pt}{282.239990pt}}
\pgflineto{\pgfpoint{110.474358pt}{282.239990pt}}
\pgfusepath{stroke}
\pgfpathmoveto{\pgfpoint{101.727028pt}{288.416840pt}}
\pgflineto{\pgfpoint{110.465500pt}{288.416840pt}}
\pgfusepath{stroke}
\pgfpathmoveto{\pgfpoint{101.727028pt}{294.593689pt}}
\pgflineto{\pgfpoint{110.456413pt}{294.593689pt}}
\pgfusepath{stroke}
\pgfpathmoveto{\pgfpoint{101.727089pt}{300.770538pt}}
\pgflineto{\pgfpoint{110.456314pt}{300.770538pt}}
\pgfusepath{stroke}
\pgfpathmoveto{\pgfpoint{101.727089pt}{306.947388pt}}
\pgflineto{\pgfpoint{110.447296pt}{306.947388pt}}
\pgfusepath{stroke}
\pgfpathmoveto{\pgfpoint{101.727089pt}{313.124207pt}}
\pgflineto{\pgfpoint{110.438232pt}{313.124207pt}}
\pgfusepath{stroke}
\pgfpathmoveto{\pgfpoint{101.736061pt}{319.301056pt}}
\pgflineto{\pgfpoint{110.429314pt}{319.301056pt}}
\pgfusepath{stroke}
\pgfpathmoveto{\pgfpoint{101.736000pt}{325.477905pt}}
\pgflineto{\pgfpoint{110.420288pt}{325.477905pt}}
\pgfusepath{stroke}
\pgfpathmoveto{\pgfpoint{101.736076pt}{331.654724pt}}
\pgflineto{\pgfpoint{110.420105pt}{331.654724pt}}
\pgfusepath{stroke}
\pgfpathmoveto{\pgfpoint{101.736076pt}{337.831604pt}}
\pgflineto{\pgfpoint{110.411064pt}{337.831604pt}}
\pgfusepath{stroke}
\pgfpathmoveto{\pgfpoint{101.744980pt}{344.008423pt}}
\pgflineto{\pgfpoint{110.402229pt}{344.008423pt}}
\pgfusepath{stroke}
\pgfpathmoveto{\pgfpoint{101.744980pt}{350.185242pt}}
\pgflineto{\pgfpoint{110.393143pt}{350.185242pt}}
\pgfusepath{stroke}
\pgfpathmoveto{\pgfpoint{101.745041pt}{356.362122pt}}
\pgflineto{\pgfpoint{110.384163pt}{356.362122pt}}
\pgfusepath{stroke}
\pgfpathmoveto{\pgfpoint{101.740532pt}{362.538940pt}}
\pgflineto{\pgfpoint{110.374992pt}{362.538940pt}}
\pgfusepath{stroke}
\pgfpathmoveto{\pgfpoint{101.745079pt}{368.715820pt}}
\pgflineto{\pgfpoint{110.370506pt}{368.715820pt}}
\pgfusepath{stroke}
\pgfpathmoveto{\pgfpoint{101.745026pt}{374.892639pt}}
\pgflineto{\pgfpoint{110.361473pt}{374.892639pt}}
\pgfusepath{stroke}
\pgfpathmoveto{\pgfpoint{101.749512pt}{381.069458pt}}
\pgflineto{\pgfpoint{110.356934pt}{381.069458pt}}
\pgfusepath{stroke}
\pgfpathmoveto{\pgfpoint{101.749573pt}{387.246338pt}}
\pgflineto{\pgfpoint{110.347893pt}{387.246338pt}}
\pgfusepath{stroke}
\pgfpathmoveto{\pgfpoint{101.749512pt}{393.423157pt}}
\pgflineto{\pgfpoint{110.338844pt}{393.423157pt}}
\pgfusepath{stroke}
\pgfpathmoveto{\pgfpoint{101.754066pt}{399.600037pt}}
\pgflineto{\pgfpoint{110.334305pt}{399.600037pt}}
\pgfusepath{stroke}
\pgfpathmoveto{\pgfpoint{110.591980pt}{158.703156pt}}
\pgflineto{\pgfpoint{110.591980pt}{152.526306pt}}
\pgfusepath{stroke}
\pgfpathmoveto{\pgfpoint{110.591980pt}{152.526306pt}}
\pgflineto{\pgfpoint{110.591980pt}{146.349472pt}}
\pgfusepath{stroke}
\pgfpathmoveto{\pgfpoint{110.591980pt}{164.880005pt}}
\pgflineto{\pgfpoint{110.591980pt}{158.703156pt}}
\pgfusepath{stroke}
\pgfpathmoveto{\pgfpoint{110.591980pt}{171.056854pt}}
\pgflineto{\pgfpoint{110.591980pt}{164.880005pt}}
\pgfusepath{stroke}
\pgfpathmoveto{\pgfpoint{110.591980pt}{152.526306pt}}
\pgflineto{\pgfpoint{110.609825pt}{152.526306pt}}
\pgfusepath{stroke}
\pgfpathmoveto{\pgfpoint{110.591980pt}{158.703156pt}}
\pgflineto{\pgfpoint{110.609825pt}{158.703156pt}}
\pgfusepath{stroke}
\pgfpathmoveto{\pgfpoint{110.591980pt}{164.880005pt}}
\pgflineto{\pgfpoint{110.609840pt}{164.880005pt}}
\pgfusepath{stroke}
\pgfpathmoveto{\pgfpoint{110.591980pt}{183.410522pt}}
\pgflineto{\pgfpoint{110.591980pt}{177.233673pt}}
\pgfusepath{stroke}
\pgfpathmoveto{\pgfpoint{110.591980pt}{177.233673pt}}
\pgflineto{\pgfpoint{110.591980pt}{171.056854pt}}
\pgfusepath{stroke}
\pgfpathmoveto{\pgfpoint{110.591980pt}{189.587372pt}}
\pgflineto{\pgfpoint{110.591980pt}{183.410522pt}}
\pgfusepath{stroke}
\pgfpathmoveto{\pgfpoint{110.591980pt}{171.056854pt}}
\pgflineto{\pgfpoint{110.609840pt}{171.056854pt}}
\pgfusepath{stroke}
\pgfpathmoveto{\pgfpoint{110.555870pt}{208.117905pt}}
\pgflineto{\pgfpoint{110.574051pt}{208.117905pt}}
\pgfusepath{stroke}
\pgfpathmoveto{\pgfpoint{110.591980pt}{214.294739pt}}
\pgflineto{\pgfpoint{110.555817pt}{214.294739pt}}
\pgfusepath{stroke}
\pgfpathmoveto{\pgfpoint{110.591980pt}{220.471588pt}}
\pgflineto{\pgfpoint{110.546799pt}{220.471588pt}}
\pgfusepath{stroke}
\pgfpathmoveto{\pgfpoint{110.591980pt}{226.648422pt}}
\pgflineto{\pgfpoint{110.537750pt}{226.648422pt}}
\pgfusepath{stroke}
\pgfpathmoveto{\pgfpoint{110.591980pt}{232.825272pt}}
\pgflineto{\pgfpoint{110.528732pt}{232.825272pt}}
\pgfusepath{stroke}
\pgfpathmoveto{\pgfpoint{110.591980pt}{239.002106pt}}
\pgflineto{\pgfpoint{110.528641pt}{239.002106pt}}
\pgfusepath{stroke}
\pgfpathmoveto{\pgfpoint{110.591980pt}{245.178955pt}}
\pgflineto{\pgfpoint{110.519630pt}{245.178955pt}}
\pgfusepath{stroke}
\pgfpathmoveto{\pgfpoint{110.591980pt}{251.355804pt}}
\pgflineto{\pgfpoint{110.510620pt}{251.355804pt}}
\pgfusepath{stroke}
\pgfpathmoveto{\pgfpoint{110.591980pt}{257.532623pt}}
\pgflineto{\pgfpoint{110.501617pt}{257.532623pt}}
\pgfusepath{stroke}
\pgfpathmoveto{\pgfpoint{110.591980pt}{263.709473pt}}
\pgflineto{\pgfpoint{110.492607pt}{263.709473pt}}
\pgfusepath{stroke}
\pgfpathmoveto{\pgfpoint{110.591980pt}{269.886322pt}}
\pgflineto{\pgfpoint{110.492508pt}{269.886322pt}}
\pgfusepath{stroke}
\pgfpathmoveto{\pgfpoint{110.591980pt}{276.063141pt}}
\pgflineto{\pgfpoint{110.483437pt}{276.063141pt}}
\pgfusepath{stroke}
\pgfpathmoveto{\pgfpoint{110.591980pt}{282.239990pt}}
\pgflineto{\pgfpoint{110.474358pt}{282.239990pt}}
\pgfusepath{stroke}
\pgfpathmoveto{\pgfpoint{110.591980pt}{288.416840pt}}
\pgflineto{\pgfpoint{110.465500pt}{288.416840pt}}
\pgfusepath{stroke}
\pgfpathmoveto{\pgfpoint{110.591980pt}{294.593689pt}}
\pgflineto{\pgfpoint{110.456413pt}{294.593689pt}}
\pgfusepath{stroke}
\pgfpathmoveto{\pgfpoint{110.591980pt}{300.770538pt}}
\pgflineto{\pgfpoint{110.456314pt}{300.770538pt}}
\pgfusepath{stroke}
\pgfpathmoveto{\pgfpoint{110.591980pt}{306.947388pt}}
\pgflineto{\pgfpoint{110.447296pt}{306.947388pt}}
\pgfusepath{stroke}
\pgfpathmoveto{\pgfpoint{110.591980pt}{313.124207pt}}
\pgflineto{\pgfpoint{110.438232pt}{313.124207pt}}
\pgfusepath{stroke}
\pgfpathmoveto{\pgfpoint{110.591980pt}{319.301056pt}}
\pgflineto{\pgfpoint{110.429314pt}{319.301056pt}}
\pgfusepath{stroke}
\pgfpathmoveto{\pgfpoint{110.591980pt}{325.477905pt}}
\pgflineto{\pgfpoint{110.420288pt}{325.477905pt}}
\pgfusepath{stroke}
\pgfpathmoveto{\pgfpoint{110.591980pt}{331.654724pt}}
\pgflineto{\pgfpoint{110.420105pt}{331.654724pt}}
\pgfusepath{stroke}
\pgfpathmoveto{\pgfpoint{110.591980pt}{337.831604pt}}
\pgflineto{\pgfpoint{110.411064pt}{337.831604pt}}
\pgfusepath{stroke}
\pgfpathmoveto{\pgfpoint{110.402229pt}{344.008423pt}}
\pgflineto{\pgfpoint{110.574005pt}{344.008423pt}}
\pgfusepath{stroke}
\pgfpathmoveto{\pgfpoint{110.393143pt}{350.185242pt}}
\pgflineto{\pgfpoint{110.564980pt}{350.185242pt}}
\pgfusepath{stroke}
\pgfpathmoveto{\pgfpoint{110.384163pt}{356.362122pt}}
\pgflineto{\pgfpoint{110.556000pt}{356.362122pt}}
\pgfusepath{stroke}
\pgfpathmoveto{\pgfpoint{110.374992pt}{362.538940pt}}
\pgflineto{\pgfpoint{110.546967pt}{362.538940pt}}
\pgfusepath{stroke}
\pgfpathmoveto{\pgfpoint{110.370506pt}{368.715820pt}}
\pgflineto{\pgfpoint{110.542549pt}{368.715820pt}}
\pgfusepath{stroke}
\pgfpathmoveto{\pgfpoint{110.361473pt}{374.892639pt}}
\pgflineto{\pgfpoint{110.533508pt}{374.892639pt}}
\pgfusepath{stroke}
\pgfpathmoveto{\pgfpoint{110.356934pt}{381.069458pt}}
\pgflineto{\pgfpoint{110.528969pt}{381.069458pt}}
\pgfusepath{stroke}
\pgfpathmoveto{\pgfpoint{110.347893pt}{387.246338pt}}
\pgflineto{\pgfpoint{110.520050pt}{387.246338pt}}
\pgfusepath{stroke}
\pgfpathmoveto{\pgfpoint{110.338844pt}{393.423157pt}}
\pgflineto{\pgfpoint{110.511002pt}{393.423157pt}}
\pgfusepath{stroke}
\pgfpathmoveto{\pgfpoint{110.334305pt}{399.600037pt}}
\pgflineto{\pgfpoint{110.506523pt}{399.600037pt}}
\pgfusepath{stroke}
\pgfpathmoveto{\pgfpoint{110.591980pt}{306.947388pt}}
\pgflineto{\pgfpoint{110.591980pt}{300.770538pt}}
\pgfusepath{stroke}
\pgfpathmoveto{\pgfpoint{110.591980pt}{300.770538pt}}
\pgflineto{\pgfpoint{110.591980pt}{294.593689pt}}
\pgfusepath{stroke}
\pgfpathmoveto{\pgfpoint{110.591980pt}{313.124207pt}}
\pgflineto{\pgfpoint{110.591980pt}{306.947388pt}}
\pgfusepath{stroke}
\pgfpathmoveto{\pgfpoint{110.591980pt}{319.301056pt}}
\pgflineto{\pgfpoint{110.591980pt}{313.124207pt}}
\pgfusepath{stroke}
\pgfpathmoveto{\pgfpoint{110.591980pt}{300.770538pt}}
\pgflineto{\pgfpoint{110.609825pt}{300.770538pt}}
\pgfusepath{stroke}
\pgfpathmoveto{\pgfpoint{110.591980pt}{306.947388pt}}
\pgflineto{\pgfpoint{110.609825pt}{306.947388pt}}
\pgfusepath{stroke}
\pgfpathmoveto{\pgfpoint{110.591980pt}{313.124207pt}}
\pgflineto{\pgfpoint{110.609779pt}{313.124207pt}}
\pgfusepath{stroke}
\pgfpathmoveto{\pgfpoint{110.591980pt}{331.654724pt}}
\pgflineto{\pgfpoint{110.591980pt}{325.477905pt}}
\pgfusepath{stroke}
\pgfpathmoveto{\pgfpoint{110.591980pt}{325.477905pt}}
\pgflineto{\pgfpoint{110.591980pt}{319.301056pt}}
\pgfusepath{stroke}
\pgfpathmoveto{\pgfpoint{110.591980pt}{337.831604pt}}
\pgflineto{\pgfpoint{110.591980pt}{331.654724pt}}
\pgfusepath{stroke}
\pgfpathmoveto{\pgfpoint{110.591980pt}{319.301056pt}}
\pgflineto{\pgfpoint{110.609779pt}{319.301056pt}}
\pgfusepath{stroke}
\pgfpathmoveto{\pgfpoint{110.591980pt}{344.008423pt}}
\pgflineto{\pgfpoint{110.574005pt}{344.008423pt}}
\pgfusepath{stroke}
\pgfpathmoveto{\pgfpoint{110.591980pt}{350.185242pt}}
\pgflineto{\pgfpoint{110.564980pt}{350.185242pt}}
\pgfusepath{stroke}
\pgfpathmoveto{\pgfpoint{110.591980pt}{356.362122pt}}
\pgflineto{\pgfpoint{110.556000pt}{356.362122pt}}
\pgfusepath{stroke}
\pgfpathmoveto{\pgfpoint{110.591980pt}{362.538940pt}}
\pgflineto{\pgfpoint{110.546967pt}{362.538940pt}}
\pgfusepath{stroke}
\pgfpathmoveto{\pgfpoint{110.591980pt}{368.715820pt}}
\pgflineto{\pgfpoint{110.542549pt}{368.715820pt}}
\pgfusepath{stroke}
\pgfpathmoveto{\pgfpoint{110.591980pt}{374.892639pt}}
\pgflineto{\pgfpoint{110.533508pt}{374.892639pt}}
\pgfusepath{stroke}
\pgfpathmoveto{\pgfpoint{110.591980pt}{381.069458pt}}
\pgflineto{\pgfpoint{110.528969pt}{381.069458pt}}
\pgfusepath{stroke}
\pgfpathmoveto{\pgfpoint{110.591980pt}{387.246338pt}}
\pgflineto{\pgfpoint{110.520050pt}{387.246338pt}}
\pgfusepath{stroke}
\pgfpathmoveto{\pgfpoint{110.591980pt}{393.423157pt}}
\pgflineto{\pgfpoint{110.511002pt}{393.423157pt}}
\pgfusepath{stroke}
\pgfpathmoveto{\pgfpoint{110.591980pt}{399.600037pt}}
\pgflineto{\pgfpoint{110.506523pt}{399.600037pt}}
\pgfusepath{stroke}
\pgfpathmoveto{\pgfpoint{110.591980pt}{374.892639pt}}
\pgflineto{\pgfpoint{110.591980pt}{368.715820pt}}
\pgfusepath{stroke}
\pgfpathmoveto{\pgfpoint{110.591980pt}{344.008423pt}}
\pgflineto{\pgfpoint{110.591980pt}{337.831604pt}}
\pgfusepath{stroke}
\pgfpathmoveto{\pgfpoint{110.591980pt}{350.185242pt}}
\pgflineto{\pgfpoint{110.591980pt}{344.008423pt}}
\pgfusepath{stroke}
\pgfpathmoveto{\pgfpoint{110.591980pt}{356.362122pt}}
\pgflineto{\pgfpoint{110.591980pt}{350.185242pt}}
\pgfusepath{stroke}
\pgfpathmoveto{\pgfpoint{110.591980pt}{362.538940pt}}
\pgflineto{\pgfpoint{110.591980pt}{356.362122pt}}
\pgfusepath{stroke}
\pgfpathmoveto{\pgfpoint{110.591980pt}{368.715820pt}}
\pgflineto{\pgfpoint{110.591980pt}{362.538940pt}}
\pgfusepath{stroke}
\pgfpathmoveto{\pgfpoint{110.591980pt}{381.069458pt}}
\pgflineto{\pgfpoint{110.591980pt}{374.892639pt}}
\pgfusepath{stroke}
\pgfpathmoveto{\pgfpoint{110.591980pt}{387.246338pt}}
\pgflineto{\pgfpoint{110.591980pt}{381.069458pt}}
\pgfusepath{stroke}
\pgfpathmoveto{\pgfpoint{110.591980pt}{393.423157pt}}
\pgflineto{\pgfpoint{110.591980pt}{387.246338pt}}
\pgfusepath{stroke}
\pgfpathmoveto{\pgfpoint{110.591980pt}{399.600037pt}}
\pgflineto{\pgfpoint{110.591980pt}{393.423157pt}}
\pgfusepath{stroke}
\pgfpathmoveto{\pgfpoint{110.591980pt}{325.477905pt}}
\pgflineto{\pgfpoint{110.609718pt}{325.477905pt}}
\pgfusepath{stroke}
\pgfpathmoveto{\pgfpoint{110.591980pt}{331.654724pt}}
\pgflineto{\pgfpoint{110.618683pt}{331.654724pt}}
\pgfusepath{stroke}
\pgfpathmoveto{\pgfpoint{110.591980pt}{337.831604pt}}
\pgflineto{\pgfpoint{110.618744pt}{337.831604pt}}
\pgfusepath{stroke}
\pgfpathmoveto{\pgfpoint{110.591980pt}{344.008423pt}}
\pgflineto{\pgfpoint{110.618683pt}{344.008423pt}}
\pgfusepath{stroke}
\pgfpathmoveto{\pgfpoint{110.591980pt}{350.185242pt}}
\pgflineto{\pgfpoint{110.618637pt}{350.185242pt}}
\pgfusepath{stroke}
\pgfpathmoveto{\pgfpoint{110.591980pt}{356.362122pt}}
\pgflineto{\pgfpoint{110.618698pt}{356.362122pt}}
\pgfusepath{stroke}
\pgfpathmoveto{\pgfpoint{110.591980pt}{362.538940pt}}
\pgflineto{\pgfpoint{110.618637pt}{362.538940pt}}
\pgfusepath{stroke}
\pgfpathmoveto{\pgfpoint{110.591980pt}{368.715820pt}}
\pgflineto{\pgfpoint{110.623131pt}{368.715820pt}}
\pgfusepath{stroke}
\pgfpathmoveto{\pgfpoint{110.591980pt}{374.892639pt}}
\pgflineto{\pgfpoint{110.623131pt}{374.892639pt}}
\pgfusepath{stroke}
\pgfpathmoveto{\pgfpoint{110.591980pt}{381.069458pt}}
\pgflineto{\pgfpoint{110.627571pt}{381.069458pt}}
\pgfusepath{stroke}
\pgfpathmoveto{\pgfpoint{110.591980pt}{387.246338pt}}
\pgflineto{\pgfpoint{110.627571pt}{387.246338pt}}
\pgfusepath{stroke}
\pgfpathmoveto{\pgfpoint{110.591980pt}{393.423157pt}}
\pgflineto{\pgfpoint{110.627533pt}{393.423157pt}}
\pgfusepath{stroke}
\pgfpathmoveto{\pgfpoint{110.591980pt}{399.600037pt}}
\pgflineto{\pgfpoint{110.632057pt}{399.600037pt}}
\pgfusepath{stroke}
\pgfpathmoveto{\pgfpoint{110.591980pt}{245.178955pt}}
\pgflineto{\pgfpoint{110.591980pt}{239.002106pt}}
\pgfusepath{stroke}
\pgfpathmoveto{\pgfpoint{110.591980pt}{220.471588pt}}
\pgflineto{\pgfpoint{110.591980pt}{214.294739pt}}
\pgfusepath{stroke}
\pgfpathmoveto{\pgfpoint{110.591980pt}{226.648422pt}}
\pgflineto{\pgfpoint{110.591980pt}{220.471588pt}}
\pgfusepath{stroke}
\pgfpathmoveto{\pgfpoint{110.591980pt}{232.825272pt}}
\pgflineto{\pgfpoint{110.591980pt}{226.648422pt}}
\pgfusepath{stroke}
\pgfpathmoveto{\pgfpoint{110.591980pt}{239.002106pt}}
\pgflineto{\pgfpoint{110.591980pt}{232.825272pt}}
\pgfusepath{stroke}
\pgfpathmoveto{\pgfpoint{110.591980pt}{251.355804pt}}
\pgflineto{\pgfpoint{110.591980pt}{245.178955pt}}
\pgfusepath{stroke}
\pgfpathmoveto{\pgfpoint{110.591980pt}{257.532623pt}}
\pgflineto{\pgfpoint{110.591980pt}{251.355804pt}}
\pgfusepath{stroke}
\pgfpathmoveto{\pgfpoint{110.591980pt}{263.709473pt}}
\pgflineto{\pgfpoint{110.591980pt}{257.532623pt}}
\pgfusepath{stroke}
\pgfpathmoveto{\pgfpoint{110.591980pt}{269.886322pt}}
\pgflineto{\pgfpoint{110.591980pt}{263.709473pt}}
\pgfusepath{stroke}
\pgfpathmoveto{\pgfpoint{110.591980pt}{276.063141pt}}
\pgflineto{\pgfpoint{110.591980pt}{269.886322pt}}
\pgfusepath{stroke}
\pgfpathmoveto{\pgfpoint{110.591980pt}{282.239990pt}}
\pgflineto{\pgfpoint{110.591980pt}{276.063141pt}}
\pgfusepath{stroke}
\pgfpathmoveto{\pgfpoint{110.591980pt}{288.416840pt}}
\pgflineto{\pgfpoint{110.591980pt}{282.239990pt}}
\pgfusepath{stroke}
\pgfpathmoveto{\pgfpoint{110.591980pt}{294.593689pt}}
\pgflineto{\pgfpoint{110.591980pt}{288.416840pt}}
\pgfusepath{stroke}
\pgfpathmoveto{\pgfpoint{110.591980pt}{195.764206pt}}
\pgflineto{\pgfpoint{110.573891pt}{195.764206pt}}
\pgfusepath{stroke}
\pgfpathmoveto{\pgfpoint{110.591980pt}{201.941055pt}}
\pgflineto{\pgfpoint{110.564903pt}{201.941055pt}}
\pgfusepath{stroke}
\pgfpathmoveto{\pgfpoint{110.591980pt}{208.117905pt}}
\pgflineto{\pgfpoint{110.574051pt}{208.117905pt}}
\pgfusepath{stroke}
\pgfpathmoveto{\pgfpoint{110.591980pt}{195.764206pt}}
\pgflineto{\pgfpoint{110.591980pt}{189.587372pt}}
\pgfusepath{stroke}
\pgfpathmoveto{\pgfpoint{110.591980pt}{201.941055pt}}
\pgflineto{\pgfpoint{110.591980pt}{195.764206pt}}
\pgfusepath{stroke}
\pgfpathmoveto{\pgfpoint{110.591980pt}{177.233673pt}}
\pgflineto{\pgfpoint{110.609779pt}{177.233673pt}}
\pgfusepath{stroke}
\pgfpathmoveto{\pgfpoint{110.591980pt}{183.410522pt}}
\pgflineto{\pgfpoint{110.618797pt}{183.410522pt}}
\pgfusepath{stroke}
\pgfpathmoveto{\pgfpoint{110.591980pt}{189.587372pt}}
\pgflineto{\pgfpoint{110.618797pt}{189.587372pt}}
\pgfusepath{stroke}
\pgfpathmoveto{\pgfpoint{110.591980pt}{195.764206pt}}
\pgflineto{\pgfpoint{110.618713pt}{195.764206pt}}
\pgfusepath{stroke}
\pgfpathmoveto{\pgfpoint{110.591980pt}{214.294739pt}}
\pgflineto{\pgfpoint{110.591980pt}{208.117905pt}}
\pgfusepath{stroke}
\pgfpathmoveto{\pgfpoint{110.591980pt}{208.117905pt}}
\pgflineto{\pgfpoint{110.591980pt}{201.941055pt}}
\pgfusepath{stroke}
\pgfpathmoveto{\pgfpoint{110.591980pt}{201.941055pt}}
\pgflineto{\pgfpoint{110.618713pt}{201.941055pt}}
\pgfusepath{stroke}
\pgfpathmoveto{\pgfpoint{110.591980pt}{208.117905pt}}
\pgflineto{\pgfpoint{110.618713pt}{208.117905pt}}
\pgfusepath{stroke}
\pgfpathmoveto{\pgfpoint{110.591980pt}{214.294739pt}}
\pgflineto{\pgfpoint{110.627724pt}{214.294739pt}}
\pgfusepath{stroke}
\pgfpathmoveto{\pgfpoint{110.591980pt}{220.471588pt}}
\pgflineto{\pgfpoint{110.627663pt}{220.471588pt}}
\pgfusepath{stroke}
\pgfpathmoveto{\pgfpoint{110.591980pt}{226.648422pt}}
\pgflineto{\pgfpoint{110.627701pt}{226.648422pt}}
\pgfusepath{stroke}
\pgfpathmoveto{\pgfpoint{110.591980pt}{232.825272pt}}
\pgflineto{\pgfpoint{110.627701pt}{232.825272pt}}
\pgfusepath{stroke}
\pgfpathmoveto{\pgfpoint{110.591980pt}{239.002106pt}}
\pgflineto{\pgfpoint{110.636642pt}{239.002106pt}}
\pgfusepath{stroke}
\pgfpathmoveto{\pgfpoint{110.591980pt}{245.178955pt}}
\pgflineto{\pgfpoint{110.636642pt}{245.178955pt}}
\pgfusepath{stroke}
\pgfpathmoveto{\pgfpoint{110.591980pt}{251.355804pt}}
\pgflineto{\pgfpoint{110.636589pt}{251.355804pt}}
\pgfusepath{stroke}
\pgfpathmoveto{\pgfpoint{110.591980pt}{257.532623pt}}
\pgflineto{\pgfpoint{110.636574pt}{257.532623pt}}
\pgfusepath{stroke}
\pgfpathmoveto{\pgfpoint{110.591980pt}{263.709473pt}}
\pgflineto{\pgfpoint{110.636574pt}{263.709473pt}}
\pgfusepath{stroke}
\pgfpathmoveto{\pgfpoint{110.591980pt}{269.886322pt}}
\pgflineto{\pgfpoint{110.645561pt}{269.886322pt}}
\pgfusepath{stroke}
\pgfpathmoveto{\pgfpoint{110.591980pt}{276.063141pt}}
\pgflineto{\pgfpoint{110.645508pt}{276.063141pt}}
\pgfusepath{stroke}
\pgfpathmoveto{\pgfpoint{110.591980pt}{282.239990pt}}
\pgflineto{\pgfpoint{110.645561pt}{282.239990pt}}
\pgfusepath{stroke}
\pgfpathmoveto{\pgfpoint{110.591980pt}{288.416840pt}}
\pgflineto{\pgfpoint{110.645561pt}{288.416840pt}}
\pgfusepath{stroke}
\pgfpathmoveto{\pgfpoint{110.591980pt}{294.593689pt}}
\pgflineto{\pgfpoint{110.645500pt}{294.593689pt}}
\pgfusepath{stroke}
\pgfpathmoveto{\pgfpoint{110.654488pt}{300.770538pt}}
\pgflineto{\pgfpoint{110.609825pt}{300.770538pt}}
\pgfusepath{stroke}
\pgfpathmoveto{\pgfpoint{110.654488pt}{306.947388pt}}
\pgflineto{\pgfpoint{110.609825pt}{306.947388pt}}
\pgfusepath{stroke}
\pgfpathmoveto{\pgfpoint{110.654488pt}{313.124207pt}}
\pgflineto{\pgfpoint{110.609779pt}{313.124207pt}}
\pgfusepath{stroke}
\pgfpathmoveto{\pgfpoint{110.654488pt}{319.301056pt}}
\pgflineto{\pgfpoint{110.609779pt}{319.301056pt}}
\pgfusepath{stroke}
\pgfpathmoveto{\pgfpoint{110.654434pt}{325.477905pt}}
\pgflineto{\pgfpoint{110.609718pt}{325.477905pt}}
\pgfusepath{stroke}
\pgfpathmoveto{\pgfpoint{110.663345pt}{331.654724pt}}
\pgflineto{\pgfpoint{110.618683pt}{331.654724pt}}
\pgfusepath{stroke}
\pgfpathmoveto{\pgfpoint{110.663406pt}{337.831604pt}}
\pgflineto{\pgfpoint{110.618744pt}{337.831604pt}}
\pgfusepath{stroke}
\pgfpathmoveto{\pgfpoint{110.663345pt}{344.008423pt}}
\pgflineto{\pgfpoint{110.618683pt}{344.008423pt}}
\pgfusepath{stroke}
\pgfpathmoveto{\pgfpoint{110.663330pt}{350.185242pt}}
\pgflineto{\pgfpoint{110.618637pt}{350.185242pt}}
\pgfusepath{stroke}
\pgfpathmoveto{\pgfpoint{110.663383pt}{356.362122pt}}
\pgflineto{\pgfpoint{110.618698pt}{356.362122pt}}
\pgfusepath{stroke}
\pgfpathmoveto{\pgfpoint{110.663269pt}{362.538940pt}}
\pgflineto{\pgfpoint{110.618637pt}{362.538940pt}}
\pgfusepath{stroke}
\pgfpathmoveto{\pgfpoint{110.667816pt}{368.715820pt}}
\pgflineto{\pgfpoint{110.623131pt}{368.715820pt}}
\pgfusepath{stroke}
\pgfpathmoveto{\pgfpoint{110.667816pt}{374.892639pt}}
\pgflineto{\pgfpoint{110.623131pt}{374.892639pt}}
\pgfusepath{stroke}
\pgfpathmoveto{\pgfpoint{110.672249pt}{381.069458pt}}
\pgflineto{\pgfpoint{110.627571pt}{381.069458pt}}
\pgfusepath{stroke}
\pgfpathmoveto{\pgfpoint{110.672249pt}{387.246338pt}}
\pgflineto{\pgfpoint{110.627571pt}{387.246338pt}}
\pgfusepath{stroke}
\pgfpathmoveto{\pgfpoint{110.672234pt}{393.423157pt}}
\pgflineto{\pgfpoint{110.627533pt}{393.423157pt}}
\pgfusepath{stroke}
\pgfpathmoveto{\pgfpoint{110.676743pt}{399.600037pt}}
\pgflineto{\pgfpoint{110.632057pt}{399.600037pt}}
\pgfusepath{stroke}
\pgfpathmoveto{\pgfpoint{110.591980pt}{103.111580pt}}
\pgflineto{\pgfpoint{110.591980pt}{96.934731pt}}
\pgfusepath{stroke}
\pgfpathmoveto{\pgfpoint{110.591980pt}{72.227356pt}}
\pgflineto{\pgfpoint{110.591980pt}{66.050522pt}}
\pgfusepath{stroke}
\pgfpathmoveto{\pgfpoint{110.591980pt}{78.404205pt}}
\pgflineto{\pgfpoint{110.591980pt}{72.227356pt}}
\pgfusepath{stroke}
\pgfpathmoveto{\pgfpoint{110.591980pt}{84.581039pt}}
\pgflineto{\pgfpoint{110.591980pt}{78.404205pt}}
\pgfusepath{stroke}
\pgfpathmoveto{\pgfpoint{110.591980pt}{90.757896pt}}
\pgflineto{\pgfpoint{110.591980pt}{84.581039pt}}
\pgfusepath{stroke}
\pgfpathmoveto{\pgfpoint{110.591980pt}{96.934731pt}}
\pgflineto{\pgfpoint{110.591980pt}{90.757896pt}}
\pgfusepath{stroke}
\pgfpathmoveto{\pgfpoint{110.591980pt}{109.288422pt}}
\pgflineto{\pgfpoint{110.591980pt}{103.111580pt}}
\pgfusepath{stroke}
\pgfpathmoveto{\pgfpoint{110.591980pt}{115.465263pt}}
\pgflineto{\pgfpoint{110.591980pt}{109.288422pt}}
\pgfusepath{stroke}
\pgfpathmoveto{\pgfpoint{110.591980pt}{121.642097pt}}
\pgflineto{\pgfpoint{110.591980pt}{115.465263pt}}
\pgfusepath{stroke}
\pgfpathmoveto{\pgfpoint{110.591980pt}{127.818947pt}}
\pgflineto{\pgfpoint{110.591980pt}{121.642097pt}}
\pgfusepath{stroke}
\pgfpathmoveto{\pgfpoint{110.591980pt}{133.995789pt}}
\pgflineto{\pgfpoint{110.591980pt}{127.818947pt}}
\pgfusepath{stroke}
\pgfpathmoveto{\pgfpoint{110.591980pt}{140.172638pt}}
\pgflineto{\pgfpoint{110.591980pt}{133.995789pt}}
\pgfusepath{stroke}
\pgfpathmoveto{\pgfpoint{110.591980pt}{146.349472pt}}
\pgflineto{\pgfpoint{110.591980pt}{140.172638pt}}
\pgfusepath{stroke}
\pgfpathmoveto{\pgfpoint{110.591980pt}{47.519989pt}}
\pgflineto{\pgfpoint{110.574127pt}{47.519989pt}}
\pgfusepath{stroke}
\pgfpathmoveto{\pgfpoint{110.591980pt}{53.696838pt}}
\pgflineto{\pgfpoint{110.574112pt}{53.696838pt}}
\pgfusepath{stroke}
\pgfpathmoveto{\pgfpoint{110.591980pt}{59.873672pt}}
\pgflineto{\pgfpoint{110.574112pt}{59.873672pt}}
\pgfusepath{stroke}
\pgfpathmoveto{\pgfpoint{110.591980pt}{53.696838pt}}
\pgflineto{\pgfpoint{110.591980pt}{47.519989pt}}
\pgfusepath{stroke}
\pgfpathmoveto{\pgfpoint{110.591980pt}{47.519989pt}}
\pgflineto{\pgfpoint{110.610008pt}{47.519989pt}}
\pgfusepath{stroke}
\pgfpathmoveto{\pgfpoint{110.591980pt}{66.050522pt}}
\pgflineto{\pgfpoint{110.591980pt}{59.873672pt}}
\pgfusepath{stroke}
\pgfpathmoveto{\pgfpoint{110.591980pt}{59.873672pt}}
\pgflineto{\pgfpoint{110.591980pt}{53.696838pt}}
\pgfusepath{stroke}
\pgfpathmoveto{\pgfpoint{119.519989pt}{47.519989pt}}
\pgflineto{\pgfpoint{110.610008pt}{47.519989pt}}
\pgfusepath{stroke}
\pgfpathmoveto{\pgfpoint{119.519989pt}{53.696838pt}}
\pgflineto{\pgfpoint{110.591980pt}{53.696838pt}}
\pgfusepath{stroke}
\pgfpathmoveto{\pgfpoint{119.519989pt}{59.873672pt}}
\pgflineto{\pgfpoint{110.591980pt}{59.873672pt}}
\pgfusepath{stroke}
\pgfpathmoveto{\pgfpoint{119.519989pt}{66.050522pt}}
\pgflineto{\pgfpoint{110.591980pt}{66.050522pt}}
\pgfusepath{stroke}
\pgfpathmoveto{\pgfpoint{119.519989pt}{72.227356pt}}
\pgflineto{\pgfpoint{110.591980pt}{72.227356pt}}
\pgfusepath{stroke}
\pgfpathmoveto{\pgfpoint{119.519989pt}{78.404205pt}}
\pgflineto{\pgfpoint{110.591980pt}{78.404205pt}}
\pgfusepath{stroke}
\pgfpathmoveto{\pgfpoint{119.519989pt}{84.581039pt}}
\pgflineto{\pgfpoint{110.591980pt}{84.581039pt}}
\pgfusepath{stroke}
\pgfpathmoveto{\pgfpoint{119.519989pt}{90.757896pt}}
\pgflineto{\pgfpoint{110.591980pt}{90.757896pt}}
\pgfusepath{stroke}
\pgfpathmoveto{\pgfpoint{119.519989pt}{96.934731pt}}
\pgflineto{\pgfpoint{110.591980pt}{96.934731pt}}
\pgfusepath{stroke}
\pgfpathmoveto{\pgfpoint{119.519989pt}{103.111580pt}}
\pgflineto{\pgfpoint{110.591980pt}{103.111580pt}}
\pgfusepath{stroke}
\pgfpathmoveto{\pgfpoint{119.519989pt}{109.288422pt}}
\pgflineto{\pgfpoint{110.591980pt}{109.288422pt}}
\pgfusepath{stroke}
\pgfpathmoveto{\pgfpoint{119.519989pt}{115.465263pt}}
\pgflineto{\pgfpoint{110.591980pt}{115.465263pt}}
\pgfusepath{stroke}
\pgfpathmoveto{\pgfpoint{119.519989pt}{121.642097pt}}
\pgflineto{\pgfpoint{110.591980pt}{121.642097pt}}
\pgfusepath{stroke}
\pgfpathmoveto{\pgfpoint{119.519989pt}{127.818947pt}}
\pgflineto{\pgfpoint{110.591980pt}{127.818947pt}}
\pgfusepath{stroke}
\pgfpathmoveto{\pgfpoint{119.519989pt}{133.995789pt}}
\pgflineto{\pgfpoint{110.591980pt}{133.995789pt}}
\pgfusepath{stroke}
\pgfpathmoveto{\pgfpoint{119.519989pt}{140.172638pt}}
\pgflineto{\pgfpoint{110.591980pt}{140.172638pt}}
\pgfusepath{stroke}
\pgfpathmoveto{\pgfpoint{119.519989pt}{146.349472pt}}
\pgflineto{\pgfpoint{110.591980pt}{146.349472pt}}
\pgfusepath{stroke}
\pgfpathmoveto{\pgfpoint{119.519989pt}{152.526306pt}}
\pgflineto{\pgfpoint{110.609825pt}{152.526306pt}}
\pgfusepath{stroke}
\pgfpathmoveto{\pgfpoint{119.519989pt}{158.703156pt}}
\pgflineto{\pgfpoint{110.609825pt}{158.703156pt}}
\pgfusepath{stroke}
\pgfpathmoveto{\pgfpoint{119.519989pt}{164.880005pt}}
\pgflineto{\pgfpoint{110.609840pt}{164.880005pt}}
\pgfusepath{stroke}
\pgfpathmoveto{\pgfpoint{119.519989pt}{171.056854pt}}
\pgflineto{\pgfpoint{110.609840pt}{171.056854pt}}
\pgfusepath{stroke}
\pgfpathmoveto{\pgfpoint{110.609779pt}{177.233673pt}}
\pgflineto{\pgfpoint{119.501907pt}{177.233673pt}}
\pgfusepath{stroke}
\pgfpathmoveto{\pgfpoint{110.618797pt}{183.410522pt}}
\pgflineto{\pgfpoint{119.492920pt}{183.410522pt}}
\pgfusepath{stroke}
\pgfpathmoveto{\pgfpoint{110.618797pt}{189.587372pt}}
\pgflineto{\pgfpoint{119.483887pt}{189.587372pt}}
\pgfusepath{stroke}
\pgfpathmoveto{\pgfpoint{110.618713pt}{195.764206pt}}
\pgflineto{\pgfpoint{119.483856pt}{195.764206pt}}
\pgfusepath{stroke}
\pgfpathmoveto{\pgfpoint{110.618713pt}{201.941055pt}}
\pgflineto{\pgfpoint{119.474785pt}{201.941055pt}}
\pgfusepath{stroke}
\pgfpathmoveto{\pgfpoint{110.618713pt}{208.117905pt}}
\pgflineto{\pgfpoint{119.465767pt}{208.117905pt}}
\pgfusepath{stroke}
\pgfpathmoveto{\pgfpoint{110.627724pt}{214.294739pt}}
\pgflineto{\pgfpoint{119.456734pt}{214.294739pt}}
\pgfusepath{stroke}
\pgfpathmoveto{\pgfpoint{110.627663pt}{220.471588pt}}
\pgflineto{\pgfpoint{119.447723pt}{220.471588pt}}
\pgfusepath{stroke}
\pgfpathmoveto{\pgfpoint{110.627701pt}{226.648422pt}}
\pgflineto{\pgfpoint{119.447655pt}{226.648422pt}}
\pgfusepath{stroke}
\pgfpathmoveto{\pgfpoint{110.627701pt}{232.825272pt}}
\pgflineto{\pgfpoint{119.438583pt}{232.825272pt}}
\pgfusepath{stroke}
\pgfpathmoveto{\pgfpoint{110.636642pt}{239.002106pt}}
\pgflineto{\pgfpoint{119.429642pt}{239.002106pt}}
\pgfusepath{stroke}
\pgfpathmoveto{\pgfpoint{110.636642pt}{245.178955pt}}
\pgflineto{\pgfpoint{119.420670pt}{245.178955pt}}
\pgfusepath{stroke}
\pgfpathmoveto{\pgfpoint{110.636589pt}{251.355804pt}}
\pgflineto{\pgfpoint{119.411621pt}{251.355804pt}}
\pgfusepath{stroke}
\pgfpathmoveto{\pgfpoint{110.636574pt}{257.532623pt}}
\pgflineto{\pgfpoint{119.411415pt}{257.532623pt}}
\pgfusepath{stroke}
\pgfpathmoveto{\pgfpoint{110.636574pt}{263.709473pt}}
\pgflineto{\pgfpoint{119.402405pt}{263.709473pt}}
\pgfusepath{stroke}
\pgfpathmoveto{\pgfpoint{110.645561pt}{269.886322pt}}
\pgflineto{\pgfpoint{119.393517pt}{269.886322pt}}
\pgfusepath{stroke}
\pgfpathmoveto{\pgfpoint{110.645508pt}{276.063141pt}}
\pgflineto{\pgfpoint{119.384415pt}{276.063141pt}}
\pgfusepath{stroke}
\pgfpathmoveto{\pgfpoint{110.645561pt}{282.239990pt}}
\pgflineto{\pgfpoint{119.384315pt}{282.239990pt}}
\pgfusepath{stroke}
\pgfpathmoveto{\pgfpoint{110.645561pt}{288.416840pt}}
\pgflineto{\pgfpoint{119.375221pt}{288.416840pt}}
\pgfusepath{stroke}
\pgfpathmoveto{\pgfpoint{110.645500pt}{294.593689pt}}
\pgflineto{\pgfpoint{119.366211pt}{294.593689pt}}
\pgfusepath{stroke}
\pgfpathmoveto{\pgfpoint{110.654488pt}{300.770538pt}}
\pgflineto{\pgfpoint{119.357391pt}{300.770538pt}}
\pgfusepath{stroke}
\pgfpathmoveto{\pgfpoint{110.654488pt}{306.947388pt}}
\pgflineto{\pgfpoint{119.348351pt}{306.947388pt}}
\pgfusepath{stroke}
\pgfpathmoveto{\pgfpoint{110.654488pt}{313.124207pt}}
\pgflineto{\pgfpoint{119.348114pt}{313.124207pt}}
\pgfusepath{stroke}
\pgfpathmoveto{\pgfpoint{110.654488pt}{319.301056pt}}
\pgflineto{\pgfpoint{119.339066pt}{319.301056pt}}
\pgfusepath{stroke}
\pgfpathmoveto{\pgfpoint{110.654434pt}{325.477905pt}}
\pgflineto{\pgfpoint{119.330032pt}{325.477905pt}}
\pgfusepath{stroke}
\pgfpathmoveto{\pgfpoint{110.663345pt}{331.654724pt}}
\pgflineto{\pgfpoint{119.321129pt}{331.654724pt}}
\pgfusepath{stroke}
\pgfpathmoveto{\pgfpoint{110.663406pt}{337.831604pt}}
\pgflineto{\pgfpoint{119.312225pt}{337.831604pt}}
\pgfusepath{stroke}
\pgfpathmoveto{\pgfpoint{110.663345pt}{344.008423pt}}
\pgflineto{\pgfpoint{119.303131pt}{344.008423pt}}
\pgfusepath{stroke}
\pgfpathmoveto{\pgfpoint{110.663330pt}{350.185242pt}}
\pgflineto{\pgfpoint{119.298439pt}{350.185242pt}}
\pgfusepath{stroke}
\pgfpathmoveto{\pgfpoint{110.663383pt}{356.362122pt}}
\pgflineto{\pgfpoint{119.289459pt}{356.362122pt}}
\pgfusepath{stroke}
\pgfpathmoveto{\pgfpoint{110.663269pt}{362.538940pt}}
\pgflineto{\pgfpoint{119.280380pt}{362.538940pt}}
\pgfusepath{stroke}
\pgfpathmoveto{\pgfpoint{110.667816pt}{368.715820pt}}
\pgflineto{\pgfpoint{119.275940pt}{368.715820pt}}
\pgfusepath{stroke}
\pgfpathmoveto{\pgfpoint{110.667816pt}{374.892639pt}}
\pgflineto{\pgfpoint{119.266846pt}{374.892639pt}}
\pgfusepath{stroke}
\pgfpathmoveto{\pgfpoint{110.672249pt}{381.069458pt}}
\pgflineto{\pgfpoint{119.262306pt}{381.069458pt}}
\pgfusepath{stroke}
\pgfpathmoveto{\pgfpoint{110.672249pt}{387.246338pt}}
\pgflineto{\pgfpoint{119.253273pt}{387.246338pt}}
\pgfusepath{stroke}
\pgfpathmoveto{\pgfpoint{110.672234pt}{393.423157pt}}
\pgflineto{\pgfpoint{119.248627pt}{393.423157pt}}
\pgfusepath{stroke}
\pgfpathmoveto{\pgfpoint{110.676743pt}{399.600037pt}}
\pgflineto{\pgfpoint{119.239746pt}{399.600037pt}}
\pgfusepath{stroke}
\pgfpathmoveto{\pgfpoint{119.519989pt}{140.172638pt}}
\pgflineto{\pgfpoint{119.519989pt}{133.995789pt}}
\pgfusepath{stroke}
\pgfpathmoveto{\pgfpoint{119.519989pt}{133.995789pt}}
\pgflineto{\pgfpoint{119.519989pt}{127.818947pt}}
\pgfusepath{stroke}
\pgfpathmoveto{\pgfpoint{119.519989pt}{146.349472pt}}
\pgflineto{\pgfpoint{119.519989pt}{140.172638pt}}
\pgfusepath{stroke}
\pgfpathmoveto{\pgfpoint{119.519989pt}{152.526306pt}}
\pgflineto{\pgfpoint{119.519989pt}{146.349472pt}}
\pgfusepath{stroke}
\pgfpathmoveto{\pgfpoint{119.519989pt}{133.995789pt}}
\pgflineto{\pgfpoint{119.538002pt}{133.995789pt}}
\pgfusepath{stroke}
\pgfpathmoveto{\pgfpoint{119.519989pt}{140.172638pt}}
\pgflineto{\pgfpoint{119.538002pt}{140.172638pt}}
\pgfusepath{stroke}
\pgfpathmoveto{\pgfpoint{119.519989pt}{146.349472pt}}
\pgflineto{\pgfpoint{119.538002pt}{146.349472pt}}
\pgfusepath{stroke}
\pgfpathmoveto{\pgfpoint{119.519989pt}{164.880005pt}}
\pgflineto{\pgfpoint{119.519989pt}{158.703156pt}}
\pgfusepath{stroke}
\pgfpathmoveto{\pgfpoint{119.519989pt}{158.703156pt}}
\pgflineto{\pgfpoint{119.519989pt}{152.526306pt}}
\pgfusepath{stroke}
\pgfpathmoveto{\pgfpoint{119.519989pt}{171.056854pt}}
\pgflineto{\pgfpoint{119.519989pt}{164.880005pt}}
\pgfusepath{stroke}
\pgfpathmoveto{\pgfpoint{119.519989pt}{152.526306pt}}
\pgflineto{\pgfpoint{119.538025pt}{152.526306pt}}
\pgfusepath{stroke}
\pgfpathmoveto{\pgfpoint{119.483887pt}{189.587372pt}}
\pgflineto{\pgfpoint{119.502075pt}{189.587372pt}}
\pgfusepath{stroke}
\pgfpathmoveto{\pgfpoint{119.519989pt}{195.764206pt}}
\pgflineto{\pgfpoint{119.483856pt}{195.764206pt}}
\pgfusepath{stroke}
\pgfpathmoveto{\pgfpoint{119.519989pt}{201.941055pt}}
\pgflineto{\pgfpoint{119.474785pt}{201.941055pt}}
\pgfusepath{stroke}
\pgfpathmoveto{\pgfpoint{119.519989pt}{208.117905pt}}
\pgflineto{\pgfpoint{119.465767pt}{208.117905pt}}
\pgfusepath{stroke}
\pgfpathmoveto{\pgfpoint{119.519989pt}{214.294739pt}}
\pgflineto{\pgfpoint{119.456734pt}{214.294739pt}}
\pgfusepath{stroke}
\pgfpathmoveto{\pgfpoint{119.519989pt}{220.471588pt}}
\pgflineto{\pgfpoint{119.447723pt}{220.471588pt}}
\pgfusepath{stroke}
\pgfpathmoveto{\pgfpoint{119.519989pt}{226.648422pt}}
\pgflineto{\pgfpoint{119.447655pt}{226.648422pt}}
\pgfusepath{stroke}
\pgfpathmoveto{\pgfpoint{119.519989pt}{232.825272pt}}
\pgflineto{\pgfpoint{119.438583pt}{232.825272pt}}
\pgfusepath{stroke}
\pgfpathmoveto{\pgfpoint{119.519989pt}{239.002106pt}}
\pgflineto{\pgfpoint{119.429642pt}{239.002106pt}}
\pgfusepath{stroke}
\pgfpathmoveto{\pgfpoint{119.519989pt}{245.178955pt}}
\pgflineto{\pgfpoint{119.420670pt}{245.178955pt}}
\pgfusepath{stroke}
\pgfpathmoveto{\pgfpoint{119.519989pt}{251.355804pt}}
\pgflineto{\pgfpoint{119.411621pt}{251.355804pt}}
\pgfusepath{stroke}
\pgfpathmoveto{\pgfpoint{119.519989pt}{257.532623pt}}
\pgflineto{\pgfpoint{119.411415pt}{257.532623pt}}
\pgfusepath{stroke}
\pgfpathmoveto{\pgfpoint{119.519989pt}{263.709473pt}}
\pgflineto{\pgfpoint{119.402405pt}{263.709473pt}}
\pgfusepath{stroke}
\pgfpathmoveto{\pgfpoint{119.519989pt}{269.886322pt}}
\pgflineto{\pgfpoint{119.393517pt}{269.886322pt}}
\pgfusepath{stroke}
\pgfpathmoveto{\pgfpoint{119.519989pt}{276.063141pt}}
\pgflineto{\pgfpoint{119.384415pt}{276.063141pt}}
\pgfusepath{stroke}
\pgfpathmoveto{\pgfpoint{119.519989pt}{282.239990pt}}
\pgflineto{\pgfpoint{119.384315pt}{282.239990pt}}
\pgfusepath{stroke}
\pgfpathmoveto{\pgfpoint{119.519989pt}{288.416840pt}}
\pgflineto{\pgfpoint{119.375221pt}{288.416840pt}}
\pgfusepath{stroke}
\pgfpathmoveto{\pgfpoint{119.519989pt}{294.593689pt}}
\pgflineto{\pgfpoint{119.366211pt}{294.593689pt}}
\pgfusepath{stroke}
\pgfpathmoveto{\pgfpoint{119.519989pt}{300.770538pt}}
\pgflineto{\pgfpoint{119.357391pt}{300.770538pt}}
\pgfusepath{stroke}
\pgfpathmoveto{\pgfpoint{119.519989pt}{306.947388pt}}
\pgflineto{\pgfpoint{119.348351pt}{306.947388pt}}
\pgfusepath{stroke}
\pgfpathmoveto{\pgfpoint{119.519989pt}{313.124207pt}}
\pgflineto{\pgfpoint{119.348114pt}{313.124207pt}}
\pgfusepath{stroke}
\pgfpathmoveto{\pgfpoint{119.519989pt}{319.301056pt}}
\pgflineto{\pgfpoint{119.339066pt}{319.301056pt}}
\pgfusepath{stroke}
\pgfpathmoveto{\pgfpoint{119.330032pt}{325.477905pt}}
\pgflineto{\pgfpoint{119.501984pt}{325.477905pt}}
\pgfusepath{stroke}
\pgfpathmoveto{\pgfpoint{119.321129pt}{331.654724pt}}
\pgflineto{\pgfpoint{119.492966pt}{331.654724pt}}
\pgfusepath{stroke}
\pgfpathmoveto{\pgfpoint{119.312225pt}{337.831604pt}}
\pgflineto{\pgfpoint{119.484009pt}{337.831604pt}}
\pgfusepath{stroke}
\pgfpathmoveto{\pgfpoint{119.303131pt}{344.008423pt}}
\pgflineto{\pgfpoint{119.474976pt}{344.008423pt}}
\pgfusepath{stroke}
\pgfpathmoveto{\pgfpoint{119.298439pt}{350.185242pt}}
\pgflineto{\pgfpoint{119.470482pt}{350.185242pt}}
\pgfusepath{stroke}
\pgfpathmoveto{\pgfpoint{119.289459pt}{356.362122pt}}
\pgflineto{\pgfpoint{119.461502pt}{356.362122pt}}
\pgfusepath{stroke}
\pgfpathmoveto{\pgfpoint{119.280380pt}{362.538940pt}}
\pgflineto{\pgfpoint{119.452477pt}{362.538940pt}}
\pgfusepath{stroke}
\pgfpathmoveto{\pgfpoint{119.275940pt}{368.715820pt}}
\pgflineto{\pgfpoint{119.448036pt}{368.715820pt}}
\pgfusepath{stroke}
\pgfpathmoveto{\pgfpoint{119.266846pt}{374.892639pt}}
\pgflineto{\pgfpoint{119.439011pt}{374.892639pt}}
\pgfusepath{stroke}
\pgfpathmoveto{\pgfpoint{119.262306pt}{381.069458pt}}
\pgflineto{\pgfpoint{119.434464pt}{381.069458pt}}
\pgfusepath{stroke}
\pgfpathmoveto{\pgfpoint{119.253273pt}{387.246338pt}}
\pgflineto{\pgfpoint{119.425545pt}{387.246338pt}}
\pgfusepath{stroke}
\pgfpathmoveto{\pgfpoint{119.248627pt}{393.423157pt}}
\pgflineto{\pgfpoint{119.420982pt}{393.423157pt}}
\pgfusepath{stroke}
\pgfpathmoveto{\pgfpoint{119.239746pt}{399.600037pt}}
\pgflineto{\pgfpoint{119.412018pt}{399.600037pt}}
\pgfusepath{stroke}
\pgfpathmoveto{\pgfpoint{119.519989pt}{288.416840pt}}
\pgflineto{\pgfpoint{119.519989pt}{282.239990pt}}
\pgfusepath{stroke}
\pgfpathmoveto{\pgfpoint{119.519989pt}{282.239990pt}}
\pgflineto{\pgfpoint{119.519989pt}{276.063141pt}}
\pgfusepath{stroke}
\pgfpathmoveto{\pgfpoint{119.519989pt}{294.593689pt}}
\pgflineto{\pgfpoint{119.519989pt}{288.416840pt}}
\pgfusepath{stroke}
\pgfpathmoveto{\pgfpoint{119.519989pt}{300.770538pt}}
\pgflineto{\pgfpoint{119.519989pt}{294.593689pt}}
\pgfusepath{stroke}
\pgfpathmoveto{\pgfpoint{119.519989pt}{282.239990pt}}
\pgflineto{\pgfpoint{119.537834pt}{282.239990pt}}
\pgfusepath{stroke}
\pgfpathmoveto{\pgfpoint{119.519989pt}{288.416840pt}}
\pgflineto{\pgfpoint{119.537834pt}{288.416840pt}}
\pgfusepath{stroke}
\pgfpathmoveto{\pgfpoint{119.519989pt}{294.593689pt}}
\pgflineto{\pgfpoint{119.537773pt}{294.593689pt}}
\pgfusepath{stroke}
\pgfpathmoveto{\pgfpoint{119.519989pt}{313.124207pt}}
\pgflineto{\pgfpoint{119.519989pt}{306.947388pt}}
\pgfusepath{stroke}
\pgfpathmoveto{\pgfpoint{119.519989pt}{306.947388pt}}
\pgflineto{\pgfpoint{119.519989pt}{300.770538pt}}
\pgfusepath{stroke}
\pgfpathmoveto{\pgfpoint{119.519989pt}{319.301056pt}}
\pgflineto{\pgfpoint{119.519989pt}{313.124207pt}}
\pgfusepath{stroke}
\pgfpathmoveto{\pgfpoint{119.519989pt}{300.770538pt}}
\pgflineto{\pgfpoint{119.537788pt}{300.770538pt}}
\pgfusepath{stroke}
\pgfpathmoveto{\pgfpoint{119.519989pt}{331.654724pt}}
\pgflineto{\pgfpoint{119.492966pt}{331.654724pt}}
\pgfusepath{stroke}
\pgfpathmoveto{\pgfpoint{119.519989pt}{337.831604pt}}
\pgflineto{\pgfpoint{119.484009pt}{337.831604pt}}
\pgfusepath{stroke}
\pgfpathmoveto{\pgfpoint{119.519989pt}{344.008423pt}}
\pgflineto{\pgfpoint{119.474976pt}{344.008423pt}}
\pgfusepath{stroke}
\pgfpathmoveto{\pgfpoint{119.519989pt}{350.185242pt}}
\pgflineto{\pgfpoint{119.470482pt}{350.185242pt}}
\pgfusepath{stroke}
\pgfpathmoveto{\pgfpoint{119.519989pt}{356.362122pt}}
\pgflineto{\pgfpoint{119.461502pt}{356.362122pt}}
\pgfusepath{stroke}
\pgfpathmoveto{\pgfpoint{119.519989pt}{362.538940pt}}
\pgflineto{\pgfpoint{119.452477pt}{362.538940pt}}
\pgfusepath{stroke}
\pgfpathmoveto{\pgfpoint{119.519989pt}{368.715820pt}}
\pgflineto{\pgfpoint{119.448036pt}{368.715820pt}}
\pgfusepath{stroke}
\pgfpathmoveto{\pgfpoint{119.519989pt}{374.892639pt}}
\pgflineto{\pgfpoint{119.439011pt}{374.892639pt}}
\pgfusepath{stroke}
\pgfpathmoveto{\pgfpoint{119.519989pt}{381.069458pt}}
\pgflineto{\pgfpoint{119.434464pt}{381.069458pt}}
\pgfusepath{stroke}
\pgfpathmoveto{\pgfpoint{119.519989pt}{387.246338pt}}
\pgflineto{\pgfpoint{119.425545pt}{387.246338pt}}
\pgfusepath{stroke}
\pgfpathmoveto{\pgfpoint{119.519989pt}{393.423157pt}}
\pgflineto{\pgfpoint{119.420982pt}{393.423157pt}}
\pgfusepath{stroke}
\pgfpathmoveto{\pgfpoint{119.519989pt}{399.600037pt}}
\pgflineto{\pgfpoint{119.412018pt}{399.600037pt}}
\pgfusepath{stroke}
\pgfpathmoveto{\pgfpoint{119.519989pt}{374.892639pt}}
\pgflineto{\pgfpoint{119.519989pt}{368.715820pt}}
\pgfusepath{stroke}
\pgfpathmoveto{\pgfpoint{119.519989pt}{325.477905pt}}
\pgflineto{\pgfpoint{119.519989pt}{319.301056pt}}
\pgfusepath{stroke}
\pgfpathmoveto{\pgfpoint{119.519989pt}{331.654724pt}}
\pgflineto{\pgfpoint{119.519989pt}{325.477905pt}}
\pgfusepath{stroke}
\pgfpathmoveto{\pgfpoint{119.519989pt}{337.831604pt}}
\pgflineto{\pgfpoint{119.519989pt}{331.654724pt}}
\pgfusepath{stroke}
\pgfpathmoveto{\pgfpoint{119.519989pt}{344.008423pt}}
\pgflineto{\pgfpoint{119.519989pt}{337.831604pt}}
\pgfusepath{stroke}
\pgfpathmoveto{\pgfpoint{119.519989pt}{350.185242pt}}
\pgflineto{\pgfpoint{119.519989pt}{344.008423pt}}
\pgfusepath{stroke}
\pgfpathmoveto{\pgfpoint{119.519989pt}{356.362122pt}}
\pgflineto{\pgfpoint{119.519989pt}{350.185242pt}}
\pgfusepath{stroke}
\pgfpathmoveto{\pgfpoint{119.519989pt}{362.538940pt}}
\pgflineto{\pgfpoint{119.519989pt}{356.362122pt}}
\pgfusepath{stroke}
\pgfpathmoveto{\pgfpoint{119.519989pt}{368.715820pt}}
\pgflineto{\pgfpoint{119.519989pt}{362.538940pt}}
\pgfusepath{stroke}
\pgfpathmoveto{\pgfpoint{119.519989pt}{381.069458pt}}
\pgflineto{\pgfpoint{119.519989pt}{374.892639pt}}
\pgfusepath{stroke}
\pgfpathmoveto{\pgfpoint{119.519989pt}{387.246338pt}}
\pgflineto{\pgfpoint{119.519989pt}{381.069458pt}}
\pgfusepath{stroke}
\pgfpathmoveto{\pgfpoint{119.519989pt}{393.423157pt}}
\pgflineto{\pgfpoint{119.519989pt}{387.246338pt}}
\pgfusepath{stroke}
\pgfpathmoveto{\pgfpoint{119.519989pt}{399.600037pt}}
\pgflineto{\pgfpoint{119.519989pt}{393.423157pt}}
\pgfusepath{stroke}
\pgfpathmoveto{\pgfpoint{119.519989pt}{325.477905pt}}
\pgflineto{\pgfpoint{119.501984pt}{325.477905pt}}
\pgfusepath{stroke}
\pgfpathmoveto{\pgfpoint{119.519989pt}{306.947388pt}}
\pgflineto{\pgfpoint{119.537788pt}{306.947388pt}}
\pgfusepath{stroke}
\pgfpathmoveto{\pgfpoint{119.519989pt}{313.124207pt}}
\pgflineto{\pgfpoint{119.546753pt}{313.124207pt}}
\pgfusepath{stroke}
\pgfpathmoveto{\pgfpoint{119.519989pt}{319.301056pt}}
\pgflineto{\pgfpoint{119.546753pt}{319.301056pt}}
\pgfusepath{stroke}
\pgfpathmoveto{\pgfpoint{119.519989pt}{325.477905pt}}
\pgflineto{\pgfpoint{119.546776pt}{325.477905pt}}
\pgfusepath{stroke}
\pgfpathmoveto{\pgfpoint{119.519989pt}{331.654724pt}}
\pgflineto{\pgfpoint{119.546722pt}{331.654724pt}}
\pgfusepath{stroke}
\pgfpathmoveto{\pgfpoint{119.519989pt}{337.831604pt}}
\pgflineto{\pgfpoint{119.546761pt}{337.831604pt}}
\pgfusepath{stroke}
\pgfpathmoveto{\pgfpoint{119.519989pt}{344.008423pt}}
\pgflineto{\pgfpoint{119.546646pt}{344.008423pt}}
\pgfusepath{stroke}
\pgfpathmoveto{\pgfpoint{119.519989pt}{350.185242pt}}
\pgflineto{\pgfpoint{119.551147pt}{350.185242pt}}
\pgfusepath{stroke}
\pgfpathmoveto{\pgfpoint{119.519989pt}{356.362122pt}}
\pgflineto{\pgfpoint{119.551147pt}{356.362122pt}}
\pgfusepath{stroke}
\pgfpathmoveto{\pgfpoint{119.519989pt}{362.538940pt}}
\pgflineto{\pgfpoint{119.551163pt}{362.538940pt}}
\pgfusepath{stroke}
\pgfpathmoveto{\pgfpoint{119.519989pt}{368.715820pt}}
\pgflineto{\pgfpoint{119.555634pt}{368.715820pt}}
\pgfusepath{stroke}
\pgfpathmoveto{\pgfpoint{119.519989pt}{374.892639pt}}
\pgflineto{\pgfpoint{119.555634pt}{374.892639pt}}
\pgfusepath{stroke}
\pgfpathmoveto{\pgfpoint{119.519989pt}{381.069458pt}}
\pgflineto{\pgfpoint{119.560066pt}{381.069458pt}}
\pgfusepath{stroke}
\pgfpathmoveto{\pgfpoint{119.519989pt}{387.246338pt}}
\pgflineto{\pgfpoint{119.560066pt}{387.246338pt}}
\pgfusepath{stroke}
\pgfpathmoveto{\pgfpoint{119.519989pt}{393.423157pt}}
\pgflineto{\pgfpoint{119.564560pt}{393.423157pt}}
\pgfusepath{stroke}
\pgfpathmoveto{\pgfpoint{119.519989pt}{399.600037pt}}
\pgflineto{\pgfpoint{119.564560pt}{399.600037pt}}
\pgfusepath{stroke}
\pgfpathmoveto{\pgfpoint{119.519989pt}{232.825272pt}}
\pgflineto{\pgfpoint{119.519989pt}{226.648422pt}}
\pgfusepath{stroke}
\pgfpathmoveto{\pgfpoint{119.519989pt}{201.941055pt}}
\pgflineto{\pgfpoint{119.519989pt}{195.764206pt}}
\pgfusepath{stroke}
\pgfpathmoveto{\pgfpoint{119.519989pt}{208.117905pt}}
\pgflineto{\pgfpoint{119.519989pt}{201.941055pt}}
\pgfusepath{stroke}
\pgfpathmoveto{\pgfpoint{119.519989pt}{214.294739pt}}
\pgflineto{\pgfpoint{119.519989pt}{208.117905pt}}
\pgfusepath{stroke}
\pgfpathmoveto{\pgfpoint{119.519989pt}{220.471588pt}}
\pgflineto{\pgfpoint{119.519989pt}{214.294739pt}}
\pgfusepath{stroke}
\pgfpathmoveto{\pgfpoint{119.519989pt}{226.648422pt}}
\pgflineto{\pgfpoint{119.519989pt}{220.471588pt}}
\pgfusepath{stroke}
\pgfpathmoveto{\pgfpoint{119.519989pt}{239.002106pt}}
\pgflineto{\pgfpoint{119.519989pt}{232.825272pt}}
\pgfusepath{stroke}
\pgfpathmoveto{\pgfpoint{119.519989pt}{245.178955pt}}
\pgflineto{\pgfpoint{119.519989pt}{239.002106pt}}
\pgfusepath{stroke}
\pgfpathmoveto{\pgfpoint{119.519989pt}{251.355804pt}}
\pgflineto{\pgfpoint{119.519989pt}{245.178955pt}}
\pgfusepath{stroke}
\pgfpathmoveto{\pgfpoint{119.519989pt}{257.532623pt}}
\pgflineto{\pgfpoint{119.519989pt}{251.355804pt}}
\pgfusepath{stroke}
\pgfpathmoveto{\pgfpoint{119.519989pt}{263.709473pt}}
\pgflineto{\pgfpoint{119.519989pt}{257.532623pt}}
\pgfusepath{stroke}
\pgfpathmoveto{\pgfpoint{119.519989pt}{269.886322pt}}
\pgflineto{\pgfpoint{119.519989pt}{263.709473pt}}
\pgfusepath{stroke}
\pgfpathmoveto{\pgfpoint{119.519989pt}{276.063141pt}}
\pgflineto{\pgfpoint{119.519989pt}{269.886322pt}}
\pgfusepath{stroke}
\pgfpathmoveto{\pgfpoint{119.519989pt}{177.233673pt}}
\pgflineto{\pgfpoint{119.501907pt}{177.233673pt}}
\pgfusepath{stroke}
\pgfpathmoveto{\pgfpoint{119.519989pt}{183.410522pt}}
\pgflineto{\pgfpoint{119.492920pt}{183.410522pt}}
\pgfusepath{stroke}
\pgfpathmoveto{\pgfpoint{119.519989pt}{189.587372pt}}
\pgflineto{\pgfpoint{119.502075pt}{189.587372pt}}
\pgfusepath{stroke}
\pgfpathmoveto{\pgfpoint{119.519989pt}{177.233673pt}}
\pgflineto{\pgfpoint{119.519989pt}{171.056854pt}}
\pgfusepath{stroke}
\pgfpathmoveto{\pgfpoint{119.519989pt}{183.410522pt}}
\pgflineto{\pgfpoint{119.519989pt}{177.233673pt}}
\pgfusepath{stroke}
\pgfpathmoveto{\pgfpoint{119.519989pt}{158.703156pt}}
\pgflineto{\pgfpoint{119.538025pt}{158.703156pt}}
\pgfusepath{stroke}
\pgfpathmoveto{\pgfpoint{119.519989pt}{164.880005pt}}
\pgflineto{\pgfpoint{119.546982pt}{164.880005pt}}
\pgfusepath{stroke}
\pgfpathmoveto{\pgfpoint{119.519989pt}{171.056854pt}}
\pgflineto{\pgfpoint{119.546982pt}{171.056854pt}}
\pgfusepath{stroke}
\pgfpathmoveto{\pgfpoint{119.519989pt}{177.233673pt}}
\pgflineto{\pgfpoint{119.547012pt}{177.233673pt}}
\pgfusepath{stroke}
\pgfpathmoveto{\pgfpoint{119.519989pt}{195.764206pt}}
\pgflineto{\pgfpoint{119.519989pt}{189.587372pt}}
\pgfusepath{stroke}
\pgfpathmoveto{\pgfpoint{119.519989pt}{189.587372pt}}
\pgflineto{\pgfpoint{119.519989pt}{183.410522pt}}
\pgfusepath{stroke}
\pgfpathmoveto{\pgfpoint{119.519989pt}{183.410522pt}}
\pgflineto{\pgfpoint{119.547012pt}{183.410522pt}}
\pgfusepath{stroke}
\pgfpathmoveto{\pgfpoint{119.519989pt}{189.587372pt}}
\pgflineto{\pgfpoint{119.547012pt}{189.587372pt}}
\pgfusepath{stroke}
\pgfpathmoveto{\pgfpoint{119.519989pt}{195.764206pt}}
\pgflineto{\pgfpoint{119.555962pt}{195.764206pt}}
\pgfusepath{stroke}
\pgfpathmoveto{\pgfpoint{119.519989pt}{201.941055pt}}
\pgflineto{\pgfpoint{119.556015pt}{201.941055pt}}
\pgfusepath{stroke}
\pgfpathmoveto{\pgfpoint{119.519989pt}{208.117905pt}}
\pgflineto{\pgfpoint{119.556053pt}{208.117905pt}}
\pgfusepath{stroke}
\pgfpathmoveto{\pgfpoint{119.519989pt}{214.294739pt}}
\pgflineto{\pgfpoint{119.556000pt}{214.294739pt}}
\pgfusepath{stroke}
\pgfpathmoveto{\pgfpoint{119.519989pt}{220.471588pt}}
\pgflineto{\pgfpoint{119.556000pt}{220.471588pt}}
\pgfusepath{stroke}
\pgfpathmoveto{\pgfpoint{119.519989pt}{226.648422pt}}
\pgflineto{\pgfpoint{119.565056pt}{226.648422pt}}
\pgfusepath{stroke}
\pgfpathmoveto{\pgfpoint{119.519989pt}{232.825272pt}}
\pgflineto{\pgfpoint{119.564995pt}{232.825272pt}}
\pgfusepath{stroke}
\pgfpathmoveto{\pgfpoint{119.519989pt}{239.002106pt}}
\pgflineto{\pgfpoint{119.565041pt}{239.002106pt}}
\pgfusepath{stroke}
\pgfpathmoveto{\pgfpoint{119.519989pt}{245.178955pt}}
\pgflineto{\pgfpoint{119.565041pt}{245.178955pt}}
\pgfusepath{stroke}
\pgfpathmoveto{\pgfpoint{119.519989pt}{251.355804pt}}
\pgflineto{\pgfpoint{119.565041pt}{251.355804pt}}
\pgfusepath{stroke}
\pgfpathmoveto{\pgfpoint{119.519989pt}{257.532623pt}}
\pgflineto{\pgfpoint{119.574036pt}{257.532623pt}}
\pgfusepath{stroke}
\pgfpathmoveto{\pgfpoint{119.519989pt}{263.709473pt}}
\pgflineto{\pgfpoint{119.573975pt}{263.709473pt}}
\pgfusepath{stroke}
\pgfpathmoveto{\pgfpoint{119.519989pt}{269.886322pt}}
\pgflineto{\pgfpoint{119.574089pt}{269.886322pt}}
\pgfusepath{stroke}
\pgfpathmoveto{\pgfpoint{119.519989pt}{276.063141pt}}
\pgflineto{\pgfpoint{119.574028pt}{276.063141pt}}
\pgfusepath{stroke}
\pgfpathmoveto{\pgfpoint{119.583015pt}{282.239990pt}}
\pgflineto{\pgfpoint{119.537834pt}{282.239990pt}}
\pgfusepath{stroke}
\pgfpathmoveto{\pgfpoint{119.583015pt}{288.416840pt}}
\pgflineto{\pgfpoint{119.537834pt}{288.416840pt}}
\pgfusepath{stroke}
\pgfpathmoveto{\pgfpoint{119.583015pt}{294.593689pt}}
\pgflineto{\pgfpoint{119.537773pt}{294.593689pt}}
\pgfusepath{stroke}
\pgfpathmoveto{\pgfpoint{119.583076pt}{300.770538pt}}
\pgflineto{\pgfpoint{119.537788pt}{300.770538pt}}
\pgfusepath{stroke}
\pgfpathmoveto{\pgfpoint{119.583076pt}{306.947388pt}}
\pgflineto{\pgfpoint{119.537788pt}{306.947388pt}}
\pgfusepath{stroke}
\pgfpathmoveto{\pgfpoint{119.591988pt}{313.124207pt}}
\pgflineto{\pgfpoint{119.546753pt}{313.124207pt}}
\pgfusepath{stroke}
\pgfpathmoveto{\pgfpoint{119.591988pt}{319.301056pt}}
\pgflineto{\pgfpoint{119.546753pt}{319.301056pt}}
\pgfusepath{stroke}
\pgfpathmoveto{\pgfpoint{119.592064pt}{325.477905pt}}
\pgflineto{\pgfpoint{119.546776pt}{325.477905pt}}
\pgfusepath{stroke}
\pgfpathmoveto{\pgfpoint{119.592064pt}{331.654724pt}}
\pgflineto{\pgfpoint{119.546722pt}{331.654724pt}}
\pgfusepath{stroke}
\pgfpathmoveto{\pgfpoint{119.592026pt}{337.831604pt}}
\pgflineto{\pgfpoint{119.546761pt}{337.831604pt}}
\pgfusepath{stroke}
\pgfpathmoveto{\pgfpoint{119.592026pt}{344.008423pt}}
\pgflineto{\pgfpoint{119.546646pt}{344.008423pt}}
\pgfusepath{stroke}
\pgfpathmoveto{\pgfpoint{119.596519pt}{350.185242pt}}
\pgflineto{\pgfpoint{119.551147pt}{350.185242pt}}
\pgfusepath{stroke}
\pgfpathmoveto{\pgfpoint{119.596519pt}{356.362122pt}}
\pgflineto{\pgfpoint{119.551147pt}{356.362122pt}}
\pgfusepath{stroke}
\pgfpathmoveto{\pgfpoint{119.596558pt}{362.538940pt}}
\pgflineto{\pgfpoint{119.551163pt}{362.538940pt}}
\pgfusepath{stroke}
\pgfpathmoveto{\pgfpoint{119.601067pt}{368.715820pt}}
\pgflineto{\pgfpoint{119.555634pt}{368.715820pt}}
\pgfusepath{stroke}
\pgfpathmoveto{\pgfpoint{119.601013pt}{374.892639pt}}
\pgflineto{\pgfpoint{119.555634pt}{374.892639pt}}
\pgfusepath{stroke}
\pgfpathmoveto{\pgfpoint{119.605560pt}{381.069458pt}}
\pgflineto{\pgfpoint{119.560066pt}{381.069458pt}}
\pgfusepath{stroke}
\pgfpathmoveto{\pgfpoint{119.605560pt}{387.246338pt}}
\pgflineto{\pgfpoint{119.560066pt}{387.246338pt}}
\pgfusepath{stroke}
\pgfpathmoveto{\pgfpoint{119.610054pt}{393.423157pt}}
\pgflineto{\pgfpoint{119.564560pt}{393.423157pt}}
\pgfusepath{stroke}
\pgfpathmoveto{\pgfpoint{119.610107pt}{399.600037pt}}
\pgflineto{\pgfpoint{119.564560pt}{399.600037pt}}
\pgfusepath{stroke}
\pgfpathmoveto{\pgfpoint{119.519989pt}{84.581039pt}}
\pgflineto{\pgfpoint{119.519989pt}{78.404205pt}}
\pgfusepath{stroke}
\pgfpathmoveto{\pgfpoint{119.519989pt}{53.696838pt}}
\pgflineto{\pgfpoint{119.519989pt}{47.519989pt}}
\pgfusepath{stroke}
\pgfpathmoveto{\pgfpoint{119.519989pt}{59.873672pt}}
\pgflineto{\pgfpoint{119.519989pt}{53.696838pt}}
\pgfusepath{stroke}
\pgfpathmoveto{\pgfpoint{119.519989pt}{66.050522pt}}
\pgflineto{\pgfpoint{119.519989pt}{59.873672pt}}
\pgfusepath{stroke}
\pgfpathmoveto{\pgfpoint{119.519989pt}{72.227356pt}}
\pgflineto{\pgfpoint{119.519989pt}{66.050522pt}}
\pgfusepath{stroke}
\pgfpathmoveto{\pgfpoint{119.519989pt}{78.404205pt}}
\pgflineto{\pgfpoint{119.519989pt}{72.227356pt}}
\pgfusepath{stroke}
\pgfpathmoveto{\pgfpoint{119.519989pt}{90.757896pt}}
\pgflineto{\pgfpoint{119.519989pt}{84.581039pt}}
\pgfusepath{stroke}
\pgfpathmoveto{\pgfpoint{119.519989pt}{96.934731pt}}
\pgflineto{\pgfpoint{119.519989pt}{90.757896pt}}
\pgfusepath{stroke}
\pgfpathmoveto{\pgfpoint{119.519989pt}{103.111580pt}}
\pgflineto{\pgfpoint{119.519989pt}{96.934731pt}}
\pgfusepath{stroke}
\pgfpathmoveto{\pgfpoint{119.519989pt}{109.288422pt}}
\pgflineto{\pgfpoint{119.519989pt}{103.111580pt}}
\pgfusepath{stroke}
\pgfpathmoveto{\pgfpoint{119.519989pt}{115.465263pt}}
\pgflineto{\pgfpoint{119.519989pt}{109.288422pt}}
\pgfusepath{stroke}
\pgfpathmoveto{\pgfpoint{119.519989pt}{121.642097pt}}
\pgflineto{\pgfpoint{119.519989pt}{115.465263pt}}
\pgfusepath{stroke}
\pgfpathmoveto{\pgfpoint{119.519989pt}{127.818947pt}}
\pgflineto{\pgfpoint{119.519989pt}{121.642097pt}}
\pgfusepath{stroke}
\pgfpathmoveto{\pgfpoint{119.519989pt}{47.519989pt}}
\pgflineto{\pgfpoint{128.429581pt}{47.519989pt}}
\pgfusepath{stroke}
\pgfpathmoveto{\pgfpoint{119.519989pt}{53.696838pt}}
\pgflineto{\pgfpoint{128.429703pt}{53.696838pt}}
\pgfusepath{stroke}
\pgfpathmoveto{\pgfpoint{128.447998pt}{59.873672pt}}
\pgflineto{\pgfpoint{119.519989pt}{59.873672pt}}
\pgfusepath{stroke}
\pgfpathmoveto{\pgfpoint{128.447998pt}{66.050522pt}}
\pgflineto{\pgfpoint{119.519989pt}{66.050522pt}}
\pgfusepath{stroke}
\pgfpathmoveto{\pgfpoint{128.447998pt}{72.227356pt}}
\pgflineto{\pgfpoint{119.519989pt}{72.227356pt}}
\pgfusepath{stroke}
\pgfpathmoveto{\pgfpoint{128.447998pt}{78.404205pt}}
\pgflineto{\pgfpoint{119.519989pt}{78.404205pt}}
\pgfusepath{stroke}
\pgfpathmoveto{\pgfpoint{128.447998pt}{84.581039pt}}
\pgflineto{\pgfpoint{119.519989pt}{84.581039pt}}
\pgfusepath{stroke}
\pgfpathmoveto{\pgfpoint{128.447998pt}{90.757896pt}}
\pgflineto{\pgfpoint{119.519989pt}{90.757896pt}}
\pgfusepath{stroke}
\pgfpathmoveto{\pgfpoint{128.447998pt}{96.934731pt}}
\pgflineto{\pgfpoint{119.519989pt}{96.934731pt}}
\pgfusepath{stroke}
\pgfpathmoveto{\pgfpoint{128.447998pt}{103.111580pt}}
\pgflineto{\pgfpoint{119.519989pt}{103.111580pt}}
\pgfusepath{stroke}
\pgfpathmoveto{\pgfpoint{128.447998pt}{109.288422pt}}
\pgflineto{\pgfpoint{119.519989pt}{109.288422pt}}
\pgfusepath{stroke}
\pgfpathmoveto{\pgfpoint{128.447998pt}{115.465263pt}}
\pgflineto{\pgfpoint{119.519989pt}{115.465263pt}}
\pgfusepath{stroke}
\pgfpathmoveto{\pgfpoint{128.447998pt}{121.642097pt}}
\pgflineto{\pgfpoint{119.519989pt}{121.642097pt}}
\pgfusepath{stroke}
\pgfpathmoveto{\pgfpoint{128.447998pt}{127.818947pt}}
\pgflineto{\pgfpoint{119.519989pt}{127.818947pt}}
\pgfusepath{stroke}
\pgfpathmoveto{\pgfpoint{128.447998pt}{133.995789pt}}
\pgflineto{\pgfpoint{119.538002pt}{133.995789pt}}
\pgfusepath{stroke}
\pgfpathmoveto{\pgfpoint{128.447998pt}{140.172638pt}}
\pgflineto{\pgfpoint{119.538002pt}{140.172638pt}}
\pgfusepath{stroke}
\pgfpathmoveto{\pgfpoint{128.447998pt}{146.349472pt}}
\pgflineto{\pgfpoint{119.538002pt}{146.349472pt}}
\pgfusepath{stroke}
\pgfpathmoveto{\pgfpoint{128.447998pt}{152.526306pt}}
\pgflineto{\pgfpoint{119.538025pt}{152.526306pt}}
\pgfusepath{stroke}
\pgfpathmoveto{\pgfpoint{128.447998pt}{158.703156pt}}
\pgflineto{\pgfpoint{119.538025pt}{158.703156pt}}
\pgfusepath{stroke}
\pgfpathmoveto{\pgfpoint{128.447998pt}{164.880005pt}}
\pgflineto{\pgfpoint{119.546982pt}{164.880005pt}}
\pgfusepath{stroke}
\pgfpathmoveto{\pgfpoint{128.447998pt}{171.056854pt}}
\pgflineto{\pgfpoint{119.546982pt}{171.056854pt}}
\pgfusepath{stroke}
\pgfpathmoveto{\pgfpoint{128.447998pt}{177.233673pt}}
\pgflineto{\pgfpoint{119.547012pt}{177.233673pt}}
\pgfusepath{stroke}
\pgfpathmoveto{\pgfpoint{128.447998pt}{183.410522pt}}
\pgflineto{\pgfpoint{119.547012pt}{183.410522pt}}
\pgfusepath{stroke}
\pgfpathmoveto{\pgfpoint{128.447998pt}{189.587372pt}}
\pgflineto{\pgfpoint{119.547012pt}{189.587372pt}}
\pgfusepath{stroke}
\pgfpathmoveto{\pgfpoint{128.447998pt}{195.764206pt}}
\pgflineto{\pgfpoint{119.555962pt}{195.764206pt}}
\pgfusepath{stroke}
\pgfpathmoveto{\pgfpoint{128.447998pt}{201.941055pt}}
\pgflineto{\pgfpoint{119.556015pt}{201.941055pt}}
\pgfusepath{stroke}
\pgfpathmoveto{\pgfpoint{128.447998pt}{208.117905pt}}
\pgflineto{\pgfpoint{119.556053pt}{208.117905pt}}
\pgfusepath{stroke}
\pgfpathmoveto{\pgfpoint{128.447998pt}{214.294739pt}}
\pgflineto{\pgfpoint{119.556000pt}{214.294739pt}}
\pgfusepath{stroke}
\pgfpathmoveto{\pgfpoint{119.556000pt}{220.471588pt}}
\pgflineto{\pgfpoint{128.429886pt}{220.471588pt}}
\pgfusepath{stroke}
\pgfpathmoveto{\pgfpoint{119.565056pt}{226.648422pt}}
\pgflineto{\pgfpoint{128.420868pt}{226.648422pt}}
\pgfusepath{stroke}
\pgfpathmoveto{\pgfpoint{119.564995pt}{232.825272pt}}
\pgflineto{\pgfpoint{128.411850pt}{232.825272pt}}
\pgfusepath{stroke}
\pgfpathmoveto{\pgfpoint{119.565041pt}{239.002106pt}}
\pgflineto{\pgfpoint{128.411819pt}{239.002106pt}}
\pgfusepath{stroke}
\pgfpathmoveto{\pgfpoint{119.565041pt}{245.178955pt}}
\pgflineto{\pgfpoint{128.402802pt}{245.178955pt}}
\pgfusepath{stroke}
\pgfpathmoveto{\pgfpoint{119.565041pt}{251.355804pt}}
\pgflineto{\pgfpoint{128.393707pt}{251.355804pt}}
\pgfusepath{stroke}
\pgfpathmoveto{\pgfpoint{119.574036pt}{257.532623pt}}
\pgflineto{\pgfpoint{128.384689pt}{257.532623pt}}
\pgfusepath{stroke}
\pgfpathmoveto{\pgfpoint{119.573975pt}{263.709473pt}}
\pgflineto{\pgfpoint{128.375702pt}{263.709473pt}}
\pgfusepath{stroke}
\pgfpathmoveto{\pgfpoint{119.574089pt}{269.886322pt}}
\pgflineto{\pgfpoint{128.375626pt}{269.886322pt}}
\pgfusepath{stroke}
\pgfpathmoveto{\pgfpoint{119.574028pt}{276.063141pt}}
\pgflineto{\pgfpoint{128.366547pt}{276.063141pt}}
\pgfusepath{stroke}
\pgfpathmoveto{\pgfpoint{119.583015pt}{282.239990pt}}
\pgflineto{\pgfpoint{128.357590pt}{282.239990pt}}
\pgfusepath{stroke}
\pgfpathmoveto{\pgfpoint{119.583015pt}{288.416840pt}}
\pgflineto{\pgfpoint{128.348557pt}{288.416840pt}}
\pgfusepath{stroke}
\pgfpathmoveto{\pgfpoint{119.583015pt}{294.593689pt}}
\pgflineto{\pgfpoint{128.339523pt}{294.593689pt}}
\pgfusepath{stroke}
\pgfpathmoveto{\pgfpoint{119.583076pt}{300.770538pt}}
\pgflineto{\pgfpoint{128.339462pt}{300.770538pt}}
\pgfusepath{stroke}
\pgfpathmoveto{\pgfpoint{119.583076pt}{306.947388pt}}
\pgflineto{\pgfpoint{128.330429pt}{306.947388pt}}
\pgfusepath{stroke}
\pgfpathmoveto{\pgfpoint{119.591988pt}{313.124207pt}}
\pgflineto{\pgfpoint{128.321472pt}{313.124207pt}}
\pgfusepath{stroke}
\pgfpathmoveto{\pgfpoint{119.591988pt}{319.301056pt}}
\pgflineto{\pgfpoint{128.312439pt}{319.301056pt}}
\pgfusepath{stroke}
\pgfpathmoveto{\pgfpoint{119.592064pt}{325.477905pt}}
\pgflineto{\pgfpoint{128.312286pt}{325.477905pt}}
\pgfusepath{stroke}
\pgfpathmoveto{\pgfpoint{119.592064pt}{331.654724pt}}
\pgflineto{\pgfpoint{128.303253pt}{331.654724pt}}
\pgfusepath{stroke}
\pgfpathmoveto{\pgfpoint{119.592026pt}{337.831604pt}}
\pgflineto{\pgfpoint{128.289825pt}{337.831604pt}}
\pgfusepath{stroke}
\pgfpathmoveto{\pgfpoint{119.592026pt}{344.008423pt}}
\pgflineto{\pgfpoint{128.280731pt}{344.008423pt}}
\pgfusepath{stroke}
\pgfpathmoveto{\pgfpoint{119.596519pt}{350.185242pt}}
\pgflineto{\pgfpoint{128.276199pt}{350.185242pt}}
\pgfusepath{stroke}
\pgfpathmoveto{\pgfpoint{119.596519pt}{356.362122pt}}
\pgflineto{\pgfpoint{128.267151pt}{356.362122pt}}
\pgfusepath{stroke}
\pgfpathmoveto{\pgfpoint{119.596558pt}{362.538940pt}}
\pgflineto{\pgfpoint{128.262543pt}{362.538940pt}}
\pgfusepath{stroke}
\pgfpathmoveto{\pgfpoint{119.601067pt}{368.715820pt}}
\pgflineto{\pgfpoint{128.253632pt}{368.715820pt}}
\pgfusepath{stroke}
\pgfpathmoveto{\pgfpoint{119.601013pt}{374.892639pt}}
\pgflineto{\pgfpoint{128.244583pt}{374.892639pt}}
\pgfusepath{stroke}
\pgfpathmoveto{\pgfpoint{119.605560pt}{381.069458pt}}
\pgflineto{\pgfpoint{128.240005pt}{381.069458pt}}
\pgfusepath{stroke}
\pgfpathmoveto{\pgfpoint{119.605560pt}{387.246338pt}}
\pgflineto{\pgfpoint{128.231018pt}{387.246338pt}}
\pgfusepath{stroke}
\pgfpathmoveto{\pgfpoint{119.610054pt}{393.423157pt}}
\pgflineto{\pgfpoint{128.226471pt}{393.423157pt}}
\pgfusepath{stroke}
\pgfpathmoveto{\pgfpoint{119.610107pt}{399.600037pt}}
\pgflineto{\pgfpoint{128.217438pt}{399.600037pt}}
\pgfusepath{stroke}
\pgfpathmoveto{\pgfpoint{128.447998pt}{158.703156pt}}
\pgflineto{\pgfpoint{128.447998pt}{152.526306pt}}
\pgfusepath{stroke}
\pgfpathmoveto{\pgfpoint{128.447998pt}{152.526306pt}}
\pgflineto{\pgfpoint{128.447998pt}{146.349472pt}}
\pgfusepath{stroke}
\pgfpathmoveto{\pgfpoint{128.447998pt}{164.880005pt}}
\pgflineto{\pgfpoint{128.447998pt}{158.703156pt}}
\pgfusepath{stroke}
\pgfpathmoveto{\pgfpoint{128.447998pt}{171.056854pt}}
\pgflineto{\pgfpoint{128.447998pt}{164.880005pt}}
\pgfusepath{stroke}
\pgfpathmoveto{\pgfpoint{128.447998pt}{177.233673pt}}
\pgflineto{\pgfpoint{128.447998pt}{171.056854pt}}
\pgfusepath{stroke}
\pgfpathmoveto{\pgfpoint{128.447998pt}{183.410522pt}}
\pgflineto{\pgfpoint{128.447998pt}{177.233673pt}}
\pgfusepath{stroke}
\pgfpathmoveto{\pgfpoint{128.447998pt}{189.587372pt}}
\pgflineto{\pgfpoint{128.447998pt}{183.410522pt}}
\pgfusepath{stroke}
\pgfpathmoveto{\pgfpoint{128.447998pt}{195.764206pt}}
\pgflineto{\pgfpoint{128.447998pt}{189.587372pt}}
\pgfusepath{stroke}
\pgfpathmoveto{\pgfpoint{128.447998pt}{189.587372pt}}
\pgflineto{\pgfpoint{128.466080pt}{189.587372pt}}
\pgfusepath{stroke}
\pgfpathmoveto{\pgfpoint{128.447998pt}{152.526306pt}}
\pgflineto{\pgfpoint{128.466476pt}{152.526306pt}}
\pgfusepath{stroke}
\pgfpathmoveto{\pgfpoint{128.447998pt}{158.703156pt}}
\pgflineto{\pgfpoint{128.466476pt}{158.703156pt}}
\pgfusepath{stroke}
\pgfpathmoveto{\pgfpoint{128.447998pt}{164.880005pt}}
\pgflineto{\pgfpoint{128.466553pt}{164.880005pt}}
\pgfusepath{stroke}
\pgfpathmoveto{\pgfpoint{128.447998pt}{171.056854pt}}
\pgflineto{\pgfpoint{128.466614pt}{171.056854pt}}
\pgfusepath{stroke}
\pgfpathmoveto{\pgfpoint{128.447998pt}{177.233673pt}}
\pgflineto{\pgfpoint{128.475632pt}{177.233673pt}}
\pgfusepath{stroke}
\pgfpathmoveto{\pgfpoint{128.447998pt}{183.410522pt}}
\pgflineto{\pgfpoint{128.475677pt}{183.410522pt}}
\pgfusepath{stroke}
\pgfpathmoveto{\pgfpoint{128.447998pt}{208.117905pt}}
\pgflineto{\pgfpoint{128.447998pt}{201.941055pt}}
\pgfusepath{stroke}
\pgfpathmoveto{\pgfpoint{128.447998pt}{201.941055pt}}
\pgflineto{\pgfpoint{128.447998pt}{195.764206pt}}
\pgfusepath{stroke}
\pgfpathmoveto{\pgfpoint{128.447998pt}{214.294739pt}}
\pgflineto{\pgfpoint{128.447998pt}{208.117905pt}}
\pgfusepath{stroke}
\pgfpathmoveto{\pgfpoint{128.411850pt}{232.825272pt}}
\pgflineto{\pgfpoint{128.429932pt}{232.825272pt}}
\pgfusepath{stroke}
\pgfpathmoveto{\pgfpoint{128.447998pt}{239.002106pt}}
\pgflineto{\pgfpoint{128.411819pt}{239.002106pt}}
\pgfusepath{stroke}
\pgfpathmoveto{\pgfpoint{128.447998pt}{245.178955pt}}
\pgflineto{\pgfpoint{128.402802pt}{245.178955pt}}
\pgfusepath{stroke}
\pgfpathmoveto{\pgfpoint{128.447998pt}{251.355804pt}}
\pgflineto{\pgfpoint{128.393707pt}{251.355804pt}}
\pgfusepath{stroke}
\pgfpathmoveto{\pgfpoint{128.447998pt}{257.532623pt}}
\pgflineto{\pgfpoint{128.384689pt}{257.532623pt}}
\pgfusepath{stroke}
\pgfpathmoveto{\pgfpoint{128.447998pt}{263.709473pt}}
\pgflineto{\pgfpoint{128.375702pt}{263.709473pt}}
\pgfusepath{stroke}
\pgfpathmoveto{\pgfpoint{128.447998pt}{269.886322pt}}
\pgflineto{\pgfpoint{128.375626pt}{269.886322pt}}
\pgfusepath{stroke}
\pgfpathmoveto{\pgfpoint{128.447998pt}{276.063141pt}}
\pgflineto{\pgfpoint{128.366547pt}{276.063141pt}}
\pgfusepath{stroke}
\pgfpathmoveto{\pgfpoint{128.447998pt}{282.239990pt}}
\pgflineto{\pgfpoint{128.357590pt}{282.239990pt}}
\pgfusepath{stroke}
\pgfpathmoveto{\pgfpoint{128.447998pt}{288.416840pt}}
\pgflineto{\pgfpoint{128.348557pt}{288.416840pt}}
\pgfusepath{stroke}
\pgfpathmoveto{\pgfpoint{128.447998pt}{294.593689pt}}
\pgflineto{\pgfpoint{128.339523pt}{294.593689pt}}
\pgfusepath{stroke}
\pgfpathmoveto{\pgfpoint{128.447998pt}{300.770538pt}}
\pgflineto{\pgfpoint{128.339462pt}{300.770538pt}}
\pgfusepath{stroke}
\pgfpathmoveto{\pgfpoint{128.447998pt}{306.947388pt}}
\pgflineto{\pgfpoint{128.330429pt}{306.947388pt}}
\pgfusepath{stroke}
\pgfpathmoveto{\pgfpoint{128.447998pt}{313.124207pt}}
\pgflineto{\pgfpoint{128.321472pt}{313.124207pt}}
\pgfusepath{stroke}
\pgfpathmoveto{\pgfpoint{128.447998pt}{319.301056pt}}
\pgflineto{\pgfpoint{128.312439pt}{319.301056pt}}
\pgfusepath{stroke}
\pgfpathmoveto{\pgfpoint{128.447998pt}{325.477905pt}}
\pgflineto{\pgfpoint{128.312286pt}{325.477905pt}}
\pgfusepath{stroke}
\pgfpathmoveto{\pgfpoint{128.447998pt}{331.654724pt}}
\pgflineto{\pgfpoint{128.303253pt}{331.654724pt}}
\pgfusepath{stroke}
\pgfpathmoveto{\pgfpoint{128.447998pt}{337.831604pt}}
\pgflineto{\pgfpoint{128.289825pt}{337.831604pt}}
\pgfusepath{stroke}
\pgfpathmoveto{\pgfpoint{128.447998pt}{344.008423pt}}
\pgflineto{\pgfpoint{128.280731pt}{344.008423pt}}
\pgfusepath{stroke}
\pgfpathmoveto{\pgfpoint{128.447998pt}{350.185242pt}}
\pgflineto{\pgfpoint{128.276199pt}{350.185242pt}}
\pgfusepath{stroke}
\pgfpathmoveto{\pgfpoint{128.447998pt}{356.362122pt}}
\pgflineto{\pgfpoint{128.267151pt}{356.362122pt}}
\pgfusepath{stroke}
\pgfpathmoveto{\pgfpoint{128.447998pt}{362.538940pt}}
\pgflineto{\pgfpoint{128.262543pt}{362.538940pt}}
\pgfusepath{stroke}
\pgfpathmoveto{\pgfpoint{128.447998pt}{368.715820pt}}
\pgflineto{\pgfpoint{128.253632pt}{368.715820pt}}
\pgfusepath{stroke}
\pgfpathmoveto{\pgfpoint{128.447998pt}{374.892639pt}}
\pgflineto{\pgfpoint{128.244583pt}{374.892639pt}}
\pgfusepath{stroke}
\pgfpathmoveto{\pgfpoint{128.447998pt}{381.069458pt}}
\pgflineto{\pgfpoint{128.240005pt}{381.069458pt}}
\pgfusepath{stroke}
\pgfpathmoveto{\pgfpoint{128.447998pt}{387.246338pt}}
\pgflineto{\pgfpoint{128.231018pt}{387.246338pt}}
\pgfusepath{stroke}
\pgfpathmoveto{\pgfpoint{128.447998pt}{393.423157pt}}
\pgflineto{\pgfpoint{128.226471pt}{393.423157pt}}
\pgfusepath{stroke}
\pgfpathmoveto{\pgfpoint{128.447998pt}{399.600037pt}}
\pgflineto{\pgfpoint{128.217438pt}{399.600037pt}}
\pgfusepath{stroke}
\pgfpathmoveto{\pgfpoint{128.447998pt}{331.654724pt}}
\pgflineto{\pgfpoint{128.447998pt}{325.477905pt}}
\pgfusepath{stroke}
\pgfpathmoveto{\pgfpoint{128.447998pt}{325.477905pt}}
\pgflineto{\pgfpoint{128.447998pt}{319.301056pt}}
\pgfusepath{stroke}
\pgfpathmoveto{\pgfpoint{128.447998pt}{337.831604pt}}
\pgflineto{\pgfpoint{128.447998pt}{331.654724pt}}
\pgfusepath{stroke}
\pgfpathmoveto{\pgfpoint{128.447998pt}{344.008423pt}}
\pgflineto{\pgfpoint{128.447998pt}{337.831604pt}}
\pgfusepath{stroke}
\pgfpathmoveto{\pgfpoint{128.447998pt}{350.185242pt}}
\pgflineto{\pgfpoint{128.447998pt}{344.008423pt}}
\pgfusepath{stroke}
\pgfpathmoveto{\pgfpoint{128.447998pt}{356.362122pt}}
\pgflineto{\pgfpoint{128.447998pt}{350.185242pt}}
\pgfusepath{stroke}
\pgfpathmoveto{\pgfpoint{128.447998pt}{325.477905pt}}
\pgflineto{\pgfpoint{128.466187pt}{325.477905pt}}
\pgfusepath{stroke}
\pgfpathmoveto{\pgfpoint{128.447998pt}{331.654724pt}}
\pgflineto{\pgfpoint{128.466125pt}{331.654724pt}}
\pgfusepath{stroke}
\pgfpathmoveto{\pgfpoint{128.447998pt}{337.831604pt}}
\pgflineto{\pgfpoint{128.461700pt}{337.831604pt}}
\pgfusepath{stroke}
\pgfpathmoveto{\pgfpoint{128.447998pt}{344.008423pt}}
\pgflineto{\pgfpoint{128.461700pt}{344.008423pt}}
\pgfusepath{stroke}
\pgfpathmoveto{\pgfpoint{128.447998pt}{350.185242pt}}
\pgflineto{\pgfpoint{128.466202pt}{350.185242pt}}
\pgfusepath{stroke}
\pgfpathmoveto{\pgfpoint{128.447998pt}{381.069458pt}}
\pgflineto{\pgfpoint{128.447998pt}{374.892639pt}}
\pgfusepath{stroke}
\pgfpathmoveto{\pgfpoint{128.447998pt}{362.538940pt}}
\pgflineto{\pgfpoint{128.447998pt}{356.362122pt}}
\pgfusepath{stroke}
\pgfpathmoveto{\pgfpoint{128.447998pt}{368.715820pt}}
\pgflineto{\pgfpoint{128.447998pt}{362.538940pt}}
\pgfusepath{stroke}
\pgfpathmoveto{\pgfpoint{128.447998pt}{374.892639pt}}
\pgflineto{\pgfpoint{128.447998pt}{368.715820pt}}
\pgfusepath{stroke}
\pgfpathmoveto{\pgfpoint{128.447998pt}{387.246338pt}}
\pgflineto{\pgfpoint{128.447998pt}{381.069458pt}}
\pgfusepath{stroke}
\pgfpathmoveto{\pgfpoint{128.447998pt}{393.423157pt}}
\pgflineto{\pgfpoint{128.447998pt}{387.246338pt}}
\pgfusepath{stroke}
\pgfpathmoveto{\pgfpoint{128.447998pt}{399.600037pt}}
\pgflineto{\pgfpoint{128.447998pt}{393.423157pt}}
\pgfusepath{stroke}
\pgfpathmoveto{\pgfpoint{128.447998pt}{356.362122pt}}
\pgflineto{\pgfpoint{128.466248pt}{356.362122pt}}
\pgfusepath{stroke}
\pgfpathmoveto{\pgfpoint{128.447998pt}{362.538940pt}}
\pgflineto{\pgfpoint{128.470749pt}{362.538940pt}}
\pgfusepath{stroke}
\pgfpathmoveto{\pgfpoint{128.447998pt}{368.715820pt}}
\pgflineto{\pgfpoint{128.470795pt}{368.715820pt}}
\pgfusepath{stroke}
\pgfpathmoveto{\pgfpoint{128.447998pt}{374.892639pt}}
\pgflineto{\pgfpoint{128.470810pt}{374.892639pt}}
\pgfusepath{stroke}
\pgfpathmoveto{\pgfpoint{128.447998pt}{381.069458pt}}
\pgflineto{\pgfpoint{128.475296pt}{381.069458pt}}
\pgfusepath{stroke}
\pgfpathmoveto{\pgfpoint{128.447998pt}{387.246338pt}}
\pgflineto{\pgfpoint{128.475357pt}{387.246338pt}}
\pgfusepath{stroke}
\pgfpathmoveto{\pgfpoint{128.447998pt}{393.423157pt}}
\pgflineto{\pgfpoint{128.479843pt}{393.423157pt}}
\pgfusepath{stroke}
\pgfpathmoveto{\pgfpoint{128.447998pt}{399.600037pt}}
\pgflineto{\pgfpoint{128.479904pt}{399.600037pt}}
\pgfusepath{stroke}
\pgfpathmoveto{\pgfpoint{128.447998pt}{276.063141pt}}
\pgflineto{\pgfpoint{128.447998pt}{269.886322pt}}
\pgfusepath{stroke}
\pgfpathmoveto{\pgfpoint{128.447998pt}{245.178955pt}}
\pgflineto{\pgfpoint{128.447998pt}{239.002106pt}}
\pgfusepath{stroke}
\pgfpathmoveto{\pgfpoint{128.447998pt}{251.355804pt}}
\pgflineto{\pgfpoint{128.447998pt}{245.178955pt}}
\pgfusepath{stroke}
\pgfpathmoveto{\pgfpoint{128.447998pt}{257.532623pt}}
\pgflineto{\pgfpoint{128.447998pt}{251.355804pt}}
\pgfusepath{stroke}
\pgfpathmoveto{\pgfpoint{128.447998pt}{263.709473pt}}
\pgflineto{\pgfpoint{128.447998pt}{257.532623pt}}
\pgfusepath{stroke}
\pgfpathmoveto{\pgfpoint{128.447998pt}{269.886322pt}}
\pgflineto{\pgfpoint{128.447998pt}{263.709473pt}}
\pgfusepath{stroke}
\pgfpathmoveto{\pgfpoint{128.447998pt}{282.239990pt}}
\pgflineto{\pgfpoint{128.447998pt}{276.063141pt}}
\pgfusepath{stroke}
\pgfpathmoveto{\pgfpoint{128.447998pt}{288.416840pt}}
\pgflineto{\pgfpoint{128.447998pt}{282.239990pt}}
\pgfusepath{stroke}
\pgfpathmoveto{\pgfpoint{128.447998pt}{294.593689pt}}
\pgflineto{\pgfpoint{128.447998pt}{288.416840pt}}
\pgfusepath{stroke}
\pgfpathmoveto{\pgfpoint{128.447998pt}{300.770538pt}}
\pgflineto{\pgfpoint{128.447998pt}{294.593689pt}}
\pgfusepath{stroke}
\pgfpathmoveto{\pgfpoint{128.447998pt}{306.947388pt}}
\pgflineto{\pgfpoint{128.447998pt}{300.770538pt}}
\pgfusepath{stroke}
\pgfpathmoveto{\pgfpoint{128.447998pt}{313.124207pt}}
\pgflineto{\pgfpoint{128.447998pt}{306.947388pt}}
\pgfusepath{stroke}
\pgfpathmoveto{\pgfpoint{128.447998pt}{319.301056pt}}
\pgflineto{\pgfpoint{128.447998pt}{313.124207pt}}
\pgfusepath{stroke}
\pgfpathmoveto{\pgfpoint{128.475677pt}{189.587372pt}}
\pgflineto{\pgfpoint{128.466080pt}{189.587372pt}}
\pgfusepath{stroke}
\pgfpathmoveto{\pgfpoint{128.447998pt}{195.764206pt}}
\pgflineto{\pgfpoint{128.475830pt}{195.764206pt}}
\pgfusepath{stroke}
\pgfpathmoveto{\pgfpoint{128.447998pt}{201.941055pt}}
\pgflineto{\pgfpoint{128.475830pt}{201.941055pt}}
\pgfusepath{stroke}
\pgfpathmoveto{\pgfpoint{128.447998pt}{208.117905pt}}
\pgflineto{\pgfpoint{128.484894pt}{208.117905pt}}
\pgfusepath{stroke}
\pgfpathmoveto{\pgfpoint{128.447998pt}{214.294739pt}}
\pgflineto{\pgfpoint{128.484894pt}{214.294739pt}}
\pgfusepath{stroke}
\pgfpathmoveto{\pgfpoint{128.447998pt}{220.471588pt}}
\pgflineto{\pgfpoint{128.447998pt}{214.294739pt}}
\pgfusepath{stroke}
\pgfpathmoveto{\pgfpoint{128.447998pt}{220.471588pt}}
\pgflineto{\pgfpoint{128.429886pt}{220.471588pt}}
\pgfusepath{stroke}
\pgfpathmoveto{\pgfpoint{128.447998pt}{220.471588pt}}
\pgflineto{\pgfpoint{128.485046pt}{220.471588pt}}
\pgfusepath{stroke}
\pgfpathmoveto{\pgfpoint{128.447998pt}{226.648422pt}}
\pgflineto{\pgfpoint{128.420868pt}{226.648422pt}}
\pgfusepath{stroke}
\pgfpathmoveto{\pgfpoint{128.447998pt}{232.825272pt}}
\pgflineto{\pgfpoint{128.429932pt}{232.825272pt}}
\pgfusepath{stroke}
\pgfpathmoveto{\pgfpoint{128.447998pt}{226.648422pt}}
\pgflineto{\pgfpoint{128.447998pt}{220.471588pt}}
\pgfusepath{stroke}
\pgfpathmoveto{\pgfpoint{128.447998pt}{239.002106pt}}
\pgflineto{\pgfpoint{128.447998pt}{232.825272pt}}
\pgfusepath{stroke}
\pgfpathmoveto{\pgfpoint{128.447998pt}{232.825272pt}}
\pgflineto{\pgfpoint{128.447998pt}{226.648422pt}}
\pgfusepath{stroke}
\pgfpathmoveto{\pgfpoint{128.447998pt}{226.648422pt}}
\pgflineto{\pgfpoint{128.485046pt}{226.648422pt}}
\pgfusepath{stroke}
\pgfpathmoveto{\pgfpoint{128.447998pt}{232.825272pt}}
\pgflineto{\pgfpoint{128.485107pt}{232.825272pt}}
\pgfusepath{stroke}
\pgfpathmoveto{\pgfpoint{128.447998pt}{239.002106pt}}
\pgflineto{\pgfpoint{128.494095pt}{239.002106pt}}
\pgfusepath{stroke}
\pgfpathmoveto{\pgfpoint{128.447998pt}{245.178955pt}}
\pgflineto{\pgfpoint{128.494156pt}{245.178955pt}}
\pgfusepath{stroke}
\pgfpathmoveto{\pgfpoint{128.447998pt}{251.355804pt}}
\pgflineto{\pgfpoint{128.494263pt}{251.355804pt}}
\pgfusepath{stroke}
\pgfpathmoveto{\pgfpoint{128.447998pt}{257.532623pt}}
\pgflineto{\pgfpoint{128.494263pt}{257.532623pt}}
\pgfusepath{stroke}
\pgfpathmoveto{\pgfpoint{128.447998pt}{263.709473pt}}
\pgflineto{\pgfpoint{128.494324pt}{263.709473pt}}
\pgfusepath{stroke}
\pgfpathmoveto{\pgfpoint{128.447998pt}{269.886322pt}}
\pgflineto{\pgfpoint{128.503311pt}{269.886322pt}}
\pgfusepath{stroke}
\pgfpathmoveto{\pgfpoint{128.447998pt}{276.063141pt}}
\pgflineto{\pgfpoint{128.503372pt}{276.063141pt}}
\pgfusepath{stroke}
\pgfpathmoveto{\pgfpoint{128.447998pt}{282.239990pt}}
\pgflineto{\pgfpoint{128.503479pt}{282.239990pt}}
\pgfusepath{stroke}
\pgfpathmoveto{\pgfpoint{128.447998pt}{288.416840pt}}
\pgflineto{\pgfpoint{128.503479pt}{288.416840pt}}
\pgfusepath{stroke}
\pgfpathmoveto{\pgfpoint{128.447998pt}{294.593689pt}}
\pgflineto{\pgfpoint{128.503601pt}{294.593689pt}}
\pgfusepath{stroke}
\pgfpathmoveto{\pgfpoint{128.447998pt}{300.770538pt}}
\pgflineto{\pgfpoint{128.512573pt}{300.770538pt}}
\pgfusepath{stroke}
\pgfpathmoveto{\pgfpoint{128.447998pt}{306.947388pt}}
\pgflineto{\pgfpoint{128.512634pt}{306.947388pt}}
\pgfusepath{stroke}
\pgfpathmoveto{\pgfpoint{128.447998pt}{313.124207pt}}
\pgflineto{\pgfpoint{128.512695pt}{313.124207pt}}
\pgfusepath{stroke}
\pgfpathmoveto{\pgfpoint{128.447998pt}{319.301056pt}}
\pgflineto{\pgfpoint{128.512756pt}{319.301056pt}}
\pgfusepath{stroke}
\pgfpathmoveto{\pgfpoint{128.521790pt}{325.477905pt}}
\pgflineto{\pgfpoint{128.466187pt}{325.477905pt}}
\pgfusepath{stroke}
\pgfpathmoveto{\pgfpoint{128.521790pt}{331.654724pt}}
\pgflineto{\pgfpoint{128.466125pt}{331.654724pt}}
\pgfusepath{stroke}
\pgfpathmoveto{\pgfpoint{128.517395pt}{337.831604pt}}
\pgflineto{\pgfpoint{128.461700pt}{337.831604pt}}
\pgfusepath{stroke}
\pgfpathmoveto{\pgfpoint{128.517395pt}{344.008423pt}}
\pgflineto{\pgfpoint{128.461700pt}{344.008423pt}}
\pgfusepath{stroke}
\pgfpathmoveto{\pgfpoint{128.521942pt}{350.185242pt}}
\pgflineto{\pgfpoint{128.466202pt}{350.185242pt}}
\pgfusepath{stroke}
\pgfpathmoveto{\pgfpoint{128.521988pt}{356.362122pt}}
\pgflineto{\pgfpoint{128.466248pt}{356.362122pt}}
\pgfusepath{stroke}
\pgfpathmoveto{\pgfpoint{128.526550pt}{362.538940pt}}
\pgflineto{\pgfpoint{128.470749pt}{362.538940pt}}
\pgfusepath{stroke}
\pgfpathmoveto{\pgfpoint{128.526657pt}{368.715820pt}}
\pgflineto{\pgfpoint{128.470795pt}{368.715820pt}}
\pgfusepath{stroke}
\pgfpathmoveto{\pgfpoint{128.526642pt}{374.892639pt}}
\pgflineto{\pgfpoint{128.470810pt}{374.892639pt}}
\pgfusepath{stroke}
\pgfpathmoveto{\pgfpoint{128.531204pt}{381.069458pt}}
\pgflineto{\pgfpoint{128.475296pt}{381.069458pt}}
\pgfusepath{stroke}
\pgfpathmoveto{\pgfpoint{128.531265pt}{387.246338pt}}
\pgflineto{\pgfpoint{128.475357pt}{387.246338pt}}
\pgfusepath{stroke}
\pgfpathmoveto{\pgfpoint{128.535812pt}{393.423157pt}}
\pgflineto{\pgfpoint{128.479843pt}{393.423157pt}}
\pgfusepath{stroke}
\pgfpathmoveto{\pgfpoint{128.535873pt}{399.600037pt}}
\pgflineto{\pgfpoint{128.479904pt}{399.600037pt}}
\pgfusepath{stroke}
\pgfpathmoveto{\pgfpoint{128.447998pt}{96.934731pt}}
\pgflineto{\pgfpoint{128.447998pt}{90.757896pt}}
\pgfusepath{stroke}
\pgfpathmoveto{\pgfpoint{128.447998pt}{66.050522pt}}
\pgflineto{\pgfpoint{128.447998pt}{59.873672pt}}
\pgfusepath{stroke}
\pgfpathmoveto{\pgfpoint{128.447998pt}{72.227356pt}}
\pgflineto{\pgfpoint{128.447998pt}{66.050522pt}}
\pgfusepath{stroke}
\pgfpathmoveto{\pgfpoint{128.447998pt}{78.404205pt}}
\pgflineto{\pgfpoint{128.447998pt}{72.227356pt}}
\pgfusepath{stroke}
\pgfpathmoveto{\pgfpoint{128.447998pt}{84.581039pt}}
\pgflineto{\pgfpoint{128.447998pt}{78.404205pt}}
\pgfusepath{stroke}
\pgfpathmoveto{\pgfpoint{128.447998pt}{90.757896pt}}
\pgflineto{\pgfpoint{128.447998pt}{84.581039pt}}
\pgfusepath{stroke}
\pgfpathmoveto{\pgfpoint{128.447998pt}{103.111580pt}}
\pgflineto{\pgfpoint{128.447998pt}{96.934731pt}}
\pgfusepath{stroke}
\pgfpathmoveto{\pgfpoint{128.447998pt}{109.288422pt}}
\pgflineto{\pgfpoint{128.447998pt}{103.111580pt}}
\pgfusepath{stroke}
\pgfpathmoveto{\pgfpoint{128.447998pt}{115.465263pt}}
\pgflineto{\pgfpoint{128.447998pt}{109.288422pt}}
\pgfusepath{stroke}
\pgfpathmoveto{\pgfpoint{128.447998pt}{121.642097pt}}
\pgflineto{\pgfpoint{128.447998pt}{115.465263pt}}
\pgfusepath{stroke}
\pgfpathmoveto{\pgfpoint{128.447998pt}{127.818947pt}}
\pgflineto{\pgfpoint{128.447998pt}{121.642097pt}}
\pgfusepath{stroke}
\pgfpathmoveto{\pgfpoint{128.447998pt}{133.995789pt}}
\pgflineto{\pgfpoint{128.447998pt}{127.818947pt}}
\pgfusepath{stroke}
\pgfpathmoveto{\pgfpoint{128.447998pt}{140.172638pt}}
\pgflineto{\pgfpoint{128.447998pt}{133.995789pt}}
\pgfusepath{stroke}
\pgfpathmoveto{\pgfpoint{128.447998pt}{146.349472pt}}
\pgflineto{\pgfpoint{128.447998pt}{140.172638pt}}
\pgfusepath{stroke}
\pgfpathmoveto{\pgfpoint{128.447998pt}{47.519989pt}}
\pgflineto{\pgfpoint{128.429581pt}{47.519989pt}}
\pgfusepath{stroke}
\pgfpathmoveto{\pgfpoint{128.447998pt}{53.696838pt}}
\pgflineto{\pgfpoint{128.429703pt}{53.696838pt}}
\pgfusepath{stroke}
\pgfpathmoveto{\pgfpoint{128.447998pt}{59.873672pt}}
\pgflineto{\pgfpoint{128.447998pt}{53.696838pt}}
\pgfusepath{stroke}
\pgfpathmoveto{\pgfpoint{128.447998pt}{53.696838pt}}
\pgflineto{\pgfpoint{128.447998pt}{47.519989pt}}
\pgfusepath{stroke}
\pgfpathmoveto{\pgfpoint{128.447998pt}{47.519989pt}}
\pgflineto{\pgfpoint{137.358093pt}{47.519989pt}}
\pgfusepath{stroke}
\pgfpathmoveto{\pgfpoint{128.447998pt}{53.696838pt}}
\pgflineto{\pgfpoint{137.358047pt}{53.696838pt}}
\pgfusepath{stroke}
\pgfpathmoveto{\pgfpoint{128.447998pt}{59.873672pt}}
\pgflineto{\pgfpoint{137.358093pt}{59.873672pt}}
\pgfusepath{stroke}
\pgfpathmoveto{\pgfpoint{128.447998pt}{66.050522pt}}
\pgflineto{\pgfpoint{137.358093pt}{66.050522pt}}
\pgfusepath{stroke}
\pgfpathmoveto{\pgfpoint{137.376007pt}{72.227356pt}}
\pgflineto{\pgfpoint{128.447998pt}{72.227356pt}}
\pgfusepath{stroke}
\pgfpathmoveto{\pgfpoint{137.376007pt}{78.404205pt}}
\pgflineto{\pgfpoint{128.447998pt}{78.404205pt}}
\pgfusepath{stroke}
\pgfpathmoveto{\pgfpoint{137.376007pt}{84.581039pt}}
\pgflineto{\pgfpoint{128.447998pt}{84.581039pt}}
\pgfusepath{stroke}
\pgfpathmoveto{\pgfpoint{137.376007pt}{90.757896pt}}
\pgflineto{\pgfpoint{128.447998pt}{90.757896pt}}
\pgfusepath{stroke}
\pgfpathmoveto{\pgfpoint{137.376007pt}{96.934731pt}}
\pgflineto{\pgfpoint{128.447998pt}{96.934731pt}}
\pgfusepath{stroke}
\pgfpathmoveto{\pgfpoint{137.376007pt}{103.111580pt}}
\pgflineto{\pgfpoint{128.447998pt}{103.111580pt}}
\pgfusepath{stroke}
\pgfpathmoveto{\pgfpoint{137.376007pt}{109.288422pt}}
\pgflineto{\pgfpoint{128.447998pt}{109.288422pt}}
\pgfusepath{stroke}
\pgfpathmoveto{\pgfpoint{137.376007pt}{115.465263pt}}
\pgflineto{\pgfpoint{128.447998pt}{115.465263pt}}
\pgfusepath{stroke}
\pgfpathmoveto{\pgfpoint{137.376007pt}{121.642097pt}}
\pgflineto{\pgfpoint{128.447998pt}{121.642097pt}}
\pgfusepath{stroke}
\pgfpathmoveto{\pgfpoint{137.376007pt}{127.818947pt}}
\pgflineto{\pgfpoint{128.447998pt}{127.818947pt}}
\pgfusepath{stroke}
\pgfpathmoveto{\pgfpoint{137.376007pt}{133.995789pt}}
\pgflineto{\pgfpoint{128.447998pt}{133.995789pt}}
\pgfusepath{stroke}
\pgfpathmoveto{\pgfpoint{137.376007pt}{140.172638pt}}
\pgflineto{\pgfpoint{128.447998pt}{140.172638pt}}
\pgfusepath{stroke}
\pgfpathmoveto{\pgfpoint{137.376007pt}{146.349472pt}}
\pgflineto{\pgfpoint{128.447998pt}{146.349472pt}}
\pgfusepath{stroke}
\pgfpathmoveto{\pgfpoint{137.376007pt}{152.526306pt}}
\pgflineto{\pgfpoint{128.466476pt}{152.526306pt}}
\pgfusepath{stroke}
\pgfpathmoveto{\pgfpoint{137.376007pt}{158.703156pt}}
\pgflineto{\pgfpoint{128.466476pt}{158.703156pt}}
\pgfusepath{stroke}
\pgfpathmoveto{\pgfpoint{137.376007pt}{164.880005pt}}
\pgflineto{\pgfpoint{128.466553pt}{164.880005pt}}
\pgfusepath{stroke}
\pgfpathmoveto{\pgfpoint{137.376007pt}{171.056854pt}}
\pgflineto{\pgfpoint{128.466614pt}{171.056854pt}}
\pgfusepath{stroke}
\pgfpathmoveto{\pgfpoint{137.376007pt}{177.233673pt}}
\pgflineto{\pgfpoint{128.475632pt}{177.233673pt}}
\pgfusepath{stroke}
\pgfpathmoveto{\pgfpoint{137.376007pt}{183.410522pt}}
\pgflineto{\pgfpoint{128.475677pt}{183.410522pt}}
\pgfusepath{stroke}
\pgfpathmoveto{\pgfpoint{137.376007pt}{189.587372pt}}
\pgflineto{\pgfpoint{128.475677pt}{189.587372pt}}
\pgfusepath{stroke}
\pgfpathmoveto{\pgfpoint{137.376007pt}{195.764206pt}}
\pgflineto{\pgfpoint{128.475830pt}{195.764206pt}}
\pgfusepath{stroke}
\pgfpathmoveto{\pgfpoint{137.376007pt}{201.941055pt}}
\pgflineto{\pgfpoint{128.475830pt}{201.941055pt}}
\pgfusepath{stroke}
\pgfpathmoveto{\pgfpoint{137.376007pt}{208.117905pt}}
\pgflineto{\pgfpoint{128.484894pt}{208.117905pt}}
\pgfusepath{stroke}
\pgfpathmoveto{\pgfpoint{137.376007pt}{214.294739pt}}
\pgflineto{\pgfpoint{128.484894pt}{214.294739pt}}
\pgfusepath{stroke}
\pgfpathmoveto{\pgfpoint{137.376007pt}{220.471588pt}}
\pgflineto{\pgfpoint{128.485046pt}{220.471588pt}}
\pgfusepath{stroke}
\pgfpathmoveto{\pgfpoint{137.376007pt}{226.648422pt}}
\pgflineto{\pgfpoint{128.485046pt}{226.648422pt}}
\pgfusepath{stroke}
\pgfpathmoveto{\pgfpoint{128.485107pt}{232.825272pt}}
\pgflineto{\pgfpoint{137.357986pt}{232.825272pt}}
\pgfusepath{stroke}
\pgfpathmoveto{\pgfpoint{128.494095pt}{239.002106pt}}
\pgflineto{\pgfpoint{137.348938pt}{239.002106pt}}
\pgfusepath{stroke}
\pgfpathmoveto{\pgfpoint{128.494156pt}{245.178955pt}}
\pgflineto{\pgfpoint{137.339890pt}{245.178955pt}}
\pgfusepath{stroke}
\pgfpathmoveto{\pgfpoint{128.494263pt}{251.355804pt}}
\pgflineto{\pgfpoint{137.339874pt}{251.355804pt}}
\pgfusepath{stroke}
\pgfpathmoveto{\pgfpoint{128.494263pt}{257.532623pt}}
\pgflineto{\pgfpoint{137.330811pt}{257.532623pt}}
\pgfusepath{stroke}
\pgfpathmoveto{\pgfpoint{128.494324pt}{263.709473pt}}
\pgflineto{\pgfpoint{137.321777pt}{263.709473pt}}
\pgfusepath{stroke}
\pgfpathmoveto{\pgfpoint{128.503311pt}{269.886322pt}}
\pgflineto{\pgfpoint{137.312805pt}{269.886322pt}}
\pgfusepath{stroke}
\pgfpathmoveto{\pgfpoint{128.503372pt}{276.063141pt}}
\pgflineto{\pgfpoint{137.303757pt}{276.063141pt}}
\pgfusepath{stroke}
\pgfpathmoveto{\pgfpoint{128.503479pt}{282.239990pt}}
\pgflineto{\pgfpoint{137.303680pt}{282.239990pt}}
\pgfusepath{stroke}
\pgfpathmoveto{\pgfpoint{128.503479pt}{288.416840pt}}
\pgflineto{\pgfpoint{137.294601pt}{288.416840pt}}
\pgfusepath{stroke}
\pgfpathmoveto{\pgfpoint{128.503601pt}{294.593689pt}}
\pgflineto{\pgfpoint{137.285599pt}{294.593689pt}}
\pgfusepath{stroke}
\pgfpathmoveto{\pgfpoint{128.512573pt}{300.770538pt}}
\pgflineto{\pgfpoint{137.276627pt}{300.770538pt}}
\pgfusepath{stroke}
\pgfpathmoveto{\pgfpoint{128.512634pt}{306.947388pt}}
\pgflineto{\pgfpoint{137.267609pt}{306.947388pt}}
\pgfusepath{stroke}
\pgfpathmoveto{\pgfpoint{128.512695pt}{313.124207pt}}
\pgflineto{\pgfpoint{137.267487pt}{313.124207pt}}
\pgfusepath{stroke}
\pgfpathmoveto{\pgfpoint{128.512756pt}{319.301056pt}}
\pgflineto{\pgfpoint{137.258438pt}{319.301056pt}}
\pgfusepath{stroke}
\pgfpathmoveto{\pgfpoint{128.521790pt}{325.477905pt}}
\pgflineto{\pgfpoint{137.249527pt}{325.477905pt}}
\pgfusepath{stroke}
\pgfpathmoveto{\pgfpoint{128.521790pt}{331.654724pt}}
\pgflineto{\pgfpoint{137.240448pt}{331.654724pt}}
\pgfusepath{stroke}
\pgfpathmoveto{\pgfpoint{128.517395pt}{337.831604pt}}
\pgflineto{\pgfpoint{137.231430pt}{337.831604pt}}
\pgfusepath{stroke}
\pgfpathmoveto{\pgfpoint{128.517395pt}{344.008423pt}}
\pgflineto{\pgfpoint{137.222336pt}{344.008423pt}}
\pgfusepath{stroke}
\pgfpathmoveto{\pgfpoint{128.521942pt}{350.185242pt}}
\pgflineto{\pgfpoint{137.217789pt}{350.185242pt}}
\pgfusepath{stroke}
\pgfpathmoveto{\pgfpoint{128.521988pt}{356.362122pt}}
\pgflineto{\pgfpoint{137.208755pt}{356.362122pt}}
\pgfusepath{stroke}
\pgfpathmoveto{\pgfpoint{128.526550pt}{362.538940pt}}
\pgflineto{\pgfpoint{137.204254pt}{362.538940pt}}
\pgfusepath{stroke}
\pgfpathmoveto{\pgfpoint{128.526657pt}{368.715820pt}}
\pgflineto{\pgfpoint{137.195282pt}{368.715820pt}}
\pgfusepath{stroke}
\pgfpathmoveto{\pgfpoint{128.526642pt}{374.892639pt}}
\pgflineto{\pgfpoint{137.190613pt}{374.892639pt}}
\pgfusepath{stroke}
\pgfpathmoveto{\pgfpoint{128.531204pt}{381.069458pt}}
\pgflineto{\pgfpoint{137.181641pt}{381.069458pt}}
\pgfusepath{stroke}
\pgfpathmoveto{\pgfpoint{128.531265pt}{387.246338pt}}
\pgflineto{\pgfpoint{137.172668pt}{387.246338pt}}
\pgfusepath{stroke}
\pgfpathmoveto{\pgfpoint{128.535812pt}{393.423157pt}}
\pgflineto{\pgfpoint{137.168060pt}{393.423157pt}}
\pgfusepath{stroke}
\pgfpathmoveto{\pgfpoint{128.535873pt}{399.600037pt}}
\pgflineto{\pgfpoint{137.159088pt}{399.600037pt}}
\pgfusepath{stroke}
\pgfpathmoveto{\pgfpoint{137.376007pt}{171.056854pt}}
\pgflineto{\pgfpoint{137.376007pt}{164.880005pt}}
\pgfusepath{stroke}
\pgfpathmoveto{\pgfpoint{137.376007pt}{164.880005pt}}
\pgflineto{\pgfpoint{137.376007pt}{158.703156pt}}
\pgfusepath{stroke}
\pgfpathmoveto{\pgfpoint{137.376007pt}{177.233673pt}}
\pgflineto{\pgfpoint{137.376007pt}{171.056854pt}}
\pgfusepath{stroke}
\pgfpathmoveto{\pgfpoint{137.376007pt}{183.410522pt}}
\pgflineto{\pgfpoint{137.376007pt}{177.233673pt}}
\pgfusepath{stroke}
\pgfpathmoveto{\pgfpoint{137.376007pt}{189.587372pt}}
\pgflineto{\pgfpoint{137.376007pt}{183.410522pt}}
\pgfusepath{stroke}
\pgfpathmoveto{\pgfpoint{137.376007pt}{195.764206pt}}
\pgflineto{\pgfpoint{137.376007pt}{189.587372pt}}
\pgfusepath{stroke}
\pgfpathmoveto{\pgfpoint{137.376007pt}{201.941055pt}}
\pgflineto{\pgfpoint{137.376007pt}{195.764206pt}}
\pgfusepath{stroke}
\pgfpathmoveto{\pgfpoint{137.376007pt}{208.117905pt}}
\pgflineto{\pgfpoint{137.376007pt}{201.941055pt}}
\pgfusepath{stroke}
\pgfpathmoveto{\pgfpoint{137.376007pt}{164.880005pt}}
\pgflineto{\pgfpoint{137.393967pt}{164.880005pt}}
\pgfusepath{stroke}
\pgfpathmoveto{\pgfpoint{137.376007pt}{171.056854pt}}
\pgflineto{\pgfpoint{137.393967pt}{171.056854pt}}
\pgfusepath{stroke}
\pgfpathmoveto{\pgfpoint{137.376007pt}{177.233673pt}}
\pgflineto{\pgfpoint{137.393921pt}{177.233673pt}}
\pgfusepath{stroke}
\pgfpathmoveto{\pgfpoint{137.376007pt}{183.410522pt}}
\pgflineto{\pgfpoint{137.393921pt}{183.410522pt}}
\pgfusepath{stroke}
\pgfpathmoveto{\pgfpoint{137.376007pt}{189.587372pt}}
\pgflineto{\pgfpoint{137.393921pt}{189.587372pt}}
\pgfusepath{stroke}
\pgfpathmoveto{\pgfpoint{137.376007pt}{195.764206pt}}
\pgflineto{\pgfpoint{137.402878pt}{195.764206pt}}
\pgfusepath{stroke}
\pgfpathmoveto{\pgfpoint{137.376007pt}{201.941055pt}}
\pgflineto{\pgfpoint{137.402939pt}{201.941055pt}}
\pgfusepath{stroke}
\pgfpathmoveto{\pgfpoint{137.376007pt}{220.471588pt}}
\pgflineto{\pgfpoint{137.376007pt}{214.294739pt}}
\pgfusepath{stroke}
\pgfpathmoveto{\pgfpoint{137.376007pt}{214.294739pt}}
\pgflineto{\pgfpoint{137.376007pt}{208.117905pt}}
\pgfusepath{stroke}
\pgfpathmoveto{\pgfpoint{137.376007pt}{226.648422pt}}
\pgflineto{\pgfpoint{137.376007pt}{220.471588pt}}
\pgfusepath{stroke}
\pgfpathmoveto{\pgfpoint{137.339890pt}{245.178955pt}}
\pgflineto{\pgfpoint{137.358078pt}{245.178955pt}}
\pgfusepath{stroke}
\pgfpathmoveto{\pgfpoint{137.376007pt}{251.355804pt}}
\pgflineto{\pgfpoint{137.339874pt}{251.355804pt}}
\pgfusepath{stroke}
\pgfpathmoveto{\pgfpoint{137.376007pt}{257.532623pt}}
\pgflineto{\pgfpoint{137.330811pt}{257.532623pt}}
\pgfusepath{stroke}
\pgfpathmoveto{\pgfpoint{137.376007pt}{263.709473pt}}
\pgflineto{\pgfpoint{137.321777pt}{263.709473pt}}
\pgfusepath{stroke}
\pgfpathmoveto{\pgfpoint{137.376007pt}{269.886322pt}}
\pgflineto{\pgfpoint{137.312805pt}{269.886322pt}}
\pgfusepath{stroke}
\pgfpathmoveto{\pgfpoint{137.376007pt}{276.063141pt}}
\pgflineto{\pgfpoint{137.303757pt}{276.063141pt}}
\pgfusepath{stroke}
\pgfpathmoveto{\pgfpoint{137.376007pt}{282.239990pt}}
\pgflineto{\pgfpoint{137.303680pt}{282.239990pt}}
\pgfusepath{stroke}
\pgfpathmoveto{\pgfpoint{137.376007pt}{288.416840pt}}
\pgflineto{\pgfpoint{137.294601pt}{288.416840pt}}
\pgfusepath{stroke}
\pgfpathmoveto{\pgfpoint{137.376007pt}{294.593689pt}}
\pgflineto{\pgfpoint{137.285599pt}{294.593689pt}}
\pgfusepath{stroke}
\pgfpathmoveto{\pgfpoint{137.376007pt}{300.770538pt}}
\pgflineto{\pgfpoint{137.276627pt}{300.770538pt}}
\pgfusepath{stroke}
\pgfpathmoveto{\pgfpoint{137.376007pt}{306.947388pt}}
\pgflineto{\pgfpoint{137.267609pt}{306.947388pt}}
\pgfusepath{stroke}
\pgfpathmoveto{\pgfpoint{137.376007pt}{313.124207pt}}
\pgflineto{\pgfpoint{137.267487pt}{313.124207pt}}
\pgfusepath{stroke}
\pgfpathmoveto{\pgfpoint{137.376007pt}{319.301056pt}}
\pgflineto{\pgfpoint{137.258438pt}{319.301056pt}}
\pgfusepath{stroke}
\pgfpathmoveto{\pgfpoint{137.376007pt}{325.477905pt}}
\pgflineto{\pgfpoint{137.249527pt}{325.477905pt}}
\pgfusepath{stroke}
\pgfpathmoveto{\pgfpoint{137.376007pt}{331.654724pt}}
\pgflineto{\pgfpoint{137.240448pt}{331.654724pt}}
\pgfusepath{stroke}
\pgfpathmoveto{\pgfpoint{137.376007pt}{337.831604pt}}
\pgflineto{\pgfpoint{137.231430pt}{337.831604pt}}
\pgfusepath{stroke}
\pgfpathmoveto{\pgfpoint{137.376007pt}{344.008423pt}}
\pgflineto{\pgfpoint{137.222336pt}{344.008423pt}}
\pgfusepath{stroke}
\pgfpathmoveto{\pgfpoint{137.376007pt}{350.185242pt}}
\pgflineto{\pgfpoint{137.217789pt}{350.185242pt}}
\pgfusepath{stroke}
\pgfpathmoveto{\pgfpoint{137.376007pt}{356.362122pt}}
\pgflineto{\pgfpoint{137.208755pt}{356.362122pt}}
\pgfusepath{stroke}
\pgfpathmoveto{\pgfpoint{137.376007pt}{362.538940pt}}
\pgflineto{\pgfpoint{137.204254pt}{362.538940pt}}
\pgfusepath{stroke}
\pgfpathmoveto{\pgfpoint{137.376007pt}{368.715820pt}}
\pgflineto{\pgfpoint{137.195282pt}{368.715820pt}}
\pgfusepath{stroke}
\pgfpathmoveto{\pgfpoint{137.376007pt}{374.892639pt}}
\pgflineto{\pgfpoint{137.190613pt}{374.892639pt}}
\pgfusepath{stroke}
\pgfpathmoveto{\pgfpoint{137.376007pt}{381.069458pt}}
\pgflineto{\pgfpoint{137.181641pt}{381.069458pt}}
\pgfusepath{stroke}
\pgfpathmoveto{\pgfpoint{137.376007pt}{387.246338pt}}
\pgflineto{\pgfpoint{137.172668pt}{387.246338pt}}
\pgfusepath{stroke}
\pgfpathmoveto{\pgfpoint{137.376007pt}{393.423157pt}}
\pgflineto{\pgfpoint{137.168060pt}{393.423157pt}}
\pgfusepath{stroke}
\pgfpathmoveto{\pgfpoint{137.376007pt}{399.600037pt}}
\pgflineto{\pgfpoint{137.159088pt}{399.600037pt}}
\pgfusepath{stroke}
\pgfpathmoveto{\pgfpoint{137.376007pt}{356.362122pt}}
\pgflineto{\pgfpoint{137.376007pt}{350.185242pt}}
\pgfusepath{stroke}
\pgfpathmoveto{\pgfpoint{137.376007pt}{350.185242pt}}
\pgflineto{\pgfpoint{137.376007pt}{344.008423pt}}
\pgfusepath{stroke}
\pgfpathmoveto{\pgfpoint{137.376007pt}{362.538940pt}}
\pgflineto{\pgfpoint{137.376007pt}{356.362122pt}}
\pgfusepath{stroke}
\pgfpathmoveto{\pgfpoint{137.376007pt}{368.715820pt}}
\pgflineto{\pgfpoint{137.376007pt}{362.538940pt}}
\pgfusepath{stroke}
\pgfpathmoveto{\pgfpoint{137.376007pt}{350.185242pt}}
\pgflineto{\pgfpoint{137.389313pt}{350.185242pt}}
\pgfusepath{stroke}
\pgfpathmoveto{\pgfpoint{137.376007pt}{356.362122pt}}
\pgflineto{\pgfpoint{137.389313pt}{356.362122pt}}
\pgfusepath{stroke}
\pgfpathmoveto{\pgfpoint{137.376007pt}{362.538940pt}}
\pgflineto{\pgfpoint{137.393799pt}{362.538940pt}}
\pgfusepath{stroke}
\pgfpathmoveto{\pgfpoint{137.376007pt}{393.423157pt}}
\pgflineto{\pgfpoint{137.376007pt}{387.246338pt}}
\pgfusepath{stroke}
\pgfpathmoveto{\pgfpoint{137.376007pt}{374.892639pt}}
\pgflineto{\pgfpoint{137.376007pt}{368.715820pt}}
\pgfusepath{stroke}
\pgfpathmoveto{\pgfpoint{137.376007pt}{381.069458pt}}
\pgflineto{\pgfpoint{137.376007pt}{374.892639pt}}
\pgfusepath{stroke}
\pgfpathmoveto{\pgfpoint{137.376007pt}{387.246338pt}}
\pgflineto{\pgfpoint{137.376007pt}{381.069458pt}}
\pgfusepath{stroke}
\pgfpathmoveto{\pgfpoint{137.376007pt}{399.600037pt}}
\pgflineto{\pgfpoint{137.376007pt}{393.423157pt}}
\pgfusepath{stroke}
\pgfpathmoveto{\pgfpoint{137.376007pt}{368.715820pt}}
\pgflineto{\pgfpoint{137.393799pt}{368.715820pt}}
\pgfusepath{stroke}
\pgfpathmoveto{\pgfpoint{137.376007pt}{374.892639pt}}
\pgflineto{\pgfpoint{137.398300pt}{374.892639pt}}
\pgfusepath{stroke}
\pgfpathmoveto{\pgfpoint{137.376007pt}{381.069458pt}}
\pgflineto{\pgfpoint{137.398239pt}{381.069458pt}}
\pgfusepath{stroke}
\pgfpathmoveto{\pgfpoint{137.376007pt}{387.246338pt}}
\pgflineto{\pgfpoint{137.398239pt}{387.246338pt}}
\pgfusepath{stroke}
\pgfpathmoveto{\pgfpoint{137.376007pt}{393.423157pt}}
\pgflineto{\pgfpoint{137.402725pt}{393.423157pt}}
\pgfusepath{stroke}
\pgfpathmoveto{\pgfpoint{137.376007pt}{399.600037pt}}
\pgflineto{\pgfpoint{137.402725pt}{399.600037pt}}
\pgfusepath{stroke}
\pgfpathmoveto{\pgfpoint{137.376007pt}{288.416840pt}}
\pgflineto{\pgfpoint{137.376007pt}{282.239990pt}}
\pgfusepath{stroke}
\pgfpathmoveto{\pgfpoint{137.376007pt}{257.532623pt}}
\pgflineto{\pgfpoint{137.376007pt}{251.355804pt}}
\pgfusepath{stroke}
\pgfpathmoveto{\pgfpoint{137.376007pt}{263.709473pt}}
\pgflineto{\pgfpoint{137.376007pt}{257.532623pt}}
\pgfusepath{stroke}
\pgfpathmoveto{\pgfpoint{137.376007pt}{269.886322pt}}
\pgflineto{\pgfpoint{137.376007pt}{263.709473pt}}
\pgfusepath{stroke}
\pgfpathmoveto{\pgfpoint{137.376007pt}{276.063141pt}}
\pgflineto{\pgfpoint{137.376007pt}{269.886322pt}}
\pgfusepath{stroke}
\pgfpathmoveto{\pgfpoint{137.376007pt}{282.239990pt}}
\pgflineto{\pgfpoint{137.376007pt}{276.063141pt}}
\pgfusepath{stroke}
\pgfpathmoveto{\pgfpoint{137.376007pt}{294.593689pt}}
\pgflineto{\pgfpoint{137.376007pt}{288.416840pt}}
\pgfusepath{stroke}
\pgfpathmoveto{\pgfpoint{137.376007pt}{300.770538pt}}
\pgflineto{\pgfpoint{137.376007pt}{294.593689pt}}
\pgfusepath{stroke}
\pgfpathmoveto{\pgfpoint{137.376007pt}{306.947388pt}}
\pgflineto{\pgfpoint{137.376007pt}{300.770538pt}}
\pgfusepath{stroke}
\pgfpathmoveto{\pgfpoint{137.376007pt}{313.124207pt}}
\pgflineto{\pgfpoint{137.376007pt}{306.947388pt}}
\pgfusepath{stroke}
\pgfpathmoveto{\pgfpoint{137.376007pt}{319.301056pt}}
\pgflineto{\pgfpoint{137.376007pt}{313.124207pt}}
\pgfusepath{stroke}
\pgfpathmoveto{\pgfpoint{137.376007pt}{325.477905pt}}
\pgflineto{\pgfpoint{137.376007pt}{319.301056pt}}
\pgfusepath{stroke}
\pgfpathmoveto{\pgfpoint{137.376007pt}{331.654724pt}}
\pgflineto{\pgfpoint{137.376007pt}{325.477905pt}}
\pgfusepath{stroke}
\pgfpathmoveto{\pgfpoint{137.376007pt}{337.831604pt}}
\pgflineto{\pgfpoint{137.376007pt}{331.654724pt}}
\pgfusepath{stroke}
\pgfpathmoveto{\pgfpoint{137.376007pt}{344.008423pt}}
\pgflineto{\pgfpoint{137.376007pt}{337.831604pt}}
\pgfusepath{stroke}
\pgfpathmoveto{\pgfpoint{137.376007pt}{208.117905pt}}
\pgflineto{\pgfpoint{137.402908pt}{208.117905pt}}
\pgfusepath{stroke}
\pgfpathmoveto{\pgfpoint{137.376007pt}{214.294739pt}}
\pgflineto{\pgfpoint{137.402863pt}{214.294739pt}}
\pgfusepath{stroke}
\pgfpathmoveto{\pgfpoint{137.376007pt}{220.471588pt}}
\pgflineto{\pgfpoint{137.411926pt}{220.471588pt}}
\pgfusepath{stroke}
\pgfpathmoveto{\pgfpoint{137.376007pt}{226.648422pt}}
\pgflineto{\pgfpoint{137.411926pt}{226.648422pt}}
\pgfusepath{stroke}
\pgfpathmoveto{\pgfpoint{137.376007pt}{232.825272pt}}
\pgflineto{\pgfpoint{137.376007pt}{226.648422pt}}
\pgfusepath{stroke}
\pgfpathmoveto{\pgfpoint{137.376007pt}{232.825272pt}}
\pgflineto{\pgfpoint{137.357986pt}{232.825272pt}}
\pgfusepath{stroke}
\pgfpathmoveto{\pgfpoint{137.376007pt}{232.825272pt}}
\pgflineto{\pgfpoint{137.411896pt}{232.825272pt}}
\pgfusepath{stroke}
\pgfpathmoveto{\pgfpoint{137.376007pt}{239.002106pt}}
\pgflineto{\pgfpoint{137.348938pt}{239.002106pt}}
\pgfusepath{stroke}
\pgfpathmoveto{\pgfpoint{137.376007pt}{245.178955pt}}
\pgflineto{\pgfpoint{137.358078pt}{245.178955pt}}
\pgfusepath{stroke}
\pgfpathmoveto{\pgfpoint{137.376007pt}{239.002106pt}}
\pgflineto{\pgfpoint{137.376007pt}{232.825272pt}}
\pgfusepath{stroke}
\pgfpathmoveto{\pgfpoint{137.376007pt}{251.355804pt}}
\pgflineto{\pgfpoint{137.376007pt}{245.178955pt}}
\pgfusepath{stroke}
\pgfpathmoveto{\pgfpoint{137.376007pt}{245.178955pt}}
\pgflineto{\pgfpoint{137.376007pt}{239.002106pt}}
\pgfusepath{stroke}
\pgfpathmoveto{\pgfpoint{137.376007pt}{239.002106pt}}
\pgflineto{\pgfpoint{137.411850pt}{239.002106pt}}
\pgfusepath{stroke}
\pgfpathmoveto{\pgfpoint{137.376007pt}{245.178955pt}}
\pgflineto{\pgfpoint{137.411850pt}{245.178955pt}}
\pgfusepath{stroke}
\pgfpathmoveto{\pgfpoint{137.376007pt}{251.355804pt}}
\pgflineto{\pgfpoint{137.420837pt}{251.355804pt}}
\pgfusepath{stroke}
\pgfpathmoveto{\pgfpoint{137.376007pt}{257.532623pt}}
\pgflineto{\pgfpoint{137.420792pt}{257.532623pt}}
\pgfusepath{stroke}
\pgfpathmoveto{\pgfpoint{137.376007pt}{263.709473pt}}
\pgflineto{\pgfpoint{137.420837pt}{263.709473pt}}
\pgfusepath{stroke}
\pgfpathmoveto{\pgfpoint{137.376007pt}{269.886322pt}}
\pgflineto{\pgfpoint{137.420837pt}{269.886322pt}}
\pgfusepath{stroke}
\pgfpathmoveto{\pgfpoint{137.376007pt}{276.063141pt}}
\pgflineto{\pgfpoint{137.420837pt}{276.063141pt}}
\pgfusepath{stroke}
\pgfpathmoveto{\pgfpoint{137.376007pt}{282.239990pt}}
\pgflineto{\pgfpoint{137.429764pt}{282.239990pt}}
\pgfusepath{stroke}
\pgfpathmoveto{\pgfpoint{137.376007pt}{288.416840pt}}
\pgflineto{\pgfpoint{137.429764pt}{288.416840pt}}
\pgfusepath{stroke}
\pgfpathmoveto{\pgfpoint{137.376007pt}{294.593689pt}}
\pgflineto{\pgfpoint{137.429825pt}{294.593689pt}}
\pgfusepath{stroke}
\pgfpathmoveto{\pgfpoint{137.376007pt}{300.770538pt}}
\pgflineto{\pgfpoint{137.429825pt}{300.770538pt}}
\pgfusepath{stroke}
\pgfpathmoveto{\pgfpoint{137.376007pt}{306.947388pt}}
\pgflineto{\pgfpoint{137.429825pt}{306.947388pt}}
\pgfusepath{stroke}
\pgfpathmoveto{\pgfpoint{137.376007pt}{313.124207pt}}
\pgflineto{\pgfpoint{137.438736pt}{313.124207pt}}
\pgfusepath{stroke}
\pgfpathmoveto{\pgfpoint{137.376007pt}{319.301056pt}}
\pgflineto{\pgfpoint{137.438736pt}{319.301056pt}}
\pgfusepath{stroke}
\pgfpathmoveto{\pgfpoint{137.376007pt}{325.477905pt}}
\pgflineto{\pgfpoint{137.438736pt}{325.477905pt}}
\pgfusepath{stroke}
\pgfpathmoveto{\pgfpoint{137.376007pt}{331.654724pt}}
\pgflineto{\pgfpoint{137.438721pt}{331.654724pt}}
\pgfusepath{stroke}
\pgfpathmoveto{\pgfpoint{137.376007pt}{337.831604pt}}
\pgflineto{\pgfpoint{137.438766pt}{337.831604pt}}
\pgfusepath{stroke}
\pgfpathmoveto{\pgfpoint{137.376007pt}{344.008423pt}}
\pgflineto{\pgfpoint{137.438751pt}{344.008423pt}}
\pgfusepath{stroke}
\pgfpathmoveto{\pgfpoint{137.443207pt}{350.185242pt}}
\pgflineto{\pgfpoint{137.389313pt}{350.185242pt}}
\pgfusepath{stroke}
\pgfpathmoveto{\pgfpoint{137.443207pt}{356.362122pt}}
\pgflineto{\pgfpoint{137.389313pt}{356.362122pt}}
\pgfusepath{stroke}
\pgfpathmoveto{\pgfpoint{137.447693pt}{362.538940pt}}
\pgflineto{\pgfpoint{137.393799pt}{362.538940pt}}
\pgfusepath{stroke}
\pgfpathmoveto{\pgfpoint{137.447693pt}{368.715820pt}}
\pgflineto{\pgfpoint{137.393799pt}{368.715820pt}}
\pgfusepath{stroke}
\pgfpathmoveto{\pgfpoint{137.452194pt}{374.892639pt}}
\pgflineto{\pgfpoint{137.398300pt}{374.892639pt}}
\pgfusepath{stroke}
\pgfpathmoveto{\pgfpoint{137.452133pt}{381.069458pt}}
\pgflineto{\pgfpoint{137.398239pt}{381.069458pt}}
\pgfusepath{stroke}
\pgfpathmoveto{\pgfpoint{137.452194pt}{387.246338pt}}
\pgflineto{\pgfpoint{137.398239pt}{387.246338pt}}
\pgfusepath{stroke}
\pgfpathmoveto{\pgfpoint{137.456619pt}{393.423157pt}}
\pgflineto{\pgfpoint{137.402725pt}{393.423157pt}}
\pgfusepath{stroke}
\pgfpathmoveto{\pgfpoint{137.456680pt}{399.600037pt}}
\pgflineto{\pgfpoint{137.402725pt}{399.600037pt}}
\pgfusepath{stroke}
\pgfpathmoveto{\pgfpoint{137.376007pt}{109.288422pt}}
\pgflineto{\pgfpoint{137.376007pt}{103.111580pt}}
\pgfusepath{stroke}
\pgfpathmoveto{\pgfpoint{137.376007pt}{78.404205pt}}
\pgflineto{\pgfpoint{137.376007pt}{72.227356pt}}
\pgfusepath{stroke}
\pgfpathmoveto{\pgfpoint{137.376007pt}{84.581039pt}}
\pgflineto{\pgfpoint{137.376007pt}{78.404205pt}}
\pgfusepath{stroke}
\pgfpathmoveto{\pgfpoint{137.376007pt}{90.757896pt}}
\pgflineto{\pgfpoint{137.376007pt}{84.581039pt}}
\pgfusepath{stroke}
\pgfpathmoveto{\pgfpoint{137.376007pt}{96.934731pt}}
\pgflineto{\pgfpoint{137.376007pt}{90.757896pt}}
\pgfusepath{stroke}
\pgfpathmoveto{\pgfpoint{137.376007pt}{103.111580pt}}
\pgflineto{\pgfpoint{137.376007pt}{96.934731pt}}
\pgfusepath{stroke}
\pgfpathmoveto{\pgfpoint{137.376007pt}{115.465263pt}}
\pgflineto{\pgfpoint{137.376007pt}{109.288422pt}}
\pgfusepath{stroke}
\pgfpathmoveto{\pgfpoint{137.376007pt}{121.642097pt}}
\pgflineto{\pgfpoint{137.376007pt}{115.465263pt}}
\pgfusepath{stroke}
\pgfpathmoveto{\pgfpoint{137.376007pt}{127.818947pt}}
\pgflineto{\pgfpoint{137.376007pt}{121.642097pt}}
\pgfusepath{stroke}
\pgfpathmoveto{\pgfpoint{137.376007pt}{133.995789pt}}
\pgflineto{\pgfpoint{137.376007pt}{127.818947pt}}
\pgfusepath{stroke}
\pgfpathmoveto{\pgfpoint{137.376007pt}{140.172638pt}}
\pgflineto{\pgfpoint{137.376007pt}{133.995789pt}}
\pgfusepath{stroke}
\pgfpathmoveto{\pgfpoint{137.376007pt}{146.349472pt}}
\pgflineto{\pgfpoint{137.376007pt}{140.172638pt}}
\pgfusepath{stroke}
\pgfpathmoveto{\pgfpoint{137.376007pt}{152.526306pt}}
\pgflineto{\pgfpoint{137.376007pt}{146.349472pt}}
\pgfusepath{stroke}
\pgfpathmoveto{\pgfpoint{137.376007pt}{158.703156pt}}
\pgflineto{\pgfpoint{137.376007pt}{152.526306pt}}
\pgfusepath{stroke}
\pgfpathmoveto{\pgfpoint{137.376007pt}{47.519989pt}}
\pgflineto{\pgfpoint{137.358093pt}{47.519989pt}}
\pgfusepath{stroke}
\pgfpathmoveto{\pgfpoint{137.376007pt}{53.696838pt}}
\pgflineto{\pgfpoint{137.358047pt}{53.696838pt}}
\pgfusepath{stroke}
\pgfpathmoveto{\pgfpoint{137.376007pt}{59.873672pt}}
\pgflineto{\pgfpoint{137.358093pt}{59.873672pt}}
\pgfusepath{stroke}
\pgfpathmoveto{\pgfpoint{137.376007pt}{66.050522pt}}
\pgflineto{\pgfpoint{137.358093pt}{66.050522pt}}
\pgfusepath{stroke}
\pgfpathmoveto{\pgfpoint{137.376007pt}{53.696838pt}}
\pgflineto{\pgfpoint{137.376007pt}{47.519989pt}}
\pgfusepath{stroke}
\pgfpathmoveto{\pgfpoint{137.376007pt}{59.873672pt}}
\pgflineto{\pgfpoint{137.376007pt}{53.696838pt}}
\pgfusepath{stroke}
\pgfpathmoveto{\pgfpoint{137.376007pt}{47.519989pt}}
\pgflineto{\pgfpoint{137.403046pt}{47.519989pt}}
\pgfusepath{stroke}
\pgfpathmoveto{\pgfpoint{137.376007pt}{53.696838pt}}
\pgflineto{\pgfpoint{137.394012pt}{53.696838pt}}
\pgfusepath{stroke}
\pgfpathmoveto{\pgfpoint{137.376007pt}{72.227356pt}}
\pgflineto{\pgfpoint{137.376007pt}{66.050522pt}}
\pgfusepath{stroke}
\pgfpathmoveto{\pgfpoint{137.376007pt}{66.050522pt}}
\pgflineto{\pgfpoint{137.376007pt}{59.873672pt}}
\pgfusepath{stroke}
\pgfpathmoveto{\pgfpoint{137.403046pt}{47.519989pt}}
\pgflineto{\pgfpoint{146.286163pt}{47.519989pt}}
\pgfusepath{stroke}
\pgfpathmoveto{\pgfpoint{137.394012pt}{53.696838pt}}
\pgflineto{\pgfpoint{146.286118pt}{53.696838pt}}
\pgfusepath{stroke}
\pgfpathmoveto{\pgfpoint{146.303986pt}{59.873672pt}}
\pgflineto{\pgfpoint{137.376007pt}{59.873672pt}}
\pgfusepath{stroke}
\pgfpathmoveto{\pgfpoint{146.303986pt}{66.050522pt}}
\pgflineto{\pgfpoint{137.376007pt}{66.050522pt}}
\pgfusepath{stroke}
\pgfpathmoveto{\pgfpoint{146.303986pt}{72.227356pt}}
\pgflineto{\pgfpoint{137.376007pt}{72.227356pt}}
\pgfusepath{stroke}
\pgfpathmoveto{\pgfpoint{146.303986pt}{78.404205pt}}
\pgflineto{\pgfpoint{137.376007pt}{78.404205pt}}
\pgfusepath{stroke}
\pgfpathmoveto{\pgfpoint{146.303986pt}{84.581039pt}}
\pgflineto{\pgfpoint{137.376007pt}{84.581039pt}}
\pgfusepath{stroke}
\pgfpathmoveto{\pgfpoint{146.303986pt}{90.757896pt}}
\pgflineto{\pgfpoint{137.376007pt}{90.757896pt}}
\pgfusepath{stroke}
\pgfpathmoveto{\pgfpoint{146.303986pt}{96.934731pt}}
\pgflineto{\pgfpoint{137.376007pt}{96.934731pt}}
\pgfusepath{stroke}
\pgfpathmoveto{\pgfpoint{146.303986pt}{103.111580pt}}
\pgflineto{\pgfpoint{137.376007pt}{103.111580pt}}
\pgfusepath{stroke}
\pgfpathmoveto{\pgfpoint{146.303986pt}{109.288422pt}}
\pgflineto{\pgfpoint{137.376007pt}{109.288422pt}}
\pgfusepath{stroke}
\pgfpathmoveto{\pgfpoint{146.303986pt}{115.465263pt}}
\pgflineto{\pgfpoint{137.376007pt}{115.465263pt}}
\pgfusepath{stroke}
\pgfpathmoveto{\pgfpoint{146.303986pt}{121.642097pt}}
\pgflineto{\pgfpoint{137.376007pt}{121.642097pt}}
\pgfusepath{stroke}
\pgfpathmoveto{\pgfpoint{146.303986pt}{127.818947pt}}
\pgflineto{\pgfpoint{137.376007pt}{127.818947pt}}
\pgfusepath{stroke}
\pgfpathmoveto{\pgfpoint{146.303986pt}{133.995789pt}}
\pgflineto{\pgfpoint{137.376007pt}{133.995789pt}}
\pgfusepath{stroke}
\pgfpathmoveto{\pgfpoint{146.303986pt}{140.172638pt}}
\pgflineto{\pgfpoint{137.376007pt}{140.172638pt}}
\pgfusepath{stroke}
\pgfpathmoveto{\pgfpoint{146.303986pt}{146.349472pt}}
\pgflineto{\pgfpoint{137.376007pt}{146.349472pt}}
\pgfusepath{stroke}
\pgfpathmoveto{\pgfpoint{146.303986pt}{152.526306pt}}
\pgflineto{\pgfpoint{137.376007pt}{152.526306pt}}
\pgfusepath{stroke}
\pgfpathmoveto{\pgfpoint{146.303986pt}{158.703156pt}}
\pgflineto{\pgfpoint{137.376007pt}{158.703156pt}}
\pgfusepath{stroke}
\pgfpathmoveto{\pgfpoint{146.303986pt}{164.880005pt}}
\pgflineto{\pgfpoint{137.393967pt}{164.880005pt}}
\pgfusepath{stroke}
\pgfpathmoveto{\pgfpoint{146.303986pt}{171.056854pt}}
\pgflineto{\pgfpoint{137.393967pt}{171.056854pt}}
\pgfusepath{stroke}
\pgfpathmoveto{\pgfpoint{146.303986pt}{177.233673pt}}
\pgflineto{\pgfpoint{137.393921pt}{177.233673pt}}
\pgfusepath{stroke}
\pgfpathmoveto{\pgfpoint{146.303986pt}{183.410522pt}}
\pgflineto{\pgfpoint{137.393921pt}{183.410522pt}}
\pgfusepath{stroke}
\pgfpathmoveto{\pgfpoint{137.393921pt}{189.587372pt}}
\pgflineto{\pgfpoint{146.285995pt}{189.587372pt}}
\pgfusepath{stroke}
\pgfpathmoveto{\pgfpoint{137.402878pt}{195.764206pt}}
\pgflineto{\pgfpoint{146.276978pt}{195.764206pt}}
\pgfusepath{stroke}
\pgfpathmoveto{\pgfpoint{137.402939pt}{201.941055pt}}
\pgflineto{\pgfpoint{146.268005pt}{201.941055pt}}
\pgfusepath{stroke}
\pgfpathmoveto{\pgfpoint{137.402908pt}{208.117905pt}}
\pgflineto{\pgfpoint{146.268021pt}{208.117905pt}}
\pgfusepath{stroke}
\pgfpathmoveto{\pgfpoint{137.402863pt}{214.294739pt}}
\pgflineto{\pgfpoint{146.259003pt}{214.294739pt}}
\pgfusepath{stroke}
\pgfpathmoveto{\pgfpoint{137.411926pt}{220.471588pt}}
\pgflineto{\pgfpoint{146.250000pt}{220.471588pt}}
\pgfusepath{stroke}
\pgfpathmoveto{\pgfpoint{137.411926pt}{226.648422pt}}
\pgflineto{\pgfpoint{146.241028pt}{226.648422pt}}
\pgfusepath{stroke}
\pgfpathmoveto{\pgfpoint{137.411896pt}{232.825272pt}}
\pgflineto{\pgfpoint{146.240982pt}{232.825272pt}}
\pgfusepath{stroke}
\pgfpathmoveto{\pgfpoint{137.411850pt}{239.002106pt}}
\pgflineto{\pgfpoint{146.232010pt}{239.002106pt}}
\pgfusepath{stroke}
\pgfpathmoveto{\pgfpoint{137.411850pt}{245.178955pt}}
\pgflineto{\pgfpoint{146.222977pt}{245.178955pt}}
\pgfusepath{stroke}
\pgfpathmoveto{\pgfpoint{137.420837pt}{251.355804pt}}
\pgflineto{\pgfpoint{146.214050pt}{251.355804pt}}
\pgfusepath{stroke}
\pgfpathmoveto{\pgfpoint{137.420792pt}{257.532623pt}}
\pgflineto{\pgfpoint{146.205032pt}{257.532623pt}}
\pgfusepath{stroke}
\pgfpathmoveto{\pgfpoint{137.420837pt}{263.709473pt}}
\pgflineto{\pgfpoint{146.204956pt}{263.709473pt}}
\pgfusepath{stroke}
\pgfpathmoveto{\pgfpoint{137.420837pt}{269.886322pt}}
\pgflineto{\pgfpoint{146.195999pt}{269.886322pt}}
\pgfusepath{stroke}
\pgfpathmoveto{\pgfpoint{137.420837pt}{276.063141pt}}
\pgflineto{\pgfpoint{146.186905pt}{276.063141pt}}
\pgfusepath{stroke}
\pgfpathmoveto{\pgfpoint{137.429764pt}{282.239990pt}}
\pgflineto{\pgfpoint{146.178055pt}{282.239990pt}}
\pgfusepath{stroke}
\pgfpathmoveto{\pgfpoint{137.429764pt}{288.416840pt}}
\pgflineto{\pgfpoint{146.169052pt}{288.416840pt}}
\pgfusepath{stroke}
\pgfpathmoveto{\pgfpoint{137.429825pt}{294.593689pt}}
\pgflineto{\pgfpoint{146.168915pt}{294.593689pt}}
\pgfusepath{stroke}
\pgfpathmoveto{\pgfpoint{137.429825pt}{300.770538pt}}
\pgflineto{\pgfpoint{146.159943pt}{300.770538pt}}
\pgfusepath{stroke}
\pgfpathmoveto{\pgfpoint{137.429825pt}{306.947388pt}}
\pgflineto{\pgfpoint{146.150955pt}{306.947388pt}}
\pgfusepath{stroke}
\pgfpathmoveto{\pgfpoint{137.438736pt}{313.124207pt}}
\pgflineto{\pgfpoint{146.142090pt}{313.124207pt}}
\pgfusepath{stroke}
\pgfpathmoveto{\pgfpoint{137.438736pt}{319.301056pt}}
\pgflineto{\pgfpoint{146.133102pt}{319.301056pt}}
\pgfusepath{stroke}
\pgfpathmoveto{\pgfpoint{137.438736pt}{325.477905pt}}
\pgflineto{\pgfpoint{146.124084pt}{325.477905pt}}
\pgfusepath{stroke}
\pgfpathmoveto{\pgfpoint{137.438721pt}{331.654724pt}}
\pgflineto{\pgfpoint{146.119522pt}{331.654724pt}}
\pgfusepath{stroke}
\pgfpathmoveto{\pgfpoint{137.438766pt}{337.831604pt}}
\pgflineto{\pgfpoint{146.110535pt}{337.831604pt}}
\pgfusepath{stroke}
\pgfpathmoveto{\pgfpoint{137.438751pt}{344.008423pt}}
\pgflineto{\pgfpoint{146.105927pt}{344.008423pt}}
\pgfusepath{stroke}
\pgfpathmoveto{\pgfpoint{137.443207pt}{350.185242pt}}
\pgflineto{\pgfpoint{146.096954pt}{350.185242pt}}
\pgfusepath{stroke}
\pgfpathmoveto{\pgfpoint{137.443207pt}{356.362122pt}}
\pgflineto{\pgfpoint{146.088043pt}{356.362122pt}}
\pgfusepath{stroke}
\pgfpathmoveto{\pgfpoint{137.447693pt}{362.538940pt}}
\pgflineto{\pgfpoint{146.083496pt}{362.538940pt}}
\pgfusepath{stroke}
\pgfpathmoveto{\pgfpoint{137.447693pt}{368.715820pt}}
\pgflineto{\pgfpoint{146.074570pt}{368.715820pt}}
\pgfusepath{stroke}
\pgfpathmoveto{\pgfpoint{137.452194pt}{374.892639pt}}
\pgflineto{\pgfpoint{146.070038pt}{374.892639pt}}
\pgfusepath{stroke}
\pgfpathmoveto{\pgfpoint{137.452133pt}{381.069458pt}}
\pgflineto{\pgfpoint{146.061005pt}{381.069458pt}}
\pgfusepath{stroke}
\pgfpathmoveto{\pgfpoint{137.452194pt}{387.246338pt}}
\pgflineto{\pgfpoint{146.052063pt}{387.246338pt}}
\pgfusepath{stroke}
\pgfpathmoveto{\pgfpoint{137.456619pt}{393.423157pt}}
\pgflineto{\pgfpoint{146.047531pt}{393.423157pt}}
\pgfusepath{stroke}
\pgfpathmoveto{\pgfpoint{137.456680pt}{399.600037pt}}
\pgflineto{\pgfpoint{146.038544pt}{399.600037pt}}
\pgfusepath{stroke}
\pgfpathmoveto{\pgfpoint{146.303986pt}{152.526306pt}}
\pgflineto{\pgfpoint{146.303986pt}{146.349472pt}}
\pgfusepath{stroke}
\pgfpathmoveto{\pgfpoint{146.303986pt}{146.349472pt}}
\pgflineto{\pgfpoint{146.303986pt}{140.172638pt}}
\pgfusepath{stroke}
\pgfpathmoveto{\pgfpoint{146.303986pt}{158.703156pt}}
\pgflineto{\pgfpoint{146.303986pt}{152.526306pt}}
\pgfusepath{stroke}
\pgfpathmoveto{\pgfpoint{146.303986pt}{164.880005pt}}
\pgflineto{\pgfpoint{146.303986pt}{158.703156pt}}
\pgfusepath{stroke}
\pgfpathmoveto{\pgfpoint{146.303986pt}{146.349472pt}}
\pgflineto{\pgfpoint{146.321777pt}{146.349472pt}}
\pgfusepath{stroke}
\pgfpathmoveto{\pgfpoint{146.303986pt}{152.526306pt}}
\pgflineto{\pgfpoint{146.321716pt}{152.526306pt}}
\pgfusepath{stroke}
\pgfpathmoveto{\pgfpoint{146.303986pt}{158.703156pt}}
\pgflineto{\pgfpoint{146.321732pt}{158.703156pt}}
\pgfusepath{stroke}
\pgfpathmoveto{\pgfpoint{146.303986pt}{177.233673pt}}
\pgflineto{\pgfpoint{146.303986pt}{171.056854pt}}
\pgfusepath{stroke}
\pgfpathmoveto{\pgfpoint{146.303986pt}{171.056854pt}}
\pgflineto{\pgfpoint{146.303986pt}{164.880005pt}}
\pgfusepath{stroke}
\pgfpathmoveto{\pgfpoint{146.303986pt}{183.410522pt}}
\pgflineto{\pgfpoint{146.303986pt}{177.233673pt}}
\pgfusepath{stroke}
\pgfpathmoveto{\pgfpoint{146.303986pt}{164.880005pt}}
\pgflineto{\pgfpoint{146.321732pt}{164.880005pt}}
\pgfusepath{stroke}
\pgfpathmoveto{\pgfpoint{146.268005pt}{201.941055pt}}
\pgflineto{\pgfpoint{146.285843pt}{201.941055pt}}
\pgfusepath{stroke}
\pgfpathmoveto{\pgfpoint{146.303986pt}{208.117905pt}}
\pgflineto{\pgfpoint{146.268021pt}{208.117905pt}}
\pgfusepath{stroke}
\pgfpathmoveto{\pgfpoint{146.303986pt}{214.294739pt}}
\pgflineto{\pgfpoint{146.259003pt}{214.294739pt}}
\pgfusepath{stroke}
\pgfpathmoveto{\pgfpoint{146.303986pt}{220.471588pt}}
\pgflineto{\pgfpoint{146.250000pt}{220.471588pt}}
\pgfusepath{stroke}
\pgfpathmoveto{\pgfpoint{146.303986pt}{226.648422pt}}
\pgflineto{\pgfpoint{146.241028pt}{226.648422pt}}
\pgfusepath{stroke}
\pgfpathmoveto{\pgfpoint{146.303986pt}{232.825272pt}}
\pgflineto{\pgfpoint{146.240982pt}{232.825272pt}}
\pgfusepath{stroke}
\pgfpathmoveto{\pgfpoint{146.303986pt}{239.002106pt}}
\pgflineto{\pgfpoint{146.232010pt}{239.002106pt}}
\pgfusepath{stroke}
\pgfpathmoveto{\pgfpoint{146.303986pt}{245.178955pt}}
\pgflineto{\pgfpoint{146.222977pt}{245.178955pt}}
\pgfusepath{stroke}
\pgfpathmoveto{\pgfpoint{146.303986pt}{251.355804pt}}
\pgflineto{\pgfpoint{146.214050pt}{251.355804pt}}
\pgfusepath{stroke}
\pgfpathmoveto{\pgfpoint{146.303986pt}{257.532623pt}}
\pgflineto{\pgfpoint{146.205032pt}{257.532623pt}}
\pgfusepath{stroke}
\pgfpathmoveto{\pgfpoint{146.303986pt}{263.709473pt}}
\pgflineto{\pgfpoint{146.204956pt}{263.709473pt}}
\pgfusepath{stroke}
\pgfpathmoveto{\pgfpoint{146.303986pt}{269.886322pt}}
\pgflineto{\pgfpoint{146.195999pt}{269.886322pt}}
\pgfusepath{stroke}
\pgfpathmoveto{\pgfpoint{146.303986pt}{276.063141pt}}
\pgflineto{\pgfpoint{146.186905pt}{276.063141pt}}
\pgfusepath{stroke}
\pgfpathmoveto{\pgfpoint{146.303986pt}{282.239990pt}}
\pgflineto{\pgfpoint{146.178055pt}{282.239990pt}}
\pgfusepath{stroke}
\pgfpathmoveto{\pgfpoint{146.303986pt}{288.416840pt}}
\pgflineto{\pgfpoint{146.169052pt}{288.416840pt}}
\pgfusepath{stroke}
\pgfpathmoveto{\pgfpoint{146.303986pt}{294.593689pt}}
\pgflineto{\pgfpoint{146.168915pt}{294.593689pt}}
\pgfusepath{stroke}
\pgfpathmoveto{\pgfpoint{146.303986pt}{300.770538pt}}
\pgflineto{\pgfpoint{146.159943pt}{300.770538pt}}
\pgfusepath{stroke}
\pgfpathmoveto{\pgfpoint{146.303986pt}{306.947388pt}}
\pgflineto{\pgfpoint{146.150955pt}{306.947388pt}}
\pgfusepath{stroke}
\pgfpathmoveto{\pgfpoint{146.303986pt}{313.124207pt}}
\pgflineto{\pgfpoint{146.142090pt}{313.124207pt}}
\pgfusepath{stroke}
\pgfpathmoveto{\pgfpoint{146.303986pt}{319.301056pt}}
\pgflineto{\pgfpoint{146.133102pt}{319.301056pt}}
\pgfusepath{stroke}
\pgfpathmoveto{\pgfpoint{146.303986pt}{325.477905pt}}
\pgflineto{\pgfpoint{146.124084pt}{325.477905pt}}
\pgfusepath{stroke}
\pgfpathmoveto{\pgfpoint{146.303986pt}{331.654724pt}}
\pgflineto{\pgfpoint{146.119522pt}{331.654724pt}}
\pgfusepath{stroke}
\pgfpathmoveto{\pgfpoint{146.303986pt}{337.831604pt}}
\pgflineto{\pgfpoint{146.110535pt}{337.831604pt}}
\pgfusepath{stroke}
\pgfpathmoveto{\pgfpoint{146.303986pt}{344.008423pt}}
\pgflineto{\pgfpoint{146.105927pt}{344.008423pt}}
\pgfusepath{stroke}
\pgfpathmoveto{\pgfpoint{146.303986pt}{350.185242pt}}
\pgflineto{\pgfpoint{146.096954pt}{350.185242pt}}
\pgfusepath{stroke}
\pgfpathmoveto{\pgfpoint{146.303986pt}{356.362122pt}}
\pgflineto{\pgfpoint{146.088043pt}{356.362122pt}}
\pgfusepath{stroke}
\pgfpathmoveto{\pgfpoint{146.303986pt}{362.538940pt}}
\pgflineto{\pgfpoint{146.083496pt}{362.538940pt}}
\pgfusepath{stroke}
\pgfpathmoveto{\pgfpoint{146.074570pt}{368.715820pt}}
\pgflineto{\pgfpoint{146.290512pt}{368.715820pt}}
\pgfusepath{stroke}
\pgfpathmoveto{\pgfpoint{146.070038pt}{374.892639pt}}
\pgflineto{\pgfpoint{146.290527pt}{374.892639pt}}
\pgfusepath{stroke}
\pgfpathmoveto{\pgfpoint{146.061005pt}{381.069458pt}}
\pgflineto{\pgfpoint{146.285980pt}{381.069458pt}}
\pgfusepath{stroke}
\pgfpathmoveto{\pgfpoint{146.052063pt}{387.246338pt}}
\pgflineto{\pgfpoint{146.281555pt}{387.246338pt}}
\pgfusepath{stroke}
\pgfpathmoveto{\pgfpoint{146.047531pt}{393.423157pt}}
\pgflineto{\pgfpoint{146.281509pt}{393.423157pt}}
\pgfusepath{stroke}
\pgfpathmoveto{\pgfpoint{146.038544pt}{399.600037pt}}
\pgflineto{\pgfpoint{146.277069pt}{399.600037pt}}
\pgfusepath{stroke}
\pgfpathmoveto{\pgfpoint{146.303986pt}{300.770538pt}}
\pgflineto{\pgfpoint{146.303986pt}{294.593689pt}}
\pgfusepath{stroke}
\pgfpathmoveto{\pgfpoint{146.303986pt}{294.593689pt}}
\pgflineto{\pgfpoint{146.303986pt}{288.416840pt}}
\pgfusepath{stroke}
\pgfpathmoveto{\pgfpoint{146.303986pt}{306.947388pt}}
\pgflineto{\pgfpoint{146.303986pt}{300.770538pt}}
\pgfusepath{stroke}
\pgfpathmoveto{\pgfpoint{146.303986pt}{313.124207pt}}
\pgflineto{\pgfpoint{146.303986pt}{306.947388pt}}
\pgfusepath{stroke}
\pgfpathmoveto{\pgfpoint{146.303986pt}{319.301056pt}}
\pgflineto{\pgfpoint{146.303986pt}{313.124207pt}}
\pgfusepath{stroke}
\pgfpathmoveto{\pgfpoint{146.303986pt}{294.593689pt}}
\pgflineto{\pgfpoint{146.322235pt}{294.593689pt}}
\pgfusepath{stroke}
\pgfpathmoveto{\pgfpoint{146.303986pt}{300.770538pt}}
\pgflineto{\pgfpoint{146.322235pt}{300.770538pt}}
\pgfusepath{stroke}
\pgfpathmoveto{\pgfpoint{146.303986pt}{306.947388pt}}
\pgflineto{\pgfpoint{146.322311pt}{306.947388pt}}
\pgfusepath{stroke}
\pgfpathmoveto{\pgfpoint{146.303986pt}{313.124207pt}}
\pgflineto{\pgfpoint{146.322296pt}{313.124207pt}}
\pgfusepath{stroke}
\pgfpathmoveto{\pgfpoint{146.303986pt}{344.008423pt}}
\pgflineto{\pgfpoint{146.303986pt}{337.831604pt}}
\pgfusepath{stroke}
\pgfpathmoveto{\pgfpoint{146.303986pt}{325.477905pt}}
\pgflineto{\pgfpoint{146.303986pt}{319.301056pt}}
\pgfusepath{stroke}
\pgfpathmoveto{\pgfpoint{146.303986pt}{331.654724pt}}
\pgflineto{\pgfpoint{146.303986pt}{325.477905pt}}
\pgfusepath{stroke}
\pgfpathmoveto{\pgfpoint{146.303986pt}{337.831604pt}}
\pgflineto{\pgfpoint{146.303986pt}{331.654724pt}}
\pgfusepath{stroke}
\pgfpathmoveto{\pgfpoint{146.303986pt}{350.185242pt}}
\pgflineto{\pgfpoint{146.303986pt}{344.008423pt}}
\pgfusepath{stroke}
\pgfpathmoveto{\pgfpoint{146.303986pt}{356.362122pt}}
\pgflineto{\pgfpoint{146.303986pt}{350.185242pt}}
\pgfusepath{stroke}
\pgfpathmoveto{\pgfpoint{146.303986pt}{362.538940pt}}
\pgflineto{\pgfpoint{146.303986pt}{356.362122pt}}
\pgfusepath{stroke}
\pgfpathmoveto{\pgfpoint{146.303986pt}{319.301056pt}}
\pgflineto{\pgfpoint{146.322296pt}{319.301056pt}}
\pgfusepath{stroke}
\pgfpathmoveto{\pgfpoint{146.303986pt}{368.715820pt}}
\pgflineto{\pgfpoint{146.290512pt}{368.715820pt}}
\pgfusepath{stroke}
\pgfpathmoveto{\pgfpoint{146.303986pt}{374.892639pt}}
\pgflineto{\pgfpoint{146.290527pt}{374.892639pt}}
\pgfusepath{stroke}
\pgfpathmoveto{\pgfpoint{146.303986pt}{381.069458pt}}
\pgflineto{\pgfpoint{146.285980pt}{381.069458pt}}
\pgfusepath{stroke}
\pgfpathmoveto{\pgfpoint{146.303986pt}{387.246338pt}}
\pgflineto{\pgfpoint{146.281555pt}{387.246338pt}}
\pgfusepath{stroke}
\pgfpathmoveto{\pgfpoint{146.303986pt}{393.423157pt}}
\pgflineto{\pgfpoint{146.281509pt}{393.423157pt}}
\pgfusepath{stroke}
\pgfpathmoveto{\pgfpoint{146.303986pt}{399.600037pt}}
\pgflineto{\pgfpoint{146.277069pt}{399.600037pt}}
\pgfusepath{stroke}
\pgfpathmoveto{\pgfpoint{146.303986pt}{368.715820pt}}
\pgflineto{\pgfpoint{146.303986pt}{362.538940pt}}
\pgfusepath{stroke}
\pgfpathmoveto{\pgfpoint{146.303986pt}{374.892639pt}}
\pgflineto{\pgfpoint{146.303986pt}{368.715820pt}}
\pgfusepath{stroke}
\pgfpathmoveto{\pgfpoint{146.303986pt}{381.069458pt}}
\pgflineto{\pgfpoint{146.303986pt}{374.892639pt}}
\pgfusepath{stroke}
\pgfpathmoveto{\pgfpoint{146.303986pt}{387.246338pt}}
\pgflineto{\pgfpoint{146.303986pt}{381.069458pt}}
\pgfusepath{stroke}
\pgfpathmoveto{\pgfpoint{146.303986pt}{393.423157pt}}
\pgflineto{\pgfpoint{146.303986pt}{387.246338pt}}
\pgfusepath{stroke}
\pgfpathmoveto{\pgfpoint{146.303986pt}{399.600037pt}}
\pgflineto{\pgfpoint{146.303986pt}{393.423157pt}}
\pgfusepath{stroke}
\pgfpathmoveto{\pgfpoint{146.303986pt}{325.477905pt}}
\pgflineto{\pgfpoint{146.322296pt}{325.477905pt}}
\pgfusepath{stroke}
\pgfpathmoveto{\pgfpoint{146.303986pt}{331.654724pt}}
\pgflineto{\pgfpoint{146.326843pt}{331.654724pt}}
\pgfusepath{stroke}
\pgfpathmoveto{\pgfpoint{146.303986pt}{337.831604pt}}
\pgflineto{\pgfpoint{146.326904pt}{337.831604pt}}
\pgfusepath{stroke}
\pgfpathmoveto{\pgfpoint{146.303986pt}{344.008423pt}}
\pgflineto{\pgfpoint{146.331390pt}{344.008423pt}}
\pgfusepath{stroke}
\pgfpathmoveto{\pgfpoint{146.303986pt}{350.185242pt}}
\pgflineto{\pgfpoint{146.331390pt}{350.185242pt}}
\pgfusepath{stroke}
\pgfpathmoveto{\pgfpoint{146.303986pt}{356.362122pt}}
\pgflineto{\pgfpoint{146.331467pt}{356.362122pt}}
\pgfusepath{stroke}
\pgfpathmoveto{\pgfpoint{146.303986pt}{362.538940pt}}
\pgflineto{\pgfpoint{146.335938pt}{362.538940pt}}
\pgfusepath{stroke}
\pgfpathmoveto{\pgfpoint{146.303986pt}{368.715820pt}}
\pgflineto{\pgfpoint{146.335999pt}{368.715820pt}}
\pgfusepath{stroke}
\pgfpathmoveto{\pgfpoint{146.303986pt}{374.892639pt}}
\pgflineto{\pgfpoint{146.340561pt}{374.892639pt}}
\pgfusepath{stroke}
\pgfpathmoveto{\pgfpoint{146.303986pt}{381.069458pt}}
\pgflineto{\pgfpoint{146.340500pt}{381.069458pt}}
\pgfusepath{stroke}
\pgfpathmoveto{\pgfpoint{146.303986pt}{387.246338pt}}
\pgflineto{\pgfpoint{146.340561pt}{387.246338pt}}
\pgfusepath{stroke}
\pgfpathmoveto{\pgfpoint{146.303986pt}{393.423157pt}}
\pgflineto{\pgfpoint{146.345093pt}{393.423157pt}}
\pgfusepath{stroke}
\pgfpathmoveto{\pgfpoint{146.303986pt}{399.600037pt}}
\pgflineto{\pgfpoint{146.345154pt}{399.600037pt}}
\pgfusepath{stroke}
\pgfpathmoveto{\pgfpoint{146.303986pt}{239.002106pt}}
\pgflineto{\pgfpoint{146.303986pt}{232.825272pt}}
\pgfusepath{stroke}
\pgfpathmoveto{\pgfpoint{146.303986pt}{214.294739pt}}
\pgflineto{\pgfpoint{146.303986pt}{208.117905pt}}
\pgfusepath{stroke}
\pgfpathmoveto{\pgfpoint{146.303986pt}{220.471588pt}}
\pgflineto{\pgfpoint{146.303986pt}{214.294739pt}}
\pgfusepath{stroke}
\pgfpathmoveto{\pgfpoint{146.303986pt}{226.648422pt}}
\pgflineto{\pgfpoint{146.303986pt}{220.471588pt}}
\pgfusepath{stroke}
\pgfpathmoveto{\pgfpoint{146.303986pt}{232.825272pt}}
\pgflineto{\pgfpoint{146.303986pt}{226.648422pt}}
\pgfusepath{stroke}
\pgfpathmoveto{\pgfpoint{146.303986pt}{245.178955pt}}
\pgflineto{\pgfpoint{146.303986pt}{239.002106pt}}
\pgfusepath{stroke}
\pgfpathmoveto{\pgfpoint{146.303986pt}{251.355804pt}}
\pgflineto{\pgfpoint{146.303986pt}{245.178955pt}}
\pgfusepath{stroke}
\pgfpathmoveto{\pgfpoint{146.303986pt}{257.532623pt}}
\pgflineto{\pgfpoint{146.303986pt}{251.355804pt}}
\pgfusepath{stroke}
\pgfpathmoveto{\pgfpoint{146.303986pt}{263.709473pt}}
\pgflineto{\pgfpoint{146.303986pt}{257.532623pt}}
\pgfusepath{stroke}
\pgfpathmoveto{\pgfpoint{146.303986pt}{269.886322pt}}
\pgflineto{\pgfpoint{146.303986pt}{263.709473pt}}
\pgfusepath{stroke}
\pgfpathmoveto{\pgfpoint{146.303986pt}{276.063141pt}}
\pgflineto{\pgfpoint{146.303986pt}{269.886322pt}}
\pgfusepath{stroke}
\pgfpathmoveto{\pgfpoint{146.303986pt}{282.239990pt}}
\pgflineto{\pgfpoint{146.303986pt}{276.063141pt}}
\pgfusepath{stroke}
\pgfpathmoveto{\pgfpoint{146.303986pt}{288.416840pt}}
\pgflineto{\pgfpoint{146.303986pt}{282.239990pt}}
\pgfusepath{stroke}
\pgfpathmoveto{\pgfpoint{146.303986pt}{189.587372pt}}
\pgflineto{\pgfpoint{146.285995pt}{189.587372pt}}
\pgfusepath{stroke}
\pgfpathmoveto{\pgfpoint{146.303986pt}{195.764206pt}}
\pgflineto{\pgfpoint{146.276978pt}{195.764206pt}}
\pgfusepath{stroke}
\pgfpathmoveto{\pgfpoint{146.303986pt}{201.941055pt}}
\pgflineto{\pgfpoint{146.285843pt}{201.941055pt}}
\pgfusepath{stroke}
\pgfpathmoveto{\pgfpoint{146.303986pt}{189.587372pt}}
\pgflineto{\pgfpoint{146.303986pt}{183.410522pt}}
\pgfusepath{stroke}
\pgfpathmoveto{\pgfpoint{146.303986pt}{195.764206pt}}
\pgflineto{\pgfpoint{146.303986pt}{189.587372pt}}
\pgfusepath{stroke}
\pgfpathmoveto{\pgfpoint{146.303986pt}{171.056854pt}}
\pgflineto{\pgfpoint{146.321671pt}{171.056854pt}}
\pgfusepath{stroke}
\pgfpathmoveto{\pgfpoint{146.303986pt}{177.233673pt}}
\pgflineto{\pgfpoint{146.330582pt}{177.233673pt}}
\pgfusepath{stroke}
\pgfpathmoveto{\pgfpoint{146.303986pt}{183.410522pt}}
\pgflineto{\pgfpoint{146.330582pt}{183.410522pt}}
\pgfusepath{stroke}
\pgfpathmoveto{\pgfpoint{146.303986pt}{189.587372pt}}
\pgflineto{\pgfpoint{146.330597pt}{189.587372pt}}
\pgfusepath{stroke}
\pgfpathmoveto{\pgfpoint{146.303986pt}{208.117905pt}}
\pgflineto{\pgfpoint{146.303986pt}{201.941055pt}}
\pgfusepath{stroke}
\pgfpathmoveto{\pgfpoint{146.303986pt}{201.941055pt}}
\pgflineto{\pgfpoint{146.303986pt}{195.764206pt}}
\pgfusepath{stroke}
\pgfpathmoveto{\pgfpoint{146.303986pt}{195.764206pt}}
\pgflineto{\pgfpoint{146.330551pt}{195.764206pt}}
\pgfusepath{stroke}
\pgfpathmoveto{\pgfpoint{146.303986pt}{201.941055pt}}
\pgflineto{\pgfpoint{146.330551pt}{201.941055pt}}
\pgfusepath{stroke}
\pgfpathmoveto{\pgfpoint{146.303986pt}{208.117905pt}}
\pgflineto{\pgfpoint{146.339554pt}{208.117905pt}}
\pgfusepath{stroke}
\pgfpathmoveto{\pgfpoint{146.303986pt}{214.294739pt}}
\pgflineto{\pgfpoint{146.339447pt}{214.294739pt}}
\pgfusepath{stroke}
\pgfpathmoveto{\pgfpoint{146.303986pt}{220.471588pt}}
\pgflineto{\pgfpoint{146.339478pt}{220.471588pt}}
\pgfusepath{stroke}
\pgfpathmoveto{\pgfpoint{146.303986pt}{226.648422pt}}
\pgflineto{\pgfpoint{146.339478pt}{226.648422pt}}
\pgfusepath{stroke}
\pgfpathmoveto{\pgfpoint{146.303986pt}{232.825272pt}}
\pgflineto{\pgfpoint{146.348419pt}{232.825272pt}}
\pgfusepath{stroke}
\pgfpathmoveto{\pgfpoint{146.303986pt}{239.002106pt}}
\pgflineto{\pgfpoint{146.348358pt}{239.002106pt}}
\pgfusepath{stroke}
\pgfpathmoveto{\pgfpoint{146.303986pt}{245.178955pt}}
\pgflineto{\pgfpoint{146.348297pt}{245.178955pt}}
\pgfusepath{stroke}
\pgfpathmoveto{\pgfpoint{146.303986pt}{251.355804pt}}
\pgflineto{\pgfpoint{146.348343pt}{251.355804pt}}
\pgfusepath{stroke}
\pgfpathmoveto{\pgfpoint{146.303986pt}{257.532623pt}}
\pgflineto{\pgfpoint{146.348297pt}{257.532623pt}}
\pgfusepath{stroke}
\pgfpathmoveto{\pgfpoint{146.303986pt}{263.709473pt}}
\pgflineto{\pgfpoint{146.357285pt}{263.709473pt}}
\pgfusepath{stroke}
\pgfpathmoveto{\pgfpoint{146.303986pt}{269.886322pt}}
\pgflineto{\pgfpoint{146.357285pt}{269.886322pt}}
\pgfusepath{stroke}
\pgfpathmoveto{\pgfpoint{146.303986pt}{276.063141pt}}
\pgflineto{\pgfpoint{146.357285pt}{276.063141pt}}
\pgfusepath{stroke}
\pgfpathmoveto{\pgfpoint{146.303986pt}{282.239990pt}}
\pgflineto{\pgfpoint{146.357224pt}{282.239990pt}}
\pgfusepath{stroke}
\pgfpathmoveto{\pgfpoint{146.303986pt}{288.416840pt}}
\pgflineto{\pgfpoint{146.357224pt}{288.416840pt}}
\pgfusepath{stroke}
\pgfpathmoveto{\pgfpoint{146.366150pt}{294.593689pt}}
\pgflineto{\pgfpoint{146.322235pt}{294.593689pt}}
\pgfusepath{stroke}
\pgfpathmoveto{\pgfpoint{146.366150pt}{300.770538pt}}
\pgflineto{\pgfpoint{146.322235pt}{300.770538pt}}
\pgfusepath{stroke}
\pgfpathmoveto{\pgfpoint{146.366150pt}{306.947388pt}}
\pgflineto{\pgfpoint{146.322311pt}{306.947388pt}}
\pgfusepath{stroke}
\pgfpathmoveto{\pgfpoint{146.366058pt}{313.124207pt}}
\pgflineto{\pgfpoint{146.322296pt}{313.124207pt}}
\pgfusepath{stroke}
\pgfpathmoveto{\pgfpoint{146.366058pt}{319.301056pt}}
\pgflineto{\pgfpoint{146.322296pt}{319.301056pt}}
\pgfusepath{stroke}
\pgfpathmoveto{\pgfpoint{146.366058pt}{325.477905pt}}
\pgflineto{\pgfpoint{146.322296pt}{325.477905pt}}
\pgfusepath{stroke}
\pgfpathmoveto{\pgfpoint{146.370499pt}{331.654724pt}}
\pgflineto{\pgfpoint{146.326843pt}{331.654724pt}}
\pgfusepath{stroke}
\pgfpathmoveto{\pgfpoint{146.370499pt}{337.831604pt}}
\pgflineto{\pgfpoint{146.326904pt}{337.831604pt}}
\pgfusepath{stroke}
\pgfpathmoveto{\pgfpoint{146.374985pt}{344.008423pt}}
\pgflineto{\pgfpoint{146.331390pt}{344.008423pt}}
\pgfusepath{stroke}
\pgfpathmoveto{\pgfpoint{146.374924pt}{350.185242pt}}
\pgflineto{\pgfpoint{146.331390pt}{350.185242pt}}
\pgfusepath{stroke}
\pgfpathmoveto{\pgfpoint{146.374969pt}{356.362122pt}}
\pgflineto{\pgfpoint{146.331467pt}{356.362122pt}}
\pgfusepath{stroke}
\pgfpathmoveto{\pgfpoint{146.379364pt}{362.538940pt}}
\pgflineto{\pgfpoint{146.335938pt}{362.538940pt}}
\pgfusepath{stroke}
\pgfpathmoveto{\pgfpoint{146.379364pt}{368.715820pt}}
\pgflineto{\pgfpoint{146.335999pt}{368.715820pt}}
\pgfusepath{stroke}
\pgfpathmoveto{\pgfpoint{146.383804pt}{374.892639pt}}
\pgflineto{\pgfpoint{146.340561pt}{374.892639pt}}
\pgfusepath{stroke}
\pgfpathmoveto{\pgfpoint{146.383804pt}{381.069458pt}}
\pgflineto{\pgfpoint{146.340500pt}{381.069458pt}}
\pgfusepath{stroke}
\pgfpathmoveto{\pgfpoint{146.383835pt}{387.246338pt}}
\pgflineto{\pgfpoint{146.340561pt}{387.246338pt}}
\pgfusepath{stroke}
\pgfpathmoveto{\pgfpoint{146.388290pt}{393.423157pt}}
\pgflineto{\pgfpoint{146.345093pt}{393.423157pt}}
\pgfusepath{stroke}
\pgfpathmoveto{\pgfpoint{146.388290pt}{399.600037pt}}
\pgflineto{\pgfpoint{146.345154pt}{399.600037pt}}
\pgfusepath{stroke}
\pgfpathmoveto{\pgfpoint{146.303986pt}{90.757896pt}}
\pgflineto{\pgfpoint{146.303986pt}{84.581039pt}}
\pgfusepath{stroke}
\pgfpathmoveto{\pgfpoint{146.303986pt}{66.050522pt}}
\pgflineto{\pgfpoint{146.303986pt}{59.873672pt}}
\pgfusepath{stroke}
\pgfpathmoveto{\pgfpoint{146.303986pt}{72.227356pt}}
\pgflineto{\pgfpoint{146.303986pt}{66.050522pt}}
\pgfusepath{stroke}
\pgfpathmoveto{\pgfpoint{146.303986pt}{78.404205pt}}
\pgflineto{\pgfpoint{146.303986pt}{72.227356pt}}
\pgfusepath{stroke}
\pgfpathmoveto{\pgfpoint{146.303986pt}{84.581039pt}}
\pgflineto{\pgfpoint{146.303986pt}{78.404205pt}}
\pgfusepath{stroke}
\pgfpathmoveto{\pgfpoint{146.303986pt}{96.934731pt}}
\pgflineto{\pgfpoint{146.303986pt}{90.757896pt}}
\pgfusepath{stroke}
\pgfpathmoveto{\pgfpoint{146.303986pt}{103.111580pt}}
\pgflineto{\pgfpoint{146.303986pt}{96.934731pt}}
\pgfusepath{stroke}
\pgfpathmoveto{\pgfpoint{146.303986pt}{109.288422pt}}
\pgflineto{\pgfpoint{146.303986pt}{103.111580pt}}
\pgfusepath{stroke}
\pgfpathmoveto{\pgfpoint{146.303986pt}{115.465263pt}}
\pgflineto{\pgfpoint{146.303986pt}{109.288422pt}}
\pgfusepath{stroke}
\pgfpathmoveto{\pgfpoint{146.303986pt}{121.642097pt}}
\pgflineto{\pgfpoint{146.303986pt}{115.465263pt}}
\pgfusepath{stroke}
\pgfpathmoveto{\pgfpoint{146.303986pt}{127.818947pt}}
\pgflineto{\pgfpoint{146.303986pt}{121.642097pt}}
\pgfusepath{stroke}
\pgfpathmoveto{\pgfpoint{146.303986pt}{133.995789pt}}
\pgflineto{\pgfpoint{146.303986pt}{127.818947pt}}
\pgfusepath{stroke}
\pgfpathmoveto{\pgfpoint{146.303986pt}{140.172638pt}}
\pgflineto{\pgfpoint{146.303986pt}{133.995789pt}}
\pgfusepath{stroke}
\pgfpathmoveto{\pgfpoint{146.303986pt}{47.519989pt}}
\pgflineto{\pgfpoint{146.286163pt}{47.519989pt}}
\pgfusepath{stroke}
\pgfpathmoveto{\pgfpoint{146.303986pt}{53.696838pt}}
\pgflineto{\pgfpoint{146.286118pt}{53.696838pt}}
\pgfusepath{stroke}
\pgfpathmoveto{\pgfpoint{146.303986pt}{59.873672pt}}
\pgflineto{\pgfpoint{146.303986pt}{53.696838pt}}
\pgfusepath{stroke}
\pgfpathmoveto{\pgfpoint{146.303986pt}{53.696838pt}}
\pgflineto{\pgfpoint{146.303986pt}{47.519989pt}}
\pgfusepath{stroke}
\pgfpathmoveto{\pgfpoint{146.303986pt}{47.519989pt}}
\pgflineto{\pgfpoint{155.214081pt}{47.519989pt}}
\pgfusepath{stroke}
\pgfpathmoveto{\pgfpoint{146.303986pt}{53.696838pt}}
\pgflineto{\pgfpoint{155.214020pt}{53.696838pt}}
\pgfusepath{stroke}
\pgfpathmoveto{\pgfpoint{146.303986pt}{59.873672pt}}
\pgflineto{\pgfpoint{155.214050pt}{59.873672pt}}
\pgfusepath{stroke}
\pgfpathmoveto{\pgfpoint{146.303986pt}{66.050522pt}}
\pgflineto{\pgfpoint{155.214050pt}{66.050522pt}}
\pgfusepath{stroke}
\pgfpathmoveto{\pgfpoint{155.231979pt}{72.227356pt}}
\pgflineto{\pgfpoint{146.303986pt}{72.227356pt}}
\pgfusepath{stroke}
\pgfpathmoveto{\pgfpoint{155.231979pt}{78.404205pt}}
\pgflineto{\pgfpoint{146.303986pt}{78.404205pt}}
\pgfusepath{stroke}
\pgfpathmoveto{\pgfpoint{155.231979pt}{84.581039pt}}
\pgflineto{\pgfpoint{146.303986pt}{84.581039pt}}
\pgfusepath{stroke}
\pgfpathmoveto{\pgfpoint{155.231979pt}{90.757896pt}}
\pgflineto{\pgfpoint{146.303986pt}{90.757896pt}}
\pgfusepath{stroke}
\pgfpathmoveto{\pgfpoint{155.231979pt}{96.934731pt}}
\pgflineto{\pgfpoint{146.303986pt}{96.934731pt}}
\pgfusepath{stroke}
\pgfpathmoveto{\pgfpoint{155.231979pt}{103.111580pt}}
\pgflineto{\pgfpoint{146.303986pt}{103.111580pt}}
\pgfusepath{stroke}
\pgfpathmoveto{\pgfpoint{155.231979pt}{109.288422pt}}
\pgflineto{\pgfpoint{146.303986pt}{109.288422pt}}
\pgfusepath{stroke}
\pgfpathmoveto{\pgfpoint{155.231979pt}{115.465263pt}}
\pgflineto{\pgfpoint{146.303986pt}{115.465263pt}}
\pgfusepath{stroke}
\pgfpathmoveto{\pgfpoint{155.231979pt}{121.642097pt}}
\pgflineto{\pgfpoint{146.303986pt}{121.642097pt}}
\pgfusepath{stroke}
\pgfpathmoveto{\pgfpoint{155.231979pt}{127.818947pt}}
\pgflineto{\pgfpoint{146.303986pt}{127.818947pt}}
\pgfusepath{stroke}
\pgfpathmoveto{\pgfpoint{155.231979pt}{133.995789pt}}
\pgflineto{\pgfpoint{146.303986pt}{133.995789pt}}
\pgfusepath{stroke}
\pgfpathmoveto{\pgfpoint{155.231979pt}{140.172638pt}}
\pgflineto{\pgfpoint{146.303986pt}{140.172638pt}}
\pgfusepath{stroke}
\pgfpathmoveto{\pgfpoint{155.231979pt}{146.349472pt}}
\pgflineto{\pgfpoint{146.321777pt}{146.349472pt}}
\pgfusepath{stroke}
\pgfpathmoveto{\pgfpoint{155.231979pt}{152.526306pt}}
\pgflineto{\pgfpoint{146.321716pt}{152.526306pt}}
\pgfusepath{stroke}
\pgfpathmoveto{\pgfpoint{155.231979pt}{158.703156pt}}
\pgflineto{\pgfpoint{146.321732pt}{158.703156pt}}
\pgfusepath{stroke}
\pgfpathmoveto{\pgfpoint{155.231979pt}{164.880005pt}}
\pgflineto{\pgfpoint{146.321732pt}{164.880005pt}}
\pgfusepath{stroke}
\pgfpathmoveto{\pgfpoint{155.231979pt}{171.056854pt}}
\pgflineto{\pgfpoint{146.321671pt}{171.056854pt}}
\pgfusepath{stroke}
\pgfpathmoveto{\pgfpoint{155.231979pt}{177.233673pt}}
\pgflineto{\pgfpoint{146.330582pt}{177.233673pt}}
\pgfusepath{stroke}
\pgfpathmoveto{\pgfpoint{155.231979pt}{183.410522pt}}
\pgflineto{\pgfpoint{146.330582pt}{183.410522pt}}
\pgfusepath{stroke}
\pgfpathmoveto{\pgfpoint{155.231979pt}{189.587372pt}}
\pgflineto{\pgfpoint{146.330597pt}{189.587372pt}}
\pgfusepath{stroke}
\pgfpathmoveto{\pgfpoint{155.231979pt}{195.764206pt}}
\pgflineto{\pgfpoint{146.330551pt}{195.764206pt}}
\pgfusepath{stroke}
\pgfpathmoveto{\pgfpoint{146.330551pt}{201.941055pt}}
\pgflineto{\pgfpoint{155.213898pt}{201.941055pt}}
\pgfusepath{stroke}
\pgfpathmoveto{\pgfpoint{146.339554pt}{208.117905pt}}
\pgflineto{\pgfpoint{155.204880pt}{208.117905pt}}
\pgfusepath{stroke}
\pgfpathmoveto{\pgfpoint{146.339447pt}{214.294739pt}}
\pgflineto{\pgfpoint{155.195862pt}{214.294739pt}}
\pgfusepath{stroke}
\pgfpathmoveto{\pgfpoint{146.339478pt}{220.471588pt}}
\pgflineto{\pgfpoint{155.195770pt}{220.471588pt}}
\pgfusepath{stroke}
\pgfpathmoveto{\pgfpoint{146.339478pt}{226.648422pt}}
\pgflineto{\pgfpoint{155.186768pt}{226.648422pt}}
\pgfusepath{stroke}
\pgfpathmoveto{\pgfpoint{146.348419pt}{232.825272pt}}
\pgflineto{\pgfpoint{155.177765pt}{232.825272pt}}
\pgfusepath{stroke}
\pgfpathmoveto{\pgfpoint{146.348358pt}{239.002106pt}}
\pgflineto{\pgfpoint{155.168716pt}{239.002106pt}}
\pgfusepath{stroke}
\pgfpathmoveto{\pgfpoint{146.348297pt}{245.178955pt}}
\pgflineto{\pgfpoint{155.159698pt}{245.178955pt}}
\pgfusepath{stroke}
\pgfpathmoveto{\pgfpoint{146.348343pt}{251.355804pt}}
\pgflineto{\pgfpoint{155.159607pt}{251.355804pt}}
\pgfusepath{stroke}
\pgfpathmoveto{\pgfpoint{146.348297pt}{257.532623pt}}
\pgflineto{\pgfpoint{155.150574pt}{257.532623pt}}
\pgfusepath{stroke}
\pgfpathmoveto{\pgfpoint{146.357285pt}{263.709473pt}}
\pgflineto{\pgfpoint{155.141632pt}{263.709473pt}}
\pgfusepath{stroke}
\pgfpathmoveto{\pgfpoint{146.357285pt}{269.886322pt}}
\pgflineto{\pgfpoint{155.132568pt}{269.886322pt}}
\pgfusepath{stroke}
\pgfpathmoveto{\pgfpoint{146.357285pt}{276.063141pt}}
\pgflineto{\pgfpoint{155.132431pt}{276.063141pt}}
\pgfusepath{stroke}
\pgfpathmoveto{\pgfpoint{146.357224pt}{282.239990pt}}
\pgflineto{\pgfpoint{155.123444pt}{282.239990pt}}
\pgfusepath{stroke}
\pgfpathmoveto{\pgfpoint{146.357224pt}{288.416840pt}}
\pgflineto{\pgfpoint{155.114349pt}{288.416840pt}}
\pgfusepath{stroke}
\pgfpathmoveto{\pgfpoint{146.366150pt}{294.593689pt}}
\pgflineto{\pgfpoint{155.105438pt}{294.593689pt}}
\pgfusepath{stroke}
\pgfpathmoveto{\pgfpoint{146.366150pt}{300.770538pt}}
\pgflineto{\pgfpoint{155.096466pt}{300.770538pt}}
\pgfusepath{stroke}
\pgfpathmoveto{\pgfpoint{146.366150pt}{306.947388pt}}
\pgflineto{\pgfpoint{155.096329pt}{306.947388pt}}
\pgfusepath{stroke}
\pgfpathmoveto{\pgfpoint{146.366058pt}{313.124207pt}}
\pgflineto{\pgfpoint{155.082794pt}{313.124207pt}}
\pgfusepath{stroke}
\pgfpathmoveto{\pgfpoint{146.366058pt}{319.301056pt}}
\pgflineto{\pgfpoint{155.073761pt}{319.301056pt}}
\pgfusepath{stroke}
\pgfpathmoveto{\pgfpoint{146.366058pt}{325.477905pt}}
\pgflineto{\pgfpoint{155.064728pt}{325.477905pt}}
\pgfusepath{stroke}
\pgfpathmoveto{\pgfpoint{146.370499pt}{331.654724pt}}
\pgflineto{\pgfpoint{155.060181pt}{331.654724pt}}
\pgfusepath{stroke}
\pgfpathmoveto{\pgfpoint{146.370499pt}{337.831604pt}}
\pgflineto{\pgfpoint{155.051193pt}{337.831604pt}}
\pgfusepath{stroke}
\pgfpathmoveto{\pgfpoint{146.374985pt}{344.008423pt}}
\pgflineto{\pgfpoint{155.046646pt}{344.008423pt}}
\pgfusepath{stroke}
\pgfpathmoveto{\pgfpoint{146.374924pt}{350.185242pt}}
\pgflineto{\pgfpoint{155.037567pt}{350.185242pt}}
\pgfusepath{stroke}
\pgfpathmoveto{\pgfpoint{146.374969pt}{356.362122pt}}
\pgflineto{\pgfpoint{155.032928pt}{356.362122pt}}
\pgfusepath{stroke}
\pgfpathmoveto{\pgfpoint{146.379364pt}{362.538940pt}}
\pgflineto{\pgfpoint{155.024017pt}{362.538940pt}}
\pgfusepath{stroke}
\pgfpathmoveto{\pgfpoint{146.379364pt}{368.715820pt}}
\pgflineto{\pgfpoint{155.014984pt}{368.715820pt}}
\pgfusepath{stroke}
\pgfpathmoveto{\pgfpoint{146.383804pt}{374.892639pt}}
\pgflineto{\pgfpoint{155.010437pt}{374.892639pt}}
\pgfusepath{stroke}
\pgfpathmoveto{\pgfpoint{146.383804pt}{381.069458pt}}
\pgflineto{\pgfpoint{155.001404pt}{381.069458pt}}
\pgfusepath{stroke}
\pgfpathmoveto{\pgfpoint{146.383835pt}{387.246338pt}}
\pgflineto{\pgfpoint{154.996765pt}{387.246338pt}}
\pgfusepath{stroke}
\pgfpathmoveto{\pgfpoint{146.388290pt}{393.423157pt}}
\pgflineto{\pgfpoint{154.987823pt}{393.423157pt}}
\pgfusepath{stroke}
\pgfpathmoveto{\pgfpoint{146.388290pt}{399.600037pt}}
\pgflineto{\pgfpoint{154.978851pt}{399.600037pt}}
\pgfusepath{stroke}
\pgfpathmoveto{\pgfpoint{155.231979pt}{164.880005pt}}
\pgflineto{\pgfpoint{155.231979pt}{158.703156pt}}
\pgfusepath{stroke}
\pgfpathmoveto{\pgfpoint{155.231979pt}{158.703156pt}}
\pgflineto{\pgfpoint{155.231979pt}{152.526306pt}}
\pgfusepath{stroke}
\pgfpathmoveto{\pgfpoint{155.231979pt}{171.056854pt}}
\pgflineto{\pgfpoint{155.231979pt}{164.880005pt}}
\pgfusepath{stroke}
\pgfpathmoveto{\pgfpoint{155.231979pt}{177.233673pt}}
\pgflineto{\pgfpoint{155.231979pt}{171.056854pt}}
\pgfusepath{stroke}
\pgfpathmoveto{\pgfpoint{155.231979pt}{158.703156pt}}
\pgflineto{\pgfpoint{155.249939pt}{158.703156pt}}
\pgfusepath{stroke}
\pgfpathmoveto{\pgfpoint{155.231979pt}{164.880005pt}}
\pgflineto{\pgfpoint{155.249939pt}{164.880005pt}}
\pgfusepath{stroke}
\pgfpathmoveto{\pgfpoint{155.231979pt}{171.056854pt}}
\pgflineto{\pgfpoint{155.249954pt}{171.056854pt}}
\pgfusepath{stroke}
\pgfpathmoveto{\pgfpoint{155.231979pt}{189.587372pt}}
\pgflineto{\pgfpoint{155.231979pt}{183.410522pt}}
\pgfusepath{stroke}
\pgfpathmoveto{\pgfpoint{155.231979pt}{183.410522pt}}
\pgflineto{\pgfpoint{155.231979pt}{177.233673pt}}
\pgfusepath{stroke}
\pgfpathmoveto{\pgfpoint{155.231979pt}{195.764206pt}}
\pgflineto{\pgfpoint{155.231979pt}{189.587372pt}}
\pgfusepath{stroke}
\pgfpathmoveto{\pgfpoint{155.231979pt}{177.233673pt}}
\pgflineto{\pgfpoint{155.249893pt}{177.233673pt}}
\pgfusepath{stroke}
\pgfpathmoveto{\pgfpoint{155.195862pt}{214.294739pt}}
\pgflineto{\pgfpoint{155.213821pt}{214.294739pt}}
\pgfusepath{stroke}
\pgfpathmoveto{\pgfpoint{155.231979pt}{220.471588pt}}
\pgflineto{\pgfpoint{155.195770pt}{220.471588pt}}
\pgfusepath{stroke}
\pgfpathmoveto{\pgfpoint{155.231979pt}{226.648422pt}}
\pgflineto{\pgfpoint{155.186768pt}{226.648422pt}}
\pgfusepath{stroke}
\pgfpathmoveto{\pgfpoint{155.231979pt}{232.825272pt}}
\pgflineto{\pgfpoint{155.177765pt}{232.825272pt}}
\pgfusepath{stroke}
\pgfpathmoveto{\pgfpoint{155.231979pt}{239.002106pt}}
\pgflineto{\pgfpoint{155.168716pt}{239.002106pt}}
\pgfusepath{stroke}
\pgfpathmoveto{\pgfpoint{155.231979pt}{245.178955pt}}
\pgflineto{\pgfpoint{155.159698pt}{245.178955pt}}
\pgfusepath{stroke}
\pgfpathmoveto{\pgfpoint{155.231979pt}{251.355804pt}}
\pgflineto{\pgfpoint{155.159607pt}{251.355804pt}}
\pgfusepath{stroke}
\pgfpathmoveto{\pgfpoint{155.231979pt}{257.532623pt}}
\pgflineto{\pgfpoint{155.150574pt}{257.532623pt}}
\pgfusepath{stroke}
\pgfpathmoveto{\pgfpoint{155.231979pt}{263.709473pt}}
\pgflineto{\pgfpoint{155.141632pt}{263.709473pt}}
\pgfusepath{stroke}
\pgfpathmoveto{\pgfpoint{155.231979pt}{269.886322pt}}
\pgflineto{\pgfpoint{155.132568pt}{269.886322pt}}
\pgfusepath{stroke}
\pgfpathmoveto{\pgfpoint{155.231979pt}{276.063141pt}}
\pgflineto{\pgfpoint{155.132431pt}{276.063141pt}}
\pgfusepath{stroke}
\pgfpathmoveto{\pgfpoint{155.231979pt}{282.239990pt}}
\pgflineto{\pgfpoint{155.123444pt}{282.239990pt}}
\pgfusepath{stroke}
\pgfpathmoveto{\pgfpoint{155.231979pt}{288.416840pt}}
\pgflineto{\pgfpoint{155.114349pt}{288.416840pt}}
\pgfusepath{stroke}
\pgfpathmoveto{\pgfpoint{155.231979pt}{294.593689pt}}
\pgflineto{\pgfpoint{155.105438pt}{294.593689pt}}
\pgfusepath{stroke}
\pgfpathmoveto{\pgfpoint{155.231979pt}{300.770538pt}}
\pgflineto{\pgfpoint{155.096466pt}{300.770538pt}}
\pgfusepath{stroke}
\pgfpathmoveto{\pgfpoint{155.231979pt}{306.947388pt}}
\pgflineto{\pgfpoint{155.096329pt}{306.947388pt}}
\pgfusepath{stroke}
\pgfpathmoveto{\pgfpoint{155.231979pt}{313.124207pt}}
\pgflineto{\pgfpoint{155.082794pt}{313.124207pt}}
\pgfusepath{stroke}
\pgfpathmoveto{\pgfpoint{155.231979pt}{319.301056pt}}
\pgflineto{\pgfpoint{155.073761pt}{319.301056pt}}
\pgfusepath{stroke}
\pgfpathmoveto{\pgfpoint{155.231979pt}{325.477905pt}}
\pgflineto{\pgfpoint{155.064728pt}{325.477905pt}}
\pgfusepath{stroke}
\pgfpathmoveto{\pgfpoint{155.231979pt}{331.654724pt}}
\pgflineto{\pgfpoint{155.060181pt}{331.654724pt}}
\pgfusepath{stroke}
\pgfpathmoveto{\pgfpoint{155.231979pt}{337.831604pt}}
\pgflineto{\pgfpoint{155.051193pt}{337.831604pt}}
\pgfusepath{stroke}
\pgfpathmoveto{\pgfpoint{155.231979pt}{344.008423pt}}
\pgflineto{\pgfpoint{155.046646pt}{344.008423pt}}
\pgfusepath{stroke}
\pgfpathmoveto{\pgfpoint{155.231979pt}{350.185242pt}}
\pgflineto{\pgfpoint{155.037567pt}{350.185242pt}}
\pgfusepath{stroke}
\pgfpathmoveto{\pgfpoint{155.231979pt}{356.362122pt}}
\pgflineto{\pgfpoint{155.032928pt}{356.362122pt}}
\pgfusepath{stroke}
\pgfpathmoveto{\pgfpoint{155.231979pt}{362.538940pt}}
\pgflineto{\pgfpoint{155.024017pt}{362.538940pt}}
\pgfusepath{stroke}
\pgfpathmoveto{\pgfpoint{155.231979pt}{368.715820pt}}
\pgflineto{\pgfpoint{155.014984pt}{368.715820pt}}
\pgfusepath{stroke}
\pgfpathmoveto{\pgfpoint{155.231979pt}{374.892639pt}}
\pgflineto{\pgfpoint{155.010437pt}{374.892639pt}}
\pgfusepath{stroke}
\pgfpathmoveto{\pgfpoint{155.001404pt}{381.069458pt}}
\pgflineto{\pgfpoint{155.218216pt}{381.069458pt}}
\pgfusepath{stroke}
\pgfpathmoveto{\pgfpoint{154.996765pt}{387.246338pt}}
\pgflineto{\pgfpoint{155.218231pt}{387.246338pt}}
\pgfusepath{stroke}
\pgfpathmoveto{\pgfpoint{154.987823pt}{393.423157pt}}
\pgflineto{\pgfpoint{155.213684pt}{393.423157pt}}
\pgfusepath{stroke}
\pgfpathmoveto{\pgfpoint{154.978851pt}{399.600037pt}}
\pgflineto{\pgfpoint{155.209122pt}{399.600037pt}}
\pgfusepath{stroke}
\pgfpathmoveto{\pgfpoint{155.231979pt}{319.301056pt}}
\pgflineto{\pgfpoint{155.231979pt}{313.124207pt}}
\pgfusepath{stroke}
\pgfpathmoveto{\pgfpoint{155.231979pt}{306.947388pt}}
\pgflineto{\pgfpoint{155.231979pt}{300.770538pt}}
\pgfusepath{stroke}
\pgfpathmoveto{\pgfpoint{155.231979pt}{313.124207pt}}
\pgflineto{\pgfpoint{155.231979pt}{306.947388pt}}
\pgfusepath{stroke}
\pgfpathmoveto{\pgfpoint{155.231979pt}{325.477905pt}}
\pgflineto{\pgfpoint{155.231979pt}{319.301056pt}}
\pgfusepath{stroke}
\pgfpathmoveto{\pgfpoint{155.231979pt}{306.947388pt}}
\pgflineto{\pgfpoint{155.250092pt}{306.947388pt}}
\pgfusepath{stroke}
\pgfpathmoveto{\pgfpoint{155.231979pt}{313.124207pt}}
\pgflineto{\pgfpoint{155.245621pt}{313.124207pt}}
\pgfusepath{stroke}
\pgfpathmoveto{\pgfpoint{155.231979pt}{319.301056pt}}
\pgflineto{\pgfpoint{155.245621pt}{319.301056pt}}
\pgfusepath{stroke}
\pgfpathmoveto{\pgfpoint{155.231979pt}{356.362122pt}}
\pgflineto{\pgfpoint{155.231979pt}{350.185242pt}}
\pgfusepath{stroke}
\pgfpathmoveto{\pgfpoint{155.231979pt}{331.654724pt}}
\pgflineto{\pgfpoint{155.231979pt}{325.477905pt}}
\pgfusepath{stroke}
\pgfpathmoveto{\pgfpoint{155.231979pt}{337.831604pt}}
\pgflineto{\pgfpoint{155.231979pt}{331.654724pt}}
\pgfusepath{stroke}
\pgfpathmoveto{\pgfpoint{155.231979pt}{344.008423pt}}
\pgflineto{\pgfpoint{155.231979pt}{337.831604pt}}
\pgfusepath{stroke}
\pgfpathmoveto{\pgfpoint{155.231979pt}{350.185242pt}}
\pgflineto{\pgfpoint{155.231979pt}{344.008423pt}}
\pgfusepath{stroke}
\pgfpathmoveto{\pgfpoint{155.231979pt}{362.538940pt}}
\pgflineto{\pgfpoint{155.231979pt}{356.362122pt}}
\pgfusepath{stroke}
\pgfpathmoveto{\pgfpoint{155.231979pt}{368.715820pt}}
\pgflineto{\pgfpoint{155.231979pt}{362.538940pt}}
\pgfusepath{stroke}
\pgfpathmoveto{\pgfpoint{155.231979pt}{374.892639pt}}
\pgflineto{\pgfpoint{155.231979pt}{368.715820pt}}
\pgfusepath{stroke}
\pgfpathmoveto{\pgfpoint{155.231979pt}{325.477905pt}}
\pgflineto{\pgfpoint{155.245636pt}{325.477905pt}}
\pgfusepath{stroke}
\pgfpathmoveto{\pgfpoint{155.231979pt}{331.654724pt}}
\pgflineto{\pgfpoint{155.250122pt}{331.654724pt}}
\pgfusepath{stroke}
\pgfpathmoveto{\pgfpoint{155.231979pt}{381.069458pt}}
\pgflineto{\pgfpoint{155.218216pt}{381.069458pt}}
\pgfusepath{stroke}
\pgfpathmoveto{\pgfpoint{155.231979pt}{387.246338pt}}
\pgflineto{\pgfpoint{155.218231pt}{387.246338pt}}
\pgfusepath{stroke}
\pgfpathmoveto{\pgfpoint{155.231979pt}{393.423157pt}}
\pgflineto{\pgfpoint{155.213684pt}{393.423157pt}}
\pgfusepath{stroke}
\pgfpathmoveto{\pgfpoint{155.231979pt}{399.600037pt}}
\pgflineto{\pgfpoint{155.209122pt}{399.600037pt}}
\pgfusepath{stroke}
\pgfpathmoveto{\pgfpoint{155.231979pt}{381.069458pt}}
\pgflineto{\pgfpoint{155.231979pt}{374.892639pt}}
\pgfusepath{stroke}
\pgfpathmoveto{\pgfpoint{155.231979pt}{387.246338pt}}
\pgflineto{\pgfpoint{155.231979pt}{381.069458pt}}
\pgfusepath{stroke}
\pgfpathmoveto{\pgfpoint{155.231979pt}{393.423157pt}}
\pgflineto{\pgfpoint{155.231979pt}{387.246338pt}}
\pgfusepath{stroke}
\pgfpathmoveto{\pgfpoint{155.231979pt}{399.600037pt}}
\pgflineto{\pgfpoint{155.231979pt}{393.423157pt}}
\pgfusepath{stroke}
\pgfpathmoveto{\pgfpoint{155.231979pt}{337.831604pt}}
\pgflineto{\pgfpoint{155.250229pt}{337.831604pt}}
\pgfusepath{stroke}
\pgfpathmoveto{\pgfpoint{155.231979pt}{344.008423pt}}
\pgflineto{\pgfpoint{155.254730pt}{344.008423pt}}
\pgfusepath{stroke}
\pgfpathmoveto{\pgfpoint{155.231979pt}{350.185242pt}}
\pgflineto{\pgfpoint{155.254730pt}{350.185242pt}}
\pgfusepath{stroke}
\pgfpathmoveto{\pgfpoint{155.231979pt}{356.362122pt}}
\pgflineto{\pgfpoint{155.259262pt}{356.362122pt}}
\pgfusepath{stroke}
\pgfpathmoveto{\pgfpoint{155.231979pt}{362.538940pt}}
\pgflineto{\pgfpoint{155.259277pt}{362.538940pt}}
\pgfusepath{stroke}
\pgfpathmoveto{\pgfpoint{155.231979pt}{368.715820pt}}
\pgflineto{\pgfpoint{155.259323pt}{368.715820pt}}
\pgfusepath{stroke}
\pgfpathmoveto{\pgfpoint{155.231979pt}{374.892639pt}}
\pgflineto{\pgfpoint{155.263824pt}{374.892639pt}}
\pgfusepath{stroke}
\pgfpathmoveto{\pgfpoint{155.231979pt}{381.069458pt}}
\pgflineto{\pgfpoint{155.263824pt}{381.069458pt}}
\pgfusepath{stroke}
\pgfpathmoveto{\pgfpoint{155.231979pt}{387.246338pt}}
\pgflineto{\pgfpoint{155.268372pt}{387.246338pt}}
\pgfusepath{stroke}
\pgfpathmoveto{\pgfpoint{155.231979pt}{393.423157pt}}
\pgflineto{\pgfpoint{155.268372pt}{393.423157pt}}
\pgfusepath{stroke}
\pgfpathmoveto{\pgfpoint{155.231979pt}{257.532623pt}}
\pgflineto{\pgfpoint{155.231979pt}{251.355804pt}}
\pgfusepath{stroke}
\pgfpathmoveto{\pgfpoint{155.231979pt}{226.648422pt}}
\pgflineto{\pgfpoint{155.231979pt}{220.471588pt}}
\pgfusepath{stroke}
\pgfpathmoveto{\pgfpoint{155.231979pt}{232.825272pt}}
\pgflineto{\pgfpoint{155.231979pt}{226.648422pt}}
\pgfusepath{stroke}
\pgfpathmoveto{\pgfpoint{155.231979pt}{239.002106pt}}
\pgflineto{\pgfpoint{155.231979pt}{232.825272pt}}
\pgfusepath{stroke}
\pgfpathmoveto{\pgfpoint{155.231979pt}{245.178955pt}}
\pgflineto{\pgfpoint{155.231979pt}{239.002106pt}}
\pgfusepath{stroke}
\pgfpathmoveto{\pgfpoint{155.231979pt}{251.355804pt}}
\pgflineto{\pgfpoint{155.231979pt}{245.178955pt}}
\pgfusepath{stroke}
\pgfpathmoveto{\pgfpoint{155.231979pt}{263.709473pt}}
\pgflineto{\pgfpoint{155.231979pt}{257.532623pt}}
\pgfusepath{stroke}
\pgfpathmoveto{\pgfpoint{155.231979pt}{269.886322pt}}
\pgflineto{\pgfpoint{155.231979pt}{263.709473pt}}
\pgfusepath{stroke}
\pgfpathmoveto{\pgfpoint{155.231979pt}{276.063141pt}}
\pgflineto{\pgfpoint{155.231979pt}{269.886322pt}}
\pgfusepath{stroke}
\pgfpathmoveto{\pgfpoint{155.231979pt}{282.239990pt}}
\pgflineto{\pgfpoint{155.231979pt}{276.063141pt}}
\pgfusepath{stroke}
\pgfpathmoveto{\pgfpoint{155.231979pt}{288.416840pt}}
\pgflineto{\pgfpoint{155.231979pt}{282.239990pt}}
\pgfusepath{stroke}
\pgfpathmoveto{\pgfpoint{155.231979pt}{294.593689pt}}
\pgflineto{\pgfpoint{155.231979pt}{288.416840pt}}
\pgfusepath{stroke}
\pgfpathmoveto{\pgfpoint{155.231979pt}{300.770538pt}}
\pgflineto{\pgfpoint{155.231979pt}{294.593689pt}}
\pgfusepath{stroke}
\pgfpathmoveto{\pgfpoint{155.231979pt}{201.941055pt}}
\pgflineto{\pgfpoint{155.213898pt}{201.941055pt}}
\pgfusepath{stroke}
\pgfpathmoveto{\pgfpoint{155.231979pt}{208.117905pt}}
\pgflineto{\pgfpoint{155.204880pt}{208.117905pt}}
\pgfusepath{stroke}
\pgfpathmoveto{\pgfpoint{155.231979pt}{214.294739pt}}
\pgflineto{\pgfpoint{155.213821pt}{214.294739pt}}
\pgfusepath{stroke}
\pgfpathmoveto{\pgfpoint{155.231979pt}{201.941055pt}}
\pgflineto{\pgfpoint{155.231979pt}{195.764206pt}}
\pgfusepath{stroke}
\pgfpathmoveto{\pgfpoint{155.231979pt}{208.117905pt}}
\pgflineto{\pgfpoint{155.231979pt}{201.941055pt}}
\pgfusepath{stroke}
\pgfpathmoveto{\pgfpoint{155.231979pt}{183.410522pt}}
\pgflineto{\pgfpoint{155.249893pt}{183.410522pt}}
\pgfusepath{stroke}
\pgfpathmoveto{\pgfpoint{155.231979pt}{189.587372pt}}
\pgflineto{\pgfpoint{155.258911pt}{189.587372pt}}
\pgfusepath{stroke}
\pgfpathmoveto{\pgfpoint{155.231979pt}{195.764206pt}}
\pgflineto{\pgfpoint{155.258911pt}{195.764206pt}}
\pgfusepath{stroke}
\pgfpathmoveto{\pgfpoint{155.231979pt}{201.941055pt}}
\pgflineto{\pgfpoint{155.258942pt}{201.941055pt}}
\pgfusepath{stroke}
\pgfpathmoveto{\pgfpoint{155.231979pt}{220.471588pt}}
\pgflineto{\pgfpoint{155.231979pt}{214.294739pt}}
\pgfusepath{stroke}
\pgfpathmoveto{\pgfpoint{155.231979pt}{214.294739pt}}
\pgflineto{\pgfpoint{155.231979pt}{208.117905pt}}
\pgfusepath{stroke}
\pgfpathmoveto{\pgfpoint{155.231979pt}{208.117905pt}}
\pgflineto{\pgfpoint{155.258881pt}{208.117905pt}}
\pgfusepath{stroke}
\pgfpathmoveto{\pgfpoint{155.231979pt}{214.294739pt}}
\pgflineto{\pgfpoint{155.258835pt}{214.294739pt}}
\pgfusepath{stroke}
\pgfpathmoveto{\pgfpoint{155.231979pt}{220.471588pt}}
\pgflineto{\pgfpoint{155.267899pt}{220.471588pt}}
\pgfusepath{stroke}
\pgfpathmoveto{\pgfpoint{155.231979pt}{226.648422pt}}
\pgflineto{\pgfpoint{155.267838pt}{226.648422pt}}
\pgfusepath{stroke}
\pgfpathmoveto{\pgfpoint{155.231979pt}{232.825272pt}}
\pgflineto{\pgfpoint{155.267868pt}{232.825272pt}}
\pgfusepath{stroke}
\pgfpathmoveto{\pgfpoint{155.231979pt}{239.002106pt}}
\pgflineto{\pgfpoint{155.267822pt}{239.002106pt}}
\pgfusepath{stroke}
\pgfpathmoveto{\pgfpoint{155.231979pt}{245.178955pt}}
\pgflineto{\pgfpoint{155.267822pt}{245.178955pt}}
\pgfusepath{stroke}
\pgfpathmoveto{\pgfpoint{155.231979pt}{251.355804pt}}
\pgflineto{\pgfpoint{155.276810pt}{251.355804pt}}
\pgfusepath{stroke}
\pgfpathmoveto{\pgfpoint{155.231979pt}{257.532623pt}}
\pgflineto{\pgfpoint{155.276810pt}{257.532623pt}}
\pgfusepath{stroke}
\pgfpathmoveto{\pgfpoint{155.231979pt}{263.709473pt}}
\pgflineto{\pgfpoint{155.276810pt}{263.709473pt}}
\pgfusepath{stroke}
\pgfpathmoveto{\pgfpoint{155.231979pt}{269.886322pt}}
\pgflineto{\pgfpoint{155.276810pt}{269.886322pt}}
\pgfusepath{stroke}
\pgfpathmoveto{\pgfpoint{155.231979pt}{276.063141pt}}
\pgflineto{\pgfpoint{155.285736pt}{276.063141pt}}
\pgfusepath{stroke}
\pgfpathmoveto{\pgfpoint{155.231979pt}{282.239990pt}}
\pgflineto{\pgfpoint{155.285736pt}{282.239990pt}}
\pgfusepath{stroke}
\pgfpathmoveto{\pgfpoint{155.231979pt}{288.416840pt}}
\pgflineto{\pgfpoint{155.285797pt}{288.416840pt}}
\pgfusepath{stroke}
\pgfpathmoveto{\pgfpoint{155.231979pt}{294.593689pt}}
\pgflineto{\pgfpoint{155.285797pt}{294.593689pt}}
\pgfusepath{stroke}
\pgfpathmoveto{\pgfpoint{155.231979pt}{300.770538pt}}
\pgflineto{\pgfpoint{155.285797pt}{300.770538pt}}
\pgfusepath{stroke}
\pgfpathmoveto{\pgfpoint{155.294739pt}{306.947388pt}}
\pgflineto{\pgfpoint{155.250092pt}{306.947388pt}}
\pgfusepath{stroke}
\pgfpathmoveto{\pgfpoint{155.290253pt}{313.124207pt}}
\pgflineto{\pgfpoint{155.245621pt}{313.124207pt}}
\pgfusepath{stroke}
\pgfpathmoveto{\pgfpoint{155.290192pt}{319.301056pt}}
\pgflineto{\pgfpoint{155.245621pt}{319.301056pt}}
\pgfusepath{stroke}
\pgfpathmoveto{\pgfpoint{155.290283pt}{325.477905pt}}
\pgflineto{\pgfpoint{155.245636pt}{325.477905pt}}
\pgfusepath{stroke}
\pgfpathmoveto{\pgfpoint{155.294693pt}{331.654724pt}}
\pgflineto{\pgfpoint{155.250122pt}{331.654724pt}}
\pgfusepath{stroke}
\pgfpathmoveto{\pgfpoint{155.294754pt}{337.831604pt}}
\pgflineto{\pgfpoint{155.250229pt}{337.831604pt}}
\pgfusepath{stroke}
\pgfpathmoveto{\pgfpoint{155.299179pt}{344.008423pt}}
\pgflineto{\pgfpoint{155.254730pt}{344.008423pt}}
\pgfusepath{stroke}
\pgfpathmoveto{\pgfpoint{155.299179pt}{350.185242pt}}
\pgflineto{\pgfpoint{155.254730pt}{350.185242pt}}
\pgfusepath{stroke}
\pgfpathmoveto{\pgfpoint{155.303665pt}{356.362122pt}}
\pgflineto{\pgfpoint{155.259262pt}{356.362122pt}}
\pgfusepath{stroke}
\pgfpathmoveto{\pgfpoint{155.303665pt}{362.538940pt}}
\pgflineto{\pgfpoint{155.259277pt}{362.538940pt}}
\pgfusepath{stroke}
\pgfpathmoveto{\pgfpoint{155.303665pt}{368.715820pt}}
\pgflineto{\pgfpoint{155.259323pt}{368.715820pt}}
\pgfusepath{stroke}
\pgfpathmoveto{\pgfpoint{155.308167pt}{374.892639pt}}
\pgflineto{\pgfpoint{155.263824pt}{374.892639pt}}
\pgfusepath{stroke}
\pgfpathmoveto{\pgfpoint{155.308105pt}{381.069458pt}}
\pgflineto{\pgfpoint{155.263824pt}{381.069458pt}}
\pgfusepath{stroke}
\pgfpathmoveto{\pgfpoint{155.312653pt}{387.246338pt}}
\pgflineto{\pgfpoint{155.268372pt}{387.246338pt}}
\pgfusepath{stroke}
\pgfpathmoveto{\pgfpoint{155.312592pt}{393.423157pt}}
\pgflineto{\pgfpoint{155.268372pt}{393.423157pt}}
\pgfusepath{stroke}
\pgfpathmoveto{\pgfpoint{155.231979pt}{109.288422pt}}
\pgflineto{\pgfpoint{155.231979pt}{103.111580pt}}
\pgfusepath{stroke}
\pgfpathmoveto{\pgfpoint{155.231979pt}{78.404205pt}}
\pgflineto{\pgfpoint{155.231979pt}{72.227356pt}}
\pgfusepath{stroke}
\pgfpathmoveto{\pgfpoint{155.231979pt}{84.581039pt}}
\pgflineto{\pgfpoint{155.231979pt}{78.404205pt}}
\pgfusepath{stroke}
\pgfpathmoveto{\pgfpoint{155.231979pt}{90.757896pt}}
\pgflineto{\pgfpoint{155.231979pt}{84.581039pt}}
\pgfusepath{stroke}
\pgfpathmoveto{\pgfpoint{155.231979pt}{96.934731pt}}
\pgflineto{\pgfpoint{155.231979pt}{90.757896pt}}
\pgfusepath{stroke}
\pgfpathmoveto{\pgfpoint{155.231979pt}{103.111580pt}}
\pgflineto{\pgfpoint{155.231979pt}{96.934731pt}}
\pgfusepath{stroke}
\pgfpathmoveto{\pgfpoint{155.231979pt}{115.465263pt}}
\pgflineto{\pgfpoint{155.231979pt}{109.288422pt}}
\pgfusepath{stroke}
\pgfpathmoveto{\pgfpoint{155.231979pt}{121.642097pt}}
\pgflineto{\pgfpoint{155.231979pt}{115.465263pt}}
\pgfusepath{stroke}
\pgfpathmoveto{\pgfpoint{155.231979pt}{127.818947pt}}
\pgflineto{\pgfpoint{155.231979pt}{121.642097pt}}
\pgfusepath{stroke}
\pgfpathmoveto{\pgfpoint{155.231979pt}{133.995789pt}}
\pgflineto{\pgfpoint{155.231979pt}{127.818947pt}}
\pgfusepath{stroke}
\pgfpathmoveto{\pgfpoint{155.231979pt}{140.172638pt}}
\pgflineto{\pgfpoint{155.231979pt}{133.995789pt}}
\pgfusepath{stroke}
\pgfpathmoveto{\pgfpoint{155.231979pt}{146.349472pt}}
\pgflineto{\pgfpoint{155.231979pt}{140.172638pt}}
\pgfusepath{stroke}
\pgfpathmoveto{\pgfpoint{155.231979pt}{152.526306pt}}
\pgflineto{\pgfpoint{155.231979pt}{146.349472pt}}
\pgfusepath{stroke}
\pgfpathmoveto{\pgfpoint{155.231979pt}{47.519989pt}}
\pgflineto{\pgfpoint{155.214081pt}{47.519989pt}}
\pgfusepath{stroke}
\pgfpathmoveto{\pgfpoint{155.231979pt}{53.696838pt}}
\pgflineto{\pgfpoint{155.214020pt}{53.696838pt}}
\pgfusepath{stroke}
\pgfpathmoveto{\pgfpoint{155.231979pt}{59.873672pt}}
\pgflineto{\pgfpoint{155.214050pt}{59.873672pt}}
\pgfusepath{stroke}
\pgfpathmoveto{\pgfpoint{155.231979pt}{66.050522pt}}
\pgflineto{\pgfpoint{155.214050pt}{66.050522pt}}
\pgfusepath{stroke}
\pgfpathmoveto{\pgfpoint{155.231979pt}{53.696838pt}}
\pgflineto{\pgfpoint{155.231979pt}{47.519989pt}}
\pgfusepath{stroke}
\pgfpathmoveto{\pgfpoint{155.231979pt}{59.873672pt}}
\pgflineto{\pgfpoint{155.231979pt}{53.696838pt}}
\pgfusepath{stroke}
\pgfpathmoveto{\pgfpoint{155.231979pt}{47.519989pt}}
\pgflineto{\pgfpoint{155.259018pt}{47.519989pt}}
\pgfusepath{stroke}
\pgfpathmoveto{\pgfpoint{155.231979pt}{53.696838pt}}
\pgflineto{\pgfpoint{155.250000pt}{53.696838pt}}
\pgfusepath{stroke}
\pgfpathmoveto{\pgfpoint{155.231979pt}{72.227356pt}}
\pgflineto{\pgfpoint{155.231979pt}{66.050522pt}}
\pgfusepath{stroke}
\pgfpathmoveto{\pgfpoint{155.231979pt}{66.050522pt}}
\pgflineto{\pgfpoint{155.231979pt}{59.873672pt}}
\pgfusepath{stroke}
\pgfpathmoveto{\pgfpoint{155.259018pt}{47.519989pt}}
\pgflineto{\pgfpoint{164.142181pt}{47.519989pt}}
\pgfusepath{stroke}
\pgfpathmoveto{\pgfpoint{164.160004pt}{53.696838pt}}
\pgflineto{\pgfpoint{155.250000pt}{53.696838pt}}
\pgfusepath{stroke}
\pgfpathmoveto{\pgfpoint{164.160004pt}{59.873672pt}}
\pgflineto{\pgfpoint{155.231979pt}{59.873672pt}}
\pgfusepath{stroke}
\pgfpathmoveto{\pgfpoint{164.160004pt}{66.050522pt}}
\pgflineto{\pgfpoint{155.231979pt}{66.050522pt}}
\pgfusepath{stroke}
\pgfpathmoveto{\pgfpoint{164.160004pt}{72.227356pt}}
\pgflineto{\pgfpoint{155.231979pt}{72.227356pt}}
\pgfusepath{stroke}
\pgfpathmoveto{\pgfpoint{164.160004pt}{78.404205pt}}
\pgflineto{\pgfpoint{155.231979pt}{78.404205pt}}
\pgfusepath{stroke}
\pgfpathmoveto{\pgfpoint{164.160004pt}{84.581039pt}}
\pgflineto{\pgfpoint{155.231979pt}{84.581039pt}}
\pgfusepath{stroke}
\pgfpathmoveto{\pgfpoint{164.160004pt}{90.757896pt}}
\pgflineto{\pgfpoint{155.231979pt}{90.757896pt}}
\pgfusepath{stroke}
\pgfpathmoveto{\pgfpoint{164.160004pt}{96.934731pt}}
\pgflineto{\pgfpoint{155.231979pt}{96.934731pt}}
\pgfusepath{stroke}
\pgfpathmoveto{\pgfpoint{164.160004pt}{103.111580pt}}
\pgflineto{\pgfpoint{155.231979pt}{103.111580pt}}
\pgfusepath{stroke}
\pgfpathmoveto{\pgfpoint{164.160004pt}{109.288422pt}}
\pgflineto{\pgfpoint{155.231979pt}{109.288422pt}}
\pgfusepath{stroke}
\pgfpathmoveto{\pgfpoint{164.160004pt}{115.465263pt}}
\pgflineto{\pgfpoint{155.231979pt}{115.465263pt}}
\pgfusepath{stroke}
\pgfpathmoveto{\pgfpoint{164.160004pt}{121.642097pt}}
\pgflineto{\pgfpoint{155.231979pt}{121.642097pt}}
\pgfusepath{stroke}
\pgfpathmoveto{\pgfpoint{164.160004pt}{127.818947pt}}
\pgflineto{\pgfpoint{155.231979pt}{127.818947pt}}
\pgfusepath{stroke}
\pgfpathmoveto{\pgfpoint{164.160004pt}{133.995789pt}}
\pgflineto{\pgfpoint{155.231979pt}{133.995789pt}}
\pgfusepath{stroke}
\pgfpathmoveto{\pgfpoint{164.160004pt}{140.172638pt}}
\pgflineto{\pgfpoint{155.231979pt}{140.172638pt}}
\pgfusepath{stroke}
\pgfpathmoveto{\pgfpoint{164.160004pt}{146.349472pt}}
\pgflineto{\pgfpoint{155.231979pt}{146.349472pt}}
\pgfusepath{stroke}
\pgfpathmoveto{\pgfpoint{164.160004pt}{152.526306pt}}
\pgflineto{\pgfpoint{155.231979pt}{152.526306pt}}
\pgfusepath{stroke}
\pgfpathmoveto{\pgfpoint{164.160004pt}{158.703156pt}}
\pgflineto{\pgfpoint{155.249939pt}{158.703156pt}}
\pgfusepath{stroke}
\pgfpathmoveto{\pgfpoint{164.160004pt}{164.880005pt}}
\pgflineto{\pgfpoint{155.249939pt}{164.880005pt}}
\pgfusepath{stroke}
\pgfpathmoveto{\pgfpoint{164.160004pt}{171.056854pt}}
\pgflineto{\pgfpoint{155.249954pt}{171.056854pt}}
\pgfusepath{stroke}
\pgfpathmoveto{\pgfpoint{164.160004pt}{177.233673pt}}
\pgflineto{\pgfpoint{155.249893pt}{177.233673pt}}
\pgfusepath{stroke}
\pgfpathmoveto{\pgfpoint{164.160004pt}{183.410522pt}}
\pgflineto{\pgfpoint{155.249893pt}{183.410522pt}}
\pgfusepath{stroke}
\pgfpathmoveto{\pgfpoint{164.160004pt}{189.587372pt}}
\pgflineto{\pgfpoint{155.258911pt}{189.587372pt}}
\pgfusepath{stroke}
\pgfpathmoveto{\pgfpoint{164.160004pt}{195.764206pt}}
\pgflineto{\pgfpoint{155.258911pt}{195.764206pt}}
\pgfusepath{stroke}
\pgfpathmoveto{\pgfpoint{164.160004pt}{201.941055pt}}
\pgflineto{\pgfpoint{155.258942pt}{201.941055pt}}
\pgfusepath{stroke}
\pgfpathmoveto{\pgfpoint{164.160004pt}{208.117905pt}}
\pgflineto{\pgfpoint{155.258881pt}{208.117905pt}}
\pgfusepath{stroke}
\pgfpathmoveto{\pgfpoint{155.258835pt}{214.294739pt}}
\pgflineto{\pgfpoint{164.141922pt}{214.294739pt}}
\pgfusepath{stroke}
\pgfpathmoveto{\pgfpoint{155.267899pt}{220.471588pt}}
\pgflineto{\pgfpoint{164.132919pt}{220.471588pt}}
\pgfusepath{stroke}
\pgfpathmoveto{\pgfpoint{155.267838pt}{226.648422pt}}
\pgflineto{\pgfpoint{164.123840pt}{226.648422pt}}
\pgfusepath{stroke}
\pgfpathmoveto{\pgfpoint{155.267868pt}{232.825272pt}}
\pgflineto{\pgfpoint{164.123795pt}{232.825272pt}}
\pgfusepath{stroke}
\pgfpathmoveto{\pgfpoint{155.267822pt}{239.002106pt}}
\pgflineto{\pgfpoint{164.114731pt}{239.002106pt}}
\pgfusepath{stroke}
\pgfpathmoveto{\pgfpoint{155.267822pt}{245.178955pt}}
\pgflineto{\pgfpoint{164.105713pt}{245.178955pt}}
\pgfusepath{stroke}
\pgfpathmoveto{\pgfpoint{155.276810pt}{251.355804pt}}
\pgflineto{\pgfpoint{164.096741pt}{251.355804pt}}
\pgfusepath{stroke}
\pgfpathmoveto{\pgfpoint{155.276810pt}{257.532623pt}}
\pgflineto{\pgfpoint{164.087646pt}{257.532623pt}}
\pgfusepath{stroke}
\pgfpathmoveto{\pgfpoint{155.276810pt}{263.709473pt}}
\pgflineto{\pgfpoint{164.087646pt}{263.709473pt}}
\pgfusepath{stroke}
\pgfpathmoveto{\pgfpoint{155.276810pt}{269.886322pt}}
\pgflineto{\pgfpoint{164.078613pt}{269.886322pt}}
\pgfusepath{stroke}
\pgfpathmoveto{\pgfpoint{155.285736pt}{276.063141pt}}
\pgflineto{\pgfpoint{164.069595pt}{276.063141pt}}
\pgfusepath{stroke}
\pgfpathmoveto{\pgfpoint{155.285736pt}{282.239990pt}}
\pgflineto{\pgfpoint{164.060547pt}{282.239990pt}}
\pgfusepath{stroke}
\pgfpathmoveto{\pgfpoint{155.285797pt}{288.416840pt}}
\pgflineto{\pgfpoint{164.060455pt}{288.416840pt}}
\pgfusepath{stroke}
\pgfpathmoveto{\pgfpoint{155.285797pt}{294.593689pt}}
\pgflineto{\pgfpoint{164.051407pt}{294.593689pt}}
\pgfusepath{stroke}
\pgfpathmoveto{\pgfpoint{155.285797pt}{300.770538pt}}
\pgflineto{\pgfpoint{164.042419pt}{300.770538pt}}
\pgfusepath{stroke}
\pgfpathmoveto{\pgfpoint{155.294739pt}{306.947388pt}}
\pgflineto{\pgfpoint{164.029022pt}{306.947388pt}}
\pgfusepath{stroke}
\pgfpathmoveto{\pgfpoint{155.290253pt}{313.124207pt}}
\pgflineto{\pgfpoint{164.024384pt}{313.124207pt}}
\pgfusepath{stroke}
\pgfpathmoveto{\pgfpoint{155.290192pt}{319.301056pt}}
\pgflineto{\pgfpoint{164.015350pt}{319.301056pt}}
\pgfusepath{stroke}
\pgfpathmoveto{\pgfpoint{155.290283pt}{325.477905pt}}
\pgflineto{\pgfpoint{164.010757pt}{325.477905pt}}
\pgfusepath{stroke}
\pgfpathmoveto{\pgfpoint{155.294693pt}{331.654724pt}}
\pgflineto{\pgfpoint{164.001770pt}{331.654724pt}}
\pgfusepath{stroke}
\pgfpathmoveto{\pgfpoint{155.294754pt}{337.831604pt}}
\pgflineto{\pgfpoint{163.992737pt}{337.831604pt}}
\pgfusepath{stroke}
\pgfpathmoveto{\pgfpoint{155.299179pt}{344.008423pt}}
\pgflineto{\pgfpoint{163.988190pt}{344.008423pt}}
\pgfusepath{stroke}
\pgfpathmoveto{\pgfpoint{155.299179pt}{350.185242pt}}
\pgflineto{\pgfpoint{163.979156pt}{350.185242pt}}
\pgfusepath{stroke}
\pgfpathmoveto{\pgfpoint{155.303665pt}{356.362122pt}}
\pgflineto{\pgfpoint{163.974670pt}{356.362122pt}}
\pgfusepath{stroke}
\pgfpathmoveto{\pgfpoint{155.303665pt}{362.538940pt}}
\pgflineto{\pgfpoint{163.965576pt}{362.538940pt}}
\pgfusepath{stroke}
\pgfpathmoveto{\pgfpoint{155.303665pt}{368.715820pt}}
\pgflineto{\pgfpoint{163.956604pt}{368.715820pt}}
\pgfusepath{stroke}
\pgfpathmoveto{\pgfpoint{155.308167pt}{374.892639pt}}
\pgflineto{\pgfpoint{163.952057pt}{374.892639pt}}
\pgfusepath{stroke}
\pgfpathmoveto{\pgfpoint{155.308105pt}{381.069458pt}}
\pgflineto{\pgfpoint{163.942947pt}{381.069458pt}}
\pgfusepath{stroke}
\pgfpathmoveto{\pgfpoint{155.312653pt}{387.246338pt}}
\pgflineto{\pgfpoint{163.938477pt}{387.246338pt}}
\pgfusepath{stroke}
\pgfpathmoveto{\pgfpoint{155.312592pt}{393.423157pt}}
\pgflineto{\pgfpoint{163.929428pt}{393.423157pt}}
\pgfusepath{stroke}
\pgfpathmoveto{\pgfpoint{164.160004pt}{152.526306pt}}
\pgflineto{\pgfpoint{164.160004pt}{146.349472pt}}
\pgfusepath{stroke}
\pgfpathmoveto{\pgfpoint{164.160004pt}{146.349472pt}}
\pgflineto{\pgfpoint{164.160004pt}{140.172638pt}}
\pgfusepath{stroke}
\pgfpathmoveto{\pgfpoint{164.160004pt}{158.703156pt}}
\pgflineto{\pgfpoint{164.160004pt}{152.526306pt}}
\pgfusepath{stroke}
\pgfpathmoveto{\pgfpoint{164.160004pt}{164.880005pt}}
\pgflineto{\pgfpoint{164.160004pt}{158.703156pt}}
\pgfusepath{stroke}
\pgfpathmoveto{\pgfpoint{164.160004pt}{171.056854pt}}
\pgflineto{\pgfpoint{164.160004pt}{164.880005pt}}
\pgfusepath{stroke}
\pgfpathmoveto{\pgfpoint{164.160004pt}{177.233673pt}}
\pgflineto{\pgfpoint{164.160004pt}{171.056854pt}}
\pgfusepath{stroke}
\pgfpathmoveto{\pgfpoint{164.160004pt}{183.410522pt}}
\pgflineto{\pgfpoint{164.160004pt}{177.233673pt}}
\pgfusepath{stroke}
\pgfpathmoveto{\pgfpoint{164.160004pt}{189.587372pt}}
\pgflineto{\pgfpoint{164.160004pt}{183.410522pt}}
\pgfusepath{stroke}
\pgfpathmoveto{\pgfpoint{164.160004pt}{146.349472pt}}
\pgflineto{\pgfpoint{164.177795pt}{146.349472pt}}
\pgfusepath{stroke}
\pgfpathmoveto{\pgfpoint{164.160004pt}{152.526306pt}}
\pgflineto{\pgfpoint{164.177734pt}{152.526306pt}}
\pgfusepath{stroke}
\pgfpathmoveto{\pgfpoint{164.160004pt}{158.703156pt}}
\pgflineto{\pgfpoint{164.177750pt}{158.703156pt}}
\pgfusepath{stroke}
\pgfpathmoveto{\pgfpoint{164.160004pt}{164.880005pt}}
\pgflineto{\pgfpoint{164.177689pt}{164.880005pt}}
\pgfusepath{stroke}
\pgfpathmoveto{\pgfpoint{164.160004pt}{171.056854pt}}
\pgflineto{\pgfpoint{164.186707pt}{171.056854pt}}
\pgfusepath{stroke}
\pgfpathmoveto{\pgfpoint{164.160004pt}{177.233673pt}}
\pgflineto{\pgfpoint{164.186600pt}{177.233673pt}}
\pgfusepath{stroke}
\pgfpathmoveto{\pgfpoint{164.160004pt}{183.410522pt}}
\pgflineto{\pgfpoint{164.186615pt}{183.410522pt}}
\pgfusepath{stroke}
\pgfpathmoveto{\pgfpoint{164.160004pt}{201.941055pt}}
\pgflineto{\pgfpoint{164.160004pt}{195.764206pt}}
\pgfusepath{stroke}
\pgfpathmoveto{\pgfpoint{164.160004pt}{195.764206pt}}
\pgflineto{\pgfpoint{164.160004pt}{189.587372pt}}
\pgfusepath{stroke}
\pgfpathmoveto{\pgfpoint{164.160004pt}{208.117905pt}}
\pgflineto{\pgfpoint{164.160004pt}{201.941055pt}}
\pgfusepath{stroke}
\pgfpathmoveto{\pgfpoint{164.123840pt}{226.648422pt}}
\pgflineto{\pgfpoint{164.141907pt}{226.648422pt}}
\pgfusepath{stroke}
\pgfpathmoveto{\pgfpoint{164.160004pt}{232.825272pt}}
\pgflineto{\pgfpoint{164.123795pt}{232.825272pt}}
\pgfusepath{stroke}
\pgfpathmoveto{\pgfpoint{164.160004pt}{239.002106pt}}
\pgflineto{\pgfpoint{164.114731pt}{239.002106pt}}
\pgfusepath{stroke}
\pgfpathmoveto{\pgfpoint{164.160004pt}{245.178955pt}}
\pgflineto{\pgfpoint{164.105713pt}{245.178955pt}}
\pgfusepath{stroke}
\pgfpathmoveto{\pgfpoint{164.160004pt}{251.355804pt}}
\pgflineto{\pgfpoint{164.096741pt}{251.355804pt}}
\pgfusepath{stroke}
\pgfpathmoveto{\pgfpoint{164.160004pt}{257.532623pt}}
\pgflineto{\pgfpoint{164.087646pt}{257.532623pt}}
\pgfusepath{stroke}
\pgfpathmoveto{\pgfpoint{164.160004pt}{263.709473pt}}
\pgflineto{\pgfpoint{164.087646pt}{263.709473pt}}
\pgfusepath{stroke}
\pgfpathmoveto{\pgfpoint{164.160004pt}{269.886322pt}}
\pgflineto{\pgfpoint{164.078613pt}{269.886322pt}}
\pgfusepath{stroke}
\pgfpathmoveto{\pgfpoint{164.160004pt}{276.063141pt}}
\pgflineto{\pgfpoint{164.069595pt}{276.063141pt}}
\pgfusepath{stroke}
\pgfpathmoveto{\pgfpoint{164.160004pt}{282.239990pt}}
\pgflineto{\pgfpoint{164.060547pt}{282.239990pt}}
\pgfusepath{stroke}
\pgfpathmoveto{\pgfpoint{164.160004pt}{288.416840pt}}
\pgflineto{\pgfpoint{164.060455pt}{288.416840pt}}
\pgfusepath{stroke}
\pgfpathmoveto{\pgfpoint{164.160004pt}{294.593689pt}}
\pgflineto{\pgfpoint{164.051407pt}{294.593689pt}}
\pgfusepath{stroke}
\pgfpathmoveto{\pgfpoint{164.160004pt}{300.770538pt}}
\pgflineto{\pgfpoint{164.042419pt}{300.770538pt}}
\pgfusepath{stroke}
\pgfpathmoveto{\pgfpoint{164.160004pt}{306.947388pt}}
\pgflineto{\pgfpoint{164.029022pt}{306.947388pt}}
\pgfusepath{stroke}
\pgfpathmoveto{\pgfpoint{164.160004pt}{313.124207pt}}
\pgflineto{\pgfpoint{164.024384pt}{313.124207pt}}
\pgfusepath{stroke}
\pgfpathmoveto{\pgfpoint{164.160004pt}{319.301056pt}}
\pgflineto{\pgfpoint{164.015350pt}{319.301056pt}}
\pgfusepath{stroke}
\pgfpathmoveto{\pgfpoint{164.160004pt}{325.477905pt}}
\pgflineto{\pgfpoint{164.010757pt}{325.477905pt}}
\pgfusepath{stroke}
\pgfpathmoveto{\pgfpoint{164.160004pt}{331.654724pt}}
\pgflineto{\pgfpoint{164.001770pt}{331.654724pt}}
\pgfusepath{stroke}
\pgfpathmoveto{\pgfpoint{164.160004pt}{337.831604pt}}
\pgflineto{\pgfpoint{163.992737pt}{337.831604pt}}
\pgfusepath{stroke}
\pgfpathmoveto{\pgfpoint{164.160004pt}{344.008423pt}}
\pgflineto{\pgfpoint{163.988190pt}{344.008423pt}}
\pgfusepath{stroke}
\pgfpathmoveto{\pgfpoint{164.160004pt}{350.185242pt}}
\pgflineto{\pgfpoint{163.979156pt}{350.185242pt}}
\pgfusepath{stroke}
\pgfpathmoveto{\pgfpoint{164.160004pt}{356.362122pt}}
\pgflineto{\pgfpoint{163.974670pt}{356.362122pt}}
\pgfusepath{stroke}
\pgfpathmoveto{\pgfpoint{164.160004pt}{362.538940pt}}
\pgflineto{\pgfpoint{163.965576pt}{362.538940pt}}
\pgfusepath{stroke}
\pgfpathmoveto{\pgfpoint{164.160004pt}{368.715820pt}}
\pgflineto{\pgfpoint{163.956604pt}{368.715820pt}}
\pgfusepath{stroke}
\pgfpathmoveto{\pgfpoint{164.160004pt}{374.892639pt}}
\pgflineto{\pgfpoint{163.952057pt}{374.892639pt}}
\pgfusepath{stroke}
\pgfpathmoveto{\pgfpoint{164.160004pt}{381.069458pt}}
\pgflineto{\pgfpoint{163.942947pt}{381.069458pt}}
\pgfusepath{stroke}
\pgfpathmoveto{\pgfpoint{164.160004pt}{387.246338pt}}
\pgflineto{\pgfpoint{163.938477pt}{387.246338pt}}
\pgfusepath{stroke}
\pgfpathmoveto{\pgfpoint{164.160004pt}{393.423157pt}}
\pgflineto{\pgfpoint{163.929428pt}{393.423157pt}}
\pgfusepath{stroke}
\pgfpathmoveto{\pgfpoint{164.160004pt}{331.654724pt}}
\pgflineto{\pgfpoint{164.160004pt}{325.477905pt}}
\pgfusepath{stroke}
\pgfpathmoveto{\pgfpoint{164.160004pt}{325.477905pt}}
\pgflineto{\pgfpoint{164.160004pt}{319.301056pt}}
\pgfusepath{stroke}
\pgfpathmoveto{\pgfpoint{164.160004pt}{337.831604pt}}
\pgflineto{\pgfpoint{164.160004pt}{331.654724pt}}
\pgfusepath{stroke}
\pgfpathmoveto{\pgfpoint{164.160004pt}{344.008423pt}}
\pgflineto{\pgfpoint{164.160004pt}{337.831604pt}}
\pgfusepath{stroke}
\pgfpathmoveto{\pgfpoint{164.160004pt}{350.185242pt}}
\pgflineto{\pgfpoint{164.160004pt}{344.008423pt}}
\pgfusepath{stroke}
\pgfpathmoveto{\pgfpoint{164.160004pt}{325.477905pt}}
\pgflineto{\pgfpoint{164.173538pt}{325.477905pt}}
\pgfusepath{stroke}
\pgfpathmoveto{\pgfpoint{164.160004pt}{331.654724pt}}
\pgflineto{\pgfpoint{164.173538pt}{331.654724pt}}
\pgfusepath{stroke}
\pgfpathmoveto{\pgfpoint{164.160004pt}{337.831604pt}}
\pgflineto{\pgfpoint{164.173599pt}{337.831604pt}}
\pgfusepath{stroke}
\pgfpathmoveto{\pgfpoint{164.160004pt}{344.008423pt}}
\pgflineto{\pgfpoint{164.178024pt}{344.008423pt}}
\pgfusepath{stroke}
\pgfpathmoveto{\pgfpoint{164.160004pt}{374.892639pt}}
\pgflineto{\pgfpoint{164.160004pt}{368.715820pt}}
\pgfusepath{stroke}
\pgfpathmoveto{\pgfpoint{164.160004pt}{356.362122pt}}
\pgflineto{\pgfpoint{164.160004pt}{350.185242pt}}
\pgfusepath{stroke}
\pgfpathmoveto{\pgfpoint{164.160004pt}{362.538940pt}}
\pgflineto{\pgfpoint{164.160004pt}{356.362122pt}}
\pgfusepath{stroke}
\pgfpathmoveto{\pgfpoint{164.160004pt}{368.715820pt}}
\pgflineto{\pgfpoint{164.160004pt}{362.538940pt}}
\pgfusepath{stroke}
\pgfpathmoveto{\pgfpoint{164.160004pt}{381.069458pt}}
\pgflineto{\pgfpoint{164.160004pt}{374.892639pt}}
\pgfusepath{stroke}
\pgfpathmoveto{\pgfpoint{164.160004pt}{387.246338pt}}
\pgflineto{\pgfpoint{164.160004pt}{381.069458pt}}
\pgfusepath{stroke}
\pgfpathmoveto{\pgfpoint{164.160004pt}{393.423157pt}}
\pgflineto{\pgfpoint{164.160004pt}{387.246338pt}}
\pgfusepath{stroke}
\pgfpathmoveto{\pgfpoint{164.160004pt}{350.185242pt}}
\pgflineto{\pgfpoint{164.178040pt}{350.185242pt}}
\pgfusepath{stroke}
\pgfpathmoveto{\pgfpoint{164.160004pt}{356.362122pt}}
\pgflineto{\pgfpoint{164.182632pt}{356.362122pt}}
\pgfusepath{stroke}
\pgfpathmoveto{\pgfpoint{164.160004pt}{362.538940pt}}
\pgflineto{\pgfpoint{164.182571pt}{362.538940pt}}
\pgfusepath{stroke}
\pgfpathmoveto{\pgfpoint{164.160004pt}{368.715820pt}}
\pgflineto{\pgfpoint{164.182587pt}{368.715820pt}}
\pgfusepath{stroke}
\pgfpathmoveto{\pgfpoint{164.160004pt}{374.892639pt}}
\pgflineto{\pgfpoint{164.187134pt}{374.892639pt}}
\pgfusepath{stroke}
\pgfpathmoveto{\pgfpoint{164.160004pt}{381.069458pt}}
\pgflineto{\pgfpoint{164.187073pt}{381.069458pt}}
\pgfusepath{stroke}
\pgfpathmoveto{\pgfpoint{164.160004pt}{387.246338pt}}
\pgflineto{\pgfpoint{164.191681pt}{387.246338pt}}
\pgfusepath{stroke}
\pgfpathmoveto{\pgfpoint{164.160004pt}{393.423157pt}}
\pgflineto{\pgfpoint{164.191620pt}{393.423157pt}}
\pgfusepath{stroke}
\pgfpathmoveto{\pgfpoint{164.160004pt}{269.886322pt}}
\pgflineto{\pgfpoint{164.160004pt}{263.709473pt}}
\pgfusepath{stroke}
\pgfpathmoveto{\pgfpoint{164.160004pt}{239.002106pt}}
\pgflineto{\pgfpoint{164.160004pt}{232.825272pt}}
\pgfusepath{stroke}
\pgfpathmoveto{\pgfpoint{164.160004pt}{245.178955pt}}
\pgflineto{\pgfpoint{164.160004pt}{239.002106pt}}
\pgfusepath{stroke}
\pgfpathmoveto{\pgfpoint{164.160004pt}{251.355804pt}}
\pgflineto{\pgfpoint{164.160004pt}{245.178955pt}}
\pgfusepath{stroke}
\pgfpathmoveto{\pgfpoint{164.160004pt}{257.532623pt}}
\pgflineto{\pgfpoint{164.160004pt}{251.355804pt}}
\pgfusepath{stroke}
\pgfpathmoveto{\pgfpoint{164.160004pt}{263.709473pt}}
\pgflineto{\pgfpoint{164.160004pt}{257.532623pt}}
\pgfusepath{stroke}
\pgfpathmoveto{\pgfpoint{164.160004pt}{276.063141pt}}
\pgflineto{\pgfpoint{164.160004pt}{269.886322pt}}
\pgfusepath{stroke}
\pgfpathmoveto{\pgfpoint{164.160004pt}{282.239990pt}}
\pgflineto{\pgfpoint{164.160004pt}{276.063141pt}}
\pgfusepath{stroke}
\pgfpathmoveto{\pgfpoint{164.160004pt}{288.416840pt}}
\pgflineto{\pgfpoint{164.160004pt}{282.239990pt}}
\pgfusepath{stroke}
\pgfpathmoveto{\pgfpoint{164.160004pt}{294.593689pt}}
\pgflineto{\pgfpoint{164.160004pt}{288.416840pt}}
\pgfusepath{stroke}
\pgfpathmoveto{\pgfpoint{164.160004pt}{300.770538pt}}
\pgflineto{\pgfpoint{164.160004pt}{294.593689pt}}
\pgfusepath{stroke}
\pgfpathmoveto{\pgfpoint{164.160004pt}{306.947388pt}}
\pgflineto{\pgfpoint{164.160004pt}{300.770538pt}}
\pgfusepath{stroke}
\pgfpathmoveto{\pgfpoint{164.160004pt}{313.124207pt}}
\pgflineto{\pgfpoint{164.160004pt}{306.947388pt}}
\pgfusepath{stroke}
\pgfpathmoveto{\pgfpoint{164.160004pt}{319.301056pt}}
\pgflineto{\pgfpoint{164.160004pt}{313.124207pt}}
\pgfusepath{stroke}
\pgfpathmoveto{\pgfpoint{164.160004pt}{189.587372pt}}
\pgflineto{\pgfpoint{164.186615pt}{189.587372pt}}
\pgfusepath{stroke}
\pgfpathmoveto{\pgfpoint{164.160004pt}{195.764206pt}}
\pgflineto{\pgfpoint{164.186569pt}{195.764206pt}}
\pgfusepath{stroke}
\pgfpathmoveto{\pgfpoint{164.160004pt}{201.941055pt}}
\pgflineto{\pgfpoint{164.195572pt}{201.941055pt}}
\pgfusepath{stroke}
\pgfpathmoveto{\pgfpoint{164.160004pt}{208.117905pt}}
\pgflineto{\pgfpoint{164.195511pt}{208.117905pt}}
\pgfusepath{stroke}
\pgfpathmoveto{\pgfpoint{164.160004pt}{214.294739pt}}
\pgflineto{\pgfpoint{164.160004pt}{208.117905pt}}
\pgfusepath{stroke}
\pgfpathmoveto{\pgfpoint{164.160004pt}{214.294739pt}}
\pgflineto{\pgfpoint{164.141922pt}{214.294739pt}}
\pgfusepath{stroke}
\pgfpathmoveto{\pgfpoint{164.160004pt}{214.294739pt}}
\pgflineto{\pgfpoint{164.195496pt}{214.294739pt}}
\pgfusepath{stroke}
\pgfpathmoveto{\pgfpoint{164.160004pt}{220.471588pt}}
\pgflineto{\pgfpoint{164.132919pt}{220.471588pt}}
\pgfusepath{stroke}
\pgfpathmoveto{\pgfpoint{164.160004pt}{226.648422pt}}
\pgflineto{\pgfpoint{164.141907pt}{226.648422pt}}
\pgfusepath{stroke}
\pgfpathmoveto{\pgfpoint{164.160004pt}{220.471588pt}}
\pgflineto{\pgfpoint{164.160004pt}{214.294739pt}}
\pgfusepath{stroke}
\pgfpathmoveto{\pgfpoint{164.160004pt}{232.825272pt}}
\pgflineto{\pgfpoint{164.160004pt}{226.648422pt}}
\pgfusepath{stroke}
\pgfpathmoveto{\pgfpoint{164.160004pt}{226.648422pt}}
\pgflineto{\pgfpoint{164.160004pt}{220.471588pt}}
\pgfusepath{stroke}
\pgfpathmoveto{\pgfpoint{164.160004pt}{220.471588pt}}
\pgflineto{\pgfpoint{164.195496pt}{220.471588pt}}
\pgfusepath{stroke}
\pgfpathmoveto{\pgfpoint{164.160004pt}{226.648422pt}}
\pgflineto{\pgfpoint{164.195496pt}{226.648422pt}}
\pgfusepath{stroke}
\pgfpathmoveto{\pgfpoint{164.160004pt}{232.825272pt}}
\pgflineto{\pgfpoint{164.204437pt}{232.825272pt}}
\pgfusepath{stroke}
\pgfpathmoveto{\pgfpoint{164.160004pt}{239.002106pt}}
\pgflineto{\pgfpoint{164.204376pt}{239.002106pt}}
\pgfusepath{stroke}
\pgfpathmoveto{\pgfpoint{164.160004pt}{245.178955pt}}
\pgflineto{\pgfpoint{164.204361pt}{245.178955pt}}
\pgfusepath{stroke}
\pgfpathmoveto{\pgfpoint{164.160004pt}{251.355804pt}}
\pgflineto{\pgfpoint{164.204361pt}{251.355804pt}}
\pgfusepath{stroke}
\pgfpathmoveto{\pgfpoint{164.160004pt}{257.532623pt}}
\pgflineto{\pgfpoint{164.204315pt}{257.532623pt}}
\pgfusepath{stroke}
\pgfpathmoveto{\pgfpoint{164.160004pt}{263.709473pt}}
\pgflineto{\pgfpoint{164.213303pt}{263.709473pt}}
\pgfusepath{stroke}
\pgfpathmoveto{\pgfpoint{164.160004pt}{269.886322pt}}
\pgflineto{\pgfpoint{164.213242pt}{269.886322pt}}
\pgfusepath{stroke}
\pgfpathmoveto{\pgfpoint{164.160004pt}{276.063141pt}}
\pgflineto{\pgfpoint{164.213242pt}{276.063141pt}}
\pgfusepath{stroke}
\pgfpathmoveto{\pgfpoint{164.160004pt}{282.239990pt}}
\pgflineto{\pgfpoint{164.213242pt}{282.239990pt}}
\pgfusepath{stroke}
\pgfpathmoveto{\pgfpoint{164.160004pt}{288.416840pt}}
\pgflineto{\pgfpoint{164.222168pt}{288.416840pt}}
\pgfusepath{stroke}
\pgfpathmoveto{\pgfpoint{164.160004pt}{294.593689pt}}
\pgflineto{\pgfpoint{164.222168pt}{294.593689pt}}
\pgfusepath{stroke}
\pgfpathmoveto{\pgfpoint{164.160004pt}{300.770538pt}}
\pgflineto{\pgfpoint{164.222107pt}{300.770538pt}}
\pgfusepath{stroke}
\pgfpathmoveto{\pgfpoint{164.160004pt}{306.947388pt}}
\pgflineto{\pgfpoint{164.217590pt}{306.947388pt}}
\pgfusepath{stroke}
\pgfpathmoveto{\pgfpoint{164.160004pt}{313.124207pt}}
\pgflineto{\pgfpoint{164.222076pt}{313.124207pt}}
\pgfusepath{stroke}
\pgfpathmoveto{\pgfpoint{164.160004pt}{319.301056pt}}
\pgflineto{\pgfpoint{164.222015pt}{319.301056pt}}
\pgfusepath{stroke}
\pgfpathmoveto{\pgfpoint{164.226578pt}{325.477905pt}}
\pgflineto{\pgfpoint{164.173538pt}{325.477905pt}}
\pgfusepath{stroke}
\pgfpathmoveto{\pgfpoint{164.226517pt}{331.654724pt}}
\pgflineto{\pgfpoint{164.173538pt}{331.654724pt}}
\pgfusepath{stroke}
\pgfpathmoveto{\pgfpoint{164.226517pt}{337.831604pt}}
\pgflineto{\pgfpoint{164.173599pt}{337.831604pt}}
\pgfusepath{stroke}
\pgfpathmoveto{\pgfpoint{164.230942pt}{344.008423pt}}
\pgflineto{\pgfpoint{164.178024pt}{344.008423pt}}
\pgfusepath{stroke}
\pgfpathmoveto{\pgfpoint{164.230896pt}{350.185242pt}}
\pgflineto{\pgfpoint{164.178040pt}{350.185242pt}}
\pgfusepath{stroke}
\pgfpathmoveto{\pgfpoint{164.235443pt}{356.362122pt}}
\pgflineto{\pgfpoint{164.182632pt}{356.362122pt}}
\pgfusepath{stroke}
\pgfpathmoveto{\pgfpoint{164.235382pt}{362.538940pt}}
\pgflineto{\pgfpoint{164.182571pt}{362.538940pt}}
\pgfusepath{stroke}
\pgfpathmoveto{\pgfpoint{164.235413pt}{368.715820pt}}
\pgflineto{\pgfpoint{164.182587pt}{368.715820pt}}
\pgfusepath{stroke}
\pgfpathmoveto{\pgfpoint{164.239822pt}{374.892639pt}}
\pgflineto{\pgfpoint{164.187134pt}{374.892639pt}}
\pgfusepath{stroke}
\pgfpathmoveto{\pgfpoint{164.239761pt}{381.069458pt}}
\pgflineto{\pgfpoint{164.187073pt}{381.069458pt}}
\pgfusepath{stroke}
\pgfpathmoveto{\pgfpoint{164.244308pt}{387.246338pt}}
\pgflineto{\pgfpoint{164.191681pt}{387.246338pt}}
\pgfusepath{stroke}
\pgfpathmoveto{\pgfpoint{164.244247pt}{393.423157pt}}
\pgflineto{\pgfpoint{164.191620pt}{393.423157pt}}
\pgfusepath{stroke}
\pgfpathmoveto{\pgfpoint{164.160004pt}{90.757896pt}}
\pgflineto{\pgfpoint{164.160004pt}{84.581039pt}}
\pgfusepath{stroke}
\pgfpathmoveto{\pgfpoint{164.160004pt}{59.873672pt}}
\pgflineto{\pgfpoint{164.160004pt}{53.696838pt}}
\pgfusepath{stroke}
\pgfpathmoveto{\pgfpoint{164.160004pt}{66.050522pt}}
\pgflineto{\pgfpoint{164.160004pt}{59.873672pt}}
\pgfusepath{stroke}
\pgfpathmoveto{\pgfpoint{164.160004pt}{72.227356pt}}
\pgflineto{\pgfpoint{164.160004pt}{66.050522pt}}
\pgfusepath{stroke}
\pgfpathmoveto{\pgfpoint{164.160004pt}{78.404205pt}}
\pgflineto{\pgfpoint{164.160004pt}{72.227356pt}}
\pgfusepath{stroke}
\pgfpathmoveto{\pgfpoint{164.160004pt}{84.581039pt}}
\pgflineto{\pgfpoint{164.160004pt}{78.404205pt}}
\pgfusepath{stroke}
\pgfpathmoveto{\pgfpoint{164.160004pt}{96.934731pt}}
\pgflineto{\pgfpoint{164.160004pt}{90.757896pt}}
\pgfusepath{stroke}
\pgfpathmoveto{\pgfpoint{164.160004pt}{103.111580pt}}
\pgflineto{\pgfpoint{164.160004pt}{96.934731pt}}
\pgfusepath{stroke}
\pgfpathmoveto{\pgfpoint{164.160004pt}{109.288422pt}}
\pgflineto{\pgfpoint{164.160004pt}{103.111580pt}}
\pgfusepath{stroke}
\pgfpathmoveto{\pgfpoint{164.160004pt}{115.465263pt}}
\pgflineto{\pgfpoint{164.160004pt}{109.288422pt}}
\pgfusepath{stroke}
\pgfpathmoveto{\pgfpoint{164.160004pt}{121.642097pt}}
\pgflineto{\pgfpoint{164.160004pt}{115.465263pt}}
\pgfusepath{stroke}
\pgfpathmoveto{\pgfpoint{164.160004pt}{127.818947pt}}
\pgflineto{\pgfpoint{164.160004pt}{121.642097pt}}
\pgfusepath{stroke}
\pgfpathmoveto{\pgfpoint{164.160004pt}{133.995789pt}}
\pgflineto{\pgfpoint{164.160004pt}{127.818947pt}}
\pgfusepath{stroke}
\pgfpathmoveto{\pgfpoint{164.160004pt}{140.172638pt}}
\pgflineto{\pgfpoint{164.160004pt}{133.995789pt}}
\pgfusepath{stroke}
\pgfpathmoveto{\pgfpoint{164.160004pt}{47.519989pt}}
\pgflineto{\pgfpoint{164.142181pt}{47.519989pt}}
\pgfusepath{stroke}
\pgfpathmoveto{\pgfpoint{164.160004pt}{53.696838pt}}
\pgflineto{\pgfpoint{164.160004pt}{47.519989pt}}
\pgfusepath{stroke}
\pgfpathmoveto{\pgfpoint{164.160004pt}{47.519989pt}}
\pgflineto{\pgfpoint{173.070084pt}{47.519989pt}}
\pgfusepath{stroke}
\pgfpathmoveto{\pgfpoint{164.160004pt}{53.696838pt}}
\pgflineto{\pgfpoint{173.070129pt}{53.696838pt}}
\pgfusepath{stroke}
\pgfpathmoveto{\pgfpoint{164.160004pt}{59.873672pt}}
\pgflineto{\pgfpoint{173.070068pt}{59.873672pt}}
\pgfusepath{stroke}
\pgfpathmoveto{\pgfpoint{173.087997pt}{66.050522pt}}
\pgflineto{\pgfpoint{164.160004pt}{66.050522pt}}
\pgfusepath{stroke}
\pgfpathmoveto{\pgfpoint{173.087997pt}{72.227356pt}}
\pgflineto{\pgfpoint{164.160004pt}{72.227356pt}}
\pgfusepath{stroke}
\pgfpathmoveto{\pgfpoint{173.087997pt}{78.404205pt}}
\pgflineto{\pgfpoint{164.160004pt}{78.404205pt}}
\pgfusepath{stroke}
\pgfpathmoveto{\pgfpoint{173.087997pt}{84.581039pt}}
\pgflineto{\pgfpoint{164.160004pt}{84.581039pt}}
\pgfusepath{stroke}
\pgfpathmoveto{\pgfpoint{173.087997pt}{90.757896pt}}
\pgflineto{\pgfpoint{164.160004pt}{90.757896pt}}
\pgfusepath{stroke}
\pgfpathmoveto{\pgfpoint{173.087997pt}{96.934731pt}}
\pgflineto{\pgfpoint{164.160004pt}{96.934731pt}}
\pgfusepath{stroke}
\pgfpathmoveto{\pgfpoint{173.087997pt}{103.111580pt}}
\pgflineto{\pgfpoint{164.160004pt}{103.111580pt}}
\pgfusepath{stroke}
\pgfpathmoveto{\pgfpoint{173.087997pt}{109.288422pt}}
\pgflineto{\pgfpoint{164.160004pt}{109.288422pt}}
\pgfusepath{stroke}
\pgfpathmoveto{\pgfpoint{173.087997pt}{115.465263pt}}
\pgflineto{\pgfpoint{164.160004pt}{115.465263pt}}
\pgfusepath{stroke}
\pgfpathmoveto{\pgfpoint{173.087997pt}{121.642097pt}}
\pgflineto{\pgfpoint{164.160004pt}{121.642097pt}}
\pgfusepath{stroke}
\pgfpathmoveto{\pgfpoint{173.087997pt}{127.818947pt}}
\pgflineto{\pgfpoint{164.160004pt}{127.818947pt}}
\pgfusepath{stroke}
\pgfpathmoveto{\pgfpoint{173.087997pt}{133.995789pt}}
\pgflineto{\pgfpoint{164.160004pt}{133.995789pt}}
\pgfusepath{stroke}
\pgfpathmoveto{\pgfpoint{173.087997pt}{140.172638pt}}
\pgflineto{\pgfpoint{164.160004pt}{140.172638pt}}
\pgfusepath{stroke}
\pgfpathmoveto{\pgfpoint{173.087997pt}{146.349472pt}}
\pgflineto{\pgfpoint{164.177795pt}{146.349472pt}}
\pgfusepath{stroke}
\pgfpathmoveto{\pgfpoint{173.087997pt}{152.526306pt}}
\pgflineto{\pgfpoint{164.177734pt}{152.526306pt}}
\pgfusepath{stroke}
\pgfpathmoveto{\pgfpoint{173.087997pt}{158.703156pt}}
\pgflineto{\pgfpoint{164.177750pt}{158.703156pt}}
\pgfusepath{stroke}
\pgfpathmoveto{\pgfpoint{173.087997pt}{164.880005pt}}
\pgflineto{\pgfpoint{164.177689pt}{164.880005pt}}
\pgfusepath{stroke}
\pgfpathmoveto{\pgfpoint{173.087997pt}{171.056854pt}}
\pgflineto{\pgfpoint{164.186707pt}{171.056854pt}}
\pgfusepath{stroke}
\pgfpathmoveto{\pgfpoint{173.087997pt}{177.233673pt}}
\pgflineto{\pgfpoint{164.186600pt}{177.233673pt}}
\pgfusepath{stroke}
\pgfpathmoveto{\pgfpoint{173.087997pt}{183.410522pt}}
\pgflineto{\pgfpoint{164.186615pt}{183.410522pt}}
\pgfusepath{stroke}
\pgfpathmoveto{\pgfpoint{173.087997pt}{189.587372pt}}
\pgflineto{\pgfpoint{164.186615pt}{189.587372pt}}
\pgfusepath{stroke}
\pgfpathmoveto{\pgfpoint{173.087997pt}{195.764206pt}}
\pgflineto{\pgfpoint{164.186569pt}{195.764206pt}}
\pgfusepath{stroke}
\pgfpathmoveto{\pgfpoint{173.087997pt}{201.941055pt}}
\pgflineto{\pgfpoint{164.195572pt}{201.941055pt}}
\pgfusepath{stroke}
\pgfpathmoveto{\pgfpoint{173.087997pt}{208.117905pt}}
\pgflineto{\pgfpoint{164.195511pt}{208.117905pt}}
\pgfusepath{stroke}
\pgfpathmoveto{\pgfpoint{173.087997pt}{214.294739pt}}
\pgflineto{\pgfpoint{164.195496pt}{214.294739pt}}
\pgfusepath{stroke}
\pgfpathmoveto{\pgfpoint{173.087997pt}{220.471588pt}}
\pgflineto{\pgfpoint{164.195496pt}{220.471588pt}}
\pgfusepath{stroke}
\pgfpathmoveto{\pgfpoint{164.195496pt}{226.648422pt}}
\pgflineto{\pgfpoint{173.069946pt}{226.648422pt}}
\pgfusepath{stroke}
\pgfpathmoveto{\pgfpoint{164.204437pt}{232.825272pt}}
\pgflineto{\pgfpoint{173.060944pt}{232.825272pt}}
\pgfusepath{stroke}
\pgfpathmoveto{\pgfpoint{164.204376pt}{239.002106pt}}
\pgflineto{\pgfpoint{173.051956pt}{239.002106pt}}
\pgfusepath{stroke}
\pgfpathmoveto{\pgfpoint{164.204361pt}{245.178955pt}}
\pgflineto{\pgfpoint{173.051910pt}{245.178955pt}}
\pgfusepath{stroke}
\pgfpathmoveto{\pgfpoint{164.204361pt}{251.355804pt}}
\pgflineto{\pgfpoint{173.042892pt}{251.355804pt}}
\pgfusepath{stroke}
\pgfpathmoveto{\pgfpoint{164.204315pt}{257.532623pt}}
\pgflineto{\pgfpoint{173.033813pt}{257.532623pt}}
\pgfusepath{stroke}
\pgfpathmoveto{\pgfpoint{164.213303pt}{263.709473pt}}
\pgflineto{\pgfpoint{173.024902pt}{263.709473pt}}
\pgfusepath{stroke}
\pgfpathmoveto{\pgfpoint{164.213242pt}{269.886322pt}}
\pgflineto{\pgfpoint{173.015884pt}{269.886322pt}}
\pgfusepath{stroke}
\pgfpathmoveto{\pgfpoint{164.213242pt}{276.063141pt}}
\pgflineto{\pgfpoint{173.015808pt}{276.063141pt}}
\pgfusepath{stroke}
\pgfpathmoveto{\pgfpoint{164.213242pt}{282.239990pt}}
\pgflineto{\pgfpoint{173.006760pt}{282.239990pt}}
\pgfusepath{stroke}
\pgfpathmoveto{\pgfpoint{164.222168pt}{288.416840pt}}
\pgflineto{\pgfpoint{172.997818pt}{288.416840pt}}
\pgfusepath{stroke}
\pgfpathmoveto{\pgfpoint{164.222168pt}{294.593689pt}}
\pgflineto{\pgfpoint{172.988846pt}{294.593689pt}}
\pgfusepath{stroke}
\pgfpathmoveto{\pgfpoint{164.222107pt}{300.770538pt}}
\pgflineto{\pgfpoint{172.979828pt}{300.770538pt}}
\pgfusepath{stroke}
\pgfpathmoveto{\pgfpoint{164.217590pt}{306.947388pt}}
\pgflineto{\pgfpoint{172.970749pt}{306.947388pt}}
\pgfusepath{stroke}
\pgfpathmoveto{\pgfpoint{164.222076pt}{313.124207pt}}
\pgflineto{\pgfpoint{172.966217pt}{313.124207pt}}
\pgfusepath{stroke}
\pgfpathmoveto{\pgfpoint{164.222015pt}{319.301056pt}}
\pgflineto{\pgfpoint{172.957184pt}{319.301056pt}}
\pgfusepath{stroke}
\pgfpathmoveto{\pgfpoint{164.226578pt}{325.477905pt}}
\pgflineto{\pgfpoint{172.952698pt}{325.477905pt}}
\pgfusepath{stroke}
\pgfpathmoveto{\pgfpoint{164.226517pt}{331.654724pt}}
\pgflineto{\pgfpoint{172.943649pt}{331.654724pt}}
\pgfusepath{stroke}
\pgfpathmoveto{\pgfpoint{164.226517pt}{337.831604pt}}
\pgflineto{\pgfpoint{172.934662pt}{337.831604pt}}
\pgfusepath{stroke}
\pgfpathmoveto{\pgfpoint{164.230942pt}{344.008423pt}}
\pgflineto{\pgfpoint{172.930191pt}{344.008423pt}}
\pgfusepath{stroke}
\pgfpathmoveto{\pgfpoint{164.230896pt}{350.185242pt}}
\pgflineto{\pgfpoint{172.921097pt}{350.185242pt}}
\pgfusepath{stroke}
\pgfpathmoveto{\pgfpoint{164.235443pt}{356.362122pt}}
\pgflineto{\pgfpoint{172.916656pt}{356.362122pt}}
\pgfusepath{stroke}
\pgfpathmoveto{\pgfpoint{164.235382pt}{362.538940pt}}
\pgflineto{\pgfpoint{172.907562pt}{362.538940pt}}
\pgfusepath{stroke}
\pgfpathmoveto{\pgfpoint{164.235413pt}{368.715820pt}}
\pgflineto{\pgfpoint{172.903000pt}{368.715820pt}}
\pgfusepath{stroke}
\pgfpathmoveto{\pgfpoint{164.239822pt}{374.892639pt}}
\pgflineto{\pgfpoint{172.894043pt}{374.892639pt}}
\pgfusepath{stroke}
\pgfpathmoveto{\pgfpoint{164.239761pt}{381.069458pt}}
\pgflineto{\pgfpoint{172.885010pt}{381.069458pt}}
\pgfusepath{stroke}
\pgfpathmoveto{\pgfpoint{164.244308pt}{387.246338pt}}
\pgflineto{\pgfpoint{172.880569pt}{387.246338pt}}
\pgfusepath{stroke}
\pgfpathmoveto{\pgfpoint{164.244247pt}{393.423157pt}}
\pgflineto{\pgfpoint{172.871475pt}{393.423157pt}}
\pgfusepath{stroke}
\pgfpathmoveto{\pgfpoint{173.087997pt}{164.880005pt}}
\pgflineto{\pgfpoint{173.087997pt}{158.703156pt}}
\pgfusepath{stroke}
\pgfpathmoveto{\pgfpoint{173.087997pt}{158.703156pt}}
\pgflineto{\pgfpoint{173.087997pt}{152.526306pt}}
\pgfusepath{stroke}
\pgfpathmoveto{\pgfpoint{173.087997pt}{171.056854pt}}
\pgflineto{\pgfpoint{173.087997pt}{164.880005pt}}
\pgfusepath{stroke}
\pgfpathmoveto{\pgfpoint{173.087997pt}{177.233673pt}}
\pgflineto{\pgfpoint{173.087997pt}{171.056854pt}}
\pgfusepath{stroke}
\pgfpathmoveto{\pgfpoint{173.087997pt}{183.410522pt}}
\pgflineto{\pgfpoint{173.087997pt}{177.233673pt}}
\pgfusepath{stroke}
\pgfpathmoveto{\pgfpoint{173.087997pt}{189.587372pt}}
\pgflineto{\pgfpoint{173.087997pt}{183.410522pt}}
\pgfusepath{stroke}
\pgfpathmoveto{\pgfpoint{173.087997pt}{195.764206pt}}
\pgflineto{\pgfpoint{173.087997pt}{189.587372pt}}
\pgfusepath{stroke}
\pgfpathmoveto{\pgfpoint{173.087997pt}{201.941055pt}}
\pgflineto{\pgfpoint{173.087997pt}{195.764206pt}}
\pgfusepath{stroke}
\pgfpathmoveto{\pgfpoint{173.087997pt}{158.703156pt}}
\pgflineto{\pgfpoint{173.105835pt}{158.703156pt}}
\pgfusepath{stroke}
\pgfpathmoveto{\pgfpoint{173.087997pt}{164.880005pt}}
\pgflineto{\pgfpoint{173.105835pt}{164.880005pt}}
\pgfusepath{stroke}
\pgfpathmoveto{\pgfpoint{173.087997pt}{171.056854pt}}
\pgflineto{\pgfpoint{173.105850pt}{171.056854pt}}
\pgfusepath{stroke}
\pgfpathmoveto{\pgfpoint{173.087997pt}{177.233673pt}}
\pgflineto{\pgfpoint{173.105804pt}{177.233673pt}}
\pgfusepath{stroke}
\pgfpathmoveto{\pgfpoint{173.087997pt}{183.410522pt}}
\pgflineto{\pgfpoint{173.114822pt}{183.410522pt}}
\pgfusepath{stroke}
\pgfpathmoveto{\pgfpoint{173.087997pt}{189.587372pt}}
\pgflineto{\pgfpoint{173.114822pt}{189.587372pt}}
\pgfusepath{stroke}
\pgfpathmoveto{\pgfpoint{173.087997pt}{195.764206pt}}
\pgflineto{\pgfpoint{173.114700pt}{195.764206pt}}
\pgfusepath{stroke}
\pgfpathmoveto{\pgfpoint{173.087997pt}{214.294739pt}}
\pgflineto{\pgfpoint{173.087997pt}{208.117905pt}}
\pgfusepath{stroke}
\pgfpathmoveto{\pgfpoint{173.087997pt}{208.117905pt}}
\pgflineto{\pgfpoint{173.087997pt}{201.941055pt}}
\pgfusepath{stroke}
\pgfpathmoveto{\pgfpoint{173.087997pt}{220.471588pt}}
\pgflineto{\pgfpoint{173.087997pt}{214.294739pt}}
\pgfusepath{stroke}
\pgfpathmoveto{\pgfpoint{173.051956pt}{239.002106pt}}
\pgflineto{\pgfpoint{173.069962pt}{239.002106pt}}
\pgfusepath{stroke}
\pgfpathmoveto{\pgfpoint{173.087997pt}{245.178955pt}}
\pgflineto{\pgfpoint{173.051910pt}{245.178955pt}}
\pgfusepath{stroke}
\pgfpathmoveto{\pgfpoint{173.087997pt}{251.355804pt}}
\pgflineto{\pgfpoint{173.042892pt}{251.355804pt}}
\pgfusepath{stroke}
\pgfpathmoveto{\pgfpoint{173.087997pt}{257.532623pt}}
\pgflineto{\pgfpoint{173.033813pt}{257.532623pt}}
\pgfusepath{stroke}
\pgfpathmoveto{\pgfpoint{173.087997pt}{263.709473pt}}
\pgflineto{\pgfpoint{173.024902pt}{263.709473pt}}
\pgfusepath{stroke}
\pgfpathmoveto{\pgfpoint{173.087997pt}{269.886322pt}}
\pgflineto{\pgfpoint{173.015884pt}{269.886322pt}}
\pgfusepath{stroke}
\pgfpathmoveto{\pgfpoint{173.087997pt}{276.063141pt}}
\pgflineto{\pgfpoint{173.015808pt}{276.063141pt}}
\pgfusepath{stroke}
\pgfpathmoveto{\pgfpoint{173.087997pt}{282.239990pt}}
\pgflineto{\pgfpoint{173.006760pt}{282.239990pt}}
\pgfusepath{stroke}
\pgfpathmoveto{\pgfpoint{173.087997pt}{288.416840pt}}
\pgflineto{\pgfpoint{172.997818pt}{288.416840pt}}
\pgfusepath{stroke}
\pgfpathmoveto{\pgfpoint{173.087997pt}{294.593689pt}}
\pgflineto{\pgfpoint{172.988846pt}{294.593689pt}}
\pgfusepath{stroke}
\pgfpathmoveto{\pgfpoint{173.087997pt}{300.770538pt}}
\pgflineto{\pgfpoint{172.979828pt}{300.770538pt}}
\pgfusepath{stroke}
\pgfpathmoveto{\pgfpoint{173.087997pt}{306.947388pt}}
\pgflineto{\pgfpoint{172.970749pt}{306.947388pt}}
\pgfusepath{stroke}
\pgfpathmoveto{\pgfpoint{173.087997pt}{313.124207pt}}
\pgflineto{\pgfpoint{172.966217pt}{313.124207pt}}
\pgfusepath{stroke}
\pgfpathmoveto{\pgfpoint{173.087997pt}{319.301056pt}}
\pgflineto{\pgfpoint{172.957184pt}{319.301056pt}}
\pgfusepath{stroke}
\pgfpathmoveto{\pgfpoint{173.087997pt}{325.477905pt}}
\pgflineto{\pgfpoint{172.952698pt}{325.477905pt}}
\pgfusepath{stroke}
\pgfpathmoveto{\pgfpoint{173.087997pt}{331.654724pt}}
\pgflineto{\pgfpoint{172.943649pt}{331.654724pt}}
\pgfusepath{stroke}
\pgfpathmoveto{\pgfpoint{173.087997pt}{337.831604pt}}
\pgflineto{\pgfpoint{172.934662pt}{337.831604pt}}
\pgfusepath{stroke}
\pgfpathmoveto{\pgfpoint{173.087997pt}{344.008423pt}}
\pgflineto{\pgfpoint{172.930191pt}{344.008423pt}}
\pgfusepath{stroke}
\pgfpathmoveto{\pgfpoint{173.087997pt}{350.185242pt}}
\pgflineto{\pgfpoint{172.921097pt}{350.185242pt}}
\pgfusepath{stroke}
\pgfpathmoveto{\pgfpoint{173.087997pt}{356.362122pt}}
\pgflineto{\pgfpoint{172.916656pt}{356.362122pt}}
\pgfusepath{stroke}
\pgfpathmoveto{\pgfpoint{173.087997pt}{362.538940pt}}
\pgflineto{\pgfpoint{172.907562pt}{362.538940pt}}
\pgfusepath{stroke}
\pgfpathmoveto{\pgfpoint{173.087997pt}{368.715820pt}}
\pgflineto{\pgfpoint{172.903000pt}{368.715820pt}}
\pgfusepath{stroke}
\pgfpathmoveto{\pgfpoint{173.087997pt}{374.892639pt}}
\pgflineto{\pgfpoint{172.894043pt}{374.892639pt}}
\pgfusepath{stroke}
\pgfpathmoveto{\pgfpoint{173.087997pt}{381.069458pt}}
\pgflineto{\pgfpoint{172.885010pt}{381.069458pt}}
\pgfusepath{stroke}
\pgfpathmoveto{\pgfpoint{173.087997pt}{387.246338pt}}
\pgflineto{\pgfpoint{172.880569pt}{387.246338pt}}
\pgfusepath{stroke}
\pgfpathmoveto{\pgfpoint{173.087997pt}{393.423157pt}}
\pgflineto{\pgfpoint{172.871475pt}{393.423157pt}}
\pgfusepath{stroke}
\pgfpathmoveto{\pgfpoint{173.087997pt}{350.185242pt}}
\pgflineto{\pgfpoint{173.087997pt}{344.008423pt}}
\pgfusepath{stroke}
\pgfpathmoveto{\pgfpoint{173.087997pt}{344.008423pt}}
\pgflineto{\pgfpoint{173.087997pt}{337.831604pt}}
\pgfusepath{stroke}
\pgfpathmoveto{\pgfpoint{173.087997pt}{356.362122pt}}
\pgflineto{\pgfpoint{173.087997pt}{350.185242pt}}
\pgfusepath{stroke}
\pgfpathmoveto{\pgfpoint{173.087997pt}{362.538940pt}}
\pgflineto{\pgfpoint{173.087997pt}{356.362122pt}}
\pgfusepath{stroke}
\pgfpathmoveto{\pgfpoint{173.087997pt}{344.008423pt}}
\pgflineto{\pgfpoint{173.101593pt}{344.008423pt}}
\pgfusepath{stroke}
\pgfpathmoveto{\pgfpoint{173.087997pt}{350.185242pt}}
\pgflineto{\pgfpoint{173.101532pt}{350.185242pt}}
\pgfusepath{stroke}
\pgfpathmoveto{\pgfpoint{173.087997pt}{356.362122pt}}
\pgflineto{\pgfpoint{173.106140pt}{356.362122pt}}
\pgfusepath{stroke}
\pgfpathmoveto{\pgfpoint{173.087997pt}{387.246338pt}}
\pgflineto{\pgfpoint{173.087997pt}{381.069458pt}}
\pgfusepath{stroke}
\pgfpathmoveto{\pgfpoint{173.087997pt}{368.715820pt}}
\pgflineto{\pgfpoint{173.087997pt}{362.538940pt}}
\pgfusepath{stroke}
\pgfpathmoveto{\pgfpoint{173.087997pt}{374.892639pt}}
\pgflineto{\pgfpoint{173.087997pt}{368.715820pt}}
\pgfusepath{stroke}
\pgfpathmoveto{\pgfpoint{173.087997pt}{381.069458pt}}
\pgflineto{\pgfpoint{173.087997pt}{374.892639pt}}
\pgfusepath{stroke}
\pgfpathmoveto{\pgfpoint{173.087997pt}{393.423157pt}}
\pgflineto{\pgfpoint{173.087997pt}{387.246338pt}}
\pgfusepath{stroke}
\pgfpathmoveto{\pgfpoint{173.087997pt}{362.538940pt}}
\pgflineto{\pgfpoint{173.106079pt}{362.538940pt}}
\pgfusepath{stroke}
\pgfpathmoveto{\pgfpoint{173.087997pt}{368.715820pt}}
\pgflineto{\pgfpoint{173.110626pt}{368.715820pt}}
\pgfusepath{stroke}
\pgfpathmoveto{\pgfpoint{173.087997pt}{374.892639pt}}
\pgflineto{\pgfpoint{173.110626pt}{374.892639pt}}
\pgfusepath{stroke}
\pgfpathmoveto{\pgfpoint{173.087997pt}{381.069458pt}}
\pgflineto{\pgfpoint{173.110580pt}{381.069458pt}}
\pgfusepath{stroke}
\pgfpathmoveto{\pgfpoint{173.087997pt}{387.246338pt}}
\pgflineto{\pgfpoint{173.115173pt}{387.246338pt}}
\pgfusepath{stroke}
\pgfpathmoveto{\pgfpoint{173.087997pt}{282.239990pt}}
\pgflineto{\pgfpoint{173.087997pt}{276.063141pt}}
\pgfusepath{stroke}
\pgfpathmoveto{\pgfpoint{173.087997pt}{251.355804pt}}
\pgflineto{\pgfpoint{173.087997pt}{245.178955pt}}
\pgfusepath{stroke}
\pgfpathmoveto{\pgfpoint{173.087997pt}{257.532623pt}}
\pgflineto{\pgfpoint{173.087997pt}{251.355804pt}}
\pgfusepath{stroke}
\pgfpathmoveto{\pgfpoint{173.087997pt}{263.709473pt}}
\pgflineto{\pgfpoint{173.087997pt}{257.532623pt}}
\pgfusepath{stroke}
\pgfpathmoveto{\pgfpoint{173.087997pt}{269.886322pt}}
\pgflineto{\pgfpoint{173.087997pt}{263.709473pt}}
\pgfusepath{stroke}
\pgfpathmoveto{\pgfpoint{173.087997pt}{276.063141pt}}
\pgflineto{\pgfpoint{173.087997pt}{269.886322pt}}
\pgfusepath{stroke}
\pgfpathmoveto{\pgfpoint{173.087997pt}{288.416840pt}}
\pgflineto{\pgfpoint{173.087997pt}{282.239990pt}}
\pgfusepath{stroke}
\pgfpathmoveto{\pgfpoint{173.087997pt}{294.593689pt}}
\pgflineto{\pgfpoint{173.087997pt}{288.416840pt}}
\pgfusepath{stroke}
\pgfpathmoveto{\pgfpoint{173.087997pt}{300.770538pt}}
\pgflineto{\pgfpoint{173.087997pt}{294.593689pt}}
\pgfusepath{stroke}
\pgfpathmoveto{\pgfpoint{173.087997pt}{306.947388pt}}
\pgflineto{\pgfpoint{173.087997pt}{300.770538pt}}
\pgfusepath{stroke}
\pgfpathmoveto{\pgfpoint{173.087997pt}{313.124207pt}}
\pgflineto{\pgfpoint{173.087997pt}{306.947388pt}}
\pgfusepath{stroke}
\pgfpathmoveto{\pgfpoint{173.087997pt}{319.301056pt}}
\pgflineto{\pgfpoint{173.087997pt}{313.124207pt}}
\pgfusepath{stroke}
\pgfpathmoveto{\pgfpoint{173.087997pt}{325.477905pt}}
\pgflineto{\pgfpoint{173.087997pt}{319.301056pt}}
\pgfusepath{stroke}
\pgfpathmoveto{\pgfpoint{173.087997pt}{331.654724pt}}
\pgflineto{\pgfpoint{173.087997pt}{325.477905pt}}
\pgfusepath{stroke}
\pgfpathmoveto{\pgfpoint{173.087997pt}{337.831604pt}}
\pgflineto{\pgfpoint{173.087997pt}{331.654724pt}}
\pgfusepath{stroke}
\pgfpathmoveto{\pgfpoint{173.087997pt}{201.941055pt}}
\pgflineto{\pgfpoint{173.114731pt}{201.941055pt}}
\pgfusepath{stroke}
\pgfpathmoveto{\pgfpoint{173.087997pt}{208.117905pt}}
\pgflineto{\pgfpoint{173.114731pt}{208.117905pt}}
\pgfusepath{stroke}
\pgfpathmoveto{\pgfpoint{173.087997pt}{214.294739pt}}
\pgflineto{\pgfpoint{173.123734pt}{214.294739pt}}
\pgfusepath{stroke}
\pgfpathmoveto{\pgfpoint{173.087997pt}{220.471588pt}}
\pgflineto{\pgfpoint{173.123688pt}{220.471588pt}}
\pgfusepath{stroke}
\pgfpathmoveto{\pgfpoint{173.087997pt}{226.648422pt}}
\pgflineto{\pgfpoint{173.087997pt}{220.471588pt}}
\pgfusepath{stroke}
\pgfpathmoveto{\pgfpoint{173.087997pt}{226.648422pt}}
\pgflineto{\pgfpoint{173.069946pt}{226.648422pt}}
\pgfusepath{stroke}
\pgfpathmoveto{\pgfpoint{173.087997pt}{226.648422pt}}
\pgflineto{\pgfpoint{173.123718pt}{226.648422pt}}
\pgfusepath{stroke}
\pgfpathmoveto{\pgfpoint{173.087997pt}{232.825272pt}}
\pgflineto{\pgfpoint{173.060944pt}{232.825272pt}}
\pgfusepath{stroke}
\pgfpathmoveto{\pgfpoint{173.087997pt}{239.002106pt}}
\pgflineto{\pgfpoint{173.069962pt}{239.002106pt}}
\pgfusepath{stroke}
\pgfpathmoveto{\pgfpoint{173.087997pt}{232.825272pt}}
\pgflineto{\pgfpoint{173.087997pt}{226.648422pt}}
\pgfusepath{stroke}
\pgfpathmoveto{\pgfpoint{173.087997pt}{245.178955pt}}
\pgflineto{\pgfpoint{173.087997pt}{239.002106pt}}
\pgfusepath{stroke}
\pgfpathmoveto{\pgfpoint{173.087997pt}{239.002106pt}}
\pgflineto{\pgfpoint{173.087997pt}{232.825272pt}}
\pgfusepath{stroke}
\pgfpathmoveto{\pgfpoint{173.087997pt}{232.825272pt}}
\pgflineto{\pgfpoint{173.123718pt}{232.825272pt}}
\pgfusepath{stroke}
\pgfpathmoveto{\pgfpoint{173.087997pt}{239.002106pt}}
\pgflineto{\pgfpoint{173.123657pt}{239.002106pt}}
\pgfusepath{stroke}
\pgfpathmoveto{\pgfpoint{173.087997pt}{245.178955pt}}
\pgflineto{\pgfpoint{173.132660pt}{245.178955pt}}
\pgfusepath{stroke}
\pgfpathmoveto{\pgfpoint{173.087997pt}{251.355804pt}}
\pgflineto{\pgfpoint{173.132599pt}{251.355804pt}}
\pgfusepath{stroke}
\pgfpathmoveto{\pgfpoint{173.087997pt}{257.532623pt}}
\pgflineto{\pgfpoint{173.132584pt}{257.532623pt}}
\pgfusepath{stroke}
\pgfpathmoveto{\pgfpoint{173.087997pt}{263.709473pt}}
\pgflineto{\pgfpoint{173.132584pt}{263.709473pt}}
\pgfusepath{stroke}
\pgfpathmoveto{\pgfpoint{173.087997pt}{269.886322pt}}
\pgflineto{\pgfpoint{173.132584pt}{269.886322pt}}
\pgfusepath{stroke}
\pgfpathmoveto{\pgfpoint{173.087997pt}{276.063141pt}}
\pgflineto{\pgfpoint{173.141525pt}{276.063141pt}}
\pgfusepath{stroke}
\pgfpathmoveto{\pgfpoint{173.087997pt}{282.239990pt}}
\pgflineto{\pgfpoint{173.141525pt}{282.239990pt}}
\pgfusepath{stroke}
\pgfpathmoveto{\pgfpoint{173.087997pt}{288.416840pt}}
\pgflineto{\pgfpoint{173.141525pt}{288.416840pt}}
\pgfusepath{stroke}
\pgfpathmoveto{\pgfpoint{173.087997pt}{294.593689pt}}
\pgflineto{\pgfpoint{173.141495pt}{294.593689pt}}
\pgfusepath{stroke}
\pgfpathmoveto{\pgfpoint{173.087997pt}{300.770538pt}}
\pgflineto{\pgfpoint{173.141495pt}{300.770538pt}}
\pgfusepath{stroke}
\pgfpathmoveto{\pgfpoint{173.087997pt}{306.947388pt}}
\pgflineto{\pgfpoint{173.141434pt}{306.947388pt}}
\pgfusepath{stroke}
\pgfpathmoveto{\pgfpoint{173.087997pt}{313.124207pt}}
\pgflineto{\pgfpoint{173.145920pt}{313.124207pt}}
\pgfusepath{stroke}
\pgfpathmoveto{\pgfpoint{173.087997pt}{319.301056pt}}
\pgflineto{\pgfpoint{173.145920pt}{319.301056pt}}
\pgfusepath{stroke}
\pgfpathmoveto{\pgfpoint{173.087997pt}{325.477905pt}}
\pgflineto{\pgfpoint{173.150421pt}{325.477905pt}}
\pgfusepath{stroke}
\pgfpathmoveto{\pgfpoint{173.087997pt}{331.654724pt}}
\pgflineto{\pgfpoint{173.150360pt}{331.654724pt}}
\pgfusepath{stroke}
\pgfpathmoveto{\pgfpoint{173.087997pt}{337.831604pt}}
\pgflineto{\pgfpoint{173.150452pt}{337.831604pt}}
\pgfusepath{stroke}
\pgfpathmoveto{\pgfpoint{173.154907pt}{344.008423pt}}
\pgflineto{\pgfpoint{173.101593pt}{344.008423pt}}
\pgfusepath{stroke}
\pgfpathmoveto{\pgfpoint{173.154800pt}{350.185242pt}}
\pgflineto{\pgfpoint{173.101532pt}{350.185242pt}}
\pgfusepath{stroke}
\pgfpathmoveto{\pgfpoint{173.159409pt}{356.362122pt}}
\pgflineto{\pgfpoint{173.106140pt}{356.362122pt}}
\pgfusepath{stroke}
\pgfpathmoveto{\pgfpoint{173.159286pt}{362.538940pt}}
\pgflineto{\pgfpoint{173.106079pt}{362.538940pt}}
\pgfusepath{stroke}
\pgfpathmoveto{\pgfpoint{173.163834pt}{368.715820pt}}
\pgflineto{\pgfpoint{173.110626pt}{368.715820pt}}
\pgfusepath{stroke}
\pgfpathmoveto{\pgfpoint{173.163834pt}{374.892639pt}}
\pgflineto{\pgfpoint{173.110626pt}{374.892639pt}}
\pgfusepath{stroke}
\pgfpathmoveto{\pgfpoint{173.163818pt}{381.069458pt}}
\pgflineto{\pgfpoint{173.110580pt}{381.069458pt}}
\pgfusepath{stroke}
\pgfpathmoveto{\pgfpoint{173.168274pt}{387.246338pt}}
\pgflineto{\pgfpoint{173.115173pt}{387.246338pt}}
\pgfusepath{stroke}
\pgfpathmoveto{\pgfpoint{173.087997pt}{103.111580pt}}
\pgflineto{\pgfpoint{173.087997pt}{96.934731pt}}
\pgfusepath{stroke}
\pgfpathmoveto{\pgfpoint{173.087997pt}{72.227356pt}}
\pgflineto{\pgfpoint{173.087997pt}{66.050522pt}}
\pgfusepath{stroke}
\pgfpathmoveto{\pgfpoint{173.087997pt}{78.404205pt}}
\pgflineto{\pgfpoint{173.087997pt}{72.227356pt}}
\pgfusepath{stroke}
\pgfpathmoveto{\pgfpoint{173.087997pt}{84.581039pt}}
\pgflineto{\pgfpoint{173.087997pt}{78.404205pt}}
\pgfusepath{stroke}
\pgfpathmoveto{\pgfpoint{173.087997pt}{90.757896pt}}
\pgflineto{\pgfpoint{173.087997pt}{84.581039pt}}
\pgfusepath{stroke}
\pgfpathmoveto{\pgfpoint{173.087997pt}{96.934731pt}}
\pgflineto{\pgfpoint{173.087997pt}{90.757896pt}}
\pgfusepath{stroke}
\pgfpathmoveto{\pgfpoint{173.087997pt}{109.288422pt}}
\pgflineto{\pgfpoint{173.087997pt}{103.111580pt}}
\pgfusepath{stroke}
\pgfpathmoveto{\pgfpoint{173.087997pt}{115.465263pt}}
\pgflineto{\pgfpoint{173.087997pt}{109.288422pt}}
\pgfusepath{stroke}
\pgfpathmoveto{\pgfpoint{173.087997pt}{121.642097pt}}
\pgflineto{\pgfpoint{173.087997pt}{115.465263pt}}
\pgfusepath{stroke}
\pgfpathmoveto{\pgfpoint{173.087997pt}{127.818947pt}}
\pgflineto{\pgfpoint{173.087997pt}{121.642097pt}}
\pgfusepath{stroke}
\pgfpathmoveto{\pgfpoint{173.087997pt}{133.995789pt}}
\pgflineto{\pgfpoint{173.087997pt}{127.818947pt}}
\pgfusepath{stroke}
\pgfpathmoveto{\pgfpoint{173.087997pt}{140.172638pt}}
\pgflineto{\pgfpoint{173.087997pt}{133.995789pt}}
\pgfusepath{stroke}
\pgfpathmoveto{\pgfpoint{173.087997pt}{146.349472pt}}
\pgflineto{\pgfpoint{173.087997pt}{140.172638pt}}
\pgfusepath{stroke}
\pgfpathmoveto{\pgfpoint{173.087997pt}{152.526306pt}}
\pgflineto{\pgfpoint{173.087997pt}{146.349472pt}}
\pgfusepath{stroke}
\pgfpathmoveto{\pgfpoint{173.087997pt}{47.519989pt}}
\pgflineto{\pgfpoint{173.070084pt}{47.519989pt}}
\pgfusepath{stroke}
\pgfpathmoveto{\pgfpoint{173.087997pt}{53.696838pt}}
\pgflineto{\pgfpoint{173.070129pt}{53.696838pt}}
\pgfusepath{stroke}
\pgfpathmoveto{\pgfpoint{173.087997pt}{59.873672pt}}
\pgflineto{\pgfpoint{173.070068pt}{59.873672pt}}
\pgfusepath{stroke}
\pgfpathmoveto{\pgfpoint{173.087997pt}{53.696838pt}}
\pgflineto{\pgfpoint{173.087997pt}{47.519989pt}}
\pgfusepath{stroke}
\pgfpathmoveto{\pgfpoint{173.087997pt}{47.519989pt}}
\pgflineto{\pgfpoint{173.106064pt}{47.519989pt}}
\pgfusepath{stroke}
\pgfpathmoveto{\pgfpoint{173.087997pt}{66.050522pt}}
\pgflineto{\pgfpoint{173.087997pt}{59.873672pt}}
\pgfusepath{stroke}
\pgfpathmoveto{\pgfpoint{173.087997pt}{59.873672pt}}
\pgflineto{\pgfpoint{173.087997pt}{53.696838pt}}
\pgfusepath{stroke}
\pgfpathmoveto{\pgfpoint{173.106064pt}{47.519989pt}}
\pgflineto{\pgfpoint{181.997803pt}{47.519989pt}}
\pgfusepath{stroke}
\pgfpathmoveto{\pgfpoint{182.015991pt}{53.696838pt}}
\pgflineto{\pgfpoint{173.087997pt}{53.696838pt}}
\pgfusepath{stroke}
\pgfpathmoveto{\pgfpoint{182.015991pt}{59.873672pt}}
\pgflineto{\pgfpoint{173.087997pt}{59.873672pt}}
\pgfusepath{stroke}
\pgfpathmoveto{\pgfpoint{182.015991pt}{66.050522pt}}
\pgflineto{\pgfpoint{173.087997pt}{66.050522pt}}
\pgfusepath{stroke}
\pgfpathmoveto{\pgfpoint{182.015991pt}{72.227356pt}}
\pgflineto{\pgfpoint{173.087997pt}{72.227356pt}}
\pgfusepath{stroke}
\pgfpathmoveto{\pgfpoint{182.015991pt}{78.404205pt}}
\pgflineto{\pgfpoint{173.087997pt}{78.404205pt}}
\pgfusepath{stroke}
\pgfpathmoveto{\pgfpoint{182.015991pt}{84.581039pt}}
\pgflineto{\pgfpoint{173.087997pt}{84.581039pt}}
\pgfusepath{stroke}
\pgfpathmoveto{\pgfpoint{182.015991pt}{90.757896pt}}
\pgflineto{\pgfpoint{173.087997pt}{90.757896pt}}
\pgfusepath{stroke}
\pgfpathmoveto{\pgfpoint{182.015991pt}{96.934731pt}}
\pgflineto{\pgfpoint{173.087997pt}{96.934731pt}}
\pgfusepath{stroke}
\pgfpathmoveto{\pgfpoint{182.015991pt}{103.111580pt}}
\pgflineto{\pgfpoint{173.087997pt}{103.111580pt}}
\pgfusepath{stroke}
\pgfpathmoveto{\pgfpoint{182.015991pt}{109.288422pt}}
\pgflineto{\pgfpoint{173.087997pt}{109.288422pt}}
\pgfusepath{stroke}
\pgfpathmoveto{\pgfpoint{182.015991pt}{115.465263pt}}
\pgflineto{\pgfpoint{173.087997pt}{115.465263pt}}
\pgfusepath{stroke}
\pgfpathmoveto{\pgfpoint{182.015991pt}{121.642097pt}}
\pgflineto{\pgfpoint{173.087997pt}{121.642097pt}}
\pgfusepath{stroke}
\pgfpathmoveto{\pgfpoint{182.015991pt}{127.818947pt}}
\pgflineto{\pgfpoint{173.087997pt}{127.818947pt}}
\pgfusepath{stroke}
\pgfpathmoveto{\pgfpoint{182.015991pt}{133.995789pt}}
\pgflineto{\pgfpoint{173.087997pt}{133.995789pt}}
\pgfusepath{stroke}
\pgfpathmoveto{\pgfpoint{182.015991pt}{140.172638pt}}
\pgflineto{\pgfpoint{173.087997pt}{140.172638pt}}
\pgfusepath{stroke}
\pgfpathmoveto{\pgfpoint{182.015991pt}{146.349472pt}}
\pgflineto{\pgfpoint{173.087997pt}{146.349472pt}}
\pgfusepath{stroke}
\pgfpathmoveto{\pgfpoint{182.015991pt}{152.526306pt}}
\pgflineto{\pgfpoint{173.087997pt}{152.526306pt}}
\pgfusepath{stroke}
\pgfpathmoveto{\pgfpoint{182.015991pt}{158.703156pt}}
\pgflineto{\pgfpoint{173.105835pt}{158.703156pt}}
\pgfusepath{stroke}
\pgfpathmoveto{\pgfpoint{182.015991pt}{164.880005pt}}
\pgflineto{\pgfpoint{173.105835pt}{164.880005pt}}
\pgfusepath{stroke}
\pgfpathmoveto{\pgfpoint{182.015991pt}{171.056854pt}}
\pgflineto{\pgfpoint{173.105850pt}{171.056854pt}}
\pgfusepath{stroke}
\pgfpathmoveto{\pgfpoint{182.015991pt}{177.233673pt}}
\pgflineto{\pgfpoint{173.105804pt}{177.233673pt}}
\pgfusepath{stroke}
\pgfpathmoveto{\pgfpoint{173.114822pt}{183.410522pt}}
\pgflineto{\pgfpoint{181.997894pt}{183.410522pt}}
\pgfusepath{stroke}
\pgfpathmoveto{\pgfpoint{173.114822pt}{189.587372pt}}
\pgflineto{\pgfpoint{181.988922pt}{189.587372pt}}
\pgfusepath{stroke}
\pgfpathmoveto{\pgfpoint{173.114700pt}{195.764206pt}}
\pgflineto{\pgfpoint{181.979828pt}{195.764206pt}}
\pgfusepath{stroke}
\pgfpathmoveto{\pgfpoint{173.114731pt}{201.941055pt}}
\pgflineto{\pgfpoint{181.979797pt}{201.941055pt}}
\pgfusepath{stroke}
\pgfpathmoveto{\pgfpoint{173.114731pt}{208.117905pt}}
\pgflineto{\pgfpoint{181.970795pt}{208.117905pt}}
\pgfusepath{stroke}
\pgfpathmoveto{\pgfpoint{173.123734pt}{214.294739pt}}
\pgflineto{\pgfpoint{181.961777pt}{214.294739pt}}
\pgfusepath{stroke}
\pgfpathmoveto{\pgfpoint{173.123688pt}{220.471588pt}}
\pgflineto{\pgfpoint{181.952728pt}{220.471588pt}}
\pgfusepath{stroke}
\pgfpathmoveto{\pgfpoint{173.123718pt}{226.648422pt}}
\pgflineto{\pgfpoint{181.952637pt}{226.648422pt}}
\pgfusepath{stroke}
\pgfpathmoveto{\pgfpoint{173.123718pt}{232.825272pt}}
\pgflineto{\pgfpoint{181.943604pt}{232.825272pt}}
\pgfusepath{stroke}
\pgfpathmoveto{\pgfpoint{173.123657pt}{239.002106pt}}
\pgflineto{\pgfpoint{181.934601pt}{239.002106pt}}
\pgfusepath{stroke}
\pgfpathmoveto{\pgfpoint{173.132660pt}{245.178955pt}}
\pgflineto{\pgfpoint{181.925583pt}{245.178955pt}}
\pgfusepath{stroke}
\pgfpathmoveto{\pgfpoint{173.132599pt}{251.355804pt}}
\pgflineto{\pgfpoint{181.916595pt}{251.355804pt}}
\pgfusepath{stroke}
\pgfpathmoveto{\pgfpoint{173.132584pt}{257.532623pt}}
\pgflineto{\pgfpoint{181.916443pt}{257.532623pt}}
\pgfusepath{stroke}
\pgfpathmoveto{\pgfpoint{173.132584pt}{263.709473pt}}
\pgflineto{\pgfpoint{181.907455pt}{263.709473pt}}
\pgfusepath{stroke}
\pgfpathmoveto{\pgfpoint{173.132584pt}{269.886322pt}}
\pgflineto{\pgfpoint{181.898407pt}{269.886322pt}}
\pgfusepath{stroke}
\pgfpathmoveto{\pgfpoint{173.141525pt}{276.063141pt}}
\pgflineto{\pgfpoint{181.889435pt}{276.063141pt}}
\pgfusepath{stroke}
\pgfpathmoveto{\pgfpoint{173.141525pt}{282.239990pt}}
\pgflineto{\pgfpoint{181.880417pt}{282.239990pt}}
\pgfusepath{stroke}
\pgfpathmoveto{\pgfpoint{173.141525pt}{288.416840pt}}
\pgflineto{\pgfpoint{181.871384pt}{288.416840pt}}
\pgfusepath{stroke}
\pgfpathmoveto{\pgfpoint{173.141495pt}{294.593689pt}}
\pgflineto{\pgfpoint{181.866791pt}{294.593689pt}}
\pgfusepath{stroke}
\pgfpathmoveto{\pgfpoint{173.141495pt}{300.770538pt}}
\pgflineto{\pgfpoint{181.857803pt}{300.770538pt}}
\pgfusepath{stroke}
\pgfpathmoveto{\pgfpoint{173.141434pt}{306.947388pt}}
\pgflineto{\pgfpoint{181.848785pt}{306.947388pt}}
\pgfusepath{stroke}
\pgfpathmoveto{\pgfpoint{173.145920pt}{313.124207pt}}
\pgflineto{\pgfpoint{181.844177pt}{313.124207pt}}
\pgfusepath{stroke}
\pgfpathmoveto{\pgfpoint{173.145920pt}{319.301056pt}}
\pgflineto{\pgfpoint{181.835190pt}{319.301056pt}}
\pgfusepath{stroke}
\pgfpathmoveto{\pgfpoint{173.150421pt}{325.477905pt}}
\pgflineto{\pgfpoint{181.830643pt}{325.477905pt}}
\pgfusepath{stroke}
\pgfpathmoveto{\pgfpoint{173.150360pt}{331.654724pt}}
\pgflineto{\pgfpoint{181.821548pt}{331.654724pt}}
\pgfusepath{stroke}
\pgfpathmoveto{\pgfpoint{173.150452pt}{337.831604pt}}
\pgflineto{\pgfpoint{181.817001pt}{337.831604pt}}
\pgfusepath{stroke}
\pgfpathmoveto{\pgfpoint{173.154907pt}{344.008423pt}}
\pgflineto{\pgfpoint{181.808029pt}{344.008423pt}}
\pgfusepath{stroke}
\pgfpathmoveto{\pgfpoint{173.154800pt}{350.185242pt}}
\pgflineto{\pgfpoint{181.798935pt}{350.185242pt}}
\pgfusepath{stroke}
\pgfpathmoveto{\pgfpoint{173.159409pt}{356.362122pt}}
\pgflineto{\pgfpoint{181.794449pt}{356.362122pt}}
\pgfusepath{stroke}
\pgfpathmoveto{\pgfpoint{173.159286pt}{362.538940pt}}
\pgflineto{\pgfpoint{181.785416pt}{362.538940pt}}
\pgfusepath{stroke}
\pgfpathmoveto{\pgfpoint{173.163834pt}{368.715820pt}}
\pgflineto{\pgfpoint{181.780930pt}{368.715820pt}}
\pgfusepath{stroke}
\pgfpathmoveto{\pgfpoint{173.163834pt}{374.892639pt}}
\pgflineto{\pgfpoint{181.771881pt}{374.892639pt}}
\pgfusepath{stroke}
\pgfpathmoveto{\pgfpoint{173.163818pt}{381.069458pt}}
\pgflineto{\pgfpoint{181.767181pt}{381.069458pt}}
\pgfusepath{stroke}
\pgfpathmoveto{\pgfpoint{173.168274pt}{387.246338pt}}
\pgflineto{\pgfpoint{181.758301pt}{387.246338pt}}
\pgfusepath{stroke}
\pgfpathmoveto{\pgfpoint{182.015991pt}{146.349472pt}}
\pgflineto{\pgfpoint{182.015991pt}{140.172638pt}}
\pgfusepath{stroke}
\pgfpathmoveto{\pgfpoint{182.015991pt}{140.172638pt}}
\pgflineto{\pgfpoint{182.015991pt}{133.995789pt}}
\pgfusepath{stroke}
\pgfpathmoveto{\pgfpoint{182.015991pt}{152.526306pt}}
\pgflineto{\pgfpoint{182.015991pt}{146.349472pt}}
\pgfusepath{stroke}
\pgfpathmoveto{\pgfpoint{182.015991pt}{158.703156pt}}
\pgflineto{\pgfpoint{182.015991pt}{152.526306pt}}
\pgfusepath{stroke}
\pgfpathmoveto{\pgfpoint{182.015991pt}{140.172638pt}}
\pgflineto{\pgfpoint{182.034241pt}{140.172638pt}}
\pgfusepath{stroke}
\pgfpathmoveto{\pgfpoint{182.015991pt}{146.349472pt}}
\pgflineto{\pgfpoint{182.034241pt}{146.349472pt}}
\pgfusepath{stroke}
\pgfpathmoveto{\pgfpoint{182.015991pt}{152.526306pt}}
\pgflineto{\pgfpoint{182.034256pt}{152.526306pt}}
\pgfusepath{stroke}
\pgfpathmoveto{\pgfpoint{182.015991pt}{171.056854pt}}
\pgflineto{\pgfpoint{182.015991pt}{164.880005pt}}
\pgfusepath{stroke}
\pgfpathmoveto{\pgfpoint{182.015991pt}{164.880005pt}}
\pgflineto{\pgfpoint{182.015991pt}{158.703156pt}}
\pgfusepath{stroke}
\pgfpathmoveto{\pgfpoint{182.015991pt}{177.233673pt}}
\pgflineto{\pgfpoint{182.015991pt}{171.056854pt}}
\pgfusepath{stroke}
\pgfpathmoveto{\pgfpoint{182.015991pt}{158.703156pt}}
\pgflineto{\pgfpoint{182.034256pt}{158.703156pt}}
\pgfusepath{stroke}
\pgfpathmoveto{\pgfpoint{181.979828pt}{195.764206pt}}
\pgflineto{\pgfpoint{181.997925pt}{195.764206pt}}
\pgfusepath{stroke}
\pgfpathmoveto{\pgfpoint{182.015991pt}{201.941055pt}}
\pgflineto{\pgfpoint{181.979797pt}{201.941055pt}}
\pgfusepath{stroke}
\pgfpathmoveto{\pgfpoint{182.015991pt}{208.117905pt}}
\pgflineto{\pgfpoint{181.970795pt}{208.117905pt}}
\pgfusepath{stroke}
\pgfpathmoveto{\pgfpoint{182.015991pt}{214.294739pt}}
\pgflineto{\pgfpoint{181.961777pt}{214.294739pt}}
\pgfusepath{stroke}
\pgfpathmoveto{\pgfpoint{182.015991pt}{220.471588pt}}
\pgflineto{\pgfpoint{181.952728pt}{220.471588pt}}
\pgfusepath{stroke}
\pgfpathmoveto{\pgfpoint{182.015991pt}{226.648422pt}}
\pgflineto{\pgfpoint{181.952637pt}{226.648422pt}}
\pgfusepath{stroke}
\pgfpathmoveto{\pgfpoint{182.015991pt}{232.825272pt}}
\pgflineto{\pgfpoint{181.943604pt}{232.825272pt}}
\pgfusepath{stroke}
\pgfpathmoveto{\pgfpoint{182.015991pt}{239.002106pt}}
\pgflineto{\pgfpoint{181.934601pt}{239.002106pt}}
\pgfusepath{stroke}
\pgfpathmoveto{\pgfpoint{182.015991pt}{245.178955pt}}
\pgflineto{\pgfpoint{181.925583pt}{245.178955pt}}
\pgfusepath{stroke}
\pgfpathmoveto{\pgfpoint{182.015991pt}{251.355804pt}}
\pgflineto{\pgfpoint{181.916595pt}{251.355804pt}}
\pgfusepath{stroke}
\pgfpathmoveto{\pgfpoint{182.015991pt}{257.532623pt}}
\pgflineto{\pgfpoint{181.916443pt}{257.532623pt}}
\pgfusepath{stroke}
\pgfpathmoveto{\pgfpoint{182.015991pt}{263.709473pt}}
\pgflineto{\pgfpoint{181.907455pt}{263.709473pt}}
\pgfusepath{stroke}
\pgfpathmoveto{\pgfpoint{182.015991pt}{269.886322pt}}
\pgflineto{\pgfpoint{181.898407pt}{269.886322pt}}
\pgfusepath{stroke}
\pgfpathmoveto{\pgfpoint{182.015991pt}{276.063141pt}}
\pgflineto{\pgfpoint{181.889435pt}{276.063141pt}}
\pgfusepath{stroke}
\pgfpathmoveto{\pgfpoint{182.015991pt}{282.239990pt}}
\pgflineto{\pgfpoint{181.880417pt}{282.239990pt}}
\pgfusepath{stroke}
\pgfpathmoveto{\pgfpoint{182.015991pt}{288.416840pt}}
\pgflineto{\pgfpoint{181.871384pt}{288.416840pt}}
\pgfusepath{stroke}
\pgfpathmoveto{\pgfpoint{182.015991pt}{294.593689pt}}
\pgflineto{\pgfpoint{181.866791pt}{294.593689pt}}
\pgfusepath{stroke}
\pgfpathmoveto{\pgfpoint{182.015991pt}{300.770538pt}}
\pgflineto{\pgfpoint{181.857803pt}{300.770538pt}}
\pgfusepath{stroke}
\pgfpathmoveto{\pgfpoint{182.015991pt}{306.947388pt}}
\pgflineto{\pgfpoint{181.848785pt}{306.947388pt}}
\pgfusepath{stroke}
\pgfpathmoveto{\pgfpoint{182.015991pt}{313.124207pt}}
\pgflineto{\pgfpoint{181.844177pt}{313.124207pt}}
\pgfusepath{stroke}
\pgfpathmoveto{\pgfpoint{182.015991pt}{319.301056pt}}
\pgflineto{\pgfpoint{181.835190pt}{319.301056pt}}
\pgfusepath{stroke}
\pgfpathmoveto{\pgfpoint{182.015991pt}{325.477905pt}}
\pgflineto{\pgfpoint{181.830643pt}{325.477905pt}}
\pgfusepath{stroke}
\pgfpathmoveto{\pgfpoint{182.015991pt}{331.654724pt}}
\pgflineto{\pgfpoint{181.821548pt}{331.654724pt}}
\pgfusepath{stroke}
\pgfpathmoveto{\pgfpoint{182.015991pt}{337.831604pt}}
\pgflineto{\pgfpoint{181.817001pt}{337.831604pt}}
\pgfusepath{stroke}
\pgfpathmoveto{\pgfpoint{182.015991pt}{344.008423pt}}
\pgflineto{\pgfpoint{181.808029pt}{344.008423pt}}
\pgfusepath{stroke}
\pgfpathmoveto{\pgfpoint{182.015991pt}{350.185242pt}}
\pgflineto{\pgfpoint{181.798935pt}{350.185242pt}}
\pgfusepath{stroke}
\pgfpathmoveto{\pgfpoint{182.015991pt}{356.362122pt}}
\pgflineto{\pgfpoint{181.794449pt}{356.362122pt}}
\pgfusepath{stroke}
\pgfpathmoveto{\pgfpoint{181.785416pt}{362.538940pt}}
\pgflineto{\pgfpoint{182.002228pt}{362.538940pt}}
\pgfusepath{stroke}
\pgfpathmoveto{\pgfpoint{181.780930pt}{368.715820pt}}
\pgflineto{\pgfpoint{182.002228pt}{368.715820pt}}
\pgfusepath{stroke}
\pgfpathmoveto{\pgfpoint{181.771881pt}{374.892639pt}}
\pgflineto{\pgfpoint{181.997650pt}{374.892639pt}}
\pgfusepath{stroke}
\pgfpathmoveto{\pgfpoint{181.767181pt}{381.069458pt}}
\pgflineto{\pgfpoint{181.997589pt}{381.069458pt}}
\pgfusepath{stroke}
\pgfpathmoveto{\pgfpoint{181.758301pt}{387.246338pt}}
\pgflineto{\pgfpoint{181.993088pt}{387.246338pt}}
\pgfusepath{stroke}
\pgfpathmoveto{\pgfpoint{182.015991pt}{300.770538pt}}
\pgflineto{\pgfpoint{182.015991pt}{294.593689pt}}
\pgfusepath{stroke}
\pgfpathmoveto{\pgfpoint{182.015991pt}{294.593689pt}}
\pgflineto{\pgfpoint{182.015991pt}{288.416840pt}}
\pgfusepath{stroke}
\pgfpathmoveto{\pgfpoint{182.015991pt}{306.947388pt}}
\pgflineto{\pgfpoint{182.015991pt}{300.770538pt}}
\pgfusepath{stroke}
\pgfpathmoveto{\pgfpoint{182.015991pt}{294.593689pt}}
\pgflineto{\pgfpoint{182.029526pt}{294.593689pt}}
\pgfusepath{stroke}
\pgfpathmoveto{\pgfpoint{182.015991pt}{300.770538pt}}
\pgflineto{\pgfpoint{182.029556pt}{300.770538pt}}
\pgfusepath{stroke}
\pgfpathmoveto{\pgfpoint{182.015991pt}{337.831604pt}}
\pgflineto{\pgfpoint{182.015991pt}{331.654724pt}}
\pgfusepath{stroke}
\pgfpathmoveto{\pgfpoint{182.015991pt}{313.124207pt}}
\pgflineto{\pgfpoint{182.015991pt}{306.947388pt}}
\pgfusepath{stroke}
\pgfpathmoveto{\pgfpoint{182.015991pt}{319.301056pt}}
\pgflineto{\pgfpoint{182.015991pt}{313.124207pt}}
\pgfusepath{stroke}
\pgfpathmoveto{\pgfpoint{182.015991pt}{325.477905pt}}
\pgflineto{\pgfpoint{182.015991pt}{319.301056pt}}
\pgfusepath{stroke}
\pgfpathmoveto{\pgfpoint{182.015991pt}{331.654724pt}}
\pgflineto{\pgfpoint{182.015991pt}{325.477905pt}}
\pgfusepath{stroke}
\pgfpathmoveto{\pgfpoint{182.015991pt}{344.008423pt}}
\pgflineto{\pgfpoint{182.015991pt}{337.831604pt}}
\pgfusepath{stroke}
\pgfpathmoveto{\pgfpoint{182.015991pt}{350.185242pt}}
\pgflineto{\pgfpoint{182.015991pt}{344.008423pt}}
\pgfusepath{stroke}
\pgfpathmoveto{\pgfpoint{182.015991pt}{356.362122pt}}
\pgflineto{\pgfpoint{182.015991pt}{350.185242pt}}
\pgfusepath{stroke}
\pgfpathmoveto{\pgfpoint{182.015991pt}{306.947388pt}}
\pgflineto{\pgfpoint{182.029556pt}{306.947388pt}}
\pgfusepath{stroke}
\pgfpathmoveto{\pgfpoint{182.015991pt}{313.124207pt}}
\pgflineto{\pgfpoint{182.034073pt}{313.124207pt}}
\pgfusepath{stroke}
\pgfpathmoveto{\pgfpoint{182.015991pt}{362.538940pt}}
\pgflineto{\pgfpoint{182.002228pt}{362.538940pt}}
\pgfusepath{stroke}
\pgfpathmoveto{\pgfpoint{182.015991pt}{368.715820pt}}
\pgflineto{\pgfpoint{182.002228pt}{368.715820pt}}
\pgfusepath{stroke}
\pgfpathmoveto{\pgfpoint{182.015991pt}{374.892639pt}}
\pgflineto{\pgfpoint{181.997650pt}{374.892639pt}}
\pgfusepath{stroke}
\pgfpathmoveto{\pgfpoint{182.015991pt}{381.069458pt}}
\pgflineto{\pgfpoint{181.997589pt}{381.069458pt}}
\pgfusepath{stroke}
\pgfpathmoveto{\pgfpoint{182.015991pt}{387.246338pt}}
\pgflineto{\pgfpoint{181.993088pt}{387.246338pt}}
\pgfusepath{stroke}
\pgfpathmoveto{\pgfpoint{182.015991pt}{362.538940pt}}
\pgflineto{\pgfpoint{182.015991pt}{356.362122pt}}
\pgfusepath{stroke}
\pgfpathmoveto{\pgfpoint{182.015991pt}{368.715820pt}}
\pgflineto{\pgfpoint{182.015991pt}{362.538940pt}}
\pgfusepath{stroke}
\pgfpathmoveto{\pgfpoint{182.015991pt}{374.892639pt}}
\pgflineto{\pgfpoint{182.015991pt}{368.715820pt}}
\pgfusepath{stroke}
\pgfpathmoveto{\pgfpoint{182.015991pt}{381.069458pt}}
\pgflineto{\pgfpoint{182.015991pt}{374.892639pt}}
\pgfusepath{stroke}
\pgfpathmoveto{\pgfpoint{182.015991pt}{387.246338pt}}
\pgflineto{\pgfpoint{182.015991pt}{381.069458pt}}
\pgfusepath{stroke}
\pgfpathmoveto{\pgfpoint{182.015991pt}{319.301056pt}}
\pgflineto{\pgfpoint{182.034073pt}{319.301056pt}}
\pgfusepath{stroke}
\pgfpathmoveto{\pgfpoint{182.015991pt}{325.477905pt}}
\pgflineto{\pgfpoint{182.038589pt}{325.477905pt}}
\pgfusepath{stroke}
\pgfpathmoveto{\pgfpoint{182.015991pt}{331.654724pt}}
\pgflineto{\pgfpoint{182.038559pt}{331.654724pt}}
\pgfusepath{stroke}
\pgfpathmoveto{\pgfpoint{182.015991pt}{337.831604pt}}
\pgflineto{\pgfpoint{182.043106pt}{337.831604pt}}
\pgfusepath{stroke}
\pgfpathmoveto{\pgfpoint{182.015991pt}{344.008423pt}}
\pgflineto{\pgfpoint{182.043106pt}{344.008423pt}}
\pgfusepath{stroke}
\pgfpathmoveto{\pgfpoint{182.015991pt}{350.185242pt}}
\pgflineto{\pgfpoint{182.043076pt}{350.185242pt}}
\pgfusepath{stroke}
\pgfpathmoveto{\pgfpoint{182.015991pt}{356.362122pt}}
\pgflineto{\pgfpoint{182.047653pt}{356.362122pt}}
\pgfusepath{stroke}
\pgfpathmoveto{\pgfpoint{182.015991pt}{362.538940pt}}
\pgflineto{\pgfpoint{182.047607pt}{362.538940pt}}
\pgfusepath{stroke}
\pgfpathmoveto{\pgfpoint{182.015991pt}{368.715820pt}}
\pgflineto{\pgfpoint{182.052155pt}{368.715820pt}}
\pgfusepath{stroke}
\pgfpathmoveto{\pgfpoint{182.015991pt}{374.892639pt}}
\pgflineto{\pgfpoint{182.052155pt}{374.892639pt}}
\pgfusepath{stroke}
\pgfpathmoveto{\pgfpoint{182.015991pt}{381.069458pt}}
\pgflineto{\pgfpoint{182.056625pt}{381.069458pt}}
\pgfusepath{stroke}
\pgfpathmoveto{\pgfpoint{182.015991pt}{232.825272pt}}
\pgflineto{\pgfpoint{182.015991pt}{226.648422pt}}
\pgfusepath{stroke}
\pgfpathmoveto{\pgfpoint{182.015991pt}{208.117905pt}}
\pgflineto{\pgfpoint{182.015991pt}{201.941055pt}}
\pgfusepath{stroke}
\pgfpathmoveto{\pgfpoint{182.015991pt}{214.294739pt}}
\pgflineto{\pgfpoint{182.015991pt}{208.117905pt}}
\pgfusepath{stroke}
\pgfpathmoveto{\pgfpoint{182.015991pt}{220.471588pt}}
\pgflineto{\pgfpoint{182.015991pt}{214.294739pt}}
\pgfusepath{stroke}
\pgfpathmoveto{\pgfpoint{182.015991pt}{226.648422pt}}
\pgflineto{\pgfpoint{182.015991pt}{220.471588pt}}
\pgfusepath{stroke}
\pgfpathmoveto{\pgfpoint{182.015991pt}{239.002106pt}}
\pgflineto{\pgfpoint{182.015991pt}{232.825272pt}}
\pgfusepath{stroke}
\pgfpathmoveto{\pgfpoint{182.015991pt}{245.178955pt}}
\pgflineto{\pgfpoint{182.015991pt}{239.002106pt}}
\pgfusepath{stroke}
\pgfpathmoveto{\pgfpoint{182.015991pt}{251.355804pt}}
\pgflineto{\pgfpoint{182.015991pt}{245.178955pt}}
\pgfusepath{stroke}
\pgfpathmoveto{\pgfpoint{182.015991pt}{257.532623pt}}
\pgflineto{\pgfpoint{182.015991pt}{251.355804pt}}
\pgfusepath{stroke}
\pgfpathmoveto{\pgfpoint{182.015991pt}{263.709473pt}}
\pgflineto{\pgfpoint{182.015991pt}{257.532623pt}}
\pgfusepath{stroke}
\pgfpathmoveto{\pgfpoint{182.015991pt}{269.886322pt}}
\pgflineto{\pgfpoint{182.015991pt}{263.709473pt}}
\pgfusepath{stroke}
\pgfpathmoveto{\pgfpoint{182.015991pt}{276.063141pt}}
\pgflineto{\pgfpoint{182.015991pt}{269.886322pt}}
\pgfusepath{stroke}
\pgfpathmoveto{\pgfpoint{182.015991pt}{282.239990pt}}
\pgflineto{\pgfpoint{182.015991pt}{276.063141pt}}
\pgfusepath{stroke}
\pgfpathmoveto{\pgfpoint{182.015991pt}{288.416840pt}}
\pgflineto{\pgfpoint{182.015991pt}{282.239990pt}}
\pgfusepath{stroke}
\pgfpathmoveto{\pgfpoint{182.015991pt}{183.410522pt}}
\pgflineto{\pgfpoint{181.997894pt}{183.410522pt}}
\pgfusepath{stroke}
\pgfpathmoveto{\pgfpoint{182.015991pt}{189.587372pt}}
\pgflineto{\pgfpoint{181.988922pt}{189.587372pt}}
\pgfusepath{stroke}
\pgfpathmoveto{\pgfpoint{182.015991pt}{195.764206pt}}
\pgflineto{\pgfpoint{181.997925pt}{195.764206pt}}
\pgfusepath{stroke}
\pgfpathmoveto{\pgfpoint{182.015991pt}{183.410522pt}}
\pgflineto{\pgfpoint{182.015991pt}{177.233673pt}}
\pgfusepath{stroke}
\pgfpathmoveto{\pgfpoint{182.015991pt}{189.587372pt}}
\pgflineto{\pgfpoint{182.015991pt}{183.410522pt}}
\pgfusepath{stroke}
\pgfpathmoveto{\pgfpoint{182.015991pt}{164.880005pt}}
\pgflineto{\pgfpoint{182.034317pt}{164.880005pt}}
\pgfusepath{stroke}
\pgfpathmoveto{\pgfpoint{182.015991pt}{171.056854pt}}
\pgflineto{\pgfpoint{182.043335pt}{171.056854pt}}
\pgfusepath{stroke}
\pgfpathmoveto{\pgfpoint{182.015991pt}{177.233673pt}}
\pgflineto{\pgfpoint{182.043335pt}{177.233673pt}}
\pgfusepath{stroke}
\pgfpathmoveto{\pgfpoint{182.015991pt}{183.410522pt}}
\pgflineto{\pgfpoint{182.043411pt}{183.410522pt}}
\pgfusepath{stroke}
\pgfpathmoveto{\pgfpoint{182.015991pt}{201.941055pt}}
\pgflineto{\pgfpoint{182.015991pt}{195.764206pt}}
\pgfusepath{stroke}
\pgfpathmoveto{\pgfpoint{182.015991pt}{195.764206pt}}
\pgflineto{\pgfpoint{182.015991pt}{189.587372pt}}
\pgfusepath{stroke}
\pgfpathmoveto{\pgfpoint{182.015991pt}{189.587372pt}}
\pgflineto{\pgfpoint{182.043411pt}{189.587372pt}}
\pgfusepath{stroke}
\pgfpathmoveto{\pgfpoint{182.015991pt}{195.764206pt}}
\pgflineto{\pgfpoint{182.043411pt}{195.764206pt}}
\pgfusepath{stroke}
\pgfpathmoveto{\pgfpoint{182.015991pt}{201.941055pt}}
\pgflineto{\pgfpoint{182.052429pt}{201.941055pt}}
\pgfusepath{stroke}
\pgfpathmoveto{\pgfpoint{182.015991pt}{208.117905pt}}
\pgflineto{\pgfpoint{182.052475pt}{208.117905pt}}
\pgfusepath{stroke}
\pgfpathmoveto{\pgfpoint{182.015991pt}{214.294739pt}}
\pgflineto{\pgfpoint{182.052521pt}{214.294739pt}}
\pgfusepath{stroke}
\pgfpathmoveto{\pgfpoint{182.015991pt}{220.471588pt}}
\pgflineto{\pgfpoint{182.052582pt}{220.471588pt}}
\pgfusepath{stroke}
\pgfpathmoveto{\pgfpoint{182.015991pt}{226.648422pt}}
\pgflineto{\pgfpoint{182.061523pt}{226.648422pt}}
\pgfusepath{stroke}
\pgfpathmoveto{\pgfpoint{182.015991pt}{232.825272pt}}
\pgflineto{\pgfpoint{182.061539pt}{232.825272pt}}
\pgfusepath{stroke}
\pgfpathmoveto{\pgfpoint{182.015991pt}{239.002106pt}}
\pgflineto{\pgfpoint{182.061569pt}{239.002106pt}}
\pgfusepath{stroke}
\pgfpathmoveto{\pgfpoint{182.015991pt}{245.178955pt}}
\pgflineto{\pgfpoint{182.061615pt}{245.178955pt}}
\pgfusepath{stroke}
\pgfpathmoveto{\pgfpoint{182.015991pt}{251.355804pt}}
\pgflineto{\pgfpoint{182.061676pt}{251.355804pt}}
\pgfusepath{stroke}
\pgfpathmoveto{\pgfpoint{182.015991pt}{257.532623pt}}
\pgflineto{\pgfpoint{182.070663pt}{257.532623pt}}
\pgfusepath{stroke}
\pgfpathmoveto{\pgfpoint{182.015991pt}{263.709473pt}}
\pgflineto{\pgfpoint{182.070663pt}{263.709473pt}}
\pgfusepath{stroke}
\pgfpathmoveto{\pgfpoint{182.015991pt}{269.886322pt}}
\pgflineto{\pgfpoint{182.070755pt}{269.886322pt}}
\pgfusepath{stroke}
\pgfpathmoveto{\pgfpoint{182.015991pt}{276.063141pt}}
\pgflineto{\pgfpoint{182.070724pt}{276.063141pt}}
\pgfusepath{stroke}
\pgfpathmoveto{\pgfpoint{182.015991pt}{282.239990pt}}
\pgflineto{\pgfpoint{182.070755pt}{282.239990pt}}
\pgfusepath{stroke}
\pgfpathmoveto{\pgfpoint{182.015991pt}{288.416840pt}}
\pgflineto{\pgfpoint{182.070786pt}{288.416840pt}}
\pgfusepath{stroke}
\pgfpathmoveto{\pgfpoint{182.075302pt}{294.593689pt}}
\pgflineto{\pgfpoint{182.029526pt}{294.593689pt}}
\pgfusepath{stroke}
\pgfpathmoveto{\pgfpoint{182.075333pt}{300.770538pt}}
\pgflineto{\pgfpoint{182.029556pt}{300.770538pt}}
\pgfusepath{stroke}
\pgfpathmoveto{\pgfpoint{182.075424pt}{306.947388pt}}
\pgflineto{\pgfpoint{182.029556pt}{306.947388pt}}
\pgfusepath{stroke}
\pgfpathmoveto{\pgfpoint{182.079880pt}{313.124207pt}}
\pgflineto{\pgfpoint{182.034073pt}{313.124207pt}}
\pgfusepath{stroke}
\pgfpathmoveto{\pgfpoint{182.079910pt}{319.301056pt}}
\pgflineto{\pgfpoint{182.034073pt}{319.301056pt}}
\pgfusepath{stroke}
\pgfpathmoveto{\pgfpoint{182.084427pt}{325.477905pt}}
\pgflineto{\pgfpoint{182.038589pt}{325.477905pt}}
\pgfusepath{stroke}
\pgfpathmoveto{\pgfpoint{182.084427pt}{331.654724pt}}
\pgflineto{\pgfpoint{182.038559pt}{331.654724pt}}
\pgfusepath{stroke}
\pgfpathmoveto{\pgfpoint{182.089035pt}{337.831604pt}}
\pgflineto{\pgfpoint{182.043106pt}{337.831604pt}}
\pgfusepath{stroke}
\pgfpathmoveto{\pgfpoint{182.089005pt}{344.008423pt}}
\pgflineto{\pgfpoint{182.043106pt}{344.008423pt}}
\pgfusepath{stroke}
\pgfpathmoveto{\pgfpoint{182.089050pt}{350.185242pt}}
\pgflineto{\pgfpoint{182.043076pt}{350.185242pt}}
\pgfusepath{stroke}
\pgfpathmoveto{\pgfpoint{182.093552pt}{356.362122pt}}
\pgflineto{\pgfpoint{182.047653pt}{356.362122pt}}
\pgfusepath{stroke}
\pgfpathmoveto{\pgfpoint{182.093552pt}{362.538940pt}}
\pgflineto{\pgfpoint{182.047607pt}{362.538940pt}}
\pgfusepath{stroke}
\pgfpathmoveto{\pgfpoint{182.098160pt}{368.715820pt}}
\pgflineto{\pgfpoint{182.052155pt}{368.715820pt}}
\pgfusepath{stroke}
\pgfpathmoveto{\pgfpoint{182.098129pt}{374.892639pt}}
\pgflineto{\pgfpoint{182.052155pt}{374.892639pt}}
\pgfusepath{stroke}
\pgfpathmoveto{\pgfpoint{182.102661pt}{381.069458pt}}
\pgflineto{\pgfpoint{182.056625pt}{381.069458pt}}
\pgfusepath{stroke}
\pgfpathmoveto{\pgfpoint{182.015991pt}{84.581039pt}}
\pgflineto{\pgfpoint{182.015991pt}{78.404205pt}}
\pgfusepath{stroke}
\pgfpathmoveto{\pgfpoint{182.015991pt}{59.873672pt}}
\pgflineto{\pgfpoint{182.015991pt}{53.696838pt}}
\pgfusepath{stroke}
\pgfpathmoveto{\pgfpoint{182.015991pt}{66.050522pt}}
\pgflineto{\pgfpoint{182.015991pt}{59.873672pt}}
\pgfusepath{stroke}
\pgfpathmoveto{\pgfpoint{182.015991pt}{72.227356pt}}
\pgflineto{\pgfpoint{182.015991pt}{66.050522pt}}
\pgfusepath{stroke}
\pgfpathmoveto{\pgfpoint{182.015991pt}{78.404205pt}}
\pgflineto{\pgfpoint{182.015991pt}{72.227356pt}}
\pgfusepath{stroke}
\pgfpathmoveto{\pgfpoint{182.015991pt}{90.757896pt}}
\pgflineto{\pgfpoint{182.015991pt}{84.581039pt}}
\pgfusepath{stroke}
\pgfpathmoveto{\pgfpoint{182.015991pt}{96.934731pt}}
\pgflineto{\pgfpoint{182.015991pt}{90.757896pt}}
\pgfusepath{stroke}
\pgfpathmoveto{\pgfpoint{182.015991pt}{103.111580pt}}
\pgflineto{\pgfpoint{182.015991pt}{96.934731pt}}
\pgfusepath{stroke}
\pgfpathmoveto{\pgfpoint{182.015991pt}{109.288422pt}}
\pgflineto{\pgfpoint{182.015991pt}{103.111580pt}}
\pgfusepath{stroke}
\pgfpathmoveto{\pgfpoint{182.015991pt}{115.465263pt}}
\pgflineto{\pgfpoint{182.015991pt}{109.288422pt}}
\pgfusepath{stroke}
\pgfpathmoveto{\pgfpoint{182.015991pt}{121.642097pt}}
\pgflineto{\pgfpoint{182.015991pt}{115.465263pt}}
\pgfusepath{stroke}
\pgfpathmoveto{\pgfpoint{182.015991pt}{127.818947pt}}
\pgflineto{\pgfpoint{182.015991pt}{121.642097pt}}
\pgfusepath{stroke}
\pgfpathmoveto{\pgfpoint{182.015991pt}{133.995789pt}}
\pgflineto{\pgfpoint{182.015991pt}{127.818947pt}}
\pgfusepath{stroke}
\pgfpathmoveto{\pgfpoint{182.015991pt}{47.519989pt}}
\pgflineto{\pgfpoint{181.997803pt}{47.519989pt}}
\pgfusepath{stroke}
\pgfpathmoveto{\pgfpoint{182.015991pt}{53.696838pt}}
\pgflineto{\pgfpoint{182.015991pt}{47.519989pt}}
\pgfusepath{stroke}
\pgfpathmoveto{\pgfpoint{182.015991pt}{47.519989pt}}
\pgflineto{\pgfpoint{190.925919pt}{47.519989pt}}
\pgfusepath{stroke}
\pgfpathmoveto{\pgfpoint{182.015991pt}{53.696838pt}}
\pgflineto{\pgfpoint{190.925934pt}{53.696838pt}}
\pgfusepath{stroke}
\pgfpathmoveto{\pgfpoint{182.015991pt}{59.873672pt}}
\pgflineto{\pgfpoint{190.925949pt}{59.873672pt}}
\pgfusepath{stroke}
\pgfpathmoveto{\pgfpoint{190.943985pt}{66.050522pt}}
\pgflineto{\pgfpoint{182.015991pt}{66.050522pt}}
\pgfusepath{stroke}
\pgfpathmoveto{\pgfpoint{190.943985pt}{72.227356pt}}
\pgflineto{\pgfpoint{182.015991pt}{72.227356pt}}
\pgfusepath{stroke}
\pgfpathmoveto{\pgfpoint{190.943985pt}{78.404205pt}}
\pgflineto{\pgfpoint{182.015991pt}{78.404205pt}}
\pgfusepath{stroke}
\pgfpathmoveto{\pgfpoint{190.943985pt}{84.581039pt}}
\pgflineto{\pgfpoint{182.015991pt}{84.581039pt}}
\pgfusepath{stroke}
\pgfpathmoveto{\pgfpoint{190.943985pt}{90.757896pt}}
\pgflineto{\pgfpoint{182.015991pt}{90.757896pt}}
\pgfusepath{stroke}
\pgfpathmoveto{\pgfpoint{190.943985pt}{96.934731pt}}
\pgflineto{\pgfpoint{182.015991pt}{96.934731pt}}
\pgfusepath{stroke}
\pgfpathmoveto{\pgfpoint{190.943985pt}{103.111580pt}}
\pgflineto{\pgfpoint{182.015991pt}{103.111580pt}}
\pgfusepath{stroke}
\pgfpathmoveto{\pgfpoint{190.943985pt}{109.288422pt}}
\pgflineto{\pgfpoint{182.015991pt}{109.288422pt}}
\pgfusepath{stroke}
\pgfpathmoveto{\pgfpoint{190.943985pt}{115.465263pt}}
\pgflineto{\pgfpoint{182.015991pt}{115.465263pt}}
\pgfusepath{stroke}
\pgfpathmoveto{\pgfpoint{190.943985pt}{121.642097pt}}
\pgflineto{\pgfpoint{182.015991pt}{121.642097pt}}
\pgfusepath{stroke}
\pgfpathmoveto{\pgfpoint{190.943985pt}{127.818947pt}}
\pgflineto{\pgfpoint{182.015991pt}{127.818947pt}}
\pgfusepath{stroke}
\pgfpathmoveto{\pgfpoint{190.943985pt}{133.995789pt}}
\pgflineto{\pgfpoint{182.015991pt}{133.995789pt}}
\pgfusepath{stroke}
\pgfpathmoveto{\pgfpoint{190.943985pt}{140.172638pt}}
\pgflineto{\pgfpoint{182.034241pt}{140.172638pt}}
\pgfusepath{stroke}
\pgfpathmoveto{\pgfpoint{190.943985pt}{146.349472pt}}
\pgflineto{\pgfpoint{182.034241pt}{146.349472pt}}
\pgfusepath{stroke}
\pgfpathmoveto{\pgfpoint{190.943985pt}{152.526306pt}}
\pgflineto{\pgfpoint{182.034256pt}{152.526306pt}}
\pgfusepath{stroke}
\pgfpathmoveto{\pgfpoint{190.943985pt}{158.703156pt}}
\pgflineto{\pgfpoint{182.034256pt}{158.703156pt}}
\pgfusepath{stroke}
\pgfpathmoveto{\pgfpoint{190.943985pt}{164.880005pt}}
\pgflineto{\pgfpoint{182.034317pt}{164.880005pt}}
\pgfusepath{stroke}
\pgfpathmoveto{\pgfpoint{190.943985pt}{171.056854pt}}
\pgflineto{\pgfpoint{182.043335pt}{171.056854pt}}
\pgfusepath{stroke}
\pgfpathmoveto{\pgfpoint{190.943985pt}{177.233673pt}}
\pgflineto{\pgfpoint{182.043335pt}{177.233673pt}}
\pgfusepath{stroke}
\pgfpathmoveto{\pgfpoint{190.943985pt}{183.410522pt}}
\pgflineto{\pgfpoint{182.043411pt}{183.410522pt}}
\pgfusepath{stroke}
\pgfpathmoveto{\pgfpoint{190.943985pt}{189.587372pt}}
\pgflineto{\pgfpoint{182.043411pt}{189.587372pt}}
\pgfusepath{stroke}
\pgfpathmoveto{\pgfpoint{182.043411pt}{195.764206pt}}
\pgflineto{\pgfpoint{190.925858pt}{195.764206pt}}
\pgfusepath{stroke}
\pgfpathmoveto{\pgfpoint{182.052429pt}{201.941055pt}}
\pgflineto{\pgfpoint{190.916901pt}{201.941055pt}}
\pgfusepath{stroke}
\pgfpathmoveto{\pgfpoint{182.052475pt}{208.117905pt}}
\pgflineto{\pgfpoint{190.907822pt}{208.117905pt}}
\pgfusepath{stroke}
\pgfpathmoveto{\pgfpoint{182.052521pt}{214.294739pt}}
\pgflineto{\pgfpoint{190.907761pt}{214.294739pt}}
\pgfusepath{stroke}
\pgfpathmoveto{\pgfpoint{182.052582pt}{220.471588pt}}
\pgflineto{\pgfpoint{190.898727pt}{220.471588pt}}
\pgfusepath{stroke}
\pgfpathmoveto{\pgfpoint{182.061523pt}{226.648422pt}}
\pgflineto{\pgfpoint{190.889755pt}{226.648422pt}}
\pgfusepath{stroke}
\pgfpathmoveto{\pgfpoint{182.061539pt}{232.825272pt}}
\pgflineto{\pgfpoint{190.880692pt}{232.825272pt}}
\pgfusepath{stroke}
\pgfpathmoveto{\pgfpoint{182.061569pt}{239.002106pt}}
\pgflineto{\pgfpoint{190.871689pt}{239.002106pt}}
\pgfusepath{stroke}
\pgfpathmoveto{\pgfpoint{182.061615pt}{245.178955pt}}
\pgflineto{\pgfpoint{190.871597pt}{245.178955pt}}
\pgfusepath{stroke}
\pgfpathmoveto{\pgfpoint{182.061676pt}{251.355804pt}}
\pgflineto{\pgfpoint{190.862579pt}{251.355804pt}}
\pgfusepath{stroke}
\pgfpathmoveto{\pgfpoint{182.070663pt}{257.532623pt}}
\pgflineto{\pgfpoint{190.853577pt}{257.532623pt}}
\pgfusepath{stroke}
\pgfpathmoveto{\pgfpoint{182.070663pt}{263.709473pt}}
\pgflineto{\pgfpoint{190.844604pt}{263.709473pt}}
\pgfusepath{stroke}
\pgfpathmoveto{\pgfpoint{182.070755pt}{269.886322pt}}
\pgflineto{\pgfpoint{190.844467pt}{269.886322pt}}
\pgfusepath{stroke}
\pgfpathmoveto{\pgfpoint{182.070724pt}{276.063141pt}}
\pgflineto{\pgfpoint{190.835403pt}{276.063141pt}}
\pgfusepath{stroke}
\pgfpathmoveto{\pgfpoint{182.070755pt}{282.239990pt}}
\pgflineto{\pgfpoint{190.821899pt}{282.239990pt}}
\pgfusepath{stroke}
\pgfpathmoveto{\pgfpoint{182.070786pt}{288.416840pt}}
\pgflineto{\pgfpoint{190.812881pt}{288.416840pt}}
\pgfusepath{stroke}
\pgfpathmoveto{\pgfpoint{182.075302pt}{294.593689pt}}
\pgflineto{\pgfpoint{190.808350pt}{294.593689pt}}
\pgfusepath{stroke}
\pgfpathmoveto{\pgfpoint{182.075333pt}{300.770538pt}}
\pgflineto{\pgfpoint{190.799332pt}{300.770538pt}}
\pgfusepath{stroke}
\pgfpathmoveto{\pgfpoint{182.075424pt}{306.947388pt}}
\pgflineto{\pgfpoint{190.794739pt}{306.947388pt}}
\pgfusepath{stroke}
\pgfpathmoveto{\pgfpoint{182.079880pt}{313.124207pt}}
\pgflineto{\pgfpoint{190.785736pt}{313.124207pt}}
\pgfusepath{stroke}
\pgfpathmoveto{\pgfpoint{182.079910pt}{319.301056pt}}
\pgflineto{\pgfpoint{190.776718pt}{319.301056pt}}
\pgfusepath{stroke}
\pgfpathmoveto{\pgfpoint{182.084427pt}{325.477905pt}}
\pgflineto{\pgfpoint{190.772186pt}{325.477905pt}}
\pgfusepath{stroke}
\pgfpathmoveto{\pgfpoint{182.084427pt}{331.654724pt}}
\pgflineto{\pgfpoint{190.763107pt}{331.654724pt}}
\pgfusepath{stroke}
\pgfpathmoveto{\pgfpoint{182.089035pt}{337.831604pt}}
\pgflineto{\pgfpoint{190.758652pt}{337.831604pt}}
\pgfusepath{stroke}
\pgfpathmoveto{\pgfpoint{182.089005pt}{344.008423pt}}
\pgflineto{\pgfpoint{190.749588pt}{344.008423pt}}
\pgfusepath{stroke}
\pgfpathmoveto{\pgfpoint{182.089050pt}{350.185242pt}}
\pgflineto{\pgfpoint{190.744934pt}{350.185242pt}}
\pgfusepath{stroke}
\pgfpathmoveto{\pgfpoint{182.093552pt}{356.362122pt}}
\pgflineto{\pgfpoint{190.736038pt}{356.362122pt}}
\pgfusepath{stroke}
\pgfpathmoveto{\pgfpoint{182.093552pt}{362.538940pt}}
\pgflineto{\pgfpoint{190.726944pt}{362.538940pt}}
\pgfusepath{stroke}
\pgfpathmoveto{\pgfpoint{182.098160pt}{368.715820pt}}
\pgflineto{\pgfpoint{190.722504pt}{368.715820pt}}
\pgfusepath{stroke}
\pgfpathmoveto{\pgfpoint{182.098129pt}{374.892639pt}}
\pgflineto{\pgfpoint{190.713409pt}{374.892639pt}}
\pgfusepath{stroke}
\pgfpathmoveto{\pgfpoint{182.102661pt}{381.069458pt}}
\pgflineto{\pgfpoint{190.708878pt}{381.069458pt}}
\pgfusepath{stroke}
\pgfpathmoveto{\pgfpoint{190.943985pt}{158.703156pt}}
\pgflineto{\pgfpoint{190.943985pt}{152.526306pt}}
\pgfusepath{stroke}
\pgfpathmoveto{\pgfpoint{190.943985pt}{152.526306pt}}
\pgflineto{\pgfpoint{190.943985pt}{146.349472pt}}
\pgfusepath{stroke}
\pgfpathmoveto{\pgfpoint{190.943985pt}{164.880005pt}}
\pgflineto{\pgfpoint{190.943985pt}{158.703156pt}}
\pgfusepath{stroke}
\pgfpathmoveto{\pgfpoint{190.943985pt}{171.056854pt}}
\pgflineto{\pgfpoint{190.943985pt}{164.880005pt}}
\pgfusepath{stroke}
\pgfpathmoveto{\pgfpoint{190.943985pt}{152.526306pt}}
\pgflineto{\pgfpoint{190.962036pt}{152.526306pt}}
\pgfusepath{stroke}
\pgfpathmoveto{\pgfpoint{190.943985pt}{158.703156pt}}
\pgflineto{\pgfpoint{190.962051pt}{158.703156pt}}
\pgfusepath{stroke}
\pgfpathmoveto{\pgfpoint{190.943985pt}{164.880005pt}}
\pgflineto{\pgfpoint{190.962082pt}{164.880005pt}}
\pgfusepath{stroke}
\pgfpathmoveto{\pgfpoint{190.943985pt}{183.410522pt}}
\pgflineto{\pgfpoint{190.943985pt}{177.233673pt}}
\pgfusepath{stroke}
\pgfpathmoveto{\pgfpoint{190.943985pt}{177.233673pt}}
\pgflineto{\pgfpoint{190.943985pt}{171.056854pt}}
\pgfusepath{stroke}
\pgfpathmoveto{\pgfpoint{190.943985pt}{189.587372pt}}
\pgflineto{\pgfpoint{190.943985pt}{183.410522pt}}
\pgfusepath{stroke}
\pgfpathmoveto{\pgfpoint{190.943985pt}{171.056854pt}}
\pgflineto{\pgfpoint{190.962082pt}{171.056854pt}}
\pgfusepath{stroke}
\pgfpathmoveto{\pgfpoint{190.907822pt}{208.117905pt}}
\pgflineto{\pgfpoint{190.926010pt}{208.117905pt}}
\pgfusepath{stroke}
\pgfpathmoveto{\pgfpoint{190.943985pt}{214.294739pt}}
\pgflineto{\pgfpoint{190.907761pt}{214.294739pt}}
\pgfusepath{stroke}
\pgfpathmoveto{\pgfpoint{190.943985pt}{220.471588pt}}
\pgflineto{\pgfpoint{190.898727pt}{220.471588pt}}
\pgfusepath{stroke}
\pgfpathmoveto{\pgfpoint{190.943985pt}{226.648422pt}}
\pgflineto{\pgfpoint{190.889755pt}{226.648422pt}}
\pgfusepath{stroke}
\pgfpathmoveto{\pgfpoint{190.943985pt}{232.825272pt}}
\pgflineto{\pgfpoint{190.880692pt}{232.825272pt}}
\pgfusepath{stroke}
\pgfpathmoveto{\pgfpoint{190.943985pt}{239.002106pt}}
\pgflineto{\pgfpoint{190.871689pt}{239.002106pt}}
\pgfusepath{stroke}
\pgfpathmoveto{\pgfpoint{190.943985pt}{245.178955pt}}
\pgflineto{\pgfpoint{190.871597pt}{245.178955pt}}
\pgfusepath{stroke}
\pgfpathmoveto{\pgfpoint{190.943985pt}{251.355804pt}}
\pgflineto{\pgfpoint{190.862579pt}{251.355804pt}}
\pgfusepath{stroke}
\pgfpathmoveto{\pgfpoint{190.943985pt}{257.532623pt}}
\pgflineto{\pgfpoint{190.853577pt}{257.532623pt}}
\pgfusepath{stroke}
\pgfpathmoveto{\pgfpoint{190.943985pt}{263.709473pt}}
\pgflineto{\pgfpoint{190.844604pt}{263.709473pt}}
\pgfusepath{stroke}
\pgfpathmoveto{\pgfpoint{190.943985pt}{269.886322pt}}
\pgflineto{\pgfpoint{190.844467pt}{269.886322pt}}
\pgfusepath{stroke}
\pgfpathmoveto{\pgfpoint{190.943985pt}{276.063141pt}}
\pgflineto{\pgfpoint{190.835403pt}{276.063141pt}}
\pgfusepath{stroke}
\pgfpathmoveto{\pgfpoint{190.943985pt}{282.239990pt}}
\pgflineto{\pgfpoint{190.821899pt}{282.239990pt}}
\pgfusepath{stroke}
\pgfpathmoveto{\pgfpoint{190.943985pt}{288.416840pt}}
\pgflineto{\pgfpoint{190.812881pt}{288.416840pt}}
\pgfusepath{stroke}
\pgfpathmoveto{\pgfpoint{190.943985pt}{294.593689pt}}
\pgflineto{\pgfpoint{190.808350pt}{294.593689pt}}
\pgfusepath{stroke}
\pgfpathmoveto{\pgfpoint{190.943985pt}{300.770538pt}}
\pgflineto{\pgfpoint{190.799332pt}{300.770538pt}}
\pgfusepath{stroke}
\pgfpathmoveto{\pgfpoint{190.943985pt}{306.947388pt}}
\pgflineto{\pgfpoint{190.794739pt}{306.947388pt}}
\pgfusepath{stroke}
\pgfpathmoveto{\pgfpoint{190.943985pt}{313.124207pt}}
\pgflineto{\pgfpoint{190.785736pt}{313.124207pt}}
\pgfusepath{stroke}
\pgfpathmoveto{\pgfpoint{190.943985pt}{319.301056pt}}
\pgflineto{\pgfpoint{190.776718pt}{319.301056pt}}
\pgfusepath{stroke}
\pgfpathmoveto{\pgfpoint{190.943985pt}{325.477905pt}}
\pgflineto{\pgfpoint{190.772186pt}{325.477905pt}}
\pgfusepath{stroke}
\pgfpathmoveto{\pgfpoint{190.943985pt}{331.654724pt}}
\pgflineto{\pgfpoint{190.763107pt}{331.654724pt}}
\pgfusepath{stroke}
\pgfpathmoveto{\pgfpoint{190.943985pt}{337.831604pt}}
\pgflineto{\pgfpoint{190.758652pt}{337.831604pt}}
\pgfusepath{stroke}
\pgfpathmoveto{\pgfpoint{190.943985pt}{344.008423pt}}
\pgflineto{\pgfpoint{190.749588pt}{344.008423pt}}
\pgfusepath{stroke}
\pgfpathmoveto{\pgfpoint{190.943985pt}{350.185242pt}}
\pgflineto{\pgfpoint{190.744934pt}{350.185242pt}}
\pgfusepath{stroke}
\pgfpathmoveto{\pgfpoint{190.943985pt}{356.362122pt}}
\pgflineto{\pgfpoint{190.736038pt}{356.362122pt}}
\pgfusepath{stroke}
\pgfpathmoveto{\pgfpoint{190.943985pt}{362.538940pt}}
\pgflineto{\pgfpoint{190.726944pt}{362.538940pt}}
\pgfusepath{stroke}
\pgfpathmoveto{\pgfpoint{190.943985pt}{368.715820pt}}
\pgflineto{\pgfpoint{190.722504pt}{368.715820pt}}
\pgfusepath{stroke}
\pgfpathmoveto{\pgfpoint{190.713409pt}{374.892639pt}}
\pgflineto{\pgfpoint{190.930573pt}{374.892639pt}}
\pgfusepath{stroke}
\pgfpathmoveto{\pgfpoint{190.708878pt}{381.069458pt}}
\pgflineto{\pgfpoint{190.930588pt}{381.069458pt}}
\pgfusepath{stroke}
\pgfpathmoveto{\pgfpoint{190.943985pt}{313.124207pt}}
\pgflineto{\pgfpoint{190.943985pt}{306.947388pt}}
\pgfusepath{stroke}
\pgfpathmoveto{\pgfpoint{190.943985pt}{306.947388pt}}
\pgflineto{\pgfpoint{190.943985pt}{300.770538pt}}
\pgfusepath{stroke}
\pgfpathmoveto{\pgfpoint{190.943985pt}{319.301056pt}}
\pgflineto{\pgfpoint{190.943985pt}{313.124207pt}}
\pgfusepath{stroke}
\pgfpathmoveto{\pgfpoint{190.943985pt}{306.947388pt}}
\pgflineto{\pgfpoint{190.957458pt}{306.947388pt}}
\pgfusepath{stroke}
\pgfpathmoveto{\pgfpoint{190.943985pt}{313.124207pt}}
\pgflineto{\pgfpoint{190.957428pt}{313.124207pt}}
\pgfusepath{stroke}
\pgfpathmoveto{\pgfpoint{190.943985pt}{350.185242pt}}
\pgflineto{\pgfpoint{190.943985pt}{344.008423pt}}
\pgfusepath{stroke}
\pgfpathmoveto{\pgfpoint{190.943985pt}{325.477905pt}}
\pgflineto{\pgfpoint{190.943985pt}{319.301056pt}}
\pgfusepath{stroke}
\pgfpathmoveto{\pgfpoint{190.943985pt}{331.654724pt}}
\pgflineto{\pgfpoint{190.943985pt}{325.477905pt}}
\pgfusepath{stroke}
\pgfpathmoveto{\pgfpoint{190.943985pt}{337.831604pt}}
\pgflineto{\pgfpoint{190.943985pt}{331.654724pt}}
\pgfusepath{stroke}
\pgfpathmoveto{\pgfpoint{190.943985pt}{344.008423pt}}
\pgflineto{\pgfpoint{190.943985pt}{337.831604pt}}
\pgfusepath{stroke}
\pgfpathmoveto{\pgfpoint{190.943985pt}{356.362122pt}}
\pgflineto{\pgfpoint{190.943985pt}{350.185242pt}}
\pgfusepath{stroke}
\pgfpathmoveto{\pgfpoint{190.943985pt}{362.538940pt}}
\pgflineto{\pgfpoint{190.943985pt}{356.362122pt}}
\pgfusepath{stroke}
\pgfpathmoveto{\pgfpoint{190.943985pt}{368.715820pt}}
\pgflineto{\pgfpoint{190.943985pt}{362.538940pt}}
\pgfusepath{stroke}
\pgfpathmoveto{\pgfpoint{190.943985pt}{319.301056pt}}
\pgflineto{\pgfpoint{190.957413pt}{319.301056pt}}
\pgfusepath{stroke}
\pgfpathmoveto{\pgfpoint{190.943985pt}{325.477905pt}}
\pgflineto{\pgfpoint{190.961929pt}{325.477905pt}}
\pgfusepath{stroke}
\pgfpathmoveto{\pgfpoint{190.943985pt}{374.892639pt}}
\pgflineto{\pgfpoint{190.930573pt}{374.892639pt}}
\pgfusepath{stroke}
\pgfpathmoveto{\pgfpoint{190.943985pt}{381.069458pt}}
\pgflineto{\pgfpoint{190.930588pt}{381.069458pt}}
\pgfusepath{stroke}
\pgfpathmoveto{\pgfpoint{190.943985pt}{374.892639pt}}
\pgflineto{\pgfpoint{190.943985pt}{368.715820pt}}
\pgfusepath{stroke}
\pgfpathmoveto{\pgfpoint{190.943985pt}{381.069458pt}}
\pgflineto{\pgfpoint{190.943985pt}{374.892639pt}}
\pgfusepath{stroke}
\pgfpathmoveto{\pgfpoint{190.943985pt}{331.654724pt}}
\pgflineto{\pgfpoint{190.961899pt}{331.654724pt}}
\pgfusepath{stroke}
\pgfpathmoveto{\pgfpoint{190.943985pt}{337.831604pt}}
\pgflineto{\pgfpoint{190.966446pt}{337.831604pt}}
\pgfusepath{stroke}
\pgfpathmoveto{\pgfpoint{190.943985pt}{344.008423pt}}
\pgflineto{\pgfpoint{190.966385pt}{344.008423pt}}
\pgfusepath{stroke}
\pgfpathmoveto{\pgfpoint{190.943985pt}{350.185242pt}}
\pgflineto{\pgfpoint{190.970871pt}{350.185242pt}}
\pgfusepath{stroke}
\pgfpathmoveto{\pgfpoint{190.943985pt}{356.362122pt}}
\pgflineto{\pgfpoint{190.970901pt}{356.362122pt}}
\pgfusepath{stroke}
\pgfpathmoveto{\pgfpoint{190.943985pt}{362.538940pt}}
\pgflineto{\pgfpoint{190.970871pt}{362.538940pt}}
\pgfusepath{stroke}
\pgfpathmoveto{\pgfpoint{190.943985pt}{368.715820pt}}
\pgflineto{\pgfpoint{190.975418pt}{368.715820pt}}
\pgfusepath{stroke}
\pgfpathmoveto{\pgfpoint{190.943985pt}{374.892639pt}}
\pgflineto{\pgfpoint{190.975372pt}{374.892639pt}}
\pgfusepath{stroke}
\pgfpathmoveto{\pgfpoint{190.943985pt}{251.355804pt}}
\pgflineto{\pgfpoint{190.943985pt}{245.178955pt}}
\pgfusepath{stroke}
\pgfpathmoveto{\pgfpoint{190.943985pt}{220.471588pt}}
\pgflineto{\pgfpoint{190.943985pt}{214.294739pt}}
\pgfusepath{stroke}
\pgfpathmoveto{\pgfpoint{190.943985pt}{226.648422pt}}
\pgflineto{\pgfpoint{190.943985pt}{220.471588pt}}
\pgfusepath{stroke}
\pgfpathmoveto{\pgfpoint{190.943985pt}{232.825272pt}}
\pgflineto{\pgfpoint{190.943985pt}{226.648422pt}}
\pgfusepath{stroke}
\pgfpathmoveto{\pgfpoint{190.943985pt}{239.002106pt}}
\pgflineto{\pgfpoint{190.943985pt}{232.825272pt}}
\pgfusepath{stroke}
\pgfpathmoveto{\pgfpoint{190.943985pt}{245.178955pt}}
\pgflineto{\pgfpoint{190.943985pt}{239.002106pt}}
\pgfusepath{stroke}
\pgfpathmoveto{\pgfpoint{190.943985pt}{257.532623pt}}
\pgflineto{\pgfpoint{190.943985pt}{251.355804pt}}
\pgfusepath{stroke}
\pgfpathmoveto{\pgfpoint{190.943985pt}{263.709473pt}}
\pgflineto{\pgfpoint{190.943985pt}{257.532623pt}}
\pgfusepath{stroke}
\pgfpathmoveto{\pgfpoint{190.943985pt}{269.886322pt}}
\pgflineto{\pgfpoint{190.943985pt}{263.709473pt}}
\pgfusepath{stroke}
\pgfpathmoveto{\pgfpoint{190.943985pt}{276.063141pt}}
\pgflineto{\pgfpoint{190.943985pt}{269.886322pt}}
\pgfusepath{stroke}
\pgfpathmoveto{\pgfpoint{190.943985pt}{282.239990pt}}
\pgflineto{\pgfpoint{190.943985pt}{276.063141pt}}
\pgfusepath{stroke}
\pgfpathmoveto{\pgfpoint{190.943985pt}{288.416840pt}}
\pgflineto{\pgfpoint{190.943985pt}{282.239990pt}}
\pgfusepath{stroke}
\pgfpathmoveto{\pgfpoint{190.943985pt}{294.593689pt}}
\pgflineto{\pgfpoint{190.943985pt}{288.416840pt}}
\pgfusepath{stroke}
\pgfpathmoveto{\pgfpoint{190.943985pt}{300.770538pt}}
\pgflineto{\pgfpoint{190.943985pt}{294.593689pt}}
\pgfusepath{stroke}
\pgfpathmoveto{\pgfpoint{190.943985pt}{195.764206pt}}
\pgflineto{\pgfpoint{190.925858pt}{195.764206pt}}
\pgfusepath{stroke}
\pgfpathmoveto{\pgfpoint{190.943985pt}{201.941055pt}}
\pgflineto{\pgfpoint{190.916901pt}{201.941055pt}}
\pgfusepath{stroke}
\pgfpathmoveto{\pgfpoint{190.943985pt}{208.117905pt}}
\pgflineto{\pgfpoint{190.926010pt}{208.117905pt}}
\pgfusepath{stroke}
\pgfpathmoveto{\pgfpoint{190.943985pt}{195.764206pt}}
\pgflineto{\pgfpoint{190.943985pt}{189.587372pt}}
\pgfusepath{stroke}
\pgfpathmoveto{\pgfpoint{190.943985pt}{201.941055pt}}
\pgflineto{\pgfpoint{190.943985pt}{195.764206pt}}
\pgfusepath{stroke}
\pgfpathmoveto{\pgfpoint{190.943985pt}{177.233673pt}}
\pgflineto{\pgfpoint{190.962082pt}{177.233673pt}}
\pgfusepath{stroke}
\pgfpathmoveto{\pgfpoint{190.943985pt}{183.410522pt}}
\pgflineto{\pgfpoint{190.971100pt}{183.410522pt}}
\pgfusepath{stroke}
\pgfpathmoveto{\pgfpoint{190.943985pt}{189.587372pt}}
\pgflineto{\pgfpoint{190.971100pt}{189.587372pt}}
\pgfusepath{stroke}
\pgfpathmoveto{\pgfpoint{190.943985pt}{195.764206pt}}
\pgflineto{\pgfpoint{190.971085pt}{195.764206pt}}
\pgfusepath{stroke}
\pgfpathmoveto{\pgfpoint{190.943985pt}{214.294739pt}}
\pgflineto{\pgfpoint{190.943985pt}{208.117905pt}}
\pgfusepath{stroke}
\pgfpathmoveto{\pgfpoint{190.943985pt}{208.117905pt}}
\pgflineto{\pgfpoint{190.943985pt}{201.941055pt}}
\pgfusepath{stroke}
\pgfpathmoveto{\pgfpoint{190.943985pt}{201.941055pt}}
\pgflineto{\pgfpoint{190.971115pt}{201.941055pt}}
\pgfusepath{stroke}
\pgfpathmoveto{\pgfpoint{190.943985pt}{208.117905pt}}
\pgflineto{\pgfpoint{190.971115pt}{208.117905pt}}
\pgfusepath{stroke}
\pgfpathmoveto{\pgfpoint{190.943985pt}{214.294739pt}}
\pgflineto{\pgfpoint{190.980133pt}{214.294739pt}}
\pgfusepath{stroke}
\pgfpathmoveto{\pgfpoint{190.943985pt}{220.471588pt}}
\pgflineto{\pgfpoint{190.980133pt}{220.471588pt}}
\pgfusepath{stroke}
\pgfpathmoveto{\pgfpoint{190.943985pt}{226.648422pt}}
\pgflineto{\pgfpoint{190.980164pt}{226.648422pt}}
\pgfusepath{stroke}
\pgfpathmoveto{\pgfpoint{190.943985pt}{232.825272pt}}
\pgflineto{\pgfpoint{190.980164pt}{232.825272pt}}
\pgfusepath{stroke}
\pgfpathmoveto{\pgfpoint{190.943985pt}{239.002106pt}}
\pgflineto{\pgfpoint{190.980164pt}{239.002106pt}}
\pgfusepath{stroke}
\pgfpathmoveto{\pgfpoint{190.943985pt}{245.178955pt}}
\pgflineto{\pgfpoint{190.989166pt}{245.178955pt}}
\pgfusepath{stroke}
\pgfpathmoveto{\pgfpoint{190.943985pt}{251.355804pt}}
\pgflineto{\pgfpoint{190.989166pt}{251.355804pt}}
\pgfusepath{stroke}
\pgfpathmoveto{\pgfpoint{190.943985pt}{257.532623pt}}
\pgflineto{\pgfpoint{190.989212pt}{257.532623pt}}
\pgfusepath{stroke}
\pgfpathmoveto{\pgfpoint{190.943985pt}{263.709473pt}}
\pgflineto{\pgfpoint{190.989182pt}{263.709473pt}}
\pgfusepath{stroke}
\pgfpathmoveto{\pgfpoint{190.943985pt}{269.886322pt}}
\pgflineto{\pgfpoint{190.998169pt}{269.886322pt}}
\pgfusepath{stroke}
\pgfpathmoveto{\pgfpoint{190.943985pt}{276.063141pt}}
\pgflineto{\pgfpoint{190.998138pt}{276.063141pt}}
\pgfusepath{stroke}
\pgfpathmoveto{\pgfpoint{190.943985pt}{282.239990pt}}
\pgflineto{\pgfpoint{190.993683pt}{282.239990pt}}
\pgfusepath{stroke}
\pgfpathmoveto{\pgfpoint{190.943985pt}{288.416840pt}}
\pgflineto{\pgfpoint{190.993683pt}{288.416840pt}}
\pgfusepath{stroke}
\pgfpathmoveto{\pgfpoint{190.943985pt}{294.593689pt}}
\pgflineto{\pgfpoint{190.998199pt}{294.593689pt}}
\pgfusepath{stroke}
\pgfpathmoveto{\pgfpoint{190.943985pt}{300.770538pt}}
\pgflineto{\pgfpoint{190.998199pt}{300.770538pt}}
\pgfusepath{stroke}
\pgfpathmoveto{\pgfpoint{191.002716pt}{306.947388pt}}
\pgflineto{\pgfpoint{190.957458pt}{306.947388pt}}
\pgfusepath{stroke}
\pgfpathmoveto{\pgfpoint{191.002716pt}{313.124207pt}}
\pgflineto{\pgfpoint{190.957428pt}{313.124207pt}}
\pgfusepath{stroke}
\pgfpathmoveto{\pgfpoint{191.002747pt}{319.301056pt}}
\pgflineto{\pgfpoint{190.957413pt}{319.301056pt}}
\pgfusepath{stroke}
\pgfpathmoveto{\pgfpoint{191.007248pt}{325.477905pt}}
\pgflineto{\pgfpoint{190.961929pt}{325.477905pt}}
\pgfusepath{stroke}
\pgfpathmoveto{\pgfpoint{191.007217pt}{331.654724pt}}
\pgflineto{\pgfpoint{190.961899pt}{331.654724pt}}
\pgfusepath{stroke}
\pgfpathmoveto{\pgfpoint{191.011795pt}{337.831604pt}}
\pgflineto{\pgfpoint{190.966446pt}{337.831604pt}}
\pgfusepath{stroke}
\pgfpathmoveto{\pgfpoint{191.011765pt}{344.008423pt}}
\pgflineto{\pgfpoint{190.966385pt}{344.008423pt}}
\pgfusepath{stroke}
\pgfpathmoveto{\pgfpoint{191.016220pt}{350.185242pt}}
\pgflineto{\pgfpoint{190.970871pt}{350.185242pt}}
\pgfusepath{stroke}
\pgfpathmoveto{\pgfpoint{191.016312pt}{356.362122pt}}
\pgflineto{\pgfpoint{190.970901pt}{356.362122pt}}
\pgfusepath{stroke}
\pgfpathmoveto{\pgfpoint{191.016251pt}{362.538940pt}}
\pgflineto{\pgfpoint{190.970871pt}{362.538940pt}}
\pgfusepath{stroke}
\pgfpathmoveto{\pgfpoint{191.020828pt}{368.715820pt}}
\pgflineto{\pgfpoint{190.975418pt}{368.715820pt}}
\pgfusepath{stroke}
\pgfpathmoveto{\pgfpoint{191.020798pt}{374.892639pt}}
\pgflineto{\pgfpoint{190.975372pt}{374.892639pt}}
\pgfusepath{stroke}
\pgfpathmoveto{\pgfpoint{190.943985pt}{103.111580pt}}
\pgflineto{\pgfpoint{190.943985pt}{96.934731pt}}
\pgfusepath{stroke}
\pgfpathmoveto{\pgfpoint{190.943985pt}{72.227356pt}}
\pgflineto{\pgfpoint{190.943985pt}{66.050522pt}}
\pgfusepath{stroke}
\pgfpathmoveto{\pgfpoint{190.943985pt}{78.404205pt}}
\pgflineto{\pgfpoint{190.943985pt}{72.227356pt}}
\pgfusepath{stroke}
\pgfpathmoveto{\pgfpoint{190.943985pt}{84.581039pt}}
\pgflineto{\pgfpoint{190.943985pt}{78.404205pt}}
\pgfusepath{stroke}
\pgfpathmoveto{\pgfpoint{190.943985pt}{90.757896pt}}
\pgflineto{\pgfpoint{190.943985pt}{84.581039pt}}
\pgfusepath{stroke}
\pgfpathmoveto{\pgfpoint{190.943985pt}{96.934731pt}}
\pgflineto{\pgfpoint{190.943985pt}{90.757896pt}}
\pgfusepath{stroke}
\pgfpathmoveto{\pgfpoint{190.943985pt}{109.288422pt}}
\pgflineto{\pgfpoint{190.943985pt}{103.111580pt}}
\pgfusepath{stroke}
\pgfpathmoveto{\pgfpoint{190.943985pt}{115.465263pt}}
\pgflineto{\pgfpoint{190.943985pt}{109.288422pt}}
\pgfusepath{stroke}
\pgfpathmoveto{\pgfpoint{190.943985pt}{121.642097pt}}
\pgflineto{\pgfpoint{190.943985pt}{115.465263pt}}
\pgfusepath{stroke}
\pgfpathmoveto{\pgfpoint{190.943985pt}{127.818947pt}}
\pgflineto{\pgfpoint{190.943985pt}{121.642097pt}}
\pgfusepath{stroke}
\pgfpathmoveto{\pgfpoint{190.943985pt}{133.995789pt}}
\pgflineto{\pgfpoint{190.943985pt}{127.818947pt}}
\pgfusepath{stroke}
\pgfpathmoveto{\pgfpoint{190.943985pt}{140.172638pt}}
\pgflineto{\pgfpoint{190.943985pt}{133.995789pt}}
\pgfusepath{stroke}
\pgfpathmoveto{\pgfpoint{190.943985pt}{146.349472pt}}
\pgflineto{\pgfpoint{190.943985pt}{140.172638pt}}
\pgfusepath{stroke}
\pgfpathmoveto{\pgfpoint{190.943985pt}{47.519989pt}}
\pgflineto{\pgfpoint{190.925919pt}{47.519989pt}}
\pgfusepath{stroke}
\pgfpathmoveto{\pgfpoint{190.943985pt}{53.696838pt}}
\pgflineto{\pgfpoint{190.925934pt}{53.696838pt}}
\pgfusepath{stroke}
\pgfpathmoveto{\pgfpoint{190.943985pt}{59.873672pt}}
\pgflineto{\pgfpoint{190.925949pt}{59.873672pt}}
\pgfusepath{stroke}
\pgfpathmoveto{\pgfpoint{190.943985pt}{53.696838pt}}
\pgflineto{\pgfpoint{190.943985pt}{47.519989pt}}
\pgfusepath{stroke}
\pgfpathmoveto{\pgfpoint{190.943985pt}{47.519989pt}}
\pgflineto{\pgfpoint{190.962036pt}{47.519989pt}}
\pgfusepath{stroke}
\pgfpathmoveto{\pgfpoint{190.943985pt}{66.050522pt}}
\pgflineto{\pgfpoint{190.943985pt}{59.873672pt}}
\pgfusepath{stroke}
\pgfpathmoveto{\pgfpoint{190.943985pt}{59.873672pt}}
\pgflineto{\pgfpoint{190.943985pt}{53.696838pt}}
\pgfusepath{stroke}
\pgfpathmoveto{\pgfpoint{199.871979pt}{47.519989pt}}
\pgflineto{\pgfpoint{190.962036pt}{47.519989pt}}
\pgfusepath{stroke}
\pgfpathmoveto{\pgfpoint{199.871979pt}{53.696838pt}}
\pgflineto{\pgfpoint{190.943985pt}{53.696838pt}}
\pgfusepath{stroke}
\pgfpathmoveto{\pgfpoint{199.871979pt}{59.873672pt}}
\pgflineto{\pgfpoint{190.943985pt}{59.873672pt}}
\pgfusepath{stroke}
\pgfpathmoveto{\pgfpoint{199.871979pt}{66.050522pt}}
\pgflineto{\pgfpoint{190.943985pt}{66.050522pt}}
\pgfusepath{stroke}
\pgfpathmoveto{\pgfpoint{199.871979pt}{72.227356pt}}
\pgflineto{\pgfpoint{190.943985pt}{72.227356pt}}
\pgfusepath{stroke}
\pgfpathmoveto{\pgfpoint{199.871979pt}{78.404205pt}}
\pgflineto{\pgfpoint{190.943985pt}{78.404205pt}}
\pgfusepath{stroke}
\pgfpathmoveto{\pgfpoint{199.871979pt}{84.581039pt}}
\pgflineto{\pgfpoint{190.943985pt}{84.581039pt}}
\pgfusepath{stroke}
\pgfpathmoveto{\pgfpoint{199.871979pt}{90.757896pt}}
\pgflineto{\pgfpoint{190.943985pt}{90.757896pt}}
\pgfusepath{stroke}
\pgfpathmoveto{\pgfpoint{199.871979pt}{96.934731pt}}
\pgflineto{\pgfpoint{190.943985pt}{96.934731pt}}
\pgfusepath{stroke}
\pgfpathmoveto{\pgfpoint{199.871979pt}{103.111580pt}}
\pgflineto{\pgfpoint{190.943985pt}{103.111580pt}}
\pgfusepath{stroke}
\pgfpathmoveto{\pgfpoint{199.871979pt}{109.288422pt}}
\pgflineto{\pgfpoint{190.943985pt}{109.288422pt}}
\pgfusepath{stroke}
\pgfpathmoveto{\pgfpoint{199.871979pt}{115.465263pt}}
\pgflineto{\pgfpoint{190.943985pt}{115.465263pt}}
\pgfusepath{stroke}
\pgfpathmoveto{\pgfpoint{199.871979pt}{121.642097pt}}
\pgflineto{\pgfpoint{190.943985pt}{121.642097pt}}
\pgfusepath{stroke}
\pgfpathmoveto{\pgfpoint{199.871979pt}{127.818947pt}}
\pgflineto{\pgfpoint{190.943985pt}{127.818947pt}}
\pgfusepath{stroke}
\pgfpathmoveto{\pgfpoint{199.871979pt}{133.995789pt}}
\pgflineto{\pgfpoint{190.943985pt}{133.995789pt}}
\pgfusepath{stroke}
\pgfpathmoveto{\pgfpoint{199.871979pt}{140.172638pt}}
\pgflineto{\pgfpoint{190.943985pt}{140.172638pt}}
\pgfusepath{stroke}
\pgfpathmoveto{\pgfpoint{199.871979pt}{146.349472pt}}
\pgflineto{\pgfpoint{190.943985pt}{146.349472pt}}
\pgfusepath{stroke}
\pgfpathmoveto{\pgfpoint{199.871979pt}{152.526306pt}}
\pgflineto{\pgfpoint{190.962036pt}{152.526306pt}}
\pgfusepath{stroke}
\pgfpathmoveto{\pgfpoint{199.871979pt}{158.703156pt}}
\pgflineto{\pgfpoint{190.962051pt}{158.703156pt}}
\pgfusepath{stroke}
\pgfpathmoveto{\pgfpoint{199.871979pt}{164.880005pt}}
\pgflineto{\pgfpoint{190.962082pt}{164.880005pt}}
\pgfusepath{stroke}
\pgfpathmoveto{\pgfpoint{199.871979pt}{171.056854pt}}
\pgflineto{\pgfpoint{190.962082pt}{171.056854pt}}
\pgfusepath{stroke}
\pgfpathmoveto{\pgfpoint{199.871979pt}{177.233673pt}}
\pgflineto{\pgfpoint{190.962082pt}{177.233673pt}}
\pgfusepath{stroke}
\pgfpathmoveto{\pgfpoint{199.871979pt}{183.410522pt}}
\pgflineto{\pgfpoint{190.971100pt}{183.410522pt}}
\pgfusepath{stroke}
\pgfpathmoveto{\pgfpoint{199.871979pt}{189.587372pt}}
\pgflineto{\pgfpoint{190.971100pt}{189.587372pt}}
\pgfusepath{stroke}
\pgfpathmoveto{\pgfpoint{199.871979pt}{195.764206pt}}
\pgflineto{\pgfpoint{190.971085pt}{195.764206pt}}
\pgfusepath{stroke}
\pgfpathmoveto{\pgfpoint{199.871979pt}{201.941055pt}}
\pgflineto{\pgfpoint{190.971115pt}{201.941055pt}}
\pgfusepath{stroke}
\pgfpathmoveto{\pgfpoint{190.971115pt}{208.117905pt}}
\pgflineto{\pgfpoint{199.853943pt}{208.117905pt}}
\pgfusepath{stroke}
\pgfpathmoveto{\pgfpoint{190.980133pt}{214.294739pt}}
\pgflineto{\pgfpoint{199.844940pt}{214.294739pt}}
\pgfusepath{stroke}
\pgfpathmoveto{\pgfpoint{190.980133pt}{220.471588pt}}
\pgflineto{\pgfpoint{199.835938pt}{220.471588pt}}
\pgfusepath{stroke}
\pgfpathmoveto{\pgfpoint{190.980164pt}{226.648422pt}}
\pgflineto{\pgfpoint{199.835876pt}{226.648422pt}}
\pgfusepath{stroke}
\pgfpathmoveto{\pgfpoint{190.980164pt}{232.825272pt}}
\pgflineto{\pgfpoint{199.826889pt}{232.825272pt}}
\pgfusepath{stroke}
\pgfpathmoveto{\pgfpoint{190.980164pt}{239.002106pt}}
\pgflineto{\pgfpoint{199.817825pt}{239.002106pt}}
\pgfusepath{stroke}
\pgfpathmoveto{\pgfpoint{190.989166pt}{245.178955pt}}
\pgflineto{\pgfpoint{199.808868pt}{245.178955pt}}
\pgfusepath{stroke}
\pgfpathmoveto{\pgfpoint{190.989166pt}{251.355804pt}}
\pgflineto{\pgfpoint{199.799896pt}{251.355804pt}}
\pgfusepath{stroke}
\pgfpathmoveto{\pgfpoint{190.989212pt}{257.532623pt}}
\pgflineto{\pgfpoint{199.799759pt}{257.532623pt}}
\pgfusepath{stroke}
\pgfpathmoveto{\pgfpoint{190.989182pt}{263.709473pt}}
\pgflineto{\pgfpoint{199.790756pt}{263.709473pt}}
\pgfusepath{stroke}
\pgfpathmoveto{\pgfpoint{190.998169pt}{269.886322pt}}
\pgflineto{\pgfpoint{199.781860pt}{269.886322pt}}
\pgfusepath{stroke}
\pgfpathmoveto{\pgfpoint{190.998138pt}{276.063141pt}}
\pgflineto{\pgfpoint{199.772797pt}{276.063141pt}}
\pgfusepath{stroke}
\pgfpathmoveto{\pgfpoint{190.993683pt}{282.239990pt}}
\pgflineto{\pgfpoint{199.763748pt}{282.239990pt}}
\pgfusepath{stroke}
\pgfpathmoveto{\pgfpoint{190.993683pt}{288.416840pt}}
\pgflineto{\pgfpoint{199.754715pt}{288.416840pt}}
\pgfusepath{stroke}
\pgfpathmoveto{\pgfpoint{190.998199pt}{294.593689pt}}
\pgflineto{\pgfpoint{199.750214pt}{294.593689pt}}
\pgfusepath{stroke}
\pgfpathmoveto{\pgfpoint{190.998199pt}{300.770538pt}}
\pgflineto{\pgfpoint{199.741211pt}{300.770538pt}}
\pgfusepath{stroke}
\pgfpathmoveto{\pgfpoint{191.002716pt}{306.947388pt}}
\pgflineto{\pgfpoint{199.736694pt}{306.947388pt}}
\pgfusepath{stroke}
\pgfpathmoveto{\pgfpoint{191.002716pt}{313.124207pt}}
\pgflineto{\pgfpoint{199.727661pt}{313.124207pt}}
\pgfusepath{stroke}
\pgfpathmoveto{\pgfpoint{191.002747pt}{319.301056pt}}
\pgflineto{\pgfpoint{199.723068pt}{319.301056pt}}
\pgfusepath{stroke}
\pgfpathmoveto{\pgfpoint{191.007248pt}{325.477905pt}}
\pgflineto{\pgfpoint{199.714157pt}{325.477905pt}}
\pgfusepath{stroke}
\pgfpathmoveto{\pgfpoint{191.007217pt}{331.654724pt}}
\pgflineto{\pgfpoint{199.705093pt}{331.654724pt}}
\pgfusepath{stroke}
\pgfpathmoveto{\pgfpoint{191.011795pt}{337.831604pt}}
\pgflineto{\pgfpoint{199.700623pt}{337.831604pt}}
\pgfusepath{stroke}
\pgfpathmoveto{\pgfpoint{191.011765pt}{344.008423pt}}
\pgflineto{\pgfpoint{199.691605pt}{344.008423pt}}
\pgfusepath{stroke}
\pgfpathmoveto{\pgfpoint{191.016220pt}{350.185242pt}}
\pgflineto{\pgfpoint{199.687042pt}{350.185242pt}}
\pgfusepath{stroke}
\pgfpathmoveto{\pgfpoint{191.016312pt}{356.362122pt}}
\pgflineto{\pgfpoint{199.678070pt}{356.362122pt}}
\pgfusepath{stroke}
\pgfpathmoveto{\pgfpoint{191.016251pt}{362.538940pt}}
\pgflineto{\pgfpoint{199.669006pt}{362.538940pt}}
\pgfusepath{stroke}
\pgfpathmoveto{\pgfpoint{191.020828pt}{368.715820pt}}
\pgflineto{\pgfpoint{199.664551pt}{368.715820pt}}
\pgfusepath{stroke}
\pgfpathmoveto{\pgfpoint{191.020798pt}{374.892639pt}}
\pgflineto{\pgfpoint{199.655487pt}{374.892639pt}}
\pgfusepath{stroke}
\pgfpathmoveto{\pgfpoint{199.871979pt}{146.349472pt}}
\pgflineto{\pgfpoint{199.871979pt}{140.172638pt}}
\pgfusepath{stroke}
\pgfpathmoveto{\pgfpoint{199.871979pt}{140.172638pt}}
\pgflineto{\pgfpoint{199.871979pt}{133.995789pt}}
\pgfusepath{stroke}
\pgfpathmoveto{\pgfpoint{199.871979pt}{152.526306pt}}
\pgflineto{\pgfpoint{199.871979pt}{146.349472pt}}
\pgfusepath{stroke}
\pgfpathmoveto{\pgfpoint{199.871979pt}{158.703156pt}}
\pgflineto{\pgfpoint{199.871979pt}{152.526306pt}}
\pgfusepath{stroke}
\pgfpathmoveto{\pgfpoint{199.871979pt}{164.880005pt}}
\pgflineto{\pgfpoint{199.871979pt}{158.703156pt}}
\pgfusepath{stroke}
\pgfpathmoveto{\pgfpoint{199.871979pt}{171.056854pt}}
\pgflineto{\pgfpoint{199.871979pt}{164.880005pt}}
\pgfusepath{stroke}
\pgfpathmoveto{\pgfpoint{199.871979pt}{177.233673pt}}
\pgflineto{\pgfpoint{199.871979pt}{171.056854pt}}
\pgfusepath{stroke}
\pgfpathmoveto{\pgfpoint{199.871979pt}{183.410522pt}}
\pgflineto{\pgfpoint{199.871979pt}{177.233673pt}}
\pgfusepath{stroke}
\pgfpathmoveto{\pgfpoint{199.871979pt}{140.172638pt}}
\pgflineto{\pgfpoint{199.890030pt}{140.172638pt}}
\pgfusepath{stroke}
\pgfpathmoveto{\pgfpoint{199.871979pt}{146.349472pt}}
\pgflineto{\pgfpoint{199.890030pt}{146.349472pt}}
\pgfusepath{stroke}
\pgfpathmoveto{\pgfpoint{199.871979pt}{152.526306pt}}
\pgflineto{\pgfpoint{199.889984pt}{152.526306pt}}
\pgfusepath{stroke}
\pgfpathmoveto{\pgfpoint{199.871979pt}{158.703156pt}}
\pgflineto{\pgfpoint{199.889984pt}{158.703156pt}}
\pgfusepath{stroke}
\pgfpathmoveto{\pgfpoint{199.871979pt}{164.880005pt}}
\pgflineto{\pgfpoint{199.899002pt}{164.880005pt}}
\pgfusepath{stroke}
\pgfpathmoveto{\pgfpoint{199.871979pt}{171.056854pt}}
\pgflineto{\pgfpoint{199.899033pt}{171.056854pt}}
\pgfusepath{stroke}
\pgfpathmoveto{\pgfpoint{199.871979pt}{177.233673pt}}
\pgflineto{\pgfpoint{199.899033pt}{177.233673pt}}
\pgfusepath{stroke}
\pgfpathmoveto{\pgfpoint{199.871979pt}{195.764206pt}}
\pgflineto{\pgfpoint{199.871979pt}{189.587372pt}}
\pgfusepath{stroke}
\pgfpathmoveto{\pgfpoint{199.871979pt}{189.587372pt}}
\pgflineto{\pgfpoint{199.871979pt}{183.410522pt}}
\pgfusepath{stroke}
\pgfpathmoveto{\pgfpoint{199.871979pt}{201.941055pt}}
\pgflineto{\pgfpoint{199.871979pt}{195.764206pt}}
\pgfusepath{stroke}
\pgfpathmoveto{\pgfpoint{199.835938pt}{220.471588pt}}
\pgflineto{\pgfpoint{199.854004pt}{220.471588pt}}
\pgfusepath{stroke}
\pgfpathmoveto{\pgfpoint{199.871979pt}{226.648422pt}}
\pgflineto{\pgfpoint{199.835876pt}{226.648422pt}}
\pgfusepath{stroke}
\pgfpathmoveto{\pgfpoint{199.871979pt}{232.825272pt}}
\pgflineto{\pgfpoint{199.826889pt}{232.825272pt}}
\pgfusepath{stroke}
\pgfpathmoveto{\pgfpoint{199.871979pt}{239.002106pt}}
\pgflineto{\pgfpoint{199.817825pt}{239.002106pt}}
\pgfusepath{stroke}
\pgfpathmoveto{\pgfpoint{199.871979pt}{245.178955pt}}
\pgflineto{\pgfpoint{199.808868pt}{245.178955pt}}
\pgfusepath{stroke}
\pgfpathmoveto{\pgfpoint{199.871979pt}{251.355804pt}}
\pgflineto{\pgfpoint{199.799896pt}{251.355804pt}}
\pgfusepath{stroke}
\pgfpathmoveto{\pgfpoint{199.871979pt}{257.532623pt}}
\pgflineto{\pgfpoint{199.799759pt}{257.532623pt}}
\pgfusepath{stroke}
\pgfpathmoveto{\pgfpoint{199.871979pt}{263.709473pt}}
\pgflineto{\pgfpoint{199.790756pt}{263.709473pt}}
\pgfusepath{stroke}
\pgfpathmoveto{\pgfpoint{199.871979pt}{269.886322pt}}
\pgflineto{\pgfpoint{199.781860pt}{269.886322pt}}
\pgfusepath{stroke}
\pgfpathmoveto{\pgfpoint{199.871979pt}{276.063141pt}}
\pgflineto{\pgfpoint{199.772797pt}{276.063141pt}}
\pgfusepath{stroke}
\pgfpathmoveto{\pgfpoint{199.871979pt}{282.239990pt}}
\pgflineto{\pgfpoint{199.763748pt}{282.239990pt}}
\pgfusepath{stroke}
\pgfpathmoveto{\pgfpoint{199.871979pt}{288.416840pt}}
\pgflineto{\pgfpoint{199.754715pt}{288.416840pt}}
\pgfusepath{stroke}
\pgfpathmoveto{\pgfpoint{199.871979pt}{294.593689pt}}
\pgflineto{\pgfpoint{199.750214pt}{294.593689pt}}
\pgfusepath{stroke}
\pgfpathmoveto{\pgfpoint{199.871979pt}{300.770538pt}}
\pgflineto{\pgfpoint{199.741211pt}{300.770538pt}}
\pgfusepath{stroke}
\pgfpathmoveto{\pgfpoint{199.871979pt}{306.947388pt}}
\pgflineto{\pgfpoint{199.736694pt}{306.947388pt}}
\pgfusepath{stroke}
\pgfpathmoveto{\pgfpoint{199.871979pt}{313.124207pt}}
\pgflineto{\pgfpoint{199.727661pt}{313.124207pt}}
\pgfusepath{stroke}
\pgfpathmoveto{\pgfpoint{199.871979pt}{319.301056pt}}
\pgflineto{\pgfpoint{199.723068pt}{319.301056pt}}
\pgfusepath{stroke}
\pgfpathmoveto{\pgfpoint{199.871979pt}{325.477905pt}}
\pgflineto{\pgfpoint{199.714157pt}{325.477905pt}}
\pgfusepath{stroke}
\pgfpathmoveto{\pgfpoint{199.871979pt}{331.654724pt}}
\pgflineto{\pgfpoint{199.705093pt}{331.654724pt}}
\pgfusepath{stroke}
\pgfpathmoveto{\pgfpoint{199.871979pt}{337.831604pt}}
\pgflineto{\pgfpoint{199.700623pt}{337.831604pt}}
\pgfusepath{stroke}
\pgfpathmoveto{\pgfpoint{199.871979pt}{344.008423pt}}
\pgflineto{\pgfpoint{199.691605pt}{344.008423pt}}
\pgfusepath{stroke}
\pgfpathmoveto{\pgfpoint{199.871979pt}{350.185242pt}}
\pgflineto{\pgfpoint{199.687042pt}{350.185242pt}}
\pgfusepath{stroke}
\pgfpathmoveto{\pgfpoint{199.871979pt}{356.362122pt}}
\pgflineto{\pgfpoint{199.678070pt}{356.362122pt}}
\pgfusepath{stroke}
\pgfpathmoveto{\pgfpoint{199.871979pt}{362.538940pt}}
\pgflineto{\pgfpoint{199.669006pt}{362.538940pt}}
\pgfusepath{stroke}
\pgfpathmoveto{\pgfpoint{199.871979pt}{368.715820pt}}
\pgflineto{\pgfpoint{199.664551pt}{368.715820pt}}
\pgfusepath{stroke}
\pgfpathmoveto{\pgfpoint{199.871979pt}{374.892639pt}}
\pgflineto{\pgfpoint{199.655487pt}{374.892639pt}}
\pgfusepath{stroke}
\pgfpathmoveto{\pgfpoint{199.871979pt}{325.477905pt}}
\pgflineto{\pgfpoint{199.871979pt}{319.301056pt}}
\pgfusepath{stroke}
\pgfpathmoveto{\pgfpoint{199.871979pt}{319.301056pt}}
\pgflineto{\pgfpoint{199.871979pt}{313.124207pt}}
\pgfusepath{stroke}
\pgfpathmoveto{\pgfpoint{199.871979pt}{331.654724pt}}
\pgflineto{\pgfpoint{199.871979pt}{325.477905pt}}
\pgfusepath{stroke}
\pgfpathmoveto{\pgfpoint{199.871979pt}{337.831604pt}}
\pgflineto{\pgfpoint{199.871979pt}{331.654724pt}}
\pgfusepath{stroke}
\pgfpathmoveto{\pgfpoint{199.871979pt}{344.008423pt}}
\pgflineto{\pgfpoint{199.871979pt}{337.831604pt}}
\pgfusepath{stroke}
\pgfpathmoveto{\pgfpoint{199.871979pt}{319.301056pt}}
\pgflineto{\pgfpoint{199.885422pt}{319.301056pt}}
\pgfusepath{stroke}
\pgfpathmoveto{\pgfpoint{199.871979pt}{325.477905pt}}
\pgflineto{\pgfpoint{199.885422pt}{325.477905pt}}
\pgfusepath{stroke}
\pgfpathmoveto{\pgfpoint{199.871979pt}{331.654724pt}}
\pgflineto{\pgfpoint{199.885391pt}{331.654724pt}}
\pgfusepath{stroke}
\pgfpathmoveto{\pgfpoint{199.871979pt}{337.831604pt}}
\pgflineto{\pgfpoint{199.889938pt}{337.831604pt}}
\pgfusepath{stroke}
\pgfpathmoveto{\pgfpoint{199.871979pt}{368.715820pt}}
\pgflineto{\pgfpoint{199.871979pt}{362.538940pt}}
\pgfusepath{stroke}
\pgfpathmoveto{\pgfpoint{199.871979pt}{350.185242pt}}
\pgflineto{\pgfpoint{199.871979pt}{344.008423pt}}
\pgfusepath{stroke}
\pgfpathmoveto{\pgfpoint{199.871979pt}{356.362122pt}}
\pgflineto{\pgfpoint{199.871979pt}{350.185242pt}}
\pgfusepath{stroke}
\pgfpathmoveto{\pgfpoint{199.871979pt}{362.538940pt}}
\pgflineto{\pgfpoint{199.871979pt}{356.362122pt}}
\pgfusepath{stroke}
\pgfpathmoveto{\pgfpoint{199.871979pt}{374.892639pt}}
\pgflineto{\pgfpoint{199.871979pt}{368.715820pt}}
\pgfusepath{stroke}
\pgfpathmoveto{\pgfpoint{199.871979pt}{344.008423pt}}
\pgflineto{\pgfpoint{199.889893pt}{344.008423pt}}
\pgfusepath{stroke}
\pgfpathmoveto{\pgfpoint{199.871979pt}{350.185242pt}}
\pgflineto{\pgfpoint{199.894379pt}{350.185242pt}}
\pgfusepath{stroke}
\pgfpathmoveto{\pgfpoint{199.871979pt}{356.362122pt}}
\pgflineto{\pgfpoint{199.894409pt}{356.362122pt}}
\pgfusepath{stroke}
\pgfpathmoveto{\pgfpoint{199.871979pt}{362.538940pt}}
\pgflineto{\pgfpoint{199.894394pt}{362.538940pt}}
\pgfusepath{stroke}
\pgfpathmoveto{\pgfpoint{199.871979pt}{368.715820pt}}
\pgflineto{\pgfpoint{199.898926pt}{368.715820pt}}
\pgfusepath{stroke}
\pgfpathmoveto{\pgfpoint{199.871979pt}{263.709473pt}}
\pgflineto{\pgfpoint{199.871979pt}{257.532623pt}}
\pgfusepath{stroke}
\pgfpathmoveto{\pgfpoint{199.871979pt}{232.825272pt}}
\pgflineto{\pgfpoint{199.871979pt}{226.648422pt}}
\pgfusepath{stroke}
\pgfpathmoveto{\pgfpoint{199.871979pt}{239.002106pt}}
\pgflineto{\pgfpoint{199.871979pt}{232.825272pt}}
\pgfusepath{stroke}
\pgfpathmoveto{\pgfpoint{199.871979pt}{245.178955pt}}
\pgflineto{\pgfpoint{199.871979pt}{239.002106pt}}
\pgfusepath{stroke}
\pgfpathmoveto{\pgfpoint{199.871979pt}{251.355804pt}}
\pgflineto{\pgfpoint{199.871979pt}{245.178955pt}}
\pgfusepath{stroke}
\pgfpathmoveto{\pgfpoint{199.871979pt}{257.532623pt}}
\pgflineto{\pgfpoint{199.871979pt}{251.355804pt}}
\pgfusepath{stroke}
\pgfpathmoveto{\pgfpoint{199.871979pt}{269.886322pt}}
\pgflineto{\pgfpoint{199.871979pt}{263.709473pt}}
\pgfusepath{stroke}
\pgfpathmoveto{\pgfpoint{199.871979pt}{276.063141pt}}
\pgflineto{\pgfpoint{199.871979pt}{269.886322pt}}
\pgfusepath{stroke}
\pgfpathmoveto{\pgfpoint{199.871979pt}{282.239990pt}}
\pgflineto{\pgfpoint{199.871979pt}{276.063141pt}}
\pgfusepath{stroke}
\pgfpathmoveto{\pgfpoint{199.871979pt}{288.416840pt}}
\pgflineto{\pgfpoint{199.871979pt}{282.239990pt}}
\pgfusepath{stroke}
\pgfpathmoveto{\pgfpoint{199.871979pt}{294.593689pt}}
\pgflineto{\pgfpoint{199.871979pt}{288.416840pt}}
\pgfusepath{stroke}
\pgfpathmoveto{\pgfpoint{199.871979pt}{300.770538pt}}
\pgflineto{\pgfpoint{199.871979pt}{294.593689pt}}
\pgfusepath{stroke}
\pgfpathmoveto{\pgfpoint{199.871979pt}{306.947388pt}}
\pgflineto{\pgfpoint{199.871979pt}{300.770538pt}}
\pgfusepath{stroke}
\pgfpathmoveto{\pgfpoint{199.871979pt}{313.124207pt}}
\pgflineto{\pgfpoint{199.871979pt}{306.947388pt}}
\pgfusepath{stroke}
\pgfpathmoveto{\pgfpoint{199.871979pt}{183.410522pt}}
\pgflineto{\pgfpoint{199.899033pt}{183.410522pt}}
\pgfusepath{stroke}
\pgfpathmoveto{\pgfpoint{199.871979pt}{189.587372pt}}
\pgflineto{\pgfpoint{199.899033pt}{189.587372pt}}
\pgfusepath{stroke}
\pgfpathmoveto{\pgfpoint{199.871979pt}{195.764206pt}}
\pgflineto{\pgfpoint{199.908005pt}{195.764206pt}}
\pgfusepath{stroke}
\pgfpathmoveto{\pgfpoint{199.871979pt}{201.941055pt}}
\pgflineto{\pgfpoint{199.908035pt}{201.941055pt}}
\pgfusepath{stroke}
\pgfpathmoveto{\pgfpoint{199.871979pt}{208.117905pt}}
\pgflineto{\pgfpoint{199.871979pt}{201.941055pt}}
\pgfusepath{stroke}
\pgfpathmoveto{\pgfpoint{199.871979pt}{208.117905pt}}
\pgflineto{\pgfpoint{199.853943pt}{208.117905pt}}
\pgfusepath{stroke}
\pgfpathmoveto{\pgfpoint{199.871979pt}{208.117905pt}}
\pgflineto{\pgfpoint{199.908081pt}{208.117905pt}}
\pgfusepath{stroke}
\pgfpathmoveto{\pgfpoint{199.871979pt}{214.294739pt}}
\pgflineto{\pgfpoint{199.844940pt}{214.294739pt}}
\pgfusepath{stroke}
\pgfpathmoveto{\pgfpoint{199.871979pt}{220.471588pt}}
\pgflineto{\pgfpoint{199.854004pt}{220.471588pt}}
\pgfusepath{stroke}
\pgfpathmoveto{\pgfpoint{199.871979pt}{214.294739pt}}
\pgflineto{\pgfpoint{199.871979pt}{208.117905pt}}
\pgfusepath{stroke}
\pgfpathmoveto{\pgfpoint{199.871979pt}{226.648422pt}}
\pgflineto{\pgfpoint{199.871979pt}{220.471588pt}}
\pgfusepath{stroke}
\pgfpathmoveto{\pgfpoint{199.871979pt}{220.471588pt}}
\pgflineto{\pgfpoint{199.871979pt}{214.294739pt}}
\pgfusepath{stroke}
\pgfpathmoveto{\pgfpoint{199.871979pt}{214.294739pt}}
\pgflineto{\pgfpoint{199.908051pt}{214.294739pt}}
\pgfusepath{stroke}
\pgfpathmoveto{\pgfpoint{199.871979pt}{220.471588pt}}
\pgflineto{\pgfpoint{199.908051pt}{220.471588pt}}
\pgfusepath{stroke}
\pgfpathmoveto{\pgfpoint{199.871979pt}{226.648422pt}}
\pgflineto{\pgfpoint{199.917023pt}{226.648422pt}}
\pgfusepath{stroke}
\pgfpathmoveto{\pgfpoint{199.871979pt}{232.825272pt}}
\pgflineto{\pgfpoint{199.917023pt}{232.825272pt}}
\pgfusepath{stroke}
\pgfpathmoveto{\pgfpoint{199.871979pt}{239.002106pt}}
\pgflineto{\pgfpoint{199.917068pt}{239.002106pt}}
\pgfusepath{stroke}
\pgfpathmoveto{\pgfpoint{199.871979pt}{245.178955pt}}
\pgflineto{\pgfpoint{199.917068pt}{245.178955pt}}
\pgfusepath{stroke}
\pgfpathmoveto{\pgfpoint{199.871979pt}{251.355804pt}}
\pgflineto{\pgfpoint{199.917068pt}{251.355804pt}}
\pgfusepath{stroke}
\pgfpathmoveto{\pgfpoint{199.871979pt}{257.532623pt}}
\pgflineto{\pgfpoint{199.926025pt}{257.532623pt}}
\pgfusepath{stroke}
\pgfpathmoveto{\pgfpoint{199.871979pt}{263.709473pt}}
\pgflineto{\pgfpoint{199.925995pt}{263.709473pt}}
\pgfusepath{stroke}
\pgfpathmoveto{\pgfpoint{199.871979pt}{269.886322pt}}
\pgflineto{\pgfpoint{199.925995pt}{269.886322pt}}
\pgfusepath{stroke}
\pgfpathmoveto{\pgfpoint{199.871979pt}{276.063141pt}}
\pgflineto{\pgfpoint{199.926025pt}{276.063141pt}}
\pgfusepath{stroke}
\pgfpathmoveto{\pgfpoint{199.871979pt}{282.239990pt}}
\pgflineto{\pgfpoint{199.926025pt}{282.239990pt}}
\pgfusepath{stroke}
\pgfpathmoveto{\pgfpoint{199.871979pt}{288.416840pt}}
\pgflineto{\pgfpoint{199.926056pt}{288.416840pt}}
\pgfusepath{stroke}
\pgfpathmoveto{\pgfpoint{199.871979pt}{294.593689pt}}
\pgflineto{\pgfpoint{199.930542pt}{294.593689pt}}
\pgfusepath{stroke}
\pgfpathmoveto{\pgfpoint{199.871979pt}{300.770538pt}}
\pgflineto{\pgfpoint{199.930573pt}{300.770538pt}}
\pgfusepath{stroke}
\pgfpathmoveto{\pgfpoint{199.871979pt}{306.947388pt}}
\pgflineto{\pgfpoint{199.935059pt}{306.947388pt}}
\pgfusepath{stroke}
\pgfpathmoveto{\pgfpoint{199.871979pt}{313.124207pt}}
\pgflineto{\pgfpoint{199.935028pt}{313.124207pt}}
\pgfusepath{stroke}
\pgfpathmoveto{\pgfpoint{199.939560pt}{319.301056pt}}
\pgflineto{\pgfpoint{199.885422pt}{319.301056pt}}
\pgfusepath{stroke}
\pgfpathmoveto{\pgfpoint{199.939560pt}{325.477905pt}}
\pgflineto{\pgfpoint{199.885422pt}{325.477905pt}}
\pgfusepath{stroke}
\pgfpathmoveto{\pgfpoint{199.939529pt}{331.654724pt}}
\pgflineto{\pgfpoint{199.885391pt}{331.654724pt}}
\pgfusepath{stroke}
\pgfpathmoveto{\pgfpoint{199.944077pt}{337.831604pt}}
\pgflineto{\pgfpoint{199.889938pt}{337.831604pt}}
\pgfusepath{stroke}
\pgfpathmoveto{\pgfpoint{199.944016pt}{344.008423pt}}
\pgflineto{\pgfpoint{199.889893pt}{344.008423pt}}
\pgfusepath{stroke}
\pgfpathmoveto{\pgfpoint{199.948532pt}{350.185242pt}}
\pgflineto{\pgfpoint{199.894379pt}{350.185242pt}}
\pgfusepath{stroke}
\pgfpathmoveto{\pgfpoint{199.948563pt}{356.362122pt}}
\pgflineto{\pgfpoint{199.894409pt}{356.362122pt}}
\pgfusepath{stroke}
\pgfpathmoveto{\pgfpoint{199.948547pt}{362.538940pt}}
\pgflineto{\pgfpoint{199.894394pt}{362.538940pt}}
\pgfusepath{stroke}
\pgfpathmoveto{\pgfpoint{199.953064pt}{368.715820pt}}
\pgflineto{\pgfpoint{199.898926pt}{368.715820pt}}
\pgfusepath{stroke}
\pgfpathmoveto{\pgfpoint{199.871979pt}{84.581039pt}}
\pgflineto{\pgfpoint{199.871979pt}{78.404205pt}}
\pgfusepath{stroke}
\pgfpathmoveto{\pgfpoint{199.871979pt}{53.696838pt}}
\pgflineto{\pgfpoint{199.871979pt}{47.519989pt}}
\pgfusepath{stroke}
\pgfpathmoveto{\pgfpoint{199.871979pt}{59.873672pt}}
\pgflineto{\pgfpoint{199.871979pt}{53.696838pt}}
\pgfusepath{stroke}
\pgfpathmoveto{\pgfpoint{199.871979pt}{66.050522pt}}
\pgflineto{\pgfpoint{199.871979pt}{59.873672pt}}
\pgfusepath{stroke}
\pgfpathmoveto{\pgfpoint{199.871979pt}{72.227356pt}}
\pgflineto{\pgfpoint{199.871979pt}{66.050522pt}}
\pgfusepath{stroke}
\pgfpathmoveto{\pgfpoint{199.871979pt}{78.404205pt}}
\pgflineto{\pgfpoint{199.871979pt}{72.227356pt}}
\pgfusepath{stroke}
\pgfpathmoveto{\pgfpoint{199.871979pt}{90.757896pt}}
\pgflineto{\pgfpoint{199.871979pt}{84.581039pt}}
\pgfusepath{stroke}
\pgfpathmoveto{\pgfpoint{199.871979pt}{96.934731pt}}
\pgflineto{\pgfpoint{199.871979pt}{90.757896pt}}
\pgfusepath{stroke}
\pgfpathmoveto{\pgfpoint{199.871979pt}{103.111580pt}}
\pgflineto{\pgfpoint{199.871979pt}{96.934731pt}}
\pgfusepath{stroke}
\pgfpathmoveto{\pgfpoint{199.871979pt}{109.288422pt}}
\pgflineto{\pgfpoint{199.871979pt}{103.111580pt}}
\pgfusepath{stroke}
\pgfpathmoveto{\pgfpoint{199.871979pt}{115.465263pt}}
\pgflineto{\pgfpoint{199.871979pt}{109.288422pt}}
\pgfusepath{stroke}
\pgfpathmoveto{\pgfpoint{199.871979pt}{121.642097pt}}
\pgflineto{\pgfpoint{199.871979pt}{115.465263pt}}
\pgfusepath{stroke}
\pgfpathmoveto{\pgfpoint{199.871979pt}{127.818947pt}}
\pgflineto{\pgfpoint{199.871979pt}{121.642097pt}}
\pgfusepath{stroke}
\pgfpathmoveto{\pgfpoint{199.871979pt}{133.995789pt}}
\pgflineto{\pgfpoint{199.871979pt}{127.818947pt}}
\pgfusepath{stroke}
\pgfpathmoveto{\pgfpoint{199.871979pt}{47.519989pt}}
\pgflineto{\pgfpoint{208.781769pt}{47.519989pt}}
\pgfusepath{stroke}
\pgfpathmoveto{\pgfpoint{199.871979pt}{53.696838pt}}
\pgflineto{\pgfpoint{208.781799pt}{53.696838pt}}
\pgfusepath{stroke}
\pgfpathmoveto{\pgfpoint{208.799988pt}{59.873672pt}}
\pgflineto{\pgfpoint{199.871979pt}{59.873672pt}}
\pgfusepath{stroke}
\pgfpathmoveto{\pgfpoint{208.799988pt}{66.050522pt}}
\pgflineto{\pgfpoint{199.871979pt}{66.050522pt}}
\pgfusepath{stroke}
\pgfpathmoveto{\pgfpoint{208.799988pt}{72.227356pt}}
\pgflineto{\pgfpoint{199.871979pt}{72.227356pt}}
\pgfusepath{stroke}
\pgfpathmoveto{\pgfpoint{208.799988pt}{78.404205pt}}
\pgflineto{\pgfpoint{199.871979pt}{78.404205pt}}
\pgfusepath{stroke}
\pgfpathmoveto{\pgfpoint{208.799988pt}{84.581039pt}}
\pgflineto{\pgfpoint{199.871979pt}{84.581039pt}}
\pgfusepath{stroke}
\pgfpathmoveto{\pgfpoint{208.799988pt}{90.757896pt}}
\pgflineto{\pgfpoint{199.871979pt}{90.757896pt}}
\pgfusepath{stroke}
\pgfpathmoveto{\pgfpoint{208.799988pt}{96.934731pt}}
\pgflineto{\pgfpoint{199.871979pt}{96.934731pt}}
\pgfusepath{stroke}
\pgfpathmoveto{\pgfpoint{208.799988pt}{103.111580pt}}
\pgflineto{\pgfpoint{199.871979pt}{103.111580pt}}
\pgfusepath{stroke}
\pgfpathmoveto{\pgfpoint{208.799988pt}{109.288422pt}}
\pgflineto{\pgfpoint{199.871979pt}{109.288422pt}}
\pgfusepath{stroke}
\pgfpathmoveto{\pgfpoint{208.799988pt}{115.465263pt}}
\pgflineto{\pgfpoint{199.871979pt}{115.465263pt}}
\pgfusepath{stroke}
\pgfpathmoveto{\pgfpoint{208.799988pt}{121.642097pt}}
\pgflineto{\pgfpoint{199.871979pt}{121.642097pt}}
\pgfusepath{stroke}
\pgfpathmoveto{\pgfpoint{208.799988pt}{127.818947pt}}
\pgflineto{\pgfpoint{199.871979pt}{127.818947pt}}
\pgfusepath{stroke}
\pgfpathmoveto{\pgfpoint{208.799988pt}{133.995789pt}}
\pgflineto{\pgfpoint{199.871979pt}{133.995789pt}}
\pgfusepath{stroke}
\pgfpathmoveto{\pgfpoint{208.799988pt}{140.172638pt}}
\pgflineto{\pgfpoint{199.890030pt}{140.172638pt}}
\pgfusepath{stroke}
\pgfpathmoveto{\pgfpoint{208.799988pt}{146.349472pt}}
\pgflineto{\pgfpoint{199.890030pt}{146.349472pt}}
\pgfusepath{stroke}
\pgfpathmoveto{\pgfpoint{208.799988pt}{152.526306pt}}
\pgflineto{\pgfpoint{199.889984pt}{152.526306pt}}
\pgfusepath{stroke}
\pgfpathmoveto{\pgfpoint{208.799988pt}{158.703156pt}}
\pgflineto{\pgfpoint{199.889984pt}{158.703156pt}}
\pgfusepath{stroke}
\pgfpathmoveto{\pgfpoint{208.799988pt}{164.880005pt}}
\pgflineto{\pgfpoint{199.899002pt}{164.880005pt}}
\pgfusepath{stroke}
\pgfpathmoveto{\pgfpoint{208.799988pt}{171.056854pt}}
\pgflineto{\pgfpoint{199.899033pt}{171.056854pt}}
\pgfusepath{stroke}
\pgfpathmoveto{\pgfpoint{208.799988pt}{177.233673pt}}
\pgflineto{\pgfpoint{199.899033pt}{177.233673pt}}
\pgfusepath{stroke}
\pgfpathmoveto{\pgfpoint{208.799988pt}{183.410522pt}}
\pgflineto{\pgfpoint{199.899033pt}{183.410522pt}}
\pgfusepath{stroke}
\pgfpathmoveto{\pgfpoint{208.799988pt}{189.587372pt}}
\pgflineto{\pgfpoint{199.899033pt}{189.587372pt}}
\pgfusepath{stroke}
\pgfpathmoveto{\pgfpoint{208.799988pt}{195.764206pt}}
\pgflineto{\pgfpoint{199.908005pt}{195.764206pt}}
\pgfusepath{stroke}
\pgfpathmoveto{\pgfpoint{208.799988pt}{201.941055pt}}
\pgflineto{\pgfpoint{199.908035pt}{201.941055pt}}
\pgfusepath{stroke}
\pgfpathmoveto{\pgfpoint{208.799988pt}{208.117905pt}}
\pgflineto{\pgfpoint{199.908081pt}{208.117905pt}}
\pgfusepath{stroke}
\pgfpathmoveto{\pgfpoint{208.799988pt}{214.294739pt}}
\pgflineto{\pgfpoint{199.908051pt}{214.294739pt}}
\pgfusepath{stroke}
\pgfpathmoveto{\pgfpoint{199.908051pt}{220.471588pt}}
\pgflineto{\pgfpoint{208.781891pt}{220.471588pt}}
\pgfusepath{stroke}
\pgfpathmoveto{\pgfpoint{199.917023pt}{226.648422pt}}
\pgflineto{\pgfpoint{208.772873pt}{226.648422pt}}
\pgfusepath{stroke}
\pgfpathmoveto{\pgfpoint{199.917023pt}{232.825272pt}}
\pgflineto{\pgfpoint{208.763824pt}{232.825272pt}}
\pgfusepath{stroke}
\pgfpathmoveto{\pgfpoint{199.917068pt}{239.002106pt}}
\pgflineto{\pgfpoint{208.763779pt}{239.002106pt}}
\pgfusepath{stroke}
\pgfpathmoveto{\pgfpoint{199.917068pt}{245.178955pt}}
\pgflineto{\pgfpoint{208.754745pt}{245.178955pt}}
\pgfusepath{stroke}
\pgfpathmoveto{\pgfpoint{199.917068pt}{251.355804pt}}
\pgflineto{\pgfpoint{208.745697pt}{251.355804pt}}
\pgfusepath{stroke}
\pgfpathmoveto{\pgfpoint{199.926025pt}{257.532623pt}}
\pgflineto{\pgfpoint{208.736710pt}{257.532623pt}}
\pgfusepath{stroke}
\pgfpathmoveto{\pgfpoint{199.925995pt}{263.709473pt}}
\pgflineto{\pgfpoint{208.727692pt}{263.709473pt}}
\pgfusepath{stroke}
\pgfpathmoveto{\pgfpoint{199.925995pt}{269.886322pt}}
\pgflineto{\pgfpoint{208.718674pt}{269.886322pt}}
\pgfusepath{stroke}
\pgfpathmoveto{\pgfpoint{199.926025pt}{276.063141pt}}
\pgflineto{\pgfpoint{208.714081pt}{276.063141pt}}
\pgfusepath{stroke}
\pgfpathmoveto{\pgfpoint{199.926025pt}{282.239990pt}}
\pgflineto{\pgfpoint{208.705032pt}{282.239990pt}}
\pgfusepath{stroke}
\pgfpathmoveto{\pgfpoint{199.926056pt}{288.416840pt}}
\pgflineto{\pgfpoint{208.700470pt}{288.416840pt}}
\pgfusepath{stroke}
\pgfpathmoveto{\pgfpoint{199.930542pt}{294.593689pt}}
\pgflineto{\pgfpoint{208.691452pt}{294.593689pt}}
\pgfusepath{stroke}
\pgfpathmoveto{\pgfpoint{199.930573pt}{300.770538pt}}
\pgflineto{\pgfpoint{208.682480pt}{300.770538pt}}
\pgfusepath{stroke}
\pgfpathmoveto{\pgfpoint{199.935059pt}{306.947388pt}}
\pgflineto{\pgfpoint{208.677948pt}{306.947388pt}}
\pgfusepath{stroke}
\pgfpathmoveto{\pgfpoint{199.935028pt}{313.124207pt}}
\pgflineto{\pgfpoint{208.668884pt}{313.124207pt}}
\pgfusepath{stroke}
\pgfpathmoveto{\pgfpoint{199.939560pt}{319.301056pt}}
\pgflineto{\pgfpoint{208.664352pt}{319.301056pt}}
\pgfusepath{stroke}
\pgfpathmoveto{\pgfpoint{199.939560pt}{325.477905pt}}
\pgflineto{\pgfpoint{208.655319pt}{325.477905pt}}
\pgfusepath{stroke}
\pgfpathmoveto{\pgfpoint{199.939529pt}{331.654724pt}}
\pgflineto{\pgfpoint{208.646240pt}{331.654724pt}}
\pgfusepath{stroke}
\pgfpathmoveto{\pgfpoint{199.944077pt}{337.831604pt}}
\pgflineto{\pgfpoint{208.641785pt}{337.831604pt}}
\pgfusepath{stroke}
\pgfpathmoveto{\pgfpoint{199.944016pt}{344.008423pt}}
\pgflineto{\pgfpoint{208.632721pt}{344.008423pt}}
\pgfusepath{stroke}
\pgfpathmoveto{\pgfpoint{199.948532pt}{350.185242pt}}
\pgflineto{\pgfpoint{208.628159pt}{350.185242pt}}
\pgfusepath{stroke}
\pgfpathmoveto{\pgfpoint{199.948563pt}{356.362122pt}}
\pgflineto{\pgfpoint{208.619171pt}{356.362122pt}}
\pgfusepath{stroke}
\pgfpathmoveto{\pgfpoint{199.948547pt}{362.538940pt}}
\pgflineto{\pgfpoint{208.614502pt}{362.538940pt}}
\pgfusepath{stroke}
\pgfpathmoveto{\pgfpoint{199.953064pt}{368.715820pt}}
\pgflineto{\pgfpoint{208.605621pt}{368.715820pt}}
\pgfusepath{stroke}
\pgfpathmoveto{\pgfpoint{208.799988pt}{158.703156pt}}
\pgflineto{\pgfpoint{208.799988pt}{152.526306pt}}
\pgfusepath{stroke}
\pgfpathmoveto{\pgfpoint{208.799988pt}{152.526306pt}}
\pgflineto{\pgfpoint{208.799988pt}{146.349472pt}}
\pgfusepath{stroke}
\pgfpathmoveto{\pgfpoint{208.799988pt}{164.880005pt}}
\pgflineto{\pgfpoint{208.799988pt}{158.703156pt}}
\pgfusepath{stroke}
\pgfpathmoveto{\pgfpoint{208.799988pt}{171.056854pt}}
\pgflineto{\pgfpoint{208.799988pt}{164.880005pt}}
\pgfusepath{stroke}
\pgfpathmoveto{\pgfpoint{208.799988pt}{177.233673pt}}
\pgflineto{\pgfpoint{208.799988pt}{171.056854pt}}
\pgfusepath{stroke}
\pgfpathmoveto{\pgfpoint{208.799988pt}{183.410522pt}}
\pgflineto{\pgfpoint{208.799988pt}{177.233673pt}}
\pgfusepath{stroke}
\pgfpathmoveto{\pgfpoint{208.799988pt}{189.587372pt}}
\pgflineto{\pgfpoint{208.799988pt}{183.410522pt}}
\pgfusepath{stroke}
\pgfpathmoveto{\pgfpoint{208.799988pt}{195.764206pt}}
\pgflineto{\pgfpoint{208.799988pt}{189.587372pt}}
\pgfusepath{stroke}
\pgfpathmoveto{\pgfpoint{208.799988pt}{152.526306pt}}
\pgflineto{\pgfpoint{208.818176pt}{152.526306pt}}
\pgfusepath{stroke}
\pgfpathmoveto{\pgfpoint{208.799988pt}{158.703156pt}}
\pgflineto{\pgfpoint{208.818237pt}{158.703156pt}}
\pgfusepath{stroke}
\pgfpathmoveto{\pgfpoint{208.799988pt}{164.880005pt}}
\pgflineto{\pgfpoint{208.818253pt}{164.880005pt}}
\pgfusepath{stroke}
\pgfpathmoveto{\pgfpoint{208.799988pt}{171.056854pt}}
\pgflineto{\pgfpoint{208.818283pt}{171.056854pt}}
\pgfusepath{stroke}
\pgfpathmoveto{\pgfpoint{208.799988pt}{177.233673pt}}
\pgflineto{\pgfpoint{208.827301pt}{177.233673pt}}
\pgfusepath{stroke}
\pgfpathmoveto{\pgfpoint{208.799988pt}{183.410522pt}}
\pgflineto{\pgfpoint{208.827332pt}{183.410522pt}}
\pgfusepath{stroke}
\pgfpathmoveto{\pgfpoint{208.799988pt}{189.587372pt}}
\pgflineto{\pgfpoint{208.827332pt}{189.587372pt}}
\pgfusepath{stroke}
\pgfpathmoveto{\pgfpoint{208.799988pt}{208.117905pt}}
\pgflineto{\pgfpoint{208.799988pt}{201.941055pt}}
\pgfusepath{stroke}
\pgfpathmoveto{\pgfpoint{208.799988pt}{201.941055pt}}
\pgflineto{\pgfpoint{208.799988pt}{195.764206pt}}
\pgfusepath{stroke}
\pgfpathmoveto{\pgfpoint{208.799988pt}{214.294739pt}}
\pgflineto{\pgfpoint{208.799988pt}{208.117905pt}}
\pgfusepath{stroke}
\pgfpathmoveto{\pgfpoint{208.799988pt}{226.648422pt}}
\pgflineto{\pgfpoint{208.772873pt}{226.648422pt}}
\pgfusepath{stroke}
\pgfpathmoveto{\pgfpoint{208.799988pt}{232.825272pt}}
\pgflineto{\pgfpoint{208.763824pt}{232.825272pt}}
\pgfusepath{stroke}
\pgfpathmoveto{\pgfpoint{208.799988pt}{239.002106pt}}
\pgflineto{\pgfpoint{208.763779pt}{239.002106pt}}
\pgfusepath{stroke}
\pgfpathmoveto{\pgfpoint{208.799988pt}{245.178955pt}}
\pgflineto{\pgfpoint{208.754745pt}{245.178955pt}}
\pgfusepath{stroke}
\pgfpathmoveto{\pgfpoint{208.799988pt}{251.355804pt}}
\pgflineto{\pgfpoint{208.745697pt}{251.355804pt}}
\pgfusepath{stroke}
\pgfpathmoveto{\pgfpoint{208.799988pt}{257.532623pt}}
\pgflineto{\pgfpoint{208.736710pt}{257.532623pt}}
\pgfusepath{stroke}
\pgfpathmoveto{\pgfpoint{208.799988pt}{263.709473pt}}
\pgflineto{\pgfpoint{208.727692pt}{263.709473pt}}
\pgfusepath{stroke}
\pgfpathmoveto{\pgfpoint{208.799988pt}{269.886322pt}}
\pgflineto{\pgfpoint{208.718674pt}{269.886322pt}}
\pgfusepath{stroke}
\pgfpathmoveto{\pgfpoint{208.799988pt}{276.063141pt}}
\pgflineto{\pgfpoint{208.714081pt}{276.063141pt}}
\pgfusepath{stroke}
\pgfpathmoveto{\pgfpoint{208.799988pt}{282.239990pt}}
\pgflineto{\pgfpoint{208.705032pt}{282.239990pt}}
\pgfusepath{stroke}
\pgfpathmoveto{\pgfpoint{208.799988pt}{288.416840pt}}
\pgflineto{\pgfpoint{208.700470pt}{288.416840pt}}
\pgfusepath{stroke}
\pgfpathmoveto{\pgfpoint{208.799988pt}{294.593689pt}}
\pgflineto{\pgfpoint{208.691452pt}{294.593689pt}}
\pgfusepath{stroke}
\pgfpathmoveto{\pgfpoint{208.799988pt}{300.770538pt}}
\pgflineto{\pgfpoint{208.682480pt}{300.770538pt}}
\pgfusepath{stroke}
\pgfpathmoveto{\pgfpoint{208.799988pt}{306.947388pt}}
\pgflineto{\pgfpoint{208.677948pt}{306.947388pt}}
\pgfusepath{stroke}
\pgfpathmoveto{\pgfpoint{208.799988pt}{313.124207pt}}
\pgflineto{\pgfpoint{208.668884pt}{313.124207pt}}
\pgfusepath{stroke}
\pgfpathmoveto{\pgfpoint{208.799988pt}{319.301056pt}}
\pgflineto{\pgfpoint{208.664352pt}{319.301056pt}}
\pgfusepath{stroke}
\pgfpathmoveto{\pgfpoint{208.799988pt}{325.477905pt}}
\pgflineto{\pgfpoint{208.655319pt}{325.477905pt}}
\pgfusepath{stroke}
\pgfpathmoveto{\pgfpoint{208.799988pt}{331.654724pt}}
\pgflineto{\pgfpoint{208.646240pt}{331.654724pt}}
\pgfusepath{stroke}
\pgfpathmoveto{\pgfpoint{208.799988pt}{337.831604pt}}
\pgflineto{\pgfpoint{208.641785pt}{337.831604pt}}
\pgfusepath{stroke}
\pgfpathmoveto{\pgfpoint{208.799988pt}{344.008423pt}}
\pgflineto{\pgfpoint{208.632721pt}{344.008423pt}}
\pgfusepath{stroke}
\pgfpathmoveto{\pgfpoint{208.799988pt}{350.185242pt}}
\pgflineto{\pgfpoint{208.628159pt}{350.185242pt}}
\pgfusepath{stroke}
\pgfpathmoveto{\pgfpoint{208.799988pt}{356.362122pt}}
\pgflineto{\pgfpoint{208.619171pt}{356.362122pt}}
\pgfusepath{stroke}
\pgfpathmoveto{\pgfpoint{208.799988pt}{362.538940pt}}
\pgflineto{\pgfpoint{208.614502pt}{362.538940pt}}
\pgfusepath{stroke}
\pgfpathmoveto{\pgfpoint{208.799988pt}{368.715820pt}}
\pgflineto{\pgfpoint{208.605621pt}{368.715820pt}}
\pgfusepath{stroke}
\pgfpathmoveto{\pgfpoint{208.799988pt}{313.124207pt}}
\pgflineto{\pgfpoint{208.799988pt}{306.947388pt}}
\pgfusepath{stroke}
\pgfpathmoveto{\pgfpoint{208.799988pt}{306.947388pt}}
\pgflineto{\pgfpoint{208.799988pt}{300.770538pt}}
\pgfusepath{stroke}
\pgfpathmoveto{\pgfpoint{208.799988pt}{319.301056pt}}
\pgflineto{\pgfpoint{208.799988pt}{313.124207pt}}
\pgfusepath{stroke}
\pgfpathmoveto{\pgfpoint{208.799988pt}{325.477905pt}}
\pgflineto{\pgfpoint{208.799988pt}{319.301056pt}}
\pgfusepath{stroke}
\pgfpathmoveto{\pgfpoint{208.799988pt}{306.947388pt}}
\pgflineto{\pgfpoint{208.813431pt}{306.947388pt}}
\pgfusepath{stroke}
\pgfpathmoveto{\pgfpoint{208.799988pt}{313.124207pt}}
\pgflineto{\pgfpoint{208.813431pt}{313.124207pt}}
\pgfusepath{stroke}
\pgfpathmoveto{\pgfpoint{208.799988pt}{319.301056pt}}
\pgflineto{\pgfpoint{208.817932pt}{319.301056pt}}
\pgfusepath{stroke}
\pgfpathmoveto{\pgfpoint{208.799988pt}{350.185242pt}}
\pgflineto{\pgfpoint{208.799988pt}{344.008423pt}}
\pgfusepath{stroke}
\pgfpathmoveto{\pgfpoint{208.799988pt}{331.654724pt}}
\pgflineto{\pgfpoint{208.799988pt}{325.477905pt}}
\pgfusepath{stroke}
\pgfpathmoveto{\pgfpoint{208.799988pt}{337.831604pt}}
\pgflineto{\pgfpoint{208.799988pt}{331.654724pt}}
\pgfusepath{stroke}
\pgfpathmoveto{\pgfpoint{208.799988pt}{344.008423pt}}
\pgflineto{\pgfpoint{208.799988pt}{337.831604pt}}
\pgfusepath{stroke}
\pgfpathmoveto{\pgfpoint{208.799988pt}{356.362122pt}}
\pgflineto{\pgfpoint{208.799988pt}{350.185242pt}}
\pgfusepath{stroke}
\pgfpathmoveto{\pgfpoint{208.799988pt}{362.538940pt}}
\pgflineto{\pgfpoint{208.799988pt}{356.362122pt}}
\pgfusepath{stroke}
\pgfpathmoveto{\pgfpoint{208.799988pt}{368.715820pt}}
\pgflineto{\pgfpoint{208.799988pt}{362.538940pt}}
\pgfusepath{stroke}
\pgfpathmoveto{\pgfpoint{208.799988pt}{325.477905pt}}
\pgflineto{\pgfpoint{208.817932pt}{325.477905pt}}
\pgfusepath{stroke}
\pgfpathmoveto{\pgfpoint{208.799988pt}{331.654724pt}}
\pgflineto{\pgfpoint{208.817902pt}{331.654724pt}}
\pgfusepath{stroke}
\pgfpathmoveto{\pgfpoint{208.799988pt}{337.831604pt}}
\pgflineto{\pgfpoint{208.822449pt}{337.831604pt}}
\pgfusepath{stroke}
\pgfpathmoveto{\pgfpoint{208.799988pt}{344.008423pt}}
\pgflineto{\pgfpoint{208.822388pt}{344.008423pt}}
\pgfusepath{stroke}
\pgfpathmoveto{\pgfpoint{208.799988pt}{350.185242pt}}
\pgflineto{\pgfpoint{208.826874pt}{350.185242pt}}
\pgfusepath{stroke}
\pgfpathmoveto{\pgfpoint{208.799988pt}{356.362122pt}}
\pgflineto{\pgfpoint{208.826904pt}{356.362122pt}}
\pgfusepath{stroke}
\pgfpathmoveto{\pgfpoint{208.799988pt}{362.538940pt}}
\pgflineto{\pgfpoint{208.831375pt}{362.538940pt}}
\pgfusepath{stroke}
\pgfpathmoveto{\pgfpoint{208.799988pt}{245.178955pt}}
\pgflineto{\pgfpoint{208.799988pt}{239.002106pt}}
\pgfusepath{stroke}
\pgfpathmoveto{\pgfpoint{208.799988pt}{220.471588pt}}
\pgflineto{\pgfpoint{208.799988pt}{214.294739pt}}
\pgfusepath{stroke}
\pgfpathmoveto{\pgfpoint{208.799988pt}{226.648422pt}}
\pgflineto{\pgfpoint{208.799988pt}{220.471588pt}}
\pgfusepath{stroke}
\pgfpathmoveto{\pgfpoint{208.799988pt}{232.825272pt}}
\pgflineto{\pgfpoint{208.799988pt}{226.648422pt}}
\pgfusepath{stroke}
\pgfpathmoveto{\pgfpoint{208.799988pt}{239.002106pt}}
\pgflineto{\pgfpoint{208.799988pt}{232.825272pt}}
\pgfusepath{stroke}
\pgfpathmoveto{\pgfpoint{208.799988pt}{251.355804pt}}
\pgflineto{\pgfpoint{208.799988pt}{245.178955pt}}
\pgfusepath{stroke}
\pgfpathmoveto{\pgfpoint{208.799988pt}{257.532623pt}}
\pgflineto{\pgfpoint{208.799988pt}{251.355804pt}}
\pgfusepath{stroke}
\pgfpathmoveto{\pgfpoint{208.799988pt}{263.709473pt}}
\pgflineto{\pgfpoint{208.799988pt}{257.532623pt}}
\pgfusepath{stroke}
\pgfpathmoveto{\pgfpoint{208.799988pt}{269.886322pt}}
\pgflineto{\pgfpoint{208.799988pt}{263.709473pt}}
\pgfusepath{stroke}
\pgfpathmoveto{\pgfpoint{208.799988pt}{276.063141pt}}
\pgflineto{\pgfpoint{208.799988pt}{269.886322pt}}
\pgfusepath{stroke}
\pgfpathmoveto{\pgfpoint{208.799988pt}{282.239990pt}}
\pgflineto{\pgfpoint{208.799988pt}{276.063141pt}}
\pgfusepath{stroke}
\pgfpathmoveto{\pgfpoint{208.799988pt}{288.416840pt}}
\pgflineto{\pgfpoint{208.799988pt}{282.239990pt}}
\pgfusepath{stroke}
\pgfpathmoveto{\pgfpoint{208.799988pt}{294.593689pt}}
\pgflineto{\pgfpoint{208.799988pt}{288.416840pt}}
\pgfusepath{stroke}
\pgfpathmoveto{\pgfpoint{208.799988pt}{300.770538pt}}
\pgflineto{\pgfpoint{208.799988pt}{294.593689pt}}
\pgfusepath{stroke}
\pgfpathmoveto{\pgfpoint{208.799988pt}{220.471588pt}}
\pgflineto{\pgfpoint{208.781891pt}{220.471588pt}}
\pgfusepath{stroke}
\pgfpathmoveto{\pgfpoint{208.799988pt}{195.764206pt}}
\pgflineto{\pgfpoint{208.827347pt}{195.764206pt}}
\pgfusepath{stroke}
\pgfpathmoveto{\pgfpoint{208.799988pt}{201.941055pt}}
\pgflineto{\pgfpoint{208.827408pt}{201.941055pt}}
\pgfusepath{stroke}
\pgfpathmoveto{\pgfpoint{208.799988pt}{208.117905pt}}
\pgflineto{\pgfpoint{208.836426pt}{208.117905pt}}
\pgfusepath{stroke}
\pgfpathmoveto{\pgfpoint{208.799988pt}{214.294739pt}}
\pgflineto{\pgfpoint{208.836456pt}{214.294739pt}}
\pgfusepath{stroke}
\pgfpathmoveto{\pgfpoint{208.799988pt}{220.471588pt}}
\pgflineto{\pgfpoint{208.836517pt}{220.471588pt}}
\pgfusepath{stroke}
\pgfpathmoveto{\pgfpoint{208.799988pt}{226.648422pt}}
\pgflineto{\pgfpoint{208.836517pt}{226.648422pt}}
\pgfusepath{stroke}
\pgfpathmoveto{\pgfpoint{208.799988pt}{232.825272pt}}
\pgflineto{\pgfpoint{208.836517pt}{232.825272pt}}
\pgfusepath{stroke}
\pgfpathmoveto{\pgfpoint{208.799988pt}{239.002106pt}}
\pgflineto{\pgfpoint{208.845520pt}{239.002106pt}}
\pgfusepath{stroke}
\pgfpathmoveto{\pgfpoint{208.799988pt}{245.178955pt}}
\pgflineto{\pgfpoint{208.845535pt}{245.178955pt}}
\pgfusepath{stroke}
\pgfpathmoveto{\pgfpoint{208.799988pt}{251.355804pt}}
\pgflineto{\pgfpoint{208.845612pt}{251.355804pt}}
\pgfusepath{stroke}
\pgfpathmoveto{\pgfpoint{208.799988pt}{257.532623pt}}
\pgflineto{\pgfpoint{208.845596pt}{257.532623pt}}
\pgfusepath{stroke}
\pgfpathmoveto{\pgfpoint{208.799988pt}{263.709473pt}}
\pgflineto{\pgfpoint{208.845627pt}{263.709473pt}}
\pgfusepath{stroke}
\pgfpathmoveto{\pgfpoint{208.799988pt}{269.886322pt}}
\pgflineto{\pgfpoint{208.845657pt}{269.886322pt}}
\pgfusepath{stroke}
\pgfpathmoveto{\pgfpoint{208.799988pt}{276.063141pt}}
\pgflineto{\pgfpoint{208.850143pt}{276.063141pt}}
\pgfusepath{stroke}
\pgfpathmoveto{\pgfpoint{208.799988pt}{282.239990pt}}
\pgflineto{\pgfpoint{208.850204pt}{282.239990pt}}
\pgfusepath{stroke}
\pgfpathmoveto{\pgfpoint{208.799988pt}{288.416840pt}}
\pgflineto{\pgfpoint{208.854752pt}{288.416840pt}}
\pgfusepath{stroke}
\pgfpathmoveto{\pgfpoint{208.799988pt}{294.593689pt}}
\pgflineto{\pgfpoint{208.854752pt}{294.593689pt}}
\pgfusepath{stroke}
\pgfpathmoveto{\pgfpoint{208.799988pt}{300.770538pt}}
\pgflineto{\pgfpoint{208.854782pt}{300.770538pt}}
\pgfusepath{stroke}
\pgfpathmoveto{\pgfpoint{208.859329pt}{306.947388pt}}
\pgflineto{\pgfpoint{208.813431pt}{306.947388pt}}
\pgfusepath{stroke}
\pgfpathmoveto{\pgfpoint{208.859329pt}{313.124207pt}}
\pgflineto{\pgfpoint{208.813431pt}{313.124207pt}}
\pgfusepath{stroke}
\pgfpathmoveto{\pgfpoint{208.863846pt}{319.301056pt}}
\pgflineto{\pgfpoint{208.817932pt}{319.301056pt}}
\pgfusepath{stroke}
\pgfpathmoveto{\pgfpoint{208.863876pt}{325.477905pt}}
\pgflineto{\pgfpoint{208.817932pt}{325.477905pt}}
\pgfusepath{stroke}
\pgfpathmoveto{\pgfpoint{208.863876pt}{331.654724pt}}
\pgflineto{\pgfpoint{208.817902pt}{331.654724pt}}
\pgfusepath{stroke}
\pgfpathmoveto{\pgfpoint{208.868454pt}{337.831604pt}}
\pgflineto{\pgfpoint{208.822449pt}{337.831604pt}}
\pgfusepath{stroke}
\pgfpathmoveto{\pgfpoint{208.868454pt}{344.008423pt}}
\pgflineto{\pgfpoint{208.822388pt}{344.008423pt}}
\pgfusepath{stroke}
\pgfpathmoveto{\pgfpoint{208.872940pt}{350.185242pt}}
\pgflineto{\pgfpoint{208.826874pt}{350.185242pt}}
\pgfusepath{stroke}
\pgfpathmoveto{\pgfpoint{208.873001pt}{356.362122pt}}
\pgflineto{\pgfpoint{208.826904pt}{356.362122pt}}
\pgfusepath{stroke}
\pgfpathmoveto{\pgfpoint{208.877502pt}{362.538940pt}}
\pgflineto{\pgfpoint{208.831375pt}{362.538940pt}}
\pgfusepath{stroke}
\pgfpathmoveto{\pgfpoint{208.799988pt}{96.934731pt}}
\pgflineto{\pgfpoint{208.799988pt}{90.757896pt}}
\pgfusepath{stroke}
\pgfpathmoveto{\pgfpoint{208.799988pt}{66.050522pt}}
\pgflineto{\pgfpoint{208.799988pt}{59.873672pt}}
\pgfusepath{stroke}
\pgfpathmoveto{\pgfpoint{208.799988pt}{72.227356pt}}
\pgflineto{\pgfpoint{208.799988pt}{66.050522pt}}
\pgfusepath{stroke}
\pgfpathmoveto{\pgfpoint{208.799988pt}{78.404205pt}}
\pgflineto{\pgfpoint{208.799988pt}{72.227356pt}}
\pgfusepath{stroke}
\pgfpathmoveto{\pgfpoint{208.799988pt}{84.581039pt}}
\pgflineto{\pgfpoint{208.799988pt}{78.404205pt}}
\pgfusepath{stroke}
\pgfpathmoveto{\pgfpoint{208.799988pt}{90.757896pt}}
\pgflineto{\pgfpoint{208.799988pt}{84.581039pt}}
\pgfusepath{stroke}
\pgfpathmoveto{\pgfpoint{208.799988pt}{103.111580pt}}
\pgflineto{\pgfpoint{208.799988pt}{96.934731pt}}
\pgfusepath{stroke}
\pgfpathmoveto{\pgfpoint{208.799988pt}{109.288422pt}}
\pgflineto{\pgfpoint{208.799988pt}{103.111580pt}}
\pgfusepath{stroke}
\pgfpathmoveto{\pgfpoint{208.799988pt}{115.465263pt}}
\pgflineto{\pgfpoint{208.799988pt}{109.288422pt}}
\pgfusepath{stroke}
\pgfpathmoveto{\pgfpoint{208.799988pt}{121.642097pt}}
\pgflineto{\pgfpoint{208.799988pt}{115.465263pt}}
\pgfusepath{stroke}
\pgfpathmoveto{\pgfpoint{208.799988pt}{127.818947pt}}
\pgflineto{\pgfpoint{208.799988pt}{121.642097pt}}
\pgfusepath{stroke}
\pgfpathmoveto{\pgfpoint{208.799988pt}{133.995789pt}}
\pgflineto{\pgfpoint{208.799988pt}{127.818947pt}}
\pgfusepath{stroke}
\pgfpathmoveto{\pgfpoint{208.799988pt}{140.172638pt}}
\pgflineto{\pgfpoint{208.799988pt}{133.995789pt}}
\pgfusepath{stroke}
\pgfpathmoveto{\pgfpoint{208.799988pt}{146.349472pt}}
\pgflineto{\pgfpoint{208.799988pt}{140.172638pt}}
\pgfusepath{stroke}
\pgfpathmoveto{\pgfpoint{208.799988pt}{47.519989pt}}
\pgflineto{\pgfpoint{208.781769pt}{47.519989pt}}
\pgfusepath{stroke}
\pgfpathmoveto{\pgfpoint{208.799988pt}{53.696838pt}}
\pgflineto{\pgfpoint{208.781799pt}{53.696838pt}}
\pgfusepath{stroke}
\pgfpathmoveto{\pgfpoint{208.799988pt}{59.873672pt}}
\pgflineto{\pgfpoint{208.799988pt}{53.696838pt}}
\pgfusepath{stroke}
\pgfpathmoveto{\pgfpoint{208.799988pt}{53.696838pt}}
\pgflineto{\pgfpoint{208.799988pt}{47.519989pt}}
\pgfusepath{stroke}
\pgfpathmoveto{\pgfpoint{208.799988pt}{47.519989pt}}
\pgflineto{\pgfpoint{217.709824pt}{47.519989pt}}
\pgfusepath{stroke}
\pgfpathmoveto{\pgfpoint{208.799988pt}{53.696838pt}}
\pgflineto{\pgfpoint{217.709854pt}{53.696838pt}}
\pgfusepath{stroke}
\pgfpathmoveto{\pgfpoint{208.799988pt}{59.873672pt}}
\pgflineto{\pgfpoint{217.709869pt}{59.873672pt}}
\pgfusepath{stroke}
\pgfpathmoveto{\pgfpoint{208.799988pt}{66.050522pt}}
\pgflineto{\pgfpoint{217.709869pt}{66.050522pt}}
\pgfusepath{stroke}
\pgfpathmoveto{\pgfpoint{217.727982pt}{72.227356pt}}
\pgflineto{\pgfpoint{208.799988pt}{72.227356pt}}
\pgfusepath{stroke}
\pgfpathmoveto{\pgfpoint{217.727982pt}{78.404205pt}}
\pgflineto{\pgfpoint{208.799988pt}{78.404205pt}}
\pgfusepath{stroke}
\pgfpathmoveto{\pgfpoint{217.727982pt}{84.581039pt}}
\pgflineto{\pgfpoint{208.799988pt}{84.581039pt}}
\pgfusepath{stroke}
\pgfpathmoveto{\pgfpoint{217.727982pt}{90.757896pt}}
\pgflineto{\pgfpoint{208.799988pt}{90.757896pt}}
\pgfusepath{stroke}
\pgfpathmoveto{\pgfpoint{217.727982pt}{96.934731pt}}
\pgflineto{\pgfpoint{208.799988pt}{96.934731pt}}
\pgfusepath{stroke}
\pgfpathmoveto{\pgfpoint{217.727982pt}{103.111580pt}}
\pgflineto{\pgfpoint{208.799988pt}{103.111580pt}}
\pgfusepath{stroke}
\pgfpathmoveto{\pgfpoint{217.727982pt}{109.288422pt}}
\pgflineto{\pgfpoint{208.799988pt}{109.288422pt}}
\pgfusepath{stroke}
\pgfpathmoveto{\pgfpoint{217.727982pt}{115.465263pt}}
\pgflineto{\pgfpoint{208.799988pt}{115.465263pt}}
\pgfusepath{stroke}
\pgfpathmoveto{\pgfpoint{217.727982pt}{121.642097pt}}
\pgflineto{\pgfpoint{208.799988pt}{121.642097pt}}
\pgfusepath{stroke}
\pgfpathmoveto{\pgfpoint{217.727982pt}{127.818947pt}}
\pgflineto{\pgfpoint{208.799988pt}{127.818947pt}}
\pgfusepath{stroke}
\pgfpathmoveto{\pgfpoint{217.727982pt}{133.995789pt}}
\pgflineto{\pgfpoint{208.799988pt}{133.995789pt}}
\pgfusepath{stroke}
\pgfpathmoveto{\pgfpoint{217.727982pt}{140.172638pt}}
\pgflineto{\pgfpoint{208.799988pt}{140.172638pt}}
\pgfusepath{stroke}
\pgfpathmoveto{\pgfpoint{217.727982pt}{146.349472pt}}
\pgflineto{\pgfpoint{208.799988pt}{146.349472pt}}
\pgfusepath{stroke}
\pgfpathmoveto{\pgfpoint{217.727982pt}{152.526306pt}}
\pgflineto{\pgfpoint{208.818176pt}{152.526306pt}}
\pgfusepath{stroke}
\pgfpathmoveto{\pgfpoint{217.727982pt}{158.703156pt}}
\pgflineto{\pgfpoint{208.818237pt}{158.703156pt}}
\pgfusepath{stroke}
\pgfpathmoveto{\pgfpoint{217.727982pt}{164.880005pt}}
\pgflineto{\pgfpoint{208.818253pt}{164.880005pt}}
\pgfusepath{stroke}
\pgfpathmoveto{\pgfpoint{217.727982pt}{171.056854pt}}
\pgflineto{\pgfpoint{208.818283pt}{171.056854pt}}
\pgfusepath{stroke}
\pgfpathmoveto{\pgfpoint{217.727982pt}{177.233673pt}}
\pgflineto{\pgfpoint{208.827301pt}{177.233673pt}}
\pgfusepath{stroke}
\pgfpathmoveto{\pgfpoint{217.727982pt}{183.410522pt}}
\pgflineto{\pgfpoint{208.827332pt}{183.410522pt}}
\pgfusepath{stroke}
\pgfpathmoveto{\pgfpoint{217.727982pt}{189.587372pt}}
\pgflineto{\pgfpoint{208.827332pt}{189.587372pt}}
\pgfusepath{stroke}
\pgfpathmoveto{\pgfpoint{217.727982pt}{195.764206pt}}
\pgflineto{\pgfpoint{208.827347pt}{195.764206pt}}
\pgfusepath{stroke}
\pgfpathmoveto{\pgfpoint{217.727982pt}{201.941055pt}}
\pgflineto{\pgfpoint{208.827408pt}{201.941055pt}}
\pgfusepath{stroke}
\pgfpathmoveto{\pgfpoint{217.727982pt}{208.117905pt}}
\pgflineto{\pgfpoint{208.836426pt}{208.117905pt}}
\pgfusepath{stroke}
\pgfpathmoveto{\pgfpoint{217.727982pt}{214.294739pt}}
\pgflineto{\pgfpoint{208.836456pt}{214.294739pt}}
\pgfusepath{stroke}
\pgfpathmoveto{\pgfpoint{217.727982pt}{220.471588pt}}
\pgflineto{\pgfpoint{208.836517pt}{220.471588pt}}
\pgfusepath{stroke}
\pgfpathmoveto{\pgfpoint{217.727982pt}{226.648422pt}}
\pgflineto{\pgfpoint{208.836517pt}{226.648422pt}}
\pgfusepath{stroke}
\pgfpathmoveto{\pgfpoint{208.836517pt}{232.825272pt}}
\pgflineto{\pgfpoint{217.709869pt}{232.825272pt}}
\pgfusepath{stroke}
\pgfpathmoveto{\pgfpoint{208.845520pt}{239.002106pt}}
\pgflineto{\pgfpoint{217.700851pt}{239.002106pt}}
\pgfusepath{stroke}
\pgfpathmoveto{\pgfpoint{208.845535pt}{245.178955pt}}
\pgflineto{\pgfpoint{217.691833pt}{245.178955pt}}
\pgfusepath{stroke}
\pgfpathmoveto{\pgfpoint{208.845612pt}{251.355804pt}}
\pgflineto{\pgfpoint{217.691742pt}{251.355804pt}}
\pgfusepath{stroke}
\pgfpathmoveto{\pgfpoint{208.845596pt}{257.532623pt}}
\pgflineto{\pgfpoint{217.678177pt}{257.532623pt}}
\pgfusepath{stroke}
\pgfpathmoveto{\pgfpoint{208.845627pt}{263.709473pt}}
\pgflineto{\pgfpoint{217.669113pt}{263.709473pt}}
\pgfusepath{stroke}
\pgfpathmoveto{\pgfpoint{208.845657pt}{269.886322pt}}
\pgflineto{\pgfpoint{217.660110pt}{269.886322pt}}
\pgfusepath{stroke}
\pgfpathmoveto{\pgfpoint{208.850143pt}{276.063141pt}}
\pgflineto{\pgfpoint{217.655533pt}{276.063141pt}}
\pgfusepath{stroke}
\pgfpathmoveto{\pgfpoint{208.850204pt}{282.239990pt}}
\pgflineto{\pgfpoint{217.646500pt}{282.239990pt}}
\pgfusepath{stroke}
\pgfpathmoveto{\pgfpoint{208.854752pt}{288.416840pt}}
\pgflineto{\pgfpoint{217.641968pt}{288.416840pt}}
\pgfusepath{stroke}
\pgfpathmoveto{\pgfpoint{208.854752pt}{294.593689pt}}
\pgflineto{\pgfpoint{217.632919pt}{294.593689pt}}
\pgfusepath{stroke}
\pgfpathmoveto{\pgfpoint{208.854782pt}{300.770538pt}}
\pgflineto{\pgfpoint{217.628342pt}{300.770538pt}}
\pgfusepath{stroke}
\pgfpathmoveto{\pgfpoint{208.859329pt}{306.947388pt}}
\pgflineto{\pgfpoint{217.619354pt}{306.947388pt}}
\pgfusepath{stroke}
\pgfpathmoveto{\pgfpoint{208.859329pt}{313.124207pt}}
\pgflineto{\pgfpoint{217.610291pt}{313.124207pt}}
\pgfusepath{stroke}
\pgfpathmoveto{\pgfpoint{208.863846pt}{319.301056pt}}
\pgflineto{\pgfpoint{217.605743pt}{319.301056pt}}
\pgfusepath{stroke}
\pgfpathmoveto{\pgfpoint{208.863876pt}{325.477905pt}}
\pgflineto{\pgfpoint{217.596710pt}{325.477905pt}}
\pgfusepath{stroke}
\pgfpathmoveto{\pgfpoint{208.863876pt}{331.654724pt}}
\pgflineto{\pgfpoint{217.592072pt}{331.654724pt}}
\pgfusepath{stroke}
\pgfpathmoveto{\pgfpoint{208.868454pt}{337.831604pt}}
\pgflineto{\pgfpoint{217.583130pt}{337.831604pt}}
\pgfusepath{stroke}
\pgfpathmoveto{\pgfpoint{208.868454pt}{344.008423pt}}
\pgflineto{\pgfpoint{217.574036pt}{344.008423pt}}
\pgfusepath{stroke}
\pgfpathmoveto{\pgfpoint{208.872940pt}{350.185242pt}}
\pgflineto{\pgfpoint{217.569489pt}{350.185242pt}}
\pgfusepath{stroke}
\pgfpathmoveto{\pgfpoint{208.873001pt}{356.362122pt}}
\pgflineto{\pgfpoint{217.560486pt}{356.362122pt}}
\pgfusepath{stroke}
\pgfpathmoveto{\pgfpoint{208.877502pt}{362.538940pt}}
\pgflineto{\pgfpoint{217.555908pt}{362.538940pt}}
\pgfusepath{stroke}
\pgfpathmoveto{\pgfpoint{217.727982pt}{171.056854pt}}
\pgflineto{\pgfpoint{217.727982pt}{164.880005pt}}
\pgfusepath{stroke}
\pgfpathmoveto{\pgfpoint{217.727982pt}{164.880005pt}}
\pgflineto{\pgfpoint{217.727982pt}{158.703156pt}}
\pgfusepath{stroke}
\pgfpathmoveto{\pgfpoint{217.727982pt}{177.233673pt}}
\pgflineto{\pgfpoint{217.727982pt}{171.056854pt}}
\pgfusepath{stroke}
\pgfpathmoveto{\pgfpoint{217.727982pt}{183.410522pt}}
\pgflineto{\pgfpoint{217.727982pt}{177.233673pt}}
\pgfusepath{stroke}
\pgfpathmoveto{\pgfpoint{217.727982pt}{189.587372pt}}
\pgflineto{\pgfpoint{217.727982pt}{183.410522pt}}
\pgfusepath{stroke}
\pgfpathmoveto{\pgfpoint{217.727982pt}{195.764206pt}}
\pgflineto{\pgfpoint{217.727982pt}{189.587372pt}}
\pgfusepath{stroke}
\pgfpathmoveto{\pgfpoint{217.727982pt}{201.941055pt}}
\pgflineto{\pgfpoint{217.727982pt}{195.764206pt}}
\pgfusepath{stroke}
\pgfpathmoveto{\pgfpoint{217.727982pt}{208.117905pt}}
\pgflineto{\pgfpoint{217.727982pt}{201.941055pt}}
\pgfusepath{stroke}
\pgfpathmoveto{\pgfpoint{217.727982pt}{164.880005pt}}
\pgflineto{\pgfpoint{217.746140pt}{164.880005pt}}
\pgfusepath{stroke}
\pgfpathmoveto{\pgfpoint{217.727982pt}{171.056854pt}}
\pgflineto{\pgfpoint{217.746140pt}{171.056854pt}}
\pgfusepath{stroke}
\pgfpathmoveto{\pgfpoint{217.727982pt}{177.233673pt}}
\pgflineto{\pgfpoint{217.746155pt}{177.233673pt}}
\pgfusepath{stroke}
\pgfpathmoveto{\pgfpoint{217.727982pt}{183.410522pt}}
\pgflineto{\pgfpoint{217.746155pt}{183.410522pt}}
\pgfusepath{stroke}
\pgfpathmoveto{\pgfpoint{217.727982pt}{189.587372pt}}
\pgflineto{\pgfpoint{217.746155pt}{189.587372pt}}
\pgfusepath{stroke}
\pgfpathmoveto{\pgfpoint{217.727982pt}{195.764206pt}}
\pgflineto{\pgfpoint{217.755173pt}{195.764206pt}}
\pgfusepath{stroke}
\pgfpathmoveto{\pgfpoint{217.727982pt}{201.941055pt}}
\pgflineto{\pgfpoint{217.755234pt}{201.941055pt}}
\pgfusepath{stroke}
\pgfpathmoveto{\pgfpoint{217.727982pt}{220.471588pt}}
\pgflineto{\pgfpoint{217.727982pt}{214.294739pt}}
\pgfusepath{stroke}
\pgfpathmoveto{\pgfpoint{217.727982pt}{214.294739pt}}
\pgflineto{\pgfpoint{217.727982pt}{208.117905pt}}
\pgfusepath{stroke}
\pgfpathmoveto{\pgfpoint{217.727982pt}{226.648422pt}}
\pgflineto{\pgfpoint{217.727982pt}{220.471588pt}}
\pgfusepath{stroke}
\pgfpathmoveto{\pgfpoint{217.691833pt}{245.178955pt}}
\pgflineto{\pgfpoint{217.714432pt}{245.178955pt}}
\pgfusepath{stroke}
\pgfpathmoveto{\pgfpoint{217.727982pt}{251.355804pt}}
\pgflineto{\pgfpoint{217.691742pt}{251.355804pt}}
\pgfusepath{stroke}
\pgfpathmoveto{\pgfpoint{217.727982pt}{257.532623pt}}
\pgflineto{\pgfpoint{217.678177pt}{257.532623pt}}
\pgfusepath{stroke}
\pgfpathmoveto{\pgfpoint{217.727982pt}{263.709473pt}}
\pgflineto{\pgfpoint{217.669113pt}{263.709473pt}}
\pgfusepath{stroke}
\pgfpathmoveto{\pgfpoint{217.727982pt}{269.886322pt}}
\pgflineto{\pgfpoint{217.660110pt}{269.886322pt}}
\pgfusepath{stroke}
\pgfpathmoveto{\pgfpoint{217.727982pt}{276.063141pt}}
\pgflineto{\pgfpoint{217.655533pt}{276.063141pt}}
\pgfusepath{stroke}
\pgfpathmoveto{\pgfpoint{217.727982pt}{282.239990pt}}
\pgflineto{\pgfpoint{217.646500pt}{282.239990pt}}
\pgfusepath{stroke}
\pgfpathmoveto{\pgfpoint{217.727982pt}{288.416840pt}}
\pgflineto{\pgfpoint{217.641968pt}{288.416840pt}}
\pgfusepath{stroke}
\pgfpathmoveto{\pgfpoint{217.727982pt}{294.593689pt}}
\pgflineto{\pgfpoint{217.632919pt}{294.593689pt}}
\pgfusepath{stroke}
\pgfpathmoveto{\pgfpoint{217.727982pt}{300.770538pt}}
\pgflineto{\pgfpoint{217.628342pt}{300.770538pt}}
\pgfusepath{stroke}
\pgfpathmoveto{\pgfpoint{217.727982pt}{306.947388pt}}
\pgflineto{\pgfpoint{217.619354pt}{306.947388pt}}
\pgfusepath{stroke}
\pgfpathmoveto{\pgfpoint{217.727982pt}{313.124207pt}}
\pgflineto{\pgfpoint{217.610291pt}{313.124207pt}}
\pgfusepath{stroke}
\pgfpathmoveto{\pgfpoint{217.727982pt}{319.301056pt}}
\pgflineto{\pgfpoint{217.605743pt}{319.301056pt}}
\pgfusepath{stroke}
\pgfpathmoveto{\pgfpoint{217.727982pt}{325.477905pt}}
\pgflineto{\pgfpoint{217.596710pt}{325.477905pt}}
\pgfusepath{stroke}
\pgfpathmoveto{\pgfpoint{217.727982pt}{331.654724pt}}
\pgflineto{\pgfpoint{217.592072pt}{331.654724pt}}
\pgfusepath{stroke}
\pgfpathmoveto{\pgfpoint{217.727982pt}{337.831604pt}}
\pgflineto{\pgfpoint{217.583130pt}{337.831604pt}}
\pgfusepath{stroke}
\pgfpathmoveto{\pgfpoint{217.727982pt}{344.008423pt}}
\pgflineto{\pgfpoint{217.574036pt}{344.008423pt}}
\pgfusepath{stroke}
\pgfpathmoveto{\pgfpoint{217.727982pt}{350.185242pt}}
\pgflineto{\pgfpoint{217.569489pt}{350.185242pt}}
\pgfusepath{stroke}
\pgfpathmoveto{\pgfpoint{217.727982pt}{356.362122pt}}
\pgflineto{\pgfpoint{217.560486pt}{356.362122pt}}
\pgfusepath{stroke}
\pgfpathmoveto{\pgfpoint{217.727982pt}{362.538940pt}}
\pgflineto{\pgfpoint{217.555908pt}{362.538940pt}}
\pgfusepath{stroke}
\pgfpathmoveto{\pgfpoint{217.727982pt}{350.185242pt}}
\pgflineto{\pgfpoint{217.727982pt}{344.008423pt}}
\pgfusepath{stroke}
\pgfpathmoveto{\pgfpoint{217.727982pt}{331.654724pt}}
\pgflineto{\pgfpoint{217.727982pt}{325.477905pt}}
\pgfusepath{stroke}
\pgfpathmoveto{\pgfpoint{217.727982pt}{337.831604pt}}
\pgflineto{\pgfpoint{217.727982pt}{331.654724pt}}
\pgfusepath{stroke}
\pgfpathmoveto{\pgfpoint{217.727982pt}{344.008423pt}}
\pgflineto{\pgfpoint{217.727982pt}{337.831604pt}}
\pgfusepath{stroke}
\pgfpathmoveto{\pgfpoint{217.727982pt}{356.362122pt}}
\pgflineto{\pgfpoint{217.727982pt}{350.185242pt}}
\pgfusepath{stroke}
\pgfpathmoveto{\pgfpoint{217.727982pt}{362.538940pt}}
\pgflineto{\pgfpoint{217.727982pt}{356.362122pt}}
\pgfusepath{stroke}
\pgfpathmoveto{\pgfpoint{217.727982pt}{331.654724pt}}
\pgflineto{\pgfpoint{217.741425pt}{331.654724pt}}
\pgfusepath{stroke}
\pgfpathmoveto{\pgfpoint{217.727982pt}{337.831604pt}}
\pgflineto{\pgfpoint{217.741486pt}{337.831604pt}}
\pgfusepath{stroke}
\pgfpathmoveto{\pgfpoint{217.727982pt}{344.008423pt}}
\pgflineto{\pgfpoint{217.741440pt}{344.008423pt}}
\pgfusepath{stroke}
\pgfpathmoveto{\pgfpoint{217.727982pt}{350.185242pt}}
\pgflineto{\pgfpoint{217.745926pt}{350.185242pt}}
\pgfusepath{stroke}
\pgfpathmoveto{\pgfpoint{217.727982pt}{356.362122pt}}
\pgflineto{\pgfpoint{217.745941pt}{356.362122pt}}
\pgfusepath{stroke}
\pgfpathmoveto{\pgfpoint{217.727982pt}{294.593689pt}}
\pgflineto{\pgfpoint{217.727982pt}{288.416840pt}}
\pgfusepath{stroke}
\pgfpathmoveto{\pgfpoint{217.727982pt}{276.063141pt}}
\pgflineto{\pgfpoint{217.727982pt}{269.886322pt}}
\pgfusepath{stroke}
\pgfpathmoveto{\pgfpoint{217.727982pt}{282.239990pt}}
\pgflineto{\pgfpoint{217.727982pt}{276.063141pt}}
\pgfusepath{stroke}
\pgfpathmoveto{\pgfpoint{217.727982pt}{288.416840pt}}
\pgflineto{\pgfpoint{217.727982pt}{282.239990pt}}
\pgfusepath{stroke}
\pgfpathmoveto{\pgfpoint{217.727982pt}{300.770538pt}}
\pgflineto{\pgfpoint{217.727982pt}{294.593689pt}}
\pgfusepath{stroke}
\pgfpathmoveto{\pgfpoint{217.727982pt}{306.947388pt}}
\pgflineto{\pgfpoint{217.727982pt}{300.770538pt}}
\pgfusepath{stroke}
\pgfpathmoveto{\pgfpoint{217.727982pt}{313.124207pt}}
\pgflineto{\pgfpoint{217.727982pt}{306.947388pt}}
\pgfusepath{stroke}
\pgfpathmoveto{\pgfpoint{217.727982pt}{319.301056pt}}
\pgflineto{\pgfpoint{217.727982pt}{313.124207pt}}
\pgfusepath{stroke}
\pgfpathmoveto{\pgfpoint{217.727982pt}{325.477905pt}}
\pgflineto{\pgfpoint{217.727982pt}{319.301056pt}}
\pgfusepath{stroke}
\pgfpathmoveto{\pgfpoint{217.727982pt}{276.063141pt}}
\pgflineto{\pgfpoint{217.746155pt}{276.063141pt}}
\pgfusepath{stroke}
\pgfpathmoveto{\pgfpoint{217.727982pt}{282.239990pt}}
\pgflineto{\pgfpoint{217.750702pt}{282.239990pt}}
\pgfusepath{stroke}
\pgfpathmoveto{\pgfpoint{217.727982pt}{288.416840pt}}
\pgflineto{\pgfpoint{217.759735pt}{288.416840pt}}
\pgfusepath{stroke}
\pgfpathmoveto{\pgfpoint{217.727982pt}{294.593689pt}}
\pgflineto{\pgfpoint{217.764313pt}{294.593689pt}}
\pgfusepath{stroke}
\pgfpathmoveto{\pgfpoint{217.727982pt}{300.770538pt}}
\pgflineto{\pgfpoint{217.782486pt}{300.770538pt}}
\pgfusepath{stroke}
\pgfpathmoveto{\pgfpoint{217.727982pt}{306.947388pt}}
\pgflineto{\pgfpoint{217.782486pt}{306.947388pt}}
\pgfusepath{stroke}
\pgfpathmoveto{\pgfpoint{217.727982pt}{313.124207pt}}
\pgflineto{\pgfpoint{217.782486pt}{313.124207pt}}
\pgfusepath{stroke}
\pgfpathmoveto{\pgfpoint{217.727982pt}{319.301056pt}}
\pgflineto{\pgfpoint{217.787003pt}{319.301056pt}}
\pgfusepath{stroke}
\pgfpathmoveto{\pgfpoint{217.727982pt}{325.477905pt}}
\pgflineto{\pgfpoint{217.787033pt}{325.477905pt}}
\pgfusepath{stroke}
\pgfpathmoveto{\pgfpoint{217.791519pt}{331.654724pt}}
\pgflineto{\pgfpoint{217.741425pt}{331.654724pt}}
\pgfusepath{stroke}
\pgfpathmoveto{\pgfpoint{217.791580pt}{337.831604pt}}
\pgflineto{\pgfpoint{217.741486pt}{337.831604pt}}
\pgfusepath{stroke}
\pgfpathmoveto{\pgfpoint{217.791580pt}{344.008423pt}}
\pgflineto{\pgfpoint{217.741440pt}{344.008423pt}}
\pgfusepath{stroke}
\pgfpathmoveto{\pgfpoint{217.796021pt}{350.185242pt}}
\pgflineto{\pgfpoint{217.745926pt}{350.185242pt}}
\pgfusepath{stroke}
\pgfpathmoveto{\pgfpoint{217.796097pt}{356.362122pt}}
\pgflineto{\pgfpoint{217.745941pt}{356.362122pt}}
\pgfusepath{stroke}
\pgfpathmoveto{\pgfpoint{217.727982pt}{257.532623pt}}
\pgflineto{\pgfpoint{217.727982pt}{251.355804pt}}
\pgfusepath{stroke}
\pgfpathmoveto{\pgfpoint{217.727982pt}{263.709473pt}}
\pgflineto{\pgfpoint{217.727982pt}{257.532623pt}}
\pgfusepath{stroke}
\pgfpathmoveto{\pgfpoint{217.727982pt}{269.886322pt}}
\pgflineto{\pgfpoint{217.727982pt}{263.709473pt}}
\pgfusepath{stroke}
\pgfpathmoveto{\pgfpoint{217.727982pt}{232.825272pt}}
\pgflineto{\pgfpoint{217.709869pt}{232.825272pt}}
\pgfusepath{stroke}
\pgfpathmoveto{\pgfpoint{217.727982pt}{239.002106pt}}
\pgflineto{\pgfpoint{217.700851pt}{239.002106pt}}
\pgfusepath{stroke}
\pgfpathmoveto{\pgfpoint{217.727982pt}{245.178955pt}}
\pgflineto{\pgfpoint{217.714432pt}{245.178955pt}}
\pgfusepath{stroke}
\pgfpathmoveto{\pgfpoint{217.727982pt}{232.825272pt}}
\pgflineto{\pgfpoint{217.727982pt}{226.648422pt}}
\pgfusepath{stroke}
\pgfpathmoveto{\pgfpoint{217.727982pt}{239.002106pt}}
\pgflineto{\pgfpoint{217.727982pt}{232.825272pt}}
\pgfusepath{stroke}
\pgfpathmoveto{\pgfpoint{217.727982pt}{245.178955pt}}
\pgflineto{\pgfpoint{217.727982pt}{239.002106pt}}
\pgfusepath{stroke}
\pgfpathmoveto{\pgfpoint{217.727982pt}{251.355804pt}}
\pgflineto{\pgfpoint{217.727982pt}{245.178955pt}}
\pgfusepath{stroke}
\pgfpathmoveto{\pgfpoint{217.727982pt}{208.117905pt}}
\pgflineto{\pgfpoint{217.755264pt}{208.117905pt}}
\pgfusepath{stroke}
\pgfpathmoveto{\pgfpoint{217.727982pt}{214.294739pt}}
\pgflineto{\pgfpoint{217.755264pt}{214.294739pt}}
\pgfusepath{stroke}
\pgfpathmoveto{\pgfpoint{217.727982pt}{220.471588pt}}
\pgflineto{\pgfpoint{217.764267pt}{220.471588pt}}
\pgfusepath{stroke}
\pgfpathmoveto{\pgfpoint{217.727982pt}{226.648422pt}}
\pgflineto{\pgfpoint{217.764267pt}{226.648422pt}}
\pgfusepath{stroke}
\pgfpathmoveto{\pgfpoint{217.727982pt}{232.825272pt}}
\pgflineto{\pgfpoint{217.764313pt}{232.825272pt}}
\pgfusepath{stroke}
\pgfpathmoveto{\pgfpoint{217.727982pt}{239.002106pt}}
\pgflineto{\pgfpoint{217.764313pt}{239.002106pt}}
\pgfusepath{stroke}
\pgfpathmoveto{\pgfpoint{217.727982pt}{245.178955pt}}
\pgflineto{\pgfpoint{217.764359pt}{245.178955pt}}
\pgfusepath{stroke}
\pgfpathmoveto{\pgfpoint{217.727982pt}{251.355804pt}}
\pgflineto{\pgfpoint{217.773346pt}{251.355804pt}}
\pgfusepath{stroke}
\pgfpathmoveto{\pgfpoint{217.727982pt}{257.532623pt}}
\pgflineto{\pgfpoint{217.768845pt}{257.532623pt}}
\pgfusepath{stroke}
\pgfpathmoveto{\pgfpoint{217.727982pt}{263.709473pt}}
\pgflineto{\pgfpoint{217.768860pt}{263.709473pt}}
\pgfusepath{stroke}
\pgfpathmoveto{\pgfpoint{217.727982pt}{269.886322pt}}
\pgflineto{\pgfpoint{217.768921pt}{269.886322pt}}
\pgfusepath{stroke}
\pgfpathmoveto{\pgfpoint{217.773392pt}{276.063141pt}}
\pgflineto{\pgfpoint{217.746155pt}{276.063141pt}}
\pgfusepath{stroke}
\pgfpathmoveto{\pgfpoint{217.773392pt}{282.239990pt}}
\pgflineto{\pgfpoint{217.750702pt}{282.239990pt}}
\pgfusepath{stroke}
\pgfpathmoveto{\pgfpoint{217.777939pt}{288.416840pt}}
\pgflineto{\pgfpoint{217.759735pt}{288.416840pt}}
\pgfusepath{stroke}
\pgfpathmoveto{\pgfpoint{217.777939pt}{294.593689pt}}
\pgflineto{\pgfpoint{217.764313pt}{294.593689pt}}
\pgfusepath{stroke}
\pgfpathmoveto{\pgfpoint{217.727982pt}{109.288422pt}}
\pgflineto{\pgfpoint{217.727982pt}{103.111580pt}}
\pgfusepath{stroke}
\pgfpathmoveto{\pgfpoint{217.727982pt}{78.404205pt}}
\pgflineto{\pgfpoint{217.727982pt}{72.227356pt}}
\pgfusepath{stroke}
\pgfpathmoveto{\pgfpoint{217.727982pt}{84.581039pt}}
\pgflineto{\pgfpoint{217.727982pt}{78.404205pt}}
\pgfusepath{stroke}
\pgfpathmoveto{\pgfpoint{217.727982pt}{90.757896pt}}
\pgflineto{\pgfpoint{217.727982pt}{84.581039pt}}
\pgfusepath{stroke}
\pgfpathmoveto{\pgfpoint{217.727982pt}{96.934731pt}}
\pgflineto{\pgfpoint{217.727982pt}{90.757896pt}}
\pgfusepath{stroke}
\pgfpathmoveto{\pgfpoint{217.727982pt}{103.111580pt}}
\pgflineto{\pgfpoint{217.727982pt}{96.934731pt}}
\pgfusepath{stroke}
\pgfpathmoveto{\pgfpoint{217.727982pt}{115.465263pt}}
\pgflineto{\pgfpoint{217.727982pt}{109.288422pt}}
\pgfusepath{stroke}
\pgfpathmoveto{\pgfpoint{217.727982pt}{121.642097pt}}
\pgflineto{\pgfpoint{217.727982pt}{115.465263pt}}
\pgfusepath{stroke}
\pgfpathmoveto{\pgfpoint{217.727982pt}{127.818947pt}}
\pgflineto{\pgfpoint{217.727982pt}{121.642097pt}}
\pgfusepath{stroke}
\pgfpathmoveto{\pgfpoint{217.727982pt}{133.995789pt}}
\pgflineto{\pgfpoint{217.727982pt}{127.818947pt}}
\pgfusepath{stroke}
\pgfpathmoveto{\pgfpoint{217.727982pt}{140.172638pt}}
\pgflineto{\pgfpoint{217.727982pt}{133.995789pt}}
\pgfusepath{stroke}
\pgfpathmoveto{\pgfpoint{217.727982pt}{146.349472pt}}
\pgflineto{\pgfpoint{217.727982pt}{140.172638pt}}
\pgfusepath{stroke}
\pgfpathmoveto{\pgfpoint{217.727982pt}{152.526306pt}}
\pgflineto{\pgfpoint{217.727982pt}{146.349472pt}}
\pgfusepath{stroke}
\pgfpathmoveto{\pgfpoint{217.727982pt}{158.703156pt}}
\pgflineto{\pgfpoint{217.727982pt}{152.526306pt}}
\pgfusepath{stroke}
\pgfpathmoveto{\pgfpoint{217.727982pt}{47.519989pt}}
\pgflineto{\pgfpoint{217.709824pt}{47.519989pt}}
\pgfusepath{stroke}
\pgfpathmoveto{\pgfpoint{217.727982pt}{53.696838pt}}
\pgflineto{\pgfpoint{217.709854pt}{53.696838pt}}
\pgfusepath{stroke}
\pgfpathmoveto{\pgfpoint{217.727982pt}{59.873672pt}}
\pgflineto{\pgfpoint{217.709869pt}{59.873672pt}}
\pgfusepath{stroke}
\pgfpathmoveto{\pgfpoint{217.727982pt}{66.050522pt}}
\pgflineto{\pgfpoint{217.709869pt}{66.050522pt}}
\pgfusepath{stroke}
\pgfpathmoveto{\pgfpoint{217.727982pt}{53.696838pt}}
\pgflineto{\pgfpoint{217.727982pt}{47.519989pt}}
\pgfusepath{stroke}
\pgfpathmoveto{\pgfpoint{217.727982pt}{59.873672pt}}
\pgflineto{\pgfpoint{217.727982pt}{53.696838pt}}
\pgfusepath{stroke}
\pgfpathmoveto{\pgfpoint{217.727982pt}{47.519989pt}}
\pgflineto{\pgfpoint{217.755051pt}{47.519989pt}}
\pgfusepath{stroke}
\pgfpathmoveto{\pgfpoint{217.727982pt}{53.696838pt}}
\pgflineto{\pgfpoint{217.746048pt}{53.696838pt}}
\pgfusepath{stroke}
\pgfpathmoveto{\pgfpoint{217.727982pt}{72.227356pt}}
\pgflineto{\pgfpoint{217.727982pt}{66.050522pt}}
\pgfusepath{stroke}
\pgfpathmoveto{\pgfpoint{217.727982pt}{66.050522pt}}
\pgflineto{\pgfpoint{217.727982pt}{59.873672pt}}
\pgfusepath{stroke}
\pgfpathmoveto{\pgfpoint{217.755051pt}{47.519989pt}}
\pgflineto{\pgfpoint{226.637787pt}{47.519989pt}}
\pgfusepath{stroke}
\pgfpathmoveto{\pgfpoint{217.746048pt}{53.696838pt}}
\pgflineto{\pgfpoint{226.637817pt}{53.696838pt}}
\pgfusepath{stroke}
\pgfpathmoveto{\pgfpoint{226.655975pt}{59.873672pt}}
\pgflineto{\pgfpoint{217.727982pt}{59.873672pt}}
\pgfusepath{stroke}
\pgfpathmoveto{\pgfpoint{226.655975pt}{66.050522pt}}
\pgflineto{\pgfpoint{217.727982pt}{66.050522pt}}
\pgfusepath{stroke}
\pgfpathmoveto{\pgfpoint{226.655975pt}{72.227356pt}}
\pgflineto{\pgfpoint{217.727982pt}{72.227356pt}}
\pgfusepath{stroke}
\pgfpathmoveto{\pgfpoint{226.655975pt}{78.404205pt}}
\pgflineto{\pgfpoint{217.727982pt}{78.404205pt}}
\pgfusepath{stroke}
\pgfpathmoveto{\pgfpoint{226.655975pt}{84.581039pt}}
\pgflineto{\pgfpoint{217.727982pt}{84.581039pt}}
\pgfusepath{stroke}
\pgfpathmoveto{\pgfpoint{226.655975pt}{90.757896pt}}
\pgflineto{\pgfpoint{217.727982pt}{90.757896pt}}
\pgfusepath{stroke}
\pgfpathmoveto{\pgfpoint{226.655975pt}{96.934731pt}}
\pgflineto{\pgfpoint{217.727982pt}{96.934731pt}}
\pgfusepath{stroke}
\pgfpathmoveto{\pgfpoint{226.655975pt}{103.111580pt}}
\pgflineto{\pgfpoint{217.727982pt}{103.111580pt}}
\pgfusepath{stroke}
\pgfpathmoveto{\pgfpoint{226.655975pt}{109.288422pt}}
\pgflineto{\pgfpoint{217.727982pt}{109.288422pt}}
\pgfusepath{stroke}
\pgfpathmoveto{\pgfpoint{226.655975pt}{115.465263pt}}
\pgflineto{\pgfpoint{217.727982pt}{115.465263pt}}
\pgfusepath{stroke}
\pgfpathmoveto{\pgfpoint{226.655975pt}{121.642097pt}}
\pgflineto{\pgfpoint{217.727982pt}{121.642097pt}}
\pgfusepath{stroke}
\pgfpathmoveto{\pgfpoint{226.655975pt}{127.818947pt}}
\pgflineto{\pgfpoint{217.727982pt}{127.818947pt}}
\pgfusepath{stroke}
\pgfpathmoveto{\pgfpoint{226.655975pt}{133.995789pt}}
\pgflineto{\pgfpoint{217.727982pt}{133.995789pt}}
\pgfusepath{stroke}
\pgfpathmoveto{\pgfpoint{226.655975pt}{140.172638pt}}
\pgflineto{\pgfpoint{217.727982pt}{140.172638pt}}
\pgfusepath{stroke}
\pgfpathmoveto{\pgfpoint{226.655975pt}{146.349472pt}}
\pgflineto{\pgfpoint{217.727982pt}{146.349472pt}}
\pgfusepath{stroke}
\pgfpathmoveto{\pgfpoint{226.655975pt}{152.526306pt}}
\pgflineto{\pgfpoint{217.727982pt}{152.526306pt}}
\pgfusepath{stroke}
\pgfpathmoveto{\pgfpoint{226.655975pt}{158.703156pt}}
\pgflineto{\pgfpoint{217.727982pt}{158.703156pt}}
\pgfusepath{stroke}
\pgfpathmoveto{\pgfpoint{226.655975pt}{164.880005pt}}
\pgflineto{\pgfpoint{217.746140pt}{164.880005pt}}
\pgfusepath{stroke}
\pgfpathmoveto{\pgfpoint{226.655975pt}{171.056854pt}}
\pgflineto{\pgfpoint{217.746140pt}{171.056854pt}}
\pgfusepath{stroke}
\pgfpathmoveto{\pgfpoint{226.655975pt}{177.233673pt}}
\pgflineto{\pgfpoint{217.746155pt}{177.233673pt}}
\pgfusepath{stroke}
\pgfpathmoveto{\pgfpoint{226.655975pt}{183.410522pt}}
\pgflineto{\pgfpoint{217.746155pt}{183.410522pt}}
\pgfusepath{stroke}
\pgfpathmoveto{\pgfpoint{217.746155pt}{189.587372pt}}
\pgflineto{\pgfpoint{226.637955pt}{189.587372pt}}
\pgfusepath{stroke}
\pgfpathmoveto{\pgfpoint{217.755173pt}{195.764206pt}}
\pgflineto{\pgfpoint{226.628998pt}{195.764206pt}}
\pgfusepath{stroke}
\pgfpathmoveto{\pgfpoint{217.755234pt}{201.941055pt}}
\pgflineto{\pgfpoint{226.620010pt}{201.941055pt}}
\pgfusepath{stroke}
\pgfpathmoveto{\pgfpoint{217.755264pt}{208.117905pt}}
\pgflineto{\pgfpoint{226.619980pt}{208.117905pt}}
\pgfusepath{stroke}
\pgfpathmoveto{\pgfpoint{217.755264pt}{214.294739pt}}
\pgflineto{\pgfpoint{226.610962pt}{214.294739pt}}
\pgfusepath{stroke}
\pgfpathmoveto{\pgfpoint{217.764267pt}{220.471588pt}}
\pgflineto{\pgfpoint{226.602020pt}{220.471588pt}}
\pgfusepath{stroke}
\pgfpathmoveto{\pgfpoint{217.764267pt}{226.648422pt}}
\pgflineto{\pgfpoint{226.592987pt}{226.648422pt}}
\pgfusepath{stroke}
\pgfpathmoveto{\pgfpoint{217.764313pt}{232.825272pt}}
\pgflineto{\pgfpoint{226.592972pt}{232.825272pt}}
\pgfusepath{stroke}
\pgfpathmoveto{\pgfpoint{217.764313pt}{239.002106pt}}
\pgflineto{\pgfpoint{226.583969pt}{239.002106pt}}
\pgfusepath{stroke}
\pgfpathmoveto{\pgfpoint{217.764359pt}{245.178955pt}}
\pgflineto{\pgfpoint{226.574951pt}{245.178955pt}}
\pgfusepath{stroke}
\pgfpathmoveto{\pgfpoint{217.773346pt}{251.355804pt}}
\pgflineto{\pgfpoint{226.561600pt}{251.355804pt}}
\pgfusepath{stroke}
\pgfpathmoveto{\pgfpoint{217.768845pt}{257.532623pt}}
\pgflineto{\pgfpoint{226.557007pt}{257.532623pt}}
\pgfusepath{stroke}
\pgfpathmoveto{\pgfpoint{217.768860pt}{263.709473pt}}
\pgflineto{\pgfpoint{226.548004pt}{263.709473pt}}
\pgfusepath{stroke}
\pgfpathmoveto{\pgfpoint{217.768921pt}{269.886322pt}}
\pgflineto{\pgfpoint{226.543442pt}{269.886322pt}}
\pgfusepath{stroke}
\pgfpathmoveto{\pgfpoint{217.773392pt}{276.063141pt}}
\pgflineto{\pgfpoint{226.534454pt}{276.063141pt}}
\pgfusepath{stroke}
\pgfpathmoveto{\pgfpoint{217.773392pt}{282.239990pt}}
\pgflineto{\pgfpoint{226.525467pt}{282.239990pt}}
\pgfusepath{stroke}
\pgfpathmoveto{\pgfpoint{217.777939pt}{288.416840pt}}
\pgflineto{\pgfpoint{226.520981pt}{288.416840pt}}
\pgfusepath{stroke}
\pgfpathmoveto{\pgfpoint{217.777939pt}{294.593689pt}}
\pgflineto{\pgfpoint{226.511963pt}{294.593689pt}}
\pgfusepath{stroke}
\pgfpathmoveto{\pgfpoint{217.782486pt}{300.770538pt}}
\pgflineto{\pgfpoint{226.507492pt}{300.770538pt}}
\pgfusepath{stroke}
\pgfpathmoveto{\pgfpoint{217.782486pt}{306.947388pt}}
\pgflineto{\pgfpoint{226.498505pt}{306.947388pt}}
\pgfusepath{stroke}
\pgfpathmoveto{\pgfpoint{217.782486pt}{313.124207pt}}
\pgflineto{\pgfpoint{226.489471pt}{313.124207pt}}
\pgfusepath{stroke}
\pgfpathmoveto{\pgfpoint{217.787003pt}{319.301056pt}}
\pgflineto{\pgfpoint{226.484985pt}{319.301056pt}}
\pgfusepath{stroke}
\pgfpathmoveto{\pgfpoint{217.787033pt}{325.477905pt}}
\pgflineto{\pgfpoint{226.475998pt}{325.477905pt}}
\pgfusepath{stroke}
\pgfpathmoveto{\pgfpoint{217.791519pt}{331.654724pt}}
\pgflineto{\pgfpoint{226.471481pt}{331.654724pt}}
\pgfusepath{stroke}
\pgfpathmoveto{\pgfpoint{217.791580pt}{337.831604pt}}
\pgflineto{\pgfpoint{226.462509pt}{337.831604pt}}
\pgfusepath{stroke}
\pgfpathmoveto{\pgfpoint{217.791580pt}{344.008423pt}}
\pgflineto{\pgfpoint{226.457916pt}{344.008423pt}}
\pgfusepath{stroke}
\pgfpathmoveto{\pgfpoint{217.796021pt}{350.185242pt}}
\pgflineto{\pgfpoint{226.448990pt}{350.185242pt}}
\pgfusepath{stroke}
\pgfpathmoveto{\pgfpoint{217.796097pt}{356.362122pt}}
\pgflineto{\pgfpoint{226.440002pt}{356.362122pt}}
\pgfusepath{stroke}
\pgfpathmoveto{\pgfpoint{226.655975pt}{152.526306pt}}
\pgflineto{\pgfpoint{226.655975pt}{146.349472pt}}
\pgfusepath{stroke}
\pgfpathmoveto{\pgfpoint{226.655975pt}{146.349472pt}}
\pgflineto{\pgfpoint{226.655975pt}{140.172638pt}}
\pgfusepath{stroke}
\pgfpathmoveto{\pgfpoint{226.655975pt}{158.703156pt}}
\pgflineto{\pgfpoint{226.655975pt}{152.526306pt}}
\pgfusepath{stroke}
\pgfpathmoveto{\pgfpoint{226.655975pt}{164.880005pt}}
\pgflineto{\pgfpoint{226.655975pt}{158.703156pt}}
\pgfusepath{stroke}
\pgfpathmoveto{\pgfpoint{226.655975pt}{146.349472pt}}
\pgflineto{\pgfpoint{226.674225pt}{146.349472pt}}
\pgfusepath{stroke}
\pgfpathmoveto{\pgfpoint{226.655975pt}{152.526306pt}}
\pgflineto{\pgfpoint{226.674194pt}{152.526306pt}}
\pgfusepath{stroke}
\pgfpathmoveto{\pgfpoint{226.655975pt}{158.703156pt}}
\pgflineto{\pgfpoint{226.674240pt}{158.703156pt}}
\pgfusepath{stroke}
\pgfpathmoveto{\pgfpoint{226.655975pt}{177.233673pt}}
\pgflineto{\pgfpoint{226.655975pt}{171.056854pt}}
\pgfusepath{stroke}
\pgfpathmoveto{\pgfpoint{226.655975pt}{171.056854pt}}
\pgflineto{\pgfpoint{226.655975pt}{164.880005pt}}
\pgfusepath{stroke}
\pgfpathmoveto{\pgfpoint{226.655975pt}{183.410522pt}}
\pgflineto{\pgfpoint{226.655975pt}{177.233673pt}}
\pgfusepath{stroke}
\pgfpathmoveto{\pgfpoint{226.655975pt}{164.880005pt}}
\pgflineto{\pgfpoint{226.674301pt}{164.880005pt}}
\pgfusepath{stroke}
\pgfpathmoveto{\pgfpoint{226.620010pt}{201.941055pt}}
\pgflineto{\pgfpoint{226.637909pt}{201.941055pt}}
\pgfusepath{stroke}
\pgfpathmoveto{\pgfpoint{226.655975pt}{208.117905pt}}
\pgflineto{\pgfpoint{226.619980pt}{208.117905pt}}
\pgfusepath{stroke}
\pgfpathmoveto{\pgfpoint{226.655975pt}{214.294739pt}}
\pgflineto{\pgfpoint{226.610962pt}{214.294739pt}}
\pgfusepath{stroke}
\pgfpathmoveto{\pgfpoint{226.655975pt}{220.471588pt}}
\pgflineto{\pgfpoint{226.602020pt}{220.471588pt}}
\pgfusepath{stroke}
\pgfpathmoveto{\pgfpoint{226.655975pt}{226.648422pt}}
\pgflineto{\pgfpoint{226.592987pt}{226.648422pt}}
\pgfusepath{stroke}
\pgfpathmoveto{\pgfpoint{226.655975pt}{232.825272pt}}
\pgflineto{\pgfpoint{226.592972pt}{232.825272pt}}
\pgfusepath{stroke}
\pgfpathmoveto{\pgfpoint{226.655975pt}{239.002106pt}}
\pgflineto{\pgfpoint{226.583969pt}{239.002106pt}}
\pgfusepath{stroke}
\pgfpathmoveto{\pgfpoint{226.655975pt}{245.178955pt}}
\pgflineto{\pgfpoint{226.574951pt}{245.178955pt}}
\pgfusepath{stroke}
\pgfpathmoveto{\pgfpoint{226.655975pt}{251.355804pt}}
\pgflineto{\pgfpoint{226.561600pt}{251.355804pt}}
\pgfusepath{stroke}
\pgfpathmoveto{\pgfpoint{226.655975pt}{257.532623pt}}
\pgflineto{\pgfpoint{226.557007pt}{257.532623pt}}
\pgfusepath{stroke}
\pgfpathmoveto{\pgfpoint{226.655975pt}{263.709473pt}}
\pgflineto{\pgfpoint{226.548004pt}{263.709473pt}}
\pgfusepath{stroke}
\pgfpathmoveto{\pgfpoint{226.655975pt}{269.886322pt}}
\pgflineto{\pgfpoint{226.543442pt}{269.886322pt}}
\pgfusepath{stroke}
\pgfpathmoveto{\pgfpoint{226.655975pt}{276.063141pt}}
\pgflineto{\pgfpoint{226.534454pt}{276.063141pt}}
\pgfusepath{stroke}
\pgfpathmoveto{\pgfpoint{226.655975pt}{282.239990pt}}
\pgflineto{\pgfpoint{226.525467pt}{282.239990pt}}
\pgfusepath{stroke}
\pgfpathmoveto{\pgfpoint{226.655975pt}{288.416840pt}}
\pgflineto{\pgfpoint{226.520981pt}{288.416840pt}}
\pgfusepath{stroke}
\pgfpathmoveto{\pgfpoint{226.655975pt}{294.593689pt}}
\pgflineto{\pgfpoint{226.511963pt}{294.593689pt}}
\pgfusepath{stroke}
\pgfpathmoveto{\pgfpoint{226.655975pt}{300.770538pt}}
\pgflineto{\pgfpoint{226.507492pt}{300.770538pt}}
\pgfusepath{stroke}
\pgfpathmoveto{\pgfpoint{226.655975pt}{306.947388pt}}
\pgflineto{\pgfpoint{226.498505pt}{306.947388pt}}
\pgfusepath{stroke}
\pgfpathmoveto{\pgfpoint{226.655975pt}{313.124207pt}}
\pgflineto{\pgfpoint{226.489471pt}{313.124207pt}}
\pgfusepath{stroke}
\pgfpathmoveto{\pgfpoint{226.655975pt}{319.301056pt}}
\pgflineto{\pgfpoint{226.484985pt}{319.301056pt}}
\pgfusepath{stroke}
\pgfpathmoveto{\pgfpoint{226.655975pt}{325.477905pt}}
\pgflineto{\pgfpoint{226.475998pt}{325.477905pt}}
\pgfusepath{stroke}
\pgfpathmoveto{\pgfpoint{226.655975pt}{331.654724pt}}
\pgflineto{\pgfpoint{226.471481pt}{331.654724pt}}
\pgfusepath{stroke}
\pgfpathmoveto{\pgfpoint{226.655975pt}{337.831604pt}}
\pgflineto{\pgfpoint{226.462509pt}{337.831604pt}}
\pgfusepath{stroke}
\pgfpathmoveto{\pgfpoint{226.655975pt}{344.008423pt}}
\pgflineto{\pgfpoint{226.457916pt}{344.008423pt}}
\pgfusepath{stroke}
\pgfpathmoveto{\pgfpoint{226.655975pt}{350.185242pt}}
\pgflineto{\pgfpoint{226.448990pt}{350.185242pt}}
\pgfusepath{stroke}
\pgfpathmoveto{\pgfpoint{226.655975pt}{356.362122pt}}
\pgflineto{\pgfpoint{226.440002pt}{356.362122pt}}
\pgfusepath{stroke}
\pgfpathmoveto{\pgfpoint{226.655975pt}{306.947388pt}}
\pgflineto{\pgfpoint{226.655975pt}{300.770538pt}}
\pgfusepath{stroke}
\pgfpathmoveto{\pgfpoint{226.655975pt}{300.770538pt}}
\pgflineto{\pgfpoint{226.655975pt}{294.593689pt}}
\pgfusepath{stroke}
\pgfpathmoveto{\pgfpoint{226.655975pt}{313.124207pt}}
\pgflineto{\pgfpoint{226.655975pt}{306.947388pt}}
\pgfusepath{stroke}
\pgfpathmoveto{\pgfpoint{226.655975pt}{300.770538pt}}
\pgflineto{\pgfpoint{226.669601pt}{300.770538pt}}
\pgfusepath{stroke}
\pgfpathmoveto{\pgfpoint{226.655975pt}{306.947388pt}}
\pgflineto{\pgfpoint{226.669617pt}{306.947388pt}}
\pgfusepath{stroke}
\pgfpathmoveto{\pgfpoint{226.655975pt}{344.008423pt}}
\pgflineto{\pgfpoint{226.655975pt}{337.831604pt}}
\pgfusepath{stroke}
\pgfpathmoveto{\pgfpoint{226.655975pt}{319.301056pt}}
\pgflineto{\pgfpoint{226.655975pt}{313.124207pt}}
\pgfusepath{stroke}
\pgfpathmoveto{\pgfpoint{226.655975pt}{325.477905pt}}
\pgflineto{\pgfpoint{226.655975pt}{319.301056pt}}
\pgfusepath{stroke}
\pgfpathmoveto{\pgfpoint{226.655975pt}{331.654724pt}}
\pgflineto{\pgfpoint{226.655975pt}{325.477905pt}}
\pgfusepath{stroke}
\pgfpathmoveto{\pgfpoint{226.655975pt}{337.831604pt}}
\pgflineto{\pgfpoint{226.655975pt}{331.654724pt}}
\pgfusepath{stroke}
\pgfpathmoveto{\pgfpoint{226.655975pt}{350.185242pt}}
\pgflineto{\pgfpoint{226.655975pt}{344.008423pt}}
\pgfusepath{stroke}
\pgfpathmoveto{\pgfpoint{226.655975pt}{356.362122pt}}
\pgflineto{\pgfpoint{226.655975pt}{350.185242pt}}
\pgfusepath{stroke}
\pgfpathmoveto{\pgfpoint{226.655975pt}{313.124207pt}}
\pgflineto{\pgfpoint{226.669601pt}{313.124207pt}}
\pgfusepath{stroke}
\pgfpathmoveto{\pgfpoint{226.655975pt}{319.301056pt}}
\pgflineto{\pgfpoint{226.674149pt}{319.301056pt}}
\pgfusepath{stroke}
\pgfpathmoveto{\pgfpoint{226.655975pt}{325.477905pt}}
\pgflineto{\pgfpoint{226.674149pt}{325.477905pt}}
\pgfusepath{stroke}
\pgfpathmoveto{\pgfpoint{226.655975pt}{331.654724pt}}
\pgflineto{\pgfpoint{226.678665pt}{331.654724pt}}
\pgfusepath{stroke}
\pgfpathmoveto{\pgfpoint{226.655975pt}{337.831604pt}}
\pgflineto{\pgfpoint{226.678696pt}{337.831604pt}}
\pgfusepath{stroke}
\pgfpathmoveto{\pgfpoint{226.655975pt}{344.008423pt}}
\pgflineto{\pgfpoint{226.683182pt}{344.008423pt}}
\pgfusepath{stroke}
\pgfpathmoveto{\pgfpoint{226.655975pt}{350.185242pt}}
\pgflineto{\pgfpoint{226.683182pt}{350.185242pt}}
\pgfusepath{stroke}
\pgfpathmoveto{\pgfpoint{226.655975pt}{239.002106pt}}
\pgflineto{\pgfpoint{226.655975pt}{232.825272pt}}
\pgfusepath{stroke}
\pgfpathmoveto{\pgfpoint{226.655975pt}{214.294739pt}}
\pgflineto{\pgfpoint{226.655975pt}{208.117905pt}}
\pgfusepath{stroke}
\pgfpathmoveto{\pgfpoint{226.655975pt}{220.471588pt}}
\pgflineto{\pgfpoint{226.655975pt}{214.294739pt}}
\pgfusepath{stroke}
\pgfpathmoveto{\pgfpoint{226.655975pt}{226.648422pt}}
\pgflineto{\pgfpoint{226.655975pt}{220.471588pt}}
\pgfusepath{stroke}
\pgfpathmoveto{\pgfpoint{226.655975pt}{232.825272pt}}
\pgflineto{\pgfpoint{226.655975pt}{226.648422pt}}
\pgfusepath{stroke}
\pgfpathmoveto{\pgfpoint{226.655975pt}{245.178955pt}}
\pgflineto{\pgfpoint{226.655975pt}{239.002106pt}}
\pgfusepath{stroke}
\pgfpathmoveto{\pgfpoint{226.655975pt}{251.355804pt}}
\pgflineto{\pgfpoint{226.655975pt}{245.178955pt}}
\pgfusepath{stroke}
\pgfpathmoveto{\pgfpoint{226.655975pt}{257.532623pt}}
\pgflineto{\pgfpoint{226.655975pt}{251.355804pt}}
\pgfusepath{stroke}
\pgfpathmoveto{\pgfpoint{226.655975pt}{263.709473pt}}
\pgflineto{\pgfpoint{226.655975pt}{257.532623pt}}
\pgfusepath{stroke}
\pgfpathmoveto{\pgfpoint{226.655975pt}{269.886322pt}}
\pgflineto{\pgfpoint{226.655975pt}{263.709473pt}}
\pgfusepath{stroke}
\pgfpathmoveto{\pgfpoint{226.655975pt}{276.063141pt}}
\pgflineto{\pgfpoint{226.655975pt}{269.886322pt}}
\pgfusepath{stroke}
\pgfpathmoveto{\pgfpoint{226.655975pt}{282.239990pt}}
\pgflineto{\pgfpoint{226.655975pt}{276.063141pt}}
\pgfusepath{stroke}
\pgfpathmoveto{\pgfpoint{226.655975pt}{288.416840pt}}
\pgflineto{\pgfpoint{226.655975pt}{282.239990pt}}
\pgfusepath{stroke}
\pgfpathmoveto{\pgfpoint{226.655975pt}{294.593689pt}}
\pgflineto{\pgfpoint{226.655975pt}{288.416840pt}}
\pgfusepath{stroke}
\pgfpathmoveto{\pgfpoint{226.655975pt}{189.587372pt}}
\pgflineto{\pgfpoint{226.637955pt}{189.587372pt}}
\pgfusepath{stroke}
\pgfpathmoveto{\pgfpoint{226.655975pt}{195.764206pt}}
\pgflineto{\pgfpoint{226.628998pt}{195.764206pt}}
\pgfusepath{stroke}
\pgfpathmoveto{\pgfpoint{226.655975pt}{201.941055pt}}
\pgflineto{\pgfpoint{226.637909pt}{201.941055pt}}
\pgfusepath{stroke}
\pgfpathmoveto{\pgfpoint{226.655975pt}{189.587372pt}}
\pgflineto{\pgfpoint{226.655975pt}{183.410522pt}}
\pgfusepath{stroke}
\pgfpathmoveto{\pgfpoint{226.655975pt}{195.764206pt}}
\pgflineto{\pgfpoint{226.655975pt}{189.587372pt}}
\pgfusepath{stroke}
\pgfpathmoveto{\pgfpoint{226.655975pt}{171.056854pt}}
\pgflineto{\pgfpoint{226.674301pt}{171.056854pt}}
\pgfusepath{stroke}
\pgfpathmoveto{\pgfpoint{226.655975pt}{177.233673pt}}
\pgflineto{\pgfpoint{226.683319pt}{177.233673pt}}
\pgfusepath{stroke}
\pgfpathmoveto{\pgfpoint{226.655975pt}{183.410522pt}}
\pgflineto{\pgfpoint{226.683319pt}{183.410522pt}}
\pgfusepath{stroke}
\pgfpathmoveto{\pgfpoint{226.655975pt}{189.587372pt}}
\pgflineto{\pgfpoint{226.683395pt}{189.587372pt}}
\pgfusepath{stroke}
\pgfpathmoveto{\pgfpoint{226.655975pt}{208.117905pt}}
\pgflineto{\pgfpoint{226.655975pt}{201.941055pt}}
\pgfusepath{stroke}
\pgfpathmoveto{\pgfpoint{226.655975pt}{201.941055pt}}
\pgflineto{\pgfpoint{226.655975pt}{195.764206pt}}
\pgfusepath{stroke}
\pgfpathmoveto{\pgfpoint{226.655975pt}{195.764206pt}}
\pgflineto{\pgfpoint{226.683365pt}{195.764206pt}}
\pgfusepath{stroke}
\pgfpathmoveto{\pgfpoint{226.655975pt}{201.941055pt}}
\pgflineto{\pgfpoint{226.683426pt}{201.941055pt}}
\pgfusepath{stroke}
\pgfpathmoveto{\pgfpoint{226.655975pt}{208.117905pt}}
\pgflineto{\pgfpoint{226.692444pt}{208.117905pt}}
\pgfusepath{stroke}
\pgfpathmoveto{\pgfpoint{226.655975pt}{214.294739pt}}
\pgflineto{\pgfpoint{226.692459pt}{214.294739pt}}
\pgfusepath{stroke}
\pgfpathmoveto{\pgfpoint{226.655975pt}{220.471588pt}}
\pgflineto{\pgfpoint{226.692535pt}{220.471588pt}}
\pgfusepath{stroke}
\pgfpathmoveto{\pgfpoint{226.655975pt}{226.648422pt}}
\pgflineto{\pgfpoint{226.692505pt}{226.648422pt}}
\pgfusepath{stroke}
\pgfpathmoveto{\pgfpoint{226.655975pt}{232.825272pt}}
\pgflineto{\pgfpoint{226.701508pt}{232.825272pt}}
\pgfusepath{stroke}
\pgfpathmoveto{\pgfpoint{226.655975pt}{239.002106pt}}
\pgflineto{\pgfpoint{226.701553pt}{239.002106pt}}
\pgfusepath{stroke}
\pgfpathmoveto{\pgfpoint{226.655975pt}{245.178955pt}}
\pgflineto{\pgfpoint{226.701553pt}{245.178955pt}}
\pgfusepath{stroke}
\pgfpathmoveto{\pgfpoint{226.655975pt}{251.355804pt}}
\pgflineto{\pgfpoint{226.697113pt}{251.355804pt}}
\pgfusepath{stroke}
\pgfpathmoveto{\pgfpoint{226.655975pt}{257.532623pt}}
\pgflineto{\pgfpoint{226.701614pt}{257.532623pt}}
\pgfusepath{stroke}
\pgfpathmoveto{\pgfpoint{226.655975pt}{263.709473pt}}
\pgflineto{\pgfpoint{226.701645pt}{263.709473pt}}
\pgfusepath{stroke}
\pgfpathmoveto{\pgfpoint{226.655975pt}{269.886322pt}}
\pgflineto{\pgfpoint{226.706192pt}{269.886322pt}}
\pgfusepath{stroke}
\pgfpathmoveto{\pgfpoint{226.655975pt}{276.063141pt}}
\pgflineto{\pgfpoint{226.706161pt}{276.063141pt}}
\pgfusepath{stroke}
\pgfpathmoveto{\pgfpoint{226.655975pt}{282.239990pt}}
\pgflineto{\pgfpoint{226.706223pt}{282.239990pt}}
\pgfusepath{stroke}
\pgfpathmoveto{\pgfpoint{226.655975pt}{288.416840pt}}
\pgflineto{\pgfpoint{226.710739pt}{288.416840pt}}
\pgfusepath{stroke}
\pgfpathmoveto{\pgfpoint{226.655975pt}{294.593689pt}}
\pgflineto{\pgfpoint{226.710770pt}{294.593689pt}}
\pgfusepath{stroke}
\pgfpathmoveto{\pgfpoint{226.715286pt}{300.770538pt}}
\pgflineto{\pgfpoint{226.669601pt}{300.770538pt}}
\pgfusepath{stroke}
\pgfpathmoveto{\pgfpoint{226.715347pt}{306.947388pt}}
\pgflineto{\pgfpoint{226.669617pt}{306.947388pt}}
\pgfusepath{stroke}
\pgfpathmoveto{\pgfpoint{226.715378pt}{313.124207pt}}
\pgflineto{\pgfpoint{226.669601pt}{313.124207pt}}
\pgfusepath{stroke}
\pgfpathmoveto{\pgfpoint{226.719864pt}{319.301056pt}}
\pgflineto{\pgfpoint{226.674149pt}{319.301056pt}}
\pgfusepath{stroke}
\pgfpathmoveto{\pgfpoint{226.719894pt}{325.477905pt}}
\pgflineto{\pgfpoint{226.674149pt}{325.477905pt}}
\pgfusepath{stroke}
\pgfpathmoveto{\pgfpoint{226.724380pt}{331.654724pt}}
\pgflineto{\pgfpoint{226.678665pt}{331.654724pt}}
\pgfusepath{stroke}
\pgfpathmoveto{\pgfpoint{226.724472pt}{337.831604pt}}
\pgflineto{\pgfpoint{226.678696pt}{337.831604pt}}
\pgfusepath{stroke}
\pgfpathmoveto{\pgfpoint{226.728989pt}{344.008423pt}}
\pgflineto{\pgfpoint{226.683182pt}{344.008423pt}}
\pgfusepath{stroke}
\pgfpathmoveto{\pgfpoint{226.728958pt}{350.185242pt}}
\pgflineto{\pgfpoint{226.683182pt}{350.185242pt}}
\pgfusepath{stroke}
\pgfpathmoveto{\pgfpoint{226.655975pt}{96.934731pt}}
\pgflineto{\pgfpoint{226.655975pt}{90.757896pt}}
\pgfusepath{stroke}
\pgfpathmoveto{\pgfpoint{226.655975pt}{66.050522pt}}
\pgflineto{\pgfpoint{226.655975pt}{59.873672pt}}
\pgfusepath{stroke}
\pgfpathmoveto{\pgfpoint{226.655975pt}{72.227356pt}}
\pgflineto{\pgfpoint{226.655975pt}{66.050522pt}}
\pgfusepath{stroke}
\pgfpathmoveto{\pgfpoint{226.655975pt}{78.404205pt}}
\pgflineto{\pgfpoint{226.655975pt}{72.227356pt}}
\pgfusepath{stroke}
\pgfpathmoveto{\pgfpoint{226.655975pt}{84.581039pt}}
\pgflineto{\pgfpoint{226.655975pt}{78.404205pt}}
\pgfusepath{stroke}
\pgfpathmoveto{\pgfpoint{226.655975pt}{90.757896pt}}
\pgflineto{\pgfpoint{226.655975pt}{84.581039pt}}
\pgfusepath{stroke}
\pgfpathmoveto{\pgfpoint{226.655975pt}{103.111580pt}}
\pgflineto{\pgfpoint{226.655975pt}{96.934731pt}}
\pgfusepath{stroke}
\pgfpathmoveto{\pgfpoint{226.655975pt}{109.288422pt}}
\pgflineto{\pgfpoint{226.655975pt}{103.111580pt}}
\pgfusepath{stroke}
\pgfpathmoveto{\pgfpoint{226.655975pt}{115.465263pt}}
\pgflineto{\pgfpoint{226.655975pt}{109.288422pt}}
\pgfusepath{stroke}
\pgfpathmoveto{\pgfpoint{226.655975pt}{121.642097pt}}
\pgflineto{\pgfpoint{226.655975pt}{115.465263pt}}
\pgfusepath{stroke}
\pgfpathmoveto{\pgfpoint{226.655975pt}{127.818947pt}}
\pgflineto{\pgfpoint{226.655975pt}{121.642097pt}}
\pgfusepath{stroke}
\pgfpathmoveto{\pgfpoint{226.655975pt}{133.995789pt}}
\pgflineto{\pgfpoint{226.655975pt}{127.818947pt}}
\pgfusepath{stroke}
\pgfpathmoveto{\pgfpoint{226.655975pt}{140.172638pt}}
\pgflineto{\pgfpoint{226.655975pt}{133.995789pt}}
\pgfusepath{stroke}
\pgfpathmoveto{\pgfpoint{226.655975pt}{47.519989pt}}
\pgflineto{\pgfpoint{226.637787pt}{47.519989pt}}
\pgfusepath{stroke}
\pgfpathmoveto{\pgfpoint{226.655975pt}{53.696838pt}}
\pgflineto{\pgfpoint{226.637817pt}{53.696838pt}}
\pgfusepath{stroke}
\pgfpathmoveto{\pgfpoint{226.655975pt}{59.873672pt}}
\pgflineto{\pgfpoint{226.655975pt}{53.696838pt}}
\pgfusepath{stroke}
\pgfpathmoveto{\pgfpoint{226.655975pt}{53.696838pt}}
\pgflineto{\pgfpoint{226.655975pt}{47.519989pt}}
\pgfusepath{stroke}
\pgfpathmoveto{\pgfpoint{226.655975pt}{47.519989pt}}
\pgflineto{\pgfpoint{235.566025pt}{47.519989pt}}
\pgfusepath{stroke}
\pgfpathmoveto{\pgfpoint{226.655975pt}{53.696838pt}}
\pgflineto{\pgfpoint{235.566025pt}{53.696838pt}}
\pgfusepath{stroke}
\pgfpathmoveto{\pgfpoint{226.655975pt}{59.873672pt}}
\pgflineto{\pgfpoint{235.565994pt}{59.873672pt}}
\pgfusepath{stroke}
\pgfpathmoveto{\pgfpoint{226.655975pt}{66.050522pt}}
\pgflineto{\pgfpoint{235.565994pt}{66.050522pt}}
\pgfusepath{stroke}
\pgfpathmoveto{\pgfpoint{235.583969pt}{72.227356pt}}
\pgflineto{\pgfpoint{226.655975pt}{72.227356pt}}
\pgfusepath{stroke}
\pgfpathmoveto{\pgfpoint{235.583969pt}{78.404205pt}}
\pgflineto{\pgfpoint{226.655975pt}{78.404205pt}}
\pgfusepath{stroke}
\pgfpathmoveto{\pgfpoint{235.583969pt}{84.581039pt}}
\pgflineto{\pgfpoint{226.655975pt}{84.581039pt}}
\pgfusepath{stroke}
\pgfpathmoveto{\pgfpoint{235.583969pt}{90.757896pt}}
\pgflineto{\pgfpoint{226.655975pt}{90.757896pt}}
\pgfusepath{stroke}
\pgfpathmoveto{\pgfpoint{235.583969pt}{96.934731pt}}
\pgflineto{\pgfpoint{226.655975pt}{96.934731pt}}
\pgfusepath{stroke}
\pgfpathmoveto{\pgfpoint{235.583969pt}{103.111580pt}}
\pgflineto{\pgfpoint{226.655975pt}{103.111580pt}}
\pgfusepath{stroke}
\pgfpathmoveto{\pgfpoint{235.583969pt}{109.288422pt}}
\pgflineto{\pgfpoint{226.655975pt}{109.288422pt}}
\pgfusepath{stroke}
\pgfpathmoveto{\pgfpoint{235.583969pt}{115.465263pt}}
\pgflineto{\pgfpoint{226.655975pt}{115.465263pt}}
\pgfusepath{stroke}
\pgfpathmoveto{\pgfpoint{235.583969pt}{121.642097pt}}
\pgflineto{\pgfpoint{226.655975pt}{121.642097pt}}
\pgfusepath{stroke}
\pgfpathmoveto{\pgfpoint{235.583969pt}{127.818947pt}}
\pgflineto{\pgfpoint{226.655975pt}{127.818947pt}}
\pgfusepath{stroke}
\pgfpathmoveto{\pgfpoint{235.583969pt}{133.995789pt}}
\pgflineto{\pgfpoint{226.655975pt}{133.995789pt}}
\pgfusepath{stroke}
\pgfpathmoveto{\pgfpoint{235.583969pt}{140.172638pt}}
\pgflineto{\pgfpoint{226.655975pt}{140.172638pt}}
\pgfusepath{stroke}
\pgfpathmoveto{\pgfpoint{235.583969pt}{146.349472pt}}
\pgflineto{\pgfpoint{226.674225pt}{146.349472pt}}
\pgfusepath{stroke}
\pgfpathmoveto{\pgfpoint{235.583969pt}{152.526306pt}}
\pgflineto{\pgfpoint{226.674194pt}{152.526306pt}}
\pgfusepath{stroke}
\pgfpathmoveto{\pgfpoint{235.583969pt}{158.703156pt}}
\pgflineto{\pgfpoint{226.674240pt}{158.703156pt}}
\pgfusepath{stroke}
\pgfpathmoveto{\pgfpoint{235.583969pt}{164.880005pt}}
\pgflineto{\pgfpoint{226.674301pt}{164.880005pt}}
\pgfusepath{stroke}
\pgfpathmoveto{\pgfpoint{235.583969pt}{171.056854pt}}
\pgflineto{\pgfpoint{226.674301pt}{171.056854pt}}
\pgfusepath{stroke}
\pgfpathmoveto{\pgfpoint{235.583969pt}{177.233673pt}}
\pgflineto{\pgfpoint{226.683319pt}{177.233673pt}}
\pgfusepath{stroke}
\pgfpathmoveto{\pgfpoint{235.583969pt}{183.410522pt}}
\pgflineto{\pgfpoint{226.683319pt}{183.410522pt}}
\pgfusepath{stroke}
\pgfpathmoveto{\pgfpoint{235.583969pt}{189.587372pt}}
\pgflineto{\pgfpoint{226.683395pt}{189.587372pt}}
\pgfusepath{stroke}
\pgfpathmoveto{\pgfpoint{235.583969pt}{195.764206pt}}
\pgflineto{\pgfpoint{226.683365pt}{195.764206pt}}
\pgfusepath{stroke}
\pgfpathmoveto{\pgfpoint{226.683426pt}{201.941055pt}}
\pgflineto{\pgfpoint{235.565887pt}{201.941055pt}}
\pgfusepath{stroke}
\pgfpathmoveto{\pgfpoint{226.692444pt}{208.117905pt}}
\pgflineto{\pgfpoint{235.556885pt}{208.117905pt}}
\pgfusepath{stroke}
\pgfpathmoveto{\pgfpoint{226.692459pt}{214.294739pt}}
\pgflineto{\pgfpoint{235.547836pt}{214.294739pt}}
\pgfusepath{stroke}
\pgfpathmoveto{\pgfpoint{226.692535pt}{220.471588pt}}
\pgflineto{\pgfpoint{235.547775pt}{220.471588pt}}
\pgfusepath{stroke}
\pgfpathmoveto{\pgfpoint{226.692505pt}{226.648422pt}}
\pgflineto{\pgfpoint{235.538742pt}{226.648422pt}}
\pgfusepath{stroke}
\pgfpathmoveto{\pgfpoint{226.701508pt}{232.825272pt}}
\pgflineto{\pgfpoint{235.529739pt}{232.825272pt}}
\pgfusepath{stroke}
\pgfpathmoveto{\pgfpoint{226.701553pt}{239.002106pt}}
\pgflineto{\pgfpoint{235.520706pt}{239.002106pt}}
\pgfusepath{stroke}
\pgfpathmoveto{\pgfpoint{226.701553pt}{245.178955pt}}
\pgflineto{\pgfpoint{235.511673pt}{245.178955pt}}
\pgfusepath{stroke}
\pgfpathmoveto{\pgfpoint{226.697113pt}{251.355804pt}}
\pgflineto{\pgfpoint{235.502594pt}{251.355804pt}}
\pgfusepath{stroke}
\pgfpathmoveto{\pgfpoint{226.701614pt}{257.532623pt}}
\pgflineto{\pgfpoint{235.498047pt}{257.532623pt}}
\pgfusepath{stroke}
\pgfpathmoveto{\pgfpoint{226.701645pt}{263.709473pt}}
\pgflineto{\pgfpoint{235.489059pt}{263.709473pt}}
\pgfusepath{stroke}
\pgfpathmoveto{\pgfpoint{226.706192pt}{269.886322pt}}
\pgflineto{\pgfpoint{235.484528pt}{269.886322pt}}
\pgfusepath{stroke}
\pgfpathmoveto{\pgfpoint{226.706161pt}{276.063141pt}}
\pgflineto{\pgfpoint{235.475464pt}{276.063141pt}}
\pgfusepath{stroke}
\pgfpathmoveto{\pgfpoint{226.706223pt}{282.239990pt}}
\pgflineto{\pgfpoint{235.466431pt}{282.239990pt}}
\pgfusepath{stroke}
\pgfpathmoveto{\pgfpoint{226.710739pt}{288.416840pt}}
\pgflineto{\pgfpoint{235.461884pt}{288.416840pt}}
\pgfusepath{stroke}
\pgfpathmoveto{\pgfpoint{226.710770pt}{294.593689pt}}
\pgflineto{\pgfpoint{235.452850pt}{294.593689pt}}
\pgfusepath{stroke}
\pgfpathmoveto{\pgfpoint{226.715286pt}{300.770538pt}}
\pgflineto{\pgfpoint{235.448364pt}{300.770538pt}}
\pgfusepath{stroke}
\pgfpathmoveto{\pgfpoint{226.715347pt}{306.947388pt}}
\pgflineto{\pgfpoint{235.439331pt}{306.947388pt}}
\pgfusepath{stroke}
\pgfpathmoveto{\pgfpoint{226.715378pt}{313.124207pt}}
\pgflineto{\pgfpoint{235.434692pt}{313.124207pt}}
\pgfusepath{stroke}
\pgfpathmoveto{\pgfpoint{226.719864pt}{319.301056pt}}
\pgflineto{\pgfpoint{235.425751pt}{319.301056pt}}
\pgfusepath{stroke}
\pgfpathmoveto{\pgfpoint{226.719894pt}{325.477905pt}}
\pgflineto{\pgfpoint{235.416702pt}{325.477905pt}}
\pgfusepath{stroke}
\pgfpathmoveto{\pgfpoint{226.724380pt}{331.654724pt}}
\pgflineto{\pgfpoint{235.412155pt}{331.654724pt}}
\pgfusepath{stroke}
\pgfpathmoveto{\pgfpoint{226.724472pt}{337.831604pt}}
\pgflineto{\pgfpoint{235.403152pt}{337.831604pt}}
\pgfusepath{stroke}
\pgfpathmoveto{\pgfpoint{226.728989pt}{344.008423pt}}
\pgflineto{\pgfpoint{235.398605pt}{344.008423pt}}
\pgfusepath{stroke}
\pgfpathmoveto{\pgfpoint{226.728958pt}{350.185242pt}}
\pgflineto{\pgfpoint{235.389542pt}{350.185242pt}}
\pgfusepath{stroke}
\pgfpathmoveto{\pgfpoint{235.583969pt}{164.880005pt}}
\pgflineto{\pgfpoint{235.583969pt}{158.703156pt}}
\pgfusepath{stroke}
\pgfpathmoveto{\pgfpoint{235.583969pt}{158.703156pt}}
\pgflineto{\pgfpoint{235.583969pt}{152.526306pt}}
\pgfusepath{stroke}
\pgfpathmoveto{\pgfpoint{235.583969pt}{171.056854pt}}
\pgflineto{\pgfpoint{235.583969pt}{164.880005pt}}
\pgfusepath{stroke}
\pgfpathmoveto{\pgfpoint{235.583969pt}{177.233673pt}}
\pgflineto{\pgfpoint{235.583969pt}{171.056854pt}}
\pgfusepath{stroke}
\pgfpathmoveto{\pgfpoint{235.583969pt}{158.703156pt}}
\pgflineto{\pgfpoint{235.601898pt}{158.703156pt}}
\pgfusepath{stroke}
\pgfpathmoveto{\pgfpoint{235.583969pt}{164.880005pt}}
\pgflineto{\pgfpoint{235.601868pt}{164.880005pt}}
\pgfusepath{stroke}
\pgfpathmoveto{\pgfpoint{235.583969pt}{171.056854pt}}
\pgflineto{\pgfpoint{235.601913pt}{171.056854pt}}
\pgfusepath{stroke}
\pgfpathmoveto{\pgfpoint{235.583969pt}{189.587372pt}}
\pgflineto{\pgfpoint{235.583969pt}{183.410522pt}}
\pgfusepath{stroke}
\pgfpathmoveto{\pgfpoint{235.583969pt}{183.410522pt}}
\pgflineto{\pgfpoint{235.583969pt}{177.233673pt}}
\pgfusepath{stroke}
\pgfpathmoveto{\pgfpoint{235.583969pt}{195.764206pt}}
\pgflineto{\pgfpoint{235.583969pt}{189.587372pt}}
\pgfusepath{stroke}
\pgfpathmoveto{\pgfpoint{235.583969pt}{177.233673pt}}
\pgflineto{\pgfpoint{235.601852pt}{177.233673pt}}
\pgfusepath{stroke}
\pgfpathmoveto{\pgfpoint{235.583969pt}{208.117905pt}}
\pgflineto{\pgfpoint{235.556885pt}{208.117905pt}}
\pgfusepath{stroke}
\pgfpathmoveto{\pgfpoint{235.583969pt}{214.294739pt}}
\pgflineto{\pgfpoint{235.547836pt}{214.294739pt}}
\pgfusepath{stroke}
\pgfpathmoveto{\pgfpoint{235.583969pt}{220.471588pt}}
\pgflineto{\pgfpoint{235.547775pt}{220.471588pt}}
\pgfusepath{stroke}
\pgfpathmoveto{\pgfpoint{235.583969pt}{226.648422pt}}
\pgflineto{\pgfpoint{235.538742pt}{226.648422pt}}
\pgfusepath{stroke}
\pgfpathmoveto{\pgfpoint{235.583969pt}{232.825272pt}}
\pgflineto{\pgfpoint{235.529739pt}{232.825272pt}}
\pgfusepath{stroke}
\pgfpathmoveto{\pgfpoint{235.583969pt}{239.002106pt}}
\pgflineto{\pgfpoint{235.520706pt}{239.002106pt}}
\pgfusepath{stroke}
\pgfpathmoveto{\pgfpoint{235.583969pt}{245.178955pt}}
\pgflineto{\pgfpoint{235.511673pt}{245.178955pt}}
\pgfusepath{stroke}
\pgfpathmoveto{\pgfpoint{235.583969pt}{251.355804pt}}
\pgflineto{\pgfpoint{235.502594pt}{251.355804pt}}
\pgfusepath{stroke}
\pgfpathmoveto{\pgfpoint{235.583969pt}{257.532623pt}}
\pgflineto{\pgfpoint{235.498047pt}{257.532623pt}}
\pgfusepath{stroke}
\pgfpathmoveto{\pgfpoint{235.583969pt}{263.709473pt}}
\pgflineto{\pgfpoint{235.489059pt}{263.709473pt}}
\pgfusepath{stroke}
\pgfpathmoveto{\pgfpoint{235.583969pt}{269.886322pt}}
\pgflineto{\pgfpoint{235.484528pt}{269.886322pt}}
\pgfusepath{stroke}
\pgfpathmoveto{\pgfpoint{235.583969pt}{276.063141pt}}
\pgflineto{\pgfpoint{235.475464pt}{276.063141pt}}
\pgfusepath{stroke}
\pgfpathmoveto{\pgfpoint{235.583969pt}{282.239990pt}}
\pgflineto{\pgfpoint{235.466431pt}{282.239990pt}}
\pgfusepath{stroke}
\pgfpathmoveto{\pgfpoint{235.583969pt}{288.416840pt}}
\pgflineto{\pgfpoint{235.461884pt}{288.416840pt}}
\pgfusepath{stroke}
\pgfpathmoveto{\pgfpoint{235.583969pt}{294.593689pt}}
\pgflineto{\pgfpoint{235.452850pt}{294.593689pt}}
\pgfusepath{stroke}
\pgfpathmoveto{\pgfpoint{235.583969pt}{300.770538pt}}
\pgflineto{\pgfpoint{235.448364pt}{300.770538pt}}
\pgfusepath{stroke}
\pgfpathmoveto{\pgfpoint{235.583969pt}{306.947388pt}}
\pgflineto{\pgfpoint{235.439331pt}{306.947388pt}}
\pgfusepath{stroke}
\pgfpathmoveto{\pgfpoint{235.583969pt}{313.124207pt}}
\pgflineto{\pgfpoint{235.434692pt}{313.124207pt}}
\pgfusepath{stroke}
\pgfpathmoveto{\pgfpoint{235.583969pt}{319.301056pt}}
\pgflineto{\pgfpoint{235.425751pt}{319.301056pt}}
\pgfusepath{stroke}
\pgfpathmoveto{\pgfpoint{235.583969pt}{325.477905pt}}
\pgflineto{\pgfpoint{235.416702pt}{325.477905pt}}
\pgfusepath{stroke}
\pgfpathmoveto{\pgfpoint{235.583969pt}{331.654724pt}}
\pgflineto{\pgfpoint{235.412155pt}{331.654724pt}}
\pgfusepath{stroke}
\pgfpathmoveto{\pgfpoint{235.583969pt}{337.831604pt}}
\pgflineto{\pgfpoint{235.403152pt}{337.831604pt}}
\pgfusepath{stroke}
\pgfpathmoveto{\pgfpoint{235.583969pt}{344.008423pt}}
\pgflineto{\pgfpoint{235.398605pt}{344.008423pt}}
\pgfusepath{stroke}
\pgfpathmoveto{\pgfpoint{235.583969pt}{350.185242pt}}
\pgflineto{\pgfpoint{235.389542pt}{350.185242pt}}
\pgfusepath{stroke}
\pgfpathmoveto{\pgfpoint{235.583969pt}{294.593689pt}}
\pgflineto{\pgfpoint{235.583969pt}{288.416840pt}}
\pgfusepath{stroke}
\pgfpathmoveto{\pgfpoint{235.583969pt}{288.416840pt}}
\pgflineto{\pgfpoint{235.583969pt}{282.239990pt}}
\pgfusepath{stroke}
\pgfpathmoveto{\pgfpoint{235.583969pt}{300.770538pt}}
\pgflineto{\pgfpoint{235.583969pt}{294.593689pt}}
\pgfusepath{stroke}
\pgfpathmoveto{\pgfpoint{235.583969pt}{306.947388pt}}
\pgflineto{\pgfpoint{235.583969pt}{300.770538pt}}
\pgfusepath{stroke}
\pgfpathmoveto{\pgfpoint{235.583969pt}{288.416840pt}}
\pgflineto{\pgfpoint{235.597702pt}{288.416840pt}}
\pgfusepath{stroke}
\pgfpathmoveto{\pgfpoint{235.583969pt}{294.593689pt}}
\pgflineto{\pgfpoint{235.597702pt}{294.593689pt}}
\pgfusepath{stroke}
\pgfpathmoveto{\pgfpoint{235.583969pt}{300.770538pt}}
\pgflineto{\pgfpoint{235.602249pt}{300.770538pt}}
\pgfusepath{stroke}
\pgfpathmoveto{\pgfpoint{235.583969pt}{331.654724pt}}
\pgflineto{\pgfpoint{235.583969pt}{325.477905pt}}
\pgfusepath{stroke}
\pgfpathmoveto{\pgfpoint{235.583969pt}{313.124207pt}}
\pgflineto{\pgfpoint{235.583969pt}{306.947388pt}}
\pgfusepath{stroke}
\pgfpathmoveto{\pgfpoint{235.583969pt}{319.301056pt}}
\pgflineto{\pgfpoint{235.583969pt}{313.124207pt}}
\pgfusepath{stroke}
\pgfpathmoveto{\pgfpoint{235.583969pt}{325.477905pt}}
\pgflineto{\pgfpoint{235.583969pt}{319.301056pt}}
\pgfusepath{stroke}
\pgfpathmoveto{\pgfpoint{235.583969pt}{337.831604pt}}
\pgflineto{\pgfpoint{235.583969pt}{331.654724pt}}
\pgfusepath{stroke}
\pgfpathmoveto{\pgfpoint{235.583969pt}{344.008423pt}}
\pgflineto{\pgfpoint{235.583969pt}{337.831604pt}}
\pgfusepath{stroke}
\pgfpathmoveto{\pgfpoint{235.583969pt}{350.185242pt}}
\pgflineto{\pgfpoint{235.583969pt}{344.008423pt}}
\pgfusepath{stroke}
\pgfpathmoveto{\pgfpoint{235.583969pt}{306.947388pt}}
\pgflineto{\pgfpoint{235.602310pt}{306.947388pt}}
\pgfusepath{stroke}
\pgfpathmoveto{\pgfpoint{235.583969pt}{313.124207pt}}
\pgflineto{\pgfpoint{235.606796pt}{313.124207pt}}
\pgfusepath{stroke}
\pgfpathmoveto{\pgfpoint{235.583969pt}{319.301056pt}}
\pgflineto{\pgfpoint{235.606827pt}{319.301056pt}}
\pgfusepath{stroke}
\pgfpathmoveto{\pgfpoint{235.583969pt}{325.477905pt}}
\pgflineto{\pgfpoint{235.606842pt}{325.477905pt}}
\pgfusepath{stroke}
\pgfpathmoveto{\pgfpoint{235.583969pt}{331.654724pt}}
\pgflineto{\pgfpoint{235.611359pt}{331.654724pt}}
\pgfusepath{stroke}
\pgfpathmoveto{\pgfpoint{235.583969pt}{337.831604pt}}
\pgflineto{\pgfpoint{235.611435pt}{337.831604pt}}
\pgfusepath{stroke}
\pgfpathmoveto{\pgfpoint{235.583969pt}{344.008423pt}}
\pgflineto{\pgfpoint{235.615921pt}{344.008423pt}}
\pgfusepath{stroke}
\pgfpathmoveto{\pgfpoint{235.583969pt}{226.648422pt}}
\pgflineto{\pgfpoint{235.583969pt}{220.471588pt}}
\pgfusepath{stroke}
\pgfpathmoveto{\pgfpoint{235.583969pt}{201.941055pt}}
\pgflineto{\pgfpoint{235.583969pt}{195.764206pt}}
\pgfusepath{stroke}
\pgfpathmoveto{\pgfpoint{235.583969pt}{208.117905pt}}
\pgflineto{\pgfpoint{235.583969pt}{201.941055pt}}
\pgfusepath{stroke}
\pgfpathmoveto{\pgfpoint{235.583969pt}{214.294739pt}}
\pgflineto{\pgfpoint{235.583969pt}{208.117905pt}}
\pgfusepath{stroke}
\pgfpathmoveto{\pgfpoint{235.583969pt}{220.471588pt}}
\pgflineto{\pgfpoint{235.583969pt}{214.294739pt}}
\pgfusepath{stroke}
\pgfpathmoveto{\pgfpoint{235.583969pt}{232.825272pt}}
\pgflineto{\pgfpoint{235.583969pt}{226.648422pt}}
\pgfusepath{stroke}
\pgfpathmoveto{\pgfpoint{235.583969pt}{239.002106pt}}
\pgflineto{\pgfpoint{235.583969pt}{232.825272pt}}
\pgfusepath{stroke}
\pgfpathmoveto{\pgfpoint{235.583969pt}{245.178955pt}}
\pgflineto{\pgfpoint{235.583969pt}{239.002106pt}}
\pgfusepath{stroke}
\pgfpathmoveto{\pgfpoint{235.583969pt}{251.355804pt}}
\pgflineto{\pgfpoint{235.583969pt}{245.178955pt}}
\pgfusepath{stroke}
\pgfpathmoveto{\pgfpoint{235.583969pt}{257.532623pt}}
\pgflineto{\pgfpoint{235.583969pt}{251.355804pt}}
\pgfusepath{stroke}
\pgfpathmoveto{\pgfpoint{235.583969pt}{263.709473pt}}
\pgflineto{\pgfpoint{235.583969pt}{257.532623pt}}
\pgfusepath{stroke}
\pgfpathmoveto{\pgfpoint{235.583969pt}{269.886322pt}}
\pgflineto{\pgfpoint{235.583969pt}{263.709473pt}}
\pgfusepath{stroke}
\pgfpathmoveto{\pgfpoint{235.583969pt}{276.063141pt}}
\pgflineto{\pgfpoint{235.583969pt}{269.886322pt}}
\pgfusepath{stroke}
\pgfpathmoveto{\pgfpoint{235.583969pt}{282.239990pt}}
\pgflineto{\pgfpoint{235.583969pt}{276.063141pt}}
\pgfusepath{stroke}
\pgfpathmoveto{\pgfpoint{235.583969pt}{201.941055pt}}
\pgflineto{\pgfpoint{235.565887pt}{201.941055pt}}
\pgfusepath{stroke}
\pgfpathmoveto{\pgfpoint{235.583969pt}{183.410522pt}}
\pgflineto{\pgfpoint{235.601822pt}{183.410522pt}}
\pgfusepath{stroke}
\pgfpathmoveto{\pgfpoint{235.583969pt}{189.587372pt}}
\pgflineto{\pgfpoint{235.610855pt}{189.587372pt}}
\pgfusepath{stroke}
\pgfpathmoveto{\pgfpoint{235.583969pt}{195.764206pt}}
\pgflineto{\pgfpoint{235.610855pt}{195.764206pt}}
\pgfusepath{stroke}
\pgfpathmoveto{\pgfpoint{235.583969pt}{201.941055pt}}
\pgflineto{\pgfpoint{235.610870pt}{201.941055pt}}
\pgfusepath{stroke}
\pgfpathmoveto{\pgfpoint{235.583969pt}{208.117905pt}}
\pgflineto{\pgfpoint{235.610840pt}{208.117905pt}}
\pgfusepath{stroke}
\pgfpathmoveto{\pgfpoint{235.583969pt}{214.294739pt}}
\pgflineto{\pgfpoint{235.610825pt}{214.294739pt}}
\pgfusepath{stroke}
\pgfpathmoveto{\pgfpoint{235.583969pt}{220.471588pt}}
\pgflineto{\pgfpoint{235.619827pt}{220.471588pt}}
\pgfusepath{stroke}
\pgfpathmoveto{\pgfpoint{235.583969pt}{226.648422pt}}
\pgflineto{\pgfpoint{235.619797pt}{226.648422pt}}
\pgfusepath{stroke}
\pgfpathmoveto{\pgfpoint{235.583969pt}{232.825272pt}}
\pgflineto{\pgfpoint{235.619766pt}{232.825272pt}}
\pgfusepath{stroke}
\pgfpathmoveto{\pgfpoint{235.583969pt}{239.002106pt}}
\pgflineto{\pgfpoint{235.619781pt}{239.002106pt}}
\pgfusepath{stroke}
\pgfpathmoveto{\pgfpoint{235.583969pt}{245.178955pt}}
\pgflineto{\pgfpoint{235.619781pt}{245.178955pt}}
\pgfusepath{stroke}
\pgfpathmoveto{\pgfpoint{235.583969pt}{251.355804pt}}
\pgflineto{\pgfpoint{235.619751pt}{251.355804pt}}
\pgfusepath{stroke}
\pgfpathmoveto{\pgfpoint{235.583969pt}{257.532623pt}}
\pgflineto{\pgfpoint{235.624283pt}{257.532623pt}}
\pgfusepath{stroke}
\pgfpathmoveto{\pgfpoint{235.583969pt}{263.709473pt}}
\pgflineto{\pgfpoint{235.624222pt}{263.709473pt}}
\pgfusepath{stroke}
\pgfpathmoveto{\pgfpoint{235.583969pt}{269.886322pt}}
\pgflineto{\pgfpoint{235.628769pt}{269.886322pt}}
\pgfusepath{stroke}
\pgfpathmoveto{\pgfpoint{235.583969pt}{276.063141pt}}
\pgflineto{\pgfpoint{235.628708pt}{276.063141pt}}
\pgfusepath{stroke}
\pgfpathmoveto{\pgfpoint{235.583969pt}{282.239990pt}}
\pgflineto{\pgfpoint{235.628708pt}{282.239990pt}}
\pgfusepath{stroke}
\pgfpathmoveto{\pgfpoint{235.633209pt}{288.416840pt}}
\pgflineto{\pgfpoint{235.597702pt}{288.416840pt}}
\pgfusepath{stroke}
\pgfpathmoveto{\pgfpoint{235.633209pt}{294.593689pt}}
\pgflineto{\pgfpoint{235.597702pt}{294.593689pt}}
\pgfusepath{stroke}
\pgfpathmoveto{\pgfpoint{235.637726pt}{300.770538pt}}
\pgflineto{\pgfpoint{235.602249pt}{300.770538pt}}
\pgfusepath{stroke}
\pgfpathmoveto{\pgfpoint{235.637726pt}{306.947388pt}}
\pgflineto{\pgfpoint{235.602310pt}{306.947388pt}}
\pgfusepath{stroke}
\pgfpathmoveto{\pgfpoint{235.642181pt}{313.124207pt}}
\pgflineto{\pgfpoint{235.606796pt}{313.124207pt}}
\pgfusepath{stroke}
\pgfpathmoveto{\pgfpoint{235.642151pt}{319.301056pt}}
\pgflineto{\pgfpoint{235.606827pt}{319.301056pt}}
\pgfusepath{stroke}
\pgfpathmoveto{\pgfpoint{235.642212pt}{325.477905pt}}
\pgflineto{\pgfpoint{235.606842pt}{325.477905pt}}
\pgfusepath{stroke}
\pgfpathmoveto{\pgfpoint{235.646652pt}{331.654724pt}}
\pgflineto{\pgfpoint{235.611359pt}{331.654724pt}}
\pgfusepath{stroke}
\pgfpathmoveto{\pgfpoint{235.646683pt}{337.831604pt}}
\pgflineto{\pgfpoint{235.611435pt}{337.831604pt}}
\pgfusepath{stroke}
\pgfpathmoveto{\pgfpoint{235.651138pt}{344.008423pt}}
\pgflineto{\pgfpoint{235.615921pt}{344.008423pt}}
\pgfusepath{stroke}
\pgfpathmoveto{\pgfpoint{235.583969pt}{109.288422pt}}
\pgflineto{\pgfpoint{235.583969pt}{103.111580pt}}
\pgfusepath{stroke}
\pgfpathmoveto{\pgfpoint{235.583969pt}{78.404205pt}}
\pgflineto{\pgfpoint{235.583969pt}{72.227356pt}}
\pgfusepath{stroke}
\pgfpathmoveto{\pgfpoint{235.583969pt}{84.581039pt}}
\pgflineto{\pgfpoint{235.583969pt}{78.404205pt}}
\pgfusepath{stroke}
\pgfpathmoveto{\pgfpoint{235.583969pt}{90.757896pt}}
\pgflineto{\pgfpoint{235.583969pt}{84.581039pt}}
\pgfusepath{stroke}
\pgfpathmoveto{\pgfpoint{235.583969pt}{96.934731pt}}
\pgflineto{\pgfpoint{235.583969pt}{90.757896pt}}
\pgfusepath{stroke}
\pgfpathmoveto{\pgfpoint{235.583969pt}{103.111580pt}}
\pgflineto{\pgfpoint{235.583969pt}{96.934731pt}}
\pgfusepath{stroke}
\pgfpathmoveto{\pgfpoint{235.583969pt}{115.465263pt}}
\pgflineto{\pgfpoint{235.583969pt}{109.288422pt}}
\pgfusepath{stroke}
\pgfpathmoveto{\pgfpoint{235.583969pt}{121.642097pt}}
\pgflineto{\pgfpoint{235.583969pt}{115.465263pt}}
\pgfusepath{stroke}
\pgfpathmoveto{\pgfpoint{235.583969pt}{127.818947pt}}
\pgflineto{\pgfpoint{235.583969pt}{121.642097pt}}
\pgfusepath{stroke}
\pgfpathmoveto{\pgfpoint{235.583969pt}{133.995789pt}}
\pgflineto{\pgfpoint{235.583969pt}{127.818947pt}}
\pgfusepath{stroke}
\pgfpathmoveto{\pgfpoint{235.583969pt}{140.172638pt}}
\pgflineto{\pgfpoint{235.583969pt}{133.995789pt}}
\pgfusepath{stroke}
\pgfpathmoveto{\pgfpoint{235.583969pt}{146.349472pt}}
\pgflineto{\pgfpoint{235.583969pt}{140.172638pt}}
\pgfusepath{stroke}
\pgfpathmoveto{\pgfpoint{235.583969pt}{152.526306pt}}
\pgflineto{\pgfpoint{235.583969pt}{146.349472pt}}
\pgfusepath{stroke}
\pgfpathmoveto{\pgfpoint{235.583969pt}{47.519989pt}}
\pgflineto{\pgfpoint{235.566025pt}{47.519989pt}}
\pgfusepath{stroke}
\pgfpathmoveto{\pgfpoint{235.583969pt}{53.696838pt}}
\pgflineto{\pgfpoint{235.566025pt}{53.696838pt}}
\pgfusepath{stroke}
\pgfpathmoveto{\pgfpoint{235.583969pt}{59.873672pt}}
\pgflineto{\pgfpoint{235.565994pt}{59.873672pt}}
\pgfusepath{stroke}
\pgfpathmoveto{\pgfpoint{235.583969pt}{66.050522pt}}
\pgflineto{\pgfpoint{235.565994pt}{66.050522pt}}
\pgfusepath{stroke}
\pgfpathmoveto{\pgfpoint{235.583969pt}{53.696838pt}}
\pgflineto{\pgfpoint{235.583969pt}{47.519989pt}}
\pgfusepath{stroke}
\pgfpathmoveto{\pgfpoint{235.583969pt}{59.873672pt}}
\pgflineto{\pgfpoint{235.583969pt}{53.696838pt}}
\pgfusepath{stroke}
\pgfpathmoveto{\pgfpoint{235.583969pt}{47.519989pt}}
\pgflineto{\pgfpoint{235.611008pt}{47.519989pt}}
\pgfusepath{stroke}
\pgfpathmoveto{\pgfpoint{235.583969pt}{53.696838pt}}
\pgflineto{\pgfpoint{235.602020pt}{53.696838pt}}
\pgfusepath{stroke}
\pgfpathmoveto{\pgfpoint{235.583969pt}{72.227356pt}}
\pgflineto{\pgfpoint{235.583969pt}{66.050522pt}}
\pgfusepath{stroke}
\pgfpathmoveto{\pgfpoint{235.583969pt}{66.050522pt}}
\pgflineto{\pgfpoint{235.583969pt}{59.873672pt}}
\pgfusepath{stroke}
\pgfpathmoveto{\pgfpoint{235.611008pt}{47.519989pt}}
\pgflineto{\pgfpoint{244.493988pt}{47.519989pt}}
\pgfusepath{stroke}
\pgfpathmoveto{\pgfpoint{244.511993pt}{53.696838pt}}
\pgflineto{\pgfpoint{235.602020pt}{53.696838pt}}
\pgfusepath{stroke}
\pgfpathmoveto{\pgfpoint{244.511993pt}{59.873672pt}}
\pgflineto{\pgfpoint{235.583969pt}{59.873672pt}}
\pgfusepath{stroke}
\pgfpathmoveto{\pgfpoint{244.511993pt}{66.050522pt}}
\pgflineto{\pgfpoint{235.583969pt}{66.050522pt}}
\pgfusepath{stroke}
\pgfpathmoveto{\pgfpoint{244.511993pt}{72.227356pt}}
\pgflineto{\pgfpoint{235.583969pt}{72.227356pt}}
\pgfusepath{stroke}
\pgfpathmoveto{\pgfpoint{244.511993pt}{78.404205pt}}
\pgflineto{\pgfpoint{235.583969pt}{78.404205pt}}
\pgfusepath{stroke}
\pgfpathmoveto{\pgfpoint{244.511993pt}{84.581039pt}}
\pgflineto{\pgfpoint{235.583969pt}{84.581039pt}}
\pgfusepath{stroke}
\pgfpathmoveto{\pgfpoint{244.511993pt}{90.757896pt}}
\pgflineto{\pgfpoint{235.583969pt}{90.757896pt}}
\pgfusepath{stroke}
\pgfpathmoveto{\pgfpoint{244.511993pt}{96.934731pt}}
\pgflineto{\pgfpoint{235.583969pt}{96.934731pt}}
\pgfusepath{stroke}
\pgfpathmoveto{\pgfpoint{244.511993pt}{103.111580pt}}
\pgflineto{\pgfpoint{235.583969pt}{103.111580pt}}
\pgfusepath{stroke}
\pgfpathmoveto{\pgfpoint{244.511993pt}{109.288422pt}}
\pgflineto{\pgfpoint{235.583969pt}{109.288422pt}}
\pgfusepath{stroke}
\pgfpathmoveto{\pgfpoint{244.511993pt}{115.465263pt}}
\pgflineto{\pgfpoint{235.583969pt}{115.465263pt}}
\pgfusepath{stroke}
\pgfpathmoveto{\pgfpoint{244.511993pt}{121.642097pt}}
\pgflineto{\pgfpoint{235.583969pt}{121.642097pt}}
\pgfusepath{stroke}
\pgfpathmoveto{\pgfpoint{244.511993pt}{127.818947pt}}
\pgflineto{\pgfpoint{235.583969pt}{127.818947pt}}
\pgfusepath{stroke}
\pgfpathmoveto{\pgfpoint{244.511993pt}{133.995789pt}}
\pgflineto{\pgfpoint{235.583969pt}{133.995789pt}}
\pgfusepath{stroke}
\pgfpathmoveto{\pgfpoint{244.511993pt}{140.172638pt}}
\pgflineto{\pgfpoint{235.583969pt}{140.172638pt}}
\pgfusepath{stroke}
\pgfpathmoveto{\pgfpoint{244.511993pt}{146.349472pt}}
\pgflineto{\pgfpoint{235.583969pt}{146.349472pt}}
\pgfusepath{stroke}
\pgfpathmoveto{\pgfpoint{244.511993pt}{152.526306pt}}
\pgflineto{\pgfpoint{235.583969pt}{152.526306pt}}
\pgfusepath{stroke}
\pgfpathmoveto{\pgfpoint{244.511993pt}{158.703156pt}}
\pgflineto{\pgfpoint{235.601898pt}{158.703156pt}}
\pgfusepath{stroke}
\pgfpathmoveto{\pgfpoint{244.511993pt}{164.880005pt}}
\pgflineto{\pgfpoint{235.601868pt}{164.880005pt}}
\pgfusepath{stroke}
\pgfpathmoveto{\pgfpoint{244.511993pt}{171.056854pt}}
\pgflineto{\pgfpoint{235.601913pt}{171.056854pt}}
\pgfusepath{stroke}
\pgfpathmoveto{\pgfpoint{244.511993pt}{177.233673pt}}
\pgflineto{\pgfpoint{235.601852pt}{177.233673pt}}
\pgfusepath{stroke}
\pgfpathmoveto{\pgfpoint{244.511993pt}{183.410522pt}}
\pgflineto{\pgfpoint{235.601822pt}{183.410522pt}}
\pgfusepath{stroke}
\pgfpathmoveto{\pgfpoint{244.511993pt}{189.587372pt}}
\pgflineto{\pgfpoint{235.610855pt}{189.587372pt}}
\pgfusepath{stroke}
\pgfpathmoveto{\pgfpoint{244.511993pt}{195.764206pt}}
\pgflineto{\pgfpoint{235.610855pt}{195.764206pt}}
\pgfusepath{stroke}
\pgfpathmoveto{\pgfpoint{244.511993pt}{201.941055pt}}
\pgflineto{\pgfpoint{235.610870pt}{201.941055pt}}
\pgfusepath{stroke}
\pgfpathmoveto{\pgfpoint{244.511993pt}{208.117905pt}}
\pgflineto{\pgfpoint{235.610840pt}{208.117905pt}}
\pgfusepath{stroke}
\pgfpathmoveto{\pgfpoint{235.610825pt}{214.294739pt}}
\pgflineto{\pgfpoint{244.493912pt}{214.294739pt}}
\pgfusepath{stroke}
\pgfpathmoveto{\pgfpoint{235.619827pt}{220.471588pt}}
\pgflineto{\pgfpoint{244.484909pt}{220.471588pt}}
\pgfusepath{stroke}
\pgfpathmoveto{\pgfpoint{235.619797pt}{226.648422pt}}
\pgflineto{\pgfpoint{244.475861pt}{226.648422pt}}
\pgfusepath{stroke}
\pgfpathmoveto{\pgfpoint{235.619766pt}{232.825272pt}}
\pgflineto{\pgfpoint{244.466843pt}{232.825272pt}}
\pgfusepath{stroke}
\pgfpathmoveto{\pgfpoint{235.619781pt}{239.002106pt}}
\pgflineto{\pgfpoint{244.462296pt}{239.002106pt}}
\pgfusepath{stroke}
\pgfpathmoveto{\pgfpoint{235.619781pt}{245.178955pt}}
\pgflineto{\pgfpoint{244.453262pt}{245.178955pt}}
\pgfusepath{stroke}
\pgfpathmoveto{\pgfpoint{235.619751pt}{251.355804pt}}
\pgflineto{\pgfpoint{244.444214pt}{251.355804pt}}
\pgfusepath{stroke}
\pgfpathmoveto{\pgfpoint{235.624283pt}{257.532623pt}}
\pgflineto{\pgfpoint{244.439651pt}{257.532623pt}}
\pgfusepath{stroke}
\pgfpathmoveto{\pgfpoint{235.624222pt}{263.709473pt}}
\pgflineto{\pgfpoint{244.430649pt}{263.709473pt}}
\pgfusepath{stroke}
\pgfpathmoveto{\pgfpoint{235.628769pt}{269.886322pt}}
\pgflineto{\pgfpoint{244.426132pt}{269.886322pt}}
\pgfusepath{stroke}
\pgfpathmoveto{\pgfpoint{235.628708pt}{276.063141pt}}
\pgflineto{\pgfpoint{244.417068pt}{276.063141pt}}
\pgfusepath{stroke}
\pgfpathmoveto{\pgfpoint{235.628708pt}{282.239990pt}}
\pgflineto{\pgfpoint{244.412476pt}{282.239990pt}}
\pgfusepath{stroke}
\pgfpathmoveto{\pgfpoint{235.633209pt}{288.416840pt}}
\pgflineto{\pgfpoint{244.403488pt}{288.416840pt}}
\pgfusepath{stroke}
\pgfpathmoveto{\pgfpoint{235.633209pt}{294.593689pt}}
\pgflineto{\pgfpoint{244.394455pt}{294.593689pt}}
\pgfusepath{stroke}
\pgfpathmoveto{\pgfpoint{235.637726pt}{300.770538pt}}
\pgflineto{\pgfpoint{244.389954pt}{300.770538pt}}
\pgfusepath{stroke}
\pgfpathmoveto{\pgfpoint{235.637726pt}{306.947388pt}}
\pgflineto{\pgfpoint{244.380936pt}{306.947388pt}}
\pgfusepath{stroke}
\pgfpathmoveto{\pgfpoint{235.642181pt}{313.124207pt}}
\pgflineto{\pgfpoint{244.376389pt}{313.124207pt}}
\pgfusepath{stroke}
\pgfpathmoveto{\pgfpoint{235.642151pt}{319.301056pt}}
\pgflineto{\pgfpoint{244.367340pt}{319.301056pt}}
\pgfusepath{stroke}
\pgfpathmoveto{\pgfpoint{235.642212pt}{325.477905pt}}
\pgflineto{\pgfpoint{244.362747pt}{325.477905pt}}
\pgfusepath{stroke}
\pgfpathmoveto{\pgfpoint{235.646652pt}{331.654724pt}}
\pgflineto{\pgfpoint{244.353760pt}{331.654724pt}}
\pgfusepath{stroke}
\pgfpathmoveto{\pgfpoint{235.646683pt}{337.831604pt}}
\pgflineto{\pgfpoint{244.344788pt}{337.831604pt}}
\pgfusepath{stroke}
\pgfpathmoveto{\pgfpoint{235.651138pt}{344.008423pt}}
\pgflineto{\pgfpoint{244.340210pt}{344.008423pt}}
\pgfusepath{stroke}
\pgfpathmoveto{\pgfpoint{244.511993pt}{152.526306pt}}
\pgflineto{\pgfpoint{244.511993pt}{146.349472pt}}
\pgfusepath{stroke}
\pgfpathmoveto{\pgfpoint{244.511993pt}{146.349472pt}}
\pgflineto{\pgfpoint{244.511993pt}{140.172638pt}}
\pgfusepath{stroke}
\pgfpathmoveto{\pgfpoint{244.511993pt}{158.703156pt}}
\pgflineto{\pgfpoint{244.511993pt}{152.526306pt}}
\pgfusepath{stroke}
\pgfpathmoveto{\pgfpoint{244.511993pt}{164.880005pt}}
\pgflineto{\pgfpoint{244.511993pt}{158.703156pt}}
\pgfusepath{stroke}
\pgfpathmoveto{\pgfpoint{244.511993pt}{171.056854pt}}
\pgflineto{\pgfpoint{244.511993pt}{164.880005pt}}
\pgfusepath{stroke}
\pgfpathmoveto{\pgfpoint{244.511993pt}{177.233673pt}}
\pgflineto{\pgfpoint{244.511993pt}{171.056854pt}}
\pgfusepath{stroke}
\pgfpathmoveto{\pgfpoint{244.511993pt}{183.410522pt}}
\pgflineto{\pgfpoint{244.511993pt}{177.233673pt}}
\pgfusepath{stroke}
\pgfpathmoveto{\pgfpoint{244.511993pt}{189.587372pt}}
\pgflineto{\pgfpoint{244.511993pt}{183.410522pt}}
\pgfusepath{stroke}
\pgfpathmoveto{\pgfpoint{244.511993pt}{146.349472pt}}
\pgflineto{\pgfpoint{244.530014pt}{146.349472pt}}
\pgfusepath{stroke}
\pgfpathmoveto{\pgfpoint{244.511993pt}{152.526306pt}}
\pgflineto{\pgfpoint{244.529984pt}{152.526306pt}}
\pgfusepath{stroke}
\pgfpathmoveto{\pgfpoint{244.511993pt}{158.703156pt}}
\pgflineto{\pgfpoint{244.530029pt}{158.703156pt}}
\pgfusepath{stroke}
\pgfpathmoveto{\pgfpoint{244.511993pt}{164.880005pt}}
\pgflineto{\pgfpoint{244.530029pt}{164.880005pt}}
\pgfusepath{stroke}
\pgfpathmoveto{\pgfpoint{244.511993pt}{171.056854pt}}
\pgflineto{\pgfpoint{244.539017pt}{171.056854pt}}
\pgfusepath{stroke}
\pgfpathmoveto{\pgfpoint{244.511993pt}{177.233673pt}}
\pgflineto{\pgfpoint{244.538986pt}{177.233673pt}}
\pgfusepath{stroke}
\pgfpathmoveto{\pgfpoint{244.511993pt}{183.410522pt}}
\pgflineto{\pgfpoint{244.538986pt}{183.410522pt}}
\pgfusepath{stroke}
\pgfpathmoveto{\pgfpoint{244.511993pt}{201.941055pt}}
\pgflineto{\pgfpoint{244.511993pt}{195.764206pt}}
\pgfusepath{stroke}
\pgfpathmoveto{\pgfpoint{244.511993pt}{195.764206pt}}
\pgflineto{\pgfpoint{244.511993pt}{189.587372pt}}
\pgfusepath{stroke}
\pgfpathmoveto{\pgfpoint{244.511993pt}{208.117905pt}}
\pgflineto{\pgfpoint{244.511993pt}{201.941055pt}}
\pgfusepath{stroke}
\pgfpathmoveto{\pgfpoint{244.511993pt}{220.471588pt}}
\pgflineto{\pgfpoint{244.484909pt}{220.471588pt}}
\pgfusepath{stroke}
\pgfpathmoveto{\pgfpoint{244.511993pt}{226.648422pt}}
\pgflineto{\pgfpoint{244.475861pt}{226.648422pt}}
\pgfusepath{stroke}
\pgfpathmoveto{\pgfpoint{244.511993pt}{232.825272pt}}
\pgflineto{\pgfpoint{244.466843pt}{232.825272pt}}
\pgfusepath{stroke}
\pgfpathmoveto{\pgfpoint{244.511993pt}{239.002106pt}}
\pgflineto{\pgfpoint{244.462296pt}{239.002106pt}}
\pgfusepath{stroke}
\pgfpathmoveto{\pgfpoint{244.511993pt}{245.178955pt}}
\pgflineto{\pgfpoint{244.453262pt}{245.178955pt}}
\pgfusepath{stroke}
\pgfpathmoveto{\pgfpoint{244.511993pt}{251.355804pt}}
\pgflineto{\pgfpoint{244.444214pt}{251.355804pt}}
\pgfusepath{stroke}
\pgfpathmoveto{\pgfpoint{244.511993pt}{257.532623pt}}
\pgflineto{\pgfpoint{244.439651pt}{257.532623pt}}
\pgfusepath{stroke}
\pgfpathmoveto{\pgfpoint{244.511993pt}{263.709473pt}}
\pgflineto{\pgfpoint{244.430649pt}{263.709473pt}}
\pgfusepath{stroke}
\pgfpathmoveto{\pgfpoint{244.511993pt}{269.886322pt}}
\pgflineto{\pgfpoint{244.426132pt}{269.886322pt}}
\pgfusepath{stroke}
\pgfpathmoveto{\pgfpoint{244.511993pt}{276.063141pt}}
\pgflineto{\pgfpoint{244.417068pt}{276.063141pt}}
\pgfusepath{stroke}
\pgfpathmoveto{\pgfpoint{244.511993pt}{282.239990pt}}
\pgflineto{\pgfpoint{244.412476pt}{282.239990pt}}
\pgfusepath{stroke}
\pgfpathmoveto{\pgfpoint{244.511993pt}{288.416840pt}}
\pgflineto{\pgfpoint{244.403488pt}{288.416840pt}}
\pgfusepath{stroke}
\pgfpathmoveto{\pgfpoint{244.511993pt}{294.593689pt}}
\pgflineto{\pgfpoint{244.394455pt}{294.593689pt}}
\pgfusepath{stroke}
\pgfpathmoveto{\pgfpoint{244.511993pt}{300.770538pt}}
\pgflineto{\pgfpoint{244.389954pt}{300.770538pt}}
\pgfusepath{stroke}
\pgfpathmoveto{\pgfpoint{244.511993pt}{306.947388pt}}
\pgflineto{\pgfpoint{244.380936pt}{306.947388pt}}
\pgfusepath{stroke}
\pgfpathmoveto{\pgfpoint{244.511993pt}{313.124207pt}}
\pgflineto{\pgfpoint{244.376389pt}{313.124207pt}}
\pgfusepath{stroke}
\pgfpathmoveto{\pgfpoint{244.511993pt}{319.301056pt}}
\pgflineto{\pgfpoint{244.367340pt}{319.301056pt}}
\pgfusepath{stroke}
\pgfpathmoveto{\pgfpoint{244.511993pt}{325.477905pt}}
\pgflineto{\pgfpoint{244.362747pt}{325.477905pt}}
\pgfusepath{stroke}
\pgfpathmoveto{\pgfpoint{244.511993pt}{331.654724pt}}
\pgflineto{\pgfpoint{244.353760pt}{331.654724pt}}
\pgfusepath{stroke}
\pgfpathmoveto{\pgfpoint{244.511993pt}{337.831604pt}}
\pgflineto{\pgfpoint{244.344788pt}{337.831604pt}}
\pgfusepath{stroke}
\pgfpathmoveto{\pgfpoint{244.511993pt}{344.008423pt}}
\pgflineto{\pgfpoint{244.340210pt}{344.008423pt}}
\pgfusepath{stroke}
\pgfpathmoveto{\pgfpoint{244.511993pt}{325.477905pt}}
\pgflineto{\pgfpoint{244.511993pt}{319.301056pt}}
\pgfusepath{stroke}
\pgfpathmoveto{\pgfpoint{244.511993pt}{300.770538pt}}
\pgflineto{\pgfpoint{244.511993pt}{294.593689pt}}
\pgfusepath{stroke}
\pgfpathmoveto{\pgfpoint{244.511993pt}{306.947388pt}}
\pgflineto{\pgfpoint{244.511993pt}{300.770538pt}}
\pgfusepath{stroke}
\pgfpathmoveto{\pgfpoint{244.511993pt}{313.124207pt}}
\pgflineto{\pgfpoint{244.511993pt}{306.947388pt}}
\pgfusepath{stroke}
\pgfpathmoveto{\pgfpoint{244.511993pt}{319.301056pt}}
\pgflineto{\pgfpoint{244.511993pt}{313.124207pt}}
\pgfusepath{stroke}
\pgfpathmoveto{\pgfpoint{244.511993pt}{331.654724pt}}
\pgflineto{\pgfpoint{244.511993pt}{325.477905pt}}
\pgfusepath{stroke}
\pgfpathmoveto{\pgfpoint{244.511993pt}{337.831604pt}}
\pgflineto{\pgfpoint{244.511993pt}{331.654724pt}}
\pgfusepath{stroke}
\pgfpathmoveto{\pgfpoint{244.511993pt}{344.008423pt}}
\pgflineto{\pgfpoint{244.511993pt}{337.831604pt}}
\pgfusepath{stroke}
\pgfpathmoveto{\pgfpoint{244.511993pt}{300.770538pt}}
\pgflineto{\pgfpoint{244.525467pt}{300.770538pt}}
\pgfusepath{stroke}
\pgfpathmoveto{\pgfpoint{244.511993pt}{306.947388pt}}
\pgflineto{\pgfpoint{244.525467pt}{306.947388pt}}
\pgfusepath{stroke}
\pgfpathmoveto{\pgfpoint{244.511993pt}{313.124207pt}}
\pgflineto{\pgfpoint{244.529953pt}{313.124207pt}}
\pgfusepath{stroke}
\pgfpathmoveto{\pgfpoint{244.511993pt}{319.301056pt}}
\pgflineto{\pgfpoint{244.529938pt}{319.301056pt}}
\pgfusepath{stroke}
\pgfpathmoveto{\pgfpoint{244.511993pt}{325.477905pt}}
\pgflineto{\pgfpoint{244.534454pt}{325.477905pt}}
\pgfusepath{stroke}
\pgfpathmoveto{\pgfpoint{244.511993pt}{331.654724pt}}
\pgflineto{\pgfpoint{244.534393pt}{331.654724pt}}
\pgfusepath{stroke}
\pgfpathmoveto{\pgfpoint{244.511993pt}{337.831604pt}}
\pgflineto{\pgfpoint{244.534454pt}{337.831604pt}}
\pgfusepath{stroke}
\pgfpathmoveto{\pgfpoint{244.511993pt}{263.709473pt}}
\pgflineto{\pgfpoint{244.511993pt}{257.532623pt}}
\pgfusepath{stroke}
\pgfpathmoveto{\pgfpoint{244.511993pt}{214.294739pt}}
\pgflineto{\pgfpoint{244.511993pt}{208.117905pt}}
\pgfusepath{stroke}
\pgfpathmoveto{\pgfpoint{244.511993pt}{220.471588pt}}
\pgflineto{\pgfpoint{244.511993pt}{214.294739pt}}
\pgfusepath{stroke}
\pgfpathmoveto{\pgfpoint{244.511993pt}{226.648422pt}}
\pgflineto{\pgfpoint{244.511993pt}{220.471588pt}}
\pgfusepath{stroke}
\pgfpathmoveto{\pgfpoint{244.511993pt}{232.825272pt}}
\pgflineto{\pgfpoint{244.511993pt}{226.648422pt}}
\pgfusepath{stroke}
\pgfpathmoveto{\pgfpoint{244.511993pt}{239.002106pt}}
\pgflineto{\pgfpoint{244.511993pt}{232.825272pt}}
\pgfusepath{stroke}
\pgfpathmoveto{\pgfpoint{244.511993pt}{245.178955pt}}
\pgflineto{\pgfpoint{244.511993pt}{239.002106pt}}
\pgfusepath{stroke}
\pgfpathmoveto{\pgfpoint{244.511993pt}{251.355804pt}}
\pgflineto{\pgfpoint{244.511993pt}{245.178955pt}}
\pgfusepath{stroke}
\pgfpathmoveto{\pgfpoint{244.511993pt}{257.532623pt}}
\pgflineto{\pgfpoint{244.511993pt}{251.355804pt}}
\pgfusepath{stroke}
\pgfpathmoveto{\pgfpoint{244.511993pt}{269.886322pt}}
\pgflineto{\pgfpoint{244.511993pt}{263.709473pt}}
\pgfusepath{stroke}
\pgfpathmoveto{\pgfpoint{244.511993pt}{276.063141pt}}
\pgflineto{\pgfpoint{244.511993pt}{269.886322pt}}
\pgfusepath{stroke}
\pgfpathmoveto{\pgfpoint{244.511993pt}{282.239990pt}}
\pgflineto{\pgfpoint{244.511993pt}{276.063141pt}}
\pgfusepath{stroke}
\pgfpathmoveto{\pgfpoint{244.511993pt}{288.416840pt}}
\pgflineto{\pgfpoint{244.511993pt}{282.239990pt}}
\pgfusepath{stroke}
\pgfpathmoveto{\pgfpoint{244.511993pt}{294.593689pt}}
\pgflineto{\pgfpoint{244.511993pt}{288.416840pt}}
\pgfusepath{stroke}
\pgfpathmoveto{\pgfpoint{244.511993pt}{214.294739pt}}
\pgflineto{\pgfpoint{244.493912pt}{214.294739pt}}
\pgfusepath{stroke}
\pgfpathmoveto{\pgfpoint{244.511993pt}{189.587372pt}}
\pgflineto{\pgfpoint{244.539017pt}{189.587372pt}}
\pgfusepath{stroke}
\pgfpathmoveto{\pgfpoint{244.511993pt}{195.764206pt}}
\pgflineto{\pgfpoint{244.538986pt}{195.764206pt}}
\pgfusepath{stroke}
\pgfpathmoveto{\pgfpoint{244.511993pt}{201.941055pt}}
\pgflineto{\pgfpoint{244.547989pt}{201.941055pt}}
\pgfusepath{stroke}
\pgfpathmoveto{\pgfpoint{244.511993pt}{208.117905pt}}
\pgflineto{\pgfpoint{244.547958pt}{208.117905pt}}
\pgfusepath{stroke}
\pgfpathmoveto{\pgfpoint{244.511993pt}{214.294739pt}}
\pgflineto{\pgfpoint{244.548004pt}{214.294739pt}}
\pgfusepath{stroke}
\pgfpathmoveto{\pgfpoint{244.511993pt}{220.471588pt}}
\pgflineto{\pgfpoint{244.548004pt}{220.471588pt}}
\pgfusepath{stroke}
\pgfpathmoveto{\pgfpoint{244.511993pt}{226.648422pt}}
\pgflineto{\pgfpoint{244.547989pt}{226.648422pt}}
\pgfusepath{stroke}
\pgfpathmoveto{\pgfpoint{244.511993pt}{232.825272pt}}
\pgflineto{\pgfpoint{244.547989pt}{232.825272pt}}
\pgfusepath{stroke}
\pgfpathmoveto{\pgfpoint{244.511993pt}{239.002106pt}}
\pgflineto{\pgfpoint{244.552475pt}{239.002106pt}}
\pgfusepath{stroke}
\pgfpathmoveto{\pgfpoint{244.511993pt}{245.178955pt}}
\pgflineto{\pgfpoint{244.552475pt}{245.178955pt}}
\pgfusepath{stroke}
\pgfpathmoveto{\pgfpoint{244.511993pt}{251.355804pt}}
\pgflineto{\pgfpoint{244.552490pt}{251.355804pt}}
\pgfusepath{stroke}
\pgfpathmoveto{\pgfpoint{244.511993pt}{257.532623pt}}
\pgflineto{\pgfpoint{244.556961pt}{257.532623pt}}
\pgfusepath{stroke}
\pgfpathmoveto{\pgfpoint{244.511993pt}{263.709473pt}}
\pgflineto{\pgfpoint{244.556961pt}{263.709473pt}}
\pgfusepath{stroke}
\pgfpathmoveto{\pgfpoint{244.511993pt}{269.886322pt}}
\pgflineto{\pgfpoint{244.561462pt}{269.886322pt}}
\pgfusepath{stroke}
\pgfpathmoveto{\pgfpoint{244.511993pt}{276.063141pt}}
\pgflineto{\pgfpoint{244.561432pt}{276.063141pt}}
\pgfusepath{stroke}
\pgfpathmoveto{\pgfpoint{244.511993pt}{282.239990pt}}
\pgflineto{\pgfpoint{244.565948pt}{282.239990pt}}
\pgfusepath{stroke}
\pgfpathmoveto{\pgfpoint{244.511993pt}{288.416840pt}}
\pgflineto{\pgfpoint{244.565948pt}{288.416840pt}}
\pgfusepath{stroke}
\pgfpathmoveto{\pgfpoint{244.511993pt}{294.593689pt}}
\pgflineto{\pgfpoint{244.565979pt}{294.593689pt}}
\pgfusepath{stroke}
\pgfpathmoveto{\pgfpoint{244.570465pt}{300.770538pt}}
\pgflineto{\pgfpoint{244.525467pt}{300.770538pt}}
\pgfusepath{stroke}
\pgfpathmoveto{\pgfpoint{244.570465pt}{306.947388pt}}
\pgflineto{\pgfpoint{244.525467pt}{306.947388pt}}
\pgfusepath{stroke}
\pgfpathmoveto{\pgfpoint{244.574936pt}{313.124207pt}}
\pgflineto{\pgfpoint{244.529953pt}{313.124207pt}}
\pgfusepath{stroke}
\pgfpathmoveto{\pgfpoint{244.574966pt}{319.301056pt}}
\pgflineto{\pgfpoint{244.529938pt}{319.301056pt}}
\pgfusepath{stroke}
\pgfpathmoveto{\pgfpoint{244.579483pt}{325.477905pt}}
\pgflineto{\pgfpoint{244.534454pt}{325.477905pt}}
\pgfusepath{stroke}
\pgfpathmoveto{\pgfpoint{244.579422pt}{331.654724pt}}
\pgflineto{\pgfpoint{244.534393pt}{331.654724pt}}
\pgfusepath{stroke}
\pgfpathmoveto{\pgfpoint{244.579483pt}{337.831604pt}}
\pgflineto{\pgfpoint{244.534454pt}{337.831604pt}}
\pgfusepath{stroke}
\pgfpathmoveto{\pgfpoint{244.511993pt}{90.757896pt}}
\pgflineto{\pgfpoint{244.511993pt}{84.581039pt}}
\pgfusepath{stroke}
\pgfpathmoveto{\pgfpoint{244.511993pt}{59.873672pt}}
\pgflineto{\pgfpoint{244.511993pt}{53.696838pt}}
\pgfusepath{stroke}
\pgfpathmoveto{\pgfpoint{244.511993pt}{66.050522pt}}
\pgflineto{\pgfpoint{244.511993pt}{59.873672pt}}
\pgfusepath{stroke}
\pgfpathmoveto{\pgfpoint{244.511993pt}{72.227356pt}}
\pgflineto{\pgfpoint{244.511993pt}{66.050522pt}}
\pgfusepath{stroke}
\pgfpathmoveto{\pgfpoint{244.511993pt}{78.404205pt}}
\pgflineto{\pgfpoint{244.511993pt}{72.227356pt}}
\pgfusepath{stroke}
\pgfpathmoveto{\pgfpoint{244.511993pt}{84.581039pt}}
\pgflineto{\pgfpoint{244.511993pt}{78.404205pt}}
\pgfusepath{stroke}
\pgfpathmoveto{\pgfpoint{244.511993pt}{96.934731pt}}
\pgflineto{\pgfpoint{244.511993pt}{90.757896pt}}
\pgfusepath{stroke}
\pgfpathmoveto{\pgfpoint{244.511993pt}{103.111580pt}}
\pgflineto{\pgfpoint{244.511993pt}{96.934731pt}}
\pgfusepath{stroke}
\pgfpathmoveto{\pgfpoint{244.511993pt}{109.288422pt}}
\pgflineto{\pgfpoint{244.511993pt}{103.111580pt}}
\pgfusepath{stroke}
\pgfpathmoveto{\pgfpoint{244.511993pt}{115.465263pt}}
\pgflineto{\pgfpoint{244.511993pt}{109.288422pt}}
\pgfusepath{stroke}
\pgfpathmoveto{\pgfpoint{244.511993pt}{121.642097pt}}
\pgflineto{\pgfpoint{244.511993pt}{115.465263pt}}
\pgfusepath{stroke}
\pgfpathmoveto{\pgfpoint{244.511993pt}{127.818947pt}}
\pgflineto{\pgfpoint{244.511993pt}{121.642097pt}}
\pgfusepath{stroke}
\pgfpathmoveto{\pgfpoint{244.511993pt}{133.995789pt}}
\pgflineto{\pgfpoint{244.511993pt}{127.818947pt}}
\pgfusepath{stroke}
\pgfpathmoveto{\pgfpoint{244.511993pt}{140.172638pt}}
\pgflineto{\pgfpoint{244.511993pt}{133.995789pt}}
\pgfusepath{stroke}
\pgfpathmoveto{\pgfpoint{244.511993pt}{47.519989pt}}
\pgflineto{\pgfpoint{244.493988pt}{47.519989pt}}
\pgfusepath{stroke}
\pgfpathmoveto{\pgfpoint{244.511993pt}{53.696838pt}}
\pgflineto{\pgfpoint{244.511993pt}{47.519989pt}}
\pgfusepath{stroke}
\pgfpathmoveto{\pgfpoint{244.511993pt}{47.519989pt}}
\pgflineto{\pgfpoint{253.422119pt}{47.519989pt}}
\pgfusepath{stroke}
\pgfpathmoveto{\pgfpoint{244.511993pt}{53.696838pt}}
\pgflineto{\pgfpoint{253.422104pt}{53.696838pt}}
\pgfusepath{stroke}
\pgfpathmoveto{\pgfpoint{244.511993pt}{59.873672pt}}
\pgflineto{\pgfpoint{253.422073pt}{59.873672pt}}
\pgfusepath{stroke}
\pgfpathmoveto{\pgfpoint{253.440002pt}{66.050522pt}}
\pgflineto{\pgfpoint{244.511993pt}{66.050522pt}}
\pgfusepath{stroke}
\pgfpathmoveto{\pgfpoint{253.440002pt}{72.227356pt}}
\pgflineto{\pgfpoint{244.511993pt}{72.227356pt}}
\pgfusepath{stroke}
\pgfpathmoveto{\pgfpoint{253.440002pt}{78.404205pt}}
\pgflineto{\pgfpoint{244.511993pt}{78.404205pt}}
\pgfusepath{stroke}
\pgfpathmoveto{\pgfpoint{253.440002pt}{84.581039pt}}
\pgflineto{\pgfpoint{244.511993pt}{84.581039pt}}
\pgfusepath{stroke}
\pgfpathmoveto{\pgfpoint{253.440002pt}{90.757896pt}}
\pgflineto{\pgfpoint{244.511993pt}{90.757896pt}}
\pgfusepath{stroke}
\pgfpathmoveto{\pgfpoint{253.440002pt}{96.934731pt}}
\pgflineto{\pgfpoint{244.511993pt}{96.934731pt}}
\pgfusepath{stroke}
\pgfpathmoveto{\pgfpoint{253.440002pt}{103.111580pt}}
\pgflineto{\pgfpoint{244.511993pt}{103.111580pt}}
\pgfusepath{stroke}
\pgfpathmoveto{\pgfpoint{253.440002pt}{109.288422pt}}
\pgflineto{\pgfpoint{244.511993pt}{109.288422pt}}
\pgfusepath{stroke}
\pgfpathmoveto{\pgfpoint{253.440002pt}{115.465263pt}}
\pgflineto{\pgfpoint{244.511993pt}{115.465263pt}}
\pgfusepath{stroke}
\pgfpathmoveto{\pgfpoint{253.440002pt}{121.642097pt}}
\pgflineto{\pgfpoint{244.511993pt}{121.642097pt}}
\pgfusepath{stroke}
\pgfpathmoveto{\pgfpoint{253.440002pt}{127.818947pt}}
\pgflineto{\pgfpoint{244.511993pt}{127.818947pt}}
\pgfusepath{stroke}
\pgfpathmoveto{\pgfpoint{253.440002pt}{133.995789pt}}
\pgflineto{\pgfpoint{244.511993pt}{133.995789pt}}
\pgfusepath{stroke}
\pgfpathmoveto{\pgfpoint{253.440002pt}{140.172638pt}}
\pgflineto{\pgfpoint{244.511993pt}{140.172638pt}}
\pgfusepath{stroke}
\pgfpathmoveto{\pgfpoint{253.440002pt}{146.349472pt}}
\pgflineto{\pgfpoint{244.530014pt}{146.349472pt}}
\pgfusepath{stroke}
\pgfpathmoveto{\pgfpoint{253.440002pt}{152.526306pt}}
\pgflineto{\pgfpoint{244.529984pt}{152.526306pt}}
\pgfusepath{stroke}
\pgfpathmoveto{\pgfpoint{253.440002pt}{158.703156pt}}
\pgflineto{\pgfpoint{244.530029pt}{158.703156pt}}
\pgfusepath{stroke}
\pgfpathmoveto{\pgfpoint{253.440002pt}{164.880005pt}}
\pgflineto{\pgfpoint{244.530029pt}{164.880005pt}}
\pgfusepath{stroke}
\pgfpathmoveto{\pgfpoint{253.440002pt}{171.056854pt}}
\pgflineto{\pgfpoint{244.539017pt}{171.056854pt}}
\pgfusepath{stroke}
\pgfpathmoveto{\pgfpoint{253.440002pt}{177.233673pt}}
\pgflineto{\pgfpoint{244.538986pt}{177.233673pt}}
\pgfusepath{stroke}
\pgfpathmoveto{\pgfpoint{253.440002pt}{183.410522pt}}
\pgflineto{\pgfpoint{244.538986pt}{183.410522pt}}
\pgfusepath{stroke}
\pgfpathmoveto{\pgfpoint{253.440002pt}{189.587372pt}}
\pgflineto{\pgfpoint{244.539017pt}{189.587372pt}}
\pgfusepath{stroke}
\pgfpathmoveto{\pgfpoint{253.440002pt}{195.764206pt}}
\pgflineto{\pgfpoint{244.538986pt}{195.764206pt}}
\pgfusepath{stroke}
\pgfpathmoveto{\pgfpoint{253.440002pt}{201.941055pt}}
\pgflineto{\pgfpoint{244.547989pt}{201.941055pt}}
\pgfusepath{stroke}
\pgfpathmoveto{\pgfpoint{253.440002pt}{208.117905pt}}
\pgflineto{\pgfpoint{244.547958pt}{208.117905pt}}
\pgfusepath{stroke}
\pgfpathmoveto{\pgfpoint{253.440002pt}{214.294739pt}}
\pgflineto{\pgfpoint{244.548004pt}{214.294739pt}}
\pgfusepath{stroke}
\pgfpathmoveto{\pgfpoint{253.440002pt}{220.471588pt}}
\pgflineto{\pgfpoint{244.548004pt}{220.471588pt}}
\pgfusepath{stroke}
\pgfpathmoveto{\pgfpoint{244.547989pt}{226.648422pt}}
\pgflineto{\pgfpoint{253.417419pt}{226.648422pt}}
\pgfusepath{stroke}
\pgfpathmoveto{\pgfpoint{244.547989pt}{232.825272pt}}
\pgflineto{\pgfpoint{253.408417pt}{232.825272pt}}
\pgfusepath{stroke}
\pgfpathmoveto{\pgfpoint{244.552475pt}{239.002106pt}}
\pgflineto{\pgfpoint{253.403915pt}{239.002106pt}}
\pgfusepath{stroke}
\pgfpathmoveto{\pgfpoint{244.552475pt}{245.178955pt}}
\pgflineto{\pgfpoint{253.394867pt}{245.178955pt}}
\pgfusepath{stroke}
\pgfpathmoveto{\pgfpoint{244.552490pt}{251.355804pt}}
\pgflineto{\pgfpoint{253.390381pt}{251.355804pt}}
\pgfusepath{stroke}
\pgfpathmoveto{\pgfpoint{244.556961pt}{257.532623pt}}
\pgflineto{\pgfpoint{253.381348pt}{257.532623pt}}
\pgfusepath{stroke}
\pgfpathmoveto{\pgfpoint{244.556961pt}{263.709473pt}}
\pgflineto{\pgfpoint{253.372299pt}{263.709473pt}}
\pgfusepath{stroke}
\pgfpathmoveto{\pgfpoint{244.561462pt}{269.886322pt}}
\pgflineto{\pgfpoint{253.367828pt}{269.886322pt}}
\pgfusepath{stroke}
\pgfpathmoveto{\pgfpoint{244.561432pt}{276.063141pt}}
\pgflineto{\pgfpoint{253.358780pt}{276.063141pt}}
\pgfusepath{stroke}
\pgfpathmoveto{\pgfpoint{244.565948pt}{282.239990pt}}
\pgflineto{\pgfpoint{253.354294pt}{282.239990pt}}
\pgfusepath{stroke}
\pgfpathmoveto{\pgfpoint{244.565948pt}{288.416840pt}}
\pgflineto{\pgfpoint{253.345261pt}{288.416840pt}}
\pgfusepath{stroke}
\pgfpathmoveto{\pgfpoint{244.565979pt}{294.593689pt}}
\pgflineto{\pgfpoint{253.340714pt}{294.593689pt}}
\pgfusepath{stroke}
\pgfpathmoveto{\pgfpoint{244.570465pt}{300.770538pt}}
\pgflineto{\pgfpoint{253.331772pt}{300.770538pt}}
\pgfusepath{stroke}
\pgfpathmoveto{\pgfpoint{244.570465pt}{306.947388pt}}
\pgflineto{\pgfpoint{253.322723pt}{306.947388pt}}
\pgfusepath{stroke}
\pgfpathmoveto{\pgfpoint{244.574936pt}{313.124207pt}}
\pgflineto{\pgfpoint{253.318207pt}{313.124207pt}}
\pgfusepath{stroke}
\pgfpathmoveto{\pgfpoint{244.574966pt}{319.301056pt}}
\pgflineto{\pgfpoint{253.309174pt}{319.301056pt}}
\pgfusepath{stroke}
\pgfpathmoveto{\pgfpoint{244.579483pt}{325.477905pt}}
\pgflineto{\pgfpoint{253.304688pt}{325.477905pt}}
\pgfusepath{stroke}
\pgfpathmoveto{\pgfpoint{244.579422pt}{331.654724pt}}
\pgflineto{\pgfpoint{253.295624pt}{331.654724pt}}
\pgfusepath{stroke}
\pgfpathmoveto{\pgfpoint{244.579483pt}{337.831604pt}}
\pgflineto{\pgfpoint{253.286682pt}{337.831604pt}}
\pgfusepath{stroke}
\pgfpathmoveto{\pgfpoint{253.440002pt}{164.880005pt}}
\pgflineto{\pgfpoint{253.440002pt}{158.703156pt}}
\pgfusepath{stroke}
\pgfpathmoveto{\pgfpoint{253.440002pt}{158.703156pt}}
\pgflineto{\pgfpoint{253.440002pt}{152.526306pt}}
\pgfusepath{stroke}
\pgfpathmoveto{\pgfpoint{253.440002pt}{171.056854pt}}
\pgflineto{\pgfpoint{253.440002pt}{164.880005pt}}
\pgfusepath{stroke}
\pgfpathmoveto{\pgfpoint{253.440002pt}{177.233673pt}}
\pgflineto{\pgfpoint{253.440002pt}{171.056854pt}}
\pgfusepath{stroke}
\pgfpathmoveto{\pgfpoint{253.440002pt}{183.410522pt}}
\pgflineto{\pgfpoint{253.440002pt}{177.233673pt}}
\pgfusepath{stroke}
\pgfpathmoveto{\pgfpoint{253.440002pt}{189.587372pt}}
\pgflineto{\pgfpoint{253.440002pt}{183.410522pt}}
\pgfusepath{stroke}
\pgfpathmoveto{\pgfpoint{253.440002pt}{195.764206pt}}
\pgflineto{\pgfpoint{253.440002pt}{189.587372pt}}
\pgfusepath{stroke}
\pgfpathmoveto{\pgfpoint{253.440002pt}{201.941055pt}}
\pgflineto{\pgfpoint{253.440002pt}{195.764206pt}}
\pgfusepath{stroke}
\pgfpathmoveto{\pgfpoint{253.440002pt}{158.703156pt}}
\pgflineto{\pgfpoint{253.457840pt}{158.703156pt}}
\pgfusepath{stroke}
\pgfpathmoveto{\pgfpoint{253.440002pt}{164.880005pt}}
\pgflineto{\pgfpoint{253.457840pt}{164.880005pt}}
\pgfusepath{stroke}
\pgfpathmoveto{\pgfpoint{253.440002pt}{171.056854pt}}
\pgflineto{\pgfpoint{253.457809pt}{171.056854pt}}
\pgfusepath{stroke}
\pgfpathmoveto{\pgfpoint{253.440002pt}{177.233673pt}}
\pgflineto{\pgfpoint{253.457779pt}{177.233673pt}}
\pgfusepath{stroke}
\pgfpathmoveto{\pgfpoint{253.440002pt}{183.410522pt}}
\pgflineto{\pgfpoint{253.457779pt}{183.410522pt}}
\pgfusepath{stroke}
\pgfpathmoveto{\pgfpoint{253.440002pt}{189.587372pt}}
\pgflineto{\pgfpoint{253.466766pt}{189.587372pt}}
\pgfusepath{stroke}
\pgfpathmoveto{\pgfpoint{253.440002pt}{195.764206pt}}
\pgflineto{\pgfpoint{253.466736pt}{195.764206pt}}
\pgfusepath{stroke}
\pgfpathmoveto{\pgfpoint{253.440002pt}{214.294739pt}}
\pgflineto{\pgfpoint{253.440002pt}{208.117905pt}}
\pgfusepath{stroke}
\pgfpathmoveto{\pgfpoint{253.440002pt}{208.117905pt}}
\pgflineto{\pgfpoint{253.440002pt}{201.941055pt}}
\pgfusepath{stroke}
\pgfpathmoveto{\pgfpoint{253.440002pt}{220.471588pt}}
\pgflineto{\pgfpoint{253.440002pt}{214.294739pt}}
\pgfusepath{stroke}
\pgfpathmoveto{\pgfpoint{253.408417pt}{232.825272pt}}
\pgflineto{\pgfpoint{253.426270pt}{232.825272pt}}
\pgfusepath{stroke}
\pgfpathmoveto{\pgfpoint{253.440002pt}{239.002106pt}}
\pgflineto{\pgfpoint{253.403915pt}{239.002106pt}}
\pgfusepath{stroke}
\pgfpathmoveto{\pgfpoint{253.440002pt}{245.178955pt}}
\pgflineto{\pgfpoint{253.394867pt}{245.178955pt}}
\pgfusepath{stroke}
\pgfpathmoveto{\pgfpoint{253.440002pt}{251.355804pt}}
\pgflineto{\pgfpoint{253.390381pt}{251.355804pt}}
\pgfusepath{stroke}
\pgfpathmoveto{\pgfpoint{253.440002pt}{257.532623pt}}
\pgflineto{\pgfpoint{253.381348pt}{257.532623pt}}
\pgfusepath{stroke}
\pgfpathmoveto{\pgfpoint{253.440002pt}{263.709473pt}}
\pgflineto{\pgfpoint{253.372299pt}{263.709473pt}}
\pgfusepath{stroke}
\pgfpathmoveto{\pgfpoint{253.440002pt}{269.886322pt}}
\pgflineto{\pgfpoint{253.367828pt}{269.886322pt}}
\pgfusepath{stroke}
\pgfpathmoveto{\pgfpoint{253.440002pt}{276.063141pt}}
\pgflineto{\pgfpoint{253.358780pt}{276.063141pt}}
\pgfusepath{stroke}
\pgfpathmoveto{\pgfpoint{253.440002pt}{282.239990pt}}
\pgflineto{\pgfpoint{253.354294pt}{282.239990pt}}
\pgfusepath{stroke}
\pgfpathmoveto{\pgfpoint{253.440002pt}{288.416840pt}}
\pgflineto{\pgfpoint{253.345261pt}{288.416840pt}}
\pgfusepath{stroke}
\pgfpathmoveto{\pgfpoint{253.440002pt}{294.593689pt}}
\pgflineto{\pgfpoint{253.340714pt}{294.593689pt}}
\pgfusepath{stroke}
\pgfpathmoveto{\pgfpoint{253.440002pt}{300.770538pt}}
\pgflineto{\pgfpoint{253.331772pt}{300.770538pt}}
\pgfusepath{stroke}
\pgfpathmoveto{\pgfpoint{253.440002pt}{306.947388pt}}
\pgflineto{\pgfpoint{253.322723pt}{306.947388pt}}
\pgfusepath{stroke}
\pgfpathmoveto{\pgfpoint{253.440002pt}{313.124207pt}}
\pgflineto{\pgfpoint{253.318207pt}{313.124207pt}}
\pgfusepath{stroke}
\pgfpathmoveto{\pgfpoint{253.440002pt}{319.301056pt}}
\pgflineto{\pgfpoint{253.309174pt}{319.301056pt}}
\pgfusepath{stroke}
\pgfpathmoveto{\pgfpoint{253.440002pt}{325.477905pt}}
\pgflineto{\pgfpoint{253.304688pt}{325.477905pt}}
\pgfusepath{stroke}
\pgfpathmoveto{\pgfpoint{253.440002pt}{331.654724pt}}
\pgflineto{\pgfpoint{253.295624pt}{331.654724pt}}
\pgfusepath{stroke}
\pgfpathmoveto{\pgfpoint{253.440002pt}{337.831604pt}}
\pgflineto{\pgfpoint{253.286682pt}{337.831604pt}}
\pgfusepath{stroke}
\pgfpathmoveto{\pgfpoint{253.440002pt}{325.477905pt}}
\pgflineto{\pgfpoint{253.440002pt}{319.301056pt}}
\pgfusepath{stroke}
\pgfpathmoveto{\pgfpoint{253.440002pt}{313.124207pt}}
\pgflineto{\pgfpoint{253.440002pt}{306.947388pt}}
\pgfusepath{stroke}
\pgfpathmoveto{\pgfpoint{253.440002pt}{319.301056pt}}
\pgflineto{\pgfpoint{253.440002pt}{313.124207pt}}
\pgfusepath{stroke}
\pgfpathmoveto{\pgfpoint{253.440002pt}{331.654724pt}}
\pgflineto{\pgfpoint{253.440002pt}{325.477905pt}}
\pgfusepath{stroke}
\pgfpathmoveto{\pgfpoint{253.440002pt}{337.831604pt}}
\pgflineto{\pgfpoint{253.440002pt}{331.654724pt}}
\pgfusepath{stroke}
\pgfpathmoveto{\pgfpoint{253.440002pt}{313.124207pt}}
\pgflineto{\pgfpoint{253.453735pt}{313.124207pt}}
\pgfusepath{stroke}
\pgfpathmoveto{\pgfpoint{253.440002pt}{319.301056pt}}
\pgflineto{\pgfpoint{253.453766pt}{319.301056pt}}
\pgfusepath{stroke}
\pgfpathmoveto{\pgfpoint{253.440002pt}{325.477905pt}}
\pgflineto{\pgfpoint{253.458313pt}{325.477905pt}}
\pgfusepath{stroke}
\pgfpathmoveto{\pgfpoint{253.440002pt}{331.654724pt}}
\pgflineto{\pgfpoint{253.458313pt}{331.654724pt}}
\pgfusepath{stroke}
\pgfpathmoveto{\pgfpoint{253.440002pt}{276.063141pt}}
\pgflineto{\pgfpoint{253.440002pt}{269.886322pt}}
\pgfusepath{stroke}
\pgfpathmoveto{\pgfpoint{253.440002pt}{245.178955pt}}
\pgflineto{\pgfpoint{253.440002pt}{239.002106pt}}
\pgfusepath{stroke}
\pgfpathmoveto{\pgfpoint{253.440002pt}{251.355804pt}}
\pgflineto{\pgfpoint{253.440002pt}{245.178955pt}}
\pgfusepath{stroke}
\pgfpathmoveto{\pgfpoint{253.440002pt}{257.532623pt}}
\pgflineto{\pgfpoint{253.440002pt}{251.355804pt}}
\pgfusepath{stroke}
\pgfpathmoveto{\pgfpoint{253.440002pt}{263.709473pt}}
\pgflineto{\pgfpoint{253.440002pt}{257.532623pt}}
\pgfusepath{stroke}
\pgfpathmoveto{\pgfpoint{253.440002pt}{269.886322pt}}
\pgflineto{\pgfpoint{253.440002pt}{263.709473pt}}
\pgfusepath{stroke}
\pgfpathmoveto{\pgfpoint{253.440002pt}{282.239990pt}}
\pgflineto{\pgfpoint{253.440002pt}{276.063141pt}}
\pgfusepath{stroke}
\pgfpathmoveto{\pgfpoint{253.440002pt}{288.416840pt}}
\pgflineto{\pgfpoint{253.440002pt}{282.239990pt}}
\pgfusepath{stroke}
\pgfpathmoveto{\pgfpoint{253.440002pt}{294.593689pt}}
\pgflineto{\pgfpoint{253.440002pt}{288.416840pt}}
\pgfusepath{stroke}
\pgfpathmoveto{\pgfpoint{253.440002pt}{300.770538pt}}
\pgflineto{\pgfpoint{253.440002pt}{294.593689pt}}
\pgfusepath{stroke}
\pgfpathmoveto{\pgfpoint{253.440002pt}{306.947388pt}}
\pgflineto{\pgfpoint{253.440002pt}{300.770538pt}}
\pgfusepath{stroke}
\pgfpathmoveto{\pgfpoint{253.440002pt}{226.648422pt}}
\pgflineto{\pgfpoint{253.417419pt}{226.648422pt}}
\pgfusepath{stroke}
\pgfpathmoveto{\pgfpoint{253.440002pt}{232.825272pt}}
\pgflineto{\pgfpoint{253.426270pt}{232.825272pt}}
\pgfusepath{stroke}
\pgfpathmoveto{\pgfpoint{253.440002pt}{239.002106pt}}
\pgflineto{\pgfpoint{253.440002pt}{232.825272pt}}
\pgfusepath{stroke}
\pgfpathmoveto{\pgfpoint{253.440002pt}{239.002106pt}}
\pgflineto{\pgfpoint{253.453552pt}{239.002106pt}}
\pgfusepath{stroke}
\pgfpathmoveto{\pgfpoint{253.440002pt}{245.178955pt}}
\pgflineto{\pgfpoint{253.458099pt}{245.178955pt}}
\pgfusepath{stroke}
\pgfpathmoveto{\pgfpoint{253.440002pt}{251.355804pt}}
\pgflineto{\pgfpoint{253.467178pt}{251.355804pt}}
\pgfusepath{stroke}
\pgfpathmoveto{\pgfpoint{253.440002pt}{257.532623pt}}
\pgflineto{\pgfpoint{253.480057pt}{257.532623pt}}
\pgfusepath{stroke}
\pgfpathmoveto{\pgfpoint{253.440002pt}{263.709473pt}}
\pgflineto{\pgfpoint{253.480072pt}{263.709473pt}}
\pgfusepath{stroke}
\pgfpathmoveto{\pgfpoint{253.440002pt}{269.886322pt}}
\pgflineto{\pgfpoint{253.484543pt}{269.886322pt}}
\pgfusepath{stroke}
\pgfpathmoveto{\pgfpoint{253.440002pt}{276.063141pt}}
\pgflineto{\pgfpoint{253.484482pt}{276.063141pt}}
\pgfusepath{stroke}
\pgfpathmoveto{\pgfpoint{253.440002pt}{282.239990pt}}
\pgflineto{\pgfpoint{253.489014pt}{282.239990pt}}
\pgfusepath{stroke}
\pgfpathmoveto{\pgfpoint{253.440002pt}{288.416840pt}}
\pgflineto{\pgfpoint{253.489014pt}{288.416840pt}}
\pgfusepath{stroke}
\pgfpathmoveto{\pgfpoint{253.440002pt}{294.593689pt}}
\pgflineto{\pgfpoint{253.493500pt}{294.593689pt}}
\pgfusepath{stroke}
\pgfpathmoveto{\pgfpoint{253.440002pt}{300.770538pt}}
\pgflineto{\pgfpoint{253.493500pt}{300.770538pt}}
\pgfusepath{stroke}
\pgfpathmoveto{\pgfpoint{253.440002pt}{306.947388pt}}
\pgflineto{\pgfpoint{253.493469pt}{306.947388pt}}
\pgfusepath{stroke}
\pgfpathmoveto{\pgfpoint{253.497925pt}{313.124207pt}}
\pgflineto{\pgfpoint{253.453735pt}{313.124207pt}}
\pgfusepath{stroke}
\pgfpathmoveto{\pgfpoint{253.497925pt}{319.301056pt}}
\pgflineto{\pgfpoint{253.453766pt}{319.301056pt}}
\pgfusepath{stroke}
\pgfpathmoveto{\pgfpoint{253.502426pt}{325.477905pt}}
\pgflineto{\pgfpoint{253.458313pt}{325.477905pt}}
\pgfusepath{stroke}
\pgfpathmoveto{\pgfpoint{253.502365pt}{331.654724pt}}
\pgflineto{\pgfpoint{253.458313pt}{331.654724pt}}
\pgfusepath{stroke}
\pgfpathmoveto{\pgfpoint{253.440002pt}{226.648422pt}}
\pgflineto{\pgfpoint{253.440002pt}{220.471588pt}}
\pgfusepath{stroke}
\pgfpathmoveto{\pgfpoint{253.440002pt}{232.825272pt}}
\pgflineto{\pgfpoint{253.440002pt}{226.648422pt}}
\pgfusepath{stroke}
\pgfpathmoveto{\pgfpoint{253.440002pt}{201.941055pt}}
\pgflineto{\pgfpoint{253.466736pt}{201.941055pt}}
\pgfusepath{stroke}
\pgfpathmoveto{\pgfpoint{253.440002pt}{208.117905pt}}
\pgflineto{\pgfpoint{253.466736pt}{208.117905pt}}
\pgfusepath{stroke}
\pgfpathmoveto{\pgfpoint{253.440002pt}{214.294739pt}}
\pgflineto{\pgfpoint{253.475708pt}{214.294739pt}}
\pgfusepath{stroke}
\pgfpathmoveto{\pgfpoint{253.440002pt}{220.471588pt}}
\pgflineto{\pgfpoint{253.475693pt}{220.471588pt}}
\pgfusepath{stroke}
\pgfpathmoveto{\pgfpoint{253.440002pt}{226.648422pt}}
\pgflineto{\pgfpoint{253.471161pt}{226.648422pt}}
\pgfusepath{stroke}
\pgfpathmoveto{\pgfpoint{253.440002pt}{232.825272pt}}
\pgflineto{\pgfpoint{253.471161pt}{232.825272pt}}
\pgfusepath{stroke}
\pgfpathmoveto{\pgfpoint{253.475647pt}{239.002106pt}}
\pgflineto{\pgfpoint{253.453552pt}{239.002106pt}}
\pgfusepath{stroke}
\pgfpathmoveto{\pgfpoint{253.475586pt}{245.178955pt}}
\pgflineto{\pgfpoint{253.458099pt}{245.178955pt}}
\pgfusepath{stroke}
\pgfpathmoveto{\pgfpoint{253.480103pt}{251.355804pt}}
\pgflineto{\pgfpoint{253.467178pt}{251.355804pt}}
\pgfusepath{stroke}
\pgfpathmoveto{\pgfpoint{253.440002pt}{103.111580pt}}
\pgflineto{\pgfpoint{253.440002pt}{96.934731pt}}
\pgfusepath{stroke}
\pgfpathmoveto{\pgfpoint{253.440002pt}{72.227356pt}}
\pgflineto{\pgfpoint{253.440002pt}{66.050522pt}}
\pgfusepath{stroke}
\pgfpathmoveto{\pgfpoint{253.440002pt}{78.404205pt}}
\pgflineto{\pgfpoint{253.440002pt}{72.227356pt}}
\pgfusepath{stroke}
\pgfpathmoveto{\pgfpoint{253.440002pt}{84.581039pt}}
\pgflineto{\pgfpoint{253.440002pt}{78.404205pt}}
\pgfusepath{stroke}
\pgfpathmoveto{\pgfpoint{253.440002pt}{90.757896pt}}
\pgflineto{\pgfpoint{253.440002pt}{84.581039pt}}
\pgfusepath{stroke}
\pgfpathmoveto{\pgfpoint{253.440002pt}{96.934731pt}}
\pgflineto{\pgfpoint{253.440002pt}{90.757896pt}}
\pgfusepath{stroke}
\pgfpathmoveto{\pgfpoint{253.440002pt}{109.288422pt}}
\pgflineto{\pgfpoint{253.440002pt}{103.111580pt}}
\pgfusepath{stroke}
\pgfpathmoveto{\pgfpoint{253.440002pt}{115.465263pt}}
\pgflineto{\pgfpoint{253.440002pt}{109.288422pt}}
\pgfusepath{stroke}
\pgfpathmoveto{\pgfpoint{253.440002pt}{121.642097pt}}
\pgflineto{\pgfpoint{253.440002pt}{115.465263pt}}
\pgfusepath{stroke}
\pgfpathmoveto{\pgfpoint{253.440002pt}{127.818947pt}}
\pgflineto{\pgfpoint{253.440002pt}{121.642097pt}}
\pgfusepath{stroke}
\pgfpathmoveto{\pgfpoint{253.440002pt}{133.995789pt}}
\pgflineto{\pgfpoint{253.440002pt}{127.818947pt}}
\pgfusepath{stroke}
\pgfpathmoveto{\pgfpoint{253.440002pt}{140.172638pt}}
\pgflineto{\pgfpoint{253.440002pt}{133.995789pt}}
\pgfusepath{stroke}
\pgfpathmoveto{\pgfpoint{253.440002pt}{146.349472pt}}
\pgflineto{\pgfpoint{253.440002pt}{140.172638pt}}
\pgfusepath{stroke}
\pgfpathmoveto{\pgfpoint{253.440002pt}{152.526306pt}}
\pgflineto{\pgfpoint{253.440002pt}{146.349472pt}}
\pgfusepath{stroke}
\pgfpathmoveto{\pgfpoint{253.440002pt}{47.519989pt}}
\pgflineto{\pgfpoint{253.422119pt}{47.519989pt}}
\pgfusepath{stroke}
\pgfpathmoveto{\pgfpoint{253.440002pt}{53.696838pt}}
\pgflineto{\pgfpoint{253.422104pt}{53.696838pt}}
\pgfusepath{stroke}
\pgfpathmoveto{\pgfpoint{253.440002pt}{59.873672pt}}
\pgflineto{\pgfpoint{253.422073pt}{59.873672pt}}
\pgfusepath{stroke}
\pgfpathmoveto{\pgfpoint{253.440002pt}{53.696838pt}}
\pgflineto{\pgfpoint{253.440002pt}{47.519989pt}}
\pgfusepath{stroke}
\pgfpathmoveto{\pgfpoint{253.440002pt}{47.519989pt}}
\pgflineto{\pgfpoint{253.458069pt}{47.519989pt}}
\pgfusepath{stroke}
\pgfpathmoveto{\pgfpoint{253.440002pt}{66.050522pt}}
\pgflineto{\pgfpoint{253.440002pt}{59.873672pt}}
\pgfusepath{stroke}
\pgfpathmoveto{\pgfpoint{253.440002pt}{59.873672pt}}
\pgflineto{\pgfpoint{253.440002pt}{53.696838pt}}
\pgfusepath{stroke}
\pgfpathmoveto{\pgfpoint{253.458069pt}{47.519989pt}}
\pgflineto{\pgfpoint{262.350098pt}{47.519989pt}}
\pgfusepath{stroke}
\pgfpathmoveto{\pgfpoint{262.367981pt}{53.696838pt}}
\pgflineto{\pgfpoint{253.440002pt}{53.696838pt}}
\pgfusepath{stroke}
\pgfpathmoveto{\pgfpoint{262.367981pt}{59.873672pt}}
\pgflineto{\pgfpoint{253.440002pt}{59.873672pt}}
\pgfusepath{stroke}
\pgfpathmoveto{\pgfpoint{262.367981pt}{66.050522pt}}
\pgflineto{\pgfpoint{253.440002pt}{66.050522pt}}
\pgfusepath{stroke}
\pgfpathmoveto{\pgfpoint{262.367981pt}{72.227356pt}}
\pgflineto{\pgfpoint{253.440002pt}{72.227356pt}}
\pgfusepath{stroke}
\pgfpathmoveto{\pgfpoint{262.367981pt}{78.404205pt}}
\pgflineto{\pgfpoint{253.440002pt}{78.404205pt}}
\pgfusepath{stroke}
\pgfpathmoveto{\pgfpoint{262.367981pt}{84.581039pt}}
\pgflineto{\pgfpoint{253.440002pt}{84.581039pt}}
\pgfusepath{stroke}
\pgfpathmoveto{\pgfpoint{262.367981pt}{90.757896pt}}
\pgflineto{\pgfpoint{253.440002pt}{90.757896pt}}
\pgfusepath{stroke}
\pgfpathmoveto{\pgfpoint{262.367981pt}{96.934731pt}}
\pgflineto{\pgfpoint{253.440002pt}{96.934731pt}}
\pgfusepath{stroke}
\pgfpathmoveto{\pgfpoint{262.367981pt}{103.111580pt}}
\pgflineto{\pgfpoint{253.440002pt}{103.111580pt}}
\pgfusepath{stroke}
\pgfpathmoveto{\pgfpoint{262.367981pt}{109.288422pt}}
\pgflineto{\pgfpoint{253.440002pt}{109.288422pt}}
\pgfusepath{stroke}
\pgfpathmoveto{\pgfpoint{262.367981pt}{115.465263pt}}
\pgflineto{\pgfpoint{253.440002pt}{115.465263pt}}
\pgfusepath{stroke}
\pgfpathmoveto{\pgfpoint{262.367981pt}{121.642097pt}}
\pgflineto{\pgfpoint{253.440002pt}{121.642097pt}}
\pgfusepath{stroke}
\pgfpathmoveto{\pgfpoint{262.367981pt}{127.818947pt}}
\pgflineto{\pgfpoint{253.440002pt}{127.818947pt}}
\pgfusepath{stroke}
\pgfpathmoveto{\pgfpoint{262.367981pt}{133.995789pt}}
\pgflineto{\pgfpoint{253.440002pt}{133.995789pt}}
\pgfusepath{stroke}
\pgfpathmoveto{\pgfpoint{262.367981pt}{140.172638pt}}
\pgflineto{\pgfpoint{253.440002pt}{140.172638pt}}
\pgfusepath{stroke}
\pgfpathmoveto{\pgfpoint{262.367981pt}{146.349472pt}}
\pgflineto{\pgfpoint{253.440002pt}{146.349472pt}}
\pgfusepath{stroke}
\pgfpathmoveto{\pgfpoint{262.367981pt}{152.526306pt}}
\pgflineto{\pgfpoint{253.440002pt}{152.526306pt}}
\pgfusepath{stroke}
\pgfpathmoveto{\pgfpoint{262.367981pt}{158.703156pt}}
\pgflineto{\pgfpoint{253.457840pt}{158.703156pt}}
\pgfusepath{stroke}
\pgfpathmoveto{\pgfpoint{262.367981pt}{164.880005pt}}
\pgflineto{\pgfpoint{253.457840pt}{164.880005pt}}
\pgfusepath{stroke}
\pgfpathmoveto{\pgfpoint{262.367981pt}{171.056854pt}}
\pgflineto{\pgfpoint{253.457809pt}{171.056854pt}}
\pgfusepath{stroke}
\pgfpathmoveto{\pgfpoint{262.367981pt}{177.233673pt}}
\pgflineto{\pgfpoint{253.457779pt}{177.233673pt}}
\pgfusepath{stroke}
\pgfpathmoveto{\pgfpoint{253.457779pt}{183.410522pt}}
\pgflineto{\pgfpoint{262.349854pt}{183.410522pt}}
\pgfusepath{stroke}
\pgfpathmoveto{\pgfpoint{253.466766pt}{189.587372pt}}
\pgflineto{\pgfpoint{262.340851pt}{189.587372pt}}
\pgfusepath{stroke}
\pgfpathmoveto{\pgfpoint{253.466736pt}{195.764206pt}}
\pgflineto{\pgfpoint{262.331787pt}{195.764206pt}}
\pgfusepath{stroke}
\pgfpathmoveto{\pgfpoint{253.466736pt}{201.941055pt}}
\pgflineto{\pgfpoint{262.331757pt}{201.941055pt}}
\pgfusepath{stroke}
\pgfpathmoveto{\pgfpoint{253.466736pt}{208.117905pt}}
\pgflineto{\pgfpoint{262.322723pt}{208.117905pt}}
\pgfusepath{stroke}
\pgfpathmoveto{\pgfpoint{253.475708pt}{214.294739pt}}
\pgflineto{\pgfpoint{262.313721pt}{214.294739pt}}
\pgfusepath{stroke}
\pgfpathmoveto{\pgfpoint{253.475693pt}{220.471588pt}}
\pgflineto{\pgfpoint{262.304688pt}{220.471588pt}}
\pgfusepath{stroke}
\pgfpathmoveto{\pgfpoint{253.471161pt}{226.648422pt}}
\pgflineto{\pgfpoint{262.295624pt}{226.648422pt}}
\pgfusepath{stroke}
\pgfpathmoveto{\pgfpoint{253.471161pt}{232.825272pt}}
\pgflineto{\pgfpoint{262.286591pt}{232.825272pt}}
\pgfusepath{stroke}
\pgfpathmoveto{\pgfpoint{253.475647pt}{239.002106pt}}
\pgflineto{\pgfpoint{262.282074pt}{239.002106pt}}
\pgfusepath{stroke}
\pgfpathmoveto{\pgfpoint{253.475586pt}{245.178955pt}}
\pgflineto{\pgfpoint{262.273041pt}{245.178955pt}}
\pgfusepath{stroke}
\pgfpathmoveto{\pgfpoint{253.480103pt}{251.355804pt}}
\pgflineto{\pgfpoint{262.268494pt}{251.355804pt}}
\pgfusepath{stroke}
\pgfpathmoveto{\pgfpoint{253.480057pt}{257.532623pt}}
\pgflineto{\pgfpoint{262.259430pt}{257.532623pt}}
\pgfusepath{stroke}
\pgfpathmoveto{\pgfpoint{253.480072pt}{263.709473pt}}
\pgflineto{\pgfpoint{262.254883pt}{263.709473pt}}
\pgfusepath{stroke}
\pgfpathmoveto{\pgfpoint{253.484543pt}{269.886322pt}}
\pgflineto{\pgfpoint{262.245911pt}{269.886322pt}}
\pgfusepath{stroke}
\pgfpathmoveto{\pgfpoint{253.484482pt}{276.063141pt}}
\pgflineto{\pgfpoint{262.236816pt}{276.063141pt}}
\pgfusepath{stroke}
\pgfpathmoveto{\pgfpoint{253.489014pt}{282.239990pt}}
\pgflineto{\pgfpoint{262.232300pt}{282.239990pt}}
\pgfusepath{stroke}
\pgfpathmoveto{\pgfpoint{253.489014pt}{288.416840pt}}
\pgflineto{\pgfpoint{262.223267pt}{288.416840pt}}
\pgfusepath{stroke}
\pgfpathmoveto{\pgfpoint{253.493500pt}{294.593689pt}}
\pgflineto{\pgfpoint{262.218750pt}{294.593689pt}}
\pgfusepath{stroke}
\pgfpathmoveto{\pgfpoint{253.493500pt}{300.770538pt}}
\pgflineto{\pgfpoint{262.209747pt}{300.770538pt}}
\pgfusepath{stroke}
\pgfpathmoveto{\pgfpoint{253.493469pt}{306.947388pt}}
\pgflineto{\pgfpoint{262.200714pt}{306.947388pt}}
\pgfusepath{stroke}
\pgfpathmoveto{\pgfpoint{253.497925pt}{313.124207pt}}
\pgflineto{\pgfpoint{262.196167pt}{313.124207pt}}
\pgfusepath{stroke}
\pgfpathmoveto{\pgfpoint{253.497925pt}{319.301056pt}}
\pgflineto{\pgfpoint{262.187134pt}{319.301056pt}}
\pgfusepath{stroke}
\pgfpathmoveto{\pgfpoint{253.502426pt}{325.477905pt}}
\pgflineto{\pgfpoint{262.182587pt}{325.477905pt}}
\pgfusepath{stroke}
\pgfpathmoveto{\pgfpoint{253.502365pt}{331.654724pt}}
\pgflineto{\pgfpoint{262.173523pt}{331.654724pt}}
\pgfusepath{stroke}
\pgfpathmoveto{\pgfpoint{262.367981pt}{146.349472pt}}
\pgflineto{\pgfpoint{262.367981pt}{140.172638pt}}
\pgfusepath{stroke}
\pgfpathmoveto{\pgfpoint{262.367981pt}{140.172638pt}}
\pgflineto{\pgfpoint{262.367981pt}{133.995789pt}}
\pgfusepath{stroke}
\pgfpathmoveto{\pgfpoint{262.367981pt}{152.526306pt}}
\pgflineto{\pgfpoint{262.367981pt}{146.349472pt}}
\pgfusepath{stroke}
\pgfpathmoveto{\pgfpoint{262.367981pt}{158.703156pt}}
\pgflineto{\pgfpoint{262.367981pt}{152.526306pt}}
\pgfusepath{stroke}
\pgfpathmoveto{\pgfpoint{262.367981pt}{140.172638pt}}
\pgflineto{\pgfpoint{262.385773pt}{140.172638pt}}
\pgfusepath{stroke}
\pgfpathmoveto{\pgfpoint{262.367981pt}{146.349472pt}}
\pgflineto{\pgfpoint{262.385742pt}{146.349472pt}}
\pgfusepath{stroke}
\pgfpathmoveto{\pgfpoint{262.367981pt}{152.526306pt}}
\pgflineto{\pgfpoint{262.385712pt}{152.526306pt}}
\pgfusepath{stroke}
\pgfpathmoveto{\pgfpoint{262.367981pt}{171.056854pt}}
\pgflineto{\pgfpoint{262.367981pt}{164.880005pt}}
\pgfusepath{stroke}
\pgfpathmoveto{\pgfpoint{262.367981pt}{164.880005pt}}
\pgflineto{\pgfpoint{262.367981pt}{158.703156pt}}
\pgfusepath{stroke}
\pgfpathmoveto{\pgfpoint{262.367981pt}{177.233673pt}}
\pgflineto{\pgfpoint{262.367981pt}{171.056854pt}}
\pgfusepath{stroke}
\pgfpathmoveto{\pgfpoint{262.367981pt}{158.703156pt}}
\pgflineto{\pgfpoint{262.385681pt}{158.703156pt}}
\pgfusepath{stroke}
\pgfpathmoveto{\pgfpoint{262.367981pt}{189.587372pt}}
\pgflineto{\pgfpoint{262.340851pt}{189.587372pt}}
\pgfusepath{stroke}
\pgfpathmoveto{\pgfpoint{262.367981pt}{195.764206pt}}
\pgflineto{\pgfpoint{262.331787pt}{195.764206pt}}
\pgfusepath{stroke}
\pgfpathmoveto{\pgfpoint{262.367981pt}{201.941055pt}}
\pgflineto{\pgfpoint{262.331757pt}{201.941055pt}}
\pgfusepath{stroke}
\pgfpathmoveto{\pgfpoint{262.367981pt}{208.117905pt}}
\pgflineto{\pgfpoint{262.322723pt}{208.117905pt}}
\pgfusepath{stroke}
\pgfpathmoveto{\pgfpoint{262.367981pt}{214.294739pt}}
\pgflineto{\pgfpoint{262.313721pt}{214.294739pt}}
\pgfusepath{stroke}
\pgfpathmoveto{\pgfpoint{262.367981pt}{220.471588pt}}
\pgflineto{\pgfpoint{262.304688pt}{220.471588pt}}
\pgfusepath{stroke}
\pgfpathmoveto{\pgfpoint{262.367981pt}{226.648422pt}}
\pgflineto{\pgfpoint{262.295624pt}{226.648422pt}}
\pgfusepath{stroke}
\pgfpathmoveto{\pgfpoint{262.367981pt}{232.825272pt}}
\pgflineto{\pgfpoint{262.286591pt}{232.825272pt}}
\pgfusepath{stroke}
\pgfpathmoveto{\pgfpoint{262.367981pt}{239.002106pt}}
\pgflineto{\pgfpoint{262.282074pt}{239.002106pt}}
\pgfusepath{stroke}
\pgfpathmoveto{\pgfpoint{262.367981pt}{245.178955pt}}
\pgflineto{\pgfpoint{262.273041pt}{245.178955pt}}
\pgfusepath{stroke}
\pgfpathmoveto{\pgfpoint{262.367981pt}{251.355804pt}}
\pgflineto{\pgfpoint{262.268494pt}{251.355804pt}}
\pgfusepath{stroke}
\pgfpathmoveto{\pgfpoint{262.367981pt}{257.532623pt}}
\pgflineto{\pgfpoint{262.259430pt}{257.532623pt}}
\pgfusepath{stroke}
\pgfpathmoveto{\pgfpoint{262.367981pt}{263.709473pt}}
\pgflineto{\pgfpoint{262.254883pt}{263.709473pt}}
\pgfusepath{stroke}
\pgfpathmoveto{\pgfpoint{262.367981pt}{269.886322pt}}
\pgflineto{\pgfpoint{262.245911pt}{269.886322pt}}
\pgfusepath{stroke}
\pgfpathmoveto{\pgfpoint{262.367981pt}{276.063141pt}}
\pgflineto{\pgfpoint{262.236816pt}{276.063141pt}}
\pgfusepath{stroke}
\pgfpathmoveto{\pgfpoint{262.367981pt}{282.239990pt}}
\pgflineto{\pgfpoint{262.232300pt}{282.239990pt}}
\pgfusepath{stroke}
\pgfpathmoveto{\pgfpoint{262.367981pt}{288.416840pt}}
\pgflineto{\pgfpoint{262.223267pt}{288.416840pt}}
\pgfusepath{stroke}
\pgfpathmoveto{\pgfpoint{262.367981pt}{294.593689pt}}
\pgflineto{\pgfpoint{262.218750pt}{294.593689pt}}
\pgfusepath{stroke}
\pgfpathmoveto{\pgfpoint{262.367981pt}{300.770538pt}}
\pgflineto{\pgfpoint{262.209747pt}{300.770538pt}}
\pgfusepath{stroke}
\pgfpathmoveto{\pgfpoint{262.367981pt}{306.947388pt}}
\pgflineto{\pgfpoint{262.200714pt}{306.947388pt}}
\pgfusepath{stroke}
\pgfpathmoveto{\pgfpoint{262.367981pt}{313.124207pt}}
\pgflineto{\pgfpoint{262.196167pt}{313.124207pt}}
\pgfusepath{stroke}
\pgfpathmoveto{\pgfpoint{262.367981pt}{319.301056pt}}
\pgflineto{\pgfpoint{262.187134pt}{319.301056pt}}
\pgfusepath{stroke}
\pgfpathmoveto{\pgfpoint{262.367981pt}{325.477905pt}}
\pgflineto{\pgfpoint{262.182587pt}{325.477905pt}}
\pgfusepath{stroke}
\pgfpathmoveto{\pgfpoint{262.367981pt}{331.654724pt}}
\pgflineto{\pgfpoint{262.173523pt}{331.654724pt}}
\pgfusepath{stroke}
\pgfpathmoveto{\pgfpoint{262.367981pt}{269.886322pt}}
\pgflineto{\pgfpoint{262.367981pt}{263.709473pt}}
\pgfusepath{stroke}
\pgfpathmoveto{\pgfpoint{262.367981pt}{263.709473pt}}
\pgflineto{\pgfpoint{262.367981pt}{257.532623pt}}
\pgfusepath{stroke}
\pgfpathmoveto{\pgfpoint{262.367981pt}{276.063141pt}}
\pgflineto{\pgfpoint{262.367981pt}{269.886322pt}}
\pgfusepath{stroke}
\pgfpathmoveto{\pgfpoint{262.367981pt}{282.239990pt}}
\pgflineto{\pgfpoint{262.367981pt}{276.063141pt}}
\pgfusepath{stroke}
\pgfpathmoveto{\pgfpoint{262.367981pt}{288.416840pt}}
\pgflineto{\pgfpoint{262.367981pt}{282.239990pt}}
\pgfusepath{stroke}
\pgfpathmoveto{\pgfpoint{262.367981pt}{263.709473pt}}
\pgflineto{\pgfpoint{262.381409pt}{263.709473pt}}
\pgfusepath{stroke}
\pgfpathmoveto{\pgfpoint{262.367981pt}{269.886322pt}}
\pgflineto{\pgfpoint{262.381409pt}{269.886322pt}}
\pgfusepath{stroke}
\pgfpathmoveto{\pgfpoint{262.367981pt}{276.063141pt}}
\pgflineto{\pgfpoint{262.381348pt}{276.063141pt}}
\pgfusepath{stroke}
\pgfpathmoveto{\pgfpoint{262.367981pt}{282.239990pt}}
\pgflineto{\pgfpoint{262.385864pt}{282.239990pt}}
\pgfusepath{stroke}
\pgfpathmoveto{\pgfpoint{262.367981pt}{313.124207pt}}
\pgflineto{\pgfpoint{262.367981pt}{306.947388pt}}
\pgfusepath{stroke}
\pgfpathmoveto{\pgfpoint{262.367981pt}{294.593689pt}}
\pgflineto{\pgfpoint{262.367981pt}{288.416840pt}}
\pgfusepath{stroke}
\pgfpathmoveto{\pgfpoint{262.367981pt}{300.770538pt}}
\pgflineto{\pgfpoint{262.367981pt}{294.593689pt}}
\pgfusepath{stroke}
\pgfpathmoveto{\pgfpoint{262.367981pt}{306.947388pt}}
\pgflineto{\pgfpoint{262.367981pt}{300.770538pt}}
\pgfusepath{stroke}
\pgfpathmoveto{\pgfpoint{262.367981pt}{319.301056pt}}
\pgflineto{\pgfpoint{262.367981pt}{313.124207pt}}
\pgfusepath{stroke}
\pgfpathmoveto{\pgfpoint{262.367981pt}{325.477905pt}}
\pgflineto{\pgfpoint{262.367981pt}{319.301056pt}}
\pgfusepath{stroke}
\pgfpathmoveto{\pgfpoint{262.367981pt}{331.654724pt}}
\pgflineto{\pgfpoint{262.367981pt}{325.477905pt}}
\pgfusepath{stroke}
\pgfpathmoveto{\pgfpoint{262.367981pt}{288.416840pt}}
\pgflineto{\pgfpoint{262.385864pt}{288.416840pt}}
\pgfusepath{stroke}
\pgfpathmoveto{\pgfpoint{262.367981pt}{294.593689pt}}
\pgflineto{\pgfpoint{262.390381pt}{294.593689pt}}
\pgfusepath{stroke}
\pgfpathmoveto{\pgfpoint{262.367981pt}{300.770538pt}}
\pgflineto{\pgfpoint{262.390381pt}{300.770538pt}}
\pgfusepath{stroke}
\pgfpathmoveto{\pgfpoint{262.367981pt}{306.947388pt}}
\pgflineto{\pgfpoint{262.390381pt}{306.947388pt}}
\pgfusepath{stroke}
\pgfpathmoveto{\pgfpoint{262.367981pt}{313.124207pt}}
\pgflineto{\pgfpoint{262.394867pt}{313.124207pt}}
\pgfusepath{stroke}
\pgfpathmoveto{\pgfpoint{262.367981pt}{319.301056pt}}
\pgflineto{\pgfpoint{262.394836pt}{319.301056pt}}
\pgfusepath{stroke}
\pgfpathmoveto{\pgfpoint{262.367981pt}{325.477905pt}}
\pgflineto{\pgfpoint{262.399353pt}{325.477905pt}}
\pgfusepath{stroke}
\pgfpathmoveto{\pgfpoint{262.367981pt}{208.117905pt}}
\pgflineto{\pgfpoint{262.367981pt}{201.941055pt}}
\pgfusepath{stroke}
\pgfpathmoveto{\pgfpoint{262.367981pt}{183.410522pt}}
\pgflineto{\pgfpoint{262.367981pt}{177.233673pt}}
\pgfusepath{stroke}
\pgfpathmoveto{\pgfpoint{262.367981pt}{189.587372pt}}
\pgflineto{\pgfpoint{262.367981pt}{183.410522pt}}
\pgfusepath{stroke}
\pgfpathmoveto{\pgfpoint{262.367981pt}{195.764206pt}}
\pgflineto{\pgfpoint{262.367981pt}{189.587372pt}}
\pgfusepath{stroke}
\pgfpathmoveto{\pgfpoint{262.367981pt}{201.941055pt}}
\pgflineto{\pgfpoint{262.367981pt}{195.764206pt}}
\pgfusepath{stroke}
\pgfpathmoveto{\pgfpoint{262.367981pt}{214.294739pt}}
\pgflineto{\pgfpoint{262.367981pt}{208.117905pt}}
\pgfusepath{stroke}
\pgfpathmoveto{\pgfpoint{262.367981pt}{220.471588pt}}
\pgflineto{\pgfpoint{262.367981pt}{214.294739pt}}
\pgfusepath{stroke}
\pgfpathmoveto{\pgfpoint{262.367981pt}{226.648422pt}}
\pgflineto{\pgfpoint{262.367981pt}{220.471588pt}}
\pgfusepath{stroke}
\pgfpathmoveto{\pgfpoint{262.367981pt}{232.825272pt}}
\pgflineto{\pgfpoint{262.367981pt}{226.648422pt}}
\pgfusepath{stroke}
\pgfpathmoveto{\pgfpoint{262.367981pt}{239.002106pt}}
\pgflineto{\pgfpoint{262.367981pt}{232.825272pt}}
\pgfusepath{stroke}
\pgfpathmoveto{\pgfpoint{262.367981pt}{245.178955pt}}
\pgflineto{\pgfpoint{262.367981pt}{239.002106pt}}
\pgfusepath{stroke}
\pgfpathmoveto{\pgfpoint{262.367981pt}{251.355804pt}}
\pgflineto{\pgfpoint{262.367981pt}{245.178955pt}}
\pgfusepath{stroke}
\pgfpathmoveto{\pgfpoint{262.367981pt}{257.532623pt}}
\pgflineto{\pgfpoint{262.367981pt}{251.355804pt}}
\pgfusepath{stroke}
\pgfpathmoveto{\pgfpoint{262.367981pt}{183.410522pt}}
\pgflineto{\pgfpoint{262.349854pt}{183.410522pt}}
\pgfusepath{stroke}
\pgfpathmoveto{\pgfpoint{262.367981pt}{164.880005pt}}
\pgflineto{\pgfpoint{262.385681pt}{164.880005pt}}
\pgfusepath{stroke}
\pgfpathmoveto{\pgfpoint{262.367981pt}{171.056854pt}}
\pgflineto{\pgfpoint{262.394623pt}{171.056854pt}}
\pgfusepath{stroke}
\pgfpathmoveto{\pgfpoint{262.367981pt}{177.233673pt}}
\pgflineto{\pgfpoint{262.394562pt}{177.233673pt}}
\pgfusepath{stroke}
\pgfpathmoveto{\pgfpoint{262.367981pt}{183.410522pt}}
\pgflineto{\pgfpoint{262.394592pt}{183.410522pt}}
\pgfusepath{stroke}
\pgfpathmoveto{\pgfpoint{262.367981pt}{189.587372pt}}
\pgflineto{\pgfpoint{262.394562pt}{189.587372pt}}
\pgfusepath{stroke}
\pgfpathmoveto{\pgfpoint{262.367981pt}{195.764206pt}}
\pgflineto{\pgfpoint{262.394531pt}{195.764206pt}}
\pgfusepath{stroke}
\pgfpathmoveto{\pgfpoint{262.367981pt}{201.941055pt}}
\pgflineto{\pgfpoint{262.403503pt}{201.941055pt}}
\pgfusepath{stroke}
\pgfpathmoveto{\pgfpoint{262.367981pt}{208.117905pt}}
\pgflineto{\pgfpoint{262.403503pt}{208.117905pt}}
\pgfusepath{stroke}
\pgfpathmoveto{\pgfpoint{262.367981pt}{214.294739pt}}
\pgflineto{\pgfpoint{262.403442pt}{214.294739pt}}
\pgfusepath{stroke}
\pgfpathmoveto{\pgfpoint{262.367981pt}{220.471588pt}}
\pgflineto{\pgfpoint{262.403442pt}{220.471588pt}}
\pgfusepath{stroke}
\pgfpathmoveto{\pgfpoint{262.367981pt}{226.648422pt}}
\pgflineto{\pgfpoint{262.403381pt}{226.648422pt}}
\pgfusepath{stroke}
\pgfpathmoveto{\pgfpoint{262.367981pt}{232.825272pt}}
\pgflineto{\pgfpoint{262.403412pt}{232.825272pt}}
\pgfusepath{stroke}
\pgfpathmoveto{\pgfpoint{262.367981pt}{239.002106pt}}
\pgflineto{\pgfpoint{262.407837pt}{239.002106pt}}
\pgfusepath{stroke}
\pgfpathmoveto{\pgfpoint{262.367981pt}{245.178955pt}}
\pgflineto{\pgfpoint{262.407837pt}{245.178955pt}}
\pgfusepath{stroke}
\pgfpathmoveto{\pgfpoint{262.367981pt}{251.355804pt}}
\pgflineto{\pgfpoint{262.412323pt}{251.355804pt}}
\pgfusepath{stroke}
\pgfpathmoveto{\pgfpoint{262.367981pt}{257.532623pt}}
\pgflineto{\pgfpoint{262.412262pt}{257.532623pt}}
\pgfusepath{stroke}
\pgfpathmoveto{\pgfpoint{262.416748pt}{263.709473pt}}
\pgflineto{\pgfpoint{262.381409pt}{263.709473pt}}
\pgfusepath{stroke}
\pgfpathmoveto{\pgfpoint{262.416748pt}{269.886322pt}}
\pgflineto{\pgfpoint{262.381409pt}{269.886322pt}}
\pgfusepath{stroke}
\pgfpathmoveto{\pgfpoint{262.416687pt}{276.063141pt}}
\pgflineto{\pgfpoint{262.381348pt}{276.063141pt}}
\pgfusepath{stroke}
\pgfpathmoveto{\pgfpoint{262.421173pt}{282.239990pt}}
\pgflineto{\pgfpoint{262.385864pt}{282.239990pt}}
\pgfusepath{stroke}
\pgfpathmoveto{\pgfpoint{262.421143pt}{288.416840pt}}
\pgflineto{\pgfpoint{262.385864pt}{288.416840pt}}
\pgfusepath{stroke}
\pgfpathmoveto{\pgfpoint{262.425629pt}{294.593689pt}}
\pgflineto{\pgfpoint{262.390381pt}{294.593689pt}}
\pgfusepath{stroke}
\pgfpathmoveto{\pgfpoint{262.425598pt}{300.770538pt}}
\pgflineto{\pgfpoint{262.390381pt}{300.770538pt}}
\pgfusepath{stroke}
\pgfpathmoveto{\pgfpoint{262.425598pt}{306.947388pt}}
\pgflineto{\pgfpoint{262.390381pt}{306.947388pt}}
\pgfusepath{stroke}
\pgfpathmoveto{\pgfpoint{262.430023pt}{313.124207pt}}
\pgflineto{\pgfpoint{262.394867pt}{313.124207pt}}
\pgfusepath{stroke}
\pgfpathmoveto{\pgfpoint{262.429993pt}{319.301056pt}}
\pgflineto{\pgfpoint{262.394836pt}{319.301056pt}}
\pgfusepath{stroke}
\pgfpathmoveto{\pgfpoint{262.434509pt}{325.477905pt}}
\pgflineto{\pgfpoint{262.399353pt}{325.477905pt}}
\pgfusepath{stroke}
\pgfpathmoveto{\pgfpoint{262.367981pt}{90.757896pt}}
\pgflineto{\pgfpoint{262.367981pt}{84.581039pt}}
\pgfusepath{stroke}
\pgfpathmoveto{\pgfpoint{262.367981pt}{59.873672pt}}
\pgflineto{\pgfpoint{262.367981pt}{53.696838pt}}
\pgfusepath{stroke}
\pgfpathmoveto{\pgfpoint{262.367981pt}{66.050522pt}}
\pgflineto{\pgfpoint{262.367981pt}{59.873672pt}}
\pgfusepath{stroke}
\pgfpathmoveto{\pgfpoint{262.367981pt}{72.227356pt}}
\pgflineto{\pgfpoint{262.367981pt}{66.050522pt}}
\pgfusepath{stroke}
\pgfpathmoveto{\pgfpoint{262.367981pt}{78.404205pt}}
\pgflineto{\pgfpoint{262.367981pt}{72.227356pt}}
\pgfusepath{stroke}
\pgfpathmoveto{\pgfpoint{262.367981pt}{84.581039pt}}
\pgflineto{\pgfpoint{262.367981pt}{78.404205pt}}
\pgfusepath{stroke}
\pgfpathmoveto{\pgfpoint{262.367981pt}{96.934731pt}}
\pgflineto{\pgfpoint{262.367981pt}{90.757896pt}}
\pgfusepath{stroke}
\pgfpathmoveto{\pgfpoint{262.367981pt}{103.111580pt}}
\pgflineto{\pgfpoint{262.367981pt}{96.934731pt}}
\pgfusepath{stroke}
\pgfpathmoveto{\pgfpoint{262.367981pt}{109.288422pt}}
\pgflineto{\pgfpoint{262.367981pt}{103.111580pt}}
\pgfusepath{stroke}
\pgfpathmoveto{\pgfpoint{262.367981pt}{115.465263pt}}
\pgflineto{\pgfpoint{262.367981pt}{109.288422pt}}
\pgfusepath{stroke}
\pgfpathmoveto{\pgfpoint{262.367981pt}{121.642097pt}}
\pgflineto{\pgfpoint{262.367981pt}{115.465263pt}}
\pgfusepath{stroke}
\pgfpathmoveto{\pgfpoint{262.367981pt}{127.818947pt}}
\pgflineto{\pgfpoint{262.367981pt}{121.642097pt}}
\pgfusepath{stroke}
\pgfpathmoveto{\pgfpoint{262.367981pt}{133.995789pt}}
\pgflineto{\pgfpoint{262.367981pt}{127.818947pt}}
\pgfusepath{stroke}
\pgfpathmoveto{\pgfpoint{262.367981pt}{47.519989pt}}
\pgflineto{\pgfpoint{262.350098pt}{47.519989pt}}
\pgfusepath{stroke}
\pgfpathmoveto{\pgfpoint{262.367981pt}{53.696838pt}}
\pgflineto{\pgfpoint{262.367981pt}{47.519989pt}}
\pgfusepath{stroke}
\pgfpathmoveto{\pgfpoint{262.367981pt}{47.519989pt}}
\pgflineto{\pgfpoint{271.277893pt}{47.519989pt}}
\pgfusepath{stroke}
\pgfpathmoveto{\pgfpoint{262.367981pt}{53.696838pt}}
\pgflineto{\pgfpoint{271.277924pt}{53.696838pt}}
\pgfusepath{stroke}
\pgfpathmoveto{\pgfpoint{262.367981pt}{59.873672pt}}
\pgflineto{\pgfpoint{271.277924pt}{59.873672pt}}
\pgfusepath{stroke}
\pgfpathmoveto{\pgfpoint{271.295990pt}{66.050522pt}}
\pgflineto{\pgfpoint{262.367981pt}{66.050522pt}}
\pgfusepath{stroke}
\pgfpathmoveto{\pgfpoint{271.295990pt}{72.227356pt}}
\pgflineto{\pgfpoint{262.367981pt}{72.227356pt}}
\pgfusepath{stroke}
\pgfpathmoveto{\pgfpoint{271.295990pt}{78.404205pt}}
\pgflineto{\pgfpoint{262.367981pt}{78.404205pt}}
\pgfusepath{stroke}
\pgfpathmoveto{\pgfpoint{271.295990pt}{84.581039pt}}
\pgflineto{\pgfpoint{262.367981pt}{84.581039pt}}
\pgfusepath{stroke}
\pgfpathmoveto{\pgfpoint{271.295990pt}{90.757896pt}}
\pgflineto{\pgfpoint{262.367981pt}{90.757896pt}}
\pgfusepath{stroke}
\pgfpathmoveto{\pgfpoint{271.295990pt}{96.934731pt}}
\pgflineto{\pgfpoint{262.367981pt}{96.934731pt}}
\pgfusepath{stroke}
\pgfpathmoveto{\pgfpoint{271.295990pt}{103.111580pt}}
\pgflineto{\pgfpoint{262.367981pt}{103.111580pt}}
\pgfusepath{stroke}
\pgfpathmoveto{\pgfpoint{271.295990pt}{109.288422pt}}
\pgflineto{\pgfpoint{262.367981pt}{109.288422pt}}
\pgfusepath{stroke}
\pgfpathmoveto{\pgfpoint{271.295990pt}{115.465263pt}}
\pgflineto{\pgfpoint{262.367981pt}{115.465263pt}}
\pgfusepath{stroke}
\pgfpathmoveto{\pgfpoint{271.295990pt}{121.642097pt}}
\pgflineto{\pgfpoint{262.367981pt}{121.642097pt}}
\pgfusepath{stroke}
\pgfpathmoveto{\pgfpoint{271.295990pt}{127.818947pt}}
\pgflineto{\pgfpoint{262.367981pt}{127.818947pt}}
\pgfusepath{stroke}
\pgfpathmoveto{\pgfpoint{271.295990pt}{133.995789pt}}
\pgflineto{\pgfpoint{262.367981pt}{133.995789pt}}
\pgfusepath{stroke}
\pgfpathmoveto{\pgfpoint{271.295990pt}{140.172638pt}}
\pgflineto{\pgfpoint{262.385773pt}{140.172638pt}}
\pgfusepath{stroke}
\pgfpathmoveto{\pgfpoint{271.295990pt}{146.349472pt}}
\pgflineto{\pgfpoint{262.385742pt}{146.349472pt}}
\pgfusepath{stroke}
\pgfpathmoveto{\pgfpoint{271.295990pt}{152.526306pt}}
\pgflineto{\pgfpoint{262.385712pt}{152.526306pt}}
\pgfusepath{stroke}
\pgfpathmoveto{\pgfpoint{271.295990pt}{158.703156pt}}
\pgflineto{\pgfpoint{262.385681pt}{158.703156pt}}
\pgfusepath{stroke}
\pgfpathmoveto{\pgfpoint{271.295990pt}{164.880005pt}}
\pgflineto{\pgfpoint{262.385681pt}{164.880005pt}}
\pgfusepath{stroke}
\pgfpathmoveto{\pgfpoint{271.295990pt}{171.056854pt}}
\pgflineto{\pgfpoint{262.394623pt}{171.056854pt}}
\pgfusepath{stroke}
\pgfpathmoveto{\pgfpoint{271.295990pt}{177.233673pt}}
\pgflineto{\pgfpoint{262.394562pt}{177.233673pt}}
\pgfusepath{stroke}
\pgfpathmoveto{\pgfpoint{271.295990pt}{183.410522pt}}
\pgflineto{\pgfpoint{262.394592pt}{183.410522pt}}
\pgfusepath{stroke}
\pgfpathmoveto{\pgfpoint{271.295990pt}{189.587372pt}}
\pgflineto{\pgfpoint{262.394562pt}{189.587372pt}}
\pgfusepath{stroke}
\pgfpathmoveto{\pgfpoint{262.394531pt}{195.764206pt}}
\pgflineto{\pgfpoint{271.277924pt}{195.764206pt}}
\pgfusepath{stroke}
\pgfpathmoveto{\pgfpoint{262.403503pt}{201.941055pt}}
\pgflineto{\pgfpoint{271.268921pt}{201.941055pt}}
\pgfusepath{stroke}
\pgfpathmoveto{\pgfpoint{262.403503pt}{208.117905pt}}
\pgflineto{\pgfpoint{271.259857pt}{208.117905pt}}
\pgfusepath{stroke}
\pgfpathmoveto{\pgfpoint{262.403442pt}{214.294739pt}}
\pgflineto{\pgfpoint{271.250854pt}{214.294739pt}}
\pgfusepath{stroke}
\pgfpathmoveto{\pgfpoint{262.403442pt}{220.471588pt}}
\pgflineto{\pgfpoint{271.246307pt}{220.471588pt}}
\pgfusepath{stroke}
\pgfpathmoveto{\pgfpoint{262.403381pt}{226.648422pt}}
\pgflineto{\pgfpoint{271.237244pt}{226.648422pt}}
\pgfusepath{stroke}
\pgfpathmoveto{\pgfpoint{262.403412pt}{232.825272pt}}
\pgflineto{\pgfpoint{271.232697pt}{232.825272pt}}
\pgfusepath{stroke}
\pgfpathmoveto{\pgfpoint{262.407837pt}{239.002106pt}}
\pgflineto{\pgfpoint{271.223694pt}{239.002106pt}}
\pgfusepath{stroke}
\pgfpathmoveto{\pgfpoint{262.407837pt}{245.178955pt}}
\pgflineto{\pgfpoint{271.214630pt}{245.178955pt}}
\pgfusepath{stroke}
\pgfpathmoveto{\pgfpoint{262.412323pt}{251.355804pt}}
\pgflineto{\pgfpoint{271.210144pt}{251.355804pt}}
\pgfusepath{stroke}
\pgfpathmoveto{\pgfpoint{262.412262pt}{257.532623pt}}
\pgflineto{\pgfpoint{271.201050pt}{257.532623pt}}
\pgfusepath{stroke}
\pgfpathmoveto{\pgfpoint{262.416748pt}{263.709473pt}}
\pgflineto{\pgfpoint{271.196564pt}{263.709473pt}}
\pgfusepath{stroke}
\pgfpathmoveto{\pgfpoint{262.416748pt}{269.886322pt}}
\pgflineto{\pgfpoint{271.187500pt}{269.886322pt}}
\pgfusepath{stroke}
\pgfpathmoveto{\pgfpoint{262.416687pt}{276.063141pt}}
\pgflineto{\pgfpoint{271.178436pt}{276.063141pt}}
\pgfusepath{stroke}
\pgfpathmoveto{\pgfpoint{262.421173pt}{282.239990pt}}
\pgflineto{\pgfpoint{271.173920pt}{282.239990pt}}
\pgfusepath{stroke}
\pgfpathmoveto{\pgfpoint{262.421143pt}{288.416840pt}}
\pgflineto{\pgfpoint{271.164886pt}{288.416840pt}}
\pgfusepath{stroke}
\pgfpathmoveto{\pgfpoint{262.425629pt}{294.593689pt}}
\pgflineto{\pgfpoint{271.160370pt}{294.593689pt}}
\pgfusepath{stroke}
\pgfpathmoveto{\pgfpoint{262.425598pt}{300.770538pt}}
\pgflineto{\pgfpoint{271.151367pt}{300.770538pt}}
\pgfusepath{stroke}
\pgfpathmoveto{\pgfpoint{262.425598pt}{306.947388pt}}
\pgflineto{\pgfpoint{271.146759pt}{306.947388pt}}
\pgfusepath{stroke}
\pgfpathmoveto{\pgfpoint{262.430023pt}{313.124207pt}}
\pgflineto{\pgfpoint{271.137787pt}{313.124207pt}}
\pgfusepath{stroke}
\pgfpathmoveto{\pgfpoint{262.429993pt}{319.301056pt}}
\pgflineto{\pgfpoint{271.128754pt}{319.301056pt}}
\pgfusepath{stroke}
\pgfpathmoveto{\pgfpoint{262.434509pt}{325.477905pt}}
\pgflineto{\pgfpoint{271.124207pt}{325.477905pt}}
\pgfusepath{stroke}
\pgfpathmoveto{\pgfpoint{271.295990pt}{158.703156pt}}
\pgflineto{\pgfpoint{271.295990pt}{152.526306pt}}
\pgfusepath{stroke}
\pgfpathmoveto{\pgfpoint{271.295990pt}{152.526306pt}}
\pgflineto{\pgfpoint{271.295990pt}{146.349472pt}}
\pgfusepath{stroke}
\pgfpathmoveto{\pgfpoint{271.295990pt}{164.880005pt}}
\pgflineto{\pgfpoint{271.295990pt}{158.703156pt}}
\pgfusepath{stroke}
\pgfpathmoveto{\pgfpoint{271.295990pt}{171.056854pt}}
\pgflineto{\pgfpoint{271.295990pt}{164.880005pt}}
\pgfusepath{stroke}
\pgfpathmoveto{\pgfpoint{271.295990pt}{152.526306pt}}
\pgflineto{\pgfpoint{271.314026pt}{152.526306pt}}
\pgfusepath{stroke}
\pgfpathmoveto{\pgfpoint{271.295990pt}{158.703156pt}}
\pgflineto{\pgfpoint{271.314087pt}{158.703156pt}}
\pgfusepath{stroke}
\pgfpathmoveto{\pgfpoint{271.295990pt}{164.880005pt}}
\pgflineto{\pgfpoint{271.314117pt}{164.880005pt}}
\pgfusepath{stroke}
\pgfpathmoveto{\pgfpoint{271.295990pt}{183.410522pt}}
\pgflineto{\pgfpoint{271.295990pt}{177.233673pt}}
\pgfusepath{stroke}
\pgfpathmoveto{\pgfpoint{271.295990pt}{177.233673pt}}
\pgflineto{\pgfpoint{271.295990pt}{171.056854pt}}
\pgfusepath{stroke}
\pgfpathmoveto{\pgfpoint{271.295990pt}{189.587372pt}}
\pgflineto{\pgfpoint{271.295990pt}{183.410522pt}}
\pgfusepath{stroke}
\pgfpathmoveto{\pgfpoint{271.295990pt}{171.056854pt}}
\pgflineto{\pgfpoint{271.314117pt}{171.056854pt}}
\pgfusepath{stroke}
\pgfpathmoveto{\pgfpoint{271.295990pt}{201.941055pt}}
\pgflineto{\pgfpoint{271.268921pt}{201.941055pt}}
\pgfusepath{stroke}
\pgfpathmoveto{\pgfpoint{271.295990pt}{208.117905pt}}
\pgflineto{\pgfpoint{271.259857pt}{208.117905pt}}
\pgfusepath{stroke}
\pgfpathmoveto{\pgfpoint{271.295990pt}{214.294739pt}}
\pgflineto{\pgfpoint{271.250854pt}{214.294739pt}}
\pgfusepath{stroke}
\pgfpathmoveto{\pgfpoint{271.295990pt}{220.471588pt}}
\pgflineto{\pgfpoint{271.246307pt}{220.471588pt}}
\pgfusepath{stroke}
\pgfpathmoveto{\pgfpoint{271.295990pt}{226.648422pt}}
\pgflineto{\pgfpoint{271.237244pt}{226.648422pt}}
\pgfusepath{stroke}
\pgfpathmoveto{\pgfpoint{271.295990pt}{232.825272pt}}
\pgflineto{\pgfpoint{271.232697pt}{232.825272pt}}
\pgfusepath{stroke}
\pgfpathmoveto{\pgfpoint{271.295990pt}{239.002106pt}}
\pgflineto{\pgfpoint{271.223694pt}{239.002106pt}}
\pgfusepath{stroke}
\pgfpathmoveto{\pgfpoint{271.295990pt}{245.178955pt}}
\pgflineto{\pgfpoint{271.214630pt}{245.178955pt}}
\pgfusepath{stroke}
\pgfpathmoveto{\pgfpoint{271.295990pt}{251.355804pt}}
\pgflineto{\pgfpoint{271.210144pt}{251.355804pt}}
\pgfusepath{stroke}
\pgfpathmoveto{\pgfpoint{271.295990pt}{257.532623pt}}
\pgflineto{\pgfpoint{271.201050pt}{257.532623pt}}
\pgfusepath{stroke}
\pgfpathmoveto{\pgfpoint{271.295990pt}{263.709473pt}}
\pgflineto{\pgfpoint{271.196564pt}{263.709473pt}}
\pgfusepath{stroke}
\pgfpathmoveto{\pgfpoint{271.295990pt}{269.886322pt}}
\pgflineto{\pgfpoint{271.187500pt}{269.886322pt}}
\pgfusepath{stroke}
\pgfpathmoveto{\pgfpoint{271.295990pt}{276.063141pt}}
\pgflineto{\pgfpoint{271.178436pt}{276.063141pt}}
\pgfusepath{stroke}
\pgfpathmoveto{\pgfpoint{271.295990pt}{282.239990pt}}
\pgflineto{\pgfpoint{271.173920pt}{282.239990pt}}
\pgfusepath{stroke}
\pgfpathmoveto{\pgfpoint{271.295990pt}{288.416840pt}}
\pgflineto{\pgfpoint{271.164886pt}{288.416840pt}}
\pgfusepath{stroke}
\pgfpathmoveto{\pgfpoint{271.295990pt}{294.593689pt}}
\pgflineto{\pgfpoint{271.160370pt}{294.593689pt}}
\pgfusepath{stroke}
\pgfpathmoveto{\pgfpoint{271.295990pt}{300.770538pt}}
\pgflineto{\pgfpoint{271.151367pt}{300.770538pt}}
\pgfusepath{stroke}
\pgfpathmoveto{\pgfpoint{271.295990pt}{306.947388pt}}
\pgflineto{\pgfpoint{271.146759pt}{306.947388pt}}
\pgfusepath{stroke}
\pgfpathmoveto{\pgfpoint{271.295990pt}{313.124207pt}}
\pgflineto{\pgfpoint{271.137787pt}{313.124207pt}}
\pgfusepath{stroke}
\pgfpathmoveto{\pgfpoint{271.295990pt}{319.301056pt}}
\pgflineto{\pgfpoint{271.128754pt}{319.301056pt}}
\pgfusepath{stroke}
\pgfpathmoveto{\pgfpoint{271.295990pt}{325.477905pt}}
\pgflineto{\pgfpoint{271.124207pt}{325.477905pt}}
\pgfusepath{stroke}
\pgfpathmoveto{\pgfpoint{271.295990pt}{306.947388pt}}
\pgflineto{\pgfpoint{271.295990pt}{300.770538pt}}
\pgfusepath{stroke}
\pgfpathmoveto{\pgfpoint{271.295990pt}{282.239990pt}}
\pgflineto{\pgfpoint{271.295990pt}{276.063141pt}}
\pgfusepath{stroke}
\pgfpathmoveto{\pgfpoint{271.295990pt}{288.416840pt}}
\pgflineto{\pgfpoint{271.295990pt}{282.239990pt}}
\pgfusepath{stroke}
\pgfpathmoveto{\pgfpoint{271.295990pt}{294.593689pt}}
\pgflineto{\pgfpoint{271.295990pt}{288.416840pt}}
\pgfusepath{stroke}
\pgfpathmoveto{\pgfpoint{271.295990pt}{300.770538pt}}
\pgflineto{\pgfpoint{271.295990pt}{294.593689pt}}
\pgfusepath{stroke}
\pgfpathmoveto{\pgfpoint{271.295990pt}{313.124207pt}}
\pgflineto{\pgfpoint{271.295990pt}{306.947388pt}}
\pgfusepath{stroke}
\pgfpathmoveto{\pgfpoint{271.295990pt}{319.301056pt}}
\pgflineto{\pgfpoint{271.295990pt}{313.124207pt}}
\pgfusepath{stroke}
\pgfpathmoveto{\pgfpoint{271.295990pt}{325.477905pt}}
\pgflineto{\pgfpoint{271.295990pt}{319.301056pt}}
\pgfusepath{stroke}
\pgfpathmoveto{\pgfpoint{271.295990pt}{282.239990pt}}
\pgflineto{\pgfpoint{271.309692pt}{282.239990pt}}
\pgfusepath{stroke}
\pgfpathmoveto{\pgfpoint{271.295990pt}{288.416840pt}}
\pgflineto{\pgfpoint{271.309723pt}{288.416840pt}}
\pgfusepath{stroke}
\pgfpathmoveto{\pgfpoint{271.295990pt}{294.593689pt}}
\pgflineto{\pgfpoint{271.314240pt}{294.593689pt}}
\pgfusepath{stroke}
\pgfpathmoveto{\pgfpoint{271.295990pt}{300.770538pt}}
\pgflineto{\pgfpoint{271.314270pt}{300.770538pt}}
\pgfusepath{stroke}
\pgfpathmoveto{\pgfpoint{271.295990pt}{306.947388pt}}
\pgflineto{\pgfpoint{271.318817pt}{306.947388pt}}
\pgfusepath{stroke}
\pgfpathmoveto{\pgfpoint{271.295990pt}{313.124207pt}}
\pgflineto{\pgfpoint{271.318817pt}{313.124207pt}}
\pgfusepath{stroke}
\pgfpathmoveto{\pgfpoint{271.295990pt}{319.301056pt}}
\pgflineto{\pgfpoint{271.318878pt}{319.301056pt}}
\pgfusepath{stroke}
\pgfpathmoveto{\pgfpoint{271.295990pt}{239.002106pt}}
\pgflineto{\pgfpoint{271.295990pt}{232.825272pt}}
\pgfusepath{stroke}
\pgfpathmoveto{\pgfpoint{271.295990pt}{195.764206pt}}
\pgflineto{\pgfpoint{271.295990pt}{189.587372pt}}
\pgfusepath{stroke}
\pgfpathmoveto{\pgfpoint{271.295990pt}{201.941055pt}}
\pgflineto{\pgfpoint{271.295990pt}{195.764206pt}}
\pgfusepath{stroke}
\pgfpathmoveto{\pgfpoint{271.295990pt}{208.117905pt}}
\pgflineto{\pgfpoint{271.295990pt}{201.941055pt}}
\pgfusepath{stroke}
\pgfpathmoveto{\pgfpoint{271.295990pt}{214.294739pt}}
\pgflineto{\pgfpoint{271.295990pt}{208.117905pt}}
\pgfusepath{stroke}
\pgfpathmoveto{\pgfpoint{271.295990pt}{220.471588pt}}
\pgflineto{\pgfpoint{271.295990pt}{214.294739pt}}
\pgfusepath{stroke}
\pgfpathmoveto{\pgfpoint{271.295990pt}{226.648422pt}}
\pgflineto{\pgfpoint{271.295990pt}{220.471588pt}}
\pgfusepath{stroke}
\pgfpathmoveto{\pgfpoint{271.295990pt}{232.825272pt}}
\pgflineto{\pgfpoint{271.295990pt}{226.648422pt}}
\pgfusepath{stroke}
\pgfpathmoveto{\pgfpoint{271.295990pt}{245.178955pt}}
\pgflineto{\pgfpoint{271.295990pt}{239.002106pt}}
\pgfusepath{stroke}
\pgfpathmoveto{\pgfpoint{271.295990pt}{251.355804pt}}
\pgflineto{\pgfpoint{271.295990pt}{245.178955pt}}
\pgfusepath{stroke}
\pgfpathmoveto{\pgfpoint{271.295990pt}{257.532623pt}}
\pgflineto{\pgfpoint{271.295990pt}{251.355804pt}}
\pgfusepath{stroke}
\pgfpathmoveto{\pgfpoint{271.295990pt}{263.709473pt}}
\pgflineto{\pgfpoint{271.295990pt}{257.532623pt}}
\pgfusepath{stroke}
\pgfpathmoveto{\pgfpoint{271.295990pt}{269.886322pt}}
\pgflineto{\pgfpoint{271.295990pt}{263.709473pt}}
\pgfusepath{stroke}
\pgfpathmoveto{\pgfpoint{271.295990pt}{276.063141pt}}
\pgflineto{\pgfpoint{271.295990pt}{269.886322pt}}
\pgfusepath{stroke}
\pgfpathmoveto{\pgfpoint{271.295990pt}{195.764206pt}}
\pgflineto{\pgfpoint{271.277924pt}{195.764206pt}}
\pgfusepath{stroke}
\pgfpathmoveto{\pgfpoint{271.295990pt}{177.233673pt}}
\pgflineto{\pgfpoint{271.314056pt}{177.233673pt}}
\pgfusepath{stroke}
\pgfpathmoveto{\pgfpoint{271.295990pt}{183.410522pt}}
\pgflineto{\pgfpoint{271.323059pt}{183.410522pt}}
\pgfusepath{stroke}
\pgfpathmoveto{\pgfpoint{271.295990pt}{189.587372pt}}
\pgflineto{\pgfpoint{271.323090pt}{189.587372pt}}
\pgfusepath{stroke}
\pgfpathmoveto{\pgfpoint{271.295990pt}{195.764206pt}}
\pgflineto{\pgfpoint{271.323090pt}{195.764206pt}}
\pgfusepath{stroke}
\pgfpathmoveto{\pgfpoint{271.295990pt}{201.941055pt}}
\pgflineto{\pgfpoint{271.323120pt}{201.941055pt}}
\pgfusepath{stroke}
\pgfpathmoveto{\pgfpoint{271.295990pt}{208.117905pt}}
\pgflineto{\pgfpoint{271.323120pt}{208.117905pt}}
\pgfusepath{stroke}
\pgfpathmoveto{\pgfpoint{271.295990pt}{214.294739pt}}
\pgflineto{\pgfpoint{271.323090pt}{214.294739pt}}
\pgfusepath{stroke}
\pgfpathmoveto{\pgfpoint{271.295990pt}{220.471588pt}}
\pgflineto{\pgfpoint{271.327637pt}{220.471588pt}}
\pgfusepath{stroke}
\pgfpathmoveto{\pgfpoint{271.295990pt}{226.648422pt}}
\pgflineto{\pgfpoint{271.327637pt}{226.648422pt}}
\pgfusepath{stroke}
\pgfpathmoveto{\pgfpoint{271.295990pt}{232.825272pt}}
\pgflineto{\pgfpoint{271.332153pt}{232.825272pt}}
\pgfusepath{stroke}
\pgfpathmoveto{\pgfpoint{271.295990pt}{239.002106pt}}
\pgflineto{\pgfpoint{271.332123pt}{239.002106pt}}
\pgfusepath{stroke}
\pgfpathmoveto{\pgfpoint{271.295990pt}{245.178955pt}}
\pgflineto{\pgfpoint{271.332184pt}{245.178955pt}}
\pgfusepath{stroke}
\pgfpathmoveto{\pgfpoint{271.295990pt}{251.355804pt}}
\pgflineto{\pgfpoint{271.336670pt}{251.355804pt}}
\pgfusepath{stroke}
\pgfpathmoveto{\pgfpoint{271.295990pt}{257.532623pt}}
\pgflineto{\pgfpoint{271.336670pt}{257.532623pt}}
\pgfusepath{stroke}
\pgfpathmoveto{\pgfpoint{271.295990pt}{263.709473pt}}
\pgflineto{\pgfpoint{271.341156pt}{263.709473pt}}
\pgfusepath{stroke}
\pgfpathmoveto{\pgfpoint{271.295990pt}{269.886322pt}}
\pgflineto{\pgfpoint{271.341156pt}{269.886322pt}}
\pgfusepath{stroke}
\pgfpathmoveto{\pgfpoint{271.295990pt}{276.063141pt}}
\pgflineto{\pgfpoint{271.341187pt}{276.063141pt}}
\pgfusepath{stroke}
\pgfpathmoveto{\pgfpoint{271.345673pt}{282.239990pt}}
\pgflineto{\pgfpoint{271.309692pt}{282.239990pt}}
\pgfusepath{stroke}
\pgfpathmoveto{\pgfpoint{271.345673pt}{288.416840pt}}
\pgflineto{\pgfpoint{271.309723pt}{288.416840pt}}
\pgfusepath{stroke}
\pgfpathmoveto{\pgfpoint{271.350220pt}{294.593689pt}}
\pgflineto{\pgfpoint{271.314240pt}{294.593689pt}}
\pgfusepath{stroke}
\pgfpathmoveto{\pgfpoint{271.350220pt}{300.770538pt}}
\pgflineto{\pgfpoint{271.314270pt}{300.770538pt}}
\pgfusepath{stroke}
\pgfpathmoveto{\pgfpoint{271.354767pt}{306.947388pt}}
\pgflineto{\pgfpoint{271.318817pt}{306.947388pt}}
\pgfusepath{stroke}
\pgfpathmoveto{\pgfpoint{271.354736pt}{313.124207pt}}
\pgflineto{\pgfpoint{271.318817pt}{313.124207pt}}
\pgfusepath{stroke}
\pgfpathmoveto{\pgfpoint{271.354767pt}{319.301056pt}}
\pgflineto{\pgfpoint{271.318878pt}{319.301056pt}}
\pgfusepath{stroke}
\pgfpathmoveto{\pgfpoint{271.295990pt}{103.111580pt}}
\pgflineto{\pgfpoint{271.295990pt}{96.934731pt}}
\pgfusepath{stroke}
\pgfpathmoveto{\pgfpoint{271.295990pt}{72.227356pt}}
\pgflineto{\pgfpoint{271.295990pt}{66.050522pt}}
\pgfusepath{stroke}
\pgfpathmoveto{\pgfpoint{271.295990pt}{78.404205pt}}
\pgflineto{\pgfpoint{271.295990pt}{72.227356pt}}
\pgfusepath{stroke}
\pgfpathmoveto{\pgfpoint{271.295990pt}{84.581039pt}}
\pgflineto{\pgfpoint{271.295990pt}{78.404205pt}}
\pgfusepath{stroke}
\pgfpathmoveto{\pgfpoint{271.295990pt}{90.757896pt}}
\pgflineto{\pgfpoint{271.295990pt}{84.581039pt}}
\pgfusepath{stroke}
\pgfpathmoveto{\pgfpoint{271.295990pt}{96.934731pt}}
\pgflineto{\pgfpoint{271.295990pt}{90.757896pt}}
\pgfusepath{stroke}
\pgfpathmoveto{\pgfpoint{271.295990pt}{109.288422pt}}
\pgflineto{\pgfpoint{271.295990pt}{103.111580pt}}
\pgfusepath{stroke}
\pgfpathmoveto{\pgfpoint{271.295990pt}{115.465263pt}}
\pgflineto{\pgfpoint{271.295990pt}{109.288422pt}}
\pgfusepath{stroke}
\pgfpathmoveto{\pgfpoint{271.295990pt}{121.642097pt}}
\pgflineto{\pgfpoint{271.295990pt}{115.465263pt}}
\pgfusepath{stroke}
\pgfpathmoveto{\pgfpoint{271.295990pt}{127.818947pt}}
\pgflineto{\pgfpoint{271.295990pt}{121.642097pt}}
\pgfusepath{stroke}
\pgfpathmoveto{\pgfpoint{271.295990pt}{133.995789pt}}
\pgflineto{\pgfpoint{271.295990pt}{127.818947pt}}
\pgfusepath{stroke}
\pgfpathmoveto{\pgfpoint{271.295990pt}{140.172638pt}}
\pgflineto{\pgfpoint{271.295990pt}{133.995789pt}}
\pgfusepath{stroke}
\pgfpathmoveto{\pgfpoint{271.295990pt}{146.349472pt}}
\pgflineto{\pgfpoint{271.295990pt}{140.172638pt}}
\pgfusepath{stroke}
\pgfpathmoveto{\pgfpoint{271.295990pt}{47.519989pt}}
\pgflineto{\pgfpoint{271.277893pt}{47.519989pt}}
\pgfusepath{stroke}
\pgfpathmoveto{\pgfpoint{271.295990pt}{53.696838pt}}
\pgflineto{\pgfpoint{271.277924pt}{53.696838pt}}
\pgfusepath{stroke}
\pgfpathmoveto{\pgfpoint{271.295990pt}{59.873672pt}}
\pgflineto{\pgfpoint{271.277924pt}{59.873672pt}}
\pgfusepath{stroke}
\pgfpathmoveto{\pgfpoint{271.295990pt}{53.696838pt}}
\pgflineto{\pgfpoint{271.295990pt}{47.519989pt}}
\pgfusepath{stroke}
\pgfpathmoveto{\pgfpoint{271.295990pt}{47.519989pt}}
\pgflineto{\pgfpoint{271.314087pt}{47.519989pt}}
\pgfusepath{stroke}
\pgfpathmoveto{\pgfpoint{271.295990pt}{66.050522pt}}
\pgflineto{\pgfpoint{271.295990pt}{59.873672pt}}
\pgfusepath{stroke}
\pgfpathmoveto{\pgfpoint{271.295990pt}{59.873672pt}}
\pgflineto{\pgfpoint{271.295990pt}{53.696838pt}}
\pgfusepath{stroke}
\pgfpathmoveto{\pgfpoint{280.223969pt}{47.519989pt}}
\pgflineto{\pgfpoint{271.314087pt}{47.519989pt}}
\pgfusepath{stroke}
\pgfpathmoveto{\pgfpoint{280.223969pt}{53.696838pt}}
\pgflineto{\pgfpoint{271.295990pt}{53.696838pt}}
\pgfusepath{stroke}
\pgfpathmoveto{\pgfpoint{280.223969pt}{59.873672pt}}
\pgflineto{\pgfpoint{271.295990pt}{59.873672pt}}
\pgfusepath{stroke}
\pgfpathmoveto{\pgfpoint{280.223969pt}{66.050522pt}}
\pgflineto{\pgfpoint{271.295990pt}{66.050522pt}}
\pgfusepath{stroke}
\pgfpathmoveto{\pgfpoint{280.223969pt}{72.227356pt}}
\pgflineto{\pgfpoint{271.295990pt}{72.227356pt}}
\pgfusepath{stroke}
\pgfpathmoveto{\pgfpoint{280.223969pt}{78.404205pt}}
\pgflineto{\pgfpoint{271.295990pt}{78.404205pt}}
\pgfusepath{stroke}
\pgfpathmoveto{\pgfpoint{280.223969pt}{84.581039pt}}
\pgflineto{\pgfpoint{271.295990pt}{84.581039pt}}
\pgfusepath{stroke}
\pgfpathmoveto{\pgfpoint{280.223969pt}{90.757896pt}}
\pgflineto{\pgfpoint{271.295990pt}{90.757896pt}}
\pgfusepath{stroke}
\pgfpathmoveto{\pgfpoint{280.223969pt}{96.934731pt}}
\pgflineto{\pgfpoint{271.295990pt}{96.934731pt}}
\pgfusepath{stroke}
\pgfpathmoveto{\pgfpoint{280.223969pt}{103.111580pt}}
\pgflineto{\pgfpoint{271.295990pt}{103.111580pt}}
\pgfusepath{stroke}
\pgfpathmoveto{\pgfpoint{280.223969pt}{109.288422pt}}
\pgflineto{\pgfpoint{271.295990pt}{109.288422pt}}
\pgfusepath{stroke}
\pgfpathmoveto{\pgfpoint{280.223969pt}{115.465263pt}}
\pgflineto{\pgfpoint{271.295990pt}{115.465263pt}}
\pgfusepath{stroke}
\pgfpathmoveto{\pgfpoint{280.223969pt}{121.642097pt}}
\pgflineto{\pgfpoint{271.295990pt}{121.642097pt}}
\pgfusepath{stroke}
\pgfpathmoveto{\pgfpoint{280.223969pt}{127.818947pt}}
\pgflineto{\pgfpoint{271.295990pt}{127.818947pt}}
\pgfusepath{stroke}
\pgfpathmoveto{\pgfpoint{280.223969pt}{133.995789pt}}
\pgflineto{\pgfpoint{271.295990pt}{133.995789pt}}
\pgfusepath{stroke}
\pgfpathmoveto{\pgfpoint{280.223969pt}{140.172638pt}}
\pgflineto{\pgfpoint{271.295990pt}{140.172638pt}}
\pgfusepath{stroke}
\pgfpathmoveto{\pgfpoint{280.223969pt}{146.349472pt}}
\pgflineto{\pgfpoint{271.295990pt}{146.349472pt}}
\pgfusepath{stroke}
\pgfpathmoveto{\pgfpoint{280.223969pt}{152.526306pt}}
\pgflineto{\pgfpoint{271.314026pt}{152.526306pt}}
\pgfusepath{stroke}
\pgfpathmoveto{\pgfpoint{280.223969pt}{158.703156pt}}
\pgflineto{\pgfpoint{271.314087pt}{158.703156pt}}
\pgfusepath{stroke}
\pgfpathmoveto{\pgfpoint{280.223969pt}{164.880005pt}}
\pgflineto{\pgfpoint{271.314117pt}{164.880005pt}}
\pgfusepath{stroke}
\pgfpathmoveto{\pgfpoint{280.223969pt}{171.056854pt}}
\pgflineto{\pgfpoint{271.314117pt}{171.056854pt}}
\pgfusepath{stroke}
\pgfpathmoveto{\pgfpoint{280.223969pt}{177.233673pt}}
\pgflineto{\pgfpoint{271.314056pt}{177.233673pt}}
\pgfusepath{stroke}
\pgfpathmoveto{\pgfpoint{280.223969pt}{183.410522pt}}
\pgflineto{\pgfpoint{271.323059pt}{183.410522pt}}
\pgfusepath{stroke}
\pgfpathmoveto{\pgfpoint{280.223969pt}{189.587372pt}}
\pgflineto{\pgfpoint{271.323090pt}{189.587372pt}}
\pgfusepath{stroke}
\pgfpathmoveto{\pgfpoint{280.223969pt}{195.764206pt}}
\pgflineto{\pgfpoint{271.323090pt}{195.764206pt}}
\pgfusepath{stroke}
\pgfpathmoveto{\pgfpoint{271.323120pt}{201.941055pt}}
\pgflineto{\pgfpoint{280.210449pt}{201.941055pt}}
\pgfusepath{stroke}
\pgfpathmoveto{\pgfpoint{271.323120pt}{208.117905pt}}
\pgflineto{\pgfpoint{280.201447pt}{208.117905pt}}
\pgfusepath{stroke}
\pgfpathmoveto{\pgfpoint{271.323090pt}{214.294739pt}}
\pgflineto{\pgfpoint{280.192383pt}{214.294739pt}}
\pgfusepath{stroke}
\pgfpathmoveto{\pgfpoint{271.327637pt}{220.471588pt}}
\pgflineto{\pgfpoint{280.187897pt}{220.471588pt}}
\pgfusepath{stroke}
\pgfpathmoveto{\pgfpoint{271.327637pt}{226.648422pt}}
\pgflineto{\pgfpoint{280.178864pt}{226.648422pt}}
\pgfusepath{stroke}
\pgfpathmoveto{\pgfpoint{271.332153pt}{232.825272pt}}
\pgflineto{\pgfpoint{280.174408pt}{232.825272pt}}
\pgfusepath{stroke}
\pgfpathmoveto{\pgfpoint{271.332123pt}{239.002106pt}}
\pgflineto{\pgfpoint{280.165344pt}{239.002106pt}}
\pgfusepath{stroke}
\pgfpathmoveto{\pgfpoint{271.332184pt}{245.178955pt}}
\pgflineto{\pgfpoint{280.160828pt}{245.178955pt}}
\pgfusepath{stroke}
\pgfpathmoveto{\pgfpoint{271.336670pt}{251.355804pt}}
\pgflineto{\pgfpoint{280.151825pt}{251.355804pt}}
\pgfusepath{stroke}
\pgfpathmoveto{\pgfpoint{271.336670pt}{257.532623pt}}
\pgflineto{\pgfpoint{280.142792pt}{257.532623pt}}
\pgfusepath{stroke}
\pgfpathmoveto{\pgfpoint{271.341156pt}{263.709473pt}}
\pgflineto{\pgfpoint{280.138306pt}{263.709473pt}}
\pgfusepath{stroke}
\pgfpathmoveto{\pgfpoint{271.341156pt}{269.886322pt}}
\pgflineto{\pgfpoint{280.129272pt}{269.886322pt}}
\pgfusepath{stroke}
\pgfpathmoveto{\pgfpoint{271.341187pt}{276.063141pt}}
\pgflineto{\pgfpoint{280.124695pt}{276.063141pt}}
\pgfusepath{stroke}
\pgfpathmoveto{\pgfpoint{271.345673pt}{282.239990pt}}
\pgflineto{\pgfpoint{280.115723pt}{282.239990pt}}
\pgfusepath{stroke}
\pgfpathmoveto{\pgfpoint{271.345673pt}{288.416840pt}}
\pgflineto{\pgfpoint{280.106689pt}{288.416840pt}}
\pgfusepath{stroke}
\pgfpathmoveto{\pgfpoint{271.350220pt}{294.593689pt}}
\pgflineto{\pgfpoint{280.102203pt}{294.593689pt}}
\pgfusepath{stroke}
\pgfpathmoveto{\pgfpoint{271.350220pt}{300.770538pt}}
\pgflineto{\pgfpoint{280.093231pt}{300.770538pt}}
\pgfusepath{stroke}
\pgfpathmoveto{\pgfpoint{271.354767pt}{306.947388pt}}
\pgflineto{\pgfpoint{280.088684pt}{306.947388pt}}
\pgfusepath{stroke}
\pgfpathmoveto{\pgfpoint{271.354736pt}{313.124207pt}}
\pgflineto{\pgfpoint{280.079681pt}{313.124207pt}}
\pgfusepath{stroke}
\pgfpathmoveto{\pgfpoint{271.354767pt}{319.301056pt}}
\pgflineto{\pgfpoint{280.075073pt}{319.301056pt}}
\pgfusepath{stroke}
\pgfpathmoveto{\pgfpoint{280.223969pt}{146.349472pt}}
\pgflineto{\pgfpoint{280.223969pt}{140.172638pt}}
\pgfusepath{stroke}
\pgfpathmoveto{\pgfpoint{280.223969pt}{140.172638pt}}
\pgflineto{\pgfpoint{280.223969pt}{133.995789pt}}
\pgfusepath{stroke}
\pgfpathmoveto{\pgfpoint{280.223969pt}{152.526306pt}}
\pgflineto{\pgfpoint{280.223969pt}{146.349472pt}}
\pgfusepath{stroke}
\pgfpathmoveto{\pgfpoint{280.223969pt}{158.703156pt}}
\pgflineto{\pgfpoint{280.223969pt}{152.526306pt}}
\pgfusepath{stroke}
\pgfpathmoveto{\pgfpoint{280.223969pt}{164.880005pt}}
\pgflineto{\pgfpoint{280.223969pt}{158.703156pt}}
\pgfusepath{stroke}
\pgfpathmoveto{\pgfpoint{280.223969pt}{171.056854pt}}
\pgflineto{\pgfpoint{280.223969pt}{164.880005pt}}
\pgfusepath{stroke}
\pgfpathmoveto{\pgfpoint{280.223969pt}{177.233673pt}}
\pgflineto{\pgfpoint{280.223969pt}{171.056854pt}}
\pgfusepath{stroke}
\pgfpathmoveto{\pgfpoint{280.223969pt}{183.410522pt}}
\pgflineto{\pgfpoint{280.223969pt}{177.233673pt}}
\pgfusepath{stroke}
\pgfpathmoveto{\pgfpoint{280.223969pt}{140.172638pt}}
\pgflineto{\pgfpoint{280.242004pt}{140.172638pt}}
\pgfusepath{stroke}
\pgfpathmoveto{\pgfpoint{280.223969pt}{146.349472pt}}
\pgflineto{\pgfpoint{280.242004pt}{146.349472pt}}
\pgfusepath{stroke}
\pgfpathmoveto{\pgfpoint{280.223969pt}{152.526306pt}}
\pgflineto{\pgfpoint{280.241974pt}{152.526306pt}}
\pgfusepath{stroke}
\pgfpathmoveto{\pgfpoint{280.223969pt}{158.703156pt}}
\pgflineto{\pgfpoint{280.241974pt}{158.703156pt}}
\pgfusepath{stroke}
\pgfpathmoveto{\pgfpoint{280.223969pt}{164.880005pt}}
\pgflineto{\pgfpoint{280.250977pt}{164.880005pt}}
\pgfusepath{stroke}
\pgfpathmoveto{\pgfpoint{280.223969pt}{171.056854pt}}
\pgflineto{\pgfpoint{280.251007pt}{171.056854pt}}
\pgfusepath{stroke}
\pgfpathmoveto{\pgfpoint{280.223969pt}{177.233673pt}}
\pgflineto{\pgfpoint{280.251007pt}{177.233673pt}}
\pgfusepath{stroke}
\pgfpathmoveto{\pgfpoint{280.223969pt}{195.764206pt}}
\pgflineto{\pgfpoint{280.223969pt}{189.587372pt}}
\pgfusepath{stroke}
\pgfpathmoveto{\pgfpoint{280.223969pt}{189.587372pt}}
\pgflineto{\pgfpoint{280.223969pt}{183.410522pt}}
\pgfusepath{stroke}
\pgfpathmoveto{\pgfpoint{280.192383pt}{214.294739pt}}
\pgflineto{\pgfpoint{280.210449pt}{214.294739pt}}
\pgfusepath{stroke}
\pgfpathmoveto{\pgfpoint{280.223969pt}{220.471588pt}}
\pgflineto{\pgfpoint{280.187897pt}{220.471588pt}}
\pgfusepath{stroke}
\pgfpathmoveto{\pgfpoint{280.223969pt}{226.648422pt}}
\pgflineto{\pgfpoint{280.178864pt}{226.648422pt}}
\pgfusepath{stroke}
\pgfpathmoveto{\pgfpoint{280.223969pt}{232.825272pt}}
\pgflineto{\pgfpoint{280.174408pt}{232.825272pt}}
\pgfusepath{stroke}
\pgfpathmoveto{\pgfpoint{280.223969pt}{239.002106pt}}
\pgflineto{\pgfpoint{280.165344pt}{239.002106pt}}
\pgfusepath{stroke}
\pgfpathmoveto{\pgfpoint{280.223969pt}{245.178955pt}}
\pgflineto{\pgfpoint{280.160828pt}{245.178955pt}}
\pgfusepath{stroke}
\pgfpathmoveto{\pgfpoint{280.223969pt}{251.355804pt}}
\pgflineto{\pgfpoint{280.151825pt}{251.355804pt}}
\pgfusepath{stroke}
\pgfpathmoveto{\pgfpoint{280.223969pt}{257.532623pt}}
\pgflineto{\pgfpoint{280.142792pt}{257.532623pt}}
\pgfusepath{stroke}
\pgfpathmoveto{\pgfpoint{280.223969pt}{263.709473pt}}
\pgflineto{\pgfpoint{280.138306pt}{263.709473pt}}
\pgfusepath{stroke}
\pgfpathmoveto{\pgfpoint{280.223969pt}{269.886322pt}}
\pgflineto{\pgfpoint{280.129272pt}{269.886322pt}}
\pgfusepath{stroke}
\pgfpathmoveto{\pgfpoint{280.223969pt}{276.063141pt}}
\pgflineto{\pgfpoint{280.124695pt}{276.063141pt}}
\pgfusepath{stroke}
\pgfpathmoveto{\pgfpoint{280.223969pt}{282.239990pt}}
\pgflineto{\pgfpoint{280.115723pt}{282.239990pt}}
\pgfusepath{stroke}
\pgfpathmoveto{\pgfpoint{280.223969pt}{288.416840pt}}
\pgflineto{\pgfpoint{280.106689pt}{288.416840pt}}
\pgfusepath{stroke}
\pgfpathmoveto{\pgfpoint{280.223969pt}{294.593689pt}}
\pgflineto{\pgfpoint{280.102203pt}{294.593689pt}}
\pgfusepath{stroke}
\pgfpathmoveto{\pgfpoint{280.223969pt}{300.770538pt}}
\pgflineto{\pgfpoint{280.093231pt}{300.770538pt}}
\pgfusepath{stroke}
\pgfpathmoveto{\pgfpoint{280.223969pt}{306.947388pt}}
\pgflineto{\pgfpoint{280.088684pt}{306.947388pt}}
\pgfusepath{stroke}
\pgfpathmoveto{\pgfpoint{280.223969pt}{313.124207pt}}
\pgflineto{\pgfpoint{280.079681pt}{313.124207pt}}
\pgfusepath{stroke}
\pgfpathmoveto{\pgfpoint{280.223969pt}{319.301056pt}}
\pgflineto{\pgfpoint{280.075073pt}{319.301056pt}}
\pgfusepath{stroke}
\pgfpathmoveto{\pgfpoint{280.223969pt}{294.593689pt}}
\pgflineto{\pgfpoint{280.223969pt}{288.416840pt}}
\pgfusepath{stroke}
\pgfpathmoveto{\pgfpoint{280.223969pt}{300.770538pt}}
\pgflineto{\pgfpoint{280.223969pt}{294.593689pt}}
\pgfusepath{stroke}
\pgfpathmoveto{\pgfpoint{280.223969pt}{306.947388pt}}
\pgflineto{\pgfpoint{280.223969pt}{300.770538pt}}
\pgfusepath{stroke}
\pgfpathmoveto{\pgfpoint{280.223969pt}{313.124207pt}}
\pgflineto{\pgfpoint{280.223969pt}{306.947388pt}}
\pgfusepath{stroke}
\pgfpathmoveto{\pgfpoint{280.223969pt}{319.301056pt}}
\pgflineto{\pgfpoint{280.223969pt}{313.124207pt}}
\pgfusepath{stroke}
\pgfpathmoveto{\pgfpoint{280.223969pt}{294.593689pt}}
\pgflineto{\pgfpoint{280.237427pt}{294.593689pt}}
\pgfusepath{stroke}
\pgfpathmoveto{\pgfpoint{280.223969pt}{300.770538pt}}
\pgflineto{\pgfpoint{280.237427pt}{300.770538pt}}
\pgfusepath{stroke}
\pgfpathmoveto{\pgfpoint{280.223969pt}{306.947388pt}}
\pgflineto{\pgfpoint{280.241913pt}{306.947388pt}}
\pgfusepath{stroke}
\pgfpathmoveto{\pgfpoint{280.223969pt}{313.124207pt}}
\pgflineto{\pgfpoint{280.241913pt}{313.124207pt}}
\pgfusepath{stroke}
\pgfpathmoveto{\pgfpoint{280.223969pt}{251.355804pt}}
\pgflineto{\pgfpoint{280.223969pt}{245.178955pt}}
\pgfusepath{stroke}
\pgfpathmoveto{\pgfpoint{280.223969pt}{226.648422pt}}
\pgflineto{\pgfpoint{280.223969pt}{220.471588pt}}
\pgfusepath{stroke}
\pgfpathmoveto{\pgfpoint{280.223969pt}{232.825272pt}}
\pgflineto{\pgfpoint{280.223969pt}{226.648422pt}}
\pgfusepath{stroke}
\pgfpathmoveto{\pgfpoint{280.223969pt}{239.002106pt}}
\pgflineto{\pgfpoint{280.223969pt}{232.825272pt}}
\pgfusepath{stroke}
\pgfpathmoveto{\pgfpoint{280.223969pt}{245.178955pt}}
\pgflineto{\pgfpoint{280.223969pt}{239.002106pt}}
\pgfusepath{stroke}
\pgfpathmoveto{\pgfpoint{280.223969pt}{257.532623pt}}
\pgflineto{\pgfpoint{280.223969pt}{251.355804pt}}
\pgfusepath{stroke}
\pgfpathmoveto{\pgfpoint{280.223969pt}{263.709473pt}}
\pgflineto{\pgfpoint{280.223969pt}{257.532623pt}}
\pgfusepath{stroke}
\pgfpathmoveto{\pgfpoint{280.223969pt}{269.886322pt}}
\pgflineto{\pgfpoint{280.223969pt}{263.709473pt}}
\pgfusepath{stroke}
\pgfpathmoveto{\pgfpoint{280.223969pt}{276.063141pt}}
\pgflineto{\pgfpoint{280.223969pt}{269.886322pt}}
\pgfusepath{stroke}
\pgfpathmoveto{\pgfpoint{280.223969pt}{282.239990pt}}
\pgflineto{\pgfpoint{280.223969pt}{276.063141pt}}
\pgfusepath{stroke}
\pgfpathmoveto{\pgfpoint{280.223969pt}{288.416840pt}}
\pgflineto{\pgfpoint{280.223969pt}{282.239990pt}}
\pgfusepath{stroke}
\pgfpathmoveto{\pgfpoint{280.223969pt}{201.941055pt}}
\pgflineto{\pgfpoint{280.210449pt}{201.941055pt}}
\pgfusepath{stroke}
\pgfpathmoveto{\pgfpoint{280.223969pt}{208.117905pt}}
\pgflineto{\pgfpoint{280.201447pt}{208.117905pt}}
\pgfusepath{stroke}
\pgfpathmoveto{\pgfpoint{280.223969pt}{214.294739pt}}
\pgflineto{\pgfpoint{280.210449pt}{214.294739pt}}
\pgfusepath{stroke}
\pgfpathmoveto{\pgfpoint{280.223969pt}{220.471588pt}}
\pgflineto{\pgfpoint{280.223969pt}{214.294739pt}}
\pgfusepath{stroke}
\pgfpathmoveto{\pgfpoint{280.223969pt}{220.471588pt}}
\pgflineto{\pgfpoint{280.242004pt}{220.471588pt}}
\pgfusepath{stroke}
\pgfpathmoveto{\pgfpoint{280.223969pt}{226.648422pt}}
\pgflineto{\pgfpoint{280.246490pt}{226.648422pt}}
\pgfusepath{stroke}
\pgfpathmoveto{\pgfpoint{280.223969pt}{232.825272pt}}
\pgflineto{\pgfpoint{280.264496pt}{232.825272pt}}
\pgfusepath{stroke}
\pgfpathmoveto{\pgfpoint{280.223969pt}{239.002106pt}}
\pgflineto{\pgfpoint{280.264496pt}{239.002106pt}}
\pgfusepath{stroke}
\pgfpathmoveto{\pgfpoint{280.223969pt}{245.178955pt}}
\pgflineto{\pgfpoint{280.269012pt}{245.178955pt}}
\pgfusepath{stroke}
\pgfpathmoveto{\pgfpoint{280.223969pt}{251.355804pt}}
\pgflineto{\pgfpoint{280.269043pt}{251.355804pt}}
\pgfusepath{stroke}
\pgfpathmoveto{\pgfpoint{280.223969pt}{257.532623pt}}
\pgflineto{\pgfpoint{280.269012pt}{257.532623pt}}
\pgfusepath{stroke}
\pgfpathmoveto{\pgfpoint{280.223969pt}{263.709473pt}}
\pgflineto{\pgfpoint{280.273499pt}{263.709473pt}}
\pgfusepath{stroke}
\pgfpathmoveto{\pgfpoint{280.223969pt}{269.886322pt}}
\pgflineto{\pgfpoint{280.273529pt}{269.886322pt}}
\pgfusepath{stroke}
\pgfpathmoveto{\pgfpoint{280.223969pt}{276.063141pt}}
\pgflineto{\pgfpoint{280.278015pt}{276.063141pt}}
\pgfusepath{stroke}
\pgfpathmoveto{\pgfpoint{280.223969pt}{282.239990pt}}
\pgflineto{\pgfpoint{280.277985pt}{282.239990pt}}
\pgfusepath{stroke}
\pgfpathmoveto{\pgfpoint{280.223969pt}{288.416840pt}}
\pgflineto{\pgfpoint{280.278046pt}{288.416840pt}}
\pgfusepath{stroke}
\pgfpathmoveto{\pgfpoint{280.282532pt}{294.593689pt}}
\pgflineto{\pgfpoint{280.237427pt}{294.593689pt}}
\pgfusepath{stroke}
\pgfpathmoveto{\pgfpoint{280.282532pt}{300.770538pt}}
\pgflineto{\pgfpoint{280.237427pt}{300.770538pt}}
\pgfusepath{stroke}
\pgfpathmoveto{\pgfpoint{280.287018pt}{306.947388pt}}
\pgflineto{\pgfpoint{280.241913pt}{306.947388pt}}
\pgfusepath{stroke}
\pgfpathmoveto{\pgfpoint{280.287018pt}{313.124207pt}}
\pgflineto{\pgfpoint{280.241913pt}{313.124207pt}}
\pgfusepath{stroke}
\pgfpathmoveto{\pgfpoint{280.223969pt}{201.941055pt}}
\pgflineto{\pgfpoint{280.223969pt}{195.764206pt}}
\pgfusepath{stroke}
\pgfpathmoveto{\pgfpoint{280.223969pt}{208.117905pt}}
\pgflineto{\pgfpoint{280.223969pt}{201.941055pt}}
\pgfusepath{stroke}
\pgfpathmoveto{\pgfpoint{280.223969pt}{214.294739pt}}
\pgflineto{\pgfpoint{280.223969pt}{208.117905pt}}
\pgfusepath{stroke}
\pgfpathmoveto{\pgfpoint{280.223969pt}{183.410522pt}}
\pgflineto{\pgfpoint{280.251007pt}{183.410522pt}}
\pgfusepath{stroke}
\pgfpathmoveto{\pgfpoint{280.223969pt}{189.587372pt}}
\pgflineto{\pgfpoint{280.251007pt}{189.587372pt}}
\pgfusepath{stroke}
\pgfpathmoveto{\pgfpoint{280.223969pt}{195.764206pt}}
\pgflineto{\pgfpoint{280.259979pt}{195.764206pt}}
\pgfusepath{stroke}
\pgfpathmoveto{\pgfpoint{280.223969pt}{201.941055pt}}
\pgflineto{\pgfpoint{280.255493pt}{201.941055pt}}
\pgfusepath{stroke}
\pgfpathmoveto{\pgfpoint{280.223969pt}{208.117905pt}}
\pgflineto{\pgfpoint{280.255493pt}{208.117905pt}}
\pgfusepath{stroke}
\pgfpathmoveto{\pgfpoint{280.223969pt}{214.294739pt}}
\pgflineto{\pgfpoint{280.255524pt}{214.294739pt}}
\pgfusepath{stroke}
\pgfpathmoveto{\pgfpoint{280.260040pt}{220.471588pt}}
\pgflineto{\pgfpoint{280.242004pt}{220.471588pt}}
\pgfusepath{stroke}
\pgfpathmoveto{\pgfpoint{280.260010pt}{226.648422pt}}
\pgflineto{\pgfpoint{280.246490pt}{226.648422pt}}
\pgfusepath{stroke}
\pgfpathmoveto{\pgfpoint{280.223969pt}{84.581039pt}}
\pgflineto{\pgfpoint{280.223969pt}{78.404205pt}}
\pgfusepath{stroke}
\pgfpathmoveto{\pgfpoint{280.223969pt}{53.696838pt}}
\pgflineto{\pgfpoint{280.223969pt}{47.519989pt}}
\pgfusepath{stroke}
\pgfpathmoveto{\pgfpoint{280.223969pt}{59.873672pt}}
\pgflineto{\pgfpoint{280.223969pt}{53.696838pt}}
\pgfusepath{stroke}
\pgfpathmoveto{\pgfpoint{280.223969pt}{66.050522pt}}
\pgflineto{\pgfpoint{280.223969pt}{59.873672pt}}
\pgfusepath{stroke}
\pgfpathmoveto{\pgfpoint{280.223969pt}{72.227356pt}}
\pgflineto{\pgfpoint{280.223969pt}{66.050522pt}}
\pgfusepath{stroke}
\pgfpathmoveto{\pgfpoint{280.223969pt}{78.404205pt}}
\pgflineto{\pgfpoint{280.223969pt}{72.227356pt}}
\pgfusepath{stroke}
\pgfpathmoveto{\pgfpoint{280.223969pt}{90.757896pt}}
\pgflineto{\pgfpoint{280.223969pt}{84.581039pt}}
\pgfusepath{stroke}
\pgfpathmoveto{\pgfpoint{280.223969pt}{96.934731pt}}
\pgflineto{\pgfpoint{280.223969pt}{90.757896pt}}
\pgfusepath{stroke}
\pgfpathmoveto{\pgfpoint{280.223969pt}{103.111580pt}}
\pgflineto{\pgfpoint{280.223969pt}{96.934731pt}}
\pgfusepath{stroke}
\pgfpathmoveto{\pgfpoint{280.223969pt}{109.288422pt}}
\pgflineto{\pgfpoint{280.223969pt}{103.111580pt}}
\pgfusepath{stroke}
\pgfpathmoveto{\pgfpoint{280.223969pt}{115.465263pt}}
\pgflineto{\pgfpoint{280.223969pt}{109.288422pt}}
\pgfusepath{stroke}
\pgfpathmoveto{\pgfpoint{280.223969pt}{121.642097pt}}
\pgflineto{\pgfpoint{280.223969pt}{115.465263pt}}
\pgfusepath{stroke}
\pgfpathmoveto{\pgfpoint{280.223969pt}{127.818947pt}}
\pgflineto{\pgfpoint{280.223969pt}{121.642097pt}}
\pgfusepath{stroke}
\pgfpathmoveto{\pgfpoint{280.223969pt}{133.995789pt}}
\pgflineto{\pgfpoint{280.223969pt}{127.818947pt}}
\pgfusepath{stroke}
\pgfpathmoveto{\pgfpoint{280.223969pt}{47.519989pt}}
\pgflineto{\pgfpoint{289.133789pt}{47.519989pt}}
\pgfusepath{stroke}
\pgfpathmoveto{\pgfpoint{280.223969pt}{53.696838pt}}
\pgflineto{\pgfpoint{289.133789pt}{53.696838pt}}
\pgfusepath{stroke}
\pgfpathmoveto{\pgfpoint{289.151978pt}{59.873672pt}}
\pgflineto{\pgfpoint{280.223969pt}{59.873672pt}}
\pgfusepath{stroke}
\pgfpathmoveto{\pgfpoint{289.151978pt}{66.050522pt}}
\pgflineto{\pgfpoint{280.223969pt}{66.050522pt}}
\pgfusepath{stroke}
\pgfpathmoveto{\pgfpoint{289.151978pt}{72.227356pt}}
\pgflineto{\pgfpoint{280.223969pt}{72.227356pt}}
\pgfusepath{stroke}
\pgfpathmoveto{\pgfpoint{289.151978pt}{78.404205pt}}
\pgflineto{\pgfpoint{280.223969pt}{78.404205pt}}
\pgfusepath{stroke}
\pgfpathmoveto{\pgfpoint{289.151978pt}{84.581039pt}}
\pgflineto{\pgfpoint{280.223969pt}{84.581039pt}}
\pgfusepath{stroke}
\pgfpathmoveto{\pgfpoint{289.151978pt}{90.757896pt}}
\pgflineto{\pgfpoint{280.223969pt}{90.757896pt}}
\pgfusepath{stroke}
\pgfpathmoveto{\pgfpoint{289.151978pt}{96.934731pt}}
\pgflineto{\pgfpoint{280.223969pt}{96.934731pt}}
\pgfusepath{stroke}
\pgfpathmoveto{\pgfpoint{289.151978pt}{103.111580pt}}
\pgflineto{\pgfpoint{280.223969pt}{103.111580pt}}
\pgfusepath{stroke}
\pgfpathmoveto{\pgfpoint{289.151978pt}{109.288422pt}}
\pgflineto{\pgfpoint{280.223969pt}{109.288422pt}}
\pgfusepath{stroke}
\pgfpathmoveto{\pgfpoint{289.151978pt}{115.465263pt}}
\pgflineto{\pgfpoint{280.223969pt}{115.465263pt}}
\pgfusepath{stroke}
\pgfpathmoveto{\pgfpoint{289.151978pt}{121.642097pt}}
\pgflineto{\pgfpoint{280.223969pt}{121.642097pt}}
\pgfusepath{stroke}
\pgfpathmoveto{\pgfpoint{289.151978pt}{127.818947pt}}
\pgflineto{\pgfpoint{280.223969pt}{127.818947pt}}
\pgfusepath{stroke}
\pgfpathmoveto{\pgfpoint{289.151978pt}{133.995789pt}}
\pgflineto{\pgfpoint{280.223969pt}{133.995789pt}}
\pgfusepath{stroke}
\pgfpathmoveto{\pgfpoint{289.151978pt}{140.172638pt}}
\pgflineto{\pgfpoint{280.242004pt}{140.172638pt}}
\pgfusepath{stroke}
\pgfpathmoveto{\pgfpoint{289.151978pt}{146.349472pt}}
\pgflineto{\pgfpoint{280.242004pt}{146.349472pt}}
\pgfusepath{stroke}
\pgfpathmoveto{\pgfpoint{289.151978pt}{152.526306pt}}
\pgflineto{\pgfpoint{280.241974pt}{152.526306pt}}
\pgfusepath{stroke}
\pgfpathmoveto{\pgfpoint{289.151978pt}{158.703156pt}}
\pgflineto{\pgfpoint{280.241974pt}{158.703156pt}}
\pgfusepath{stroke}
\pgfpathmoveto{\pgfpoint{289.151978pt}{164.880005pt}}
\pgflineto{\pgfpoint{280.250977pt}{164.880005pt}}
\pgfusepath{stroke}
\pgfpathmoveto{\pgfpoint{289.151978pt}{171.056854pt}}
\pgflineto{\pgfpoint{280.251007pt}{171.056854pt}}
\pgfusepath{stroke}
\pgfpathmoveto{\pgfpoint{289.151978pt}{177.233673pt}}
\pgflineto{\pgfpoint{280.251007pt}{177.233673pt}}
\pgfusepath{stroke}
\pgfpathmoveto{\pgfpoint{289.151978pt}{183.410522pt}}
\pgflineto{\pgfpoint{280.251007pt}{183.410522pt}}
\pgfusepath{stroke}
\pgfpathmoveto{\pgfpoint{289.151978pt}{189.587372pt}}
\pgflineto{\pgfpoint{280.251007pt}{189.587372pt}}
\pgfusepath{stroke}
\pgfpathmoveto{\pgfpoint{289.151978pt}{195.764206pt}}
\pgflineto{\pgfpoint{280.259979pt}{195.764206pt}}
\pgfusepath{stroke}
\pgfpathmoveto{\pgfpoint{289.151978pt}{201.941055pt}}
\pgflineto{\pgfpoint{280.255493pt}{201.941055pt}}
\pgfusepath{stroke}
\pgfpathmoveto{\pgfpoint{289.151978pt}{208.117905pt}}
\pgflineto{\pgfpoint{280.255493pt}{208.117905pt}}
\pgfusepath{stroke}
\pgfpathmoveto{\pgfpoint{289.151978pt}{214.294739pt}}
\pgflineto{\pgfpoint{280.255524pt}{214.294739pt}}
\pgfusepath{stroke}
\pgfpathmoveto{\pgfpoint{289.151978pt}{220.471588pt}}
\pgflineto{\pgfpoint{280.260040pt}{220.471588pt}}
\pgfusepath{stroke}
\pgfpathmoveto{\pgfpoint{280.260010pt}{226.648422pt}}
\pgflineto{\pgfpoint{289.138397pt}{226.648422pt}}
\pgfusepath{stroke}
\pgfpathmoveto{\pgfpoint{280.264496pt}{232.825272pt}}
\pgflineto{\pgfpoint{289.138428pt}{232.825272pt}}
\pgfusepath{stroke}
\pgfpathmoveto{\pgfpoint{280.264496pt}{239.002106pt}}
\pgflineto{\pgfpoint{289.133911pt}{239.002106pt}}
\pgfusepath{stroke}
\pgfpathmoveto{\pgfpoint{280.269012pt}{245.178955pt}}
\pgflineto{\pgfpoint{289.133911pt}{245.178955pt}}
\pgfusepath{stroke}
\pgfpathmoveto{\pgfpoint{280.269043pt}{251.355804pt}}
\pgflineto{\pgfpoint{289.129364pt}{251.355804pt}}
\pgfusepath{stroke}
\pgfpathmoveto{\pgfpoint{280.269012pt}{257.532623pt}}
\pgflineto{\pgfpoint{289.124817pt}{257.532623pt}}
\pgfusepath{stroke}
\pgfpathmoveto{\pgfpoint{280.273499pt}{263.709473pt}}
\pgflineto{\pgfpoint{289.124847pt}{263.709473pt}}
\pgfusepath{stroke}
\pgfpathmoveto{\pgfpoint{280.273529pt}{269.886322pt}}
\pgflineto{\pgfpoint{289.120331pt}{269.886322pt}}
\pgfusepath{stroke}
\pgfpathmoveto{\pgfpoint{280.278015pt}{276.063141pt}}
\pgflineto{\pgfpoint{289.120300pt}{276.063141pt}}
\pgfusepath{stroke}
\pgfpathmoveto{\pgfpoint{280.277985pt}{282.239990pt}}
\pgflineto{\pgfpoint{289.115814pt}{282.239990pt}}
\pgfusepath{stroke}
\pgfpathmoveto{\pgfpoint{280.278046pt}{288.416840pt}}
\pgflineto{\pgfpoint{289.115784pt}{288.416840pt}}
\pgfusepath{stroke}
\pgfpathmoveto{\pgfpoint{280.282532pt}{294.593689pt}}
\pgflineto{\pgfpoint{289.111298pt}{294.593689pt}}
\pgfusepath{stroke}
\pgfpathmoveto{\pgfpoint{280.282532pt}{300.770538pt}}
\pgflineto{\pgfpoint{289.106781pt}{300.770538pt}}
\pgfusepath{stroke}
\pgfpathmoveto{\pgfpoint{280.287018pt}{306.947388pt}}
\pgflineto{\pgfpoint{289.106781pt}{306.947388pt}}
\pgfusepath{stroke}
\pgfpathmoveto{\pgfpoint{280.287018pt}{313.124207pt}}
\pgflineto{\pgfpoint{289.102234pt}{313.124207pt}}
\pgfusepath{stroke}
\pgfpathmoveto{\pgfpoint{289.151978pt}{158.703156pt}}
\pgflineto{\pgfpoint{289.151978pt}{152.526306pt}}
\pgfusepath{stroke}
\pgfpathmoveto{\pgfpoint{289.151978pt}{152.526306pt}}
\pgflineto{\pgfpoint{289.151978pt}{146.349472pt}}
\pgfusepath{stroke}
\pgfpathmoveto{\pgfpoint{289.151978pt}{164.880005pt}}
\pgflineto{\pgfpoint{289.151978pt}{158.703156pt}}
\pgfusepath{stroke}
\pgfpathmoveto{\pgfpoint{289.151978pt}{171.056854pt}}
\pgflineto{\pgfpoint{289.151978pt}{164.880005pt}}
\pgfusepath{stroke}
\pgfpathmoveto{\pgfpoint{289.151978pt}{177.233673pt}}
\pgflineto{\pgfpoint{289.151978pt}{171.056854pt}}
\pgfusepath{stroke}
\pgfpathmoveto{\pgfpoint{289.151978pt}{183.410522pt}}
\pgflineto{\pgfpoint{289.151978pt}{177.233673pt}}
\pgfusepath{stroke}
\pgfpathmoveto{\pgfpoint{289.151978pt}{189.587372pt}}
\pgflineto{\pgfpoint{289.151978pt}{183.410522pt}}
\pgfusepath{stroke}
\pgfpathmoveto{\pgfpoint{289.151978pt}{183.410522pt}}
\pgflineto{\pgfpoint{289.165527pt}{183.410522pt}}
\pgfusepath{stroke}
\pgfpathmoveto{\pgfpoint{289.151978pt}{152.526306pt}}
\pgflineto{\pgfpoint{289.170197pt}{152.526306pt}}
\pgfusepath{stroke}
\pgfpathmoveto{\pgfpoint{289.151978pt}{158.703156pt}}
\pgflineto{\pgfpoint{289.170197pt}{158.703156pt}}
\pgfusepath{stroke}
\pgfpathmoveto{\pgfpoint{289.151978pt}{164.880005pt}}
\pgflineto{\pgfpoint{289.170258pt}{164.880005pt}}
\pgfusepath{stroke}
\pgfpathmoveto{\pgfpoint{289.151978pt}{171.056854pt}}
\pgflineto{\pgfpoint{289.170258pt}{171.056854pt}}
\pgfusepath{stroke}
\pgfpathmoveto{\pgfpoint{289.151978pt}{177.233673pt}}
\pgflineto{\pgfpoint{289.179291pt}{177.233673pt}}
\pgfusepath{stroke}
\pgfpathmoveto{\pgfpoint{289.151978pt}{201.941055pt}}
\pgflineto{\pgfpoint{289.151978pt}{195.764206pt}}
\pgfusepath{stroke}
\pgfpathmoveto{\pgfpoint{289.151978pt}{195.764206pt}}
\pgflineto{\pgfpoint{289.151978pt}{189.587372pt}}
\pgfusepath{stroke}
\pgfpathmoveto{\pgfpoint{289.151978pt}{208.117905pt}}
\pgflineto{\pgfpoint{289.151978pt}{201.941055pt}}
\pgfusepath{stroke}
\pgfpathmoveto{\pgfpoint{289.151978pt}{214.294739pt}}
\pgflineto{\pgfpoint{289.151978pt}{208.117905pt}}
\pgfusepath{stroke}
\pgfpathmoveto{\pgfpoint{289.151978pt}{220.471588pt}}
\pgflineto{\pgfpoint{289.151978pt}{214.294739pt}}
\pgfusepath{stroke}
\pgfpathmoveto{\pgfpoint{289.151978pt}{251.355804pt}}
\pgflineto{\pgfpoint{289.129364pt}{251.355804pt}}
\pgfusepath{stroke}
\pgfpathmoveto{\pgfpoint{289.151978pt}{257.532623pt}}
\pgflineto{\pgfpoint{289.124817pt}{257.532623pt}}
\pgfusepath{stroke}
\pgfpathmoveto{\pgfpoint{289.151978pt}{263.709473pt}}
\pgflineto{\pgfpoint{289.124847pt}{263.709473pt}}
\pgfusepath{stroke}
\pgfpathmoveto{\pgfpoint{289.151978pt}{269.886322pt}}
\pgflineto{\pgfpoint{289.120331pt}{269.886322pt}}
\pgfusepath{stroke}
\pgfpathmoveto{\pgfpoint{289.151978pt}{276.063141pt}}
\pgflineto{\pgfpoint{289.120300pt}{276.063141pt}}
\pgfusepath{stroke}
\pgfpathmoveto{\pgfpoint{289.151978pt}{282.239990pt}}
\pgflineto{\pgfpoint{289.115814pt}{282.239990pt}}
\pgfusepath{stroke}
\pgfpathmoveto{\pgfpoint{289.151978pt}{288.416840pt}}
\pgflineto{\pgfpoint{289.115784pt}{288.416840pt}}
\pgfusepath{stroke}
\pgfpathmoveto{\pgfpoint{289.151978pt}{294.593689pt}}
\pgflineto{\pgfpoint{289.111298pt}{294.593689pt}}
\pgfusepath{stroke}
\pgfpathmoveto{\pgfpoint{289.151978pt}{300.770538pt}}
\pgflineto{\pgfpoint{289.106781pt}{300.770538pt}}
\pgfusepath{stroke}
\pgfpathmoveto{\pgfpoint{289.151978pt}{306.947388pt}}
\pgflineto{\pgfpoint{289.106781pt}{306.947388pt}}
\pgfusepath{stroke}
\pgfpathmoveto{\pgfpoint{289.151978pt}{313.124207pt}}
\pgflineto{\pgfpoint{289.102234pt}{313.124207pt}}
\pgfusepath{stroke}
\pgfpathmoveto{\pgfpoint{289.151978pt}{282.239990pt}}
\pgflineto{\pgfpoint{289.151978pt}{276.063141pt}}
\pgfusepath{stroke}
\pgfpathmoveto{\pgfpoint{289.151978pt}{251.355804pt}}
\pgflineto{\pgfpoint{289.151978pt}{245.178955pt}}
\pgfusepath{stroke}
\pgfpathmoveto{\pgfpoint{289.151978pt}{257.532623pt}}
\pgflineto{\pgfpoint{289.151978pt}{251.355804pt}}
\pgfusepath{stroke}
\pgfpathmoveto{\pgfpoint{289.151978pt}{263.709473pt}}
\pgflineto{\pgfpoint{289.151978pt}{257.532623pt}}
\pgfusepath{stroke}
\pgfpathmoveto{\pgfpoint{289.151978pt}{269.886322pt}}
\pgflineto{\pgfpoint{289.151978pt}{263.709473pt}}
\pgfusepath{stroke}
\pgfpathmoveto{\pgfpoint{289.151978pt}{276.063141pt}}
\pgflineto{\pgfpoint{289.151978pt}{269.886322pt}}
\pgfusepath{stroke}
\pgfpathmoveto{\pgfpoint{289.151978pt}{288.416840pt}}
\pgflineto{\pgfpoint{289.151978pt}{282.239990pt}}
\pgfusepath{stroke}
\pgfpathmoveto{\pgfpoint{289.151978pt}{294.593689pt}}
\pgflineto{\pgfpoint{289.151978pt}{288.416840pt}}
\pgfusepath{stroke}
\pgfpathmoveto{\pgfpoint{289.151978pt}{300.770538pt}}
\pgflineto{\pgfpoint{289.151978pt}{294.593689pt}}
\pgfusepath{stroke}
\pgfpathmoveto{\pgfpoint{289.151978pt}{306.947388pt}}
\pgflineto{\pgfpoint{289.151978pt}{300.770538pt}}
\pgfusepath{stroke}
\pgfpathmoveto{\pgfpoint{289.151978pt}{313.124207pt}}
\pgflineto{\pgfpoint{289.151978pt}{306.947388pt}}
\pgfusepath{stroke}
\pgfpathmoveto{\pgfpoint{289.151978pt}{245.178955pt}}
\pgflineto{\pgfpoint{289.133911pt}{245.178955pt}}
\pgfusepath{stroke}
\pgfpathmoveto{\pgfpoint{289.179321pt}{183.410522pt}}
\pgflineto{\pgfpoint{289.165527pt}{183.410522pt}}
\pgfusepath{stroke}
\pgfpathmoveto{\pgfpoint{289.151978pt}{189.587372pt}}
\pgflineto{\pgfpoint{289.179352pt}{189.587372pt}}
\pgfusepath{stroke}
\pgfpathmoveto{\pgfpoint{289.151978pt}{195.764206pt}}
\pgflineto{\pgfpoint{289.174835pt}{195.764206pt}}
\pgfusepath{stroke}
\pgfpathmoveto{\pgfpoint{289.151978pt}{201.941055pt}}
\pgflineto{\pgfpoint{289.179352pt}{201.941055pt}}
\pgfusepath{stroke}
\pgfpathmoveto{\pgfpoint{289.151978pt}{226.648422pt}}
\pgflineto{\pgfpoint{289.138397pt}{226.648422pt}}
\pgfusepath{stroke}
\pgfpathmoveto{\pgfpoint{289.151978pt}{232.825272pt}}
\pgflineto{\pgfpoint{289.138428pt}{232.825272pt}}
\pgfusepath{stroke}
\pgfpathmoveto{\pgfpoint{289.151978pt}{239.002106pt}}
\pgflineto{\pgfpoint{289.133911pt}{239.002106pt}}
\pgfusepath{stroke}
\pgfpathmoveto{\pgfpoint{289.151978pt}{232.825272pt}}
\pgflineto{\pgfpoint{289.151978pt}{226.648422pt}}
\pgfusepath{stroke}
\pgfpathmoveto{\pgfpoint{289.151978pt}{226.648422pt}}
\pgflineto{\pgfpoint{289.151978pt}{220.471588pt}}
\pgfusepath{stroke}
\pgfpathmoveto{\pgfpoint{289.151978pt}{239.002106pt}}
\pgflineto{\pgfpoint{289.151978pt}{232.825272pt}}
\pgfusepath{stroke}
\pgfpathmoveto{\pgfpoint{289.151978pt}{245.178955pt}}
\pgflineto{\pgfpoint{289.151978pt}{239.002106pt}}
\pgfusepath{stroke}
\pgfpathmoveto{\pgfpoint{289.151978pt}{208.117905pt}}
\pgflineto{\pgfpoint{289.179413pt}{208.117905pt}}
\pgfusepath{stroke}
\pgfpathmoveto{\pgfpoint{289.151978pt}{214.294739pt}}
\pgflineto{\pgfpoint{289.183929pt}{214.294739pt}}
\pgfusepath{stroke}
\pgfpathmoveto{\pgfpoint{289.151978pt}{220.471588pt}}
\pgflineto{\pgfpoint{289.183960pt}{220.471588pt}}
\pgfusepath{stroke}
\pgfpathmoveto{\pgfpoint{289.151978pt}{226.648422pt}}
\pgflineto{\pgfpoint{289.183960pt}{226.648422pt}}
\pgfusepath{stroke}
\pgfpathmoveto{\pgfpoint{289.151978pt}{232.825272pt}}
\pgflineto{\pgfpoint{289.188477pt}{232.825272pt}}
\pgfusepath{stroke}
\pgfpathmoveto{\pgfpoint{289.151978pt}{239.002106pt}}
\pgflineto{\pgfpoint{289.188507pt}{239.002106pt}}
\pgfusepath{stroke}
\pgfpathmoveto{\pgfpoint{289.151978pt}{245.178955pt}}
\pgflineto{\pgfpoint{289.193054pt}{245.178955pt}}
\pgfusepath{stroke}
\pgfpathmoveto{\pgfpoint{289.151978pt}{251.355804pt}}
\pgflineto{\pgfpoint{289.193085pt}{251.355804pt}}
\pgfusepath{stroke}
\pgfpathmoveto{\pgfpoint{289.151978pt}{257.532623pt}}
\pgflineto{\pgfpoint{289.193115pt}{257.532623pt}}
\pgfusepath{stroke}
\pgfpathmoveto{\pgfpoint{289.151978pt}{263.709473pt}}
\pgflineto{\pgfpoint{289.197601pt}{263.709473pt}}
\pgfusepath{stroke}
\pgfpathmoveto{\pgfpoint{289.151978pt}{269.886322pt}}
\pgflineto{\pgfpoint{289.197632pt}{269.886322pt}}
\pgfusepath{stroke}
\pgfpathmoveto{\pgfpoint{289.151978pt}{276.063141pt}}
\pgflineto{\pgfpoint{289.202148pt}{276.063141pt}}
\pgfusepath{stroke}
\pgfpathmoveto{\pgfpoint{289.151978pt}{282.239990pt}}
\pgflineto{\pgfpoint{289.202179pt}{282.239990pt}}
\pgfusepath{stroke}
\pgfpathmoveto{\pgfpoint{289.151978pt}{288.416840pt}}
\pgflineto{\pgfpoint{289.206726pt}{288.416840pt}}
\pgfusepath{stroke}
\pgfpathmoveto{\pgfpoint{289.151978pt}{294.593689pt}}
\pgflineto{\pgfpoint{289.206726pt}{294.593689pt}}
\pgfusepath{stroke}
\pgfpathmoveto{\pgfpoint{289.151978pt}{300.770538pt}}
\pgflineto{\pgfpoint{289.206757pt}{300.770538pt}}
\pgfusepath{stroke}
\pgfpathmoveto{\pgfpoint{289.151978pt}{306.947388pt}}
\pgflineto{\pgfpoint{289.211304pt}{306.947388pt}}
\pgfusepath{stroke}
\pgfpathmoveto{\pgfpoint{289.151978pt}{96.934731pt}}
\pgflineto{\pgfpoint{289.151978pt}{90.757896pt}}
\pgfusepath{stroke}
\pgfpathmoveto{\pgfpoint{289.151978pt}{66.050522pt}}
\pgflineto{\pgfpoint{289.151978pt}{59.873672pt}}
\pgfusepath{stroke}
\pgfpathmoveto{\pgfpoint{289.151978pt}{72.227356pt}}
\pgflineto{\pgfpoint{289.151978pt}{66.050522pt}}
\pgfusepath{stroke}
\pgfpathmoveto{\pgfpoint{289.151978pt}{78.404205pt}}
\pgflineto{\pgfpoint{289.151978pt}{72.227356pt}}
\pgfusepath{stroke}
\pgfpathmoveto{\pgfpoint{289.151978pt}{84.581039pt}}
\pgflineto{\pgfpoint{289.151978pt}{78.404205pt}}
\pgfusepath{stroke}
\pgfpathmoveto{\pgfpoint{289.151978pt}{90.757896pt}}
\pgflineto{\pgfpoint{289.151978pt}{84.581039pt}}
\pgfusepath{stroke}
\pgfpathmoveto{\pgfpoint{289.151978pt}{103.111580pt}}
\pgflineto{\pgfpoint{289.151978pt}{96.934731pt}}
\pgfusepath{stroke}
\pgfpathmoveto{\pgfpoint{289.151978pt}{109.288422pt}}
\pgflineto{\pgfpoint{289.151978pt}{103.111580pt}}
\pgfusepath{stroke}
\pgfpathmoveto{\pgfpoint{289.151978pt}{115.465263pt}}
\pgflineto{\pgfpoint{289.151978pt}{109.288422pt}}
\pgfusepath{stroke}
\pgfpathmoveto{\pgfpoint{289.151978pt}{121.642097pt}}
\pgflineto{\pgfpoint{289.151978pt}{115.465263pt}}
\pgfusepath{stroke}
\pgfpathmoveto{\pgfpoint{289.151978pt}{127.818947pt}}
\pgflineto{\pgfpoint{289.151978pt}{121.642097pt}}
\pgfusepath{stroke}
\pgfpathmoveto{\pgfpoint{289.151978pt}{133.995789pt}}
\pgflineto{\pgfpoint{289.151978pt}{127.818947pt}}
\pgfusepath{stroke}
\pgfpathmoveto{\pgfpoint{289.151978pt}{140.172638pt}}
\pgflineto{\pgfpoint{289.151978pt}{133.995789pt}}
\pgfusepath{stroke}
\pgfpathmoveto{\pgfpoint{289.151978pt}{146.349472pt}}
\pgflineto{\pgfpoint{289.151978pt}{140.172638pt}}
\pgfusepath{stroke}
\pgfpathmoveto{\pgfpoint{289.151978pt}{47.519989pt}}
\pgflineto{\pgfpoint{289.133789pt}{47.519989pt}}
\pgfusepath{stroke}
\pgfpathmoveto{\pgfpoint{289.151978pt}{53.696838pt}}
\pgflineto{\pgfpoint{289.133789pt}{53.696838pt}}
\pgfusepath{stroke}
\pgfpathmoveto{\pgfpoint{289.151978pt}{59.873672pt}}
\pgflineto{\pgfpoint{289.151978pt}{53.696838pt}}
\pgfusepath{stroke}
\pgfpathmoveto{\pgfpoint{289.151978pt}{53.696838pt}}
\pgflineto{\pgfpoint{289.151978pt}{47.519989pt}}
\pgfusepath{stroke}
\pgfpathmoveto{\pgfpoint{289.151978pt}{47.519989pt}}
\pgflineto{\pgfpoint{298.061829pt}{47.519989pt}}
\pgfusepath{stroke}
\pgfpathmoveto{\pgfpoint{289.151978pt}{53.696838pt}}
\pgflineto{\pgfpoint{298.061859pt}{53.696838pt}}
\pgfusepath{stroke}
\pgfpathmoveto{\pgfpoint{289.151978pt}{59.873672pt}}
\pgflineto{\pgfpoint{298.061859pt}{59.873672pt}}
\pgfusepath{stroke}
\pgfpathmoveto{\pgfpoint{289.151978pt}{66.050522pt}}
\pgflineto{\pgfpoint{298.061859pt}{66.050522pt}}
\pgfusepath{stroke}
\pgfpathmoveto{\pgfpoint{298.079987pt}{72.227356pt}}
\pgflineto{\pgfpoint{289.151978pt}{72.227356pt}}
\pgfusepath{stroke}
\pgfpathmoveto{\pgfpoint{298.079987pt}{78.404205pt}}
\pgflineto{\pgfpoint{289.151978pt}{78.404205pt}}
\pgfusepath{stroke}
\pgfpathmoveto{\pgfpoint{298.079987pt}{84.581039pt}}
\pgflineto{\pgfpoint{289.151978pt}{84.581039pt}}
\pgfusepath{stroke}
\pgfpathmoveto{\pgfpoint{298.079987pt}{90.757896pt}}
\pgflineto{\pgfpoint{289.151978pt}{90.757896pt}}
\pgfusepath{stroke}
\pgfpathmoveto{\pgfpoint{298.079987pt}{96.934731pt}}
\pgflineto{\pgfpoint{289.151978pt}{96.934731pt}}
\pgfusepath{stroke}
\pgfpathmoveto{\pgfpoint{298.079987pt}{103.111580pt}}
\pgflineto{\pgfpoint{289.151978pt}{103.111580pt}}
\pgfusepath{stroke}
\pgfpathmoveto{\pgfpoint{298.079987pt}{109.288422pt}}
\pgflineto{\pgfpoint{289.151978pt}{109.288422pt}}
\pgfusepath{stroke}
\pgfpathmoveto{\pgfpoint{298.079987pt}{115.465263pt}}
\pgflineto{\pgfpoint{289.151978pt}{115.465263pt}}
\pgfusepath{stroke}
\pgfpathmoveto{\pgfpoint{298.079987pt}{121.642097pt}}
\pgflineto{\pgfpoint{289.151978pt}{121.642097pt}}
\pgfusepath{stroke}
\pgfpathmoveto{\pgfpoint{298.079987pt}{127.818947pt}}
\pgflineto{\pgfpoint{289.151978pt}{127.818947pt}}
\pgfusepath{stroke}
\pgfpathmoveto{\pgfpoint{298.079987pt}{133.995789pt}}
\pgflineto{\pgfpoint{289.151978pt}{133.995789pt}}
\pgfusepath{stroke}
\pgfpathmoveto{\pgfpoint{298.079987pt}{140.172638pt}}
\pgflineto{\pgfpoint{289.151978pt}{140.172638pt}}
\pgfusepath{stroke}
\pgfpathmoveto{\pgfpoint{298.079987pt}{146.349472pt}}
\pgflineto{\pgfpoint{289.151978pt}{146.349472pt}}
\pgfusepath{stroke}
\pgfpathmoveto{\pgfpoint{298.079987pt}{152.526306pt}}
\pgflineto{\pgfpoint{289.170197pt}{152.526306pt}}
\pgfusepath{stroke}
\pgfpathmoveto{\pgfpoint{298.079987pt}{158.703156pt}}
\pgflineto{\pgfpoint{289.170197pt}{158.703156pt}}
\pgfusepath{stroke}
\pgfpathmoveto{\pgfpoint{298.079987pt}{164.880005pt}}
\pgflineto{\pgfpoint{289.170258pt}{164.880005pt}}
\pgfusepath{stroke}
\pgfpathmoveto{\pgfpoint{298.079987pt}{171.056854pt}}
\pgflineto{\pgfpoint{289.170258pt}{171.056854pt}}
\pgfusepath{stroke}
\pgfpathmoveto{\pgfpoint{298.079987pt}{177.233673pt}}
\pgflineto{\pgfpoint{289.179291pt}{177.233673pt}}
\pgfusepath{stroke}
\pgfpathmoveto{\pgfpoint{298.079987pt}{183.410522pt}}
\pgflineto{\pgfpoint{289.179321pt}{183.410522pt}}
\pgfusepath{stroke}
\pgfpathmoveto{\pgfpoint{298.079987pt}{189.587372pt}}
\pgflineto{\pgfpoint{289.179352pt}{189.587372pt}}
\pgfusepath{stroke}
\pgfpathmoveto{\pgfpoint{298.079987pt}{195.764206pt}}
\pgflineto{\pgfpoint{289.174835pt}{195.764206pt}}
\pgfusepath{stroke}
\pgfpathmoveto{\pgfpoint{298.079987pt}{201.941055pt}}
\pgflineto{\pgfpoint{289.179352pt}{201.941055pt}}
\pgfusepath{stroke}
\pgfpathmoveto{\pgfpoint{298.079987pt}{208.117905pt}}
\pgflineto{\pgfpoint{289.179413pt}{208.117905pt}}
\pgfusepath{stroke}
\pgfpathmoveto{\pgfpoint{298.079987pt}{214.294739pt}}
\pgflineto{\pgfpoint{289.183929pt}{214.294739pt}}
\pgfusepath{stroke}
\pgfpathmoveto{\pgfpoint{298.079987pt}{220.471588pt}}
\pgflineto{\pgfpoint{289.183960pt}{220.471588pt}}
\pgfusepath{stroke}
\pgfpathmoveto{\pgfpoint{298.079987pt}{226.648422pt}}
\pgflineto{\pgfpoint{289.183960pt}{226.648422pt}}
\pgfusepath{stroke}
\pgfpathmoveto{\pgfpoint{298.079987pt}{232.825272pt}}
\pgflineto{\pgfpoint{289.188477pt}{232.825272pt}}
\pgfusepath{stroke}
\pgfpathmoveto{\pgfpoint{289.188507pt}{239.002106pt}}
\pgflineto{\pgfpoint{298.066528pt}{239.002106pt}}
\pgfusepath{stroke}
\pgfpathmoveto{\pgfpoint{289.193054pt}{245.178955pt}}
\pgflineto{\pgfpoint{298.066528pt}{245.178955pt}}
\pgfusepath{stroke}
\pgfpathmoveto{\pgfpoint{289.193085pt}{251.355804pt}}
\pgflineto{\pgfpoint{298.062042pt}{251.355804pt}}
\pgfusepath{stroke}
\pgfpathmoveto{\pgfpoint{289.193115pt}{257.532623pt}}
\pgflineto{\pgfpoint{298.062012pt}{257.532623pt}}
\pgfusepath{stroke}
\pgfpathmoveto{\pgfpoint{289.197601pt}{263.709473pt}}
\pgflineto{\pgfpoint{298.057556pt}{263.709473pt}}
\pgfusepath{stroke}
\pgfpathmoveto{\pgfpoint{289.197632pt}{269.886322pt}}
\pgflineto{\pgfpoint{298.053040pt}{269.886322pt}}
\pgfusepath{stroke}
\pgfpathmoveto{\pgfpoint{289.202148pt}{276.063141pt}}
\pgflineto{\pgfpoint{298.053009pt}{276.063141pt}}
\pgfusepath{stroke}
\pgfpathmoveto{\pgfpoint{289.202179pt}{282.239990pt}}
\pgflineto{\pgfpoint{298.048523pt}{282.239990pt}}
\pgfusepath{stroke}
\pgfpathmoveto{\pgfpoint{289.206726pt}{288.416840pt}}
\pgflineto{\pgfpoint{298.048523pt}{288.416840pt}}
\pgfusepath{stroke}
\pgfpathmoveto{\pgfpoint{289.206726pt}{294.593689pt}}
\pgflineto{\pgfpoint{298.044037pt}{294.593689pt}}
\pgfusepath{stroke}
\pgfpathmoveto{\pgfpoint{289.206757pt}{300.770538pt}}
\pgflineto{\pgfpoint{298.039551pt}{300.770538pt}}
\pgfusepath{stroke}
\pgfpathmoveto{\pgfpoint{289.211304pt}{306.947388pt}}
\pgflineto{\pgfpoint{298.039581pt}{306.947388pt}}
\pgfusepath{stroke}
\pgfpathmoveto{\pgfpoint{298.079987pt}{171.056854pt}}
\pgflineto{\pgfpoint{298.079987pt}{164.880005pt}}
\pgfusepath{stroke}
\pgfpathmoveto{\pgfpoint{298.079987pt}{164.880005pt}}
\pgflineto{\pgfpoint{298.079987pt}{158.703156pt}}
\pgfusepath{stroke}
\pgfpathmoveto{\pgfpoint{298.079987pt}{177.233673pt}}
\pgflineto{\pgfpoint{298.079987pt}{171.056854pt}}
\pgfusepath{stroke}
\pgfpathmoveto{\pgfpoint{298.079987pt}{183.410522pt}}
\pgflineto{\pgfpoint{298.079987pt}{177.233673pt}}
\pgfusepath{stroke}
\pgfpathmoveto{\pgfpoint{298.079987pt}{189.587372pt}}
\pgflineto{\pgfpoint{298.079987pt}{183.410522pt}}
\pgfusepath{stroke}
\pgfpathmoveto{\pgfpoint{298.079987pt}{164.880005pt}}
\pgflineto{\pgfpoint{298.098145pt}{164.880005pt}}
\pgfusepath{stroke}
\pgfpathmoveto{\pgfpoint{298.079987pt}{171.056854pt}}
\pgflineto{\pgfpoint{298.098145pt}{171.056854pt}}
\pgfusepath{stroke}
\pgfpathmoveto{\pgfpoint{298.079987pt}{177.233673pt}}
\pgflineto{\pgfpoint{298.098145pt}{177.233673pt}}
\pgfusepath{stroke}
\pgfpathmoveto{\pgfpoint{298.079987pt}{183.410522pt}}
\pgflineto{\pgfpoint{298.098145pt}{183.410522pt}}
\pgfusepath{stroke}
\pgfpathmoveto{\pgfpoint{298.079987pt}{214.294739pt}}
\pgflineto{\pgfpoint{298.079987pt}{208.117905pt}}
\pgfusepath{stroke}
\pgfpathmoveto{\pgfpoint{298.079987pt}{195.764206pt}}
\pgflineto{\pgfpoint{298.079987pt}{189.587372pt}}
\pgfusepath{stroke}
\pgfpathmoveto{\pgfpoint{298.079987pt}{201.941055pt}}
\pgflineto{\pgfpoint{298.079987pt}{195.764206pt}}
\pgfusepath{stroke}
\pgfpathmoveto{\pgfpoint{298.079987pt}{208.117905pt}}
\pgflineto{\pgfpoint{298.079987pt}{201.941055pt}}
\pgfusepath{stroke}
\pgfpathmoveto{\pgfpoint{298.079987pt}{220.471588pt}}
\pgflineto{\pgfpoint{298.079987pt}{214.294739pt}}
\pgfusepath{stroke}
\pgfpathmoveto{\pgfpoint{298.079987pt}{226.648422pt}}
\pgflineto{\pgfpoint{298.079987pt}{220.471588pt}}
\pgfusepath{stroke}
\pgfpathmoveto{\pgfpoint{298.079987pt}{232.825272pt}}
\pgflineto{\pgfpoint{298.079987pt}{226.648422pt}}
\pgfusepath{stroke}
\pgfpathmoveto{\pgfpoint{298.079987pt}{189.587372pt}}
\pgflineto{\pgfpoint{298.098175pt}{189.587372pt}}
\pgfusepath{stroke}
\pgfpathmoveto{\pgfpoint{298.079987pt}{251.355804pt}}
\pgflineto{\pgfpoint{298.062042pt}{251.355804pt}}
\pgfusepath{stroke}
\pgfpathmoveto{\pgfpoint{298.079987pt}{257.532623pt}}
\pgflineto{\pgfpoint{298.062012pt}{257.532623pt}}
\pgfusepath{stroke}
\pgfpathmoveto{\pgfpoint{298.079987pt}{263.709473pt}}
\pgflineto{\pgfpoint{298.057556pt}{263.709473pt}}
\pgfusepath{stroke}
\pgfpathmoveto{\pgfpoint{298.079987pt}{269.886322pt}}
\pgflineto{\pgfpoint{298.053040pt}{269.886322pt}}
\pgfusepath{stroke}
\pgfpathmoveto{\pgfpoint{298.079987pt}{276.063141pt}}
\pgflineto{\pgfpoint{298.053009pt}{276.063141pt}}
\pgfusepath{stroke}
\pgfpathmoveto{\pgfpoint{298.079987pt}{282.239990pt}}
\pgflineto{\pgfpoint{298.048523pt}{282.239990pt}}
\pgfusepath{stroke}
\pgfpathmoveto{\pgfpoint{298.079987pt}{288.416840pt}}
\pgflineto{\pgfpoint{298.048523pt}{288.416840pt}}
\pgfusepath{stroke}
\pgfpathmoveto{\pgfpoint{298.079987pt}{294.593689pt}}
\pgflineto{\pgfpoint{298.044037pt}{294.593689pt}}
\pgfusepath{stroke}
\pgfpathmoveto{\pgfpoint{298.079987pt}{300.770538pt}}
\pgflineto{\pgfpoint{298.039551pt}{300.770538pt}}
\pgfusepath{stroke}
\pgfpathmoveto{\pgfpoint{298.079987pt}{306.947388pt}}
\pgflineto{\pgfpoint{298.039581pt}{306.947388pt}}
\pgfusepath{stroke}
\pgfpathmoveto{\pgfpoint{298.079987pt}{306.947388pt}}
\pgflineto{\pgfpoint{298.079987pt}{300.770538pt}}
\pgfusepath{stroke}
\pgfpathmoveto{\pgfpoint{298.079987pt}{263.709473pt}}
\pgflineto{\pgfpoint{298.079987pt}{257.532623pt}}
\pgfusepath{stroke}
\pgfpathmoveto{\pgfpoint{298.079987pt}{239.002106pt}}
\pgflineto{\pgfpoint{298.079987pt}{232.825272pt}}
\pgfusepath{stroke}
\pgfpathmoveto{\pgfpoint{298.079987pt}{245.178955pt}}
\pgflineto{\pgfpoint{298.079987pt}{239.002106pt}}
\pgfusepath{stroke}
\pgfpathmoveto{\pgfpoint{298.079987pt}{251.355804pt}}
\pgflineto{\pgfpoint{298.079987pt}{245.178955pt}}
\pgfusepath{stroke}
\pgfpathmoveto{\pgfpoint{298.079987pt}{257.532623pt}}
\pgflineto{\pgfpoint{298.079987pt}{251.355804pt}}
\pgfusepath{stroke}
\pgfpathmoveto{\pgfpoint{298.079987pt}{269.886322pt}}
\pgflineto{\pgfpoint{298.079987pt}{263.709473pt}}
\pgfusepath{stroke}
\pgfpathmoveto{\pgfpoint{298.079987pt}{276.063141pt}}
\pgflineto{\pgfpoint{298.079987pt}{269.886322pt}}
\pgfusepath{stroke}
\pgfpathmoveto{\pgfpoint{298.079987pt}{282.239990pt}}
\pgflineto{\pgfpoint{298.079987pt}{276.063141pt}}
\pgfusepath{stroke}
\pgfpathmoveto{\pgfpoint{298.079987pt}{288.416840pt}}
\pgflineto{\pgfpoint{298.079987pt}{282.239990pt}}
\pgfusepath{stroke}
\pgfpathmoveto{\pgfpoint{298.079987pt}{294.593689pt}}
\pgflineto{\pgfpoint{298.079987pt}{288.416840pt}}
\pgfusepath{stroke}
\pgfpathmoveto{\pgfpoint{298.079987pt}{300.770538pt}}
\pgflineto{\pgfpoint{298.079987pt}{294.593689pt}}
\pgfusepath{stroke}
\pgfpathmoveto{\pgfpoint{298.079987pt}{239.002106pt}}
\pgflineto{\pgfpoint{298.066528pt}{239.002106pt}}
\pgfusepath{stroke}
\pgfpathmoveto{\pgfpoint{298.079987pt}{245.178955pt}}
\pgflineto{\pgfpoint{298.066528pt}{245.178955pt}}
\pgfusepath{stroke}
\pgfpathmoveto{\pgfpoint{298.079987pt}{195.764206pt}}
\pgflineto{\pgfpoint{298.098175pt}{195.764206pt}}
\pgfusepath{stroke}
\pgfpathmoveto{\pgfpoint{298.079987pt}{201.941055pt}}
\pgflineto{\pgfpoint{298.102692pt}{201.941055pt}}
\pgfusepath{stroke}
\pgfpathmoveto{\pgfpoint{298.079987pt}{208.117905pt}}
\pgflineto{\pgfpoint{298.102722pt}{208.117905pt}}
\pgfusepath{stroke}
\pgfpathmoveto{\pgfpoint{298.079987pt}{214.294739pt}}
\pgflineto{\pgfpoint{298.107239pt}{214.294739pt}}
\pgfusepath{stroke}
\pgfpathmoveto{\pgfpoint{298.079987pt}{220.471588pt}}
\pgflineto{\pgfpoint{298.107239pt}{220.471588pt}}
\pgfusepath{stroke}
\pgfpathmoveto{\pgfpoint{298.079987pt}{226.648422pt}}
\pgflineto{\pgfpoint{298.107269pt}{226.648422pt}}
\pgfusepath{stroke}
\pgfpathmoveto{\pgfpoint{298.079987pt}{232.825272pt}}
\pgflineto{\pgfpoint{298.111786pt}{232.825272pt}}
\pgfusepath{stroke}
\pgfpathmoveto{\pgfpoint{298.079987pt}{239.002106pt}}
\pgflineto{\pgfpoint{298.111816pt}{239.002106pt}}
\pgfusepath{stroke}
\pgfpathmoveto{\pgfpoint{298.079987pt}{245.178955pt}}
\pgflineto{\pgfpoint{298.116333pt}{245.178955pt}}
\pgfusepath{stroke}
\pgfpathmoveto{\pgfpoint{298.079987pt}{251.355804pt}}
\pgflineto{\pgfpoint{298.116364pt}{251.355804pt}}
\pgfusepath{stroke}
\pgfpathmoveto{\pgfpoint{298.079987pt}{257.532623pt}}
\pgflineto{\pgfpoint{298.120850pt}{257.532623pt}}
\pgfusepath{stroke}
\pgfpathmoveto{\pgfpoint{298.079987pt}{263.709473pt}}
\pgflineto{\pgfpoint{298.120880pt}{263.709473pt}}
\pgfusepath{stroke}
\pgfpathmoveto{\pgfpoint{298.079987pt}{269.886322pt}}
\pgflineto{\pgfpoint{298.120880pt}{269.886322pt}}
\pgfusepath{stroke}
\pgfpathmoveto{\pgfpoint{298.079987pt}{276.063141pt}}
\pgflineto{\pgfpoint{298.125397pt}{276.063141pt}}
\pgfusepath{stroke}
\pgfpathmoveto{\pgfpoint{298.079987pt}{282.239990pt}}
\pgflineto{\pgfpoint{298.125397pt}{282.239990pt}}
\pgfusepath{stroke}
\pgfpathmoveto{\pgfpoint{298.079987pt}{288.416840pt}}
\pgflineto{\pgfpoint{298.129944pt}{288.416840pt}}
\pgfusepath{stroke}
\pgfpathmoveto{\pgfpoint{298.079987pt}{294.593689pt}}
\pgflineto{\pgfpoint{298.129944pt}{294.593689pt}}
\pgfusepath{stroke}
\pgfpathmoveto{\pgfpoint{298.079987pt}{300.770538pt}}
\pgflineto{\pgfpoint{298.130005pt}{300.770538pt}}
\pgfusepath{stroke}
\pgfpathmoveto{\pgfpoint{298.079987pt}{109.288422pt}}
\pgflineto{\pgfpoint{298.079987pt}{103.111580pt}}
\pgfusepath{stroke}
\pgfpathmoveto{\pgfpoint{298.079987pt}{78.404205pt}}
\pgflineto{\pgfpoint{298.079987pt}{72.227356pt}}
\pgfusepath{stroke}
\pgfpathmoveto{\pgfpoint{298.079987pt}{84.581039pt}}
\pgflineto{\pgfpoint{298.079987pt}{78.404205pt}}
\pgfusepath{stroke}
\pgfpathmoveto{\pgfpoint{298.079987pt}{90.757896pt}}
\pgflineto{\pgfpoint{298.079987pt}{84.581039pt}}
\pgfusepath{stroke}
\pgfpathmoveto{\pgfpoint{298.079987pt}{96.934731pt}}
\pgflineto{\pgfpoint{298.079987pt}{90.757896pt}}
\pgfusepath{stroke}
\pgfpathmoveto{\pgfpoint{298.079987pt}{103.111580pt}}
\pgflineto{\pgfpoint{298.079987pt}{96.934731pt}}
\pgfusepath{stroke}
\pgfpathmoveto{\pgfpoint{298.079987pt}{115.465263pt}}
\pgflineto{\pgfpoint{298.079987pt}{109.288422pt}}
\pgfusepath{stroke}
\pgfpathmoveto{\pgfpoint{298.079987pt}{121.642097pt}}
\pgflineto{\pgfpoint{298.079987pt}{115.465263pt}}
\pgfusepath{stroke}
\pgfpathmoveto{\pgfpoint{298.079987pt}{127.818947pt}}
\pgflineto{\pgfpoint{298.079987pt}{121.642097pt}}
\pgfusepath{stroke}
\pgfpathmoveto{\pgfpoint{298.079987pt}{133.995789pt}}
\pgflineto{\pgfpoint{298.079987pt}{127.818947pt}}
\pgfusepath{stroke}
\pgfpathmoveto{\pgfpoint{298.079987pt}{140.172638pt}}
\pgflineto{\pgfpoint{298.079987pt}{133.995789pt}}
\pgfusepath{stroke}
\pgfpathmoveto{\pgfpoint{298.079987pt}{146.349472pt}}
\pgflineto{\pgfpoint{298.079987pt}{140.172638pt}}
\pgfusepath{stroke}
\pgfpathmoveto{\pgfpoint{298.079987pt}{152.526306pt}}
\pgflineto{\pgfpoint{298.079987pt}{146.349472pt}}
\pgfusepath{stroke}
\pgfpathmoveto{\pgfpoint{298.079987pt}{158.703156pt}}
\pgflineto{\pgfpoint{298.079987pt}{152.526306pt}}
\pgfusepath{stroke}
\pgfpathmoveto{\pgfpoint{298.079987pt}{47.519989pt}}
\pgflineto{\pgfpoint{298.061829pt}{47.519989pt}}
\pgfusepath{stroke}
\pgfpathmoveto{\pgfpoint{298.079987pt}{53.696838pt}}
\pgflineto{\pgfpoint{298.061859pt}{53.696838pt}}
\pgfusepath{stroke}
\pgfpathmoveto{\pgfpoint{298.079987pt}{59.873672pt}}
\pgflineto{\pgfpoint{298.061859pt}{59.873672pt}}
\pgfusepath{stroke}
\pgfpathmoveto{\pgfpoint{298.079987pt}{66.050522pt}}
\pgflineto{\pgfpoint{298.061859pt}{66.050522pt}}
\pgfusepath{stroke}
\pgfpathmoveto{\pgfpoint{298.079987pt}{53.696838pt}}
\pgflineto{\pgfpoint{298.079987pt}{47.519989pt}}
\pgfusepath{stroke}
\pgfpathmoveto{\pgfpoint{298.079987pt}{59.873672pt}}
\pgflineto{\pgfpoint{298.079987pt}{53.696838pt}}
\pgfusepath{stroke}
\pgfpathmoveto{\pgfpoint{298.079987pt}{47.519989pt}}
\pgflineto{\pgfpoint{298.107056pt}{47.519989pt}}
\pgfusepath{stroke}
\pgfpathmoveto{\pgfpoint{298.079987pt}{53.696838pt}}
\pgflineto{\pgfpoint{298.098083pt}{53.696838pt}}
\pgfusepath{stroke}
\pgfpathmoveto{\pgfpoint{298.079987pt}{72.227356pt}}
\pgflineto{\pgfpoint{298.079987pt}{66.050522pt}}
\pgfusepath{stroke}
\pgfpathmoveto{\pgfpoint{298.079987pt}{66.050522pt}}
\pgflineto{\pgfpoint{298.079987pt}{59.873672pt}}
\pgfusepath{stroke}
\pgfpathmoveto{\pgfpoint{298.107056pt}{47.519989pt}}
\pgflineto{\pgfpoint{306.989777pt}{47.519989pt}}
\pgfusepath{stroke}
\pgfpathmoveto{\pgfpoint{298.098083pt}{53.696838pt}}
\pgflineto{\pgfpoint{306.989807pt}{53.696838pt}}
\pgfusepath{stroke}
\pgfpathmoveto{\pgfpoint{307.007965pt}{59.873672pt}}
\pgflineto{\pgfpoint{298.079987pt}{59.873672pt}}
\pgfusepath{stroke}
\pgfpathmoveto{\pgfpoint{307.007965pt}{66.050522pt}}
\pgflineto{\pgfpoint{298.079987pt}{66.050522pt}}
\pgfusepath{stroke}
\pgfpathmoveto{\pgfpoint{307.007965pt}{72.227356pt}}
\pgflineto{\pgfpoint{298.079987pt}{72.227356pt}}
\pgfusepath{stroke}
\pgfpathmoveto{\pgfpoint{307.007965pt}{78.404205pt}}
\pgflineto{\pgfpoint{298.079987pt}{78.404205pt}}
\pgfusepath{stroke}
\pgfpathmoveto{\pgfpoint{307.007965pt}{84.581039pt}}
\pgflineto{\pgfpoint{298.079987pt}{84.581039pt}}
\pgfusepath{stroke}
\pgfpathmoveto{\pgfpoint{307.007965pt}{90.757896pt}}
\pgflineto{\pgfpoint{298.079987pt}{90.757896pt}}
\pgfusepath{stroke}
\pgfpathmoveto{\pgfpoint{307.007965pt}{96.934731pt}}
\pgflineto{\pgfpoint{298.079987pt}{96.934731pt}}
\pgfusepath{stroke}
\pgfpathmoveto{\pgfpoint{307.007965pt}{103.111580pt}}
\pgflineto{\pgfpoint{298.079987pt}{103.111580pt}}
\pgfusepath{stroke}
\pgfpathmoveto{\pgfpoint{307.007965pt}{109.288422pt}}
\pgflineto{\pgfpoint{298.079987pt}{109.288422pt}}
\pgfusepath{stroke}
\pgfpathmoveto{\pgfpoint{307.007965pt}{115.465263pt}}
\pgflineto{\pgfpoint{298.079987pt}{115.465263pt}}
\pgfusepath{stroke}
\pgfpathmoveto{\pgfpoint{307.007965pt}{121.642097pt}}
\pgflineto{\pgfpoint{298.079987pt}{121.642097pt}}
\pgfusepath{stroke}
\pgfpathmoveto{\pgfpoint{307.007965pt}{127.818947pt}}
\pgflineto{\pgfpoint{298.079987pt}{127.818947pt}}
\pgfusepath{stroke}
\pgfpathmoveto{\pgfpoint{307.007965pt}{133.995789pt}}
\pgflineto{\pgfpoint{298.079987pt}{133.995789pt}}
\pgfusepath{stroke}
\pgfpathmoveto{\pgfpoint{307.007965pt}{140.172638pt}}
\pgflineto{\pgfpoint{298.079987pt}{140.172638pt}}
\pgfusepath{stroke}
\pgfpathmoveto{\pgfpoint{307.007965pt}{146.349472pt}}
\pgflineto{\pgfpoint{298.079987pt}{146.349472pt}}
\pgfusepath{stroke}
\pgfpathmoveto{\pgfpoint{307.007965pt}{152.526306pt}}
\pgflineto{\pgfpoint{298.079987pt}{152.526306pt}}
\pgfusepath{stroke}
\pgfpathmoveto{\pgfpoint{307.007965pt}{158.703156pt}}
\pgflineto{\pgfpoint{298.079987pt}{158.703156pt}}
\pgfusepath{stroke}
\pgfpathmoveto{\pgfpoint{307.007965pt}{164.880005pt}}
\pgflineto{\pgfpoint{298.098145pt}{164.880005pt}}
\pgfusepath{stroke}
\pgfpathmoveto{\pgfpoint{307.007965pt}{171.056854pt}}
\pgflineto{\pgfpoint{298.098145pt}{171.056854pt}}
\pgfusepath{stroke}
\pgfpathmoveto{\pgfpoint{307.007965pt}{177.233673pt}}
\pgflineto{\pgfpoint{298.098145pt}{177.233673pt}}
\pgfusepath{stroke}
\pgfpathmoveto{\pgfpoint{307.007965pt}{183.410522pt}}
\pgflineto{\pgfpoint{298.098145pt}{183.410522pt}}
\pgfusepath{stroke}
\pgfpathmoveto{\pgfpoint{307.007965pt}{189.587372pt}}
\pgflineto{\pgfpoint{298.098175pt}{189.587372pt}}
\pgfusepath{stroke}
\pgfpathmoveto{\pgfpoint{307.007965pt}{195.764206pt}}
\pgflineto{\pgfpoint{298.098175pt}{195.764206pt}}
\pgfusepath{stroke}
\pgfpathmoveto{\pgfpoint{307.007965pt}{201.941055pt}}
\pgflineto{\pgfpoint{298.102692pt}{201.941055pt}}
\pgfusepath{stroke}
\pgfpathmoveto{\pgfpoint{307.007965pt}{208.117905pt}}
\pgflineto{\pgfpoint{298.102722pt}{208.117905pt}}
\pgfusepath{stroke}
\pgfpathmoveto{\pgfpoint{307.007965pt}{214.294739pt}}
\pgflineto{\pgfpoint{298.107239pt}{214.294739pt}}
\pgfusepath{stroke}
\pgfpathmoveto{\pgfpoint{307.007965pt}{220.471588pt}}
\pgflineto{\pgfpoint{298.107239pt}{220.471588pt}}
\pgfusepath{stroke}
\pgfpathmoveto{\pgfpoint{307.007965pt}{226.648422pt}}
\pgflineto{\pgfpoint{298.107269pt}{226.648422pt}}
\pgfusepath{stroke}
\pgfpathmoveto{\pgfpoint{298.111786pt}{232.825272pt}}
\pgflineto{\pgfpoint{306.994415pt}{232.825272pt}}
\pgfusepath{stroke}
\pgfpathmoveto{\pgfpoint{298.111816pt}{239.002106pt}}
\pgflineto{\pgfpoint{306.989899pt}{239.002106pt}}
\pgfusepath{stroke}
\pgfpathmoveto{\pgfpoint{298.116333pt}{245.178955pt}}
\pgflineto{\pgfpoint{306.989899pt}{245.178955pt}}
\pgfusepath{stroke}
\pgfpathmoveto{\pgfpoint{298.116364pt}{251.355804pt}}
\pgflineto{\pgfpoint{306.985382pt}{251.355804pt}}
\pgfusepath{stroke}
\pgfpathmoveto{\pgfpoint{298.120850pt}{257.532623pt}}
\pgflineto{\pgfpoint{306.985352pt}{257.532623pt}}
\pgfusepath{stroke}
\pgfpathmoveto{\pgfpoint{298.120880pt}{263.709473pt}}
\pgflineto{\pgfpoint{306.980835pt}{263.709473pt}}
\pgfusepath{stroke}
\pgfpathmoveto{\pgfpoint{298.120880pt}{269.886322pt}}
\pgflineto{\pgfpoint{306.980835pt}{269.886322pt}}
\pgfusepath{stroke}
\pgfpathmoveto{\pgfpoint{298.125397pt}{276.063141pt}}
\pgflineto{\pgfpoint{306.976288pt}{276.063141pt}}
\pgfusepath{stroke}
\pgfpathmoveto{\pgfpoint{298.125397pt}{282.239990pt}}
\pgflineto{\pgfpoint{306.971802pt}{282.239990pt}}
\pgfusepath{stroke}
\pgfpathmoveto{\pgfpoint{298.129944pt}{288.416840pt}}
\pgflineto{\pgfpoint{306.971771pt}{288.416840pt}}
\pgfusepath{stroke}
\pgfpathmoveto{\pgfpoint{298.129944pt}{294.593689pt}}
\pgflineto{\pgfpoint{306.967285pt}{294.593689pt}}
\pgfusepath{stroke}
\pgfpathmoveto{\pgfpoint{298.130005pt}{300.770538pt}}
\pgflineto{\pgfpoint{306.967255pt}{300.770538pt}}
\pgfusepath{stroke}
\pgfpathmoveto{\pgfpoint{307.007965pt}{152.526306pt}}
\pgflineto{\pgfpoint{307.007965pt}{146.349472pt}}
\pgfusepath{stroke}
\pgfpathmoveto{\pgfpoint{307.007965pt}{146.349472pt}}
\pgflineto{\pgfpoint{307.007965pt}{140.172638pt}}
\pgfusepath{stroke}
\pgfpathmoveto{\pgfpoint{307.007965pt}{158.703156pt}}
\pgflineto{\pgfpoint{307.007965pt}{152.526306pt}}
\pgfusepath{stroke}
\pgfpathmoveto{\pgfpoint{307.007965pt}{164.880005pt}}
\pgflineto{\pgfpoint{307.007965pt}{158.703156pt}}
\pgfusepath{stroke}
\pgfpathmoveto{\pgfpoint{307.007965pt}{171.056854pt}}
\pgflineto{\pgfpoint{307.007965pt}{164.880005pt}}
\pgfusepath{stroke}
\pgfpathmoveto{\pgfpoint{307.007965pt}{177.233673pt}}
\pgflineto{\pgfpoint{307.007965pt}{171.056854pt}}
\pgfusepath{stroke}
\pgfpathmoveto{\pgfpoint{307.007965pt}{146.349472pt}}
\pgflineto{\pgfpoint{307.026184pt}{146.349472pt}}
\pgfusepath{stroke}
\pgfpathmoveto{\pgfpoint{307.007965pt}{152.526306pt}}
\pgflineto{\pgfpoint{307.026184pt}{152.526306pt}}
\pgfusepath{stroke}
\pgfpathmoveto{\pgfpoint{307.007965pt}{158.703156pt}}
\pgflineto{\pgfpoint{307.026245pt}{158.703156pt}}
\pgfusepath{stroke}
\pgfpathmoveto{\pgfpoint{307.007965pt}{164.880005pt}}
\pgflineto{\pgfpoint{307.026245pt}{164.880005pt}}
\pgfusepath{stroke}
\pgfpathmoveto{\pgfpoint{307.007965pt}{171.056854pt}}
\pgflineto{\pgfpoint{307.026306pt}{171.056854pt}}
\pgfusepath{stroke}
\pgfpathmoveto{\pgfpoint{307.007965pt}{201.941055pt}}
\pgflineto{\pgfpoint{307.007965pt}{195.764206pt}}
\pgfusepath{stroke}
\pgfpathmoveto{\pgfpoint{307.007965pt}{183.410522pt}}
\pgflineto{\pgfpoint{307.007965pt}{177.233673pt}}
\pgfusepath{stroke}
\pgfpathmoveto{\pgfpoint{307.007965pt}{189.587372pt}}
\pgflineto{\pgfpoint{307.007965pt}{183.410522pt}}
\pgfusepath{stroke}
\pgfpathmoveto{\pgfpoint{307.007965pt}{195.764206pt}}
\pgflineto{\pgfpoint{307.007965pt}{189.587372pt}}
\pgfusepath{stroke}
\pgfpathmoveto{\pgfpoint{307.007965pt}{208.117905pt}}
\pgflineto{\pgfpoint{307.007965pt}{201.941055pt}}
\pgfusepath{stroke}
\pgfpathmoveto{\pgfpoint{307.007965pt}{214.294739pt}}
\pgflineto{\pgfpoint{307.007965pt}{208.117905pt}}
\pgfusepath{stroke}
\pgfpathmoveto{\pgfpoint{307.007965pt}{220.471588pt}}
\pgflineto{\pgfpoint{307.007965pt}{214.294739pt}}
\pgfusepath{stroke}
\pgfpathmoveto{\pgfpoint{307.007965pt}{226.648422pt}}
\pgflineto{\pgfpoint{307.007965pt}{220.471588pt}}
\pgfusepath{stroke}
\pgfpathmoveto{\pgfpoint{307.007965pt}{177.233673pt}}
\pgflineto{\pgfpoint{307.026276pt}{177.233673pt}}
\pgfusepath{stroke}
\pgfpathmoveto{\pgfpoint{307.007965pt}{245.178955pt}}
\pgflineto{\pgfpoint{306.989899pt}{245.178955pt}}
\pgfusepath{stroke}
\pgfpathmoveto{\pgfpoint{307.007965pt}{251.355804pt}}
\pgflineto{\pgfpoint{306.985382pt}{251.355804pt}}
\pgfusepath{stroke}
\pgfpathmoveto{\pgfpoint{307.007965pt}{257.532623pt}}
\pgflineto{\pgfpoint{306.985352pt}{257.532623pt}}
\pgfusepath{stroke}
\pgfpathmoveto{\pgfpoint{307.007965pt}{263.709473pt}}
\pgflineto{\pgfpoint{306.980835pt}{263.709473pt}}
\pgfusepath{stroke}
\pgfpathmoveto{\pgfpoint{307.007965pt}{269.886322pt}}
\pgflineto{\pgfpoint{306.980835pt}{269.886322pt}}
\pgfusepath{stroke}
\pgfpathmoveto{\pgfpoint{307.007965pt}{276.063141pt}}
\pgflineto{\pgfpoint{306.976288pt}{276.063141pt}}
\pgfusepath{stroke}
\pgfpathmoveto{\pgfpoint{307.007965pt}{282.239990pt}}
\pgflineto{\pgfpoint{306.971802pt}{282.239990pt}}
\pgfusepath{stroke}
\pgfpathmoveto{\pgfpoint{307.007965pt}{288.416840pt}}
\pgflineto{\pgfpoint{306.971771pt}{288.416840pt}}
\pgfusepath{stroke}
\pgfpathmoveto{\pgfpoint{307.007965pt}{294.593689pt}}
\pgflineto{\pgfpoint{306.967285pt}{294.593689pt}}
\pgfusepath{stroke}
\pgfpathmoveto{\pgfpoint{307.007965pt}{300.770538pt}}
\pgflineto{\pgfpoint{306.967255pt}{300.770538pt}}
\pgfusepath{stroke}
\pgfpathmoveto{\pgfpoint{307.007965pt}{300.770538pt}}
\pgflineto{\pgfpoint{307.007965pt}{294.593689pt}}
\pgfusepath{stroke}
\pgfpathmoveto{\pgfpoint{307.007965pt}{263.709473pt}}
\pgflineto{\pgfpoint{307.007965pt}{257.532623pt}}
\pgfusepath{stroke}
\pgfpathmoveto{\pgfpoint{307.007965pt}{232.825272pt}}
\pgflineto{\pgfpoint{307.007965pt}{226.648422pt}}
\pgfusepath{stroke}
\pgfpathmoveto{\pgfpoint{307.007965pt}{239.002106pt}}
\pgflineto{\pgfpoint{307.007965pt}{232.825272pt}}
\pgfusepath{stroke}
\pgfpathmoveto{\pgfpoint{307.007965pt}{245.178955pt}}
\pgflineto{\pgfpoint{307.007965pt}{239.002106pt}}
\pgfusepath{stroke}
\pgfpathmoveto{\pgfpoint{307.007965pt}{251.355804pt}}
\pgflineto{\pgfpoint{307.007965pt}{245.178955pt}}
\pgfusepath{stroke}
\pgfpathmoveto{\pgfpoint{307.007965pt}{257.532623pt}}
\pgflineto{\pgfpoint{307.007965pt}{251.355804pt}}
\pgfusepath{stroke}
\pgfpathmoveto{\pgfpoint{307.007965pt}{269.886322pt}}
\pgflineto{\pgfpoint{307.007965pt}{263.709473pt}}
\pgfusepath{stroke}
\pgfpathmoveto{\pgfpoint{307.007965pt}{276.063141pt}}
\pgflineto{\pgfpoint{307.007965pt}{269.886322pt}}
\pgfusepath{stroke}
\pgfpathmoveto{\pgfpoint{307.007965pt}{282.239990pt}}
\pgflineto{\pgfpoint{307.007965pt}{276.063141pt}}
\pgfusepath{stroke}
\pgfpathmoveto{\pgfpoint{307.007965pt}{288.416840pt}}
\pgflineto{\pgfpoint{307.007965pt}{282.239990pt}}
\pgfusepath{stroke}
\pgfpathmoveto{\pgfpoint{307.007965pt}{294.593689pt}}
\pgflineto{\pgfpoint{307.007965pt}{288.416840pt}}
\pgfusepath{stroke}
\pgfpathmoveto{\pgfpoint{307.007965pt}{232.825272pt}}
\pgflineto{\pgfpoint{306.994415pt}{232.825272pt}}
\pgfusepath{stroke}
\pgfpathmoveto{\pgfpoint{307.007965pt}{239.002106pt}}
\pgflineto{\pgfpoint{306.989899pt}{239.002106pt}}
\pgfusepath{stroke}
\pgfpathmoveto{\pgfpoint{307.007965pt}{183.410522pt}}
\pgflineto{\pgfpoint{307.030823pt}{183.410522pt}}
\pgfusepath{stroke}
\pgfpathmoveto{\pgfpoint{307.007965pt}{189.587372pt}}
\pgflineto{\pgfpoint{307.030853pt}{189.587372pt}}
\pgfusepath{stroke}
\pgfpathmoveto{\pgfpoint{307.007965pt}{195.764206pt}}
\pgflineto{\pgfpoint{307.030853pt}{195.764206pt}}
\pgfusepath{stroke}
\pgfpathmoveto{\pgfpoint{307.007965pt}{201.941055pt}}
\pgflineto{\pgfpoint{307.035400pt}{201.941055pt}}
\pgfusepath{stroke}
\pgfpathmoveto{\pgfpoint{307.007965pt}{208.117905pt}}
\pgflineto{\pgfpoint{307.035400pt}{208.117905pt}}
\pgfusepath{stroke}
\pgfpathmoveto{\pgfpoint{307.007965pt}{214.294739pt}}
\pgflineto{\pgfpoint{307.039948pt}{214.294739pt}}
\pgfusepath{stroke}
\pgfpathmoveto{\pgfpoint{307.007965pt}{220.471588pt}}
\pgflineto{\pgfpoint{307.039978pt}{220.471588pt}}
\pgfusepath{stroke}
\pgfpathmoveto{\pgfpoint{307.007965pt}{226.648422pt}}
\pgflineto{\pgfpoint{307.044495pt}{226.648422pt}}
\pgfusepath{stroke}
\pgfpathmoveto{\pgfpoint{307.007965pt}{232.825272pt}}
\pgflineto{\pgfpoint{307.044495pt}{232.825272pt}}
\pgfusepath{stroke}
\pgfpathmoveto{\pgfpoint{307.007965pt}{239.002106pt}}
\pgflineto{\pgfpoint{307.044556pt}{239.002106pt}}
\pgfusepath{stroke}
\pgfpathmoveto{\pgfpoint{307.007965pt}{245.178955pt}}
\pgflineto{\pgfpoint{307.049042pt}{245.178955pt}}
\pgfusepath{stroke}
\pgfpathmoveto{\pgfpoint{307.007965pt}{251.355804pt}}
\pgflineto{\pgfpoint{307.049103pt}{251.355804pt}}
\pgfusepath{stroke}
\pgfpathmoveto{\pgfpoint{307.007965pt}{257.532623pt}}
\pgflineto{\pgfpoint{307.053589pt}{257.532623pt}}
\pgfusepath{stroke}
\pgfpathmoveto{\pgfpoint{307.007965pt}{263.709473pt}}
\pgflineto{\pgfpoint{307.053619pt}{263.709473pt}}
\pgfusepath{stroke}
\pgfpathmoveto{\pgfpoint{307.007965pt}{269.886322pt}}
\pgflineto{\pgfpoint{307.058167pt}{269.886322pt}}
\pgfusepath{stroke}
\pgfpathmoveto{\pgfpoint{307.007965pt}{276.063141pt}}
\pgflineto{\pgfpoint{307.058136pt}{276.063141pt}}
\pgfusepath{stroke}
\pgfpathmoveto{\pgfpoint{307.007965pt}{282.239990pt}}
\pgflineto{\pgfpoint{307.058197pt}{282.239990pt}}
\pgfusepath{stroke}
\pgfpathmoveto{\pgfpoint{307.007965pt}{288.416840pt}}
\pgflineto{\pgfpoint{307.062744pt}{288.416840pt}}
\pgfusepath{stroke}
\pgfpathmoveto{\pgfpoint{307.007965pt}{294.593689pt}}
\pgflineto{\pgfpoint{307.062744pt}{294.593689pt}}
\pgfusepath{stroke}
\pgfpathmoveto{\pgfpoint{307.007965pt}{96.934731pt}}
\pgflineto{\pgfpoint{307.007965pt}{90.757896pt}}
\pgfusepath{stroke}
\pgfpathmoveto{\pgfpoint{307.007965pt}{66.050522pt}}
\pgflineto{\pgfpoint{307.007965pt}{59.873672pt}}
\pgfusepath{stroke}
\pgfpathmoveto{\pgfpoint{307.007965pt}{72.227356pt}}
\pgflineto{\pgfpoint{307.007965pt}{66.050522pt}}
\pgfusepath{stroke}
\pgfpathmoveto{\pgfpoint{307.007965pt}{78.404205pt}}
\pgflineto{\pgfpoint{307.007965pt}{72.227356pt}}
\pgfusepath{stroke}
\pgfpathmoveto{\pgfpoint{307.007965pt}{84.581039pt}}
\pgflineto{\pgfpoint{307.007965pt}{78.404205pt}}
\pgfusepath{stroke}
\pgfpathmoveto{\pgfpoint{307.007965pt}{90.757896pt}}
\pgflineto{\pgfpoint{307.007965pt}{84.581039pt}}
\pgfusepath{stroke}
\pgfpathmoveto{\pgfpoint{307.007965pt}{103.111580pt}}
\pgflineto{\pgfpoint{307.007965pt}{96.934731pt}}
\pgfusepath{stroke}
\pgfpathmoveto{\pgfpoint{307.007965pt}{109.288422pt}}
\pgflineto{\pgfpoint{307.007965pt}{103.111580pt}}
\pgfusepath{stroke}
\pgfpathmoveto{\pgfpoint{307.007965pt}{115.465263pt}}
\pgflineto{\pgfpoint{307.007965pt}{109.288422pt}}
\pgfusepath{stroke}
\pgfpathmoveto{\pgfpoint{307.007965pt}{121.642097pt}}
\pgflineto{\pgfpoint{307.007965pt}{115.465263pt}}
\pgfusepath{stroke}
\pgfpathmoveto{\pgfpoint{307.007965pt}{127.818947pt}}
\pgflineto{\pgfpoint{307.007965pt}{121.642097pt}}
\pgfusepath{stroke}
\pgfpathmoveto{\pgfpoint{307.007965pt}{133.995789pt}}
\pgflineto{\pgfpoint{307.007965pt}{127.818947pt}}
\pgfusepath{stroke}
\pgfpathmoveto{\pgfpoint{307.007965pt}{140.172638pt}}
\pgflineto{\pgfpoint{307.007965pt}{133.995789pt}}
\pgfusepath{stroke}
\pgfpathmoveto{\pgfpoint{307.007965pt}{47.519989pt}}
\pgflineto{\pgfpoint{306.989777pt}{47.519989pt}}
\pgfusepath{stroke}
\pgfpathmoveto{\pgfpoint{307.007965pt}{53.696838pt}}
\pgflineto{\pgfpoint{306.989807pt}{53.696838pt}}
\pgfusepath{stroke}
\pgfpathmoveto{\pgfpoint{307.007965pt}{59.873672pt}}
\pgflineto{\pgfpoint{307.007965pt}{53.696838pt}}
\pgfusepath{stroke}
\pgfpathmoveto{\pgfpoint{307.007965pt}{53.696838pt}}
\pgflineto{\pgfpoint{307.007965pt}{47.519989pt}}
\pgfusepath{stroke}
\pgfpathmoveto{\pgfpoint{315.935974pt}{47.519989pt}}
\pgflineto{\pgfpoint{307.007965pt}{47.519989pt}}
\pgfusepath{stroke}
\pgfpathmoveto{\pgfpoint{315.935974pt}{53.696838pt}}
\pgflineto{\pgfpoint{307.007965pt}{53.696838pt}}
\pgfusepath{stroke}
\pgfpathmoveto{\pgfpoint{315.935974pt}{59.873672pt}}
\pgflineto{\pgfpoint{307.007965pt}{59.873672pt}}
\pgfusepath{stroke}
\pgfpathmoveto{\pgfpoint{315.935974pt}{66.050522pt}}
\pgflineto{\pgfpoint{307.007965pt}{66.050522pt}}
\pgfusepath{stroke}
\pgfpathmoveto{\pgfpoint{315.935974pt}{72.227356pt}}
\pgflineto{\pgfpoint{307.007965pt}{72.227356pt}}
\pgfusepath{stroke}
\pgfpathmoveto{\pgfpoint{315.935974pt}{78.404205pt}}
\pgflineto{\pgfpoint{307.007965pt}{78.404205pt}}
\pgfusepath{stroke}
\pgfpathmoveto{\pgfpoint{315.935974pt}{84.581039pt}}
\pgflineto{\pgfpoint{307.007965pt}{84.581039pt}}
\pgfusepath{stroke}
\pgfpathmoveto{\pgfpoint{315.935974pt}{90.757896pt}}
\pgflineto{\pgfpoint{307.007965pt}{90.757896pt}}
\pgfusepath{stroke}
\pgfpathmoveto{\pgfpoint{315.935974pt}{96.934731pt}}
\pgflineto{\pgfpoint{307.007965pt}{96.934731pt}}
\pgfusepath{stroke}
\pgfpathmoveto{\pgfpoint{315.935974pt}{103.111580pt}}
\pgflineto{\pgfpoint{307.007965pt}{103.111580pt}}
\pgfusepath{stroke}
\pgfpathmoveto{\pgfpoint{315.935974pt}{109.288422pt}}
\pgflineto{\pgfpoint{307.007965pt}{109.288422pt}}
\pgfusepath{stroke}
\pgfpathmoveto{\pgfpoint{315.935974pt}{115.465263pt}}
\pgflineto{\pgfpoint{307.007965pt}{115.465263pt}}
\pgfusepath{stroke}
\pgfpathmoveto{\pgfpoint{315.935974pt}{121.642097pt}}
\pgflineto{\pgfpoint{307.007965pt}{121.642097pt}}
\pgfusepath{stroke}
\pgfpathmoveto{\pgfpoint{315.935974pt}{127.818947pt}}
\pgflineto{\pgfpoint{307.007965pt}{127.818947pt}}
\pgfusepath{stroke}
\pgfpathmoveto{\pgfpoint{315.935974pt}{133.995789pt}}
\pgflineto{\pgfpoint{307.007965pt}{133.995789pt}}
\pgfusepath{stroke}
\pgfpathmoveto{\pgfpoint{315.935974pt}{140.172638pt}}
\pgflineto{\pgfpoint{307.007965pt}{140.172638pt}}
\pgfusepath{stroke}
\pgfpathmoveto{\pgfpoint{315.935974pt}{146.349472pt}}
\pgflineto{\pgfpoint{307.026184pt}{146.349472pt}}
\pgfusepath{stroke}
\pgfpathmoveto{\pgfpoint{315.935974pt}{152.526306pt}}
\pgflineto{\pgfpoint{307.026184pt}{152.526306pt}}
\pgfusepath{stroke}
\pgfpathmoveto{\pgfpoint{315.935974pt}{158.703156pt}}
\pgflineto{\pgfpoint{307.026245pt}{158.703156pt}}
\pgfusepath{stroke}
\pgfpathmoveto{\pgfpoint{315.935974pt}{164.880005pt}}
\pgflineto{\pgfpoint{307.026245pt}{164.880005pt}}
\pgfusepath{stroke}
\pgfpathmoveto{\pgfpoint{315.935974pt}{171.056854pt}}
\pgflineto{\pgfpoint{307.026306pt}{171.056854pt}}
\pgfusepath{stroke}
\pgfpathmoveto{\pgfpoint{315.935974pt}{177.233673pt}}
\pgflineto{\pgfpoint{307.026276pt}{177.233673pt}}
\pgfusepath{stroke}
\pgfpathmoveto{\pgfpoint{315.935974pt}{183.410522pt}}
\pgflineto{\pgfpoint{307.030823pt}{183.410522pt}}
\pgfusepath{stroke}
\pgfpathmoveto{\pgfpoint{315.935974pt}{189.587372pt}}
\pgflineto{\pgfpoint{307.030853pt}{189.587372pt}}
\pgfusepath{stroke}
\pgfpathmoveto{\pgfpoint{315.935974pt}{195.764206pt}}
\pgflineto{\pgfpoint{307.030853pt}{195.764206pt}}
\pgfusepath{stroke}
\pgfpathmoveto{\pgfpoint{315.935974pt}{201.941055pt}}
\pgflineto{\pgfpoint{307.035400pt}{201.941055pt}}
\pgfusepath{stroke}
\pgfpathmoveto{\pgfpoint{307.035400pt}{208.117905pt}}
\pgflineto{\pgfpoint{315.922363pt}{208.117905pt}}
\pgfusepath{stroke}
\pgfpathmoveto{\pgfpoint{307.039948pt}{214.294739pt}}
\pgflineto{\pgfpoint{315.922363pt}{214.294739pt}}
\pgfusepath{stroke}
\pgfpathmoveto{\pgfpoint{307.039978pt}{220.471588pt}}
\pgflineto{\pgfpoint{315.917816pt}{220.471588pt}}
\pgfusepath{stroke}
\pgfpathmoveto{\pgfpoint{307.044495pt}{226.648422pt}}
\pgflineto{\pgfpoint{315.917847pt}{226.648422pt}}
\pgfusepath{stroke}
\pgfpathmoveto{\pgfpoint{307.044495pt}{232.825272pt}}
\pgflineto{\pgfpoint{315.913269pt}{232.825272pt}}
\pgfusepath{stroke}
\pgfpathmoveto{\pgfpoint{307.044556pt}{239.002106pt}}
\pgflineto{\pgfpoint{315.913269pt}{239.002106pt}}
\pgfusepath{stroke}
\pgfpathmoveto{\pgfpoint{307.049042pt}{245.178955pt}}
\pgflineto{\pgfpoint{315.908752pt}{245.178955pt}}
\pgfusepath{stroke}
\pgfpathmoveto{\pgfpoint{307.049103pt}{251.355804pt}}
\pgflineto{\pgfpoint{315.904175pt}{251.355804pt}}
\pgfusepath{stroke}
\pgfpathmoveto{\pgfpoint{307.053589pt}{257.532623pt}}
\pgflineto{\pgfpoint{315.904175pt}{257.532623pt}}
\pgfusepath{stroke}
\pgfpathmoveto{\pgfpoint{307.053619pt}{263.709473pt}}
\pgflineto{\pgfpoint{315.899628pt}{263.709473pt}}
\pgfusepath{stroke}
\pgfpathmoveto{\pgfpoint{307.058167pt}{269.886322pt}}
\pgflineto{\pgfpoint{315.899658pt}{269.886322pt}}
\pgfusepath{stroke}
\pgfpathmoveto{\pgfpoint{307.058136pt}{276.063141pt}}
\pgflineto{\pgfpoint{315.895050pt}{276.063141pt}}
\pgfusepath{stroke}
\pgfpathmoveto{\pgfpoint{307.058197pt}{282.239990pt}}
\pgflineto{\pgfpoint{315.890533pt}{282.239990pt}}
\pgfusepath{stroke}
\pgfpathmoveto{\pgfpoint{307.062744pt}{288.416840pt}}
\pgflineto{\pgfpoint{315.890533pt}{288.416840pt}}
\pgfusepath{stroke}
\pgfpathmoveto{\pgfpoint{307.062744pt}{294.593689pt}}
\pgflineto{\pgfpoint{315.886017pt}{294.593689pt}}
\pgfusepath{stroke}
\pgfpathmoveto{\pgfpoint{315.935974pt}{140.172638pt}}
\pgflineto{\pgfpoint{315.935974pt}{133.995789pt}}
\pgfusepath{stroke}
\pgfpathmoveto{\pgfpoint{315.935974pt}{133.995789pt}}
\pgflineto{\pgfpoint{315.935974pt}{127.818947pt}}
\pgfusepath{stroke}
\pgfpathmoveto{\pgfpoint{315.935974pt}{146.349472pt}}
\pgflineto{\pgfpoint{315.935974pt}{140.172638pt}}
\pgfusepath{stroke}
\pgfpathmoveto{\pgfpoint{315.935974pt}{152.526306pt}}
\pgflineto{\pgfpoint{315.935974pt}{146.349472pt}}
\pgfusepath{stroke}
\pgfpathmoveto{\pgfpoint{315.935974pt}{158.703156pt}}
\pgflineto{\pgfpoint{315.935974pt}{152.526306pt}}
\pgfusepath{stroke}
\pgfpathmoveto{\pgfpoint{315.935974pt}{164.880005pt}}
\pgflineto{\pgfpoint{315.935974pt}{158.703156pt}}
\pgfusepath{stroke}
\pgfpathmoveto{\pgfpoint{315.935974pt}{171.056854pt}}
\pgflineto{\pgfpoint{315.935974pt}{164.880005pt}}
\pgfusepath{stroke}
\pgfpathmoveto{\pgfpoint{315.935974pt}{164.880005pt}}
\pgflineto{\pgfpoint{315.949554pt}{164.880005pt}}
\pgfusepath{stroke}
\pgfpathmoveto{\pgfpoint{315.935974pt}{133.995789pt}}
\pgflineto{\pgfpoint{315.954010pt}{133.995789pt}}
\pgfusepath{stroke}
\pgfpathmoveto{\pgfpoint{315.935974pt}{140.172638pt}}
\pgflineto{\pgfpoint{315.954041pt}{140.172638pt}}
\pgfusepath{stroke}
\pgfpathmoveto{\pgfpoint{315.935974pt}{146.349472pt}}
\pgflineto{\pgfpoint{315.954071pt}{146.349472pt}}
\pgfusepath{stroke}
\pgfpathmoveto{\pgfpoint{315.935974pt}{152.526306pt}}
\pgflineto{\pgfpoint{315.954041pt}{152.526306pt}}
\pgfusepath{stroke}
\pgfpathmoveto{\pgfpoint{315.935974pt}{158.703156pt}}
\pgflineto{\pgfpoint{315.963043pt}{158.703156pt}}
\pgfusepath{stroke}
\pgfpathmoveto{\pgfpoint{315.935974pt}{183.410522pt}}
\pgflineto{\pgfpoint{315.935974pt}{177.233673pt}}
\pgfusepath{stroke}
\pgfpathmoveto{\pgfpoint{315.935974pt}{177.233673pt}}
\pgflineto{\pgfpoint{315.935974pt}{171.056854pt}}
\pgfusepath{stroke}
\pgfpathmoveto{\pgfpoint{315.935974pt}{189.587372pt}}
\pgflineto{\pgfpoint{315.935974pt}{183.410522pt}}
\pgfusepath{stroke}
\pgfpathmoveto{\pgfpoint{315.935974pt}{195.764206pt}}
\pgflineto{\pgfpoint{315.935974pt}{189.587372pt}}
\pgfusepath{stroke}
\pgfpathmoveto{\pgfpoint{315.935974pt}{201.941055pt}}
\pgflineto{\pgfpoint{315.935974pt}{195.764206pt}}
\pgfusepath{stroke}
\pgfpathmoveto{\pgfpoint{315.935974pt}{232.825272pt}}
\pgflineto{\pgfpoint{315.913269pt}{232.825272pt}}
\pgfusepath{stroke}
\pgfpathmoveto{\pgfpoint{315.935974pt}{239.002106pt}}
\pgflineto{\pgfpoint{315.913269pt}{239.002106pt}}
\pgfusepath{stroke}
\pgfpathmoveto{\pgfpoint{315.935974pt}{245.178955pt}}
\pgflineto{\pgfpoint{315.908752pt}{245.178955pt}}
\pgfusepath{stroke}
\pgfpathmoveto{\pgfpoint{315.935974pt}{251.355804pt}}
\pgflineto{\pgfpoint{315.904175pt}{251.355804pt}}
\pgfusepath{stroke}
\pgfpathmoveto{\pgfpoint{315.935974pt}{257.532623pt}}
\pgflineto{\pgfpoint{315.904175pt}{257.532623pt}}
\pgfusepath{stroke}
\pgfpathmoveto{\pgfpoint{315.935974pt}{263.709473pt}}
\pgflineto{\pgfpoint{315.899628pt}{263.709473pt}}
\pgfusepath{stroke}
\pgfpathmoveto{\pgfpoint{315.935974pt}{269.886322pt}}
\pgflineto{\pgfpoint{315.899658pt}{269.886322pt}}
\pgfusepath{stroke}
\pgfpathmoveto{\pgfpoint{315.935974pt}{276.063141pt}}
\pgflineto{\pgfpoint{315.895050pt}{276.063141pt}}
\pgfusepath{stroke}
\pgfpathmoveto{\pgfpoint{315.935974pt}{282.239990pt}}
\pgflineto{\pgfpoint{315.890533pt}{282.239990pt}}
\pgfusepath{stroke}
\pgfpathmoveto{\pgfpoint{315.935974pt}{288.416840pt}}
\pgflineto{\pgfpoint{315.890533pt}{288.416840pt}}
\pgfusepath{stroke}
\pgfpathmoveto{\pgfpoint{315.935974pt}{294.593689pt}}
\pgflineto{\pgfpoint{315.886017pt}{294.593689pt}}
\pgfusepath{stroke}
\pgfpathmoveto{\pgfpoint{315.935974pt}{263.709473pt}}
\pgflineto{\pgfpoint{315.935974pt}{257.532623pt}}
\pgfusepath{stroke}
\pgfpathmoveto{\pgfpoint{315.935974pt}{232.825272pt}}
\pgflineto{\pgfpoint{315.935974pt}{226.648422pt}}
\pgfusepath{stroke}
\pgfpathmoveto{\pgfpoint{315.935974pt}{239.002106pt}}
\pgflineto{\pgfpoint{315.935974pt}{232.825272pt}}
\pgfusepath{stroke}
\pgfpathmoveto{\pgfpoint{315.935974pt}{245.178955pt}}
\pgflineto{\pgfpoint{315.935974pt}{239.002106pt}}
\pgfusepath{stroke}
\pgfpathmoveto{\pgfpoint{315.935974pt}{251.355804pt}}
\pgflineto{\pgfpoint{315.935974pt}{245.178955pt}}
\pgfusepath{stroke}
\pgfpathmoveto{\pgfpoint{315.935974pt}{257.532623pt}}
\pgflineto{\pgfpoint{315.935974pt}{251.355804pt}}
\pgfusepath{stroke}
\pgfpathmoveto{\pgfpoint{315.935974pt}{269.886322pt}}
\pgflineto{\pgfpoint{315.935974pt}{263.709473pt}}
\pgfusepath{stroke}
\pgfpathmoveto{\pgfpoint{315.935974pt}{276.063141pt}}
\pgflineto{\pgfpoint{315.935974pt}{269.886322pt}}
\pgfusepath{stroke}
\pgfpathmoveto{\pgfpoint{315.935974pt}{282.239990pt}}
\pgflineto{\pgfpoint{315.935974pt}{276.063141pt}}
\pgfusepath{stroke}
\pgfpathmoveto{\pgfpoint{315.935974pt}{288.416840pt}}
\pgflineto{\pgfpoint{315.935974pt}{282.239990pt}}
\pgfusepath{stroke}
\pgfpathmoveto{\pgfpoint{315.935974pt}{294.593689pt}}
\pgflineto{\pgfpoint{315.935974pt}{288.416840pt}}
\pgfusepath{stroke}
\pgfpathmoveto{\pgfpoint{315.935974pt}{226.648422pt}}
\pgflineto{\pgfpoint{315.917847pt}{226.648422pt}}
\pgfusepath{stroke}
\pgfpathmoveto{\pgfpoint{315.963074pt}{164.880005pt}}
\pgflineto{\pgfpoint{315.949554pt}{164.880005pt}}
\pgfusepath{stroke}
\pgfpathmoveto{\pgfpoint{315.935974pt}{171.056854pt}}
\pgflineto{\pgfpoint{315.958557pt}{171.056854pt}}
\pgfusepath{stroke}
\pgfpathmoveto{\pgfpoint{315.935974pt}{177.233673pt}}
\pgflineto{\pgfpoint{315.958557pt}{177.233673pt}}
\pgfusepath{stroke}
\pgfpathmoveto{\pgfpoint{315.935974pt}{183.410522pt}}
\pgflineto{\pgfpoint{315.963074pt}{183.410522pt}}
\pgfusepath{stroke}
\pgfpathmoveto{\pgfpoint{315.935974pt}{208.117905pt}}
\pgflineto{\pgfpoint{315.922363pt}{208.117905pt}}
\pgfusepath{stroke}
\pgfpathmoveto{\pgfpoint{315.935974pt}{214.294739pt}}
\pgflineto{\pgfpoint{315.922363pt}{214.294739pt}}
\pgfusepath{stroke}
\pgfpathmoveto{\pgfpoint{315.935974pt}{220.471588pt}}
\pgflineto{\pgfpoint{315.917816pt}{220.471588pt}}
\pgfusepath{stroke}
\pgfpathmoveto{\pgfpoint{315.935974pt}{214.294739pt}}
\pgflineto{\pgfpoint{315.935974pt}{208.117905pt}}
\pgfusepath{stroke}
\pgfpathmoveto{\pgfpoint{315.935974pt}{208.117905pt}}
\pgflineto{\pgfpoint{315.935974pt}{201.941055pt}}
\pgfusepath{stroke}
\pgfpathmoveto{\pgfpoint{315.935974pt}{220.471588pt}}
\pgflineto{\pgfpoint{315.935974pt}{214.294739pt}}
\pgfusepath{stroke}
\pgfpathmoveto{\pgfpoint{315.935974pt}{226.648422pt}}
\pgflineto{\pgfpoint{315.935974pt}{220.471588pt}}
\pgfusepath{stroke}
\pgfpathmoveto{\pgfpoint{315.935974pt}{189.587372pt}}
\pgflineto{\pgfpoint{315.963074pt}{189.587372pt}}
\pgfusepath{stroke}
\pgfpathmoveto{\pgfpoint{315.935974pt}{195.764206pt}}
\pgflineto{\pgfpoint{315.967590pt}{195.764206pt}}
\pgfusepath{stroke}
\pgfpathmoveto{\pgfpoint{315.935974pt}{201.941055pt}}
\pgflineto{\pgfpoint{315.967590pt}{201.941055pt}}
\pgfusepath{stroke}
\pgfpathmoveto{\pgfpoint{315.935974pt}{208.117905pt}}
\pgflineto{\pgfpoint{315.967590pt}{208.117905pt}}
\pgfusepath{stroke}
\pgfpathmoveto{\pgfpoint{315.935974pt}{214.294739pt}}
\pgflineto{\pgfpoint{315.972107pt}{214.294739pt}}
\pgfusepath{stroke}
\pgfpathmoveto{\pgfpoint{315.935974pt}{220.471588pt}}
\pgflineto{\pgfpoint{315.972107pt}{220.471588pt}}
\pgfusepath{stroke}
\pgfpathmoveto{\pgfpoint{315.935974pt}{226.648422pt}}
\pgflineto{\pgfpoint{315.976624pt}{226.648422pt}}
\pgfusepath{stroke}
\pgfpathmoveto{\pgfpoint{315.935974pt}{232.825272pt}}
\pgflineto{\pgfpoint{315.976624pt}{232.825272pt}}
\pgfusepath{stroke}
\pgfpathmoveto{\pgfpoint{315.935974pt}{239.002106pt}}
\pgflineto{\pgfpoint{315.981110pt}{239.002106pt}}
\pgfusepath{stroke}
\pgfpathmoveto{\pgfpoint{315.935974pt}{245.178955pt}}
\pgflineto{\pgfpoint{315.981140pt}{245.178955pt}}
\pgfusepath{stroke}
\pgfpathmoveto{\pgfpoint{315.935974pt}{251.355804pt}}
\pgflineto{\pgfpoint{315.981140pt}{251.355804pt}}
\pgfusepath{stroke}
\pgfpathmoveto{\pgfpoint{315.935974pt}{257.532623pt}}
\pgflineto{\pgfpoint{315.985657pt}{257.532623pt}}
\pgfusepath{stroke}
\pgfpathmoveto{\pgfpoint{315.935974pt}{263.709473pt}}
\pgflineto{\pgfpoint{315.985626pt}{263.709473pt}}
\pgfusepath{stroke}
\pgfpathmoveto{\pgfpoint{315.935974pt}{269.886322pt}}
\pgflineto{\pgfpoint{315.990173pt}{269.886322pt}}
\pgfusepath{stroke}
\pgfpathmoveto{\pgfpoint{315.935974pt}{276.063141pt}}
\pgflineto{\pgfpoint{315.990173pt}{276.063141pt}}
\pgfusepath{stroke}
\pgfpathmoveto{\pgfpoint{315.935974pt}{282.239990pt}}
\pgflineto{\pgfpoint{315.990173pt}{282.239990pt}}
\pgfusepath{stroke}
\pgfpathmoveto{\pgfpoint{315.935974pt}{288.416840pt}}
\pgflineto{\pgfpoint{315.994720pt}{288.416840pt}}
\pgfusepath{stroke}
\pgfpathmoveto{\pgfpoint{315.935974pt}{78.404205pt}}
\pgflineto{\pgfpoint{315.935974pt}{72.227356pt}}
\pgfusepath{stroke}
\pgfpathmoveto{\pgfpoint{315.935974pt}{53.696838pt}}
\pgflineto{\pgfpoint{315.935974pt}{47.519989pt}}
\pgfusepath{stroke}
\pgfpathmoveto{\pgfpoint{315.935974pt}{59.873672pt}}
\pgflineto{\pgfpoint{315.935974pt}{53.696838pt}}
\pgfusepath{stroke}
\pgfpathmoveto{\pgfpoint{315.935974pt}{66.050522pt}}
\pgflineto{\pgfpoint{315.935974pt}{59.873672pt}}
\pgfusepath{stroke}
\pgfpathmoveto{\pgfpoint{315.935974pt}{72.227356pt}}
\pgflineto{\pgfpoint{315.935974pt}{66.050522pt}}
\pgfusepath{stroke}
\pgfpathmoveto{\pgfpoint{315.935974pt}{84.581039pt}}
\pgflineto{\pgfpoint{315.935974pt}{78.404205pt}}
\pgfusepath{stroke}
\pgfpathmoveto{\pgfpoint{315.935974pt}{90.757896pt}}
\pgflineto{\pgfpoint{315.935974pt}{84.581039pt}}
\pgfusepath{stroke}
\pgfpathmoveto{\pgfpoint{315.935974pt}{96.934731pt}}
\pgflineto{\pgfpoint{315.935974pt}{90.757896pt}}
\pgfusepath{stroke}
\pgfpathmoveto{\pgfpoint{315.935974pt}{103.111580pt}}
\pgflineto{\pgfpoint{315.935974pt}{96.934731pt}}
\pgfusepath{stroke}
\pgfpathmoveto{\pgfpoint{315.935974pt}{109.288422pt}}
\pgflineto{\pgfpoint{315.935974pt}{103.111580pt}}
\pgfusepath{stroke}
\pgfpathmoveto{\pgfpoint{315.935974pt}{115.465263pt}}
\pgflineto{\pgfpoint{315.935974pt}{109.288422pt}}
\pgfusepath{stroke}
\pgfpathmoveto{\pgfpoint{315.935974pt}{121.642097pt}}
\pgflineto{\pgfpoint{315.935974pt}{115.465263pt}}
\pgfusepath{stroke}
\pgfpathmoveto{\pgfpoint{315.935974pt}{127.818947pt}}
\pgflineto{\pgfpoint{315.935974pt}{121.642097pt}}
\pgfusepath{stroke}
\pgfpathmoveto{\pgfpoint{315.935974pt}{47.519989pt}}
\pgflineto{\pgfpoint{324.845978pt}{47.519989pt}}
\pgfusepath{stroke}
\pgfpathmoveto{\pgfpoint{324.863983pt}{53.696838pt}}
\pgflineto{\pgfpoint{315.935974pt}{53.696838pt}}
\pgfusepath{stroke}
\pgfpathmoveto{\pgfpoint{324.863983pt}{59.873672pt}}
\pgflineto{\pgfpoint{315.935974pt}{59.873672pt}}
\pgfusepath{stroke}
\pgfpathmoveto{\pgfpoint{324.863983pt}{66.050522pt}}
\pgflineto{\pgfpoint{315.935974pt}{66.050522pt}}
\pgfusepath{stroke}
\pgfpathmoveto{\pgfpoint{324.863983pt}{72.227356pt}}
\pgflineto{\pgfpoint{315.935974pt}{72.227356pt}}
\pgfusepath{stroke}
\pgfpathmoveto{\pgfpoint{324.863983pt}{78.404205pt}}
\pgflineto{\pgfpoint{315.935974pt}{78.404205pt}}
\pgfusepath{stroke}
\pgfpathmoveto{\pgfpoint{324.863983pt}{84.581039pt}}
\pgflineto{\pgfpoint{315.935974pt}{84.581039pt}}
\pgfusepath{stroke}
\pgfpathmoveto{\pgfpoint{324.863983pt}{90.757896pt}}
\pgflineto{\pgfpoint{315.935974pt}{90.757896pt}}
\pgfusepath{stroke}
\pgfpathmoveto{\pgfpoint{324.863983pt}{96.934731pt}}
\pgflineto{\pgfpoint{315.935974pt}{96.934731pt}}
\pgfusepath{stroke}
\pgfpathmoveto{\pgfpoint{324.863983pt}{103.111580pt}}
\pgflineto{\pgfpoint{315.935974pt}{103.111580pt}}
\pgfusepath{stroke}
\pgfpathmoveto{\pgfpoint{324.863983pt}{109.288422pt}}
\pgflineto{\pgfpoint{315.935974pt}{109.288422pt}}
\pgfusepath{stroke}
\pgfpathmoveto{\pgfpoint{324.863983pt}{115.465263pt}}
\pgflineto{\pgfpoint{315.935974pt}{115.465263pt}}
\pgfusepath{stroke}
\pgfpathmoveto{\pgfpoint{324.863983pt}{121.642097pt}}
\pgflineto{\pgfpoint{315.935974pt}{121.642097pt}}
\pgfusepath{stroke}
\pgfpathmoveto{\pgfpoint{324.863983pt}{127.818947pt}}
\pgflineto{\pgfpoint{315.935974pt}{127.818947pt}}
\pgfusepath{stroke}
\pgfpathmoveto{\pgfpoint{324.863983pt}{133.995789pt}}
\pgflineto{\pgfpoint{315.954010pt}{133.995789pt}}
\pgfusepath{stroke}
\pgfpathmoveto{\pgfpoint{324.863983pt}{140.172638pt}}
\pgflineto{\pgfpoint{315.954041pt}{140.172638pt}}
\pgfusepath{stroke}
\pgfpathmoveto{\pgfpoint{324.863983pt}{146.349472pt}}
\pgflineto{\pgfpoint{315.954071pt}{146.349472pt}}
\pgfusepath{stroke}
\pgfpathmoveto{\pgfpoint{324.863983pt}{152.526306pt}}
\pgflineto{\pgfpoint{315.954041pt}{152.526306pt}}
\pgfusepath{stroke}
\pgfpathmoveto{\pgfpoint{324.863983pt}{158.703156pt}}
\pgflineto{\pgfpoint{315.963043pt}{158.703156pt}}
\pgfusepath{stroke}
\pgfpathmoveto{\pgfpoint{324.863983pt}{164.880005pt}}
\pgflineto{\pgfpoint{315.963074pt}{164.880005pt}}
\pgfusepath{stroke}
\pgfpathmoveto{\pgfpoint{324.863983pt}{171.056854pt}}
\pgflineto{\pgfpoint{315.958557pt}{171.056854pt}}
\pgfusepath{stroke}
\pgfpathmoveto{\pgfpoint{324.863983pt}{177.233673pt}}
\pgflineto{\pgfpoint{315.958557pt}{177.233673pt}}
\pgfusepath{stroke}
\pgfpathmoveto{\pgfpoint{324.863983pt}{183.410522pt}}
\pgflineto{\pgfpoint{315.963074pt}{183.410522pt}}
\pgfusepath{stroke}
\pgfpathmoveto{\pgfpoint{324.863983pt}{189.587372pt}}
\pgflineto{\pgfpoint{315.963074pt}{189.587372pt}}
\pgfusepath{stroke}
\pgfpathmoveto{\pgfpoint{324.863983pt}{195.764206pt}}
\pgflineto{\pgfpoint{315.967590pt}{195.764206pt}}
\pgfusepath{stroke}
\pgfpathmoveto{\pgfpoint{324.863983pt}{201.941055pt}}
\pgflineto{\pgfpoint{315.967590pt}{201.941055pt}}
\pgfusepath{stroke}
\pgfpathmoveto{\pgfpoint{324.863983pt}{208.117905pt}}
\pgflineto{\pgfpoint{315.967590pt}{208.117905pt}}
\pgfusepath{stroke}
\pgfpathmoveto{\pgfpoint{324.863983pt}{214.294739pt}}
\pgflineto{\pgfpoint{315.972107pt}{214.294739pt}}
\pgfusepath{stroke}
\pgfpathmoveto{\pgfpoint{315.972107pt}{220.471588pt}}
\pgflineto{\pgfpoint{324.850525pt}{220.471588pt}}
\pgfusepath{stroke}
\pgfpathmoveto{\pgfpoint{315.976624pt}{226.648422pt}}
\pgflineto{\pgfpoint{324.850525pt}{226.648422pt}}
\pgfusepath{stroke}
\pgfpathmoveto{\pgfpoint{315.976624pt}{232.825272pt}}
\pgflineto{\pgfpoint{324.846039pt}{232.825272pt}}
\pgfusepath{stroke}
\pgfpathmoveto{\pgfpoint{315.981110pt}{239.002106pt}}
\pgflineto{\pgfpoint{324.846008pt}{239.002106pt}}
\pgfusepath{stroke}
\pgfpathmoveto{\pgfpoint{315.981140pt}{245.178955pt}}
\pgflineto{\pgfpoint{324.841522pt}{245.178955pt}}
\pgfusepath{stroke}
\pgfpathmoveto{\pgfpoint{315.981140pt}{251.355804pt}}
\pgflineto{\pgfpoint{324.837036pt}{251.355804pt}}
\pgfusepath{stroke}
\pgfpathmoveto{\pgfpoint{315.985657pt}{257.532623pt}}
\pgflineto{\pgfpoint{324.837036pt}{257.532623pt}}
\pgfusepath{stroke}
\pgfpathmoveto{\pgfpoint{315.985626pt}{263.709473pt}}
\pgflineto{\pgfpoint{324.832520pt}{263.709473pt}}
\pgfusepath{stroke}
\pgfpathmoveto{\pgfpoint{315.990173pt}{269.886322pt}}
\pgflineto{\pgfpoint{324.832550pt}{269.886322pt}}
\pgfusepath{stroke}
\pgfpathmoveto{\pgfpoint{315.990173pt}{276.063141pt}}
\pgflineto{\pgfpoint{324.828003pt}{276.063141pt}}
\pgfusepath{stroke}
\pgfpathmoveto{\pgfpoint{315.990173pt}{282.239990pt}}
\pgflineto{\pgfpoint{324.828003pt}{282.239990pt}}
\pgfusepath{stroke}
\pgfpathmoveto{\pgfpoint{315.994720pt}{288.416840pt}}
\pgflineto{\pgfpoint{324.823547pt}{288.416840pt}}
\pgfusepath{stroke}
\pgfpathmoveto{\pgfpoint{324.863983pt}{152.526306pt}}
\pgflineto{\pgfpoint{324.863983pt}{146.349472pt}}
\pgfusepath{stroke}
\pgfpathmoveto{\pgfpoint{324.863983pt}{146.349472pt}}
\pgflineto{\pgfpoint{324.863983pt}{140.172638pt}}
\pgfusepath{stroke}
\pgfpathmoveto{\pgfpoint{324.863983pt}{158.703156pt}}
\pgflineto{\pgfpoint{324.863983pt}{152.526306pt}}
\pgfusepath{stroke}
\pgfpathmoveto{\pgfpoint{324.863983pt}{164.880005pt}}
\pgflineto{\pgfpoint{324.863983pt}{158.703156pt}}
\pgfusepath{stroke}
\pgfpathmoveto{\pgfpoint{324.863983pt}{171.056854pt}}
\pgflineto{\pgfpoint{324.863983pt}{164.880005pt}}
\pgfusepath{stroke}
\pgfpathmoveto{\pgfpoint{324.863983pt}{146.349472pt}}
\pgflineto{\pgfpoint{324.881989pt}{146.349472pt}}
\pgfusepath{stroke}
\pgfpathmoveto{\pgfpoint{324.863983pt}{152.526306pt}}
\pgflineto{\pgfpoint{324.881958pt}{152.526306pt}}
\pgfusepath{stroke}
\pgfpathmoveto{\pgfpoint{324.863983pt}{158.703156pt}}
\pgflineto{\pgfpoint{324.881927pt}{158.703156pt}}
\pgfusepath{stroke}
\pgfpathmoveto{\pgfpoint{324.863983pt}{164.880005pt}}
\pgflineto{\pgfpoint{324.881989pt}{164.880005pt}}
\pgfusepath{stroke}
\pgfpathmoveto{\pgfpoint{324.863983pt}{195.764206pt}}
\pgflineto{\pgfpoint{324.863983pt}{189.587372pt}}
\pgfusepath{stroke}
\pgfpathmoveto{\pgfpoint{324.863983pt}{177.233673pt}}
\pgflineto{\pgfpoint{324.863983pt}{171.056854pt}}
\pgfusepath{stroke}
\pgfpathmoveto{\pgfpoint{324.863983pt}{183.410522pt}}
\pgflineto{\pgfpoint{324.863983pt}{177.233673pt}}
\pgfusepath{stroke}
\pgfpathmoveto{\pgfpoint{324.863983pt}{189.587372pt}}
\pgflineto{\pgfpoint{324.863983pt}{183.410522pt}}
\pgfusepath{stroke}
\pgfpathmoveto{\pgfpoint{324.863983pt}{201.941055pt}}
\pgflineto{\pgfpoint{324.863983pt}{195.764206pt}}
\pgfusepath{stroke}
\pgfpathmoveto{\pgfpoint{324.863983pt}{208.117905pt}}
\pgflineto{\pgfpoint{324.863983pt}{201.941055pt}}
\pgfusepath{stroke}
\pgfpathmoveto{\pgfpoint{324.863983pt}{214.294739pt}}
\pgflineto{\pgfpoint{324.863983pt}{208.117905pt}}
\pgfusepath{stroke}
\pgfpathmoveto{\pgfpoint{324.863983pt}{171.056854pt}}
\pgflineto{\pgfpoint{324.881989pt}{171.056854pt}}
\pgfusepath{stroke}
\pgfpathmoveto{\pgfpoint{324.863983pt}{239.002106pt}}
\pgflineto{\pgfpoint{324.846008pt}{239.002106pt}}
\pgfusepath{stroke}
\pgfpathmoveto{\pgfpoint{324.863983pt}{245.178955pt}}
\pgflineto{\pgfpoint{324.841522pt}{245.178955pt}}
\pgfusepath{stroke}
\pgfpathmoveto{\pgfpoint{324.863983pt}{251.355804pt}}
\pgflineto{\pgfpoint{324.837036pt}{251.355804pt}}
\pgfusepath{stroke}
\pgfpathmoveto{\pgfpoint{324.863983pt}{257.532623pt}}
\pgflineto{\pgfpoint{324.837036pt}{257.532623pt}}
\pgfusepath{stroke}
\pgfpathmoveto{\pgfpoint{324.863983pt}{263.709473pt}}
\pgflineto{\pgfpoint{324.832520pt}{263.709473pt}}
\pgfusepath{stroke}
\pgfpathmoveto{\pgfpoint{324.863983pt}{269.886322pt}}
\pgflineto{\pgfpoint{324.832550pt}{269.886322pt}}
\pgfusepath{stroke}
\pgfpathmoveto{\pgfpoint{324.863983pt}{276.063141pt}}
\pgflineto{\pgfpoint{324.828003pt}{276.063141pt}}
\pgfusepath{stroke}
\pgfpathmoveto{\pgfpoint{324.863983pt}{282.239990pt}}
\pgflineto{\pgfpoint{324.828003pt}{282.239990pt}}
\pgfusepath{stroke}
\pgfpathmoveto{\pgfpoint{324.863983pt}{288.416840pt}}
\pgflineto{\pgfpoint{324.823547pt}{288.416840pt}}
\pgfusepath{stroke}
\pgfpathmoveto{\pgfpoint{324.863983pt}{263.709473pt}}
\pgflineto{\pgfpoint{324.863983pt}{257.532623pt}}
\pgfusepath{stroke}
\pgfpathmoveto{\pgfpoint{324.863983pt}{232.825272pt}}
\pgflineto{\pgfpoint{324.863983pt}{226.648422pt}}
\pgfusepath{stroke}
\pgfpathmoveto{\pgfpoint{324.863983pt}{239.002106pt}}
\pgflineto{\pgfpoint{324.863983pt}{232.825272pt}}
\pgfusepath{stroke}
\pgfpathmoveto{\pgfpoint{324.863983pt}{245.178955pt}}
\pgflineto{\pgfpoint{324.863983pt}{239.002106pt}}
\pgfusepath{stroke}
\pgfpathmoveto{\pgfpoint{324.863983pt}{251.355804pt}}
\pgflineto{\pgfpoint{324.863983pt}{245.178955pt}}
\pgfusepath{stroke}
\pgfpathmoveto{\pgfpoint{324.863983pt}{257.532623pt}}
\pgflineto{\pgfpoint{324.863983pt}{251.355804pt}}
\pgfusepath{stroke}
\pgfpathmoveto{\pgfpoint{324.863983pt}{269.886322pt}}
\pgflineto{\pgfpoint{324.863983pt}{263.709473pt}}
\pgfusepath{stroke}
\pgfpathmoveto{\pgfpoint{324.863983pt}{276.063141pt}}
\pgflineto{\pgfpoint{324.863983pt}{269.886322pt}}
\pgfusepath{stroke}
\pgfpathmoveto{\pgfpoint{324.863983pt}{282.239990pt}}
\pgflineto{\pgfpoint{324.863983pt}{276.063141pt}}
\pgfusepath{stroke}
\pgfpathmoveto{\pgfpoint{324.863983pt}{288.416840pt}}
\pgflineto{\pgfpoint{324.863983pt}{282.239990pt}}
\pgfusepath{stroke}
\pgfpathmoveto{\pgfpoint{324.863983pt}{226.648422pt}}
\pgflineto{\pgfpoint{324.850525pt}{226.648422pt}}
\pgfusepath{stroke}
\pgfpathmoveto{\pgfpoint{324.863983pt}{232.825272pt}}
\pgflineto{\pgfpoint{324.846039pt}{232.825272pt}}
\pgfusepath{stroke}
\pgfpathmoveto{\pgfpoint{324.863983pt}{177.233673pt}}
\pgflineto{\pgfpoint{324.881958pt}{177.233673pt}}
\pgfusepath{stroke}
\pgfpathmoveto{\pgfpoint{324.863983pt}{183.410522pt}}
\pgflineto{\pgfpoint{324.886444pt}{183.410522pt}}
\pgfusepath{stroke}
\pgfpathmoveto{\pgfpoint{324.863983pt}{189.587372pt}}
\pgflineto{\pgfpoint{324.886475pt}{189.587372pt}}
\pgfusepath{stroke}
\pgfpathmoveto{\pgfpoint{324.863983pt}{195.764206pt}}
\pgflineto{\pgfpoint{324.890961pt}{195.764206pt}}
\pgfusepath{stroke}
\pgfpathmoveto{\pgfpoint{324.863983pt}{220.471588pt}}
\pgflineto{\pgfpoint{324.850525pt}{220.471588pt}}
\pgfusepath{stroke}
\pgfpathmoveto{\pgfpoint{324.863983pt}{226.648422pt}}
\pgflineto{\pgfpoint{324.863983pt}{220.471588pt}}
\pgfusepath{stroke}
\pgfpathmoveto{\pgfpoint{324.863983pt}{220.471588pt}}
\pgflineto{\pgfpoint{324.863983pt}{214.294739pt}}
\pgfusepath{stroke}
\pgfpathmoveto{\pgfpoint{324.863983pt}{201.941055pt}}
\pgflineto{\pgfpoint{324.890961pt}{201.941055pt}}
\pgfusepath{stroke}
\pgfpathmoveto{\pgfpoint{324.863983pt}{208.117905pt}}
\pgflineto{\pgfpoint{324.895447pt}{208.117905pt}}
\pgfusepath{stroke}
\pgfpathmoveto{\pgfpoint{324.863983pt}{214.294739pt}}
\pgflineto{\pgfpoint{324.895447pt}{214.294739pt}}
\pgfusepath{stroke}
\pgfpathmoveto{\pgfpoint{324.863983pt}{220.471588pt}}
\pgflineto{\pgfpoint{324.895447pt}{220.471588pt}}
\pgfusepath{stroke}
\pgfpathmoveto{\pgfpoint{324.863983pt}{226.648422pt}}
\pgflineto{\pgfpoint{324.899933pt}{226.648422pt}}
\pgfusepath{stroke}
\pgfpathmoveto{\pgfpoint{324.863983pt}{232.825272pt}}
\pgflineto{\pgfpoint{324.899963pt}{232.825272pt}}
\pgfusepath{stroke}
\pgfpathmoveto{\pgfpoint{324.863983pt}{239.002106pt}}
\pgflineto{\pgfpoint{324.904480pt}{239.002106pt}}
\pgfusepath{stroke}
\pgfpathmoveto{\pgfpoint{324.863983pt}{245.178955pt}}
\pgflineto{\pgfpoint{324.904449pt}{245.178955pt}}
\pgfusepath{stroke}
\pgfpathmoveto{\pgfpoint{324.863983pt}{251.355804pt}}
\pgflineto{\pgfpoint{324.904480pt}{251.355804pt}}
\pgfusepath{stroke}
\pgfpathmoveto{\pgfpoint{324.863983pt}{257.532623pt}}
\pgflineto{\pgfpoint{324.908966pt}{257.532623pt}}
\pgfusepath{stroke}
\pgfpathmoveto{\pgfpoint{324.863983pt}{263.709473pt}}
\pgflineto{\pgfpoint{324.908936pt}{263.709473pt}}
\pgfusepath{stroke}
\pgfpathmoveto{\pgfpoint{324.863983pt}{269.886322pt}}
\pgflineto{\pgfpoint{324.913452pt}{269.886322pt}}
\pgfusepath{stroke}
\pgfpathmoveto{\pgfpoint{324.863983pt}{276.063141pt}}
\pgflineto{\pgfpoint{324.913391pt}{276.063141pt}}
\pgfusepath{stroke}
\pgfpathmoveto{\pgfpoint{324.863983pt}{282.239990pt}}
\pgflineto{\pgfpoint{324.917938pt}{282.239990pt}}
\pgfusepath{stroke}
\pgfpathmoveto{\pgfpoint{324.863983pt}{90.757896pt}}
\pgflineto{\pgfpoint{324.863983pt}{84.581039pt}}
\pgfusepath{stroke}
\pgfpathmoveto{\pgfpoint{324.863983pt}{59.873672pt}}
\pgflineto{\pgfpoint{324.863983pt}{53.696838pt}}
\pgfusepath{stroke}
\pgfpathmoveto{\pgfpoint{324.863983pt}{66.050522pt}}
\pgflineto{\pgfpoint{324.863983pt}{59.873672pt}}
\pgfusepath{stroke}
\pgfpathmoveto{\pgfpoint{324.863983pt}{72.227356pt}}
\pgflineto{\pgfpoint{324.863983pt}{66.050522pt}}
\pgfusepath{stroke}
\pgfpathmoveto{\pgfpoint{324.863983pt}{78.404205pt}}
\pgflineto{\pgfpoint{324.863983pt}{72.227356pt}}
\pgfusepath{stroke}
\pgfpathmoveto{\pgfpoint{324.863983pt}{84.581039pt}}
\pgflineto{\pgfpoint{324.863983pt}{78.404205pt}}
\pgfusepath{stroke}
\pgfpathmoveto{\pgfpoint{324.863983pt}{96.934731pt}}
\pgflineto{\pgfpoint{324.863983pt}{90.757896pt}}
\pgfusepath{stroke}
\pgfpathmoveto{\pgfpoint{324.863983pt}{103.111580pt}}
\pgflineto{\pgfpoint{324.863983pt}{96.934731pt}}
\pgfusepath{stroke}
\pgfpathmoveto{\pgfpoint{324.863983pt}{109.288422pt}}
\pgflineto{\pgfpoint{324.863983pt}{103.111580pt}}
\pgfusepath{stroke}
\pgfpathmoveto{\pgfpoint{324.863983pt}{115.465263pt}}
\pgflineto{\pgfpoint{324.863983pt}{109.288422pt}}
\pgfusepath{stroke}
\pgfpathmoveto{\pgfpoint{324.863983pt}{121.642097pt}}
\pgflineto{\pgfpoint{324.863983pt}{115.465263pt}}
\pgfusepath{stroke}
\pgfpathmoveto{\pgfpoint{324.863983pt}{127.818947pt}}
\pgflineto{\pgfpoint{324.863983pt}{121.642097pt}}
\pgfusepath{stroke}
\pgfpathmoveto{\pgfpoint{324.863983pt}{133.995789pt}}
\pgflineto{\pgfpoint{324.863983pt}{127.818947pt}}
\pgfusepath{stroke}
\pgfpathmoveto{\pgfpoint{324.863983pt}{140.172638pt}}
\pgflineto{\pgfpoint{324.863983pt}{133.995789pt}}
\pgfusepath{stroke}
\pgfpathmoveto{\pgfpoint{324.863983pt}{47.519989pt}}
\pgflineto{\pgfpoint{324.845978pt}{47.519989pt}}
\pgfusepath{stroke}
\pgfpathmoveto{\pgfpoint{324.863983pt}{53.696838pt}}
\pgflineto{\pgfpoint{324.863983pt}{47.519989pt}}
\pgfusepath{stroke}
\pgfpathmoveto{\pgfpoint{324.863983pt}{47.519989pt}}
\pgflineto{\pgfpoint{333.774109pt}{47.519989pt}}
\pgfusepath{stroke}
\pgfpathmoveto{\pgfpoint{324.863983pt}{53.696838pt}}
\pgflineto{\pgfpoint{333.774109pt}{53.696838pt}}
\pgfusepath{stroke}
\pgfpathmoveto{\pgfpoint{324.863983pt}{59.873672pt}}
\pgflineto{\pgfpoint{333.774109pt}{59.873672pt}}
\pgfusepath{stroke}
\pgfpathmoveto{\pgfpoint{333.791992pt}{66.050522pt}}
\pgflineto{\pgfpoint{324.863983pt}{66.050522pt}}
\pgfusepath{stroke}
\pgfpathmoveto{\pgfpoint{333.791992pt}{72.227356pt}}
\pgflineto{\pgfpoint{324.863983pt}{72.227356pt}}
\pgfusepath{stroke}
\pgfpathmoveto{\pgfpoint{333.791992pt}{78.404205pt}}
\pgflineto{\pgfpoint{324.863983pt}{78.404205pt}}
\pgfusepath{stroke}
\pgfpathmoveto{\pgfpoint{333.791992pt}{84.581039pt}}
\pgflineto{\pgfpoint{324.863983pt}{84.581039pt}}
\pgfusepath{stroke}
\pgfpathmoveto{\pgfpoint{333.791992pt}{90.757896pt}}
\pgflineto{\pgfpoint{324.863983pt}{90.757896pt}}
\pgfusepath{stroke}
\pgfpathmoveto{\pgfpoint{333.791992pt}{96.934731pt}}
\pgflineto{\pgfpoint{324.863983pt}{96.934731pt}}
\pgfusepath{stroke}
\pgfpathmoveto{\pgfpoint{333.791992pt}{103.111580pt}}
\pgflineto{\pgfpoint{324.863983pt}{103.111580pt}}
\pgfusepath{stroke}
\pgfpathmoveto{\pgfpoint{333.791992pt}{109.288422pt}}
\pgflineto{\pgfpoint{324.863983pt}{109.288422pt}}
\pgfusepath{stroke}
\pgfpathmoveto{\pgfpoint{333.791992pt}{115.465263pt}}
\pgflineto{\pgfpoint{324.863983pt}{115.465263pt}}
\pgfusepath{stroke}
\pgfpathmoveto{\pgfpoint{333.791992pt}{121.642097pt}}
\pgflineto{\pgfpoint{324.863983pt}{121.642097pt}}
\pgfusepath{stroke}
\pgfpathmoveto{\pgfpoint{333.791992pt}{127.818947pt}}
\pgflineto{\pgfpoint{324.863983pt}{127.818947pt}}
\pgfusepath{stroke}
\pgfpathmoveto{\pgfpoint{333.791992pt}{133.995789pt}}
\pgflineto{\pgfpoint{324.863983pt}{133.995789pt}}
\pgfusepath{stroke}
\pgfpathmoveto{\pgfpoint{333.791992pt}{140.172638pt}}
\pgflineto{\pgfpoint{324.863983pt}{140.172638pt}}
\pgfusepath{stroke}
\pgfpathmoveto{\pgfpoint{333.791992pt}{146.349472pt}}
\pgflineto{\pgfpoint{324.881989pt}{146.349472pt}}
\pgfusepath{stroke}
\pgfpathmoveto{\pgfpoint{333.791992pt}{152.526306pt}}
\pgflineto{\pgfpoint{324.881958pt}{152.526306pt}}
\pgfusepath{stroke}
\pgfpathmoveto{\pgfpoint{333.791992pt}{158.703156pt}}
\pgflineto{\pgfpoint{324.881927pt}{158.703156pt}}
\pgfusepath{stroke}
\pgfpathmoveto{\pgfpoint{333.791992pt}{164.880005pt}}
\pgflineto{\pgfpoint{324.881989pt}{164.880005pt}}
\pgfusepath{stroke}
\pgfpathmoveto{\pgfpoint{333.791992pt}{171.056854pt}}
\pgflineto{\pgfpoint{324.881989pt}{171.056854pt}}
\pgfusepath{stroke}
\pgfpathmoveto{\pgfpoint{333.791992pt}{177.233673pt}}
\pgflineto{\pgfpoint{324.881958pt}{177.233673pt}}
\pgfusepath{stroke}
\pgfpathmoveto{\pgfpoint{333.791992pt}{183.410522pt}}
\pgflineto{\pgfpoint{324.886444pt}{183.410522pt}}
\pgfusepath{stroke}
\pgfpathmoveto{\pgfpoint{333.791992pt}{189.587372pt}}
\pgflineto{\pgfpoint{324.886475pt}{189.587372pt}}
\pgfusepath{stroke}
\pgfpathmoveto{\pgfpoint{333.791992pt}{195.764206pt}}
\pgflineto{\pgfpoint{324.890961pt}{195.764206pt}}
\pgfusepath{stroke}
\pgfpathmoveto{\pgfpoint{333.791992pt}{201.941055pt}}
\pgflineto{\pgfpoint{324.890961pt}{201.941055pt}}
\pgfusepath{stroke}
\pgfpathmoveto{\pgfpoint{333.791992pt}{208.117905pt}}
\pgflineto{\pgfpoint{324.895447pt}{208.117905pt}}
\pgfusepath{stroke}
\pgfpathmoveto{\pgfpoint{333.791992pt}{214.294739pt}}
\pgflineto{\pgfpoint{324.895447pt}{214.294739pt}}
\pgfusepath{stroke}
\pgfpathmoveto{\pgfpoint{333.791992pt}{220.471588pt}}
\pgflineto{\pgfpoint{324.895447pt}{220.471588pt}}
\pgfusepath{stroke}
\pgfpathmoveto{\pgfpoint{333.791992pt}{226.648422pt}}
\pgflineto{\pgfpoint{324.899933pt}{226.648422pt}}
\pgfusepath{stroke}
\pgfpathmoveto{\pgfpoint{324.899963pt}{232.825272pt}}
\pgflineto{\pgfpoint{333.778412pt}{232.825272pt}}
\pgfusepath{stroke}
\pgfpathmoveto{\pgfpoint{324.904480pt}{239.002106pt}}
\pgflineto{\pgfpoint{333.778381pt}{239.002106pt}}
\pgfusepath{stroke}
\pgfpathmoveto{\pgfpoint{324.904449pt}{245.178955pt}}
\pgflineto{\pgfpoint{333.773865pt}{245.178955pt}}
\pgfusepath{stroke}
\pgfpathmoveto{\pgfpoint{324.904480pt}{251.355804pt}}
\pgflineto{\pgfpoint{333.773865pt}{251.355804pt}}
\pgfusepath{stroke}
\pgfpathmoveto{\pgfpoint{324.908966pt}{257.532623pt}}
\pgflineto{\pgfpoint{333.769318pt}{257.532623pt}}
\pgfusepath{stroke}
\pgfpathmoveto{\pgfpoint{324.908936pt}{263.709473pt}}
\pgflineto{\pgfpoint{333.764801pt}{263.709473pt}}
\pgfusepath{stroke}
\pgfpathmoveto{\pgfpoint{324.913452pt}{269.886322pt}}
\pgflineto{\pgfpoint{333.764801pt}{269.886322pt}}
\pgfusepath{stroke}
\pgfpathmoveto{\pgfpoint{324.913391pt}{276.063141pt}}
\pgflineto{\pgfpoint{333.760254pt}{276.063141pt}}
\pgfusepath{stroke}
\pgfpathmoveto{\pgfpoint{324.917938pt}{282.239990pt}}
\pgflineto{\pgfpoint{333.760254pt}{282.239990pt}}
\pgfusepath{stroke}
\pgfpathmoveto{\pgfpoint{333.791992pt}{171.056854pt}}
\pgflineto{\pgfpoint{333.791992pt}{164.880005pt}}
\pgfusepath{stroke}
\pgfpathmoveto{\pgfpoint{333.791992pt}{164.880005pt}}
\pgflineto{\pgfpoint{333.791992pt}{158.703156pt}}
\pgfusepath{stroke}
\pgfpathmoveto{\pgfpoint{333.791992pt}{177.233673pt}}
\pgflineto{\pgfpoint{333.791992pt}{171.056854pt}}
\pgfusepath{stroke}
\pgfpathmoveto{\pgfpoint{333.791992pt}{183.410522pt}}
\pgflineto{\pgfpoint{333.791992pt}{177.233673pt}}
\pgfusepath{stroke}
\pgfpathmoveto{\pgfpoint{333.791992pt}{164.880005pt}}
\pgflineto{\pgfpoint{333.805328pt}{164.880005pt}}
\pgfusepath{stroke}
\pgfpathmoveto{\pgfpoint{333.791992pt}{171.056854pt}}
\pgflineto{\pgfpoint{333.805298pt}{171.056854pt}}
\pgfusepath{stroke}
\pgfpathmoveto{\pgfpoint{333.791992pt}{177.233673pt}}
\pgflineto{\pgfpoint{333.809784pt}{177.233673pt}}
\pgfusepath{stroke}
\pgfpathmoveto{\pgfpoint{333.791992pt}{208.117905pt}}
\pgflineto{\pgfpoint{333.791992pt}{201.941055pt}}
\pgfusepath{stroke}
\pgfpathmoveto{\pgfpoint{333.791992pt}{189.587372pt}}
\pgflineto{\pgfpoint{333.791992pt}{183.410522pt}}
\pgfusepath{stroke}
\pgfpathmoveto{\pgfpoint{333.791992pt}{195.764206pt}}
\pgflineto{\pgfpoint{333.791992pt}{189.587372pt}}
\pgfusepath{stroke}
\pgfpathmoveto{\pgfpoint{333.791992pt}{201.941055pt}}
\pgflineto{\pgfpoint{333.791992pt}{195.764206pt}}
\pgfusepath{stroke}
\pgfpathmoveto{\pgfpoint{333.791992pt}{214.294739pt}}
\pgflineto{\pgfpoint{333.791992pt}{208.117905pt}}
\pgfusepath{stroke}
\pgfpathmoveto{\pgfpoint{333.791992pt}{220.471588pt}}
\pgflineto{\pgfpoint{333.791992pt}{214.294739pt}}
\pgfusepath{stroke}
\pgfpathmoveto{\pgfpoint{333.791992pt}{226.648422pt}}
\pgflineto{\pgfpoint{333.791992pt}{220.471588pt}}
\pgfusepath{stroke}
\pgfpathmoveto{\pgfpoint{333.791992pt}{183.410522pt}}
\pgflineto{\pgfpoint{333.809753pt}{183.410522pt}}
\pgfusepath{stroke}
\pgfpathmoveto{\pgfpoint{333.791992pt}{239.002106pt}}
\pgflineto{\pgfpoint{333.778381pt}{239.002106pt}}
\pgfusepath{stroke}
\pgfpathmoveto{\pgfpoint{333.791992pt}{245.178955pt}}
\pgflineto{\pgfpoint{333.773865pt}{245.178955pt}}
\pgfusepath{stroke}
\pgfpathmoveto{\pgfpoint{333.791992pt}{251.355804pt}}
\pgflineto{\pgfpoint{333.773865pt}{251.355804pt}}
\pgfusepath{stroke}
\pgfpathmoveto{\pgfpoint{333.791992pt}{257.532623pt}}
\pgflineto{\pgfpoint{333.769318pt}{257.532623pt}}
\pgfusepath{stroke}
\pgfpathmoveto{\pgfpoint{333.791992pt}{263.709473pt}}
\pgflineto{\pgfpoint{333.764801pt}{263.709473pt}}
\pgfusepath{stroke}
\pgfpathmoveto{\pgfpoint{333.791992pt}{269.886322pt}}
\pgflineto{\pgfpoint{333.764801pt}{269.886322pt}}
\pgfusepath{stroke}
\pgfpathmoveto{\pgfpoint{333.791992pt}{276.063141pt}}
\pgflineto{\pgfpoint{333.760254pt}{276.063141pt}}
\pgfusepath{stroke}
\pgfpathmoveto{\pgfpoint{333.791992pt}{282.239990pt}}
\pgflineto{\pgfpoint{333.760254pt}{282.239990pt}}
\pgfusepath{stroke}
\pgfpathmoveto{\pgfpoint{333.791992pt}{282.239990pt}}
\pgflineto{\pgfpoint{333.791992pt}{276.063141pt}}
\pgfusepath{stroke}
\pgfpathmoveto{\pgfpoint{333.791992pt}{245.178955pt}}
\pgflineto{\pgfpoint{333.791992pt}{239.002106pt}}
\pgfusepath{stroke}
\pgfpathmoveto{\pgfpoint{333.791992pt}{232.825272pt}}
\pgflineto{\pgfpoint{333.791992pt}{226.648422pt}}
\pgfusepath{stroke}
\pgfpathmoveto{\pgfpoint{333.791992pt}{239.002106pt}}
\pgflineto{\pgfpoint{333.791992pt}{232.825272pt}}
\pgfusepath{stroke}
\pgfpathmoveto{\pgfpoint{333.791992pt}{251.355804pt}}
\pgflineto{\pgfpoint{333.791992pt}{245.178955pt}}
\pgfusepath{stroke}
\pgfpathmoveto{\pgfpoint{333.791992pt}{257.532623pt}}
\pgflineto{\pgfpoint{333.791992pt}{251.355804pt}}
\pgfusepath{stroke}
\pgfpathmoveto{\pgfpoint{333.791992pt}{263.709473pt}}
\pgflineto{\pgfpoint{333.791992pt}{257.532623pt}}
\pgfusepath{stroke}
\pgfpathmoveto{\pgfpoint{333.791992pt}{269.886322pt}}
\pgflineto{\pgfpoint{333.791992pt}{263.709473pt}}
\pgfusepath{stroke}
\pgfpathmoveto{\pgfpoint{333.791992pt}{276.063141pt}}
\pgflineto{\pgfpoint{333.791992pt}{269.886322pt}}
\pgfusepath{stroke}
\pgfpathmoveto{\pgfpoint{333.791992pt}{232.825272pt}}
\pgflineto{\pgfpoint{333.778412pt}{232.825272pt}}
\pgfusepath{stroke}
\pgfpathmoveto{\pgfpoint{333.791992pt}{189.587372pt}}
\pgflineto{\pgfpoint{333.809753pt}{189.587372pt}}
\pgfusepath{stroke}
\pgfpathmoveto{\pgfpoint{333.791992pt}{195.764206pt}}
\pgflineto{\pgfpoint{333.814240pt}{195.764206pt}}
\pgfusepath{stroke}
\pgfpathmoveto{\pgfpoint{333.791992pt}{201.941055pt}}
\pgflineto{\pgfpoint{333.814178pt}{201.941055pt}}
\pgfusepath{stroke}
\pgfpathmoveto{\pgfpoint{333.791992pt}{208.117905pt}}
\pgflineto{\pgfpoint{333.818726pt}{208.117905pt}}
\pgfusepath{stroke}
\pgfpathmoveto{\pgfpoint{333.791992pt}{214.294739pt}}
\pgflineto{\pgfpoint{333.818695pt}{214.294739pt}}
\pgfusepath{stroke}
\pgfpathmoveto{\pgfpoint{333.791992pt}{220.471588pt}}
\pgflineto{\pgfpoint{333.818695pt}{220.471588pt}}
\pgfusepath{stroke}
\pgfpathmoveto{\pgfpoint{333.791992pt}{226.648422pt}}
\pgflineto{\pgfpoint{333.823151pt}{226.648422pt}}
\pgfusepath{stroke}
\pgfpathmoveto{\pgfpoint{333.791992pt}{232.825272pt}}
\pgflineto{\pgfpoint{333.823151pt}{232.825272pt}}
\pgfusepath{stroke}
\pgfpathmoveto{\pgfpoint{333.791992pt}{239.002106pt}}
\pgflineto{\pgfpoint{333.827637pt}{239.002106pt}}
\pgfusepath{stroke}
\pgfpathmoveto{\pgfpoint{333.791992pt}{245.178955pt}}
\pgflineto{\pgfpoint{333.827606pt}{245.178955pt}}
\pgfusepath{stroke}
\pgfpathmoveto{\pgfpoint{333.791992pt}{251.355804pt}}
\pgflineto{\pgfpoint{333.832092pt}{251.355804pt}}
\pgfusepath{stroke}
\pgfpathmoveto{\pgfpoint{333.791992pt}{257.532623pt}}
\pgflineto{\pgfpoint{333.832031pt}{257.532623pt}}
\pgfusepath{stroke}
\pgfpathmoveto{\pgfpoint{333.791992pt}{263.709473pt}}
\pgflineto{\pgfpoint{333.832092pt}{263.709473pt}}
\pgfusepath{stroke}
\pgfpathmoveto{\pgfpoint{333.791992pt}{269.886322pt}}
\pgflineto{\pgfpoint{333.836548pt}{269.886322pt}}
\pgfusepath{stroke}
\pgfpathmoveto{\pgfpoint{333.791992pt}{276.063141pt}}
\pgflineto{\pgfpoint{333.836517pt}{276.063141pt}}
\pgfusepath{stroke}
\pgfpathmoveto{\pgfpoint{333.791992pt}{103.111580pt}}
\pgflineto{\pgfpoint{333.791992pt}{96.934731pt}}
\pgfusepath{stroke}
\pgfpathmoveto{\pgfpoint{333.791992pt}{72.227356pt}}
\pgflineto{\pgfpoint{333.791992pt}{66.050522pt}}
\pgfusepath{stroke}
\pgfpathmoveto{\pgfpoint{333.791992pt}{78.404205pt}}
\pgflineto{\pgfpoint{333.791992pt}{72.227356pt}}
\pgfusepath{stroke}
\pgfpathmoveto{\pgfpoint{333.791992pt}{84.581039pt}}
\pgflineto{\pgfpoint{333.791992pt}{78.404205pt}}
\pgfusepath{stroke}
\pgfpathmoveto{\pgfpoint{333.791992pt}{90.757896pt}}
\pgflineto{\pgfpoint{333.791992pt}{84.581039pt}}
\pgfusepath{stroke}
\pgfpathmoveto{\pgfpoint{333.791992pt}{96.934731pt}}
\pgflineto{\pgfpoint{333.791992pt}{90.757896pt}}
\pgfusepath{stroke}
\pgfpathmoveto{\pgfpoint{333.791992pt}{109.288422pt}}
\pgflineto{\pgfpoint{333.791992pt}{103.111580pt}}
\pgfusepath{stroke}
\pgfpathmoveto{\pgfpoint{333.791992pt}{115.465263pt}}
\pgflineto{\pgfpoint{333.791992pt}{109.288422pt}}
\pgfusepath{stroke}
\pgfpathmoveto{\pgfpoint{333.791992pt}{121.642097pt}}
\pgflineto{\pgfpoint{333.791992pt}{115.465263pt}}
\pgfusepath{stroke}
\pgfpathmoveto{\pgfpoint{333.791992pt}{127.818947pt}}
\pgflineto{\pgfpoint{333.791992pt}{121.642097pt}}
\pgfusepath{stroke}
\pgfpathmoveto{\pgfpoint{333.791992pt}{133.995789pt}}
\pgflineto{\pgfpoint{333.791992pt}{127.818947pt}}
\pgfusepath{stroke}
\pgfpathmoveto{\pgfpoint{333.791992pt}{140.172638pt}}
\pgflineto{\pgfpoint{333.791992pt}{133.995789pt}}
\pgfusepath{stroke}
\pgfpathmoveto{\pgfpoint{333.791992pt}{146.349472pt}}
\pgflineto{\pgfpoint{333.791992pt}{140.172638pt}}
\pgfusepath{stroke}
\pgfpathmoveto{\pgfpoint{333.791992pt}{152.526306pt}}
\pgflineto{\pgfpoint{333.791992pt}{146.349472pt}}
\pgfusepath{stroke}
\pgfpathmoveto{\pgfpoint{333.791992pt}{158.703156pt}}
\pgflineto{\pgfpoint{333.791992pt}{152.526306pt}}
\pgfusepath{stroke}
\pgfpathmoveto{\pgfpoint{333.791992pt}{47.519989pt}}
\pgflineto{\pgfpoint{333.774109pt}{47.519989pt}}
\pgfusepath{stroke}
\pgfpathmoveto{\pgfpoint{333.791992pt}{53.696838pt}}
\pgflineto{\pgfpoint{333.774109pt}{53.696838pt}}
\pgfusepath{stroke}
\pgfpathmoveto{\pgfpoint{333.791992pt}{59.873672pt}}
\pgflineto{\pgfpoint{333.774109pt}{59.873672pt}}
\pgfusepath{stroke}
\pgfpathmoveto{\pgfpoint{333.791992pt}{53.696838pt}}
\pgflineto{\pgfpoint{333.791992pt}{47.519989pt}}
\pgfusepath{stroke}
\pgfpathmoveto{\pgfpoint{333.791992pt}{47.519989pt}}
\pgflineto{\pgfpoint{333.810089pt}{47.519989pt}}
\pgfusepath{stroke}
\pgfpathmoveto{\pgfpoint{333.791992pt}{66.050522pt}}
\pgflineto{\pgfpoint{333.791992pt}{59.873672pt}}
\pgfusepath{stroke}
\pgfpathmoveto{\pgfpoint{333.791992pt}{59.873672pt}}
\pgflineto{\pgfpoint{333.791992pt}{53.696838pt}}
\pgfusepath{stroke}
\pgfpathmoveto{\pgfpoint{333.810089pt}{47.519989pt}}
\pgflineto{\pgfpoint{342.702118pt}{47.519989pt}}
\pgfusepath{stroke}
\pgfpathmoveto{\pgfpoint{342.719971pt}{53.696838pt}}
\pgflineto{\pgfpoint{333.791992pt}{53.696838pt}}
\pgfusepath{stroke}
\pgfpathmoveto{\pgfpoint{342.719971pt}{59.873672pt}}
\pgflineto{\pgfpoint{333.791992pt}{59.873672pt}}
\pgfusepath{stroke}
\pgfpathmoveto{\pgfpoint{342.719971pt}{66.050522pt}}
\pgflineto{\pgfpoint{333.791992pt}{66.050522pt}}
\pgfusepath{stroke}
\pgfpathmoveto{\pgfpoint{342.719971pt}{72.227356pt}}
\pgflineto{\pgfpoint{333.791992pt}{72.227356pt}}
\pgfusepath{stroke}
\pgfpathmoveto{\pgfpoint{342.719971pt}{78.404205pt}}
\pgflineto{\pgfpoint{333.791992pt}{78.404205pt}}
\pgfusepath{stroke}
\pgfpathmoveto{\pgfpoint{342.719971pt}{84.581039pt}}
\pgflineto{\pgfpoint{333.791992pt}{84.581039pt}}
\pgfusepath{stroke}
\pgfpathmoveto{\pgfpoint{342.719971pt}{90.757896pt}}
\pgflineto{\pgfpoint{333.791992pt}{90.757896pt}}
\pgfusepath{stroke}
\pgfpathmoveto{\pgfpoint{342.719971pt}{96.934731pt}}
\pgflineto{\pgfpoint{333.791992pt}{96.934731pt}}
\pgfusepath{stroke}
\pgfpathmoveto{\pgfpoint{342.719971pt}{103.111580pt}}
\pgflineto{\pgfpoint{333.791992pt}{103.111580pt}}
\pgfusepath{stroke}
\pgfpathmoveto{\pgfpoint{342.719971pt}{109.288422pt}}
\pgflineto{\pgfpoint{333.791992pt}{109.288422pt}}
\pgfusepath{stroke}
\pgfpathmoveto{\pgfpoint{342.719971pt}{115.465263pt}}
\pgflineto{\pgfpoint{333.791992pt}{115.465263pt}}
\pgfusepath{stroke}
\pgfpathmoveto{\pgfpoint{342.719971pt}{121.642097pt}}
\pgflineto{\pgfpoint{333.791992pt}{121.642097pt}}
\pgfusepath{stroke}
\pgfpathmoveto{\pgfpoint{342.719971pt}{127.818947pt}}
\pgflineto{\pgfpoint{333.791992pt}{127.818947pt}}
\pgfusepath{stroke}
\pgfpathmoveto{\pgfpoint{342.719971pt}{133.995789pt}}
\pgflineto{\pgfpoint{333.791992pt}{133.995789pt}}
\pgfusepath{stroke}
\pgfpathmoveto{\pgfpoint{342.719971pt}{140.172638pt}}
\pgflineto{\pgfpoint{333.791992pt}{140.172638pt}}
\pgfusepath{stroke}
\pgfpathmoveto{\pgfpoint{342.719971pt}{146.349472pt}}
\pgflineto{\pgfpoint{333.791992pt}{146.349472pt}}
\pgfusepath{stroke}
\pgfpathmoveto{\pgfpoint{342.719971pt}{152.526306pt}}
\pgflineto{\pgfpoint{333.791992pt}{152.526306pt}}
\pgfusepath{stroke}
\pgfpathmoveto{\pgfpoint{342.719971pt}{158.703156pt}}
\pgflineto{\pgfpoint{333.791992pt}{158.703156pt}}
\pgfusepath{stroke}
\pgfpathmoveto{\pgfpoint{342.719971pt}{164.880005pt}}
\pgflineto{\pgfpoint{333.805328pt}{164.880005pt}}
\pgfusepath{stroke}
\pgfpathmoveto{\pgfpoint{342.719971pt}{171.056854pt}}
\pgflineto{\pgfpoint{333.805298pt}{171.056854pt}}
\pgfusepath{stroke}
\pgfpathmoveto{\pgfpoint{342.719971pt}{177.233673pt}}
\pgflineto{\pgfpoint{333.809784pt}{177.233673pt}}
\pgfusepath{stroke}
\pgfpathmoveto{\pgfpoint{342.719971pt}{183.410522pt}}
\pgflineto{\pgfpoint{333.809753pt}{183.410522pt}}
\pgfusepath{stroke}
\pgfpathmoveto{\pgfpoint{342.719971pt}{189.587372pt}}
\pgflineto{\pgfpoint{333.809753pt}{189.587372pt}}
\pgfusepath{stroke}
\pgfpathmoveto{\pgfpoint{342.719971pt}{195.764206pt}}
\pgflineto{\pgfpoint{333.814240pt}{195.764206pt}}
\pgfusepath{stroke}
\pgfpathmoveto{\pgfpoint{342.719971pt}{201.941055pt}}
\pgflineto{\pgfpoint{333.814178pt}{201.941055pt}}
\pgfusepath{stroke}
\pgfpathmoveto{\pgfpoint{342.719971pt}{208.117905pt}}
\pgflineto{\pgfpoint{333.818726pt}{208.117905pt}}
\pgfusepath{stroke}
\pgfpathmoveto{\pgfpoint{333.818695pt}{214.294739pt}}
\pgflineto{\pgfpoint{342.706421pt}{214.294739pt}}
\pgfusepath{stroke}
\pgfpathmoveto{\pgfpoint{333.818695pt}{220.471588pt}}
\pgflineto{\pgfpoint{342.706421pt}{220.471588pt}}
\pgfusepath{stroke}
\pgfpathmoveto{\pgfpoint{333.823151pt}{226.648422pt}}
\pgflineto{\pgfpoint{342.701874pt}{226.648422pt}}
\pgfusepath{stroke}
\pgfpathmoveto{\pgfpoint{333.823151pt}{232.825272pt}}
\pgflineto{\pgfpoint{342.697388pt}{232.825272pt}}
\pgfusepath{stroke}
\pgfpathmoveto{\pgfpoint{333.827637pt}{239.002106pt}}
\pgflineto{\pgfpoint{342.697357pt}{239.002106pt}}
\pgfusepath{stroke}
\pgfpathmoveto{\pgfpoint{333.827606pt}{245.178955pt}}
\pgflineto{\pgfpoint{342.692871pt}{245.178955pt}}
\pgfusepath{stroke}
\pgfpathmoveto{\pgfpoint{333.832092pt}{251.355804pt}}
\pgflineto{\pgfpoint{342.692871pt}{251.355804pt}}
\pgfusepath{stroke}
\pgfpathmoveto{\pgfpoint{333.832031pt}{257.532623pt}}
\pgflineto{\pgfpoint{342.688293pt}{257.532623pt}}
\pgfusepath{stroke}
\pgfpathmoveto{\pgfpoint{333.832092pt}{263.709473pt}}
\pgflineto{\pgfpoint{342.688324pt}{263.709473pt}}
\pgfusepath{stroke}
\pgfpathmoveto{\pgfpoint{333.836548pt}{269.886322pt}}
\pgflineto{\pgfpoint{342.683838pt}{269.886322pt}}
\pgfusepath{stroke}
\pgfpathmoveto{\pgfpoint{333.836517pt}{276.063141pt}}
\pgflineto{\pgfpoint{342.679260pt}{276.063141pt}}
\pgfusepath{stroke}
\pgfpathmoveto{\pgfpoint{342.719971pt}{152.526306pt}}
\pgflineto{\pgfpoint{342.719971pt}{146.349472pt}}
\pgfusepath{stroke}
\pgfpathmoveto{\pgfpoint{342.719971pt}{140.172638pt}}
\pgflineto{\pgfpoint{342.719971pt}{133.995789pt}}
\pgfusepath{stroke}
\pgfpathmoveto{\pgfpoint{342.719971pt}{146.349472pt}}
\pgflineto{\pgfpoint{342.719971pt}{140.172638pt}}
\pgfusepath{stroke}
\pgfpathmoveto{\pgfpoint{342.719971pt}{158.703156pt}}
\pgflineto{\pgfpoint{342.719971pt}{152.526306pt}}
\pgfusepath{stroke}
\pgfpathmoveto{\pgfpoint{342.719971pt}{140.172638pt}}
\pgflineto{\pgfpoint{342.737732pt}{140.172638pt}}
\pgfusepath{stroke}
\pgfpathmoveto{\pgfpoint{342.719971pt}{146.349472pt}}
\pgflineto{\pgfpoint{342.733215pt}{146.349472pt}}
\pgfusepath{stroke}
\pgfpathmoveto{\pgfpoint{342.719971pt}{152.526306pt}}
\pgflineto{\pgfpoint{342.733185pt}{152.526306pt}}
\pgfusepath{stroke}
\pgfpathmoveto{\pgfpoint{342.719971pt}{189.587372pt}}
\pgflineto{\pgfpoint{342.719971pt}{183.410522pt}}
\pgfusepath{stroke}
\pgfpathmoveto{\pgfpoint{342.719971pt}{164.880005pt}}
\pgflineto{\pgfpoint{342.719971pt}{158.703156pt}}
\pgfusepath{stroke}
\pgfpathmoveto{\pgfpoint{342.719971pt}{171.056854pt}}
\pgflineto{\pgfpoint{342.719971pt}{164.880005pt}}
\pgfusepath{stroke}
\pgfpathmoveto{\pgfpoint{342.719971pt}{177.233673pt}}
\pgflineto{\pgfpoint{342.719971pt}{171.056854pt}}
\pgfusepath{stroke}
\pgfpathmoveto{\pgfpoint{342.719971pt}{183.410522pt}}
\pgflineto{\pgfpoint{342.719971pt}{177.233673pt}}
\pgfusepath{stroke}
\pgfpathmoveto{\pgfpoint{342.719971pt}{195.764206pt}}
\pgflineto{\pgfpoint{342.719971pt}{189.587372pt}}
\pgfusepath{stroke}
\pgfpathmoveto{\pgfpoint{342.719971pt}{201.941055pt}}
\pgflineto{\pgfpoint{342.719971pt}{195.764206pt}}
\pgfusepath{stroke}
\pgfpathmoveto{\pgfpoint{342.719971pt}{208.117905pt}}
\pgflineto{\pgfpoint{342.719971pt}{201.941055pt}}
\pgfusepath{stroke}
\pgfpathmoveto{\pgfpoint{342.719971pt}{158.703156pt}}
\pgflineto{\pgfpoint{342.733154pt}{158.703156pt}}
\pgfusepath{stroke}
\pgfpathmoveto{\pgfpoint{342.719971pt}{164.880005pt}}
\pgflineto{\pgfpoint{342.737640pt}{164.880005pt}}
\pgfusepath{stroke}
\pgfpathmoveto{\pgfpoint{342.719971pt}{226.648422pt}}
\pgflineto{\pgfpoint{342.701874pt}{226.648422pt}}
\pgfusepath{stroke}
\pgfpathmoveto{\pgfpoint{342.719971pt}{232.825272pt}}
\pgflineto{\pgfpoint{342.697388pt}{232.825272pt}}
\pgfusepath{stroke}
\pgfpathmoveto{\pgfpoint{342.719971pt}{239.002106pt}}
\pgflineto{\pgfpoint{342.697357pt}{239.002106pt}}
\pgfusepath{stroke}
\pgfpathmoveto{\pgfpoint{342.719971pt}{245.178955pt}}
\pgflineto{\pgfpoint{342.692871pt}{245.178955pt}}
\pgfusepath{stroke}
\pgfpathmoveto{\pgfpoint{342.719971pt}{251.355804pt}}
\pgflineto{\pgfpoint{342.692871pt}{251.355804pt}}
\pgfusepath{stroke}
\pgfpathmoveto{\pgfpoint{342.719971pt}{257.532623pt}}
\pgflineto{\pgfpoint{342.688293pt}{257.532623pt}}
\pgfusepath{stroke}
\pgfpathmoveto{\pgfpoint{342.719971pt}{263.709473pt}}
\pgflineto{\pgfpoint{342.688324pt}{263.709473pt}}
\pgfusepath{stroke}
\pgfpathmoveto{\pgfpoint{342.719971pt}{269.886322pt}}
\pgflineto{\pgfpoint{342.683838pt}{269.886322pt}}
\pgfusepath{stroke}
\pgfpathmoveto{\pgfpoint{342.719971pt}{276.063141pt}}
\pgflineto{\pgfpoint{342.679260pt}{276.063141pt}}
\pgfusepath{stroke}
\pgfpathmoveto{\pgfpoint{342.719971pt}{245.178955pt}}
\pgflineto{\pgfpoint{342.719971pt}{239.002106pt}}
\pgfusepath{stroke}
\pgfpathmoveto{\pgfpoint{342.719971pt}{214.294739pt}}
\pgflineto{\pgfpoint{342.719971pt}{208.117905pt}}
\pgfusepath{stroke}
\pgfpathmoveto{\pgfpoint{342.719971pt}{220.471588pt}}
\pgflineto{\pgfpoint{342.719971pt}{214.294739pt}}
\pgfusepath{stroke}
\pgfpathmoveto{\pgfpoint{342.719971pt}{226.648422pt}}
\pgflineto{\pgfpoint{342.719971pt}{220.471588pt}}
\pgfusepath{stroke}
\pgfpathmoveto{\pgfpoint{342.719971pt}{232.825272pt}}
\pgflineto{\pgfpoint{342.719971pt}{226.648422pt}}
\pgfusepath{stroke}
\pgfpathmoveto{\pgfpoint{342.719971pt}{239.002106pt}}
\pgflineto{\pgfpoint{342.719971pt}{232.825272pt}}
\pgfusepath{stroke}
\pgfpathmoveto{\pgfpoint{342.719971pt}{251.355804pt}}
\pgflineto{\pgfpoint{342.719971pt}{245.178955pt}}
\pgfusepath{stroke}
\pgfpathmoveto{\pgfpoint{342.719971pt}{257.532623pt}}
\pgflineto{\pgfpoint{342.719971pt}{251.355804pt}}
\pgfusepath{stroke}
\pgfpathmoveto{\pgfpoint{342.719971pt}{263.709473pt}}
\pgflineto{\pgfpoint{342.719971pt}{257.532623pt}}
\pgfusepath{stroke}
\pgfpathmoveto{\pgfpoint{342.719971pt}{269.886322pt}}
\pgflineto{\pgfpoint{342.719971pt}{263.709473pt}}
\pgfusepath{stroke}
\pgfpathmoveto{\pgfpoint{342.719971pt}{276.063141pt}}
\pgflineto{\pgfpoint{342.719971pt}{269.886322pt}}
\pgfusepath{stroke}
\pgfpathmoveto{\pgfpoint{342.719971pt}{214.294739pt}}
\pgflineto{\pgfpoint{342.706421pt}{214.294739pt}}
\pgfusepath{stroke}
\pgfpathmoveto{\pgfpoint{342.719971pt}{220.471588pt}}
\pgflineto{\pgfpoint{342.706421pt}{220.471588pt}}
\pgfusepath{stroke}
\pgfpathmoveto{\pgfpoint{342.719971pt}{171.056854pt}}
\pgflineto{\pgfpoint{342.737610pt}{171.056854pt}}
\pgfusepath{stroke}
\pgfpathmoveto{\pgfpoint{342.719971pt}{177.233673pt}}
\pgflineto{\pgfpoint{342.742096pt}{177.233673pt}}
\pgfusepath{stroke}
\pgfpathmoveto{\pgfpoint{342.719971pt}{183.410522pt}}
\pgflineto{\pgfpoint{342.742096pt}{183.410522pt}}
\pgfusepath{stroke}
\pgfpathmoveto{\pgfpoint{342.719971pt}{189.587372pt}}
\pgflineto{\pgfpoint{342.746552pt}{189.587372pt}}
\pgfusepath{stroke}
\pgfpathmoveto{\pgfpoint{342.719971pt}{195.764206pt}}
\pgflineto{\pgfpoint{342.746521pt}{195.764206pt}}
\pgfusepath{stroke}
\pgfpathmoveto{\pgfpoint{342.719971pt}{201.941055pt}}
\pgflineto{\pgfpoint{342.746490pt}{201.941055pt}}
\pgfusepath{stroke}
\pgfpathmoveto{\pgfpoint{342.719971pt}{208.117905pt}}
\pgflineto{\pgfpoint{342.750977pt}{208.117905pt}}
\pgfusepath{stroke}
\pgfpathmoveto{\pgfpoint{342.719971pt}{214.294739pt}}
\pgflineto{\pgfpoint{342.750946pt}{214.294739pt}}
\pgfusepath{stroke}
\pgfpathmoveto{\pgfpoint{342.719971pt}{220.471588pt}}
\pgflineto{\pgfpoint{342.755432pt}{220.471588pt}}
\pgfusepath{stroke}
\pgfpathmoveto{\pgfpoint{342.719971pt}{226.648422pt}}
\pgflineto{\pgfpoint{342.755402pt}{226.648422pt}}
\pgfusepath{stroke}
\pgfpathmoveto{\pgfpoint{342.719971pt}{232.825272pt}}
\pgflineto{\pgfpoint{342.755402pt}{232.825272pt}}
\pgfusepath{stroke}
\pgfpathmoveto{\pgfpoint{342.719971pt}{239.002106pt}}
\pgflineto{\pgfpoint{342.759857pt}{239.002106pt}}
\pgfusepath{stroke}
\pgfpathmoveto{\pgfpoint{342.719971pt}{245.178955pt}}
\pgflineto{\pgfpoint{342.759827pt}{245.178955pt}}
\pgfusepath{stroke}
\pgfpathmoveto{\pgfpoint{342.719971pt}{251.355804pt}}
\pgflineto{\pgfpoint{342.764313pt}{251.355804pt}}
\pgfusepath{stroke}
\pgfpathmoveto{\pgfpoint{342.719971pt}{257.532623pt}}
\pgflineto{\pgfpoint{342.764252pt}{257.532623pt}}
\pgfusepath{stroke}
\pgfpathmoveto{\pgfpoint{342.719971pt}{263.709473pt}}
\pgflineto{\pgfpoint{342.768768pt}{263.709473pt}}
\pgfusepath{stroke}
\pgfpathmoveto{\pgfpoint{342.719971pt}{269.886322pt}}
\pgflineto{\pgfpoint{342.768738pt}{269.886322pt}}
\pgfusepath{stroke}
\pgfpathmoveto{\pgfpoint{342.719971pt}{90.757896pt}}
\pgflineto{\pgfpoint{342.719971pt}{84.581039pt}}
\pgfusepath{stroke}
\pgfpathmoveto{\pgfpoint{342.719971pt}{59.873672pt}}
\pgflineto{\pgfpoint{342.719971pt}{53.696838pt}}
\pgfusepath{stroke}
\pgfpathmoveto{\pgfpoint{342.719971pt}{66.050522pt}}
\pgflineto{\pgfpoint{342.719971pt}{59.873672pt}}
\pgfusepath{stroke}
\pgfpathmoveto{\pgfpoint{342.719971pt}{72.227356pt}}
\pgflineto{\pgfpoint{342.719971pt}{66.050522pt}}
\pgfusepath{stroke}
\pgfpathmoveto{\pgfpoint{342.719971pt}{78.404205pt}}
\pgflineto{\pgfpoint{342.719971pt}{72.227356pt}}
\pgfusepath{stroke}
\pgfpathmoveto{\pgfpoint{342.719971pt}{84.581039pt}}
\pgflineto{\pgfpoint{342.719971pt}{78.404205pt}}
\pgfusepath{stroke}
\pgfpathmoveto{\pgfpoint{342.719971pt}{96.934731pt}}
\pgflineto{\pgfpoint{342.719971pt}{90.757896pt}}
\pgfusepath{stroke}
\pgfpathmoveto{\pgfpoint{342.719971pt}{103.111580pt}}
\pgflineto{\pgfpoint{342.719971pt}{96.934731pt}}
\pgfusepath{stroke}
\pgfpathmoveto{\pgfpoint{342.719971pt}{109.288422pt}}
\pgflineto{\pgfpoint{342.719971pt}{103.111580pt}}
\pgfusepath{stroke}
\pgfpathmoveto{\pgfpoint{342.719971pt}{115.465263pt}}
\pgflineto{\pgfpoint{342.719971pt}{109.288422pt}}
\pgfusepath{stroke}
\pgfpathmoveto{\pgfpoint{342.719971pt}{121.642097pt}}
\pgflineto{\pgfpoint{342.719971pt}{115.465263pt}}
\pgfusepath{stroke}
\pgfpathmoveto{\pgfpoint{342.719971pt}{127.818947pt}}
\pgflineto{\pgfpoint{342.719971pt}{121.642097pt}}
\pgfusepath{stroke}
\pgfpathmoveto{\pgfpoint{342.719971pt}{133.995789pt}}
\pgflineto{\pgfpoint{342.719971pt}{127.818947pt}}
\pgfusepath{stroke}
\pgfpathmoveto{\pgfpoint{342.719971pt}{47.519989pt}}
\pgflineto{\pgfpoint{342.702118pt}{47.519989pt}}
\pgfusepath{stroke}
\pgfpathmoveto{\pgfpoint{342.719971pt}{53.696838pt}}
\pgflineto{\pgfpoint{342.719971pt}{47.519989pt}}
\pgfusepath{stroke}
\pgfpathmoveto{\pgfpoint{342.719971pt}{47.519989pt}}
\pgflineto{\pgfpoint{351.629883pt}{47.519989pt}}
\pgfusepath{stroke}
\pgfpathmoveto{\pgfpoint{342.719971pt}{53.696838pt}}
\pgflineto{\pgfpoint{351.629883pt}{53.696838pt}}
\pgfusepath{stroke}
\pgfpathmoveto{\pgfpoint{342.719971pt}{59.873672pt}}
\pgflineto{\pgfpoint{351.629883pt}{59.873672pt}}
\pgfusepath{stroke}
\pgfpathmoveto{\pgfpoint{351.647980pt}{66.050522pt}}
\pgflineto{\pgfpoint{342.719971pt}{66.050522pt}}
\pgfusepath{stroke}
\pgfpathmoveto{\pgfpoint{351.647980pt}{72.227356pt}}
\pgflineto{\pgfpoint{342.719971pt}{72.227356pt}}
\pgfusepath{stroke}
\pgfpathmoveto{\pgfpoint{351.647980pt}{78.404205pt}}
\pgflineto{\pgfpoint{342.719971pt}{78.404205pt}}
\pgfusepath{stroke}
\pgfpathmoveto{\pgfpoint{351.647980pt}{84.581039pt}}
\pgflineto{\pgfpoint{342.719971pt}{84.581039pt}}
\pgfusepath{stroke}
\pgfpathmoveto{\pgfpoint{351.647980pt}{90.757896pt}}
\pgflineto{\pgfpoint{342.719971pt}{90.757896pt}}
\pgfusepath{stroke}
\pgfpathmoveto{\pgfpoint{351.647980pt}{96.934731pt}}
\pgflineto{\pgfpoint{342.719971pt}{96.934731pt}}
\pgfusepath{stroke}
\pgfpathmoveto{\pgfpoint{351.647980pt}{103.111580pt}}
\pgflineto{\pgfpoint{342.719971pt}{103.111580pt}}
\pgfusepath{stroke}
\pgfpathmoveto{\pgfpoint{351.647980pt}{109.288422pt}}
\pgflineto{\pgfpoint{342.719971pt}{109.288422pt}}
\pgfusepath{stroke}
\pgfpathmoveto{\pgfpoint{351.647980pt}{115.465263pt}}
\pgflineto{\pgfpoint{342.719971pt}{115.465263pt}}
\pgfusepath{stroke}
\pgfpathmoveto{\pgfpoint{351.647980pt}{121.642097pt}}
\pgflineto{\pgfpoint{342.719971pt}{121.642097pt}}
\pgfusepath{stroke}
\pgfpathmoveto{\pgfpoint{351.647980pt}{127.818947pt}}
\pgflineto{\pgfpoint{342.719971pt}{127.818947pt}}
\pgfusepath{stroke}
\pgfpathmoveto{\pgfpoint{351.647980pt}{133.995789pt}}
\pgflineto{\pgfpoint{342.719971pt}{133.995789pt}}
\pgfusepath{stroke}
\pgfpathmoveto{\pgfpoint{351.647980pt}{140.172638pt}}
\pgflineto{\pgfpoint{342.737732pt}{140.172638pt}}
\pgfusepath{stroke}
\pgfpathmoveto{\pgfpoint{351.647980pt}{146.349472pt}}
\pgflineto{\pgfpoint{342.733215pt}{146.349472pt}}
\pgfusepath{stroke}
\pgfpathmoveto{\pgfpoint{351.647980pt}{152.526306pt}}
\pgflineto{\pgfpoint{342.733185pt}{152.526306pt}}
\pgfusepath{stroke}
\pgfpathmoveto{\pgfpoint{351.647980pt}{158.703156pt}}
\pgflineto{\pgfpoint{342.733154pt}{158.703156pt}}
\pgfusepath{stroke}
\pgfpathmoveto{\pgfpoint{351.647980pt}{164.880005pt}}
\pgflineto{\pgfpoint{342.737640pt}{164.880005pt}}
\pgfusepath{stroke}
\pgfpathmoveto{\pgfpoint{351.647980pt}{171.056854pt}}
\pgflineto{\pgfpoint{342.737610pt}{171.056854pt}}
\pgfusepath{stroke}
\pgfpathmoveto{\pgfpoint{351.647980pt}{177.233673pt}}
\pgflineto{\pgfpoint{342.742096pt}{177.233673pt}}
\pgfusepath{stroke}
\pgfpathmoveto{\pgfpoint{351.647980pt}{183.410522pt}}
\pgflineto{\pgfpoint{342.742096pt}{183.410522pt}}
\pgfusepath{stroke}
\pgfpathmoveto{\pgfpoint{351.647980pt}{189.587372pt}}
\pgflineto{\pgfpoint{342.746552pt}{189.587372pt}}
\pgfusepath{stroke}
\pgfpathmoveto{\pgfpoint{351.647980pt}{195.764206pt}}
\pgflineto{\pgfpoint{342.746521pt}{195.764206pt}}
\pgfusepath{stroke}
\pgfpathmoveto{\pgfpoint{351.647980pt}{201.941055pt}}
\pgflineto{\pgfpoint{342.746490pt}{201.941055pt}}
\pgfusepath{stroke}
\pgfpathmoveto{\pgfpoint{351.647980pt}{208.117905pt}}
\pgflineto{\pgfpoint{342.750977pt}{208.117905pt}}
\pgfusepath{stroke}
\pgfpathmoveto{\pgfpoint{351.647980pt}{214.294739pt}}
\pgflineto{\pgfpoint{342.750946pt}{214.294739pt}}
\pgfusepath{stroke}
\pgfpathmoveto{\pgfpoint{351.647980pt}{220.471588pt}}
\pgflineto{\pgfpoint{342.755432pt}{220.471588pt}}
\pgfusepath{stroke}
\pgfpathmoveto{\pgfpoint{351.647980pt}{226.648422pt}}
\pgflineto{\pgfpoint{342.755402pt}{226.648422pt}}
\pgfusepath{stroke}
\pgfpathmoveto{\pgfpoint{351.647980pt}{232.825272pt}}
\pgflineto{\pgfpoint{342.755402pt}{232.825272pt}}
\pgfusepath{stroke}
\pgfpathmoveto{\pgfpoint{342.759857pt}{239.002106pt}}
\pgflineto{\pgfpoint{351.634369pt}{239.002106pt}}
\pgfusepath{stroke}
\pgfpathmoveto{\pgfpoint{342.759827pt}{245.178955pt}}
\pgflineto{\pgfpoint{351.629852pt}{245.178955pt}}
\pgfusepath{stroke}
\pgfpathmoveto{\pgfpoint{342.764313pt}{251.355804pt}}
\pgflineto{\pgfpoint{351.629852pt}{251.355804pt}}
\pgfusepath{stroke}
\pgfpathmoveto{\pgfpoint{342.764252pt}{257.532623pt}}
\pgflineto{\pgfpoint{351.625305pt}{257.532623pt}}
\pgfusepath{stroke}
\pgfpathmoveto{\pgfpoint{342.768768pt}{263.709473pt}}
\pgflineto{\pgfpoint{351.625305pt}{263.709473pt}}
\pgfusepath{stroke}
\pgfpathmoveto{\pgfpoint{342.768738pt}{269.886322pt}}
\pgflineto{\pgfpoint{351.620789pt}{269.886322pt}}
\pgfusepath{stroke}
\pgfpathmoveto{\pgfpoint{351.647980pt}{164.880005pt}}
\pgflineto{\pgfpoint{351.647980pt}{158.703156pt}}
\pgfusepath{stroke}
\pgfpathmoveto{\pgfpoint{351.647980pt}{158.703156pt}}
\pgflineto{\pgfpoint{351.647980pt}{152.526306pt}}
\pgfusepath{stroke}
\pgfpathmoveto{\pgfpoint{351.647980pt}{171.056854pt}}
\pgflineto{\pgfpoint{351.647980pt}{164.880005pt}}
\pgfusepath{stroke}
\pgfpathmoveto{\pgfpoint{351.647980pt}{177.233673pt}}
\pgflineto{\pgfpoint{351.647980pt}{171.056854pt}}
\pgfusepath{stroke}
\pgfpathmoveto{\pgfpoint{351.647980pt}{183.410522pt}}
\pgflineto{\pgfpoint{351.647980pt}{177.233673pt}}
\pgfusepath{stroke}
\pgfpathmoveto{\pgfpoint{351.647980pt}{158.703156pt}}
\pgflineto{\pgfpoint{351.661499pt}{158.703156pt}}
\pgfusepath{stroke}
\pgfpathmoveto{\pgfpoint{351.647980pt}{164.880005pt}}
\pgflineto{\pgfpoint{351.661530pt}{164.880005pt}}
\pgfusepath{stroke}
\pgfpathmoveto{\pgfpoint{351.647980pt}{171.056854pt}}
\pgflineto{\pgfpoint{351.661530pt}{171.056854pt}}
\pgfusepath{stroke}
\pgfpathmoveto{\pgfpoint{351.647980pt}{177.233673pt}}
\pgflineto{\pgfpoint{351.666046pt}{177.233673pt}}
\pgfusepath{stroke}
\pgfpathmoveto{\pgfpoint{351.647980pt}{208.117905pt}}
\pgflineto{\pgfpoint{351.647980pt}{201.941055pt}}
\pgfusepath{stroke}
\pgfpathmoveto{\pgfpoint{351.647980pt}{189.587372pt}}
\pgflineto{\pgfpoint{351.647980pt}{183.410522pt}}
\pgfusepath{stroke}
\pgfpathmoveto{\pgfpoint{351.647980pt}{195.764206pt}}
\pgflineto{\pgfpoint{351.647980pt}{189.587372pt}}
\pgfusepath{stroke}
\pgfpathmoveto{\pgfpoint{351.647980pt}{201.941055pt}}
\pgflineto{\pgfpoint{351.647980pt}{195.764206pt}}
\pgfusepath{stroke}
\pgfpathmoveto{\pgfpoint{351.647980pt}{214.294739pt}}
\pgflineto{\pgfpoint{351.647980pt}{208.117905pt}}
\pgfusepath{stroke}
\pgfpathmoveto{\pgfpoint{351.647980pt}{220.471588pt}}
\pgflineto{\pgfpoint{351.647980pt}{214.294739pt}}
\pgfusepath{stroke}
\pgfpathmoveto{\pgfpoint{351.647980pt}{226.648422pt}}
\pgflineto{\pgfpoint{351.647980pt}{220.471588pt}}
\pgfusepath{stroke}
\pgfpathmoveto{\pgfpoint{351.647980pt}{232.825272pt}}
\pgflineto{\pgfpoint{351.647980pt}{226.648422pt}}
\pgfusepath{stroke}
\pgfpathmoveto{\pgfpoint{351.647980pt}{183.410522pt}}
\pgflineto{\pgfpoint{351.666046pt}{183.410522pt}}
\pgfusepath{stroke}
\pgfpathmoveto{\pgfpoint{351.647980pt}{239.002106pt}}
\pgflineto{\pgfpoint{351.634369pt}{239.002106pt}}
\pgfusepath{stroke}
\pgfpathmoveto{\pgfpoint{351.647980pt}{245.178955pt}}
\pgflineto{\pgfpoint{351.629852pt}{245.178955pt}}
\pgfusepath{stroke}
\pgfpathmoveto{\pgfpoint{351.647980pt}{251.355804pt}}
\pgflineto{\pgfpoint{351.629852pt}{251.355804pt}}
\pgfusepath{stroke}
\pgfpathmoveto{\pgfpoint{351.647980pt}{257.532623pt}}
\pgflineto{\pgfpoint{351.625305pt}{257.532623pt}}
\pgfusepath{stroke}
\pgfpathmoveto{\pgfpoint{351.647980pt}{263.709473pt}}
\pgflineto{\pgfpoint{351.625305pt}{263.709473pt}}
\pgfusepath{stroke}
\pgfpathmoveto{\pgfpoint{351.647980pt}{269.886322pt}}
\pgflineto{\pgfpoint{351.620789pt}{269.886322pt}}
\pgfusepath{stroke}
\pgfpathmoveto{\pgfpoint{351.647980pt}{239.002106pt}}
\pgflineto{\pgfpoint{351.647980pt}{232.825272pt}}
\pgfusepath{stroke}
\pgfpathmoveto{\pgfpoint{351.647980pt}{245.178955pt}}
\pgflineto{\pgfpoint{351.647980pt}{239.002106pt}}
\pgfusepath{stroke}
\pgfpathmoveto{\pgfpoint{351.647980pt}{251.355804pt}}
\pgflineto{\pgfpoint{351.647980pt}{245.178955pt}}
\pgfusepath{stroke}
\pgfpathmoveto{\pgfpoint{351.647980pt}{257.532623pt}}
\pgflineto{\pgfpoint{351.647980pt}{251.355804pt}}
\pgfusepath{stroke}
\pgfpathmoveto{\pgfpoint{351.647980pt}{263.709473pt}}
\pgflineto{\pgfpoint{351.647980pt}{257.532623pt}}
\pgfusepath{stroke}
\pgfpathmoveto{\pgfpoint{351.647980pt}{269.886322pt}}
\pgflineto{\pgfpoint{351.647980pt}{263.709473pt}}
\pgfusepath{stroke}
\pgfpathmoveto{\pgfpoint{351.647980pt}{189.587372pt}}
\pgflineto{\pgfpoint{351.670563pt}{189.587372pt}}
\pgfusepath{stroke}
\pgfpathmoveto{\pgfpoint{351.647980pt}{195.764206pt}}
\pgflineto{\pgfpoint{351.670532pt}{195.764206pt}}
\pgfusepath{stroke}
\pgfpathmoveto{\pgfpoint{351.647980pt}{201.941055pt}}
\pgflineto{\pgfpoint{351.670563pt}{201.941055pt}}
\pgfusepath{stroke}
\pgfpathmoveto{\pgfpoint{351.647980pt}{208.117905pt}}
\pgflineto{\pgfpoint{351.675079pt}{208.117905pt}}
\pgfusepath{stroke}
\pgfpathmoveto{\pgfpoint{351.647980pt}{214.294739pt}}
\pgflineto{\pgfpoint{351.675079pt}{214.294739pt}}
\pgfusepath{stroke}
\pgfpathmoveto{\pgfpoint{351.647980pt}{220.471588pt}}
\pgflineto{\pgfpoint{351.679596pt}{220.471588pt}}
\pgfusepath{stroke}
\pgfpathmoveto{\pgfpoint{351.647980pt}{226.648422pt}}
\pgflineto{\pgfpoint{351.679565pt}{226.648422pt}}
\pgfusepath{stroke}
\pgfpathmoveto{\pgfpoint{351.647980pt}{232.825272pt}}
\pgflineto{\pgfpoint{351.684113pt}{232.825272pt}}
\pgfusepath{stroke}
\pgfpathmoveto{\pgfpoint{351.647980pt}{239.002106pt}}
\pgflineto{\pgfpoint{351.684082pt}{239.002106pt}}
\pgfusepath{stroke}
\pgfpathmoveto{\pgfpoint{351.647980pt}{245.178955pt}}
\pgflineto{\pgfpoint{351.684113pt}{245.178955pt}}
\pgfusepath{stroke}
\pgfpathmoveto{\pgfpoint{351.647980pt}{251.355804pt}}
\pgflineto{\pgfpoint{351.688629pt}{251.355804pt}}
\pgfusepath{stroke}
\pgfpathmoveto{\pgfpoint{351.647980pt}{257.532623pt}}
\pgflineto{\pgfpoint{351.688599pt}{257.532623pt}}
\pgfusepath{stroke}
\pgfpathmoveto{\pgfpoint{351.647980pt}{263.709473pt}}
\pgflineto{\pgfpoint{351.693115pt}{263.709473pt}}
\pgfusepath{stroke}
\pgfpathmoveto{\pgfpoint{351.647980pt}{103.111580pt}}
\pgflineto{\pgfpoint{351.647980pt}{96.934731pt}}
\pgfusepath{stroke}
\pgfpathmoveto{\pgfpoint{351.647980pt}{72.227356pt}}
\pgflineto{\pgfpoint{351.647980pt}{66.050522pt}}
\pgfusepath{stroke}
\pgfpathmoveto{\pgfpoint{351.647980pt}{78.404205pt}}
\pgflineto{\pgfpoint{351.647980pt}{72.227356pt}}
\pgfusepath{stroke}
\pgfpathmoveto{\pgfpoint{351.647980pt}{84.581039pt}}
\pgflineto{\pgfpoint{351.647980pt}{78.404205pt}}
\pgfusepath{stroke}
\pgfpathmoveto{\pgfpoint{351.647980pt}{90.757896pt}}
\pgflineto{\pgfpoint{351.647980pt}{84.581039pt}}
\pgfusepath{stroke}
\pgfpathmoveto{\pgfpoint{351.647980pt}{96.934731pt}}
\pgflineto{\pgfpoint{351.647980pt}{90.757896pt}}
\pgfusepath{stroke}
\pgfpathmoveto{\pgfpoint{351.647980pt}{109.288422pt}}
\pgflineto{\pgfpoint{351.647980pt}{103.111580pt}}
\pgfusepath{stroke}
\pgfpathmoveto{\pgfpoint{351.647980pt}{115.465263pt}}
\pgflineto{\pgfpoint{351.647980pt}{109.288422pt}}
\pgfusepath{stroke}
\pgfpathmoveto{\pgfpoint{351.647980pt}{121.642097pt}}
\pgflineto{\pgfpoint{351.647980pt}{115.465263pt}}
\pgfusepath{stroke}
\pgfpathmoveto{\pgfpoint{351.647980pt}{127.818947pt}}
\pgflineto{\pgfpoint{351.647980pt}{121.642097pt}}
\pgfusepath{stroke}
\pgfpathmoveto{\pgfpoint{351.647980pt}{133.995789pt}}
\pgflineto{\pgfpoint{351.647980pt}{127.818947pt}}
\pgfusepath{stroke}
\pgfpathmoveto{\pgfpoint{351.647980pt}{140.172638pt}}
\pgflineto{\pgfpoint{351.647980pt}{133.995789pt}}
\pgfusepath{stroke}
\pgfpathmoveto{\pgfpoint{351.647980pt}{146.349472pt}}
\pgflineto{\pgfpoint{351.647980pt}{140.172638pt}}
\pgfusepath{stroke}
\pgfpathmoveto{\pgfpoint{351.647980pt}{152.526306pt}}
\pgflineto{\pgfpoint{351.647980pt}{146.349472pt}}
\pgfusepath{stroke}
\pgfpathmoveto{\pgfpoint{351.647980pt}{47.519989pt}}
\pgflineto{\pgfpoint{351.629883pt}{47.519989pt}}
\pgfusepath{stroke}
\pgfpathmoveto{\pgfpoint{351.647980pt}{53.696838pt}}
\pgflineto{\pgfpoint{351.629883pt}{53.696838pt}}
\pgfusepath{stroke}
\pgfpathmoveto{\pgfpoint{351.647980pt}{59.873672pt}}
\pgflineto{\pgfpoint{351.629883pt}{59.873672pt}}
\pgfusepath{stroke}
\pgfpathmoveto{\pgfpoint{351.647980pt}{53.696838pt}}
\pgflineto{\pgfpoint{351.647980pt}{47.519989pt}}
\pgfusepath{stroke}
\pgfpathmoveto{\pgfpoint{351.647980pt}{47.519989pt}}
\pgflineto{\pgfpoint{351.666077pt}{47.519989pt}}
\pgfusepath{stroke}
\pgfpathmoveto{\pgfpoint{351.647980pt}{66.050522pt}}
\pgflineto{\pgfpoint{351.647980pt}{59.873672pt}}
\pgfusepath{stroke}
\pgfpathmoveto{\pgfpoint{351.647980pt}{59.873672pt}}
\pgflineto{\pgfpoint{351.647980pt}{53.696838pt}}
\pgfusepath{stroke}
\pgfpathmoveto{\pgfpoint{360.575958pt}{47.519989pt}}
\pgflineto{\pgfpoint{351.666077pt}{47.519989pt}}
\pgfusepath{stroke}
\pgfpathmoveto{\pgfpoint{360.575958pt}{53.696838pt}}
\pgflineto{\pgfpoint{351.647980pt}{53.696838pt}}
\pgfusepath{stroke}
\pgfpathmoveto{\pgfpoint{360.575958pt}{59.873672pt}}
\pgflineto{\pgfpoint{351.647980pt}{59.873672pt}}
\pgfusepath{stroke}
\pgfpathmoveto{\pgfpoint{360.575958pt}{66.050522pt}}
\pgflineto{\pgfpoint{351.647980pt}{66.050522pt}}
\pgfusepath{stroke}
\pgfpathmoveto{\pgfpoint{360.575958pt}{72.227356pt}}
\pgflineto{\pgfpoint{351.647980pt}{72.227356pt}}
\pgfusepath{stroke}
\pgfpathmoveto{\pgfpoint{360.575958pt}{78.404205pt}}
\pgflineto{\pgfpoint{351.647980pt}{78.404205pt}}
\pgfusepath{stroke}
\pgfpathmoveto{\pgfpoint{360.575958pt}{84.581039pt}}
\pgflineto{\pgfpoint{351.647980pt}{84.581039pt}}
\pgfusepath{stroke}
\pgfpathmoveto{\pgfpoint{360.575958pt}{90.757896pt}}
\pgflineto{\pgfpoint{351.647980pt}{90.757896pt}}
\pgfusepath{stroke}
\pgfpathmoveto{\pgfpoint{360.575958pt}{96.934731pt}}
\pgflineto{\pgfpoint{351.647980pt}{96.934731pt}}
\pgfusepath{stroke}
\pgfpathmoveto{\pgfpoint{360.575958pt}{103.111580pt}}
\pgflineto{\pgfpoint{351.647980pt}{103.111580pt}}
\pgfusepath{stroke}
\pgfpathmoveto{\pgfpoint{360.575958pt}{109.288422pt}}
\pgflineto{\pgfpoint{351.647980pt}{109.288422pt}}
\pgfusepath{stroke}
\pgfpathmoveto{\pgfpoint{360.575958pt}{115.465263pt}}
\pgflineto{\pgfpoint{351.647980pt}{115.465263pt}}
\pgfusepath{stroke}
\pgfpathmoveto{\pgfpoint{360.575958pt}{121.642097pt}}
\pgflineto{\pgfpoint{351.647980pt}{121.642097pt}}
\pgfusepath{stroke}
\pgfpathmoveto{\pgfpoint{360.575958pt}{127.818947pt}}
\pgflineto{\pgfpoint{351.647980pt}{127.818947pt}}
\pgfusepath{stroke}
\pgfpathmoveto{\pgfpoint{360.575958pt}{133.995789pt}}
\pgflineto{\pgfpoint{351.647980pt}{133.995789pt}}
\pgfusepath{stroke}
\pgfpathmoveto{\pgfpoint{360.575958pt}{140.172638pt}}
\pgflineto{\pgfpoint{351.647980pt}{140.172638pt}}
\pgfusepath{stroke}
\pgfpathmoveto{\pgfpoint{360.575958pt}{146.349472pt}}
\pgflineto{\pgfpoint{351.647980pt}{146.349472pt}}
\pgfusepath{stroke}
\pgfpathmoveto{\pgfpoint{360.575958pt}{152.526306pt}}
\pgflineto{\pgfpoint{351.647980pt}{152.526306pt}}
\pgfusepath{stroke}
\pgfpathmoveto{\pgfpoint{360.575958pt}{158.703156pt}}
\pgflineto{\pgfpoint{351.661499pt}{158.703156pt}}
\pgfusepath{stroke}
\pgfpathmoveto{\pgfpoint{360.575958pt}{164.880005pt}}
\pgflineto{\pgfpoint{351.661530pt}{164.880005pt}}
\pgfusepath{stroke}
\pgfpathmoveto{\pgfpoint{360.575958pt}{171.056854pt}}
\pgflineto{\pgfpoint{351.661530pt}{171.056854pt}}
\pgfusepath{stroke}
\pgfpathmoveto{\pgfpoint{360.575958pt}{177.233673pt}}
\pgflineto{\pgfpoint{351.666046pt}{177.233673pt}}
\pgfusepath{stroke}
\pgfpathmoveto{\pgfpoint{360.575958pt}{183.410522pt}}
\pgflineto{\pgfpoint{351.666046pt}{183.410522pt}}
\pgfusepath{stroke}
\pgfpathmoveto{\pgfpoint{360.575958pt}{189.587372pt}}
\pgflineto{\pgfpoint{351.670563pt}{189.587372pt}}
\pgfusepath{stroke}
\pgfpathmoveto{\pgfpoint{360.575958pt}{195.764206pt}}
\pgflineto{\pgfpoint{351.670532pt}{195.764206pt}}
\pgfusepath{stroke}
\pgfpathmoveto{\pgfpoint{360.575958pt}{201.941055pt}}
\pgflineto{\pgfpoint{351.670563pt}{201.941055pt}}
\pgfusepath{stroke}
\pgfpathmoveto{\pgfpoint{360.575958pt}{208.117905pt}}
\pgflineto{\pgfpoint{351.675079pt}{208.117905pt}}
\pgfusepath{stroke}
\pgfpathmoveto{\pgfpoint{360.575958pt}{214.294739pt}}
\pgflineto{\pgfpoint{351.675079pt}{214.294739pt}}
\pgfusepath{stroke}
\pgfpathmoveto{\pgfpoint{360.575958pt}{220.471588pt}}
\pgflineto{\pgfpoint{351.679596pt}{220.471588pt}}
\pgfusepath{stroke}
\pgfpathmoveto{\pgfpoint{351.679565pt}{226.648422pt}}
\pgflineto{\pgfpoint{360.562378pt}{226.648422pt}}
\pgfusepath{stroke}
\pgfpathmoveto{\pgfpoint{351.684113pt}{232.825272pt}}
\pgflineto{\pgfpoint{360.562408pt}{232.825272pt}}
\pgfusepath{stroke}
\pgfpathmoveto{\pgfpoint{351.684082pt}{239.002106pt}}
\pgflineto{\pgfpoint{360.557831pt}{239.002106pt}}
\pgfusepath{stroke}
\pgfpathmoveto{\pgfpoint{351.684113pt}{245.178955pt}}
\pgflineto{\pgfpoint{360.553345pt}{245.178955pt}}
\pgfusepath{stroke}
\pgfpathmoveto{\pgfpoint{351.688629pt}{251.355804pt}}
\pgflineto{\pgfpoint{360.553314pt}{251.355804pt}}
\pgfusepath{stroke}
\pgfpathmoveto{\pgfpoint{351.688599pt}{257.532623pt}}
\pgflineto{\pgfpoint{360.548767pt}{257.532623pt}}
\pgfusepath{stroke}
\pgfpathmoveto{\pgfpoint{351.693115pt}{263.709473pt}}
\pgflineto{\pgfpoint{360.548767pt}{263.709473pt}}
\pgfusepath{stroke}
\pgfpathmoveto{\pgfpoint{360.575958pt}{152.526306pt}}
\pgflineto{\pgfpoint{360.575958pt}{146.349472pt}}
\pgfusepath{stroke}
\pgfpathmoveto{\pgfpoint{360.575958pt}{146.349472pt}}
\pgflineto{\pgfpoint{360.575958pt}{140.172638pt}}
\pgfusepath{stroke}
\pgfpathmoveto{\pgfpoint{360.575958pt}{158.703156pt}}
\pgflineto{\pgfpoint{360.575958pt}{152.526306pt}}
\pgfusepath{stroke}
\pgfpathmoveto{\pgfpoint{360.575958pt}{164.880005pt}}
\pgflineto{\pgfpoint{360.575958pt}{158.703156pt}}
\pgfusepath{stroke}
\pgfpathmoveto{\pgfpoint{360.575958pt}{171.056854pt}}
\pgflineto{\pgfpoint{360.575958pt}{164.880005pt}}
\pgfusepath{stroke}
\pgfpathmoveto{\pgfpoint{360.575958pt}{146.349472pt}}
\pgflineto{\pgfpoint{360.589447pt}{146.349472pt}}
\pgfusepath{stroke}
\pgfpathmoveto{\pgfpoint{360.575958pt}{152.526306pt}}
\pgflineto{\pgfpoint{360.589447pt}{152.526306pt}}
\pgfusepath{stroke}
\pgfpathmoveto{\pgfpoint{360.575958pt}{158.703156pt}}
\pgflineto{\pgfpoint{360.593994pt}{158.703156pt}}
\pgfusepath{stroke}
\pgfpathmoveto{\pgfpoint{360.575958pt}{164.880005pt}}
\pgflineto{\pgfpoint{360.593994pt}{164.880005pt}}
\pgfusepath{stroke}
\pgfpathmoveto{\pgfpoint{360.575958pt}{201.941055pt}}
\pgflineto{\pgfpoint{360.575958pt}{195.764206pt}}
\pgfusepath{stroke}
\pgfpathmoveto{\pgfpoint{360.575958pt}{177.233673pt}}
\pgflineto{\pgfpoint{360.575958pt}{171.056854pt}}
\pgfusepath{stroke}
\pgfpathmoveto{\pgfpoint{360.575958pt}{183.410522pt}}
\pgflineto{\pgfpoint{360.575958pt}{177.233673pt}}
\pgfusepath{stroke}
\pgfpathmoveto{\pgfpoint{360.575958pt}{189.587372pt}}
\pgflineto{\pgfpoint{360.575958pt}{183.410522pt}}
\pgfusepath{stroke}
\pgfpathmoveto{\pgfpoint{360.575958pt}{195.764206pt}}
\pgflineto{\pgfpoint{360.575958pt}{189.587372pt}}
\pgfusepath{stroke}
\pgfpathmoveto{\pgfpoint{360.575958pt}{208.117905pt}}
\pgflineto{\pgfpoint{360.575958pt}{201.941055pt}}
\pgfusepath{stroke}
\pgfpathmoveto{\pgfpoint{360.575958pt}{214.294739pt}}
\pgflineto{\pgfpoint{360.575958pt}{208.117905pt}}
\pgfusepath{stroke}
\pgfpathmoveto{\pgfpoint{360.575958pt}{220.471588pt}}
\pgflineto{\pgfpoint{360.575958pt}{214.294739pt}}
\pgfusepath{stroke}
\pgfpathmoveto{\pgfpoint{360.575958pt}{171.056854pt}}
\pgflineto{\pgfpoint{360.593994pt}{171.056854pt}}
\pgfusepath{stroke}
\pgfpathmoveto{\pgfpoint{360.575958pt}{226.648422pt}}
\pgflineto{\pgfpoint{360.562378pt}{226.648422pt}}
\pgfusepath{stroke}
\pgfpathmoveto{\pgfpoint{360.575958pt}{232.825272pt}}
\pgflineto{\pgfpoint{360.562408pt}{232.825272pt}}
\pgfusepath{stroke}
\pgfpathmoveto{\pgfpoint{360.575958pt}{239.002106pt}}
\pgflineto{\pgfpoint{360.557831pt}{239.002106pt}}
\pgfusepath{stroke}
\pgfpathmoveto{\pgfpoint{360.575958pt}{245.178955pt}}
\pgflineto{\pgfpoint{360.553345pt}{245.178955pt}}
\pgfusepath{stroke}
\pgfpathmoveto{\pgfpoint{360.575958pt}{251.355804pt}}
\pgflineto{\pgfpoint{360.553314pt}{251.355804pt}}
\pgfusepath{stroke}
\pgfpathmoveto{\pgfpoint{360.575958pt}{257.532623pt}}
\pgflineto{\pgfpoint{360.548767pt}{257.532623pt}}
\pgfusepath{stroke}
\pgfpathmoveto{\pgfpoint{360.575958pt}{263.709473pt}}
\pgflineto{\pgfpoint{360.548767pt}{263.709473pt}}
\pgfusepath{stroke}
\pgfpathmoveto{\pgfpoint{360.575958pt}{263.709473pt}}
\pgflineto{\pgfpoint{360.575958pt}{257.532623pt}}
\pgfusepath{stroke}
\pgfpathmoveto{\pgfpoint{360.575958pt}{226.648422pt}}
\pgflineto{\pgfpoint{360.575958pt}{220.471588pt}}
\pgfusepath{stroke}
\pgfpathmoveto{\pgfpoint{360.575958pt}{232.825272pt}}
\pgflineto{\pgfpoint{360.575958pt}{226.648422pt}}
\pgfusepath{stroke}
\pgfpathmoveto{\pgfpoint{360.575958pt}{239.002106pt}}
\pgflineto{\pgfpoint{360.575958pt}{232.825272pt}}
\pgfusepath{stroke}
\pgfpathmoveto{\pgfpoint{360.575958pt}{245.178955pt}}
\pgflineto{\pgfpoint{360.575958pt}{239.002106pt}}
\pgfusepath{stroke}
\pgfpathmoveto{\pgfpoint{360.575958pt}{251.355804pt}}
\pgflineto{\pgfpoint{360.575958pt}{245.178955pt}}
\pgfusepath{stroke}
\pgfpathmoveto{\pgfpoint{360.575958pt}{257.532623pt}}
\pgflineto{\pgfpoint{360.575958pt}{251.355804pt}}
\pgfusepath{stroke}
\pgfpathmoveto{\pgfpoint{360.575958pt}{177.233673pt}}
\pgflineto{\pgfpoint{360.598480pt}{177.233673pt}}
\pgfusepath{stroke}
\pgfpathmoveto{\pgfpoint{360.575958pt}{183.410522pt}}
\pgflineto{\pgfpoint{360.598450pt}{183.410522pt}}
\pgfusepath{stroke}
\pgfpathmoveto{\pgfpoint{360.575958pt}{189.587372pt}}
\pgflineto{\pgfpoint{360.602997pt}{189.587372pt}}
\pgfusepath{stroke}
\pgfpathmoveto{\pgfpoint{360.575958pt}{195.764206pt}}
\pgflineto{\pgfpoint{360.602966pt}{195.764206pt}}
\pgfusepath{stroke}
\pgfpathmoveto{\pgfpoint{360.575958pt}{201.941055pt}}
\pgflineto{\pgfpoint{360.607483pt}{201.941055pt}}
\pgfusepath{stroke}
\pgfpathmoveto{\pgfpoint{360.575958pt}{208.117905pt}}
\pgflineto{\pgfpoint{360.607483pt}{208.117905pt}}
\pgfusepath{stroke}
\pgfpathmoveto{\pgfpoint{360.575958pt}{214.294739pt}}
\pgflineto{\pgfpoint{360.607513pt}{214.294739pt}}
\pgfusepath{stroke}
\pgfpathmoveto{\pgfpoint{360.575958pt}{220.471588pt}}
\pgflineto{\pgfpoint{360.612000pt}{220.471588pt}}
\pgfusepath{stroke}
\pgfpathmoveto{\pgfpoint{360.575958pt}{226.648422pt}}
\pgflineto{\pgfpoint{360.611969pt}{226.648422pt}}
\pgfusepath{stroke}
\pgfpathmoveto{\pgfpoint{360.575958pt}{232.825272pt}}
\pgflineto{\pgfpoint{360.616455pt}{232.825272pt}}
\pgfusepath{stroke}
\pgfpathmoveto{\pgfpoint{360.575958pt}{239.002106pt}}
\pgflineto{\pgfpoint{360.616486pt}{239.002106pt}}
\pgfusepath{stroke}
\pgfpathmoveto{\pgfpoint{360.575958pt}{245.178955pt}}
\pgflineto{\pgfpoint{360.616486pt}{245.178955pt}}
\pgfusepath{stroke}
\pgfpathmoveto{\pgfpoint{360.575958pt}{251.355804pt}}
\pgflineto{\pgfpoint{360.621002pt}{251.355804pt}}
\pgfusepath{stroke}
\pgfpathmoveto{\pgfpoint{360.575958pt}{84.581039pt}}
\pgflineto{\pgfpoint{360.575958pt}{78.404205pt}}
\pgfusepath{stroke}
\pgfpathmoveto{\pgfpoint{360.575958pt}{53.696838pt}}
\pgflineto{\pgfpoint{360.575958pt}{47.519989pt}}
\pgfusepath{stroke}
\pgfpathmoveto{\pgfpoint{360.575958pt}{59.873672pt}}
\pgflineto{\pgfpoint{360.575958pt}{53.696838pt}}
\pgfusepath{stroke}
\pgfpathmoveto{\pgfpoint{360.575958pt}{66.050522pt}}
\pgflineto{\pgfpoint{360.575958pt}{59.873672pt}}
\pgfusepath{stroke}
\pgfpathmoveto{\pgfpoint{360.575958pt}{72.227356pt}}
\pgflineto{\pgfpoint{360.575958pt}{66.050522pt}}
\pgfusepath{stroke}
\pgfpathmoveto{\pgfpoint{360.575958pt}{78.404205pt}}
\pgflineto{\pgfpoint{360.575958pt}{72.227356pt}}
\pgfusepath{stroke}
\pgfpathmoveto{\pgfpoint{360.575958pt}{90.757896pt}}
\pgflineto{\pgfpoint{360.575958pt}{84.581039pt}}
\pgfusepath{stroke}
\pgfpathmoveto{\pgfpoint{360.575958pt}{96.934731pt}}
\pgflineto{\pgfpoint{360.575958pt}{90.757896pt}}
\pgfusepath{stroke}
\pgfpathmoveto{\pgfpoint{360.575958pt}{103.111580pt}}
\pgflineto{\pgfpoint{360.575958pt}{96.934731pt}}
\pgfusepath{stroke}
\pgfpathmoveto{\pgfpoint{360.575958pt}{109.288422pt}}
\pgflineto{\pgfpoint{360.575958pt}{103.111580pt}}
\pgfusepath{stroke}
\pgfpathmoveto{\pgfpoint{360.575958pt}{115.465263pt}}
\pgflineto{\pgfpoint{360.575958pt}{109.288422pt}}
\pgfusepath{stroke}
\pgfpathmoveto{\pgfpoint{360.575958pt}{121.642097pt}}
\pgflineto{\pgfpoint{360.575958pt}{115.465263pt}}
\pgfusepath{stroke}
\pgfpathmoveto{\pgfpoint{360.575958pt}{127.818947pt}}
\pgflineto{\pgfpoint{360.575958pt}{121.642097pt}}
\pgfusepath{stroke}
\pgfpathmoveto{\pgfpoint{360.575958pt}{133.995789pt}}
\pgflineto{\pgfpoint{360.575958pt}{127.818947pt}}
\pgfusepath{stroke}
\pgfpathmoveto{\pgfpoint{360.575958pt}{140.172638pt}}
\pgflineto{\pgfpoint{360.575958pt}{133.995789pt}}
\pgfusepath{stroke}
\pgfpathmoveto{\pgfpoint{360.575958pt}{47.519989pt}}
\pgflineto{\pgfpoint{369.485870pt}{47.519989pt}}
\pgfusepath{stroke}
\pgfpathmoveto{\pgfpoint{360.575958pt}{53.696838pt}}
\pgflineto{\pgfpoint{369.485870pt}{53.696838pt}}
\pgfusepath{stroke}
\pgfpathmoveto{\pgfpoint{369.503998pt}{59.873672pt}}
\pgflineto{\pgfpoint{360.575958pt}{59.873672pt}}
\pgfusepath{stroke}
\pgfpathmoveto{\pgfpoint{369.503998pt}{66.050522pt}}
\pgflineto{\pgfpoint{360.575958pt}{66.050522pt}}
\pgfusepath{stroke}
\pgfpathmoveto{\pgfpoint{369.503998pt}{72.227356pt}}
\pgflineto{\pgfpoint{360.575958pt}{72.227356pt}}
\pgfusepath{stroke}
\pgfpathmoveto{\pgfpoint{369.503998pt}{78.404205pt}}
\pgflineto{\pgfpoint{360.575958pt}{78.404205pt}}
\pgfusepath{stroke}
\pgfpathmoveto{\pgfpoint{369.503998pt}{84.581039pt}}
\pgflineto{\pgfpoint{360.575958pt}{84.581039pt}}
\pgfusepath{stroke}
\pgfpathmoveto{\pgfpoint{369.503998pt}{90.757896pt}}
\pgflineto{\pgfpoint{360.575958pt}{90.757896pt}}
\pgfusepath{stroke}
\pgfpathmoveto{\pgfpoint{369.503998pt}{96.934731pt}}
\pgflineto{\pgfpoint{360.575958pt}{96.934731pt}}
\pgfusepath{stroke}
\pgfpathmoveto{\pgfpoint{369.503998pt}{103.111580pt}}
\pgflineto{\pgfpoint{360.575958pt}{103.111580pt}}
\pgfusepath{stroke}
\pgfpathmoveto{\pgfpoint{369.503998pt}{109.288422pt}}
\pgflineto{\pgfpoint{360.575958pt}{109.288422pt}}
\pgfusepath{stroke}
\pgfpathmoveto{\pgfpoint{369.503998pt}{115.465263pt}}
\pgflineto{\pgfpoint{360.575958pt}{115.465263pt}}
\pgfusepath{stroke}
\pgfpathmoveto{\pgfpoint{369.503998pt}{121.642097pt}}
\pgflineto{\pgfpoint{360.575958pt}{121.642097pt}}
\pgfusepath{stroke}
\pgfpathmoveto{\pgfpoint{369.503998pt}{127.818947pt}}
\pgflineto{\pgfpoint{360.575958pt}{127.818947pt}}
\pgfusepath{stroke}
\pgfpathmoveto{\pgfpoint{369.503998pt}{133.995789pt}}
\pgflineto{\pgfpoint{360.575958pt}{133.995789pt}}
\pgfusepath{stroke}
\pgfpathmoveto{\pgfpoint{369.503998pt}{140.172638pt}}
\pgflineto{\pgfpoint{360.575958pt}{140.172638pt}}
\pgfusepath{stroke}
\pgfpathmoveto{\pgfpoint{369.503998pt}{146.349472pt}}
\pgflineto{\pgfpoint{360.589447pt}{146.349472pt}}
\pgfusepath{stroke}
\pgfpathmoveto{\pgfpoint{369.503998pt}{152.526306pt}}
\pgflineto{\pgfpoint{360.589447pt}{152.526306pt}}
\pgfusepath{stroke}
\pgfpathmoveto{\pgfpoint{369.503998pt}{158.703156pt}}
\pgflineto{\pgfpoint{360.593994pt}{158.703156pt}}
\pgfusepath{stroke}
\pgfpathmoveto{\pgfpoint{369.503998pt}{164.880005pt}}
\pgflineto{\pgfpoint{360.593994pt}{164.880005pt}}
\pgfusepath{stroke}
\pgfpathmoveto{\pgfpoint{369.503998pt}{171.056854pt}}
\pgflineto{\pgfpoint{360.593994pt}{171.056854pt}}
\pgfusepath{stroke}
\pgfpathmoveto{\pgfpoint{369.503998pt}{177.233673pt}}
\pgflineto{\pgfpoint{360.598480pt}{177.233673pt}}
\pgfusepath{stroke}
\pgfpathmoveto{\pgfpoint{369.503998pt}{183.410522pt}}
\pgflineto{\pgfpoint{360.598450pt}{183.410522pt}}
\pgfusepath{stroke}
\pgfpathmoveto{\pgfpoint{369.503998pt}{189.587372pt}}
\pgflineto{\pgfpoint{360.602997pt}{189.587372pt}}
\pgfusepath{stroke}
\pgfpathmoveto{\pgfpoint{369.503998pt}{195.764206pt}}
\pgflineto{\pgfpoint{360.602966pt}{195.764206pt}}
\pgfusepath{stroke}
\pgfpathmoveto{\pgfpoint{369.503998pt}{201.941055pt}}
\pgflineto{\pgfpoint{360.607483pt}{201.941055pt}}
\pgfusepath{stroke}
\pgfpathmoveto{\pgfpoint{369.503998pt}{208.117905pt}}
\pgflineto{\pgfpoint{360.607483pt}{208.117905pt}}
\pgfusepath{stroke}
\pgfpathmoveto{\pgfpoint{369.503998pt}{214.294739pt}}
\pgflineto{\pgfpoint{360.607513pt}{214.294739pt}}
\pgfusepath{stroke}
\pgfpathmoveto{\pgfpoint{369.503998pt}{220.471588pt}}
\pgflineto{\pgfpoint{360.612000pt}{220.471588pt}}
\pgfusepath{stroke}
\pgfpathmoveto{\pgfpoint{369.503998pt}{226.648422pt}}
\pgflineto{\pgfpoint{360.611969pt}{226.648422pt}}
\pgfusepath{stroke}
\pgfpathmoveto{\pgfpoint{369.503998pt}{232.825272pt}}
\pgflineto{\pgfpoint{360.616455pt}{232.825272pt}}
\pgfusepath{stroke}
\pgfpathmoveto{\pgfpoint{360.616486pt}{239.002106pt}}
\pgflineto{\pgfpoint{369.490479pt}{239.002106pt}}
\pgfusepath{stroke}
\pgfpathmoveto{\pgfpoint{360.616486pt}{245.178955pt}}
\pgflineto{\pgfpoint{369.490448pt}{245.178955pt}}
\pgfusepath{stroke}
\pgfpathmoveto{\pgfpoint{360.621002pt}{251.355804pt}}
\pgflineto{\pgfpoint{369.485992pt}{251.355804pt}}
\pgfusepath{stroke}
\pgfpathmoveto{\pgfpoint{369.503998pt}{164.880005pt}}
\pgflineto{\pgfpoint{369.503998pt}{158.703156pt}}
\pgfusepath{stroke}
\pgfpathmoveto{\pgfpoint{369.503998pt}{158.703156pt}}
\pgflineto{\pgfpoint{369.503998pt}{152.526306pt}}
\pgfusepath{stroke}
\pgfpathmoveto{\pgfpoint{369.503998pt}{171.056854pt}}
\pgflineto{\pgfpoint{369.503998pt}{164.880005pt}}
\pgfusepath{stroke}
\pgfpathmoveto{\pgfpoint{369.503998pt}{177.233673pt}}
\pgflineto{\pgfpoint{369.503998pt}{171.056854pt}}
\pgfusepath{stroke}
\pgfpathmoveto{\pgfpoint{369.503998pt}{183.410522pt}}
\pgflineto{\pgfpoint{369.503998pt}{177.233673pt}}
\pgfusepath{stroke}
\pgfpathmoveto{\pgfpoint{369.503998pt}{158.703156pt}}
\pgflineto{\pgfpoint{369.517578pt}{158.703156pt}}
\pgfusepath{stroke}
\pgfpathmoveto{\pgfpoint{369.503998pt}{164.880005pt}}
\pgflineto{\pgfpoint{369.517609pt}{164.880005pt}}
\pgfusepath{stroke}
\pgfpathmoveto{\pgfpoint{369.503998pt}{171.056854pt}}
\pgflineto{\pgfpoint{369.522125pt}{171.056854pt}}
\pgfusepath{stroke}
\pgfpathmoveto{\pgfpoint{369.503998pt}{177.233673pt}}
\pgflineto{\pgfpoint{369.522125pt}{177.233673pt}}
\pgfusepath{stroke}
\pgfpathmoveto{\pgfpoint{369.503998pt}{214.294739pt}}
\pgflineto{\pgfpoint{369.503998pt}{208.117905pt}}
\pgfusepath{stroke}
\pgfpathmoveto{\pgfpoint{369.503998pt}{189.587372pt}}
\pgflineto{\pgfpoint{369.503998pt}{183.410522pt}}
\pgfusepath{stroke}
\pgfpathmoveto{\pgfpoint{369.503998pt}{195.764206pt}}
\pgflineto{\pgfpoint{369.503998pt}{189.587372pt}}
\pgfusepath{stroke}
\pgfpathmoveto{\pgfpoint{369.503998pt}{201.941055pt}}
\pgflineto{\pgfpoint{369.503998pt}{195.764206pt}}
\pgfusepath{stroke}
\pgfpathmoveto{\pgfpoint{369.503998pt}{208.117905pt}}
\pgflineto{\pgfpoint{369.503998pt}{201.941055pt}}
\pgfusepath{stroke}
\pgfpathmoveto{\pgfpoint{369.503998pt}{220.471588pt}}
\pgflineto{\pgfpoint{369.503998pt}{214.294739pt}}
\pgfusepath{stroke}
\pgfpathmoveto{\pgfpoint{369.503998pt}{226.648422pt}}
\pgflineto{\pgfpoint{369.503998pt}{220.471588pt}}
\pgfusepath{stroke}
\pgfpathmoveto{\pgfpoint{369.503998pt}{232.825272pt}}
\pgflineto{\pgfpoint{369.503998pt}{226.648422pt}}
\pgfusepath{stroke}
\pgfpathmoveto{\pgfpoint{369.503998pt}{183.410522pt}}
\pgflineto{\pgfpoint{369.522156pt}{183.410522pt}}
\pgfusepath{stroke}
\pgfpathmoveto{\pgfpoint{369.503998pt}{239.002106pt}}
\pgflineto{\pgfpoint{369.490479pt}{239.002106pt}}
\pgfusepath{stroke}
\pgfpathmoveto{\pgfpoint{369.503998pt}{245.178955pt}}
\pgflineto{\pgfpoint{369.490448pt}{245.178955pt}}
\pgfusepath{stroke}
\pgfpathmoveto{\pgfpoint{369.503998pt}{251.355804pt}}
\pgflineto{\pgfpoint{369.485992pt}{251.355804pt}}
\pgfusepath{stroke}
\pgfpathmoveto{\pgfpoint{369.503998pt}{239.002106pt}}
\pgflineto{\pgfpoint{369.503998pt}{232.825272pt}}
\pgfusepath{stroke}
\pgfpathmoveto{\pgfpoint{369.503998pt}{245.178955pt}}
\pgflineto{\pgfpoint{369.503998pt}{239.002106pt}}
\pgfusepath{stroke}
\pgfpathmoveto{\pgfpoint{369.503998pt}{251.355804pt}}
\pgflineto{\pgfpoint{369.503998pt}{245.178955pt}}
\pgfusepath{stroke}
\pgfpathmoveto{\pgfpoint{369.503998pt}{189.587372pt}}
\pgflineto{\pgfpoint{369.526672pt}{189.587372pt}}
\pgfusepath{stroke}
\pgfpathmoveto{\pgfpoint{369.503998pt}{195.764206pt}}
\pgflineto{\pgfpoint{369.526672pt}{195.764206pt}}
\pgfusepath{stroke}
\pgfpathmoveto{\pgfpoint{369.503998pt}{201.941055pt}}
\pgflineto{\pgfpoint{369.531219pt}{201.941055pt}}
\pgfusepath{stroke}
\pgfpathmoveto{\pgfpoint{369.503998pt}{208.117905pt}}
\pgflineto{\pgfpoint{369.531219pt}{208.117905pt}}
\pgfusepath{stroke}
\pgfpathmoveto{\pgfpoint{369.503998pt}{214.294739pt}}
\pgflineto{\pgfpoint{369.535736pt}{214.294739pt}}
\pgfusepath{stroke}
\pgfpathmoveto{\pgfpoint{369.503998pt}{220.471588pt}}
\pgflineto{\pgfpoint{369.535736pt}{220.471588pt}}
\pgfusepath{stroke}
\pgfpathmoveto{\pgfpoint{369.503998pt}{226.648422pt}}
\pgflineto{\pgfpoint{369.535767pt}{226.648422pt}}
\pgfusepath{stroke}
\pgfpathmoveto{\pgfpoint{369.503998pt}{232.825272pt}}
\pgflineto{\pgfpoint{369.540253pt}{232.825272pt}}
\pgfusepath{stroke}
\pgfpathmoveto{\pgfpoint{369.503998pt}{239.002106pt}}
\pgflineto{\pgfpoint{369.540283pt}{239.002106pt}}
\pgfusepath{stroke}
\pgfpathmoveto{\pgfpoint{369.503998pt}{96.934731pt}}
\pgflineto{\pgfpoint{369.503998pt}{90.757896pt}}
\pgfusepath{stroke}
\pgfpathmoveto{\pgfpoint{369.503998pt}{66.050522pt}}
\pgflineto{\pgfpoint{369.503998pt}{59.873672pt}}
\pgfusepath{stroke}
\pgfpathmoveto{\pgfpoint{369.503998pt}{72.227356pt}}
\pgflineto{\pgfpoint{369.503998pt}{66.050522pt}}
\pgfusepath{stroke}
\pgfpathmoveto{\pgfpoint{369.503998pt}{78.404205pt}}
\pgflineto{\pgfpoint{369.503998pt}{72.227356pt}}
\pgfusepath{stroke}
\pgfpathmoveto{\pgfpoint{369.503998pt}{84.581039pt}}
\pgflineto{\pgfpoint{369.503998pt}{78.404205pt}}
\pgfusepath{stroke}
\pgfpathmoveto{\pgfpoint{369.503998pt}{90.757896pt}}
\pgflineto{\pgfpoint{369.503998pt}{84.581039pt}}
\pgfusepath{stroke}
\pgfpathmoveto{\pgfpoint{369.503998pt}{103.111580pt}}
\pgflineto{\pgfpoint{369.503998pt}{96.934731pt}}
\pgfusepath{stroke}
\pgfpathmoveto{\pgfpoint{369.503998pt}{109.288422pt}}
\pgflineto{\pgfpoint{369.503998pt}{103.111580pt}}
\pgfusepath{stroke}
\pgfpathmoveto{\pgfpoint{369.503998pt}{115.465263pt}}
\pgflineto{\pgfpoint{369.503998pt}{109.288422pt}}
\pgfusepath{stroke}
\pgfpathmoveto{\pgfpoint{369.503998pt}{121.642097pt}}
\pgflineto{\pgfpoint{369.503998pt}{115.465263pt}}
\pgfusepath{stroke}
\pgfpathmoveto{\pgfpoint{369.503998pt}{127.818947pt}}
\pgflineto{\pgfpoint{369.503998pt}{121.642097pt}}
\pgfusepath{stroke}
\pgfpathmoveto{\pgfpoint{369.503998pt}{133.995789pt}}
\pgflineto{\pgfpoint{369.503998pt}{127.818947pt}}
\pgfusepath{stroke}
\pgfpathmoveto{\pgfpoint{369.503998pt}{140.172638pt}}
\pgflineto{\pgfpoint{369.503998pt}{133.995789pt}}
\pgfusepath{stroke}
\pgfpathmoveto{\pgfpoint{369.503998pt}{146.349472pt}}
\pgflineto{\pgfpoint{369.503998pt}{140.172638pt}}
\pgfusepath{stroke}
\pgfpathmoveto{\pgfpoint{369.503998pt}{152.526306pt}}
\pgflineto{\pgfpoint{369.503998pt}{146.349472pt}}
\pgfusepath{stroke}
\pgfpathmoveto{\pgfpoint{369.503998pt}{47.519989pt}}
\pgflineto{\pgfpoint{369.485870pt}{47.519989pt}}
\pgfusepath{stroke}
\pgfpathmoveto{\pgfpoint{369.503998pt}{53.696838pt}}
\pgflineto{\pgfpoint{369.485870pt}{53.696838pt}}
\pgfusepath{stroke}
\pgfpathmoveto{\pgfpoint{369.503998pt}{59.873672pt}}
\pgflineto{\pgfpoint{369.503998pt}{53.696838pt}}
\pgfusepath{stroke}
\pgfpathmoveto{\pgfpoint{369.503998pt}{53.696838pt}}
\pgflineto{\pgfpoint{369.503998pt}{47.519989pt}}
\pgfusepath{stroke}
\pgfpathmoveto{\pgfpoint{378.431976pt}{47.519989pt}}
\pgflineto{\pgfpoint{369.503998pt}{47.519989pt}}
\pgfusepath{stroke}
\pgfpathmoveto{\pgfpoint{378.431976pt}{53.696838pt}}
\pgflineto{\pgfpoint{369.503998pt}{53.696838pt}}
\pgfusepath{stroke}
\pgfpathmoveto{\pgfpoint{378.431976pt}{59.873672pt}}
\pgflineto{\pgfpoint{369.503998pt}{59.873672pt}}
\pgfusepath{stroke}
\pgfpathmoveto{\pgfpoint{378.431976pt}{66.050522pt}}
\pgflineto{\pgfpoint{369.503998pt}{66.050522pt}}
\pgfusepath{stroke}
\pgfpathmoveto{\pgfpoint{378.431976pt}{72.227356pt}}
\pgflineto{\pgfpoint{369.503998pt}{72.227356pt}}
\pgfusepath{stroke}
\pgfpathmoveto{\pgfpoint{378.431976pt}{78.404205pt}}
\pgflineto{\pgfpoint{369.503998pt}{78.404205pt}}
\pgfusepath{stroke}
\pgfpathmoveto{\pgfpoint{378.431976pt}{84.581039pt}}
\pgflineto{\pgfpoint{369.503998pt}{84.581039pt}}
\pgfusepath{stroke}
\pgfpathmoveto{\pgfpoint{378.431976pt}{90.757896pt}}
\pgflineto{\pgfpoint{369.503998pt}{90.757896pt}}
\pgfusepath{stroke}
\pgfpathmoveto{\pgfpoint{378.431976pt}{96.934731pt}}
\pgflineto{\pgfpoint{369.503998pt}{96.934731pt}}
\pgfusepath{stroke}
\pgfpathmoveto{\pgfpoint{378.431976pt}{103.111580pt}}
\pgflineto{\pgfpoint{369.503998pt}{103.111580pt}}
\pgfusepath{stroke}
\pgfpathmoveto{\pgfpoint{378.431976pt}{109.288422pt}}
\pgflineto{\pgfpoint{369.503998pt}{109.288422pt}}
\pgfusepath{stroke}
\pgfpathmoveto{\pgfpoint{378.431976pt}{115.465263pt}}
\pgflineto{\pgfpoint{369.503998pt}{115.465263pt}}
\pgfusepath{stroke}
\pgfpathmoveto{\pgfpoint{378.431976pt}{121.642097pt}}
\pgflineto{\pgfpoint{369.503998pt}{121.642097pt}}
\pgfusepath{stroke}
\pgfpathmoveto{\pgfpoint{378.431976pt}{127.818947pt}}
\pgflineto{\pgfpoint{369.503998pt}{127.818947pt}}
\pgfusepath{stroke}
\pgfpathmoveto{\pgfpoint{378.431976pt}{133.995789pt}}
\pgflineto{\pgfpoint{369.503998pt}{133.995789pt}}
\pgfusepath{stroke}
\pgfpathmoveto{\pgfpoint{378.431976pt}{140.172638pt}}
\pgflineto{\pgfpoint{369.503998pt}{140.172638pt}}
\pgfusepath{stroke}
\pgfpathmoveto{\pgfpoint{378.431976pt}{146.349472pt}}
\pgflineto{\pgfpoint{369.503998pt}{146.349472pt}}
\pgfusepath{stroke}
\pgfpathmoveto{\pgfpoint{378.431976pt}{152.526306pt}}
\pgflineto{\pgfpoint{369.503998pt}{152.526306pt}}
\pgfusepath{stroke}
\pgfpathmoveto{\pgfpoint{378.431976pt}{158.703156pt}}
\pgflineto{\pgfpoint{369.517578pt}{158.703156pt}}
\pgfusepath{stroke}
\pgfpathmoveto{\pgfpoint{378.431976pt}{164.880005pt}}
\pgflineto{\pgfpoint{369.517609pt}{164.880005pt}}
\pgfusepath{stroke}
\pgfpathmoveto{\pgfpoint{378.431976pt}{171.056854pt}}
\pgflineto{\pgfpoint{369.522125pt}{171.056854pt}}
\pgfusepath{stroke}
\pgfpathmoveto{\pgfpoint{378.431976pt}{177.233673pt}}
\pgflineto{\pgfpoint{369.522125pt}{177.233673pt}}
\pgfusepath{stroke}
\pgfpathmoveto{\pgfpoint{378.431976pt}{183.410522pt}}
\pgflineto{\pgfpoint{369.522156pt}{183.410522pt}}
\pgfusepath{stroke}
\pgfpathmoveto{\pgfpoint{378.431976pt}{189.587372pt}}
\pgflineto{\pgfpoint{369.526672pt}{189.587372pt}}
\pgfusepath{stroke}
\pgfpathmoveto{\pgfpoint{378.431976pt}{195.764206pt}}
\pgflineto{\pgfpoint{369.526672pt}{195.764206pt}}
\pgfusepath{stroke}
\pgfpathmoveto{\pgfpoint{378.431976pt}{201.941055pt}}
\pgflineto{\pgfpoint{369.531219pt}{201.941055pt}}
\pgfusepath{stroke}
\pgfpathmoveto{\pgfpoint{369.531219pt}{208.117905pt}}
\pgflineto{\pgfpoint{378.418396pt}{208.117905pt}}
\pgfusepath{stroke}
\pgfpathmoveto{\pgfpoint{369.535736pt}{214.294739pt}}
\pgflineto{\pgfpoint{378.418396pt}{214.294739pt}}
\pgfusepath{stroke}
\pgfpathmoveto{\pgfpoint{369.535736pt}{220.471588pt}}
\pgflineto{\pgfpoint{378.413849pt}{220.471588pt}}
\pgfusepath{stroke}
\pgfpathmoveto{\pgfpoint{369.535767pt}{226.648422pt}}
\pgflineto{\pgfpoint{378.409332pt}{226.648422pt}}
\pgfusepath{stroke}
\pgfpathmoveto{\pgfpoint{369.540253pt}{232.825272pt}}
\pgflineto{\pgfpoint{378.409332pt}{232.825272pt}}
\pgfusepath{stroke}
\pgfpathmoveto{\pgfpoint{369.540283pt}{239.002106pt}}
\pgflineto{\pgfpoint{378.404785pt}{239.002106pt}}
\pgfusepath{stroke}
\pgfpathmoveto{\pgfpoint{378.431976pt}{146.349472pt}}
\pgflineto{\pgfpoint{378.431976pt}{140.172638pt}}
\pgfusepath{stroke}
\pgfpathmoveto{\pgfpoint{378.431976pt}{140.172638pt}}
\pgflineto{\pgfpoint{378.431976pt}{133.995789pt}}
\pgfusepath{stroke}
\pgfpathmoveto{\pgfpoint{378.431976pt}{152.526306pt}}
\pgflineto{\pgfpoint{378.431976pt}{146.349472pt}}
\pgfusepath{stroke}
\pgfpathmoveto{\pgfpoint{378.431976pt}{140.172638pt}}
\pgflineto{\pgfpoint{378.445618pt}{140.172638pt}}
\pgfusepath{stroke}
\pgfpathmoveto{\pgfpoint{378.431976pt}{146.349472pt}}
\pgflineto{\pgfpoint{378.445618pt}{146.349472pt}}
\pgfusepath{stroke}
\pgfpathmoveto{\pgfpoint{378.431976pt}{183.410522pt}}
\pgflineto{\pgfpoint{378.431976pt}{177.233673pt}}
\pgfusepath{stroke}
\pgfpathmoveto{\pgfpoint{378.431976pt}{158.703156pt}}
\pgflineto{\pgfpoint{378.431976pt}{152.526306pt}}
\pgfusepath{stroke}
\pgfpathmoveto{\pgfpoint{378.431976pt}{164.880005pt}}
\pgflineto{\pgfpoint{378.431976pt}{158.703156pt}}
\pgfusepath{stroke}
\pgfpathmoveto{\pgfpoint{378.431976pt}{171.056854pt}}
\pgflineto{\pgfpoint{378.431976pt}{164.880005pt}}
\pgfusepath{stroke}
\pgfpathmoveto{\pgfpoint{378.431976pt}{177.233673pt}}
\pgflineto{\pgfpoint{378.431976pt}{171.056854pt}}
\pgfusepath{stroke}
\pgfpathmoveto{\pgfpoint{378.431976pt}{189.587372pt}}
\pgflineto{\pgfpoint{378.431976pt}{183.410522pt}}
\pgfusepath{stroke}
\pgfpathmoveto{\pgfpoint{378.431976pt}{195.764206pt}}
\pgflineto{\pgfpoint{378.431976pt}{189.587372pt}}
\pgfusepath{stroke}
\pgfpathmoveto{\pgfpoint{378.431976pt}{201.941055pt}}
\pgflineto{\pgfpoint{378.431976pt}{195.764206pt}}
\pgfusepath{stroke}
\pgfpathmoveto{\pgfpoint{378.431976pt}{152.526306pt}}
\pgflineto{\pgfpoint{378.445587pt}{152.526306pt}}
\pgfusepath{stroke}
\pgfpathmoveto{\pgfpoint{378.431976pt}{158.703156pt}}
\pgflineto{\pgfpoint{378.450134pt}{158.703156pt}}
\pgfusepath{stroke}
\pgfpathmoveto{\pgfpoint{378.431976pt}{208.117905pt}}
\pgflineto{\pgfpoint{378.418396pt}{208.117905pt}}
\pgfusepath{stroke}
\pgfpathmoveto{\pgfpoint{378.431976pt}{214.294739pt}}
\pgflineto{\pgfpoint{378.418396pt}{214.294739pt}}
\pgfusepath{stroke}
\pgfpathmoveto{\pgfpoint{378.431976pt}{220.471588pt}}
\pgflineto{\pgfpoint{378.413849pt}{220.471588pt}}
\pgfusepath{stroke}
\pgfpathmoveto{\pgfpoint{378.431976pt}{226.648422pt}}
\pgflineto{\pgfpoint{378.409332pt}{226.648422pt}}
\pgfusepath{stroke}
\pgfpathmoveto{\pgfpoint{378.431976pt}{232.825272pt}}
\pgflineto{\pgfpoint{378.409332pt}{232.825272pt}}
\pgfusepath{stroke}
\pgfpathmoveto{\pgfpoint{378.431976pt}{239.002106pt}}
\pgflineto{\pgfpoint{378.404785pt}{239.002106pt}}
\pgfusepath{stroke}
\pgfpathmoveto{\pgfpoint{378.431976pt}{208.117905pt}}
\pgflineto{\pgfpoint{378.431976pt}{201.941055pt}}
\pgfusepath{stroke}
\pgfpathmoveto{\pgfpoint{378.431976pt}{214.294739pt}}
\pgflineto{\pgfpoint{378.431976pt}{208.117905pt}}
\pgfusepath{stroke}
\pgfpathmoveto{\pgfpoint{378.431976pt}{220.471588pt}}
\pgflineto{\pgfpoint{378.431976pt}{214.294739pt}}
\pgfusepath{stroke}
\pgfpathmoveto{\pgfpoint{378.431976pt}{226.648422pt}}
\pgflineto{\pgfpoint{378.431976pt}{220.471588pt}}
\pgfusepath{stroke}
\pgfpathmoveto{\pgfpoint{378.431976pt}{232.825272pt}}
\pgflineto{\pgfpoint{378.431976pt}{226.648422pt}}
\pgfusepath{stroke}
\pgfpathmoveto{\pgfpoint{378.431976pt}{239.002106pt}}
\pgflineto{\pgfpoint{378.431976pt}{232.825272pt}}
\pgfusepath{stroke}
\pgfpathmoveto{\pgfpoint{378.431976pt}{164.880005pt}}
\pgflineto{\pgfpoint{378.450134pt}{164.880005pt}}
\pgfusepath{stroke}
\pgfpathmoveto{\pgfpoint{378.431976pt}{171.056854pt}}
\pgflineto{\pgfpoint{378.454651pt}{171.056854pt}}
\pgfusepath{stroke}
\pgfpathmoveto{\pgfpoint{378.431976pt}{177.233673pt}}
\pgflineto{\pgfpoint{378.454651pt}{177.233673pt}}
\pgfusepath{stroke}
\pgfpathmoveto{\pgfpoint{378.431976pt}{183.410522pt}}
\pgflineto{\pgfpoint{378.459198pt}{183.410522pt}}
\pgfusepath{stroke}
\pgfpathmoveto{\pgfpoint{378.431976pt}{189.587372pt}}
\pgflineto{\pgfpoint{378.459229pt}{189.587372pt}}
\pgfusepath{stroke}
\pgfpathmoveto{\pgfpoint{378.431976pt}{195.764206pt}}
\pgflineto{\pgfpoint{378.459229pt}{195.764206pt}}
\pgfusepath{stroke}
\pgfpathmoveto{\pgfpoint{378.431976pt}{201.941055pt}}
\pgflineto{\pgfpoint{378.463745pt}{201.941055pt}}
\pgfusepath{stroke}
\pgfpathmoveto{\pgfpoint{378.431976pt}{208.117905pt}}
\pgflineto{\pgfpoint{378.463745pt}{208.117905pt}}
\pgfusepath{stroke}
\pgfpathmoveto{\pgfpoint{378.431976pt}{214.294739pt}}
\pgflineto{\pgfpoint{378.468262pt}{214.294739pt}}
\pgfusepath{stroke}
\pgfpathmoveto{\pgfpoint{378.431976pt}{220.471588pt}}
\pgflineto{\pgfpoint{378.468262pt}{220.471588pt}}
\pgfusepath{stroke}
\pgfpathmoveto{\pgfpoint{378.431976pt}{226.648422pt}}
\pgflineto{\pgfpoint{378.468292pt}{226.648422pt}}
\pgfusepath{stroke}
\pgfpathmoveto{\pgfpoint{378.431976pt}{84.581039pt}}
\pgflineto{\pgfpoint{378.431976pt}{78.404205pt}}
\pgfusepath{stroke}
\pgfpathmoveto{\pgfpoint{378.431976pt}{53.696838pt}}
\pgflineto{\pgfpoint{378.431976pt}{47.519989pt}}
\pgfusepath{stroke}
\pgfpathmoveto{\pgfpoint{378.431976pt}{59.873672pt}}
\pgflineto{\pgfpoint{378.431976pt}{53.696838pt}}
\pgfusepath{stroke}
\pgfpathmoveto{\pgfpoint{378.431976pt}{66.050522pt}}
\pgflineto{\pgfpoint{378.431976pt}{59.873672pt}}
\pgfusepath{stroke}
\pgfpathmoveto{\pgfpoint{378.431976pt}{72.227356pt}}
\pgflineto{\pgfpoint{378.431976pt}{66.050522pt}}
\pgfusepath{stroke}
\pgfpathmoveto{\pgfpoint{378.431976pt}{78.404205pt}}
\pgflineto{\pgfpoint{378.431976pt}{72.227356pt}}
\pgfusepath{stroke}
\pgfpathmoveto{\pgfpoint{378.431976pt}{90.757896pt}}
\pgflineto{\pgfpoint{378.431976pt}{84.581039pt}}
\pgfusepath{stroke}
\pgfpathmoveto{\pgfpoint{378.431976pt}{96.934731pt}}
\pgflineto{\pgfpoint{378.431976pt}{90.757896pt}}
\pgfusepath{stroke}
\pgfpathmoveto{\pgfpoint{378.431976pt}{103.111580pt}}
\pgflineto{\pgfpoint{378.431976pt}{96.934731pt}}
\pgfusepath{stroke}
\pgfpathmoveto{\pgfpoint{378.431976pt}{109.288422pt}}
\pgflineto{\pgfpoint{378.431976pt}{103.111580pt}}
\pgfusepath{stroke}
\pgfpathmoveto{\pgfpoint{378.431976pt}{115.465263pt}}
\pgflineto{\pgfpoint{378.431976pt}{109.288422pt}}
\pgfusepath{stroke}
\pgfpathmoveto{\pgfpoint{378.431976pt}{121.642097pt}}
\pgflineto{\pgfpoint{378.431976pt}{115.465263pt}}
\pgfusepath{stroke}
\pgfpathmoveto{\pgfpoint{378.431976pt}{127.818947pt}}
\pgflineto{\pgfpoint{378.431976pt}{121.642097pt}}
\pgfusepath{stroke}
\pgfpathmoveto{\pgfpoint{378.431976pt}{133.995789pt}}
\pgflineto{\pgfpoint{378.431976pt}{127.818947pt}}
\pgfusepath{stroke}
\pgfpathmoveto{\pgfpoint{378.431976pt}{47.519989pt}}
\pgflineto{\pgfpoint{387.341858pt}{47.519989pt}}
\pgfusepath{stroke}
\pgfpathmoveto{\pgfpoint{378.431976pt}{53.696838pt}}
\pgflineto{\pgfpoint{387.341888pt}{53.696838pt}}
\pgfusepath{stroke}
\pgfpathmoveto{\pgfpoint{387.359985pt}{59.873672pt}}
\pgflineto{\pgfpoint{378.431976pt}{59.873672pt}}
\pgfusepath{stroke}
\pgfpathmoveto{\pgfpoint{387.359985pt}{66.050522pt}}
\pgflineto{\pgfpoint{378.431976pt}{66.050522pt}}
\pgfusepath{stroke}
\pgfpathmoveto{\pgfpoint{387.359985pt}{72.227356pt}}
\pgflineto{\pgfpoint{378.431976pt}{72.227356pt}}
\pgfusepath{stroke}
\pgfpathmoveto{\pgfpoint{387.359985pt}{78.404205pt}}
\pgflineto{\pgfpoint{378.431976pt}{78.404205pt}}
\pgfusepath{stroke}
\pgfpathmoveto{\pgfpoint{387.359985pt}{84.581039pt}}
\pgflineto{\pgfpoint{378.431976pt}{84.581039pt}}
\pgfusepath{stroke}
\pgfpathmoveto{\pgfpoint{387.359985pt}{90.757896pt}}
\pgflineto{\pgfpoint{378.431976pt}{90.757896pt}}
\pgfusepath{stroke}
\pgfpathmoveto{\pgfpoint{387.359985pt}{96.934731pt}}
\pgflineto{\pgfpoint{378.431976pt}{96.934731pt}}
\pgfusepath{stroke}
\pgfpathmoveto{\pgfpoint{387.359985pt}{103.111580pt}}
\pgflineto{\pgfpoint{378.431976pt}{103.111580pt}}
\pgfusepath{stroke}
\pgfpathmoveto{\pgfpoint{387.359985pt}{109.288422pt}}
\pgflineto{\pgfpoint{378.431976pt}{109.288422pt}}
\pgfusepath{stroke}
\pgfpathmoveto{\pgfpoint{387.359985pt}{115.465263pt}}
\pgflineto{\pgfpoint{378.431976pt}{115.465263pt}}
\pgfusepath{stroke}
\pgfpathmoveto{\pgfpoint{387.359985pt}{121.642097pt}}
\pgflineto{\pgfpoint{378.431976pt}{121.642097pt}}
\pgfusepath{stroke}
\pgfpathmoveto{\pgfpoint{387.359985pt}{127.818947pt}}
\pgflineto{\pgfpoint{378.431976pt}{127.818947pt}}
\pgfusepath{stroke}
\pgfpathmoveto{\pgfpoint{387.359985pt}{133.995789pt}}
\pgflineto{\pgfpoint{378.431976pt}{133.995789pt}}
\pgfusepath{stroke}
\pgfpathmoveto{\pgfpoint{387.359985pt}{140.172638pt}}
\pgflineto{\pgfpoint{378.445618pt}{140.172638pt}}
\pgfusepath{stroke}
\pgfpathmoveto{\pgfpoint{387.359985pt}{146.349472pt}}
\pgflineto{\pgfpoint{378.445618pt}{146.349472pt}}
\pgfusepath{stroke}
\pgfpathmoveto{\pgfpoint{387.359985pt}{152.526306pt}}
\pgflineto{\pgfpoint{378.445587pt}{152.526306pt}}
\pgfusepath{stroke}
\pgfpathmoveto{\pgfpoint{387.359985pt}{158.703156pt}}
\pgflineto{\pgfpoint{378.450134pt}{158.703156pt}}
\pgfusepath{stroke}
\pgfpathmoveto{\pgfpoint{387.359985pt}{164.880005pt}}
\pgflineto{\pgfpoint{378.450134pt}{164.880005pt}}
\pgfusepath{stroke}
\pgfpathmoveto{\pgfpoint{387.359985pt}{171.056854pt}}
\pgflineto{\pgfpoint{378.454651pt}{171.056854pt}}
\pgfusepath{stroke}
\pgfpathmoveto{\pgfpoint{387.359985pt}{177.233673pt}}
\pgflineto{\pgfpoint{378.454651pt}{177.233673pt}}
\pgfusepath{stroke}
\pgfpathmoveto{\pgfpoint{387.359985pt}{183.410522pt}}
\pgflineto{\pgfpoint{378.459198pt}{183.410522pt}}
\pgfusepath{stroke}
\pgfpathmoveto{\pgfpoint{387.359985pt}{189.587372pt}}
\pgflineto{\pgfpoint{378.459229pt}{189.587372pt}}
\pgfusepath{stroke}
\pgfpathmoveto{\pgfpoint{387.359985pt}{195.764206pt}}
\pgflineto{\pgfpoint{378.459229pt}{195.764206pt}}
\pgfusepath{stroke}
\pgfpathmoveto{\pgfpoint{387.359985pt}{201.941055pt}}
\pgflineto{\pgfpoint{378.463745pt}{201.941055pt}}
\pgfusepath{stroke}
\pgfpathmoveto{\pgfpoint{387.359985pt}{208.117905pt}}
\pgflineto{\pgfpoint{378.463745pt}{208.117905pt}}
\pgfusepath{stroke}
\pgfpathmoveto{\pgfpoint{387.359985pt}{214.294739pt}}
\pgflineto{\pgfpoint{378.468262pt}{214.294739pt}}
\pgfusepath{stroke}
\pgfpathmoveto{\pgfpoint{387.359985pt}{220.471588pt}}
\pgflineto{\pgfpoint{378.468262pt}{220.471588pt}}
\pgfusepath{stroke}
\pgfpathmoveto{\pgfpoint{387.359985pt}{226.648422pt}}
\pgflineto{\pgfpoint{378.468292pt}{226.648422pt}}
\pgfusepath{stroke}
\pgfpathmoveto{\pgfpoint{387.359985pt}{158.703156pt}}
\pgflineto{\pgfpoint{387.359985pt}{152.526306pt}}
\pgfusepath{stroke}
\pgfpathmoveto{\pgfpoint{387.359985pt}{152.526306pt}}
\pgflineto{\pgfpoint{387.359985pt}{146.349472pt}}
\pgfusepath{stroke}
\pgfpathmoveto{\pgfpoint{387.359985pt}{164.880005pt}}
\pgflineto{\pgfpoint{387.359985pt}{158.703156pt}}
\pgfusepath{stroke}
\pgfpathmoveto{\pgfpoint{387.359985pt}{171.056854pt}}
\pgflineto{\pgfpoint{387.359985pt}{164.880005pt}}
\pgfusepath{stroke}
\pgfpathmoveto{\pgfpoint{387.359985pt}{177.233673pt}}
\pgflineto{\pgfpoint{387.359985pt}{171.056854pt}}
\pgfusepath{stroke}
\pgfpathmoveto{\pgfpoint{387.359985pt}{152.526306pt}}
\pgflineto{\pgfpoint{387.373566pt}{152.526306pt}}
\pgfusepath{stroke}
\pgfpathmoveto{\pgfpoint{387.359985pt}{158.703156pt}}
\pgflineto{\pgfpoint{387.373596pt}{158.703156pt}}
\pgfusepath{stroke}
\pgfpathmoveto{\pgfpoint{387.359985pt}{164.880005pt}}
\pgflineto{\pgfpoint{387.373627pt}{164.880005pt}}
\pgfusepath{stroke}
\pgfpathmoveto{\pgfpoint{387.359985pt}{171.056854pt}}
\pgflineto{\pgfpoint{387.378143pt}{171.056854pt}}
\pgfusepath{stroke}
\pgfpathmoveto{\pgfpoint{387.359985pt}{201.941055pt}}
\pgflineto{\pgfpoint{387.359985pt}{195.764206pt}}
\pgfusepath{stroke}
\pgfpathmoveto{\pgfpoint{387.359985pt}{183.410522pt}}
\pgflineto{\pgfpoint{387.359985pt}{177.233673pt}}
\pgfusepath{stroke}
\pgfpathmoveto{\pgfpoint{387.359985pt}{189.587372pt}}
\pgflineto{\pgfpoint{387.359985pt}{183.410522pt}}
\pgfusepath{stroke}
\pgfpathmoveto{\pgfpoint{387.359985pt}{195.764206pt}}
\pgflineto{\pgfpoint{387.359985pt}{189.587372pt}}
\pgfusepath{stroke}
\pgfpathmoveto{\pgfpoint{387.359985pt}{208.117905pt}}
\pgflineto{\pgfpoint{387.359985pt}{201.941055pt}}
\pgfusepath{stroke}
\pgfpathmoveto{\pgfpoint{387.359985pt}{214.294739pt}}
\pgflineto{\pgfpoint{387.359985pt}{208.117905pt}}
\pgfusepath{stroke}
\pgfpathmoveto{\pgfpoint{387.359985pt}{220.471588pt}}
\pgflineto{\pgfpoint{387.359985pt}{214.294739pt}}
\pgfusepath{stroke}
\pgfpathmoveto{\pgfpoint{387.359985pt}{226.648422pt}}
\pgflineto{\pgfpoint{387.359985pt}{220.471588pt}}
\pgfusepath{stroke}
\pgfpathmoveto{\pgfpoint{387.359985pt}{177.233673pt}}
\pgflineto{\pgfpoint{387.378113pt}{177.233673pt}}
\pgfusepath{stroke}
\pgfpathmoveto{\pgfpoint{387.359985pt}{183.410522pt}}
\pgflineto{\pgfpoint{387.382660pt}{183.410522pt}}
\pgfusepath{stroke}
\pgfpathmoveto{\pgfpoint{387.359985pt}{189.587372pt}}
\pgflineto{\pgfpoint{387.382660pt}{189.587372pt}}
\pgfusepath{stroke}
\pgfpathmoveto{\pgfpoint{387.359985pt}{195.764206pt}}
\pgflineto{\pgfpoint{387.382690pt}{195.764206pt}}
\pgfusepath{stroke}
\pgfpathmoveto{\pgfpoint{387.359985pt}{201.941055pt}}
\pgflineto{\pgfpoint{387.387207pt}{201.941055pt}}
\pgfusepath{stroke}
\pgfpathmoveto{\pgfpoint{387.359985pt}{208.117905pt}}
\pgflineto{\pgfpoint{387.387238pt}{208.117905pt}}
\pgfusepath{stroke}
\pgfpathmoveto{\pgfpoint{387.359985pt}{214.294739pt}}
\pgflineto{\pgfpoint{387.391724pt}{214.294739pt}}
\pgfusepath{stroke}
\pgfpathmoveto{\pgfpoint{387.359985pt}{96.934731pt}}
\pgflineto{\pgfpoint{387.359985pt}{90.757896pt}}
\pgfusepath{stroke}
\pgfpathmoveto{\pgfpoint{387.359985pt}{66.050522pt}}
\pgflineto{\pgfpoint{387.359985pt}{59.873672pt}}
\pgfusepath{stroke}
\pgfpathmoveto{\pgfpoint{387.359985pt}{72.227356pt}}
\pgflineto{\pgfpoint{387.359985pt}{66.050522pt}}
\pgfusepath{stroke}
\pgfpathmoveto{\pgfpoint{387.359985pt}{78.404205pt}}
\pgflineto{\pgfpoint{387.359985pt}{72.227356pt}}
\pgfusepath{stroke}
\pgfpathmoveto{\pgfpoint{387.359985pt}{84.581039pt}}
\pgflineto{\pgfpoint{387.359985pt}{78.404205pt}}
\pgfusepath{stroke}
\pgfpathmoveto{\pgfpoint{387.359985pt}{90.757896pt}}
\pgflineto{\pgfpoint{387.359985pt}{84.581039pt}}
\pgfusepath{stroke}
\pgfpathmoveto{\pgfpoint{387.359985pt}{103.111580pt}}
\pgflineto{\pgfpoint{387.359985pt}{96.934731pt}}
\pgfusepath{stroke}
\pgfpathmoveto{\pgfpoint{387.359985pt}{109.288422pt}}
\pgflineto{\pgfpoint{387.359985pt}{103.111580pt}}
\pgfusepath{stroke}
\pgfpathmoveto{\pgfpoint{387.359985pt}{115.465263pt}}
\pgflineto{\pgfpoint{387.359985pt}{109.288422pt}}
\pgfusepath{stroke}
\pgfpathmoveto{\pgfpoint{387.359985pt}{121.642097pt}}
\pgflineto{\pgfpoint{387.359985pt}{115.465263pt}}
\pgfusepath{stroke}
\pgfpathmoveto{\pgfpoint{387.359985pt}{127.818947pt}}
\pgflineto{\pgfpoint{387.359985pt}{121.642097pt}}
\pgfusepath{stroke}
\pgfpathmoveto{\pgfpoint{387.359985pt}{133.995789pt}}
\pgflineto{\pgfpoint{387.359985pt}{127.818947pt}}
\pgfusepath{stroke}
\pgfpathmoveto{\pgfpoint{387.359985pt}{140.172638pt}}
\pgflineto{\pgfpoint{387.359985pt}{133.995789pt}}
\pgfusepath{stroke}
\pgfpathmoveto{\pgfpoint{387.359985pt}{146.349472pt}}
\pgflineto{\pgfpoint{387.359985pt}{140.172638pt}}
\pgfusepath{stroke}
\pgfpathmoveto{\pgfpoint{387.359985pt}{47.519989pt}}
\pgflineto{\pgfpoint{387.341858pt}{47.519989pt}}
\pgfusepath{stroke}
\pgfpathmoveto{\pgfpoint{387.359985pt}{53.696838pt}}
\pgflineto{\pgfpoint{387.341888pt}{53.696838pt}}
\pgfusepath{stroke}
\pgfpathmoveto{\pgfpoint{387.359985pt}{59.873672pt}}
\pgflineto{\pgfpoint{387.359985pt}{53.696838pt}}
\pgfusepath{stroke}
\pgfpathmoveto{\pgfpoint{387.359985pt}{53.696838pt}}
\pgflineto{\pgfpoint{387.359985pt}{47.519989pt}}
\pgfusepath{stroke}
\pgfpathmoveto{\pgfpoint{396.287964pt}{47.519989pt}}
\pgflineto{\pgfpoint{387.359985pt}{47.519989pt}}
\pgfusepath{stroke}
\pgfpathmoveto{\pgfpoint{396.287964pt}{53.696838pt}}
\pgflineto{\pgfpoint{387.359985pt}{53.696838pt}}
\pgfusepath{stroke}
\pgfpathmoveto{\pgfpoint{396.287964pt}{59.873672pt}}
\pgflineto{\pgfpoint{387.359985pt}{59.873672pt}}
\pgfusepath{stroke}
\pgfpathmoveto{\pgfpoint{396.287964pt}{66.050522pt}}
\pgflineto{\pgfpoint{387.359985pt}{66.050522pt}}
\pgfusepath{stroke}
\pgfpathmoveto{\pgfpoint{396.287964pt}{72.227356pt}}
\pgflineto{\pgfpoint{387.359985pt}{72.227356pt}}
\pgfusepath{stroke}
\pgfpathmoveto{\pgfpoint{396.287964pt}{78.404205pt}}
\pgflineto{\pgfpoint{387.359985pt}{78.404205pt}}
\pgfusepath{stroke}
\pgfpathmoveto{\pgfpoint{396.287964pt}{84.581039pt}}
\pgflineto{\pgfpoint{387.359985pt}{84.581039pt}}
\pgfusepath{stroke}
\pgfpathmoveto{\pgfpoint{396.287964pt}{90.757896pt}}
\pgflineto{\pgfpoint{387.359985pt}{90.757896pt}}
\pgfusepath{stroke}
\pgfpathmoveto{\pgfpoint{396.287964pt}{96.934731pt}}
\pgflineto{\pgfpoint{387.359985pt}{96.934731pt}}
\pgfusepath{stroke}
\pgfpathmoveto{\pgfpoint{396.287964pt}{103.111580pt}}
\pgflineto{\pgfpoint{387.359985pt}{103.111580pt}}
\pgfusepath{stroke}
\pgfpathmoveto{\pgfpoint{396.287964pt}{109.288422pt}}
\pgflineto{\pgfpoint{387.359985pt}{109.288422pt}}
\pgfusepath{stroke}
\pgfpathmoveto{\pgfpoint{396.287964pt}{115.465263pt}}
\pgflineto{\pgfpoint{387.359985pt}{115.465263pt}}
\pgfusepath{stroke}
\pgfpathmoveto{\pgfpoint{396.287964pt}{121.642097pt}}
\pgflineto{\pgfpoint{387.359985pt}{121.642097pt}}
\pgfusepath{stroke}
\pgfpathmoveto{\pgfpoint{396.287964pt}{127.818947pt}}
\pgflineto{\pgfpoint{387.359985pt}{127.818947pt}}
\pgfusepath{stroke}
\pgfpathmoveto{\pgfpoint{396.287964pt}{133.995789pt}}
\pgflineto{\pgfpoint{387.359985pt}{133.995789pt}}
\pgfusepath{stroke}
\pgfpathmoveto{\pgfpoint{396.287964pt}{140.172638pt}}
\pgflineto{\pgfpoint{387.359985pt}{140.172638pt}}
\pgfusepath{stroke}
\pgfpathmoveto{\pgfpoint{396.287964pt}{146.349472pt}}
\pgflineto{\pgfpoint{387.359985pt}{146.349472pt}}
\pgfusepath{stroke}
\pgfpathmoveto{\pgfpoint{396.287964pt}{152.526306pt}}
\pgflineto{\pgfpoint{387.373566pt}{152.526306pt}}
\pgfusepath{stroke}
\pgfpathmoveto{\pgfpoint{396.287964pt}{158.703156pt}}
\pgflineto{\pgfpoint{387.373596pt}{158.703156pt}}
\pgfusepath{stroke}
\pgfpathmoveto{\pgfpoint{396.287964pt}{164.880005pt}}
\pgflineto{\pgfpoint{387.373627pt}{164.880005pt}}
\pgfusepath{stroke}
\pgfpathmoveto{\pgfpoint{396.287964pt}{171.056854pt}}
\pgflineto{\pgfpoint{387.378143pt}{171.056854pt}}
\pgfusepath{stroke}
\pgfpathmoveto{\pgfpoint{396.287964pt}{177.233673pt}}
\pgflineto{\pgfpoint{387.378113pt}{177.233673pt}}
\pgfusepath{stroke}
\pgfpathmoveto{\pgfpoint{396.287964pt}{183.410522pt}}
\pgflineto{\pgfpoint{387.382660pt}{183.410522pt}}
\pgfusepath{stroke}
\pgfpathmoveto{\pgfpoint{396.287964pt}{189.587372pt}}
\pgflineto{\pgfpoint{387.382660pt}{189.587372pt}}
\pgfusepath{stroke}
\pgfpathmoveto{\pgfpoint{396.287964pt}{195.764206pt}}
\pgflineto{\pgfpoint{387.382690pt}{195.764206pt}}
\pgfusepath{stroke}
\pgfpathmoveto{\pgfpoint{396.287964pt}{201.941055pt}}
\pgflineto{\pgfpoint{387.387207pt}{201.941055pt}}
\pgfusepath{stroke}
\pgfpathmoveto{\pgfpoint{396.287964pt}{208.117905pt}}
\pgflineto{\pgfpoint{387.387238pt}{208.117905pt}}
\pgfusepath{stroke}
\pgfpathmoveto{\pgfpoint{396.287964pt}{214.294739pt}}
\pgflineto{\pgfpoint{387.391724pt}{214.294739pt}}
\pgfusepath{stroke}
\pgfpathmoveto{\pgfpoint{396.287964pt}{146.349472pt}}
\pgflineto{\pgfpoint{396.287964pt}{140.172638pt}}
\pgfusepath{stroke}
\pgfpathmoveto{\pgfpoint{396.287964pt}{140.172638pt}}
\pgflineto{\pgfpoint{396.287964pt}{133.995789pt}}
\pgfusepath{stroke}
\pgfpathmoveto{\pgfpoint{396.287964pt}{152.526306pt}}
\pgflineto{\pgfpoint{396.287964pt}{146.349472pt}}
\pgfusepath{stroke}
\pgfpathmoveto{\pgfpoint{396.287964pt}{158.703156pt}}
\pgflineto{\pgfpoint{396.287964pt}{152.526306pt}}
\pgfusepath{stroke}
\pgfpathmoveto{\pgfpoint{396.287964pt}{164.880005pt}}
\pgflineto{\pgfpoint{396.287964pt}{158.703156pt}}
\pgfusepath{stroke}
\pgfpathmoveto{\pgfpoint{396.287964pt}{140.172638pt}}
\pgflineto{\pgfpoint{396.301422pt}{140.172638pt}}
\pgfusepath{stroke}
\pgfpathmoveto{\pgfpoint{396.287964pt}{146.349472pt}}
\pgflineto{\pgfpoint{396.301392pt}{146.349472pt}}
\pgfusepath{stroke}
\pgfpathmoveto{\pgfpoint{396.287964pt}{152.526306pt}}
\pgflineto{\pgfpoint{396.305878pt}{152.526306pt}}
\pgfusepath{stroke}
\pgfpathmoveto{\pgfpoint{396.287964pt}{158.703156pt}}
\pgflineto{\pgfpoint{396.305878pt}{158.703156pt}}
\pgfusepath{stroke}
\pgfpathmoveto{\pgfpoint{396.287964pt}{195.764206pt}}
\pgflineto{\pgfpoint{396.287964pt}{189.587372pt}}
\pgfusepath{stroke}
\pgfpathmoveto{\pgfpoint{396.287964pt}{171.056854pt}}
\pgflineto{\pgfpoint{396.287964pt}{164.880005pt}}
\pgfusepath{stroke}
\pgfpathmoveto{\pgfpoint{396.287964pt}{177.233673pt}}
\pgflineto{\pgfpoint{396.287964pt}{171.056854pt}}
\pgfusepath{stroke}
\pgfpathmoveto{\pgfpoint{396.287964pt}{183.410522pt}}
\pgflineto{\pgfpoint{396.287964pt}{177.233673pt}}
\pgfusepath{stroke}
\pgfpathmoveto{\pgfpoint{396.287964pt}{189.587372pt}}
\pgflineto{\pgfpoint{396.287964pt}{183.410522pt}}
\pgfusepath{stroke}
\pgfpathmoveto{\pgfpoint{396.287964pt}{201.941055pt}}
\pgflineto{\pgfpoint{396.287964pt}{195.764206pt}}
\pgfusepath{stroke}
\pgfpathmoveto{\pgfpoint{396.287964pt}{208.117905pt}}
\pgflineto{\pgfpoint{396.287964pt}{201.941055pt}}
\pgfusepath{stroke}
\pgfpathmoveto{\pgfpoint{396.287964pt}{214.294739pt}}
\pgflineto{\pgfpoint{396.287964pt}{208.117905pt}}
\pgfusepath{stroke}
\pgfpathmoveto{\pgfpoint{396.287964pt}{164.880005pt}}
\pgflineto{\pgfpoint{396.305878pt}{164.880005pt}}
\pgfusepath{stroke}
\pgfpathmoveto{\pgfpoint{396.287964pt}{171.056854pt}}
\pgflineto{\pgfpoint{396.310364pt}{171.056854pt}}
\pgfusepath{stroke}
\pgfpathmoveto{\pgfpoint{396.287964pt}{177.233673pt}}
\pgflineto{\pgfpoint{396.310333pt}{177.233673pt}}
\pgfusepath{stroke}
\pgfpathmoveto{\pgfpoint{396.287964pt}{183.410522pt}}
\pgflineto{\pgfpoint{396.314850pt}{183.410522pt}}
\pgfusepath{stroke}
\pgfpathmoveto{\pgfpoint{396.287964pt}{189.587372pt}}
\pgflineto{\pgfpoint{396.314850pt}{189.587372pt}}
\pgfusepath{stroke}
\pgfpathmoveto{\pgfpoint{396.287964pt}{195.764206pt}}
\pgflineto{\pgfpoint{396.319336pt}{195.764206pt}}
\pgfusepath{stroke}
\pgfpathmoveto{\pgfpoint{396.287964pt}{201.941055pt}}
\pgflineto{\pgfpoint{396.319336pt}{201.941055pt}}
\pgfusepath{stroke}
\pgfpathmoveto{\pgfpoint{396.287964pt}{78.404205pt}}
\pgflineto{\pgfpoint{396.287964pt}{72.227356pt}}
\pgfusepath{stroke}
\pgfpathmoveto{\pgfpoint{396.287964pt}{53.696838pt}}
\pgflineto{\pgfpoint{396.287964pt}{47.519989pt}}
\pgfusepath{stroke}
\pgfpathmoveto{\pgfpoint{396.287964pt}{59.873672pt}}
\pgflineto{\pgfpoint{396.287964pt}{53.696838pt}}
\pgfusepath{stroke}
\pgfpathmoveto{\pgfpoint{396.287964pt}{66.050522pt}}
\pgflineto{\pgfpoint{396.287964pt}{59.873672pt}}
\pgfusepath{stroke}
\pgfpathmoveto{\pgfpoint{396.287964pt}{72.227356pt}}
\pgflineto{\pgfpoint{396.287964pt}{66.050522pt}}
\pgfusepath{stroke}
\pgfpathmoveto{\pgfpoint{396.287964pt}{84.581039pt}}
\pgflineto{\pgfpoint{396.287964pt}{78.404205pt}}
\pgfusepath{stroke}
\pgfpathmoveto{\pgfpoint{396.287964pt}{90.757896pt}}
\pgflineto{\pgfpoint{396.287964pt}{84.581039pt}}
\pgfusepath{stroke}
\pgfpathmoveto{\pgfpoint{396.287964pt}{96.934731pt}}
\pgflineto{\pgfpoint{396.287964pt}{90.757896pt}}
\pgfusepath{stroke}
\pgfpathmoveto{\pgfpoint{396.287964pt}{103.111580pt}}
\pgflineto{\pgfpoint{396.287964pt}{96.934731pt}}
\pgfusepath{stroke}
\pgfpathmoveto{\pgfpoint{396.287964pt}{109.288422pt}}
\pgflineto{\pgfpoint{396.287964pt}{103.111580pt}}
\pgfusepath{stroke}
\pgfpathmoveto{\pgfpoint{396.287964pt}{115.465263pt}}
\pgflineto{\pgfpoint{396.287964pt}{109.288422pt}}
\pgfusepath{stroke}
\pgfpathmoveto{\pgfpoint{396.287964pt}{121.642097pt}}
\pgflineto{\pgfpoint{396.287964pt}{115.465263pt}}
\pgfusepath{stroke}
\pgfpathmoveto{\pgfpoint{396.287964pt}{127.818947pt}}
\pgflineto{\pgfpoint{396.287964pt}{121.642097pt}}
\pgfusepath{stroke}
\pgfpathmoveto{\pgfpoint{396.287964pt}{133.995789pt}}
\pgflineto{\pgfpoint{396.287964pt}{127.818947pt}}
\pgfusepath{stroke}
\pgfpathmoveto{\pgfpoint{396.287964pt}{47.519989pt}}
\pgflineto{\pgfpoint{405.202393pt}{47.519989pt}}
\pgfusepath{stroke}
\pgfpathmoveto{\pgfpoint{405.216003pt}{53.696838pt}}
\pgflineto{\pgfpoint{396.287964pt}{53.696838pt}}
\pgfusepath{stroke}
\pgfpathmoveto{\pgfpoint{405.216003pt}{59.873672pt}}
\pgflineto{\pgfpoint{396.287964pt}{59.873672pt}}
\pgfusepath{stroke}
\pgfpathmoveto{\pgfpoint{405.216003pt}{66.050522pt}}
\pgflineto{\pgfpoint{396.287964pt}{66.050522pt}}
\pgfusepath{stroke}
\pgfpathmoveto{\pgfpoint{405.216003pt}{72.227356pt}}
\pgflineto{\pgfpoint{396.287964pt}{72.227356pt}}
\pgfusepath{stroke}
\pgfpathmoveto{\pgfpoint{405.216003pt}{78.404205pt}}
\pgflineto{\pgfpoint{396.287964pt}{78.404205pt}}
\pgfusepath{stroke}
\pgfpathmoveto{\pgfpoint{405.216003pt}{84.581039pt}}
\pgflineto{\pgfpoint{396.287964pt}{84.581039pt}}
\pgfusepath{stroke}
\pgfpathmoveto{\pgfpoint{405.216003pt}{90.757896pt}}
\pgflineto{\pgfpoint{396.287964pt}{90.757896pt}}
\pgfusepath{stroke}
\pgfpathmoveto{\pgfpoint{405.216003pt}{96.934731pt}}
\pgflineto{\pgfpoint{396.287964pt}{96.934731pt}}
\pgfusepath{stroke}
\pgfpathmoveto{\pgfpoint{405.216003pt}{103.111580pt}}
\pgflineto{\pgfpoint{396.287964pt}{103.111580pt}}
\pgfusepath{stroke}
\pgfpathmoveto{\pgfpoint{405.216003pt}{109.288422pt}}
\pgflineto{\pgfpoint{396.287964pt}{109.288422pt}}
\pgfusepath{stroke}
\pgfpathmoveto{\pgfpoint{405.216003pt}{115.465263pt}}
\pgflineto{\pgfpoint{396.287964pt}{115.465263pt}}
\pgfusepath{stroke}
\pgfpathmoveto{\pgfpoint{405.216003pt}{121.642097pt}}
\pgflineto{\pgfpoint{396.287964pt}{121.642097pt}}
\pgfusepath{stroke}
\pgfpathmoveto{\pgfpoint{405.216003pt}{127.818947pt}}
\pgflineto{\pgfpoint{396.287964pt}{127.818947pt}}
\pgfusepath{stroke}
\pgfpathmoveto{\pgfpoint{405.216003pt}{133.995789pt}}
\pgflineto{\pgfpoint{396.287964pt}{133.995789pt}}
\pgfusepath{stroke}
\pgfpathmoveto{\pgfpoint{405.216003pt}{140.172638pt}}
\pgflineto{\pgfpoint{396.301422pt}{140.172638pt}}
\pgfusepath{stroke}
\pgfpathmoveto{\pgfpoint{405.216003pt}{146.349472pt}}
\pgflineto{\pgfpoint{396.301392pt}{146.349472pt}}
\pgfusepath{stroke}
\pgfpathmoveto{\pgfpoint{405.216003pt}{152.526306pt}}
\pgflineto{\pgfpoint{396.305878pt}{152.526306pt}}
\pgfusepath{stroke}
\pgfpathmoveto{\pgfpoint{405.216003pt}{158.703156pt}}
\pgflineto{\pgfpoint{396.305878pt}{158.703156pt}}
\pgfusepath{stroke}
\pgfpathmoveto{\pgfpoint{405.216003pt}{164.880005pt}}
\pgflineto{\pgfpoint{396.305878pt}{164.880005pt}}
\pgfusepath{stroke}
\pgfpathmoveto{\pgfpoint{405.216003pt}{171.056854pt}}
\pgflineto{\pgfpoint{396.310364pt}{171.056854pt}}
\pgfusepath{stroke}
\pgfpathmoveto{\pgfpoint{405.216003pt}{177.233673pt}}
\pgflineto{\pgfpoint{396.310333pt}{177.233673pt}}
\pgfusepath{stroke}
\pgfpathmoveto{\pgfpoint{405.216003pt}{183.410522pt}}
\pgflineto{\pgfpoint{396.314850pt}{183.410522pt}}
\pgfusepath{stroke}
\pgfpathmoveto{\pgfpoint{405.216003pt}{189.587372pt}}
\pgflineto{\pgfpoint{396.314850pt}{189.587372pt}}
\pgfusepath{stroke}
\pgfpathmoveto{\pgfpoint{405.216003pt}{195.764206pt}}
\pgflineto{\pgfpoint{396.319336pt}{195.764206pt}}
\pgfusepath{stroke}
\pgfpathmoveto{\pgfpoint{405.216003pt}{201.941055pt}}
\pgflineto{\pgfpoint{396.319336pt}{201.941055pt}}
\pgfusepath{stroke}
\pgfpathmoveto{\pgfpoint{405.216003pt}{140.172638pt}}
\pgflineto{\pgfpoint{405.216003pt}{133.995789pt}}
\pgfusepath{stroke}
\pgfpathmoveto{\pgfpoint{405.216003pt}{133.995789pt}}
\pgflineto{\pgfpoint{405.216003pt}{127.818947pt}}
\pgfusepath{stroke}
\pgfpathmoveto{\pgfpoint{405.216003pt}{146.349472pt}}
\pgflineto{\pgfpoint{405.216003pt}{140.172638pt}}
\pgfusepath{stroke}
\pgfpathmoveto{\pgfpoint{405.216003pt}{152.526306pt}}
\pgflineto{\pgfpoint{405.216003pt}{146.349472pt}}
\pgfusepath{stroke}
\pgfpathmoveto{\pgfpoint{405.216003pt}{158.703156pt}}
\pgflineto{\pgfpoint{405.216003pt}{152.526306pt}}
\pgfusepath{stroke}
\pgfpathmoveto{\pgfpoint{405.216003pt}{133.995789pt}}
\pgflineto{\pgfpoint{405.229584pt}{133.995789pt}}
\pgfusepath{stroke}
\pgfpathmoveto{\pgfpoint{405.216003pt}{140.172638pt}}
\pgflineto{\pgfpoint{405.229584pt}{140.172638pt}}
\pgfusepath{stroke}
\pgfpathmoveto{\pgfpoint{405.216003pt}{146.349472pt}}
\pgflineto{\pgfpoint{405.229584pt}{146.349472pt}}
\pgfusepath{stroke}
\pgfpathmoveto{\pgfpoint{405.216003pt}{152.526306pt}}
\pgflineto{\pgfpoint{405.234100pt}{152.526306pt}}
\pgfusepath{stroke}
\pgfpathmoveto{\pgfpoint{405.216003pt}{183.410522pt}}
\pgflineto{\pgfpoint{405.216003pt}{177.233673pt}}
\pgfusepath{stroke}
\pgfpathmoveto{\pgfpoint{405.216003pt}{164.880005pt}}
\pgflineto{\pgfpoint{405.216003pt}{158.703156pt}}
\pgfusepath{stroke}
\pgfpathmoveto{\pgfpoint{405.216003pt}{171.056854pt}}
\pgflineto{\pgfpoint{405.216003pt}{164.880005pt}}
\pgfusepath{stroke}
\pgfpathmoveto{\pgfpoint{405.216003pt}{177.233673pt}}
\pgflineto{\pgfpoint{405.216003pt}{171.056854pt}}
\pgfusepath{stroke}
\pgfpathmoveto{\pgfpoint{405.216003pt}{189.587372pt}}
\pgflineto{\pgfpoint{405.216003pt}{183.410522pt}}
\pgfusepath{stroke}
\pgfpathmoveto{\pgfpoint{405.216003pt}{195.764206pt}}
\pgflineto{\pgfpoint{405.216003pt}{189.587372pt}}
\pgfusepath{stroke}
\pgfpathmoveto{\pgfpoint{405.216003pt}{201.941055pt}}
\pgflineto{\pgfpoint{405.216003pt}{195.764206pt}}
\pgfusepath{stroke}
\pgfpathmoveto{\pgfpoint{405.216003pt}{158.703156pt}}
\pgflineto{\pgfpoint{405.234131pt}{158.703156pt}}
\pgfusepath{stroke}
\pgfpathmoveto{\pgfpoint{405.216003pt}{164.880005pt}}
\pgflineto{\pgfpoint{405.238678pt}{164.880005pt}}
\pgfusepath{stroke}
\pgfpathmoveto{\pgfpoint{405.216003pt}{171.056854pt}}
\pgflineto{\pgfpoint{405.238678pt}{171.056854pt}}
\pgfusepath{stroke}
\pgfpathmoveto{\pgfpoint{405.216003pt}{177.233673pt}}
\pgflineto{\pgfpoint{405.238678pt}{177.233673pt}}
\pgfusepath{stroke}
\pgfpathmoveto{\pgfpoint{405.216003pt}{183.410522pt}}
\pgflineto{\pgfpoint{405.243195pt}{183.410522pt}}
\pgfusepath{stroke}
\pgfpathmoveto{\pgfpoint{405.216003pt}{189.587372pt}}
\pgflineto{\pgfpoint{405.243225pt}{189.587372pt}}
\pgfusepath{stroke}
\pgfpathmoveto{\pgfpoint{405.216003pt}{96.934731pt}}
\pgflineto{\pgfpoint{405.216003pt}{90.757896pt}}
\pgfusepath{stroke}
\pgfpathmoveto{\pgfpoint{405.216003pt}{59.873672pt}}
\pgflineto{\pgfpoint{405.216003pt}{53.696838pt}}
\pgfusepath{stroke}
\pgfpathmoveto{\pgfpoint{405.216003pt}{66.050522pt}}
\pgflineto{\pgfpoint{405.216003pt}{59.873672pt}}
\pgfusepath{stroke}
\pgfpathmoveto{\pgfpoint{405.216003pt}{72.227356pt}}
\pgflineto{\pgfpoint{405.216003pt}{66.050522pt}}
\pgfusepath{stroke}
\pgfpathmoveto{\pgfpoint{405.216003pt}{78.404205pt}}
\pgflineto{\pgfpoint{405.216003pt}{72.227356pt}}
\pgfusepath{stroke}
\pgfpathmoveto{\pgfpoint{405.216003pt}{84.581039pt}}
\pgflineto{\pgfpoint{405.216003pt}{78.404205pt}}
\pgfusepath{stroke}
\pgfpathmoveto{\pgfpoint{405.216003pt}{90.757896pt}}
\pgflineto{\pgfpoint{405.216003pt}{84.581039pt}}
\pgfusepath{stroke}
\pgfpathmoveto{\pgfpoint{405.216003pt}{103.111580pt}}
\pgflineto{\pgfpoint{405.216003pt}{96.934731pt}}
\pgfusepath{stroke}
\pgfpathmoveto{\pgfpoint{405.216003pt}{109.288422pt}}
\pgflineto{\pgfpoint{405.216003pt}{103.111580pt}}
\pgfusepath{stroke}
\pgfpathmoveto{\pgfpoint{405.216003pt}{115.465263pt}}
\pgflineto{\pgfpoint{405.216003pt}{109.288422pt}}
\pgfusepath{stroke}
\pgfpathmoveto{\pgfpoint{405.216003pt}{121.642097pt}}
\pgflineto{\pgfpoint{405.216003pt}{115.465263pt}}
\pgfusepath{stroke}
\pgfpathmoveto{\pgfpoint{405.216003pt}{127.818947pt}}
\pgflineto{\pgfpoint{405.216003pt}{121.642097pt}}
\pgfusepath{stroke}
\pgfpathmoveto{\pgfpoint{405.216003pt}{47.519989pt}}
\pgflineto{\pgfpoint{405.202393pt}{47.519989pt}}
\pgfusepath{stroke}
\pgfpathmoveto{\pgfpoint{405.216003pt}{53.696838pt}}
\pgflineto{\pgfpoint{405.216003pt}{47.519989pt}}
\pgfusepath{stroke}
\pgfpathmoveto{\pgfpoint{414.143982pt}{47.519989pt}}
\pgflineto{\pgfpoint{405.216003pt}{47.519989pt}}
\pgfusepath{stroke}
\pgfpathmoveto{\pgfpoint{414.143982pt}{53.696838pt}}
\pgflineto{\pgfpoint{405.216003pt}{53.696838pt}}
\pgfusepath{stroke}
\pgfpathmoveto{\pgfpoint{414.143982pt}{59.873672pt}}
\pgflineto{\pgfpoint{405.216003pt}{59.873672pt}}
\pgfusepath{stroke}
\pgfpathmoveto{\pgfpoint{414.143982pt}{66.050522pt}}
\pgflineto{\pgfpoint{405.216003pt}{66.050522pt}}
\pgfusepath{stroke}
\pgfpathmoveto{\pgfpoint{414.143982pt}{72.227356pt}}
\pgflineto{\pgfpoint{405.216003pt}{72.227356pt}}
\pgfusepath{stroke}
\pgfpathmoveto{\pgfpoint{414.143982pt}{78.404205pt}}
\pgflineto{\pgfpoint{405.216003pt}{78.404205pt}}
\pgfusepath{stroke}
\pgfpathmoveto{\pgfpoint{414.143982pt}{84.581039pt}}
\pgflineto{\pgfpoint{405.216003pt}{84.581039pt}}
\pgfusepath{stroke}
\pgfpathmoveto{\pgfpoint{414.143982pt}{90.757896pt}}
\pgflineto{\pgfpoint{405.216003pt}{90.757896pt}}
\pgfusepath{stroke}
\pgfpathmoveto{\pgfpoint{414.143982pt}{96.934731pt}}
\pgflineto{\pgfpoint{405.216003pt}{96.934731pt}}
\pgfusepath{stroke}
\pgfpathmoveto{\pgfpoint{414.143982pt}{103.111580pt}}
\pgflineto{\pgfpoint{405.216003pt}{103.111580pt}}
\pgfusepath{stroke}
\pgfpathmoveto{\pgfpoint{414.143982pt}{109.288422pt}}
\pgflineto{\pgfpoint{405.216003pt}{109.288422pt}}
\pgfusepath{stroke}
\pgfpathmoveto{\pgfpoint{414.143982pt}{115.465263pt}}
\pgflineto{\pgfpoint{405.216003pt}{115.465263pt}}
\pgfusepath{stroke}
\pgfpathmoveto{\pgfpoint{414.143982pt}{121.642097pt}}
\pgflineto{\pgfpoint{405.216003pt}{121.642097pt}}
\pgfusepath{stroke}
\pgfpathmoveto{\pgfpoint{414.143982pt}{127.818947pt}}
\pgflineto{\pgfpoint{405.216003pt}{127.818947pt}}
\pgfusepath{stroke}
\pgfpathmoveto{\pgfpoint{414.143982pt}{133.995789pt}}
\pgflineto{\pgfpoint{405.229584pt}{133.995789pt}}
\pgfusepath{stroke}
\pgfpathmoveto{\pgfpoint{414.143982pt}{140.172638pt}}
\pgflineto{\pgfpoint{405.229584pt}{140.172638pt}}
\pgfusepath{stroke}
\pgfpathmoveto{\pgfpoint{414.143982pt}{146.349472pt}}
\pgflineto{\pgfpoint{405.229584pt}{146.349472pt}}
\pgfusepath{stroke}
\pgfpathmoveto{\pgfpoint{414.143982pt}{152.526306pt}}
\pgflineto{\pgfpoint{405.234100pt}{152.526306pt}}
\pgfusepath{stroke}
\pgfpathmoveto{\pgfpoint{414.143982pt}{158.703156pt}}
\pgflineto{\pgfpoint{405.234131pt}{158.703156pt}}
\pgfusepath{stroke}
\pgfpathmoveto{\pgfpoint{414.143982pt}{164.880005pt}}
\pgflineto{\pgfpoint{405.238678pt}{164.880005pt}}
\pgfusepath{stroke}
\pgfpathmoveto{\pgfpoint{414.143982pt}{171.056854pt}}
\pgflineto{\pgfpoint{405.238678pt}{171.056854pt}}
\pgfusepath{stroke}
\pgfpathmoveto{\pgfpoint{414.143982pt}{177.233673pt}}
\pgflineto{\pgfpoint{405.238678pt}{177.233673pt}}
\pgfusepath{stroke}
\pgfpathmoveto{\pgfpoint{414.143982pt}{183.410522pt}}
\pgflineto{\pgfpoint{405.243195pt}{183.410522pt}}
\pgfusepath{stroke}
\pgfpathmoveto{\pgfpoint{414.143982pt}{189.587372pt}}
\pgflineto{\pgfpoint{405.243225pt}{189.587372pt}}
\pgfusepath{stroke}
\pgfpathmoveto{\pgfpoint{414.143982pt}{140.172638pt}}
\pgflineto{\pgfpoint{414.143982pt}{133.995789pt}}
\pgfusepath{stroke}
\pgfpathmoveto{\pgfpoint{414.143982pt}{133.995789pt}}
\pgflineto{\pgfpoint{414.143982pt}{127.818947pt}}
\pgfusepath{stroke}
\pgfpathmoveto{\pgfpoint{414.143982pt}{146.349472pt}}
\pgflineto{\pgfpoint{414.143982pt}{140.172638pt}}
\pgfusepath{stroke}
\pgfpathmoveto{\pgfpoint{414.143982pt}{133.995789pt}}
\pgflineto{\pgfpoint{414.157593pt}{133.995789pt}}
\pgfusepath{stroke}
\pgfpathmoveto{\pgfpoint{414.143982pt}{140.172638pt}}
\pgflineto{\pgfpoint{414.157623pt}{140.172638pt}}
\pgfusepath{stroke}
\pgfpathmoveto{\pgfpoint{414.143982pt}{177.233673pt}}
\pgflineto{\pgfpoint{414.143982pt}{171.056854pt}}
\pgfusepath{stroke}
\pgfpathmoveto{\pgfpoint{414.143982pt}{152.526306pt}}
\pgflineto{\pgfpoint{414.143982pt}{146.349472pt}}
\pgfusepath{stroke}
\pgfpathmoveto{\pgfpoint{414.143982pt}{158.703156pt}}
\pgflineto{\pgfpoint{414.143982pt}{152.526306pt}}
\pgfusepath{stroke}
\pgfpathmoveto{\pgfpoint{414.143982pt}{164.880005pt}}
\pgflineto{\pgfpoint{414.143982pt}{158.703156pt}}
\pgfusepath{stroke}
\pgfpathmoveto{\pgfpoint{414.143982pt}{171.056854pt}}
\pgflineto{\pgfpoint{414.143982pt}{164.880005pt}}
\pgfusepath{stroke}
\pgfpathmoveto{\pgfpoint{414.143982pt}{183.410522pt}}
\pgflineto{\pgfpoint{414.143982pt}{177.233673pt}}
\pgfusepath{stroke}
\pgfpathmoveto{\pgfpoint{414.143982pt}{189.587372pt}}
\pgflineto{\pgfpoint{414.143982pt}{183.410522pt}}
\pgfusepath{stroke}
\pgfpathmoveto{\pgfpoint{414.143982pt}{146.349472pt}}
\pgflineto{\pgfpoint{414.157623pt}{146.349472pt}}
\pgfusepath{stroke}
\pgfpathmoveto{\pgfpoint{414.143982pt}{152.526306pt}}
\pgflineto{\pgfpoint{414.162109pt}{152.526306pt}}
\pgfusepath{stroke}
\pgfpathmoveto{\pgfpoint{414.143982pt}{158.703156pt}}
\pgflineto{\pgfpoint{414.162140pt}{158.703156pt}}
\pgfusepath{stroke}
\pgfpathmoveto{\pgfpoint{414.143982pt}{164.880005pt}}
\pgflineto{\pgfpoint{414.166656pt}{164.880005pt}}
\pgfusepath{stroke}
\pgfpathmoveto{\pgfpoint{414.143982pt}{171.056854pt}}
\pgflineto{\pgfpoint{414.166687pt}{171.056854pt}}
\pgfusepath{stroke}
\pgfpathmoveto{\pgfpoint{414.143982pt}{177.233673pt}}
\pgflineto{\pgfpoint{414.171204pt}{177.233673pt}}
\pgfusepath{stroke}
\pgfpathmoveto{\pgfpoint{414.143982pt}{72.227356pt}}
\pgflineto{\pgfpoint{414.143982pt}{66.050522pt}}
\pgfusepath{stroke}
\pgfpathmoveto{\pgfpoint{414.143982pt}{53.696838pt}}
\pgflineto{\pgfpoint{414.143982pt}{47.519989pt}}
\pgfusepath{stroke}
\pgfpathmoveto{\pgfpoint{414.143982pt}{59.873672pt}}
\pgflineto{\pgfpoint{414.143982pt}{53.696838pt}}
\pgfusepath{stroke}
\pgfpathmoveto{\pgfpoint{414.143982pt}{66.050522pt}}
\pgflineto{\pgfpoint{414.143982pt}{59.873672pt}}
\pgfusepath{stroke}
\pgfpathmoveto{\pgfpoint{414.143982pt}{78.404205pt}}
\pgflineto{\pgfpoint{414.143982pt}{72.227356pt}}
\pgfusepath{stroke}
\pgfpathmoveto{\pgfpoint{414.143982pt}{84.581039pt}}
\pgflineto{\pgfpoint{414.143982pt}{78.404205pt}}
\pgfusepath{stroke}
\pgfpathmoveto{\pgfpoint{414.143982pt}{90.757896pt}}
\pgflineto{\pgfpoint{414.143982pt}{84.581039pt}}
\pgfusepath{stroke}
\pgfpathmoveto{\pgfpoint{414.143982pt}{96.934731pt}}
\pgflineto{\pgfpoint{414.143982pt}{90.757896pt}}
\pgfusepath{stroke}
\pgfpathmoveto{\pgfpoint{414.143982pt}{103.111580pt}}
\pgflineto{\pgfpoint{414.143982pt}{96.934731pt}}
\pgfusepath{stroke}
\pgfpathmoveto{\pgfpoint{414.143982pt}{109.288422pt}}
\pgflineto{\pgfpoint{414.143982pt}{103.111580pt}}
\pgfusepath{stroke}
\pgfpathmoveto{\pgfpoint{414.143982pt}{115.465263pt}}
\pgflineto{\pgfpoint{414.143982pt}{109.288422pt}}
\pgfusepath{stroke}
\pgfpathmoveto{\pgfpoint{414.143982pt}{121.642097pt}}
\pgflineto{\pgfpoint{414.143982pt}{115.465263pt}}
\pgfusepath{stroke}
\pgfpathmoveto{\pgfpoint{414.143982pt}{127.818947pt}}
\pgflineto{\pgfpoint{414.143982pt}{121.642097pt}}
\pgfusepath{stroke}
\pgfpathmoveto{\pgfpoint{414.143982pt}{47.519989pt}}
\pgflineto{\pgfpoint{423.053925pt}{47.519989pt}}
\pgfusepath{stroke}
\pgfpathmoveto{\pgfpoint{423.071960pt}{53.696838pt}}
\pgflineto{\pgfpoint{414.143982pt}{53.696838pt}}
\pgfusepath{stroke}
\pgfpathmoveto{\pgfpoint{423.071960pt}{59.873672pt}}
\pgflineto{\pgfpoint{414.143982pt}{59.873672pt}}
\pgfusepath{stroke}
\pgfpathmoveto{\pgfpoint{423.071960pt}{66.050522pt}}
\pgflineto{\pgfpoint{414.143982pt}{66.050522pt}}
\pgfusepath{stroke}
\pgfpathmoveto{\pgfpoint{423.071960pt}{72.227356pt}}
\pgflineto{\pgfpoint{414.143982pt}{72.227356pt}}
\pgfusepath{stroke}
\pgfpathmoveto{\pgfpoint{423.071960pt}{78.404205pt}}
\pgflineto{\pgfpoint{414.143982pt}{78.404205pt}}
\pgfusepath{stroke}
\pgfpathmoveto{\pgfpoint{423.071960pt}{84.581039pt}}
\pgflineto{\pgfpoint{414.143982pt}{84.581039pt}}
\pgfusepath{stroke}
\pgfpathmoveto{\pgfpoint{423.071960pt}{90.757896pt}}
\pgflineto{\pgfpoint{414.143982pt}{90.757896pt}}
\pgfusepath{stroke}
\pgfpathmoveto{\pgfpoint{423.071960pt}{96.934731pt}}
\pgflineto{\pgfpoint{414.143982pt}{96.934731pt}}
\pgfusepath{stroke}
\pgfpathmoveto{\pgfpoint{423.071960pt}{103.111580pt}}
\pgflineto{\pgfpoint{414.143982pt}{103.111580pt}}
\pgfusepath{stroke}
\pgfpathmoveto{\pgfpoint{423.071960pt}{109.288422pt}}
\pgflineto{\pgfpoint{414.143982pt}{109.288422pt}}
\pgfusepath{stroke}
\pgfpathmoveto{\pgfpoint{423.071960pt}{115.465263pt}}
\pgflineto{\pgfpoint{414.143982pt}{115.465263pt}}
\pgfusepath{stroke}
\pgfpathmoveto{\pgfpoint{423.071960pt}{121.642097pt}}
\pgflineto{\pgfpoint{414.143982pt}{121.642097pt}}
\pgfusepath{stroke}
\pgfpathmoveto{\pgfpoint{423.071960pt}{127.818947pt}}
\pgflineto{\pgfpoint{414.143982pt}{127.818947pt}}
\pgfusepath{stroke}
\pgfpathmoveto{\pgfpoint{423.071960pt}{133.995789pt}}
\pgflineto{\pgfpoint{414.157593pt}{133.995789pt}}
\pgfusepath{stroke}
\pgfpathmoveto{\pgfpoint{423.071960pt}{140.172638pt}}
\pgflineto{\pgfpoint{414.157623pt}{140.172638pt}}
\pgfusepath{stroke}
\pgfpathmoveto{\pgfpoint{423.071960pt}{146.349472pt}}
\pgflineto{\pgfpoint{414.157623pt}{146.349472pt}}
\pgfusepath{stroke}
\pgfpathmoveto{\pgfpoint{423.071960pt}{152.526306pt}}
\pgflineto{\pgfpoint{414.162109pt}{152.526306pt}}
\pgfusepath{stroke}
\pgfpathmoveto{\pgfpoint{423.071960pt}{158.703156pt}}
\pgflineto{\pgfpoint{414.162140pt}{158.703156pt}}
\pgfusepath{stroke}
\pgfpathmoveto{\pgfpoint{423.071960pt}{164.880005pt}}
\pgflineto{\pgfpoint{414.166656pt}{164.880005pt}}
\pgfusepath{stroke}
\pgfpathmoveto{\pgfpoint{423.071960pt}{171.056854pt}}
\pgflineto{\pgfpoint{414.166687pt}{171.056854pt}}
\pgfusepath{stroke}
\pgfpathmoveto{\pgfpoint{423.071960pt}{177.233673pt}}
\pgflineto{\pgfpoint{414.171204pt}{177.233673pt}}
\pgfusepath{stroke}
\pgfpathmoveto{\pgfpoint{423.071960pt}{164.880005pt}}
\pgflineto{\pgfpoint{423.071960pt}{158.703156pt}}
\pgfusepath{stroke}
\pgfpathmoveto{\pgfpoint{423.071960pt}{146.349472pt}}
\pgflineto{\pgfpoint{423.071960pt}{140.172638pt}}
\pgfusepath{stroke}
\pgfpathmoveto{\pgfpoint{423.071960pt}{152.526306pt}}
\pgflineto{\pgfpoint{423.071960pt}{146.349472pt}}
\pgfusepath{stroke}
\pgfpathmoveto{\pgfpoint{423.071960pt}{158.703156pt}}
\pgflineto{\pgfpoint{423.071960pt}{152.526306pt}}
\pgfusepath{stroke}
\pgfpathmoveto{\pgfpoint{423.071960pt}{171.056854pt}}
\pgflineto{\pgfpoint{423.071960pt}{164.880005pt}}
\pgfusepath{stroke}
\pgfpathmoveto{\pgfpoint{423.071960pt}{177.233673pt}}
\pgflineto{\pgfpoint{423.071960pt}{171.056854pt}}
\pgfusepath{stroke}
\pgfpathmoveto{\pgfpoint{423.071960pt}{146.349472pt}}
\pgflineto{\pgfpoint{423.085480pt}{146.349472pt}}
\pgfusepath{stroke}
\pgfpathmoveto{\pgfpoint{423.071960pt}{152.526306pt}}
\pgflineto{\pgfpoint{423.085480pt}{152.526306pt}}
\pgfusepath{stroke}
\pgfpathmoveto{\pgfpoint{423.071960pt}{158.703156pt}}
\pgflineto{\pgfpoint{423.085480pt}{158.703156pt}}
\pgfusepath{stroke}
\pgfpathmoveto{\pgfpoint{423.071960pt}{164.880005pt}}
\pgflineto{\pgfpoint{423.089996pt}{164.880005pt}}
\pgfusepath{stroke}
\pgfpathmoveto{\pgfpoint{423.071960pt}{171.056854pt}}
\pgflineto{\pgfpoint{423.090027pt}{171.056854pt}}
\pgfusepath{stroke}
\pgfpathmoveto{\pgfpoint{423.071960pt}{109.288422pt}}
\pgflineto{\pgfpoint{423.071960pt}{103.111580pt}}
\pgfusepath{stroke}
\pgfpathmoveto{\pgfpoint{423.071960pt}{59.873672pt}}
\pgflineto{\pgfpoint{423.071960pt}{53.696838pt}}
\pgfusepath{stroke}
\pgfpathmoveto{\pgfpoint{423.071960pt}{66.050522pt}}
\pgflineto{\pgfpoint{423.071960pt}{59.873672pt}}
\pgfusepath{stroke}
\pgfpathmoveto{\pgfpoint{423.071960pt}{72.227356pt}}
\pgflineto{\pgfpoint{423.071960pt}{66.050522pt}}
\pgfusepath{stroke}
\pgfpathmoveto{\pgfpoint{423.071960pt}{78.404205pt}}
\pgflineto{\pgfpoint{423.071960pt}{72.227356pt}}
\pgfusepath{stroke}
\pgfpathmoveto{\pgfpoint{423.071960pt}{84.581039pt}}
\pgflineto{\pgfpoint{423.071960pt}{78.404205pt}}
\pgfusepath{stroke}
\pgfpathmoveto{\pgfpoint{423.071960pt}{90.757896pt}}
\pgflineto{\pgfpoint{423.071960pt}{84.581039pt}}
\pgfusepath{stroke}
\pgfpathmoveto{\pgfpoint{423.071960pt}{96.934731pt}}
\pgflineto{\pgfpoint{423.071960pt}{90.757896pt}}
\pgfusepath{stroke}
\pgfpathmoveto{\pgfpoint{423.071960pt}{103.111580pt}}
\pgflineto{\pgfpoint{423.071960pt}{96.934731pt}}
\pgfusepath{stroke}
\pgfpathmoveto{\pgfpoint{423.071960pt}{115.465263pt}}
\pgflineto{\pgfpoint{423.071960pt}{109.288422pt}}
\pgfusepath{stroke}
\pgfpathmoveto{\pgfpoint{423.071960pt}{121.642097pt}}
\pgflineto{\pgfpoint{423.071960pt}{115.465263pt}}
\pgfusepath{stroke}
\pgfpathmoveto{\pgfpoint{423.071960pt}{127.818947pt}}
\pgflineto{\pgfpoint{423.071960pt}{121.642097pt}}
\pgfusepath{stroke}
\pgfpathmoveto{\pgfpoint{423.071960pt}{133.995789pt}}
\pgflineto{\pgfpoint{423.071960pt}{127.818947pt}}
\pgfusepath{stroke}
\pgfpathmoveto{\pgfpoint{423.071960pt}{140.172638pt}}
\pgflineto{\pgfpoint{423.071960pt}{133.995789pt}}
\pgfusepath{stroke}
\pgfpathmoveto{\pgfpoint{423.071960pt}{47.519989pt}}
\pgflineto{\pgfpoint{423.053925pt}{47.519989pt}}
\pgfusepath{stroke}
\pgfpathmoveto{\pgfpoint{423.071960pt}{53.696838pt}}
\pgflineto{\pgfpoint{423.071960pt}{47.519989pt}}
\pgfusepath{stroke}
\pgfpathmoveto{\pgfpoint{432.000000pt}{47.519989pt}}
\pgflineto{\pgfpoint{423.071960pt}{47.519989pt}}
\pgfusepath{stroke}
\pgfpathmoveto{\pgfpoint{432.000000pt}{53.696838pt}}
\pgflineto{\pgfpoint{423.071960pt}{53.696838pt}}
\pgfusepath{stroke}
\pgfpathmoveto{\pgfpoint{432.000000pt}{59.873672pt}}
\pgflineto{\pgfpoint{423.071960pt}{59.873672pt}}
\pgfusepath{stroke}
\pgfpathmoveto{\pgfpoint{432.000000pt}{66.050522pt}}
\pgflineto{\pgfpoint{423.071960pt}{66.050522pt}}
\pgfusepath{stroke}
\pgfpathmoveto{\pgfpoint{432.000000pt}{72.227356pt}}
\pgflineto{\pgfpoint{423.071960pt}{72.227356pt}}
\pgfusepath{stroke}
\pgfpathmoveto{\pgfpoint{432.000000pt}{78.404205pt}}
\pgflineto{\pgfpoint{423.071960pt}{78.404205pt}}
\pgfusepath{stroke}
\pgfpathmoveto{\pgfpoint{432.000000pt}{84.581039pt}}
\pgflineto{\pgfpoint{423.071960pt}{84.581039pt}}
\pgfusepath{stroke}
\pgfpathmoveto{\pgfpoint{432.000000pt}{90.757896pt}}
\pgflineto{\pgfpoint{423.071960pt}{90.757896pt}}
\pgfusepath{stroke}
\pgfpathmoveto{\pgfpoint{432.000000pt}{96.934731pt}}
\pgflineto{\pgfpoint{423.071960pt}{96.934731pt}}
\pgfusepath{stroke}
\pgfpathmoveto{\pgfpoint{432.000000pt}{103.111580pt}}
\pgflineto{\pgfpoint{423.071960pt}{103.111580pt}}
\pgfusepath{stroke}
\pgfpathmoveto{\pgfpoint{432.000000pt}{109.288422pt}}
\pgflineto{\pgfpoint{423.071960pt}{109.288422pt}}
\pgfusepath{stroke}
\pgfpathmoveto{\pgfpoint{432.000000pt}{115.465263pt}}
\pgflineto{\pgfpoint{423.071960pt}{115.465263pt}}
\pgfusepath{stroke}
\pgfpathmoveto{\pgfpoint{432.000000pt}{121.642097pt}}
\pgflineto{\pgfpoint{423.071960pt}{121.642097pt}}
\pgfusepath{stroke}
\pgfpathmoveto{\pgfpoint{432.000000pt}{127.818947pt}}
\pgflineto{\pgfpoint{423.071960pt}{127.818947pt}}
\pgfusepath{stroke}
\pgfpathmoveto{\pgfpoint{432.000000pt}{133.995789pt}}
\pgflineto{\pgfpoint{423.071960pt}{133.995789pt}}
\pgfusepath{stroke}
\pgfpathmoveto{\pgfpoint{432.000000pt}{140.172638pt}}
\pgflineto{\pgfpoint{423.071960pt}{140.172638pt}}
\pgfusepath{stroke}
\pgfpathmoveto{\pgfpoint{432.000000pt}{146.349472pt}}
\pgflineto{\pgfpoint{423.085480pt}{146.349472pt}}
\pgfusepath{stroke}
\pgfpathmoveto{\pgfpoint{432.000000pt}{152.526306pt}}
\pgflineto{\pgfpoint{423.085480pt}{152.526306pt}}
\pgfusepath{stroke}
\pgfpathmoveto{\pgfpoint{432.000000pt}{158.703156pt}}
\pgflineto{\pgfpoint{423.085480pt}{158.703156pt}}
\pgfusepath{stroke}
\pgfpathmoveto{\pgfpoint{432.000000pt}{164.880005pt}}
\pgflineto{\pgfpoint{423.089996pt}{164.880005pt}}
\pgfusepath{stroke}
\pgfpathmoveto{\pgfpoint{432.000000pt}{171.056854pt}}
\pgflineto{\pgfpoint{423.090027pt}{171.056854pt}}
\pgfusepath{stroke}
\pgfpathmoveto{\pgfpoint{432.000000pt}{158.703156pt}}
\pgflineto{\pgfpoint{432.000000pt}{152.526306pt}}
\pgfusepath{stroke}
\pgfpathmoveto{\pgfpoint{432.000000pt}{133.995789pt}}
\pgflineto{\pgfpoint{432.000000pt}{127.818947pt}}
\pgfusepath{stroke}
\pgfpathmoveto{\pgfpoint{432.000000pt}{140.172638pt}}
\pgflineto{\pgfpoint{432.000000pt}{133.995789pt}}
\pgfusepath{stroke}
\pgfpathmoveto{\pgfpoint{432.000000pt}{146.349472pt}}
\pgflineto{\pgfpoint{432.000000pt}{140.172638pt}}
\pgfusepath{stroke}
\pgfpathmoveto{\pgfpoint{432.000000pt}{152.526306pt}}
\pgflineto{\pgfpoint{432.000000pt}{146.349472pt}}
\pgfusepath{stroke}
\pgfpathmoveto{\pgfpoint{432.000000pt}{164.880005pt}}
\pgflineto{\pgfpoint{432.000000pt}{158.703156pt}}
\pgfusepath{stroke}
\pgfpathmoveto{\pgfpoint{432.000000pt}{171.056854pt}}
\pgflineto{\pgfpoint{432.000000pt}{164.880005pt}}
\pgfusepath{stroke}
\pgfpathmoveto{\pgfpoint{432.000000pt}{133.995789pt}}
\pgflineto{\pgfpoint{432.013580pt}{133.995789pt}}
\pgfusepath{stroke}
\pgfpathmoveto{\pgfpoint{432.000000pt}{140.172638pt}}
\pgflineto{\pgfpoint{432.013611pt}{140.172638pt}}
\pgfusepath{stroke}
\pgfpathmoveto{\pgfpoint{432.000000pt}{146.349472pt}}
\pgflineto{\pgfpoint{432.018097pt}{146.349472pt}}
\pgfusepath{stroke}
\pgfpathmoveto{\pgfpoint{432.000000pt}{152.526306pt}}
\pgflineto{\pgfpoint{432.018097pt}{152.526306pt}}
\pgfusepath{stroke}
\pgfpathmoveto{\pgfpoint{432.000000pt}{158.703156pt}}
\pgflineto{\pgfpoint{432.022644pt}{158.703156pt}}
\pgfusepath{stroke}
\pgfpathmoveto{\pgfpoint{432.000000pt}{96.934731pt}}
\pgflineto{\pgfpoint{432.000000pt}{90.757896pt}}
\pgfusepath{stroke}
\pgfpathmoveto{\pgfpoint{432.000000pt}{53.696838pt}}
\pgflineto{\pgfpoint{432.000000pt}{47.519989pt}}
\pgfusepath{stroke}
\pgfpathmoveto{\pgfpoint{432.000000pt}{59.873672pt}}
\pgflineto{\pgfpoint{432.000000pt}{53.696838pt}}
\pgfusepath{stroke}
\pgfpathmoveto{\pgfpoint{432.000000pt}{66.050522pt}}
\pgflineto{\pgfpoint{432.000000pt}{59.873672pt}}
\pgfusepath{stroke}
\pgfpathmoveto{\pgfpoint{432.000000pt}{72.227356pt}}
\pgflineto{\pgfpoint{432.000000pt}{66.050522pt}}
\pgfusepath{stroke}
\pgfpathmoveto{\pgfpoint{432.000000pt}{78.404205pt}}
\pgflineto{\pgfpoint{432.000000pt}{72.227356pt}}
\pgfusepath{stroke}
\pgfpathmoveto{\pgfpoint{432.000000pt}{84.581039pt}}
\pgflineto{\pgfpoint{432.000000pt}{78.404205pt}}
\pgfusepath{stroke}
\pgfpathmoveto{\pgfpoint{432.000000pt}{90.757896pt}}
\pgflineto{\pgfpoint{432.000000pt}{84.581039pt}}
\pgfusepath{stroke}
\pgfpathmoveto{\pgfpoint{432.000000pt}{103.111580pt}}
\pgflineto{\pgfpoint{432.000000pt}{96.934731pt}}
\pgfusepath{stroke}
\pgfpathmoveto{\pgfpoint{432.000000pt}{109.288422pt}}
\pgflineto{\pgfpoint{432.000000pt}{103.111580pt}}
\pgfusepath{stroke}
\pgfpathmoveto{\pgfpoint{432.000000pt}{115.465263pt}}
\pgflineto{\pgfpoint{432.000000pt}{109.288422pt}}
\pgfusepath{stroke}
\pgfpathmoveto{\pgfpoint{432.000000pt}{121.642097pt}}
\pgflineto{\pgfpoint{432.000000pt}{115.465263pt}}
\pgfusepath{stroke}
\pgfpathmoveto{\pgfpoint{432.000000pt}{127.818947pt}}
\pgflineto{\pgfpoint{432.000000pt}{121.642097pt}}
\pgfusepath{stroke}
\pgfpathmoveto{\pgfpoint{440.927979pt}{47.519989pt}}
\pgflineto{\pgfpoint{432.000000pt}{47.519989pt}}
\pgfusepath{stroke}
\pgfpathmoveto{\pgfpoint{440.927979pt}{53.696838pt}}
\pgflineto{\pgfpoint{432.000000pt}{53.696838pt}}
\pgfusepath{stroke}
\pgfpathmoveto{\pgfpoint{432.000000pt}{59.873672pt}}
\pgflineto{\pgfpoint{440.914398pt}{59.873672pt}}
\pgfusepath{stroke}
\pgfpathmoveto{\pgfpoint{432.000000pt}{66.050522pt}}
\pgflineto{\pgfpoint{440.914398pt}{66.050522pt}}
\pgfusepath{stroke}
\pgfpathmoveto{\pgfpoint{440.927979pt}{72.227356pt}}
\pgflineto{\pgfpoint{432.000000pt}{72.227356pt}}
\pgfusepath{stroke}
\pgfpathmoveto{\pgfpoint{440.927979pt}{78.404205pt}}
\pgflineto{\pgfpoint{432.000000pt}{78.404205pt}}
\pgfusepath{stroke}
\pgfpathmoveto{\pgfpoint{440.927979pt}{84.581039pt}}
\pgflineto{\pgfpoint{432.000000pt}{84.581039pt}}
\pgfusepath{stroke}
\pgfpathmoveto{\pgfpoint{440.927979pt}{90.757896pt}}
\pgflineto{\pgfpoint{432.000000pt}{90.757896pt}}
\pgfusepath{stroke}
\pgfpathmoveto{\pgfpoint{440.927979pt}{96.934731pt}}
\pgflineto{\pgfpoint{432.000000pt}{96.934731pt}}
\pgfusepath{stroke}
\pgfpathmoveto{\pgfpoint{440.927979pt}{103.111580pt}}
\pgflineto{\pgfpoint{432.000000pt}{103.111580pt}}
\pgfusepath{stroke}
\pgfpathmoveto{\pgfpoint{440.927979pt}{109.288422pt}}
\pgflineto{\pgfpoint{432.000000pt}{109.288422pt}}
\pgfusepath{stroke}
\pgfpathmoveto{\pgfpoint{440.927979pt}{115.465263pt}}
\pgflineto{\pgfpoint{432.000000pt}{115.465263pt}}
\pgfusepath{stroke}
\pgfpathmoveto{\pgfpoint{440.927979pt}{121.642097pt}}
\pgflineto{\pgfpoint{432.000000pt}{121.642097pt}}
\pgfusepath{stroke}
\pgfpathmoveto{\pgfpoint{440.927979pt}{127.818947pt}}
\pgflineto{\pgfpoint{432.000000pt}{127.818947pt}}
\pgfusepath{stroke}
\pgfpathmoveto{\pgfpoint{440.927979pt}{133.995789pt}}
\pgflineto{\pgfpoint{432.013580pt}{133.995789pt}}
\pgfusepath{stroke}
\pgfpathmoveto{\pgfpoint{440.927979pt}{140.172638pt}}
\pgflineto{\pgfpoint{432.013611pt}{140.172638pt}}
\pgfusepath{stroke}
\pgfpathmoveto{\pgfpoint{440.927979pt}{146.349472pt}}
\pgflineto{\pgfpoint{432.018097pt}{146.349472pt}}
\pgfusepath{stroke}
\pgfpathmoveto{\pgfpoint{440.927979pt}{152.526306pt}}
\pgflineto{\pgfpoint{432.018097pt}{152.526306pt}}
\pgfusepath{stroke}
\pgfpathmoveto{\pgfpoint{440.927979pt}{158.703156pt}}
\pgflineto{\pgfpoint{432.022644pt}{158.703156pt}}
\pgfusepath{stroke}
\pgfpathmoveto{\pgfpoint{440.927979pt}{146.349472pt}}
\pgflineto{\pgfpoint{440.927979pt}{140.172638pt}}
\pgfusepath{stroke}
\pgfpathmoveto{\pgfpoint{440.927979pt}{152.526306pt}}
\pgflineto{\pgfpoint{440.927979pt}{146.349472pt}}
\pgfusepath{stroke}
\pgfpathmoveto{\pgfpoint{440.927979pt}{158.703156pt}}
\pgflineto{\pgfpoint{440.927979pt}{152.526306pt}}
\pgfusepath{stroke}
\pgfpathmoveto{\pgfpoint{440.927979pt}{146.349472pt}}
\pgflineto{\pgfpoint{440.941528pt}{146.349472pt}}
\pgfusepath{stroke}
\pgfpathmoveto{\pgfpoint{440.927979pt}{109.288422pt}}
\pgflineto{\pgfpoint{440.927979pt}{103.111580pt}}
\pgfusepath{stroke}
\pgfpathmoveto{\pgfpoint{440.927979pt}{53.696838pt}}
\pgflineto{\pgfpoint{440.927979pt}{47.519989pt}}
\pgfusepath{stroke}
\pgfpathmoveto{\pgfpoint{440.927979pt}{78.404205pt}}
\pgflineto{\pgfpoint{440.927979pt}{72.227356pt}}
\pgfusepath{stroke}
\pgfpathmoveto{\pgfpoint{440.927979pt}{84.581039pt}}
\pgflineto{\pgfpoint{440.927979pt}{78.404205pt}}
\pgfusepath{stroke}
\pgfpathmoveto{\pgfpoint{440.927979pt}{90.757896pt}}
\pgflineto{\pgfpoint{440.927979pt}{84.581039pt}}
\pgfusepath{stroke}
\pgfpathmoveto{\pgfpoint{440.927979pt}{96.934731pt}}
\pgflineto{\pgfpoint{440.927979pt}{90.757896pt}}
\pgfusepath{stroke}
\pgfpathmoveto{\pgfpoint{440.927979pt}{103.111580pt}}
\pgflineto{\pgfpoint{440.927979pt}{96.934731pt}}
\pgfusepath{stroke}
\pgfpathmoveto{\pgfpoint{440.927979pt}{115.465263pt}}
\pgflineto{\pgfpoint{440.927979pt}{109.288422pt}}
\pgfusepath{stroke}
\pgfpathmoveto{\pgfpoint{440.927979pt}{121.642097pt}}
\pgflineto{\pgfpoint{440.927979pt}{115.465263pt}}
\pgfusepath{stroke}
\pgfpathmoveto{\pgfpoint{440.927979pt}{127.818947pt}}
\pgflineto{\pgfpoint{440.927979pt}{121.642097pt}}
\pgfusepath{stroke}
\pgfpathmoveto{\pgfpoint{440.927979pt}{133.995789pt}}
\pgflineto{\pgfpoint{440.927979pt}{127.818947pt}}
\pgfusepath{stroke}
\pgfpathmoveto{\pgfpoint{440.927979pt}{140.172638pt}}
\pgflineto{\pgfpoint{440.927979pt}{133.995789pt}}
\pgfusepath{stroke}
\pgfpathmoveto{\pgfpoint{440.927979pt}{59.873672pt}}
\pgflineto{\pgfpoint{440.914398pt}{59.873672pt}}
\pgfusepath{stroke}
\pgfpathmoveto{\pgfpoint{440.927979pt}{66.050522pt}}
\pgflineto{\pgfpoint{440.914398pt}{66.050522pt}}
\pgfusepath{stroke}
\pgfpathmoveto{\pgfpoint{440.927979pt}{59.873672pt}}
\pgflineto{\pgfpoint{440.927979pt}{53.696838pt}}
\pgfusepath{stroke}
\pgfpathmoveto{\pgfpoint{440.927979pt}{47.519989pt}}
\pgflineto{\pgfpoint{440.946045pt}{47.519989pt}}
\pgfusepath{stroke}
\pgfpathmoveto{\pgfpoint{440.927979pt}{53.696838pt}}
\pgflineto{\pgfpoint{440.941528pt}{53.696838pt}}
\pgfusepath{stroke}
\pgfpathmoveto{\pgfpoint{440.927979pt}{72.227356pt}}
\pgflineto{\pgfpoint{440.927979pt}{66.050522pt}}
\pgfusepath{stroke}
\pgfpathmoveto{\pgfpoint{440.927979pt}{66.050522pt}}
\pgflineto{\pgfpoint{440.927979pt}{59.873672pt}}
\pgfusepath{stroke}
\pgfpathmoveto{\pgfpoint{449.855957pt}{47.519989pt}}
\pgflineto{\pgfpoint{440.946045pt}{47.519989pt}}
\pgfusepath{stroke}
\pgfpathmoveto{\pgfpoint{449.855957pt}{53.696838pt}}
\pgflineto{\pgfpoint{440.941528pt}{53.696838pt}}
\pgfusepath{stroke}
\pgfpathmoveto{\pgfpoint{449.855957pt}{59.873672pt}}
\pgflineto{\pgfpoint{440.927979pt}{59.873672pt}}
\pgfusepath{stroke}
\pgfpathmoveto{\pgfpoint{449.855957pt}{66.050522pt}}
\pgflineto{\pgfpoint{440.927979pt}{66.050522pt}}
\pgfusepath{stroke}
\pgfpathmoveto{\pgfpoint{449.855957pt}{72.227356pt}}
\pgflineto{\pgfpoint{440.927979pt}{72.227356pt}}
\pgfusepath{stroke}
\pgfpathmoveto{\pgfpoint{449.855957pt}{78.404205pt}}
\pgflineto{\pgfpoint{440.927979pt}{78.404205pt}}
\pgfusepath{stroke}
\pgfpathmoveto{\pgfpoint{449.855957pt}{84.581039pt}}
\pgflineto{\pgfpoint{440.927979pt}{84.581039pt}}
\pgfusepath{stroke}
\pgfpathmoveto{\pgfpoint{449.855957pt}{90.757896pt}}
\pgflineto{\pgfpoint{440.927979pt}{90.757896pt}}
\pgfusepath{stroke}
\pgfpathmoveto{\pgfpoint{449.855957pt}{96.934731pt}}
\pgflineto{\pgfpoint{440.927979pt}{96.934731pt}}
\pgfusepath{stroke}
\pgfpathmoveto{\pgfpoint{449.855957pt}{103.111580pt}}
\pgflineto{\pgfpoint{440.927979pt}{103.111580pt}}
\pgfusepath{stroke}
\pgfpathmoveto{\pgfpoint{449.855957pt}{109.288422pt}}
\pgflineto{\pgfpoint{440.927979pt}{109.288422pt}}
\pgfusepath{stroke}
\pgfpathmoveto{\pgfpoint{449.855957pt}{115.465263pt}}
\pgflineto{\pgfpoint{440.927979pt}{115.465263pt}}
\pgfusepath{stroke}
\pgfpathmoveto{\pgfpoint{449.855957pt}{121.642097pt}}
\pgflineto{\pgfpoint{440.927979pt}{121.642097pt}}
\pgfusepath{stroke}
\pgfpathmoveto{\pgfpoint{449.855957pt}{127.818947pt}}
\pgflineto{\pgfpoint{440.927979pt}{127.818947pt}}
\pgfusepath{stroke}
\pgfpathmoveto{\pgfpoint{449.855957pt}{133.995789pt}}
\pgflineto{\pgfpoint{440.927979pt}{133.995789pt}}
\pgfusepath{stroke}
\pgfpathmoveto{\pgfpoint{449.855957pt}{140.172638pt}}
\pgflineto{\pgfpoint{440.927979pt}{140.172638pt}}
\pgfusepath{stroke}
\pgfpathmoveto{\pgfpoint{449.855957pt}{146.349472pt}}
\pgflineto{\pgfpoint{440.941528pt}{146.349472pt}}
\pgfusepath{stroke}
\pgfpathmoveto{\pgfpoint{449.855957pt}{127.818947pt}}
\pgflineto{\pgfpoint{449.855957pt}{121.642097pt}}
\pgfusepath{stroke}
\pgfpathmoveto{\pgfpoint{449.855957pt}{133.995789pt}}
\pgflineto{\pgfpoint{449.855957pt}{127.818947pt}}
\pgfusepath{stroke}
\pgfpathmoveto{\pgfpoint{449.855957pt}{140.172638pt}}
\pgflineto{\pgfpoint{449.855957pt}{133.995789pt}}
\pgfusepath{stroke}
\pgfpathmoveto{\pgfpoint{449.855957pt}{146.349472pt}}
\pgflineto{\pgfpoint{449.855957pt}{140.172638pt}}
\pgfusepath{stroke}
\pgfpathmoveto{\pgfpoint{449.855957pt}{127.818947pt}}
\pgflineto{\pgfpoint{449.869385pt}{127.818947pt}}
\pgfusepath{stroke}
\pgfpathmoveto{\pgfpoint{449.855957pt}{133.995789pt}}
\pgflineto{\pgfpoint{449.869354pt}{133.995789pt}}
\pgfusepath{stroke}
\pgfpathmoveto{\pgfpoint{449.855957pt}{90.757896pt}}
\pgflineto{\pgfpoint{449.855957pt}{84.581039pt}}
\pgfusepath{stroke}
\pgfpathmoveto{\pgfpoint{449.855957pt}{53.696838pt}}
\pgflineto{\pgfpoint{449.855957pt}{47.519989pt}}
\pgfusepath{stroke}
\pgfpathmoveto{\pgfpoint{449.855957pt}{59.873672pt}}
\pgflineto{\pgfpoint{449.855957pt}{53.696838pt}}
\pgfusepath{stroke}
\pgfpathmoveto{\pgfpoint{449.855957pt}{66.050522pt}}
\pgflineto{\pgfpoint{449.855957pt}{59.873672pt}}
\pgfusepath{stroke}
\pgfpathmoveto{\pgfpoint{449.855957pt}{72.227356pt}}
\pgflineto{\pgfpoint{449.855957pt}{66.050522pt}}
\pgfusepath{stroke}
\pgfpathmoveto{\pgfpoint{449.855957pt}{78.404205pt}}
\pgflineto{\pgfpoint{449.855957pt}{72.227356pt}}
\pgfusepath{stroke}
\pgfpathmoveto{\pgfpoint{449.855957pt}{84.581039pt}}
\pgflineto{\pgfpoint{449.855957pt}{78.404205pt}}
\pgfusepath{stroke}
\pgfpathmoveto{\pgfpoint{449.855957pt}{96.934731pt}}
\pgflineto{\pgfpoint{449.855957pt}{90.757896pt}}
\pgfusepath{stroke}
\pgfpathmoveto{\pgfpoint{449.855957pt}{103.111580pt}}
\pgflineto{\pgfpoint{449.855957pt}{96.934731pt}}
\pgfusepath{stroke}
\pgfpathmoveto{\pgfpoint{449.855957pt}{109.288422pt}}
\pgflineto{\pgfpoint{449.855957pt}{103.111580pt}}
\pgfusepath{stroke}
\pgfpathmoveto{\pgfpoint{449.855957pt}{115.465263pt}}
\pgflineto{\pgfpoint{449.855957pt}{109.288422pt}}
\pgfusepath{stroke}
\pgfpathmoveto{\pgfpoint{449.855957pt}{121.642097pt}}
\pgflineto{\pgfpoint{449.855957pt}{115.465263pt}}
\pgfusepath{stroke}
\pgfpathmoveto{\pgfpoint{449.855957pt}{47.519989pt}}
\pgflineto{\pgfpoint{458.770447pt}{47.519989pt}}
\pgfusepath{stroke}
\pgfpathmoveto{\pgfpoint{458.783936pt}{53.696838pt}}
\pgflineto{\pgfpoint{449.855957pt}{53.696838pt}}
\pgfusepath{stroke}
\pgfpathmoveto{\pgfpoint{458.783936pt}{59.873672pt}}
\pgflineto{\pgfpoint{449.855957pt}{59.873672pt}}
\pgfusepath{stroke}
\pgfpathmoveto{\pgfpoint{458.783936pt}{66.050522pt}}
\pgflineto{\pgfpoint{449.855957pt}{66.050522pt}}
\pgfusepath{stroke}
\pgfpathmoveto{\pgfpoint{458.783936pt}{72.227356pt}}
\pgflineto{\pgfpoint{449.855957pt}{72.227356pt}}
\pgfusepath{stroke}
\pgfpathmoveto{\pgfpoint{458.783936pt}{78.404205pt}}
\pgflineto{\pgfpoint{449.855957pt}{78.404205pt}}
\pgfusepath{stroke}
\pgfpathmoveto{\pgfpoint{458.783936pt}{84.581039pt}}
\pgflineto{\pgfpoint{449.855957pt}{84.581039pt}}
\pgfusepath{stroke}
\pgfpathmoveto{\pgfpoint{458.783936pt}{90.757896pt}}
\pgflineto{\pgfpoint{449.855957pt}{90.757896pt}}
\pgfusepath{stroke}
\pgfpathmoveto{\pgfpoint{458.783936pt}{96.934731pt}}
\pgflineto{\pgfpoint{449.855957pt}{96.934731pt}}
\pgfusepath{stroke}
\pgfpathmoveto{\pgfpoint{458.783936pt}{103.111580pt}}
\pgflineto{\pgfpoint{449.855957pt}{103.111580pt}}
\pgfusepath{stroke}
\pgfpathmoveto{\pgfpoint{458.783936pt}{109.288422pt}}
\pgflineto{\pgfpoint{449.855957pt}{109.288422pt}}
\pgfusepath{stroke}
\pgfpathmoveto{\pgfpoint{458.783936pt}{115.465263pt}}
\pgflineto{\pgfpoint{449.855957pt}{115.465263pt}}
\pgfusepath{stroke}
\pgfpathmoveto{\pgfpoint{458.783936pt}{121.642097pt}}
\pgflineto{\pgfpoint{449.855957pt}{121.642097pt}}
\pgfusepath{stroke}
\pgfpathmoveto{\pgfpoint{458.783936pt}{127.818947pt}}
\pgflineto{\pgfpoint{449.869385pt}{127.818947pt}}
\pgfusepath{stroke}
\pgfpathmoveto{\pgfpoint{458.783936pt}{133.995789pt}}
\pgflineto{\pgfpoint{449.869354pt}{133.995789pt}}
\pgfusepath{stroke}
\pgfpathmoveto{\pgfpoint{458.783936pt}{127.818947pt}}
\pgflineto{\pgfpoint{458.783936pt}{121.642097pt}}
\pgfusepath{stroke}
\pgfpathmoveto{\pgfpoint{458.783936pt}{133.995789pt}}
\pgflineto{\pgfpoint{458.783936pt}{127.818947pt}}
\pgfusepath{stroke}
\pgfpathmoveto{\pgfpoint{458.783936pt}{90.757896pt}}
\pgflineto{\pgfpoint{458.783936pt}{84.581039pt}}
\pgfusepath{stroke}
\pgfpathmoveto{\pgfpoint{458.783936pt}{59.873672pt}}
\pgflineto{\pgfpoint{458.783936pt}{53.696838pt}}
\pgfusepath{stroke}
\pgfpathmoveto{\pgfpoint{458.783936pt}{66.050522pt}}
\pgflineto{\pgfpoint{458.783936pt}{59.873672pt}}
\pgfusepath{stroke}
\pgfpathmoveto{\pgfpoint{458.783936pt}{72.227356pt}}
\pgflineto{\pgfpoint{458.783936pt}{66.050522pt}}
\pgfusepath{stroke}
\pgfpathmoveto{\pgfpoint{458.783936pt}{78.404205pt}}
\pgflineto{\pgfpoint{458.783936pt}{72.227356pt}}
\pgfusepath{stroke}
\pgfpathmoveto{\pgfpoint{458.783936pt}{84.581039pt}}
\pgflineto{\pgfpoint{458.783936pt}{78.404205pt}}
\pgfusepath{stroke}
\pgfpathmoveto{\pgfpoint{458.783936pt}{96.934731pt}}
\pgflineto{\pgfpoint{458.783936pt}{90.757896pt}}
\pgfusepath{stroke}
\pgfpathmoveto{\pgfpoint{458.783936pt}{103.111580pt}}
\pgflineto{\pgfpoint{458.783936pt}{96.934731pt}}
\pgfusepath{stroke}
\pgfpathmoveto{\pgfpoint{458.783936pt}{109.288422pt}}
\pgflineto{\pgfpoint{458.783936pt}{103.111580pt}}
\pgfusepath{stroke}
\pgfpathmoveto{\pgfpoint{458.783936pt}{115.465263pt}}
\pgflineto{\pgfpoint{458.783936pt}{109.288422pt}}
\pgfusepath{stroke}
\pgfpathmoveto{\pgfpoint{458.783936pt}{121.642097pt}}
\pgflineto{\pgfpoint{458.783936pt}{115.465263pt}}
\pgfusepath{stroke}
\pgfpathmoveto{\pgfpoint{458.783936pt}{47.519989pt}}
\pgflineto{\pgfpoint{458.770447pt}{47.519989pt}}
\pgfusepath{stroke}
\pgfpathmoveto{\pgfpoint{458.783936pt}{53.696838pt}}
\pgflineto{\pgfpoint{458.783936pt}{47.519989pt}}
\pgfusepath{stroke}
\pgfpathmoveto{\pgfpoint{467.711975pt}{47.519989pt}}
\pgflineto{\pgfpoint{458.783936pt}{47.519989pt}}
\pgfusepath{stroke}
\pgfpathmoveto{\pgfpoint{467.711975pt}{53.696838pt}}
\pgflineto{\pgfpoint{458.783936pt}{53.696838pt}}
\pgfusepath{stroke}
\pgfpathmoveto{\pgfpoint{467.711975pt}{59.873672pt}}
\pgflineto{\pgfpoint{458.783936pt}{59.873672pt}}
\pgfusepath{stroke}
\pgfpathmoveto{\pgfpoint{467.711975pt}{66.050522pt}}
\pgflineto{\pgfpoint{458.783936pt}{66.050522pt}}
\pgfusepath{stroke}
\pgfpathmoveto{\pgfpoint{467.711975pt}{72.227356pt}}
\pgflineto{\pgfpoint{458.783936pt}{72.227356pt}}
\pgfusepath{stroke}
\pgfpathmoveto{\pgfpoint{467.711975pt}{78.404205pt}}
\pgflineto{\pgfpoint{458.783936pt}{78.404205pt}}
\pgfusepath{stroke}
\pgfpathmoveto{\pgfpoint{467.711975pt}{84.581039pt}}
\pgflineto{\pgfpoint{458.783936pt}{84.581039pt}}
\pgfusepath{stroke}
\pgfpathmoveto{\pgfpoint{467.711975pt}{90.757896pt}}
\pgflineto{\pgfpoint{458.783936pt}{90.757896pt}}
\pgfusepath{stroke}
\pgfpathmoveto{\pgfpoint{467.711975pt}{96.934731pt}}
\pgflineto{\pgfpoint{458.783936pt}{96.934731pt}}
\pgfusepath{stroke}
\pgfpathmoveto{\pgfpoint{467.711975pt}{103.111580pt}}
\pgflineto{\pgfpoint{458.783936pt}{103.111580pt}}
\pgfusepath{stroke}
\pgfpathmoveto{\pgfpoint{467.711975pt}{109.288422pt}}
\pgflineto{\pgfpoint{458.783936pt}{109.288422pt}}
\pgfusepath{stroke}
\pgfpathmoveto{\pgfpoint{467.711975pt}{115.465263pt}}
\pgflineto{\pgfpoint{458.783936pt}{115.465263pt}}
\pgfusepath{stroke}
\pgfpathmoveto{\pgfpoint{467.711975pt}{121.642097pt}}
\pgflineto{\pgfpoint{458.783936pt}{121.642097pt}}
\pgfusepath{stroke}
\pgfpathmoveto{\pgfpoint{467.711975pt}{109.288422pt}}
\pgflineto{\pgfpoint{467.711975pt}{103.111580pt}}
\pgfusepath{stroke}
\pgfpathmoveto{\pgfpoint{467.711975pt}{115.465263pt}}
\pgflineto{\pgfpoint{467.711975pt}{109.288422pt}}
\pgfusepath{stroke}
\pgfpathmoveto{\pgfpoint{467.711975pt}{121.642097pt}}
\pgflineto{\pgfpoint{467.711975pt}{115.465263pt}}
\pgfusepath{stroke}
\pgfpathmoveto{\pgfpoint{467.711975pt}{109.288422pt}}
\pgflineto{\pgfpoint{467.725555pt}{109.288422pt}}
\pgfusepath{stroke}
\pgfpathmoveto{\pgfpoint{467.711975pt}{72.227356pt}}
\pgflineto{\pgfpoint{467.711975pt}{66.050522pt}}
\pgfusepath{stroke}
\pgfpathmoveto{\pgfpoint{467.711975pt}{53.696838pt}}
\pgflineto{\pgfpoint{467.711975pt}{47.519989pt}}
\pgfusepath{stroke}
\pgfpathmoveto{\pgfpoint{467.711975pt}{59.873672pt}}
\pgflineto{\pgfpoint{467.711975pt}{53.696838pt}}
\pgfusepath{stroke}
\pgfpathmoveto{\pgfpoint{467.711975pt}{66.050522pt}}
\pgflineto{\pgfpoint{467.711975pt}{59.873672pt}}
\pgfusepath{stroke}
\pgfpathmoveto{\pgfpoint{467.711975pt}{78.404205pt}}
\pgflineto{\pgfpoint{467.711975pt}{72.227356pt}}
\pgfusepath{stroke}
\pgfpathmoveto{\pgfpoint{467.711975pt}{84.581039pt}}
\pgflineto{\pgfpoint{467.711975pt}{78.404205pt}}
\pgfusepath{stroke}
\pgfpathmoveto{\pgfpoint{467.711975pt}{90.757896pt}}
\pgflineto{\pgfpoint{467.711975pt}{84.581039pt}}
\pgfusepath{stroke}
\pgfpathmoveto{\pgfpoint{467.711975pt}{96.934731pt}}
\pgflineto{\pgfpoint{467.711975pt}{90.757896pt}}
\pgfusepath{stroke}
\pgfpathmoveto{\pgfpoint{467.711975pt}{103.111580pt}}
\pgflineto{\pgfpoint{467.711975pt}{96.934731pt}}
\pgfusepath{stroke}
\pgfpathmoveto{\pgfpoint{476.639954pt}{47.519989pt}}
\pgflineto{\pgfpoint{467.711975pt}{47.519989pt}}
\pgfusepath{stroke}
\pgfpathmoveto{\pgfpoint{476.639954pt}{53.696838pt}}
\pgflineto{\pgfpoint{467.711975pt}{53.696838pt}}
\pgfusepath{stroke}
\pgfpathmoveto{\pgfpoint{476.639954pt}{59.873672pt}}
\pgflineto{\pgfpoint{467.711975pt}{59.873672pt}}
\pgfusepath{stroke}
\pgfpathmoveto{\pgfpoint{476.639954pt}{66.050522pt}}
\pgflineto{\pgfpoint{467.711975pt}{66.050522pt}}
\pgfusepath{stroke}
\pgfpathmoveto{\pgfpoint{476.639954pt}{72.227356pt}}
\pgflineto{\pgfpoint{467.711975pt}{72.227356pt}}
\pgfusepath{stroke}
\pgfpathmoveto{\pgfpoint{476.639954pt}{78.404205pt}}
\pgflineto{\pgfpoint{467.711975pt}{78.404205pt}}
\pgfusepath{stroke}
\pgfpathmoveto{\pgfpoint{476.639954pt}{84.581039pt}}
\pgflineto{\pgfpoint{467.711975pt}{84.581039pt}}
\pgfusepath{stroke}
\pgfpathmoveto{\pgfpoint{476.639954pt}{90.757896pt}}
\pgflineto{\pgfpoint{467.711975pt}{90.757896pt}}
\pgfusepath{stroke}
\pgfpathmoveto{\pgfpoint{476.639954pt}{96.934731pt}}
\pgflineto{\pgfpoint{467.711975pt}{96.934731pt}}
\pgfusepath{stroke}
\pgfpathmoveto{\pgfpoint{476.639954pt}{103.111580pt}}
\pgflineto{\pgfpoint{467.711975pt}{103.111580pt}}
\pgfusepath{stroke}
\pgfpathmoveto{\pgfpoint{476.639954pt}{109.288422pt}}
\pgflineto{\pgfpoint{467.725555pt}{109.288422pt}}
\pgfusepath{stroke}
\pgfpathmoveto{\pgfpoint{476.639954pt}{109.288422pt}}
\pgflineto{\pgfpoint{476.639954pt}{103.111580pt}}
\pgfusepath{stroke}
\pgfpathmoveto{\pgfpoint{476.639954pt}{72.227356pt}}
\pgflineto{\pgfpoint{476.639954pt}{66.050522pt}}
\pgfusepath{stroke}
\pgfpathmoveto{\pgfpoint{476.639954pt}{53.696838pt}}
\pgflineto{\pgfpoint{476.639954pt}{47.519989pt}}
\pgfusepath{stroke}
\pgfpathmoveto{\pgfpoint{476.639954pt}{59.873672pt}}
\pgflineto{\pgfpoint{476.639954pt}{53.696838pt}}
\pgfusepath{stroke}
\pgfpathmoveto{\pgfpoint{476.639954pt}{66.050522pt}}
\pgflineto{\pgfpoint{476.639954pt}{59.873672pt}}
\pgfusepath{stroke}
\pgfpathmoveto{\pgfpoint{476.639954pt}{78.404205pt}}
\pgflineto{\pgfpoint{476.639954pt}{72.227356pt}}
\pgfusepath{stroke}
\pgfpathmoveto{\pgfpoint{476.639954pt}{84.581039pt}}
\pgflineto{\pgfpoint{476.639954pt}{78.404205pt}}
\pgfusepath{stroke}
\pgfpathmoveto{\pgfpoint{476.639954pt}{90.757896pt}}
\pgflineto{\pgfpoint{476.639954pt}{84.581039pt}}
\pgfusepath{stroke}
\pgfpathmoveto{\pgfpoint{476.639954pt}{96.934731pt}}
\pgflineto{\pgfpoint{476.639954pt}{90.757896pt}}
\pgfusepath{stroke}
\pgfpathmoveto{\pgfpoint{476.639954pt}{103.111580pt}}
\pgflineto{\pgfpoint{476.639954pt}{96.934731pt}}
\pgfusepath{stroke}
\pgfpathmoveto{\pgfpoint{485.567963pt}{47.519989pt}}
\pgflineto{\pgfpoint{476.639954pt}{47.519989pt}}
\pgfusepath{stroke}
\pgfpathmoveto{\pgfpoint{485.567963pt}{53.696838pt}}
\pgflineto{\pgfpoint{476.639954pt}{53.696838pt}}
\pgfusepath{stroke}
\pgfpathmoveto{\pgfpoint{485.567963pt}{59.873672pt}}
\pgflineto{\pgfpoint{476.639954pt}{59.873672pt}}
\pgfusepath{stroke}
\pgfpathmoveto{\pgfpoint{485.567963pt}{66.050522pt}}
\pgflineto{\pgfpoint{476.639954pt}{66.050522pt}}
\pgfusepath{stroke}
\pgfpathmoveto{\pgfpoint{485.567963pt}{72.227356pt}}
\pgflineto{\pgfpoint{476.639954pt}{72.227356pt}}
\pgfusepath{stroke}
\pgfpathmoveto{\pgfpoint{485.567963pt}{78.404205pt}}
\pgflineto{\pgfpoint{476.639954pt}{78.404205pt}}
\pgfusepath{stroke}
\pgfpathmoveto{\pgfpoint{485.567963pt}{84.581039pt}}
\pgflineto{\pgfpoint{476.639954pt}{84.581039pt}}
\pgfusepath{stroke}
\pgfpathmoveto{\pgfpoint{485.567963pt}{90.757896pt}}
\pgflineto{\pgfpoint{476.639954pt}{90.757896pt}}
\pgfusepath{stroke}
\pgfpathmoveto{\pgfpoint{485.567963pt}{96.934731pt}}
\pgflineto{\pgfpoint{476.639954pt}{96.934731pt}}
\pgfusepath{stroke}
\pgfpathmoveto{\pgfpoint{485.567963pt}{90.757896pt}}
\pgflineto{\pgfpoint{485.567963pt}{84.581039pt}}
\pgfusepath{stroke}
\pgfpathmoveto{\pgfpoint{485.567963pt}{96.934731pt}}
\pgflineto{\pgfpoint{485.567963pt}{90.757896pt}}
\pgfusepath{stroke}
\pgfpathmoveto{\pgfpoint{485.567963pt}{53.696838pt}}
\pgflineto{\pgfpoint{485.567963pt}{47.519989pt}}
\pgfusepath{stroke}
\pgfpathmoveto{\pgfpoint{485.567963pt}{59.873672pt}}
\pgflineto{\pgfpoint{485.567963pt}{53.696838pt}}
\pgfusepath{stroke}
\pgfpathmoveto{\pgfpoint{485.567963pt}{66.050522pt}}
\pgflineto{\pgfpoint{485.567963pt}{59.873672pt}}
\pgfusepath{stroke}
\pgfpathmoveto{\pgfpoint{485.567963pt}{72.227356pt}}
\pgflineto{\pgfpoint{485.567963pt}{66.050522pt}}
\pgfusepath{stroke}
\pgfpathmoveto{\pgfpoint{485.567963pt}{78.404205pt}}
\pgflineto{\pgfpoint{485.567963pt}{72.227356pt}}
\pgfusepath{stroke}
\pgfpathmoveto{\pgfpoint{485.567963pt}{84.581039pt}}
\pgflineto{\pgfpoint{485.567963pt}{78.404205pt}}
\pgfusepath{stroke}
\pgfpathmoveto{\pgfpoint{494.495972pt}{47.519989pt}}
\pgflineto{\pgfpoint{485.567963pt}{47.519989pt}}
\pgfusepath{stroke}
\pgfpathmoveto{\pgfpoint{494.495972pt}{53.696838pt}}
\pgflineto{\pgfpoint{485.567963pt}{53.696838pt}}
\pgfusepath{stroke}
\pgfpathmoveto{\pgfpoint{494.495972pt}{59.873672pt}}
\pgflineto{\pgfpoint{485.567963pt}{59.873672pt}}
\pgfusepath{stroke}
\pgfpathmoveto{\pgfpoint{494.495972pt}{66.050522pt}}
\pgflineto{\pgfpoint{485.567963pt}{66.050522pt}}
\pgfusepath{stroke}
\pgfpathmoveto{\pgfpoint{494.495972pt}{72.227356pt}}
\pgflineto{\pgfpoint{485.567963pt}{72.227356pt}}
\pgfusepath{stroke}
\pgfpathmoveto{\pgfpoint{494.495972pt}{78.404205pt}}
\pgflineto{\pgfpoint{485.567963pt}{78.404205pt}}
\pgfusepath{stroke}
\pgfpathmoveto{\pgfpoint{494.495972pt}{84.581039pt}}
\pgflineto{\pgfpoint{485.567963pt}{84.581039pt}}
\pgfusepath{stroke}
\pgfpathmoveto{\pgfpoint{494.495972pt}{53.696838pt}}
\pgflineto{\pgfpoint{494.495972pt}{47.519989pt}}
\pgfusepath{stroke}
\pgfpathmoveto{\pgfpoint{494.495972pt}{59.873672pt}}
\pgflineto{\pgfpoint{494.495972pt}{53.696838pt}}
\pgfusepath{stroke}
\pgfpathmoveto{\pgfpoint{494.495972pt}{66.050522pt}}
\pgflineto{\pgfpoint{494.495972pt}{59.873672pt}}
\pgfusepath{stroke}
\pgfpathmoveto{\pgfpoint{494.495972pt}{72.227356pt}}
\pgflineto{\pgfpoint{494.495972pt}{66.050522pt}}
\pgfusepath{stroke}
\pgfpathmoveto{\pgfpoint{494.495972pt}{78.404205pt}}
\pgflineto{\pgfpoint{494.495972pt}{72.227356pt}}
\pgfusepath{stroke}
\pgfpathmoveto{\pgfpoint{494.495972pt}{84.581039pt}}
\pgflineto{\pgfpoint{494.495972pt}{78.404205pt}}
\pgfusepath{stroke}
\pgfpathmoveto{\pgfpoint{503.423981pt}{47.519989pt}}
\pgflineto{\pgfpoint{494.495972pt}{47.519989pt}}
\pgfusepath{stroke}
\pgfpathmoveto{\pgfpoint{503.423981pt}{53.696838pt}}
\pgflineto{\pgfpoint{494.495972pt}{53.696838pt}}
\pgfusepath{stroke}
\pgfpathmoveto{\pgfpoint{503.423981pt}{59.873672pt}}
\pgflineto{\pgfpoint{494.495972pt}{59.873672pt}}
\pgfusepath{stroke}
\pgfpathmoveto{\pgfpoint{503.423981pt}{66.050522pt}}
\pgflineto{\pgfpoint{494.495972pt}{66.050522pt}}
\pgfusepath{stroke}
\pgfpathmoveto{\pgfpoint{503.423981pt}{72.227356pt}}
\pgflineto{\pgfpoint{494.495972pt}{72.227356pt}}
\pgfusepath{stroke}
\pgfpathmoveto{\pgfpoint{503.423981pt}{78.404205pt}}
\pgflineto{\pgfpoint{494.495972pt}{78.404205pt}}
\pgfusepath{stroke}
\pgfpathmoveto{\pgfpoint{503.423981pt}{53.696838pt}}
\pgflineto{\pgfpoint{503.423981pt}{47.519989pt}}
\pgfusepath{stroke}
\pgfpathmoveto{\pgfpoint{503.423981pt}{59.873672pt}}
\pgflineto{\pgfpoint{503.423981pt}{53.696838pt}}
\pgfusepath{stroke}
\pgfpathmoveto{\pgfpoint{503.423981pt}{66.050522pt}}
\pgflineto{\pgfpoint{503.423981pt}{59.873672pt}}
\pgfusepath{stroke}
\pgfpathmoveto{\pgfpoint{503.423981pt}{72.227356pt}}
\pgflineto{\pgfpoint{503.423981pt}{66.050522pt}}
\pgfusepath{stroke}
\pgfpathmoveto{\pgfpoint{503.423981pt}{78.404205pt}}
\pgflineto{\pgfpoint{503.423981pt}{72.227356pt}}
\pgfusepath{stroke}
\pgfpathmoveto{\pgfpoint{512.351990pt}{47.519989pt}}
\pgflineto{\pgfpoint{503.423981pt}{47.519989pt}}
\pgfusepath{stroke}
\pgfpathmoveto{\pgfpoint{512.351990pt}{53.696838pt}}
\pgflineto{\pgfpoint{503.423981pt}{53.696838pt}}
\pgfusepath{stroke}
\pgfpathmoveto{\pgfpoint{512.351990pt}{59.873672pt}}
\pgflineto{\pgfpoint{503.423981pt}{59.873672pt}}
\pgfusepath{stroke}
\pgfpathmoveto{\pgfpoint{512.351990pt}{66.050522pt}}
\pgflineto{\pgfpoint{503.423981pt}{66.050522pt}}
\pgfusepath{stroke}
\pgfpathmoveto{\pgfpoint{512.351990pt}{53.696838pt}}
\pgflineto{\pgfpoint{512.351990pt}{47.519989pt}}
\pgfusepath{stroke}
\pgfpathmoveto{\pgfpoint{512.351990pt}{59.873672pt}}
\pgflineto{\pgfpoint{512.351990pt}{53.696838pt}}
\pgfusepath{stroke}
\pgfpathmoveto{\pgfpoint{512.351990pt}{66.050522pt}}
\pgflineto{\pgfpoint{512.351990pt}{59.873672pt}}
\pgfusepath{stroke}
\pgfpathmoveto{\pgfpoint{521.279968pt}{47.519989pt}}
\pgflineto{\pgfpoint{512.351990pt}{47.519989pt}}
\pgfusepath{stroke}
\pgfpathmoveto{\pgfpoint{521.279968pt}{53.696838pt}}
\pgflineto{\pgfpoint{512.351990pt}{53.696838pt}}
\pgfusepath{stroke}
\pgfpathmoveto{\pgfpoint{521.279968pt}{53.696838pt}}
\pgflineto{\pgfpoint{521.279968pt}{47.519989pt}}
\pgfusepath{stroke}
\end{pgfscope}
\end{pgfpicture}
}
\end{enumerate}
\end{document}
