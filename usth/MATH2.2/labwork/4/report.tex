\documentclass[a4paper,12pt]{article}
\usepackage[english,vietnamese]{babel}
\usepackage{amsmath}
\usepackage{booktabs}
\usepackage{enumerate}
\usepackage{lmodern}
\usepackage{tikz}

\newcommand{\exercise}[1]{\noindent\textbf{#1.}}

\title{Numerical Methods: Labwork 4 Report}
\author{Nguyễn Gia Phong--BI9-184}
\date{\dateenglish\today}

\begin{document}
\maketitle
\section{Curve Fitting Problems}
\exercise{3}  From the given table, we define the two vectors
\begin{verbatim}
octave> x = [0.00000 0.78540 1.57080 2.35620 ...
>            3.14159 3.92699 4.71239 5.49779 6.28319];
octave> fx = [0.00000 0.70711 1.00000 0.70711 ...
>             0.00000 -0.70711 -1.00000 -0.70711 0.00000];
\end{verbatim}

\begin{enumerate}[(a)]
  \item \verb|f(3.00000)| and \verb|f(4.50000)| can be interpolated by
\begin{verbatim}
octave> points = [3.00000 4.50000];
octave> linear = interp1 (x, fx, points)
linear =
   0.12748  -0.92080
\end{verbatim}
    To further illustrate this, we can then plot these point along
    with the linearly interpolated line:
    \verb|plot (points, linear, "o", x, fx)|
    \begin{figure}[!h]
      \centering
      \scalebox{0.36}{% Title: gl2ps_renderer figure
% Creator: GL2PS 1.4.0, (C) 1999-2017 C. Geuzaine
% For: Octave
% CreationDate: Fri Oct 25 16:15:42 2019
\begin{pgfpicture}
\color[rgb]{1.000000,1.000000,1.000000}
\pgfpathrectanglecorners{\pgfpoint{0pt}{0pt}}{\pgfpoint{576pt}{432pt}}
\pgfusepath{fill}
\begin{pgfscope}
\pgfpathrectangle{\pgfpoint{0pt}{0pt}}{\pgfpoint{576pt}{432pt}}
\pgfusepath{fill}
\pgfpathrectangle{\pgfpoint{0pt}{0pt}}{\pgfpoint{576pt}{432pt}}
\pgfusepath{clip}
\pgfpathmoveto{\pgfpoint{74.880005pt}{399.599976pt}}
\pgflineto{\pgfpoint{521.279968pt}{47.519989pt}}
\pgflineto{\pgfpoint{74.880005pt}{47.519989pt}}
\pgfpathclose
\pgfusepath{fill,stroke}
\pgfpathmoveto{\pgfpoint{74.880005pt}{399.599976pt}}
\pgflineto{\pgfpoint{521.279968pt}{399.599976pt}}
\pgflineto{\pgfpoint{521.279968pt}{47.519989pt}}
\pgfpathclose
\pgfusepath{fill,stroke}
\color[rgb]{0.150000,0.150000,0.150000}
\pgfsetlinewidth{0.500000pt}
\pgfpathmoveto{\pgfpoint{74.880005pt}{51.985016pt}}
\pgflineto{\pgfpoint{74.880005pt}{47.519989pt}}
\pgfusepath{stroke}
\pgfpathmoveto{\pgfpoint{74.880005pt}{395.134979pt}}
\pgflineto{\pgfpoint{74.880005pt}{399.599976pt}}
\pgfusepath{stroke}
\pgfpathmoveto{\pgfpoint{138.651428pt}{51.985016pt}}
\pgflineto{\pgfpoint{138.651428pt}{47.519989pt}}
\pgfusepath{stroke}
\pgfpathmoveto{\pgfpoint{138.651428pt}{395.134979pt}}
\pgflineto{\pgfpoint{138.651428pt}{399.599976pt}}
\pgfusepath{stroke}
\pgfpathmoveto{\pgfpoint{202.422852pt}{51.985016pt}}
\pgflineto{\pgfpoint{202.422852pt}{47.519989pt}}
\pgfusepath{stroke}
\pgfpathmoveto{\pgfpoint{202.422852pt}{395.134979pt}}
\pgflineto{\pgfpoint{202.422852pt}{399.599976pt}}
\pgfusepath{stroke}
\pgfpathmoveto{\pgfpoint{266.194275pt}{51.985016pt}}
\pgflineto{\pgfpoint{266.194275pt}{47.519989pt}}
\pgfusepath{stroke}
\pgfpathmoveto{\pgfpoint{266.194275pt}{395.134979pt}}
\pgflineto{\pgfpoint{266.194275pt}{399.599976pt}}
\pgfusepath{stroke}
\pgfpathmoveto{\pgfpoint{329.965698pt}{51.985016pt}}
\pgflineto{\pgfpoint{329.965698pt}{47.519989pt}}
\pgfusepath{stroke}
\pgfpathmoveto{\pgfpoint{329.965698pt}{395.134979pt}}
\pgflineto{\pgfpoint{329.965698pt}{399.599976pt}}
\pgfusepath{stroke}
\pgfpathmoveto{\pgfpoint{393.737152pt}{51.985016pt}}
\pgflineto{\pgfpoint{393.737152pt}{47.519989pt}}
\pgfusepath{stroke}
\pgfpathmoveto{\pgfpoint{393.737152pt}{395.134979pt}}
\pgflineto{\pgfpoint{393.737152pt}{399.599976pt}}
\pgfusepath{stroke}
\pgfpathmoveto{\pgfpoint{457.508575pt}{51.985016pt}}
\pgflineto{\pgfpoint{457.508575pt}{47.519989pt}}
\pgfusepath{stroke}
\pgfpathmoveto{\pgfpoint{457.508575pt}{395.134979pt}}
\pgflineto{\pgfpoint{457.508575pt}{399.599976pt}}
\pgfusepath{stroke}
\pgfpathmoveto{\pgfpoint{521.279968pt}{51.985016pt}}
\pgflineto{\pgfpoint{521.279968pt}{47.519989pt}}
\pgfusepath{stroke}
\pgfpathmoveto{\pgfpoint{521.279968pt}{395.134979pt}}
\pgflineto{\pgfpoint{521.279968pt}{399.599976pt}}
\pgfusepath{stroke}
{
\pgftransformshift{\pgfpoint{74.880005pt}{40.018295pt}}
\pgfnode{rectangle}{north}{\fontsize{10}{0}\selectfont\textcolor[rgb]{0.15,0.15,0.15}{{0}}}{}{\pgfusepath{discard}}}
{
\pgftransformshift{\pgfpoint{138.651428pt}{40.018295pt}}
\pgfnode{rectangle}{north}{\fontsize{10}{0}\selectfont\textcolor[rgb]{0.15,0.15,0.15}{{1}}}{}{\pgfusepath{discard}}}
{
\pgftransformshift{\pgfpoint{202.422852pt}{40.018295pt}}
\pgfnode{rectangle}{north}{\fontsize{10}{0}\selectfont\textcolor[rgb]{0.15,0.15,0.15}{{2}}}{}{\pgfusepath{discard}}}
{
\pgftransformshift{\pgfpoint{266.194305pt}{40.018295pt}}
\pgfnode{rectangle}{north}{\fontsize{10}{0}\selectfont\textcolor[rgb]{0.15,0.15,0.15}{{3}}}{}{\pgfusepath{discard}}}
{
\pgftransformshift{\pgfpoint{329.965698pt}{40.018295pt}}
\pgfnode{rectangle}{north}{\fontsize{10}{0}\selectfont\textcolor[rgb]{0.15,0.15,0.15}{{4}}}{}{\pgfusepath{discard}}}
{
\pgftransformshift{\pgfpoint{393.737122pt}{40.018295pt}}
\pgfnode{rectangle}{north}{\fontsize{10}{0}\selectfont\textcolor[rgb]{0.15,0.15,0.15}{{5}}}{}{\pgfusepath{discard}}}
{
\pgftransformshift{\pgfpoint{457.508606pt}{40.018295pt}}
\pgfnode{rectangle}{north}{\fontsize{10}{0}\selectfont\textcolor[rgb]{0.15,0.15,0.15}{{6}}}{}{\pgfusepath{discard}}}
{
\pgftransformshift{\pgfpoint{521.279968pt}{40.018295pt}}
\pgfnode{rectangle}{north}{\fontsize{10}{0}\selectfont\textcolor[rgb]{0.15,0.15,0.15}{{7}}}{}{\pgfusepath{discard}}}
\pgfpathmoveto{\pgfpoint{79.347992pt}{47.519989pt}}
\pgflineto{\pgfpoint{74.880005pt}{47.519989pt}}
\pgfusepath{stroke}
\pgfpathmoveto{\pgfpoint{516.812012pt}{47.519989pt}}
\pgflineto{\pgfpoint{521.279968pt}{47.519989pt}}
\pgfusepath{stroke}
\pgfpathmoveto{\pgfpoint{79.347992pt}{135.539993pt}}
\pgflineto{\pgfpoint{74.880005pt}{135.539993pt}}
\pgfusepath{stroke}
\pgfpathmoveto{\pgfpoint{516.812012pt}{135.539993pt}}
\pgflineto{\pgfpoint{521.279968pt}{135.539993pt}}
\pgfusepath{stroke}
\pgfpathmoveto{\pgfpoint{79.347992pt}{223.559998pt}}
\pgflineto{\pgfpoint{74.880005pt}{223.559998pt}}
\pgfusepath{stroke}
\pgfpathmoveto{\pgfpoint{516.812012pt}{223.559998pt}}
\pgflineto{\pgfpoint{521.279968pt}{223.559998pt}}
\pgfusepath{stroke}
\pgfpathmoveto{\pgfpoint{79.347992pt}{311.579987pt}}
\pgflineto{\pgfpoint{74.880005pt}{311.579987pt}}
\pgfusepath{stroke}
\pgfpathmoveto{\pgfpoint{516.812012pt}{311.579987pt}}
\pgflineto{\pgfpoint{521.279968pt}{311.579987pt}}
\pgfusepath{stroke}
\pgfpathmoveto{\pgfpoint{79.347992pt}{399.599976pt}}
\pgflineto{\pgfpoint{74.880005pt}{399.599976pt}}
\pgfusepath{stroke}
\pgfpathmoveto{\pgfpoint{516.812012pt}{399.599976pt}}
\pgflineto{\pgfpoint{521.279968pt}{399.599976pt}}
\pgfusepath{stroke}
{
\pgftransformshift{\pgfpoint{69.875519pt}{47.519989pt}}
\pgfnode{rectangle}{east}{\fontsize{10}{0}\selectfont\textcolor[rgb]{0.15,0.15,0.15}{{-1}}}{}{\pgfusepath{discard}}}
{
\pgftransformshift{\pgfpoint{69.875519pt}{135.539993pt}}
\pgfnode{rectangle}{east}{\fontsize{10}{0}\selectfont\textcolor[rgb]{0.15,0.15,0.15}{{-0.5}}}{}{\pgfusepath{discard}}}
{
\pgftransformshift{\pgfpoint{69.875519pt}{223.559998pt}}
\pgfnode{rectangle}{east}{\fontsize{10}{0}\selectfont\textcolor[rgb]{0.15,0.15,0.15}{{0}}}{}{\pgfusepath{discard}}}
{
\pgftransformshift{\pgfpoint{69.875519pt}{311.579987pt}}
\pgfnode{rectangle}{east}{\fontsize{10}{0}\selectfont\textcolor[rgb]{0.15,0.15,0.15}{{0.5}}}{}{\pgfusepath{discard}}}
{
\pgftransformshift{\pgfpoint{69.875519pt}{399.599976pt}}
\pgfnode{rectangle}{east}{\fontsize{10}{0}\selectfont\textcolor[rgb]{0.15,0.15,0.15}{{1}}}{}{\pgfusepath{discard}}}
\pgfsetrectcap
\pgfsetdash{{16pt}{0pt}}{0pt}
\pgfpathmoveto{\pgfpoint{521.279968pt}{47.519989pt}}
\pgflineto{\pgfpoint{74.880005pt}{47.519989pt}}
\pgfusepath{stroke}
\pgfpathmoveto{\pgfpoint{521.279968pt}{399.599976pt}}
\pgflineto{\pgfpoint{74.880005pt}{399.599976pt}}
\pgfusepath{stroke}
\pgfpathmoveto{\pgfpoint{74.880005pt}{399.599976pt}}
\pgflineto{\pgfpoint{74.880005pt}{47.519989pt}}
\pgfusepath{stroke}
\pgfpathmoveto{\pgfpoint{521.279968pt}{399.599976pt}}
\pgflineto{\pgfpoint{521.279968pt}{47.519989pt}}
\pgfusepath{stroke}
\color[rgb]{0.000000,0.447000,0.741000}
\pgfsetroundcap
\pgfsetroundjoin
\pgfsetdash{}{0pt}
\pgfpathmoveto{\pgfpoint{268.621338pt}{244.237823pt}}
\pgflineto{\pgfpoint{269.194275pt}{246.001175pt}}
\pgfusepath{stroke}
\pgfpathmoveto{\pgfpoint{267.121338pt}{243.148010pt}}
\pgflineto{\pgfpoint{268.621338pt}{244.237823pt}}
\pgfusepath{stroke}
\pgfpathmoveto{\pgfpoint{265.267242pt}{243.148010pt}}
\pgflineto{\pgfpoint{267.121338pt}{243.148010pt}}
\pgfusepath{stroke}
\pgfpathmoveto{\pgfpoint{263.767242pt}{244.237823pt}}
\pgflineto{\pgfpoint{265.267242pt}{243.148010pt}}
\pgfusepath{stroke}
\pgfpathmoveto{\pgfpoint{263.194275pt}{246.001175pt}}
\pgflineto{\pgfpoint{263.767242pt}{244.237823pt}}
\pgfusepath{stroke}
\pgfpathmoveto{\pgfpoint{263.767242pt}{247.764526pt}}
\pgflineto{\pgfpoint{263.194275pt}{246.001175pt}}
\pgfusepath{stroke}
\pgfpathmoveto{\pgfpoint{265.267242pt}{248.854340pt}}
\pgflineto{\pgfpoint{263.767242pt}{247.764526pt}}
\pgfusepath{stroke}
\pgfpathmoveto{\pgfpoint{267.121338pt}{248.854340pt}}
\pgflineto{\pgfpoint{265.267242pt}{248.854340pt}}
\pgfusepath{stroke}
\pgfpathmoveto{\pgfpoint{268.621338pt}{247.764526pt}}
\pgflineto{\pgfpoint{267.121338pt}{248.854340pt}}
\pgfusepath{stroke}
\pgfpathmoveto{\pgfpoint{269.194275pt}{246.001175pt}}
\pgflineto{\pgfpoint{268.621338pt}{247.764526pt}}
\pgfusepath{stroke}
\pgfpathmoveto{\pgfpoint{364.278503pt}{59.699738pt}}
\pgflineto{\pgfpoint{364.851440pt}{61.463104pt}}
\pgfusepath{stroke}
\pgfpathmoveto{\pgfpoint{362.778503pt}{58.609924pt}}
\pgflineto{\pgfpoint{364.278503pt}{59.699738pt}}
\pgfusepath{stroke}
\pgfpathmoveto{\pgfpoint{360.924377pt}{58.609924pt}}
\pgflineto{\pgfpoint{362.778503pt}{58.609924pt}}
\pgfusepath{stroke}
\pgfpathmoveto{\pgfpoint{359.424377pt}{59.699738pt}}
\pgflineto{\pgfpoint{360.924377pt}{58.609924pt}}
\pgfusepath{stroke}
\pgfpathmoveto{\pgfpoint{358.851440pt}{61.463104pt}}
\pgflineto{\pgfpoint{359.424377pt}{59.699738pt}}
\pgfusepath{stroke}
\pgfpathmoveto{\pgfpoint{359.424377pt}{63.226456pt}}
\pgflineto{\pgfpoint{358.851440pt}{61.463104pt}}
\pgfusepath{stroke}
\pgfpathmoveto{\pgfpoint{360.924377pt}{64.316269pt}}
\pgflineto{\pgfpoint{359.424377pt}{63.226456pt}}
\pgfusepath{stroke}
\pgfpathmoveto{\pgfpoint{362.778503pt}{64.316269pt}}
\pgflineto{\pgfpoint{360.924377pt}{64.316269pt}}
\pgfusepath{stroke}
\pgfpathmoveto{\pgfpoint{364.278503pt}{63.226456pt}}
\pgflineto{\pgfpoint{362.778503pt}{64.316269pt}}
\pgfusepath{stroke}
\pgfpathmoveto{\pgfpoint{364.851440pt}{61.463104pt}}
\pgflineto{\pgfpoint{364.278503pt}{63.226456pt}}
\pgfusepath{stroke}
\color[rgb]{0.850000,0.325000,0.098000}
\pgfsetbuttcap
\pgfpathmoveto{\pgfpoint{124.966080pt}{348.039642pt}}
\pgflineto{\pgfpoint{74.880005pt}{223.559998pt}}
\pgfusepath{stroke}
\pgfpathmoveto{\pgfpoint{175.052155pt}{399.599976pt}}
\pgflineto{\pgfpoint{124.966080pt}{348.039642pt}}
\pgfusepath{stroke}
\pgfpathmoveto{\pgfpoint{225.138245pt}{348.039642pt}}
\pgflineto{\pgfpoint{175.052155pt}{399.599976pt}}
\pgfusepath{stroke}
\pgfpathmoveto{\pgfpoint{275.223694pt}{223.559998pt}}
\pgflineto{\pgfpoint{225.138245pt}{348.039642pt}}
\pgfusepath{stroke}
\pgfpathmoveto{\pgfpoint{325.309753pt}{99.080345pt}}
\pgflineto{\pgfpoint{275.223694pt}{223.559998pt}}
\pgfusepath{stroke}
\pgfpathmoveto{\pgfpoint{375.395813pt}{47.519989pt}}
\pgflineto{\pgfpoint{325.309753pt}{99.080345pt}}
\pgfusepath{stroke}
\pgfpathmoveto{\pgfpoint{425.481934pt}{99.080345pt}}
\pgflineto{\pgfpoint{375.395813pt}{47.519989pt}}
\pgfusepath{stroke}
\pgfpathmoveto{\pgfpoint{475.567993pt}{223.559998pt}}
\pgflineto{\pgfpoint{425.481934pt}{99.080345pt}}
\pgfusepath{stroke}
\end{pgfscope}
\end{pgfpicture}
}
    \end{figure}

  \item For convenience purposes, we define a thin wrapper around \verb|interp1|
\begin{verbatim}
octave> interpolate = @(X, method) interp1 (
> x, fx, X, method, "extrap");
\end{verbatim}
    Anonymous function had to be used because named functions somehow do not
    support closure.  Now we can use \verb|interpolate (points, method)|
    to approximate \verb|f(3.00000)| and \verb|f(4.50000)|
    and obtain the table below
    \begin{center}
      \begin{tabular}{c r r r}
        \toprule
        method & nearest & cubic & spline \\
        \midrule
        f(3.00000) & 0 & 0.13528 & 0.14073 \\
        f(4.50000) & -1 & -0.96943 & -0.97745\\
        \bottomrule
      \end{tabular}
    \end{center}

    Next, we use some plots to better visualize these interpolation methods.
\begin{verbatim}
octave> interplot = @(mark, line, method) plot (
> mark, interpolate (mark, method), "o",
> line, interpolate (line, method));
octave> B = linspace (x(1), x(end));
octave> interplot (points, B, "nearest")
\end{verbatim}
\scalebox{0.61}{% Title: gl2ps_renderer figure
% Creator: GL2PS 1.4.0, (C) 1999-2017 C. Geuzaine
% For: Octave
% CreationDate: Fri Oct 25 16:19:46 2019
\begin{pgfpicture}
\color[rgb]{1.000000,1.000000,1.000000}
\pgfpathrectanglecorners{\pgfpoint{0pt}{0pt}}{\pgfpoint{576pt}{432pt}}
\pgfusepath{fill}
\begin{pgfscope}
\pgfpathrectangle{\pgfpoint{0pt}{0pt}}{\pgfpoint{576pt}{432pt}}
\pgfusepath{fill}
\pgfpathrectangle{\pgfpoint{0pt}{0pt}}{\pgfpoint{576pt}{432pt}}
\pgfusepath{clip}
\pgfpathmoveto{\pgfpoint{74.880005pt}{399.599976pt}}
\pgflineto{\pgfpoint{521.279968pt}{47.519989pt}}
\pgflineto{\pgfpoint{74.880005pt}{47.519989pt}}
\pgfpathclose
\pgfusepath{fill,stroke}
\pgfpathmoveto{\pgfpoint{74.880005pt}{399.599976pt}}
\pgflineto{\pgfpoint{521.279968pt}{399.599976pt}}
\pgflineto{\pgfpoint{521.279968pt}{47.519989pt}}
\pgfpathclose
\pgfusepath{fill,stroke}
\color[rgb]{0.150000,0.150000,0.150000}
\pgfsetlinewidth{0.500000pt}
\pgfpathmoveto{\pgfpoint{74.880005pt}{51.985016pt}}
\pgflineto{\pgfpoint{74.880005pt}{47.519989pt}}
\pgfusepath{stroke}
\pgfpathmoveto{\pgfpoint{74.880005pt}{395.134979pt}}
\pgflineto{\pgfpoint{74.880005pt}{399.599976pt}}
\pgfusepath{stroke}
\pgfpathmoveto{\pgfpoint{138.651428pt}{51.985016pt}}
\pgflineto{\pgfpoint{138.651428pt}{47.519989pt}}
\pgfusepath{stroke}
\pgfpathmoveto{\pgfpoint{138.651428pt}{395.134979pt}}
\pgflineto{\pgfpoint{138.651428pt}{399.599976pt}}
\pgfusepath{stroke}
\pgfpathmoveto{\pgfpoint{202.422852pt}{51.985016pt}}
\pgflineto{\pgfpoint{202.422852pt}{47.519989pt}}
\pgfusepath{stroke}
\pgfpathmoveto{\pgfpoint{202.422852pt}{395.134979pt}}
\pgflineto{\pgfpoint{202.422852pt}{399.599976pt}}
\pgfusepath{stroke}
\pgfpathmoveto{\pgfpoint{266.194275pt}{51.985016pt}}
\pgflineto{\pgfpoint{266.194275pt}{47.519989pt}}
\pgfusepath{stroke}
\pgfpathmoveto{\pgfpoint{266.194275pt}{395.134979pt}}
\pgflineto{\pgfpoint{266.194275pt}{399.599976pt}}
\pgfusepath{stroke}
\pgfpathmoveto{\pgfpoint{329.965698pt}{51.985016pt}}
\pgflineto{\pgfpoint{329.965698pt}{47.519989pt}}
\pgfusepath{stroke}
\pgfpathmoveto{\pgfpoint{329.965698pt}{395.134979pt}}
\pgflineto{\pgfpoint{329.965698pt}{399.599976pt}}
\pgfusepath{stroke}
\pgfpathmoveto{\pgfpoint{393.737152pt}{51.985016pt}}
\pgflineto{\pgfpoint{393.737152pt}{47.519989pt}}
\pgfusepath{stroke}
\pgfpathmoveto{\pgfpoint{393.737152pt}{395.134979pt}}
\pgflineto{\pgfpoint{393.737152pt}{399.599976pt}}
\pgfusepath{stroke}
\pgfpathmoveto{\pgfpoint{457.508575pt}{51.985016pt}}
\pgflineto{\pgfpoint{457.508575pt}{47.519989pt}}
\pgfusepath{stroke}
\pgfpathmoveto{\pgfpoint{457.508575pt}{395.134979pt}}
\pgflineto{\pgfpoint{457.508575pt}{399.599976pt}}
\pgfusepath{stroke}
\pgfpathmoveto{\pgfpoint{521.279968pt}{51.985016pt}}
\pgflineto{\pgfpoint{521.279968pt}{47.519989pt}}
\pgfusepath{stroke}
\pgfpathmoveto{\pgfpoint{521.279968pt}{395.134979pt}}
\pgflineto{\pgfpoint{521.279968pt}{399.599976pt}}
\pgfusepath{stroke}
{
\pgftransformshift{\pgfpoint{74.880005pt}{40.018295pt}}
\pgfnode{rectangle}{north}{\fontsize{10}{0}\selectfont\textcolor[rgb]{0.15,0.15,0.15}{{0}}}{}{\pgfusepath{discard}}}
{
\pgftransformshift{\pgfpoint{138.651428pt}{40.018295pt}}
\pgfnode{rectangle}{north}{\fontsize{10}{0}\selectfont\textcolor[rgb]{0.15,0.15,0.15}{{1}}}{}{\pgfusepath{discard}}}
{
\pgftransformshift{\pgfpoint{202.422852pt}{40.018295pt}}
\pgfnode{rectangle}{north}{\fontsize{10}{0}\selectfont\textcolor[rgb]{0.15,0.15,0.15}{{2}}}{}{\pgfusepath{discard}}}
{
\pgftransformshift{\pgfpoint{266.194305pt}{40.018295pt}}
\pgfnode{rectangle}{north}{\fontsize{10}{0}\selectfont\textcolor[rgb]{0.15,0.15,0.15}{{3}}}{}{\pgfusepath{discard}}}
{
\pgftransformshift{\pgfpoint{329.965698pt}{40.018295pt}}
\pgfnode{rectangle}{north}{\fontsize{10}{0}\selectfont\textcolor[rgb]{0.15,0.15,0.15}{{4}}}{}{\pgfusepath{discard}}}
{
\pgftransformshift{\pgfpoint{393.737122pt}{40.018295pt}}
\pgfnode{rectangle}{north}{\fontsize{10}{0}\selectfont\textcolor[rgb]{0.15,0.15,0.15}{{5}}}{}{\pgfusepath{discard}}}
{
\pgftransformshift{\pgfpoint{457.508606pt}{40.018295pt}}
\pgfnode{rectangle}{north}{\fontsize{10}{0}\selectfont\textcolor[rgb]{0.15,0.15,0.15}{{6}}}{}{\pgfusepath{discard}}}
{
\pgftransformshift{\pgfpoint{521.279968pt}{40.018295pt}}
\pgfnode{rectangle}{north}{\fontsize{10}{0}\selectfont\textcolor[rgb]{0.15,0.15,0.15}{{7}}}{}{\pgfusepath{discard}}}
\pgfpathmoveto{\pgfpoint{79.347992pt}{47.519989pt}}
\pgflineto{\pgfpoint{74.880005pt}{47.519989pt}}
\pgfusepath{stroke}
\pgfpathmoveto{\pgfpoint{516.812012pt}{47.519989pt}}
\pgflineto{\pgfpoint{521.279968pt}{47.519989pt}}
\pgfusepath{stroke}
\pgfpathmoveto{\pgfpoint{79.347992pt}{135.539993pt}}
\pgflineto{\pgfpoint{74.880005pt}{135.539993pt}}
\pgfusepath{stroke}
\pgfpathmoveto{\pgfpoint{516.812012pt}{135.539993pt}}
\pgflineto{\pgfpoint{521.279968pt}{135.539993pt}}
\pgfusepath{stroke}
\pgfpathmoveto{\pgfpoint{79.347992pt}{223.559998pt}}
\pgflineto{\pgfpoint{74.880005pt}{223.559998pt}}
\pgfusepath{stroke}
\pgfpathmoveto{\pgfpoint{516.812012pt}{223.559998pt}}
\pgflineto{\pgfpoint{521.279968pt}{223.559998pt}}
\pgfusepath{stroke}
\pgfpathmoveto{\pgfpoint{79.347992pt}{311.579987pt}}
\pgflineto{\pgfpoint{74.880005pt}{311.579987pt}}
\pgfusepath{stroke}
\pgfpathmoveto{\pgfpoint{516.812012pt}{311.579987pt}}
\pgflineto{\pgfpoint{521.279968pt}{311.579987pt}}
\pgfusepath{stroke}
\pgfpathmoveto{\pgfpoint{79.347992pt}{399.599976pt}}
\pgflineto{\pgfpoint{74.880005pt}{399.599976pt}}
\pgfusepath{stroke}
\pgfpathmoveto{\pgfpoint{516.812012pt}{399.599976pt}}
\pgflineto{\pgfpoint{521.279968pt}{399.599976pt}}
\pgfusepath{stroke}
{
\pgftransformshift{\pgfpoint{69.875519pt}{47.519989pt}}
\pgfnode{rectangle}{east}{\fontsize{10}{0}\selectfont\textcolor[rgb]{0.15,0.15,0.15}{{-1}}}{}{\pgfusepath{discard}}}
{
\pgftransformshift{\pgfpoint{69.875519pt}{135.539993pt}}
\pgfnode{rectangle}{east}{\fontsize{10}{0}\selectfont\textcolor[rgb]{0.15,0.15,0.15}{{-0.5}}}{}{\pgfusepath{discard}}}
{
\pgftransformshift{\pgfpoint{69.875519pt}{223.559998pt}}
\pgfnode{rectangle}{east}{\fontsize{10}{0}\selectfont\textcolor[rgb]{0.15,0.15,0.15}{{0}}}{}{\pgfusepath{discard}}}
{
\pgftransformshift{\pgfpoint{69.875519pt}{311.579987pt}}
\pgfnode{rectangle}{east}{\fontsize{10}{0}\selectfont\textcolor[rgb]{0.15,0.15,0.15}{{0.5}}}{}{\pgfusepath{discard}}}
{
\pgftransformshift{\pgfpoint{69.875519pt}{399.599976pt}}
\pgfnode{rectangle}{east}{\fontsize{10}{0}\selectfont\textcolor[rgb]{0.15,0.15,0.15}{{1}}}{}{\pgfusepath{discard}}}
\pgfsetrectcap
\pgfsetdash{{16pt}{0pt}}{0pt}
\pgfpathmoveto{\pgfpoint{521.279968pt}{47.519989pt}}
\pgflineto{\pgfpoint{74.880005pt}{47.519989pt}}
\pgfusepath{stroke}
\pgfpathmoveto{\pgfpoint{521.279968pt}{399.599976pt}}
\pgflineto{\pgfpoint{74.880005pt}{399.599976pt}}
\pgfusepath{stroke}
\pgfpathmoveto{\pgfpoint{74.880005pt}{399.599976pt}}
\pgflineto{\pgfpoint{74.880005pt}{47.519989pt}}
\pgfusepath{stroke}
\pgfpathmoveto{\pgfpoint{521.279968pt}{399.599976pt}}
\pgflineto{\pgfpoint{521.279968pt}{47.519989pt}}
\pgfusepath{stroke}
\color[rgb]{0.000000,0.447000,0.741000}
\pgfsetroundcap
\pgfsetroundjoin
\pgfsetdash{}{0pt}
\pgfpathmoveto{\pgfpoint{268.621338pt}{221.796631pt}}
\pgflineto{\pgfpoint{269.194275pt}{223.559998pt}}
\pgfusepath{stroke}
\pgfpathmoveto{\pgfpoint{267.121338pt}{220.706818pt}}
\pgflineto{\pgfpoint{268.621338pt}{221.796631pt}}
\pgfusepath{stroke}
\pgfpathmoveto{\pgfpoint{265.267242pt}{220.706818pt}}
\pgflineto{\pgfpoint{267.121338pt}{220.706818pt}}
\pgfusepath{stroke}
\pgfpathmoveto{\pgfpoint{263.767242pt}{221.796631pt}}
\pgflineto{\pgfpoint{265.267242pt}{220.706818pt}}
\pgfusepath{stroke}
\pgfpathmoveto{\pgfpoint{263.194275pt}{223.559998pt}}
\pgflineto{\pgfpoint{263.767242pt}{221.796631pt}}
\pgfusepath{stroke}
\pgfpathmoveto{\pgfpoint{263.767242pt}{225.323349pt}}
\pgflineto{\pgfpoint{263.194275pt}{223.559998pt}}
\pgfusepath{stroke}
\pgfpathmoveto{\pgfpoint{265.267242pt}{226.413162pt}}
\pgflineto{\pgfpoint{263.767242pt}{225.323349pt}}
\pgfusepath{stroke}
\pgfpathmoveto{\pgfpoint{267.121338pt}{226.413162pt}}
\pgflineto{\pgfpoint{265.267242pt}{226.413162pt}}
\pgfusepath{stroke}
\pgfpathmoveto{\pgfpoint{268.621338pt}{225.323349pt}}
\pgflineto{\pgfpoint{267.121338pt}{226.413162pt}}
\pgfusepath{stroke}
\pgfpathmoveto{\pgfpoint{269.194275pt}{223.559998pt}}
\pgflineto{\pgfpoint{268.621338pt}{225.323349pt}}
\pgfusepath{stroke}
\pgfpathmoveto{\pgfpoint{364.278503pt}{45.756622pt}}
\pgflineto{\pgfpoint{364.851440pt}{47.519974pt}}
\pgfusepath{stroke}
\pgfpathmoveto{\pgfpoint{362.778503pt}{44.666809pt}}
\pgflineto{\pgfpoint{364.278503pt}{45.756622pt}}
\pgfusepath{stroke}
\pgfpathmoveto{\pgfpoint{360.924377pt}{44.666809pt}}
\pgflineto{\pgfpoint{362.778503pt}{44.666809pt}}
\pgfusepath{stroke}
\pgfpathmoveto{\pgfpoint{359.424377pt}{45.756622pt}}
\pgflineto{\pgfpoint{360.924377pt}{44.666809pt}}
\pgfusepath{stroke}
\pgfpathmoveto{\pgfpoint{358.851440pt}{47.519974pt}}
\pgflineto{\pgfpoint{359.424377pt}{45.756622pt}}
\pgfusepath{stroke}
\pgfpathmoveto{\pgfpoint{359.424377pt}{49.283340pt}}
\pgflineto{\pgfpoint{358.851440pt}{47.519974pt}}
\pgfusepath{stroke}
\pgfpathmoveto{\pgfpoint{360.924377pt}{50.373154pt}}
\pgflineto{\pgfpoint{359.424377pt}{49.283340pt}}
\pgfusepath{stroke}
\pgfpathmoveto{\pgfpoint{362.778503pt}{50.373154pt}}
\pgflineto{\pgfpoint{360.924377pt}{50.373154pt}}
\pgfusepath{stroke}
\pgfpathmoveto{\pgfpoint{364.278503pt}{49.283340pt}}
\pgflineto{\pgfpoint{362.778503pt}{50.373154pt}}
\pgfusepath{stroke}
\pgfpathmoveto{\pgfpoint{364.851440pt}{47.519974pt}}
\pgflineto{\pgfpoint{364.278503pt}{49.283340pt}}
\pgfusepath{stroke}
\color[rgb]{0.850000,0.325000,0.098000}
\pgfsetbuttcap
\pgfpathmoveto{\pgfpoint{78.927338pt}{223.559998pt}}
\pgflineto{\pgfpoint{74.880005pt}{223.559998pt}}
\pgfusepath{stroke}
\pgfpathmoveto{\pgfpoint{82.974701pt}{223.559998pt}}
\pgflineto{\pgfpoint{78.927338pt}{223.559998pt}}
\pgfusepath{stroke}
\pgfpathmoveto{\pgfpoint{87.022049pt}{223.559998pt}}
\pgflineto{\pgfpoint{82.974701pt}{223.559998pt}}
\pgfusepath{stroke}
\pgfpathmoveto{\pgfpoint{91.069412pt}{223.559998pt}}
\pgflineto{\pgfpoint{87.022049pt}{223.559998pt}}
\pgfusepath{stroke}
\pgfpathmoveto{\pgfpoint{95.116760pt}{223.559998pt}}
\pgflineto{\pgfpoint{91.069412pt}{223.559998pt}}
\pgfusepath{stroke}
\pgfpathmoveto{\pgfpoint{99.164124pt}{223.559998pt}}
\pgflineto{\pgfpoint{95.116760pt}{223.559998pt}}
\pgfusepath{stroke}
\pgfpathmoveto{\pgfpoint{103.211472pt}{348.039642pt}}
\pgflineto{\pgfpoint{99.164124pt}{223.559998pt}}
\pgfusepath{stroke}
\pgfpathmoveto{\pgfpoint{107.258820pt}{348.039642pt}}
\pgflineto{\pgfpoint{103.211472pt}{348.039642pt}}
\pgfusepath{stroke}
\pgfpathmoveto{\pgfpoint{111.306183pt}{348.039642pt}}
\pgflineto{\pgfpoint{107.258820pt}{348.039642pt}}
\pgfusepath{stroke}
\pgfpathmoveto{\pgfpoint{115.353531pt}{348.039642pt}}
\pgflineto{\pgfpoint{111.306183pt}{348.039642pt}}
\pgfusepath{stroke}
\pgfpathmoveto{\pgfpoint{119.400894pt}{348.039642pt}}
\pgflineto{\pgfpoint{115.353531pt}{348.039642pt}}
\pgfusepath{stroke}
\pgfpathmoveto{\pgfpoint{123.448242pt}{348.039642pt}}
\pgflineto{\pgfpoint{119.400894pt}{348.039642pt}}
\pgfusepath{stroke}
\pgfpathmoveto{\pgfpoint{127.495605pt}{348.039642pt}}
\pgflineto{\pgfpoint{123.448242pt}{348.039642pt}}
\pgfusepath{stroke}
\pgfpathmoveto{\pgfpoint{131.542938pt}{348.039642pt}}
\pgflineto{\pgfpoint{127.495605pt}{348.039642pt}}
\pgfusepath{stroke}
\pgfpathmoveto{\pgfpoint{135.590302pt}{348.039642pt}}
\pgflineto{\pgfpoint{131.542938pt}{348.039642pt}}
\pgfusepath{stroke}
\pgfpathmoveto{\pgfpoint{139.637650pt}{348.039642pt}}
\pgflineto{\pgfpoint{135.590302pt}{348.039642pt}}
\pgfusepath{stroke}
\pgfpathmoveto{\pgfpoint{143.684998pt}{348.039642pt}}
\pgflineto{\pgfpoint{139.637650pt}{348.039642pt}}
\pgfusepath{stroke}
\pgfpathmoveto{\pgfpoint{147.732361pt}{348.039642pt}}
\pgflineto{\pgfpoint{143.684998pt}{348.039642pt}}
\pgfusepath{stroke}
\pgfpathmoveto{\pgfpoint{151.779724pt}{399.599976pt}}
\pgflineto{\pgfpoint{147.732361pt}{348.039642pt}}
\pgfusepath{stroke}
\pgfpathmoveto{\pgfpoint{155.827072pt}{399.599976pt}}
\pgflineto{\pgfpoint{151.779724pt}{399.599976pt}}
\pgfusepath{stroke}
\pgfpathmoveto{\pgfpoint{159.874420pt}{399.599976pt}}
\pgflineto{\pgfpoint{155.827072pt}{399.599976pt}}
\pgfusepath{stroke}
\pgfpathmoveto{\pgfpoint{163.921783pt}{399.599976pt}}
\pgflineto{\pgfpoint{159.874420pt}{399.599976pt}}
\pgfusepath{stroke}
\pgfpathmoveto{\pgfpoint{167.969131pt}{399.599976pt}}
\pgflineto{\pgfpoint{163.921783pt}{399.599976pt}}
\pgfusepath{stroke}
\pgfpathmoveto{\pgfpoint{172.016479pt}{399.599976pt}}
\pgflineto{\pgfpoint{167.969131pt}{399.599976pt}}
\pgfusepath{stroke}
\pgfpathmoveto{\pgfpoint{176.063843pt}{399.599976pt}}
\pgflineto{\pgfpoint{172.016479pt}{399.599976pt}}
\pgfusepath{stroke}
\pgfpathmoveto{\pgfpoint{180.111191pt}{399.599976pt}}
\pgflineto{\pgfpoint{176.063843pt}{399.599976pt}}
\pgfusepath{stroke}
\pgfpathmoveto{\pgfpoint{184.158539pt}{399.599976pt}}
\pgflineto{\pgfpoint{180.111191pt}{399.599976pt}}
\pgfusepath{stroke}
\pgfpathmoveto{\pgfpoint{188.205902pt}{399.599976pt}}
\pgflineto{\pgfpoint{184.158539pt}{399.599976pt}}
\pgfusepath{stroke}
\pgfpathmoveto{\pgfpoint{192.253250pt}{399.599976pt}}
\pgflineto{\pgfpoint{188.205902pt}{399.599976pt}}
\pgfusepath{stroke}
\pgfpathmoveto{\pgfpoint{196.300598pt}{399.599976pt}}
\pgflineto{\pgfpoint{192.253250pt}{399.599976pt}}
\pgfusepath{stroke}
\pgfpathmoveto{\pgfpoint{200.347961pt}{348.039642pt}}
\pgflineto{\pgfpoint{196.300598pt}{399.599976pt}}
\pgfusepath{stroke}
\pgfpathmoveto{\pgfpoint{204.395294pt}{348.039642pt}}
\pgflineto{\pgfpoint{200.347961pt}{348.039642pt}}
\pgfusepath{stroke}
\pgfpathmoveto{\pgfpoint{208.442657pt}{348.039642pt}}
\pgflineto{\pgfpoint{204.395294pt}{348.039642pt}}
\pgfusepath{stroke}
\pgfpathmoveto{\pgfpoint{212.490021pt}{348.039642pt}}
\pgflineto{\pgfpoint{208.442657pt}{348.039642pt}}
\pgfusepath{stroke}
\pgfpathmoveto{\pgfpoint{216.537369pt}{348.039642pt}}
\pgflineto{\pgfpoint{212.490021pt}{348.039642pt}}
\pgfusepath{stroke}
\pgfpathmoveto{\pgfpoint{220.584732pt}{348.039642pt}}
\pgflineto{\pgfpoint{216.537369pt}{348.039642pt}}
\pgfusepath{stroke}
\pgfpathmoveto{\pgfpoint{224.632080pt}{348.039642pt}}
\pgflineto{\pgfpoint{220.584732pt}{348.039642pt}}
\pgfusepath{stroke}
\pgfpathmoveto{\pgfpoint{228.679443pt}{348.039642pt}}
\pgflineto{\pgfpoint{224.632080pt}{348.039642pt}}
\pgfusepath{stroke}
\pgfpathmoveto{\pgfpoint{232.726791pt}{348.039642pt}}
\pgflineto{\pgfpoint{228.679443pt}{348.039642pt}}
\pgfusepath{stroke}
\pgfpathmoveto{\pgfpoint{236.774155pt}{348.039642pt}}
\pgflineto{\pgfpoint{232.726791pt}{348.039642pt}}
\pgfusepath{stroke}
\pgfpathmoveto{\pgfpoint{240.821503pt}{348.039642pt}}
\pgflineto{\pgfpoint{236.774155pt}{348.039642pt}}
\pgfusepath{stroke}
\pgfpathmoveto{\pgfpoint{244.868835pt}{348.039642pt}}
\pgflineto{\pgfpoint{240.821503pt}{348.039642pt}}
\pgfusepath{stroke}
\pgfpathmoveto{\pgfpoint{248.916199pt}{348.039642pt}}
\pgflineto{\pgfpoint{244.868835pt}{348.039642pt}}
\pgfusepath{stroke}
\pgfpathmoveto{\pgfpoint{252.963562pt}{223.559998pt}}
\pgflineto{\pgfpoint{248.916199pt}{348.039642pt}}
\pgfusepath{stroke}
\pgfpathmoveto{\pgfpoint{257.010895pt}{223.559998pt}}
\pgflineto{\pgfpoint{252.963562pt}{223.559998pt}}
\pgfusepath{stroke}
\pgfpathmoveto{\pgfpoint{261.058258pt}{223.559998pt}}
\pgflineto{\pgfpoint{257.010895pt}{223.559998pt}}
\pgfusepath{stroke}
\pgfpathmoveto{\pgfpoint{265.105621pt}{223.559998pt}}
\pgflineto{\pgfpoint{261.058258pt}{223.559998pt}}
\pgfusepath{stroke}
\pgfpathmoveto{\pgfpoint{269.152985pt}{223.559998pt}}
\pgflineto{\pgfpoint{265.105621pt}{223.559998pt}}
\pgfusepath{stroke}
\pgfpathmoveto{\pgfpoint{273.200317pt}{223.559998pt}}
\pgflineto{\pgfpoint{269.152985pt}{223.559998pt}}
\pgfusepath{stroke}
\pgfpathmoveto{\pgfpoint{277.247681pt}{223.559998pt}}
\pgflineto{\pgfpoint{273.200317pt}{223.559998pt}}
\pgfusepath{stroke}
\pgfpathmoveto{\pgfpoint{281.295044pt}{223.559998pt}}
\pgflineto{\pgfpoint{277.247681pt}{223.559998pt}}
\pgfusepath{stroke}
\pgfpathmoveto{\pgfpoint{285.342377pt}{223.559998pt}}
\pgflineto{\pgfpoint{281.295044pt}{223.559998pt}}
\pgfusepath{stroke}
\pgfpathmoveto{\pgfpoint{289.389740pt}{223.559998pt}}
\pgflineto{\pgfpoint{285.342377pt}{223.559998pt}}
\pgfusepath{stroke}
\pgfpathmoveto{\pgfpoint{293.437103pt}{223.559998pt}}
\pgflineto{\pgfpoint{289.389740pt}{223.559998pt}}
\pgfusepath{stroke}
\pgfpathmoveto{\pgfpoint{297.484436pt}{223.559998pt}}
\pgflineto{\pgfpoint{293.437103pt}{223.559998pt}}
\pgfusepath{stroke}
\pgfpathmoveto{\pgfpoint{301.531799pt}{99.080345pt}}
\pgflineto{\pgfpoint{297.484436pt}{223.559998pt}}
\pgfusepath{stroke}
\pgfpathmoveto{\pgfpoint{305.579163pt}{99.080345pt}}
\pgflineto{\pgfpoint{301.531799pt}{99.080345pt}}
\pgfusepath{stroke}
\pgfpathmoveto{\pgfpoint{309.626495pt}{99.080345pt}}
\pgflineto{\pgfpoint{305.579163pt}{99.080345pt}}
\pgfusepath{stroke}
\pgfpathmoveto{\pgfpoint{313.673859pt}{99.080345pt}}
\pgflineto{\pgfpoint{309.626495pt}{99.080345pt}}
\pgfusepath{stroke}
\pgfpathmoveto{\pgfpoint{317.721222pt}{99.080345pt}}
\pgflineto{\pgfpoint{313.673859pt}{99.080345pt}}
\pgfusepath{stroke}
\pgfpathmoveto{\pgfpoint{321.768555pt}{99.080345pt}}
\pgflineto{\pgfpoint{317.721222pt}{99.080345pt}}
\pgfusepath{stroke}
\pgfpathmoveto{\pgfpoint{325.815918pt}{99.080345pt}}
\pgflineto{\pgfpoint{321.768555pt}{99.080345pt}}
\pgfusepath{stroke}
\pgfpathmoveto{\pgfpoint{329.863281pt}{99.080345pt}}
\pgflineto{\pgfpoint{325.815918pt}{99.080345pt}}
\pgfusepath{stroke}
\pgfpathmoveto{\pgfpoint{333.910614pt}{99.080345pt}}
\pgflineto{\pgfpoint{329.863281pt}{99.080345pt}}
\pgfusepath{stroke}
\pgfpathmoveto{\pgfpoint{337.957977pt}{99.080345pt}}
\pgflineto{\pgfpoint{333.910614pt}{99.080345pt}}
\pgfusepath{stroke}
\pgfpathmoveto{\pgfpoint{342.005310pt}{99.080345pt}}
\pgflineto{\pgfpoint{337.957977pt}{99.080345pt}}
\pgfusepath{stroke}
\pgfpathmoveto{\pgfpoint{346.052673pt}{99.080345pt}}
\pgflineto{\pgfpoint{342.005310pt}{99.080345pt}}
\pgfusepath{stroke}
\pgfpathmoveto{\pgfpoint{350.100037pt}{99.080345pt}}
\pgflineto{\pgfpoint{346.052673pt}{99.080345pt}}
\pgfusepath{stroke}
\pgfpathmoveto{\pgfpoint{354.147369pt}{47.519989pt}}
\pgflineto{\pgfpoint{350.100037pt}{99.080345pt}}
\pgfusepath{stroke}
\pgfpathmoveto{\pgfpoint{358.194733pt}{47.519989pt}}
\pgflineto{\pgfpoint{354.147369pt}{47.519989pt}}
\pgfusepath{stroke}
\pgfpathmoveto{\pgfpoint{362.242096pt}{47.519989pt}}
\pgflineto{\pgfpoint{358.194733pt}{47.519989pt}}
\pgfusepath{stroke}
\pgfpathmoveto{\pgfpoint{366.289459pt}{47.519989pt}}
\pgflineto{\pgfpoint{362.242096pt}{47.519989pt}}
\pgfusepath{stroke}
\pgfpathmoveto{\pgfpoint{370.336823pt}{47.519989pt}}
\pgflineto{\pgfpoint{366.289459pt}{47.519989pt}}
\pgfusepath{stroke}
\pgfpathmoveto{\pgfpoint{374.384155pt}{47.519989pt}}
\pgflineto{\pgfpoint{370.336823pt}{47.519989pt}}
\pgfusepath{stroke}
\pgfpathmoveto{\pgfpoint{378.431519pt}{47.519989pt}}
\pgflineto{\pgfpoint{374.384155pt}{47.519989pt}}
\pgfusepath{stroke}
\pgfpathmoveto{\pgfpoint{382.478882pt}{47.519989pt}}
\pgflineto{\pgfpoint{378.431519pt}{47.519989pt}}
\pgfusepath{stroke}
\pgfpathmoveto{\pgfpoint{386.526245pt}{47.519989pt}}
\pgflineto{\pgfpoint{382.478882pt}{47.519989pt}}
\pgfusepath{stroke}
\pgfpathmoveto{\pgfpoint{390.573578pt}{47.519989pt}}
\pgflineto{\pgfpoint{386.526245pt}{47.519989pt}}
\pgfusepath{stroke}
\pgfpathmoveto{\pgfpoint{394.620941pt}{47.519989pt}}
\pgflineto{\pgfpoint{390.573578pt}{47.519989pt}}
\pgfusepath{stroke}
\pgfpathmoveto{\pgfpoint{398.668304pt}{47.519989pt}}
\pgflineto{\pgfpoint{394.620941pt}{47.519989pt}}
\pgfusepath{stroke}
\pgfpathmoveto{\pgfpoint{402.715637pt}{99.080345pt}}
\pgflineto{\pgfpoint{398.668304pt}{47.519989pt}}
\pgfusepath{stroke}
\pgfpathmoveto{\pgfpoint{406.763000pt}{99.080345pt}}
\pgflineto{\pgfpoint{402.715637pt}{99.080345pt}}
\pgfusepath{stroke}
\pgfpathmoveto{\pgfpoint{410.810364pt}{99.080345pt}}
\pgflineto{\pgfpoint{406.763000pt}{99.080345pt}}
\pgfusepath{stroke}
\pgfpathmoveto{\pgfpoint{414.857697pt}{99.080345pt}}
\pgflineto{\pgfpoint{410.810364pt}{99.080345pt}}
\pgfusepath{stroke}
\pgfpathmoveto{\pgfpoint{418.905060pt}{99.080345pt}}
\pgflineto{\pgfpoint{414.857697pt}{99.080345pt}}
\pgfusepath{stroke}
\pgfpathmoveto{\pgfpoint{422.952423pt}{99.080345pt}}
\pgflineto{\pgfpoint{418.905060pt}{99.080345pt}}
\pgfusepath{stroke}
\pgfpathmoveto{\pgfpoint{426.999756pt}{99.080345pt}}
\pgflineto{\pgfpoint{422.952423pt}{99.080345pt}}
\pgfusepath{stroke}
\pgfpathmoveto{\pgfpoint{431.047119pt}{99.080345pt}}
\pgflineto{\pgfpoint{426.999756pt}{99.080345pt}}
\pgfusepath{stroke}
\pgfpathmoveto{\pgfpoint{435.094482pt}{99.080345pt}}
\pgflineto{\pgfpoint{431.047119pt}{99.080345pt}}
\pgfusepath{stroke}
\pgfpathmoveto{\pgfpoint{439.141785pt}{99.080345pt}}
\pgflineto{\pgfpoint{435.094482pt}{99.080345pt}}
\pgfusepath{stroke}
\pgfpathmoveto{\pgfpoint{443.189148pt}{99.080345pt}}
\pgflineto{\pgfpoint{439.141785pt}{99.080345pt}}
\pgfusepath{stroke}
\pgfpathmoveto{\pgfpoint{447.236511pt}{99.080345pt}}
\pgflineto{\pgfpoint{443.189148pt}{99.080345pt}}
\pgfusepath{stroke}
\pgfpathmoveto{\pgfpoint{451.283875pt}{223.559998pt}}
\pgflineto{\pgfpoint{447.236511pt}{99.080345pt}}
\pgfusepath{stroke}
\pgfpathmoveto{\pgfpoint{455.331238pt}{223.559998pt}}
\pgflineto{\pgfpoint{451.283875pt}{223.559998pt}}
\pgfusepath{stroke}
\pgfpathmoveto{\pgfpoint{459.378601pt}{223.559998pt}}
\pgflineto{\pgfpoint{455.331238pt}{223.559998pt}}
\pgfusepath{stroke}
\pgfpathmoveto{\pgfpoint{463.425964pt}{223.559998pt}}
\pgflineto{\pgfpoint{459.378601pt}{223.559998pt}}
\pgfusepath{stroke}
\pgfpathmoveto{\pgfpoint{467.473267pt}{223.559998pt}}
\pgflineto{\pgfpoint{463.425964pt}{223.559998pt}}
\pgfusepath{stroke}
\pgfpathmoveto{\pgfpoint{471.520630pt}{223.559998pt}}
\pgflineto{\pgfpoint{467.473267pt}{223.559998pt}}
\pgfusepath{stroke}
\pgfpathmoveto{\pgfpoint{475.567993pt}{223.559998pt}}
\pgflineto{\pgfpoint{471.520630pt}{223.559998pt}}
\pgfusepath{stroke}
\end{pgfscope}
\end{pgfpicture}
}
\begin{verbatim}
octave> interplot (points, B, "cubic")
\end{verbatim}
\scalebox{0.61}{% Title: gl2ps_renderer figure
% Creator: GL2PS 1.4.0, (C) 1999-2017 C. Geuzaine
% For: Octave
% CreationDate: Fri Oct 25 16:20:21 2019
\begin{pgfpicture}
\color[rgb]{1.000000,1.000000,1.000000}
\pgfpathrectanglecorners{\pgfpoint{0pt}{0pt}}{\pgfpoint{576pt}{432pt}}
\pgfusepath{fill}
\begin{pgfscope}
\pgfpathrectangle{\pgfpoint{0pt}{0pt}}{\pgfpoint{576pt}{432pt}}
\pgfusepath{fill}
\pgfpathrectangle{\pgfpoint{0pt}{0pt}}{\pgfpoint{576pt}{432pt}}
\pgfusepath{clip}
\pgfpathmoveto{\pgfpoint{74.880005pt}{399.599976pt}}
\pgflineto{\pgfpoint{521.279968pt}{47.519989pt}}
\pgflineto{\pgfpoint{74.880005pt}{47.519989pt}}
\pgfpathclose
\pgfusepath{fill,stroke}
\pgfpathmoveto{\pgfpoint{74.880005pt}{399.599976pt}}
\pgflineto{\pgfpoint{521.279968pt}{399.599976pt}}
\pgflineto{\pgfpoint{521.279968pt}{47.519989pt}}
\pgfpathclose
\pgfusepath{fill,stroke}
\color[rgb]{0.150000,0.150000,0.150000}
\pgfsetlinewidth{0.500000pt}
\pgfpathmoveto{\pgfpoint{74.880005pt}{51.985016pt}}
\pgflineto{\pgfpoint{74.880005pt}{47.519989pt}}
\pgfusepath{stroke}
\pgfpathmoveto{\pgfpoint{74.880005pt}{395.134979pt}}
\pgflineto{\pgfpoint{74.880005pt}{399.599976pt}}
\pgfusepath{stroke}
\pgfpathmoveto{\pgfpoint{138.651428pt}{51.985016pt}}
\pgflineto{\pgfpoint{138.651428pt}{47.519989pt}}
\pgfusepath{stroke}
\pgfpathmoveto{\pgfpoint{138.651428pt}{395.134979pt}}
\pgflineto{\pgfpoint{138.651428pt}{399.599976pt}}
\pgfusepath{stroke}
\pgfpathmoveto{\pgfpoint{202.422852pt}{51.985016pt}}
\pgflineto{\pgfpoint{202.422852pt}{47.519989pt}}
\pgfusepath{stroke}
\pgfpathmoveto{\pgfpoint{202.422852pt}{395.134979pt}}
\pgflineto{\pgfpoint{202.422852pt}{399.599976pt}}
\pgfusepath{stroke}
\pgfpathmoveto{\pgfpoint{266.194275pt}{51.985016pt}}
\pgflineto{\pgfpoint{266.194275pt}{47.519989pt}}
\pgfusepath{stroke}
\pgfpathmoveto{\pgfpoint{266.194275pt}{395.134979pt}}
\pgflineto{\pgfpoint{266.194275pt}{399.599976pt}}
\pgfusepath{stroke}
\pgfpathmoveto{\pgfpoint{329.965698pt}{51.985016pt}}
\pgflineto{\pgfpoint{329.965698pt}{47.519989pt}}
\pgfusepath{stroke}
\pgfpathmoveto{\pgfpoint{329.965698pt}{395.134979pt}}
\pgflineto{\pgfpoint{329.965698pt}{399.599976pt}}
\pgfusepath{stroke}
\pgfpathmoveto{\pgfpoint{393.737152pt}{51.985016pt}}
\pgflineto{\pgfpoint{393.737152pt}{47.519989pt}}
\pgfusepath{stroke}
\pgfpathmoveto{\pgfpoint{393.737152pt}{395.134979pt}}
\pgflineto{\pgfpoint{393.737152pt}{399.599976pt}}
\pgfusepath{stroke}
\pgfpathmoveto{\pgfpoint{457.508575pt}{51.985016pt}}
\pgflineto{\pgfpoint{457.508575pt}{47.519989pt}}
\pgfusepath{stroke}
\pgfpathmoveto{\pgfpoint{457.508575pt}{395.134979pt}}
\pgflineto{\pgfpoint{457.508575pt}{399.599976pt}}
\pgfusepath{stroke}
\pgfpathmoveto{\pgfpoint{521.279968pt}{51.985016pt}}
\pgflineto{\pgfpoint{521.279968pt}{47.519989pt}}
\pgfusepath{stroke}
\pgfpathmoveto{\pgfpoint{521.279968pt}{395.134979pt}}
\pgflineto{\pgfpoint{521.279968pt}{399.599976pt}}
\pgfusepath{stroke}
{
\pgftransformshift{\pgfpoint{74.880005pt}{40.018295pt}}
\pgfnode{rectangle}{north}{\fontsize{10}{0}\selectfont\textcolor[rgb]{0.15,0.15,0.15}{{0}}}{}{\pgfusepath{discard}}}
{
\pgftransformshift{\pgfpoint{138.651428pt}{40.018295pt}}
\pgfnode{rectangle}{north}{\fontsize{10}{0}\selectfont\textcolor[rgb]{0.15,0.15,0.15}{{1}}}{}{\pgfusepath{discard}}}
{
\pgftransformshift{\pgfpoint{202.422852pt}{40.018295pt}}
\pgfnode{rectangle}{north}{\fontsize{10}{0}\selectfont\textcolor[rgb]{0.15,0.15,0.15}{{2}}}{}{\pgfusepath{discard}}}
{
\pgftransformshift{\pgfpoint{266.194305pt}{40.018295pt}}
\pgfnode{rectangle}{north}{\fontsize{10}{0}\selectfont\textcolor[rgb]{0.15,0.15,0.15}{{3}}}{}{\pgfusepath{discard}}}
{
\pgftransformshift{\pgfpoint{329.965698pt}{40.018295pt}}
\pgfnode{rectangle}{north}{\fontsize{10}{0}\selectfont\textcolor[rgb]{0.15,0.15,0.15}{{4}}}{}{\pgfusepath{discard}}}
{
\pgftransformshift{\pgfpoint{393.737122pt}{40.018295pt}}
\pgfnode{rectangle}{north}{\fontsize{10}{0}\selectfont\textcolor[rgb]{0.15,0.15,0.15}{{5}}}{}{\pgfusepath{discard}}}
{
\pgftransformshift{\pgfpoint{457.508606pt}{40.018295pt}}
\pgfnode{rectangle}{north}{\fontsize{10}{0}\selectfont\textcolor[rgb]{0.15,0.15,0.15}{{6}}}{}{\pgfusepath{discard}}}
{
\pgftransformshift{\pgfpoint{521.279968pt}{40.018295pt}}
\pgfnode{rectangle}{north}{\fontsize{10}{0}\selectfont\textcolor[rgb]{0.15,0.15,0.15}{{7}}}{}{\pgfusepath{discard}}}
\pgfpathmoveto{\pgfpoint{79.347992pt}{47.519989pt}}
\pgflineto{\pgfpoint{74.880005pt}{47.519989pt}}
\pgfusepath{stroke}
\pgfpathmoveto{\pgfpoint{516.812012pt}{47.519989pt}}
\pgflineto{\pgfpoint{521.279968pt}{47.519989pt}}
\pgfusepath{stroke}
\pgfpathmoveto{\pgfpoint{79.347992pt}{135.539993pt}}
\pgflineto{\pgfpoint{74.880005pt}{135.539993pt}}
\pgfusepath{stroke}
\pgfpathmoveto{\pgfpoint{516.812012pt}{135.539993pt}}
\pgflineto{\pgfpoint{521.279968pt}{135.539993pt}}
\pgfusepath{stroke}
\pgfpathmoveto{\pgfpoint{79.347992pt}{223.559998pt}}
\pgflineto{\pgfpoint{74.880005pt}{223.559998pt}}
\pgfusepath{stroke}
\pgfpathmoveto{\pgfpoint{516.812012pt}{223.559998pt}}
\pgflineto{\pgfpoint{521.279968pt}{223.559998pt}}
\pgfusepath{stroke}
\pgfpathmoveto{\pgfpoint{79.347992pt}{311.579987pt}}
\pgflineto{\pgfpoint{74.880005pt}{311.579987pt}}
\pgfusepath{stroke}
\pgfpathmoveto{\pgfpoint{516.812012pt}{311.579987pt}}
\pgflineto{\pgfpoint{521.279968pt}{311.579987pt}}
\pgfusepath{stroke}
\pgfpathmoveto{\pgfpoint{79.347992pt}{399.599976pt}}
\pgflineto{\pgfpoint{74.880005pt}{399.599976pt}}
\pgfusepath{stroke}
\pgfpathmoveto{\pgfpoint{516.812012pt}{399.599976pt}}
\pgflineto{\pgfpoint{521.279968pt}{399.599976pt}}
\pgfusepath{stroke}
{
\pgftransformshift{\pgfpoint{69.875519pt}{47.519989pt}}
\pgfnode{rectangle}{east}{\fontsize{10}{0}\selectfont\textcolor[rgb]{0.15,0.15,0.15}{{-1}}}{}{\pgfusepath{discard}}}
{
\pgftransformshift{\pgfpoint{69.875519pt}{135.539993pt}}
\pgfnode{rectangle}{east}{\fontsize{10}{0}\selectfont\textcolor[rgb]{0.15,0.15,0.15}{{-0.5}}}{}{\pgfusepath{discard}}}
{
\pgftransformshift{\pgfpoint{69.875519pt}{223.559998pt}}
\pgfnode{rectangle}{east}{\fontsize{10}{0}\selectfont\textcolor[rgb]{0.15,0.15,0.15}{{0}}}{}{\pgfusepath{discard}}}
{
\pgftransformshift{\pgfpoint{69.875519pt}{311.579987pt}}
\pgfnode{rectangle}{east}{\fontsize{10}{0}\selectfont\textcolor[rgb]{0.15,0.15,0.15}{{0.5}}}{}{\pgfusepath{discard}}}
{
\pgftransformshift{\pgfpoint{69.875519pt}{399.599976pt}}
\pgfnode{rectangle}{east}{\fontsize{10}{0}\selectfont\textcolor[rgb]{0.15,0.15,0.15}{{1}}}{}{\pgfusepath{discard}}}
\pgfsetrectcap
\pgfsetdash{{16pt}{0pt}}{0pt}
\pgfpathmoveto{\pgfpoint{521.279968pt}{47.519989pt}}
\pgflineto{\pgfpoint{74.880005pt}{47.519989pt}}
\pgfusepath{stroke}
\pgfpathmoveto{\pgfpoint{521.279968pt}{399.599976pt}}
\pgflineto{\pgfpoint{74.880005pt}{399.599976pt}}
\pgfusepath{stroke}
\pgfpathmoveto{\pgfpoint{74.880005pt}{399.599976pt}}
\pgflineto{\pgfpoint{74.880005pt}{47.519989pt}}
\pgfusepath{stroke}
\pgfpathmoveto{\pgfpoint{521.279968pt}{399.599976pt}}
\pgflineto{\pgfpoint{521.279968pt}{47.519989pt}}
\pgfusepath{stroke}
\color[rgb]{0.000000,0.447000,0.741000}
\pgfsetroundcap
\pgfsetroundjoin
\pgfsetdash{}{0pt}
\pgfpathmoveto{\pgfpoint{268.621338pt}{245.611435pt}}
\pgflineto{\pgfpoint{269.194275pt}{247.374786pt}}
\pgfusepath{stroke}
\pgfpathmoveto{\pgfpoint{267.121338pt}{244.521622pt}}
\pgflineto{\pgfpoint{268.621338pt}{245.611435pt}}
\pgfusepath{stroke}
\pgfpathmoveto{\pgfpoint{265.267242pt}{244.521622pt}}
\pgflineto{\pgfpoint{267.121338pt}{244.521622pt}}
\pgfusepath{stroke}
\pgfpathmoveto{\pgfpoint{263.767242pt}{245.611435pt}}
\pgflineto{\pgfpoint{265.267242pt}{244.521622pt}}
\pgfusepath{stroke}
\pgfpathmoveto{\pgfpoint{263.194275pt}{247.374786pt}}
\pgflineto{\pgfpoint{263.767242pt}{245.611435pt}}
\pgfusepath{stroke}
\pgfpathmoveto{\pgfpoint{263.767242pt}{249.138153pt}}
\pgflineto{\pgfpoint{263.194275pt}{247.374786pt}}
\pgfusepath{stroke}
\pgfpathmoveto{\pgfpoint{265.267242pt}{250.227966pt}}
\pgflineto{\pgfpoint{263.767242pt}{249.138153pt}}
\pgfusepath{stroke}
\pgfpathmoveto{\pgfpoint{267.121338pt}{250.227966pt}}
\pgflineto{\pgfpoint{265.267242pt}{250.227966pt}}
\pgfusepath{stroke}
\pgfpathmoveto{\pgfpoint{268.621338pt}{249.138153pt}}
\pgflineto{\pgfpoint{267.121338pt}{250.227966pt}}
\pgfusepath{stroke}
\pgfpathmoveto{\pgfpoint{269.194275pt}{247.374786pt}}
\pgflineto{\pgfpoint{268.621338pt}{249.138153pt}}
\pgfusepath{stroke}
\pgfpathmoveto{\pgfpoint{364.278503pt}{51.138596pt}}
\pgflineto{\pgfpoint{364.851440pt}{52.901947pt}}
\pgfusepath{stroke}
\pgfpathmoveto{\pgfpoint{362.778503pt}{50.048782pt}}
\pgflineto{\pgfpoint{364.278503pt}{51.138596pt}}
\pgfusepath{stroke}
\pgfpathmoveto{\pgfpoint{360.924377pt}{50.048782pt}}
\pgflineto{\pgfpoint{362.778503pt}{50.048782pt}}
\pgfusepath{stroke}
\pgfpathmoveto{\pgfpoint{359.424377pt}{51.138596pt}}
\pgflineto{\pgfpoint{360.924377pt}{50.048782pt}}
\pgfusepath{stroke}
\pgfpathmoveto{\pgfpoint{358.851440pt}{52.901947pt}}
\pgflineto{\pgfpoint{359.424377pt}{51.138596pt}}
\pgfusepath{stroke}
\pgfpathmoveto{\pgfpoint{359.424377pt}{54.665314pt}}
\pgflineto{\pgfpoint{358.851440pt}{52.901947pt}}
\pgfusepath{stroke}
\pgfpathmoveto{\pgfpoint{360.924377pt}{55.755112pt}}
\pgflineto{\pgfpoint{359.424377pt}{54.665314pt}}
\pgfusepath{stroke}
\pgfpathmoveto{\pgfpoint{362.778503pt}{55.755112pt}}
\pgflineto{\pgfpoint{360.924377pt}{55.755112pt}}
\pgfusepath{stroke}
\pgfpathmoveto{\pgfpoint{364.278503pt}{54.665314pt}}
\pgflineto{\pgfpoint{362.778503pt}{55.755112pt}}
\pgfusepath{stroke}
\pgfpathmoveto{\pgfpoint{364.851440pt}{52.901947pt}}
\pgflineto{\pgfpoint{364.278503pt}{54.665314pt}}
\pgfusepath{stroke}
\color[rgb]{0.850000,0.325000,0.098000}
\pgfsetbuttcap
\pgfpathmoveto{\pgfpoint{78.927338pt}{236.417740pt}}
\pgflineto{\pgfpoint{74.880005pt}{223.559998pt}}
\pgfusepath{stroke}
\pgfpathmoveto{\pgfpoint{82.974701pt}{248.948746pt}}
\pgflineto{\pgfpoint{78.927338pt}{236.417740pt}}
\pgfusepath{stroke}
\pgfpathmoveto{\pgfpoint{87.022049pt}{261.105194pt}}
\pgflineto{\pgfpoint{82.974701pt}{248.948746pt}}
\pgfusepath{stroke}
\pgfpathmoveto{\pgfpoint{91.069412pt}{272.839294pt}}
\pgflineto{\pgfpoint{87.022049pt}{261.105194pt}}
\pgfusepath{stroke}
\pgfpathmoveto{\pgfpoint{95.116760pt}{284.103180pt}}
\pgflineto{\pgfpoint{91.069412pt}{272.839294pt}}
\pgfusepath{stroke}
\pgfpathmoveto{\pgfpoint{99.164124pt}{294.849121pt}}
\pgflineto{\pgfpoint{95.116760pt}{284.103180pt}}
\pgfusepath{stroke}
\pgfpathmoveto{\pgfpoint{103.211472pt}{305.029205pt}}
\pgflineto{\pgfpoint{99.164124pt}{294.849121pt}}
\pgfusepath{stroke}
\pgfpathmoveto{\pgfpoint{107.258820pt}{314.595703pt}}
\pgflineto{\pgfpoint{103.211472pt}{305.029205pt}}
\pgfusepath{stroke}
\pgfpathmoveto{\pgfpoint{111.306183pt}{323.500732pt}}
\pgflineto{\pgfpoint{107.258820pt}{314.595703pt}}
\pgfusepath{stroke}
\pgfpathmoveto{\pgfpoint{115.353531pt}{331.696533pt}}
\pgflineto{\pgfpoint{111.306183pt}{323.500732pt}}
\pgfusepath{stroke}
\pgfpathmoveto{\pgfpoint{119.400894pt}{339.135254pt}}
\pgflineto{\pgfpoint{115.353531pt}{331.696533pt}}
\pgfusepath{stroke}
\pgfpathmoveto{\pgfpoint{123.448242pt}{345.769104pt}}
\pgflineto{\pgfpoint{119.400894pt}{339.135254pt}}
\pgfusepath{stroke}
\pgfpathmoveto{\pgfpoint{127.495605pt}{351.740906pt}}
\pgflineto{\pgfpoint{123.448242pt}{345.769104pt}}
\pgfusepath{stroke}
\pgfpathmoveto{\pgfpoint{131.542938pt}{357.698700pt}}
\pgflineto{\pgfpoint{127.495605pt}{351.740906pt}}
\pgfusepath{stroke}
\pgfpathmoveto{\pgfpoint{135.590302pt}{363.616638pt}}
\pgflineto{\pgfpoint{131.542938pt}{357.698700pt}}
\pgfusepath{stroke}
\pgfpathmoveto{\pgfpoint{139.637650pt}{369.399048pt}}
\pgflineto{\pgfpoint{135.590302pt}{363.616638pt}}
\pgfusepath{stroke}
\pgfpathmoveto{\pgfpoint{143.684998pt}{374.950439pt}}
\pgflineto{\pgfpoint{139.637650pt}{369.399048pt}}
\pgfusepath{stroke}
\pgfpathmoveto{\pgfpoint{147.732361pt}{380.175018pt}}
\pgflineto{\pgfpoint{143.684998pt}{374.950439pt}}
\pgfusepath{stroke}
\pgfpathmoveto{\pgfpoint{151.779724pt}{384.977264pt}}
\pgflineto{\pgfpoint{147.732361pt}{380.175018pt}}
\pgfusepath{stroke}
\pgfpathmoveto{\pgfpoint{155.827072pt}{389.261536pt}}
\pgflineto{\pgfpoint{151.779724pt}{384.977264pt}}
\pgfusepath{stroke}
\pgfpathmoveto{\pgfpoint{159.874420pt}{392.932190pt}}
\pgflineto{\pgfpoint{155.827072pt}{389.261536pt}}
\pgfusepath{stroke}
\pgfpathmoveto{\pgfpoint{163.921783pt}{395.893646pt}}
\pgflineto{\pgfpoint{159.874420pt}{392.932190pt}}
\pgfusepath{stroke}
\pgfpathmoveto{\pgfpoint{167.969131pt}{398.050232pt}}
\pgflineto{\pgfpoint{163.921783pt}{395.893646pt}}
\pgfusepath{stroke}
\pgfpathmoveto{\pgfpoint{172.016479pt}{399.306366pt}}
\pgflineto{\pgfpoint{167.969131pt}{398.050232pt}}
\pgfusepath{stroke}
\pgfpathmoveto{\pgfpoint{176.063843pt}{399.566895pt}}
\pgflineto{\pgfpoint{172.016479pt}{399.306366pt}}
\pgfusepath{stroke}
\pgfpathmoveto{\pgfpoint{180.111191pt}{398.796936pt}}
\pgflineto{\pgfpoint{176.063843pt}{399.566895pt}}
\pgfusepath{stroke}
\pgfpathmoveto{\pgfpoint{184.158539pt}{397.078705pt}}
\pgflineto{\pgfpoint{180.111191pt}{398.796936pt}}
\pgfusepath{stroke}
\pgfpathmoveto{\pgfpoint{188.205902pt}{394.507812pt}}
\pgflineto{\pgfpoint{184.158539pt}{397.078705pt}}
\pgfusepath{stroke}
\pgfpathmoveto{\pgfpoint{192.253250pt}{391.179871pt}}
\pgflineto{\pgfpoint{188.205902pt}{394.507812pt}}
\pgfusepath{stroke}
\pgfpathmoveto{\pgfpoint{196.300598pt}{387.190491pt}}
\pgflineto{\pgfpoint{192.253250pt}{391.179871pt}}
\pgfusepath{stroke}
\pgfpathmoveto{\pgfpoint{200.347961pt}{382.635315pt}}
\pgflineto{\pgfpoint{196.300598pt}{387.190491pt}}
\pgfusepath{stroke}
\pgfpathmoveto{\pgfpoint{204.395294pt}{377.609985pt}}
\pgflineto{\pgfpoint{200.347961pt}{382.635315pt}}
\pgfusepath{stroke}
\pgfpathmoveto{\pgfpoint{208.442657pt}{372.210083pt}}
\pgflineto{\pgfpoint{204.395294pt}{377.609985pt}}
\pgfusepath{stroke}
\pgfpathmoveto{\pgfpoint{212.490021pt}{366.531250pt}}
\pgflineto{\pgfpoint{208.442657pt}{372.210083pt}}
\pgfusepath{stroke}
\pgfpathmoveto{\pgfpoint{216.537369pt}{360.669128pt}}
\pgflineto{\pgfpoint{212.490021pt}{366.531250pt}}
\pgfusepath{stroke}
\pgfpathmoveto{\pgfpoint{220.584732pt}{354.719299pt}}
\pgflineto{\pgfpoint{216.537369pt}{360.669128pt}}
\pgfusepath{stroke}
\pgfpathmoveto{\pgfpoint{224.632080pt}{348.777405pt}}
\pgflineto{\pgfpoint{220.584732pt}{354.719299pt}}
\pgfusepath{stroke}
\pgfpathmoveto{\pgfpoint{228.679443pt}{342.386871pt}}
\pgflineto{\pgfpoint{224.632080pt}{348.777405pt}}
\pgfusepath{stroke}
\pgfpathmoveto{\pgfpoint{232.726791pt}{334.803802pt}}
\pgflineto{\pgfpoint{228.679443pt}{342.386871pt}}
\pgfusepath{stroke}
\pgfpathmoveto{\pgfpoint{236.774155pt}{326.179993pt}}
\pgflineto{\pgfpoint{232.726791pt}{334.803802pt}}
\pgfusepath{stroke}
\pgfpathmoveto{\pgfpoint{240.821503pt}{316.678711pt}}
\pgflineto{\pgfpoint{236.774155pt}{326.179993pt}}
\pgfusepath{stroke}
\pgfpathmoveto{\pgfpoint{244.868835pt}{306.463165pt}}
\pgflineto{\pgfpoint{240.821503pt}{316.678711pt}}
\pgfusepath{stroke}
\pgfpathmoveto{\pgfpoint{248.916199pt}{295.696655pt}}
\pgflineto{\pgfpoint{244.868835pt}{306.463165pt}}
\pgfusepath{stroke}
\pgfpathmoveto{\pgfpoint{252.963562pt}{284.542419pt}}
\pgflineto{\pgfpoint{248.916199pt}{295.696655pt}}
\pgfusepath{stroke}
\pgfpathmoveto{\pgfpoint{257.010895pt}{273.163727pt}}
\pgflineto{\pgfpoint{252.963562pt}{284.542419pt}}
\pgfusepath{stroke}
\pgfpathmoveto{\pgfpoint{261.058258pt}{261.723816pt}}
\pgflineto{\pgfpoint{257.010895pt}{273.163727pt}}
\pgfusepath{stroke}
\pgfpathmoveto{\pgfpoint{265.105621pt}{250.385971pt}}
\pgflineto{\pgfpoint{261.058258pt}{261.723816pt}}
\pgfusepath{stroke}
\pgfpathmoveto{\pgfpoint{269.152985pt}{239.313431pt}}
\pgflineto{\pgfpoint{265.105621pt}{250.385971pt}}
\pgfusepath{stroke}
\pgfpathmoveto{\pgfpoint{273.200317pt}{228.669464pt}}
\pgflineto{\pgfpoint{269.152985pt}{239.313431pt}}
\pgfusepath{stroke}
\pgfpathmoveto{\pgfpoint{277.247681pt}{218.448898pt}}
\pgflineto{\pgfpoint{273.200317pt}{228.669464pt}}
\pgfusepath{stroke}
\pgfpathmoveto{\pgfpoint{281.295044pt}{207.804901pt}}
\pgflineto{\pgfpoint{277.247681pt}{218.448898pt}}
\pgfusepath{stroke}
\pgfpathmoveto{\pgfpoint{285.342377pt}{196.732422pt}}
\pgflineto{\pgfpoint{281.295044pt}{207.804901pt}}
\pgfusepath{stroke}
\pgfpathmoveto{\pgfpoint{289.389740pt}{185.394684pt}}
\pgflineto{\pgfpoint{285.342377pt}{196.732422pt}}
\pgfusepath{stroke}
\pgfpathmoveto{\pgfpoint{293.437103pt}{173.954926pt}}
\pgflineto{\pgfpoint{289.389740pt}{185.394684pt}}
\pgfusepath{stroke}
\pgfpathmoveto{\pgfpoint{297.484436pt}{162.576431pt}}
\pgflineto{\pgfpoint{293.437103pt}{173.954926pt}}
\pgfusepath{stroke}
\pgfpathmoveto{\pgfpoint{301.531799pt}{151.422409pt}}
\pgflineto{\pgfpoint{297.484436pt}{162.576431pt}}
\pgfusepath{stroke}
\pgfpathmoveto{\pgfpoint{305.579163pt}{140.656113pt}}
\pgflineto{\pgfpoint{301.531799pt}{151.422409pt}}
\pgfusepath{stroke}
\pgfpathmoveto{\pgfpoint{309.626495pt}{130.440796pt}}
\pgflineto{\pgfpoint{305.579163pt}{140.656113pt}}
\pgfusepath{stroke}
\pgfpathmoveto{\pgfpoint{313.673859pt}{120.939674pt}}
\pgflineto{\pgfpoint{309.626495pt}{130.440796pt}}
\pgfusepath{stroke}
\pgfpathmoveto{\pgfpoint{317.721222pt}{112.316017pt}}
\pgflineto{\pgfpoint{313.673859pt}{120.939674pt}}
\pgfusepath{stroke}
\pgfpathmoveto{\pgfpoint{321.768555pt}{104.733055pt}}
\pgflineto{\pgfpoint{317.721222pt}{112.316017pt}}
\pgfusepath{stroke}
\pgfpathmoveto{\pgfpoint{325.815918pt}{98.342590pt}}
\pgflineto{\pgfpoint{321.768555pt}{104.733055pt}}
\pgfusepath{stroke}
\pgfpathmoveto{\pgfpoint{329.863281pt}{92.400719pt}}
\pgflineto{\pgfpoint{325.815918pt}{98.342590pt}}
\pgfusepath{stroke}
\pgfpathmoveto{\pgfpoint{333.910614pt}{86.450897pt}}
\pgflineto{\pgfpoint{329.863281pt}{92.400719pt}}
\pgfusepath{stroke}
\pgfpathmoveto{\pgfpoint{337.957977pt}{80.588760pt}}
\pgflineto{\pgfpoint{333.910614pt}{86.450897pt}}
\pgfusepath{stroke}
\pgfpathmoveto{\pgfpoint{342.005310pt}{74.909943pt}}
\pgflineto{\pgfpoint{337.957977pt}{80.588760pt}}
\pgfusepath{stroke}
\pgfpathmoveto{\pgfpoint{346.052673pt}{69.510040pt}}
\pgflineto{\pgfpoint{342.005310pt}{74.909943pt}}
\pgfusepath{stroke}
\pgfpathmoveto{\pgfpoint{350.100037pt}{64.484695pt}}
\pgflineto{\pgfpoint{346.052673pt}{69.510040pt}}
\pgfusepath{stroke}
\pgfpathmoveto{\pgfpoint{354.147369pt}{59.929520pt}}
\pgflineto{\pgfpoint{350.100037pt}{64.484695pt}}
\pgfusepath{stroke}
\pgfpathmoveto{\pgfpoint{358.194733pt}{55.940140pt}}
\pgflineto{\pgfpoint{354.147369pt}{59.929520pt}}
\pgfusepath{stroke}
\pgfpathmoveto{\pgfpoint{362.242096pt}{52.612198pt}}
\pgflineto{\pgfpoint{358.194733pt}{55.940140pt}}
\pgfusepath{stroke}
\pgfpathmoveto{\pgfpoint{366.289459pt}{50.041290pt}}
\pgflineto{\pgfpoint{362.242096pt}{52.612198pt}}
\pgfusepath{stroke}
\pgfpathmoveto{\pgfpoint{370.336823pt}{48.323044pt}}
\pgflineto{\pgfpoint{366.289459pt}{50.041290pt}}
\pgfusepath{stroke}
\pgfpathmoveto{\pgfpoint{374.384155pt}{47.553101pt}}
\pgflineto{\pgfpoint{370.336823pt}{48.323044pt}}
\pgfusepath{stroke}
\pgfpathmoveto{\pgfpoint{378.431519pt}{47.813629pt}}
\pgflineto{\pgfpoint{374.384155pt}{47.553101pt}}
\pgfusepath{stroke}
\pgfpathmoveto{\pgfpoint{382.478882pt}{49.069733pt}}
\pgflineto{\pgfpoint{378.431519pt}{47.813629pt}}
\pgfusepath{stroke}
\pgfpathmoveto{\pgfpoint{386.526245pt}{51.226334pt}}
\pgflineto{\pgfpoint{382.478882pt}{49.069733pt}}
\pgfusepath{stroke}
\pgfpathmoveto{\pgfpoint{390.573578pt}{54.187775pt}}
\pgflineto{\pgfpoint{386.526245pt}{51.226334pt}}
\pgfusepath{stroke}
\pgfpathmoveto{\pgfpoint{394.620941pt}{57.858444pt}}
\pgflineto{\pgfpoint{390.573578pt}{54.187775pt}}
\pgfusepath{stroke}
\pgfpathmoveto{\pgfpoint{398.668304pt}{62.142731pt}}
\pgflineto{\pgfpoint{394.620941pt}{57.858444pt}}
\pgfusepath{stroke}
\pgfpathmoveto{\pgfpoint{402.715637pt}{66.944962pt}}
\pgflineto{\pgfpoint{398.668304pt}{62.142731pt}}
\pgfusepath{stroke}
\pgfpathmoveto{\pgfpoint{406.763000pt}{72.169571pt}}
\pgflineto{\pgfpoint{402.715637pt}{66.944962pt}}
\pgfusepath{stroke}
\pgfpathmoveto{\pgfpoint{410.810364pt}{77.720917pt}}
\pgflineto{\pgfpoint{406.763000pt}{72.169571pt}}
\pgfusepath{stroke}
\pgfpathmoveto{\pgfpoint{414.857697pt}{83.503357pt}}
\pgflineto{\pgfpoint{410.810364pt}{77.720917pt}}
\pgfusepath{stroke}
\pgfpathmoveto{\pgfpoint{418.905060pt}{89.421288pt}}
\pgflineto{\pgfpoint{414.857697pt}{83.503357pt}}
\pgfusepath{stroke}
\pgfpathmoveto{\pgfpoint{422.952423pt}{95.379089pt}}
\pgflineto{\pgfpoint{418.905060pt}{89.421288pt}}
\pgfusepath{stroke}
\pgfpathmoveto{\pgfpoint{426.999756pt}{101.350883pt}}
\pgflineto{\pgfpoint{422.952423pt}{95.379089pt}}
\pgfusepath{stroke}
\pgfpathmoveto{\pgfpoint{431.047119pt}{107.984734pt}}
\pgflineto{\pgfpoint{426.999756pt}{101.350883pt}}
\pgfusepath{stroke}
\pgfpathmoveto{\pgfpoint{435.094482pt}{115.423470pt}}
\pgflineto{\pgfpoint{431.047119pt}{107.984734pt}}
\pgfusepath{stroke}
\pgfpathmoveto{\pgfpoint{439.141785pt}{123.619255pt}}
\pgflineto{\pgfpoint{435.094482pt}{115.423470pt}}
\pgfusepath{stroke}
\pgfpathmoveto{\pgfpoint{443.189148pt}{132.524292pt}}
\pgflineto{\pgfpoint{439.141785pt}{123.619255pt}}
\pgfusepath{stroke}
\pgfpathmoveto{\pgfpoint{447.236511pt}{142.090775pt}}
\pgflineto{\pgfpoint{443.189148pt}{132.524292pt}}
\pgfusepath{stroke}
\pgfpathmoveto{\pgfpoint{451.283875pt}{152.270874pt}}
\pgflineto{\pgfpoint{447.236511pt}{142.090775pt}}
\pgfusepath{stroke}
\pgfpathmoveto{\pgfpoint{455.331238pt}{163.016800pt}}
\pgflineto{\pgfpoint{451.283875pt}{152.270874pt}}
\pgfusepath{stroke}
\pgfpathmoveto{\pgfpoint{459.378601pt}{174.280701pt}}
\pgflineto{\pgfpoint{455.331238pt}{163.016800pt}}
\pgfusepath{stroke}
\pgfpathmoveto{\pgfpoint{463.425964pt}{186.014786pt}}
\pgflineto{\pgfpoint{459.378601pt}{174.280701pt}}
\pgfusepath{stroke}
\pgfpathmoveto{\pgfpoint{467.473267pt}{198.171249pt}}
\pgflineto{\pgfpoint{463.425964pt}{186.014786pt}}
\pgfusepath{stroke}
\pgfpathmoveto{\pgfpoint{471.520630pt}{210.702255pt}}
\pgflineto{\pgfpoint{467.473267pt}{198.171249pt}}
\pgfusepath{stroke}
\pgfpathmoveto{\pgfpoint{475.567993pt}{223.559998pt}}
\pgflineto{\pgfpoint{471.520630pt}{210.702255pt}}
\pgfusepath{stroke}
\end{pgfscope}
\end{pgfpicture}
}
\begin{verbatim}
octave> interplot (points, B, "spline")
\end{verbatim}
\scalebox{0.61}{% Title: gl2ps_renderer figure
% Creator: GL2PS 1.4.0, (C) 1999-2017 C. Geuzaine
% For: Octave
% CreationDate: Fri Oct 25 16:20:52 2019
\begin{pgfpicture}
\color[rgb]{1.000000,1.000000,1.000000}
\pgfpathrectanglecorners{\pgfpoint{0pt}{0pt}}{\pgfpoint{576pt}{432pt}}
\pgfusepath{fill}
\begin{pgfscope}
\pgfpathrectangle{\pgfpoint{0pt}{0pt}}{\pgfpoint{576pt}{432pt}}
\pgfusepath{fill}
\pgfpathrectangle{\pgfpoint{0pt}{0pt}}{\pgfpoint{576pt}{432pt}}
\pgfusepath{clip}
\pgfpathmoveto{\pgfpoint{74.880005pt}{399.599976pt}}
\pgflineto{\pgfpoint{521.279968pt}{47.519989pt}}
\pgflineto{\pgfpoint{74.880005pt}{47.519989pt}}
\pgfpathclose
\pgfusepath{fill,stroke}
\pgfpathmoveto{\pgfpoint{74.880005pt}{399.599976pt}}
\pgflineto{\pgfpoint{521.279968pt}{399.599976pt}}
\pgflineto{\pgfpoint{521.279968pt}{47.519989pt}}
\pgfpathclose
\pgfusepath{fill,stroke}
\color[rgb]{0.150000,0.150000,0.150000}
\pgfsetlinewidth{0.500000pt}
\pgfpathmoveto{\pgfpoint{74.880005pt}{51.985016pt}}
\pgflineto{\pgfpoint{74.880005pt}{47.519989pt}}
\pgfusepath{stroke}
\pgfpathmoveto{\pgfpoint{74.880005pt}{395.134979pt}}
\pgflineto{\pgfpoint{74.880005pt}{399.599976pt}}
\pgfusepath{stroke}
\pgfpathmoveto{\pgfpoint{138.651428pt}{51.985016pt}}
\pgflineto{\pgfpoint{138.651428pt}{47.519989pt}}
\pgfusepath{stroke}
\pgfpathmoveto{\pgfpoint{138.651428pt}{395.134979pt}}
\pgflineto{\pgfpoint{138.651428pt}{399.599976pt}}
\pgfusepath{stroke}
\pgfpathmoveto{\pgfpoint{202.422852pt}{51.985016pt}}
\pgflineto{\pgfpoint{202.422852pt}{47.519989pt}}
\pgfusepath{stroke}
\pgfpathmoveto{\pgfpoint{202.422852pt}{395.134979pt}}
\pgflineto{\pgfpoint{202.422852pt}{399.599976pt}}
\pgfusepath{stroke}
\pgfpathmoveto{\pgfpoint{266.194275pt}{51.985016pt}}
\pgflineto{\pgfpoint{266.194275pt}{47.519989pt}}
\pgfusepath{stroke}
\pgfpathmoveto{\pgfpoint{266.194275pt}{395.134979pt}}
\pgflineto{\pgfpoint{266.194275pt}{399.599976pt}}
\pgfusepath{stroke}
\pgfpathmoveto{\pgfpoint{329.965698pt}{51.985016pt}}
\pgflineto{\pgfpoint{329.965698pt}{47.519989pt}}
\pgfusepath{stroke}
\pgfpathmoveto{\pgfpoint{329.965698pt}{395.134979pt}}
\pgflineto{\pgfpoint{329.965698pt}{399.599976pt}}
\pgfusepath{stroke}
\pgfpathmoveto{\pgfpoint{393.737152pt}{51.985016pt}}
\pgflineto{\pgfpoint{393.737152pt}{47.519989pt}}
\pgfusepath{stroke}
\pgfpathmoveto{\pgfpoint{393.737152pt}{395.134979pt}}
\pgflineto{\pgfpoint{393.737152pt}{399.599976pt}}
\pgfusepath{stroke}
\pgfpathmoveto{\pgfpoint{457.508575pt}{51.985016pt}}
\pgflineto{\pgfpoint{457.508575pt}{47.519989pt}}
\pgfusepath{stroke}
\pgfpathmoveto{\pgfpoint{457.508575pt}{395.134979pt}}
\pgflineto{\pgfpoint{457.508575pt}{399.599976pt}}
\pgfusepath{stroke}
\pgfpathmoveto{\pgfpoint{521.279968pt}{51.985016pt}}
\pgflineto{\pgfpoint{521.279968pt}{47.519989pt}}
\pgfusepath{stroke}
\pgfpathmoveto{\pgfpoint{521.279968pt}{395.134979pt}}
\pgflineto{\pgfpoint{521.279968pt}{399.599976pt}}
\pgfusepath{stroke}
{
\pgftransformshift{\pgfpoint{74.880005pt}{40.018295pt}}
\pgfnode{rectangle}{north}{\fontsize{10}{0}\selectfont\textcolor[rgb]{0.15,0.15,0.15}{{0}}}{}{\pgfusepath{discard}}}
{
\pgftransformshift{\pgfpoint{138.651428pt}{40.018295pt}}
\pgfnode{rectangle}{north}{\fontsize{10}{0}\selectfont\textcolor[rgb]{0.15,0.15,0.15}{{1}}}{}{\pgfusepath{discard}}}
{
\pgftransformshift{\pgfpoint{202.422852pt}{40.018295pt}}
\pgfnode{rectangle}{north}{\fontsize{10}{0}\selectfont\textcolor[rgb]{0.15,0.15,0.15}{{2}}}{}{\pgfusepath{discard}}}
{
\pgftransformshift{\pgfpoint{266.194305pt}{40.018295pt}}
\pgfnode{rectangle}{north}{\fontsize{10}{0}\selectfont\textcolor[rgb]{0.15,0.15,0.15}{{3}}}{}{\pgfusepath{discard}}}
{
\pgftransformshift{\pgfpoint{329.965698pt}{40.018295pt}}
\pgfnode{rectangle}{north}{\fontsize{10}{0}\selectfont\textcolor[rgb]{0.15,0.15,0.15}{{4}}}{}{\pgfusepath{discard}}}
{
\pgftransformshift{\pgfpoint{393.737122pt}{40.018295pt}}
\pgfnode{rectangle}{north}{\fontsize{10}{0}\selectfont\textcolor[rgb]{0.15,0.15,0.15}{{5}}}{}{\pgfusepath{discard}}}
{
\pgftransformshift{\pgfpoint{457.508606pt}{40.018295pt}}
\pgfnode{rectangle}{north}{\fontsize{10}{0}\selectfont\textcolor[rgb]{0.15,0.15,0.15}{{6}}}{}{\pgfusepath{discard}}}
{
\pgftransformshift{\pgfpoint{521.279968pt}{40.018295pt}}
\pgfnode{rectangle}{north}{\fontsize{10}{0}\selectfont\textcolor[rgb]{0.15,0.15,0.15}{{7}}}{}{\pgfusepath{discard}}}
\pgfpathmoveto{\pgfpoint{79.347992pt}{47.519989pt}}
\pgflineto{\pgfpoint{74.880005pt}{47.519989pt}}
\pgfusepath{stroke}
\pgfpathmoveto{\pgfpoint{516.812012pt}{47.519989pt}}
\pgflineto{\pgfpoint{521.279968pt}{47.519989pt}}
\pgfusepath{stroke}
\pgfpathmoveto{\pgfpoint{79.347992pt}{135.539993pt}}
\pgflineto{\pgfpoint{74.880005pt}{135.539993pt}}
\pgfusepath{stroke}
\pgfpathmoveto{\pgfpoint{516.812012pt}{135.539993pt}}
\pgflineto{\pgfpoint{521.279968pt}{135.539993pt}}
\pgfusepath{stroke}
\pgfpathmoveto{\pgfpoint{79.347992pt}{223.559998pt}}
\pgflineto{\pgfpoint{74.880005pt}{223.559998pt}}
\pgfusepath{stroke}
\pgfpathmoveto{\pgfpoint{516.812012pt}{223.559998pt}}
\pgflineto{\pgfpoint{521.279968pt}{223.559998pt}}
\pgfusepath{stroke}
\pgfpathmoveto{\pgfpoint{79.347992pt}{311.579987pt}}
\pgflineto{\pgfpoint{74.880005pt}{311.579987pt}}
\pgfusepath{stroke}
\pgfpathmoveto{\pgfpoint{516.812012pt}{311.579987pt}}
\pgflineto{\pgfpoint{521.279968pt}{311.579987pt}}
\pgfusepath{stroke}
\pgfpathmoveto{\pgfpoint{79.347992pt}{399.599976pt}}
\pgflineto{\pgfpoint{74.880005pt}{399.599976pt}}
\pgfusepath{stroke}
\pgfpathmoveto{\pgfpoint{516.812012pt}{399.599976pt}}
\pgflineto{\pgfpoint{521.279968pt}{399.599976pt}}
\pgfusepath{stroke}
{
\pgftransformshift{\pgfpoint{69.875519pt}{47.519989pt}}
\pgfnode{rectangle}{east}{\fontsize{10}{0}\selectfont\textcolor[rgb]{0.15,0.15,0.15}{{-1}}}{}{\pgfusepath{discard}}}
{
\pgftransformshift{\pgfpoint{69.875519pt}{135.539993pt}}
\pgfnode{rectangle}{east}{\fontsize{10}{0}\selectfont\textcolor[rgb]{0.15,0.15,0.15}{{-0.5}}}{}{\pgfusepath{discard}}}
{
\pgftransformshift{\pgfpoint{69.875519pt}{223.559998pt}}
\pgfnode{rectangle}{east}{\fontsize{10}{0}\selectfont\textcolor[rgb]{0.15,0.15,0.15}{{0}}}{}{\pgfusepath{discard}}}
{
\pgftransformshift{\pgfpoint{69.875519pt}{311.579987pt}}
\pgfnode{rectangle}{east}{\fontsize{10}{0}\selectfont\textcolor[rgb]{0.15,0.15,0.15}{{0.5}}}{}{\pgfusepath{discard}}}
{
\pgftransformshift{\pgfpoint{69.875519pt}{399.599976pt}}
\pgfnode{rectangle}{east}{\fontsize{10}{0}\selectfont\textcolor[rgb]{0.15,0.15,0.15}{{1}}}{}{\pgfusepath{discard}}}
\pgfsetrectcap
\pgfsetdash{{16pt}{0pt}}{0pt}
\pgfpathmoveto{\pgfpoint{521.279968pt}{47.519989pt}}
\pgflineto{\pgfpoint{74.880005pt}{47.519989pt}}
\pgfusepath{stroke}
\pgfpathmoveto{\pgfpoint{521.279968pt}{399.599976pt}}
\pgflineto{\pgfpoint{74.880005pt}{399.599976pt}}
\pgfusepath{stroke}
\pgfpathmoveto{\pgfpoint{74.880005pt}{399.599976pt}}
\pgflineto{\pgfpoint{74.880005pt}{47.519989pt}}
\pgfusepath{stroke}
\pgfpathmoveto{\pgfpoint{521.279968pt}{399.599976pt}}
\pgflineto{\pgfpoint{521.279968pt}{47.519989pt}}
\pgfusepath{stroke}
\color[rgb]{0.000000,0.447000,0.741000}
\pgfsetroundcap
\pgfsetroundjoin
\pgfsetdash{}{0pt}
\pgfpathmoveto{\pgfpoint{268.621338pt}{246.571579pt}}
\pgflineto{\pgfpoint{269.194275pt}{248.334930pt}}
\pgfusepath{stroke}
\pgfpathmoveto{\pgfpoint{267.121338pt}{245.481766pt}}
\pgflineto{\pgfpoint{268.621338pt}{246.571579pt}}
\pgfusepath{stroke}
\pgfpathmoveto{\pgfpoint{265.267242pt}{245.481766pt}}
\pgflineto{\pgfpoint{267.121338pt}{245.481766pt}}
\pgfusepath{stroke}
\pgfpathmoveto{\pgfpoint{263.767242pt}{246.571579pt}}
\pgflineto{\pgfpoint{265.267242pt}{245.481766pt}}
\pgfusepath{stroke}
\pgfpathmoveto{\pgfpoint{263.194275pt}{248.334930pt}}
\pgflineto{\pgfpoint{263.767242pt}{246.571579pt}}
\pgfusepath{stroke}
\pgfpathmoveto{\pgfpoint{263.767242pt}{250.098297pt}}
\pgflineto{\pgfpoint{263.194275pt}{248.334930pt}}
\pgfusepath{stroke}
\pgfpathmoveto{\pgfpoint{265.267242pt}{251.188110pt}}
\pgflineto{\pgfpoint{263.767242pt}{250.098297pt}}
\pgfusepath{stroke}
\pgfpathmoveto{\pgfpoint{267.121338pt}{251.188110pt}}
\pgflineto{\pgfpoint{265.267242pt}{251.188110pt}}
\pgfusepath{stroke}
\pgfpathmoveto{\pgfpoint{268.621338pt}{250.098297pt}}
\pgflineto{\pgfpoint{267.121338pt}{251.188110pt}}
\pgfusepath{stroke}
\pgfpathmoveto{\pgfpoint{269.194275pt}{248.334930pt}}
\pgflineto{\pgfpoint{268.621338pt}{250.098297pt}}
\pgfusepath{stroke}
\pgfpathmoveto{\pgfpoint{364.278503pt}{49.727097pt}}
\pgflineto{\pgfpoint{364.851440pt}{51.490463pt}}
\pgfusepath{stroke}
\pgfpathmoveto{\pgfpoint{362.778503pt}{48.637283pt}}
\pgflineto{\pgfpoint{364.278503pt}{49.727097pt}}
\pgfusepath{stroke}
\pgfpathmoveto{\pgfpoint{360.924377pt}{48.637283pt}}
\pgflineto{\pgfpoint{362.778503pt}{48.637283pt}}
\pgfusepath{stroke}
\pgfpathmoveto{\pgfpoint{359.424377pt}{49.727097pt}}
\pgflineto{\pgfpoint{360.924377pt}{48.637283pt}}
\pgfusepath{stroke}
\pgfpathmoveto{\pgfpoint{358.851440pt}{51.490463pt}}
\pgflineto{\pgfpoint{359.424377pt}{49.727097pt}}
\pgfusepath{stroke}
\pgfpathmoveto{\pgfpoint{359.424377pt}{53.253815pt}}
\pgflineto{\pgfpoint{358.851440pt}{51.490463pt}}
\pgfusepath{stroke}
\pgfpathmoveto{\pgfpoint{360.924377pt}{54.343628pt}}
\pgflineto{\pgfpoint{359.424377pt}{53.253815pt}}
\pgfusepath{stroke}
\pgfpathmoveto{\pgfpoint{362.778503pt}{54.343628pt}}
\pgflineto{\pgfpoint{360.924377pt}{54.343628pt}}
\pgfusepath{stroke}
\pgfpathmoveto{\pgfpoint{364.278503pt}{53.253815pt}}
\pgflineto{\pgfpoint{362.778503pt}{54.343628pt}}
\pgfusepath{stroke}
\pgfpathmoveto{\pgfpoint{364.851440pt}{51.490463pt}}
\pgflineto{\pgfpoint{364.278503pt}{53.253815pt}}
\pgfusepath{stroke}
\color[rgb]{0.850000,0.325000,0.098000}
\pgfsetbuttcap
\pgfpathmoveto{\pgfpoint{78.927338pt}{235.294525pt}}
\pgflineto{\pgfpoint{74.880005pt}{223.559998pt}}
\pgfusepath{stroke}
\pgfpathmoveto{\pgfpoint{82.974701pt}{246.813751pt}}
\pgflineto{\pgfpoint{78.927338pt}{235.294525pt}}
\pgfusepath{stroke}
\pgfpathmoveto{\pgfpoint{87.022049pt}{258.094757pt}}
\pgflineto{\pgfpoint{82.974701pt}{246.813751pt}}
\pgfusepath{stroke}
\pgfpathmoveto{\pgfpoint{91.069412pt}{269.114594pt}}
\pgflineto{\pgfpoint{87.022049pt}{258.094757pt}}
\pgfusepath{stroke}
\pgfpathmoveto{\pgfpoint{95.116760pt}{279.850342pt}}
\pgflineto{\pgfpoint{91.069412pt}{269.114594pt}}
\pgfusepath{stroke}
\pgfpathmoveto{\pgfpoint{99.164124pt}{290.279022pt}}
\pgflineto{\pgfpoint{95.116760pt}{279.850342pt}}
\pgfusepath{stroke}
\pgfpathmoveto{\pgfpoint{103.211472pt}{300.377777pt}}
\pgflineto{\pgfpoint{99.164124pt}{290.279022pt}}
\pgfusepath{stroke}
\pgfpathmoveto{\pgfpoint{107.258820pt}{310.123627pt}}
\pgflineto{\pgfpoint{103.211472pt}{300.377777pt}}
\pgfusepath{stroke}
\pgfpathmoveto{\pgfpoint{111.306183pt}{319.493652pt}}
\pgflineto{\pgfpoint{107.258820pt}{310.123627pt}}
\pgfusepath{stroke}
\pgfpathmoveto{\pgfpoint{115.353531pt}{328.464935pt}}
\pgflineto{\pgfpoint{111.306183pt}{319.493652pt}}
\pgfusepath{stroke}
\pgfpathmoveto{\pgfpoint{119.400894pt}{337.014496pt}}
\pgflineto{\pgfpoint{115.353531pt}{328.464935pt}}
\pgfusepath{stroke}
\pgfpathmoveto{\pgfpoint{123.448242pt}{345.119446pt}}
\pgflineto{\pgfpoint{119.400894pt}{337.014496pt}}
\pgfusepath{stroke}
\pgfpathmoveto{\pgfpoint{127.495605pt}{352.756836pt}}
\pgflineto{\pgfpoint{123.448242pt}{345.119446pt}}
\pgfusepath{stroke}
\pgfpathmoveto{\pgfpoint{131.542938pt}{359.903748pt}}
\pgflineto{\pgfpoint{127.495605pt}{352.756836pt}}
\pgfusepath{stroke}
\pgfpathmoveto{\pgfpoint{135.590302pt}{366.537201pt}}
\pgflineto{\pgfpoint{131.542938pt}{359.903748pt}}
\pgfusepath{stroke}
\pgfpathmoveto{\pgfpoint{139.637650pt}{372.634338pt}}
\pgflineto{\pgfpoint{135.590302pt}{366.537201pt}}
\pgfusepath{stroke}
\pgfpathmoveto{\pgfpoint{143.684998pt}{378.172180pt}}
\pgflineto{\pgfpoint{139.637650pt}{372.634338pt}}
\pgfusepath{stroke}
\pgfpathmoveto{\pgfpoint{147.732361pt}{383.127808pt}}
\pgflineto{\pgfpoint{143.684998pt}{378.172180pt}}
\pgfusepath{stroke}
\pgfpathmoveto{\pgfpoint{151.779724pt}{387.478241pt}}
\pgflineto{\pgfpoint{147.732361pt}{383.127808pt}}
\pgfusepath{stroke}
\pgfpathmoveto{\pgfpoint{155.827072pt}{391.200623pt}}
\pgflineto{\pgfpoint{151.779724pt}{387.478241pt}}
\pgfusepath{stroke}
\pgfpathmoveto{\pgfpoint{159.874420pt}{394.271973pt}}
\pgflineto{\pgfpoint{155.827072pt}{391.200623pt}}
\pgfusepath{stroke}
\pgfpathmoveto{\pgfpoint{163.921783pt}{396.669403pt}}
\pgflineto{\pgfpoint{159.874420pt}{394.271973pt}}
\pgfusepath{stroke}
\pgfpathmoveto{\pgfpoint{167.969131pt}{398.369934pt}}
\pgflineto{\pgfpoint{163.921783pt}{396.669403pt}}
\pgfusepath{stroke}
\pgfpathmoveto{\pgfpoint{172.016479pt}{399.350647pt}}
\pgflineto{\pgfpoint{167.969131pt}{398.369934pt}}
\pgfusepath{stroke}
\pgfpathmoveto{\pgfpoint{176.063843pt}{399.588684pt}}
\pgflineto{\pgfpoint{172.016479pt}{399.350647pt}}
\pgfusepath{stroke}
\pgfpathmoveto{\pgfpoint{180.111191pt}{399.074524pt}}
\pgflineto{\pgfpoint{176.063843pt}{399.588684pt}}
\pgfusepath{stroke}
\pgfpathmoveto{\pgfpoint{184.158539pt}{397.824219pt}}
\pgflineto{\pgfpoint{180.111191pt}{399.074524pt}}
\pgfusepath{stroke}
\pgfpathmoveto{\pgfpoint{188.205902pt}{395.856812pt}}
\pgflineto{\pgfpoint{184.158539pt}{397.824219pt}}
\pgfusepath{stroke}
\pgfpathmoveto{\pgfpoint{192.253250pt}{393.191376pt}}
\pgflineto{\pgfpoint{188.205902pt}{395.856812pt}}
\pgfusepath{stroke}
\pgfpathmoveto{\pgfpoint{196.300598pt}{389.846893pt}}
\pgflineto{\pgfpoint{192.253250pt}{393.191376pt}}
\pgfusepath{stroke}
\pgfpathmoveto{\pgfpoint{200.347961pt}{385.842468pt}}
\pgflineto{\pgfpoint{196.300598pt}{389.846893pt}}
\pgfusepath{stroke}
\pgfpathmoveto{\pgfpoint{204.395294pt}{381.197113pt}}
\pgflineto{\pgfpoint{200.347961pt}{385.842468pt}}
\pgfusepath{stroke}
\pgfpathmoveto{\pgfpoint{208.442657pt}{375.929871pt}}
\pgflineto{\pgfpoint{204.395294pt}{381.197113pt}}
\pgfusepath{stroke}
\pgfpathmoveto{\pgfpoint{212.490021pt}{370.059814pt}}
\pgflineto{\pgfpoint{208.442657pt}{375.929871pt}}
\pgfusepath{stroke}
\pgfpathmoveto{\pgfpoint{216.537369pt}{363.605957pt}}
\pgflineto{\pgfpoint{212.490021pt}{370.059814pt}}
\pgfusepath{stroke}
\pgfpathmoveto{\pgfpoint{220.584732pt}{356.587341pt}}
\pgflineto{\pgfpoint{216.537369pt}{363.605957pt}}
\pgfusepath{stroke}
\pgfpathmoveto{\pgfpoint{224.632080pt}{349.023071pt}}
\pgflineto{\pgfpoint{220.584732pt}{356.587341pt}}
\pgfusepath{stroke}
\pgfpathmoveto{\pgfpoint{228.679443pt}{340.934723pt}}
\pgflineto{\pgfpoint{224.632080pt}{349.023071pt}}
\pgfusepath{stroke}
\pgfpathmoveto{\pgfpoint{232.726791pt}{332.359161pt}}
\pgflineto{\pgfpoint{228.679443pt}{340.934723pt}}
\pgfusepath{stroke}
\pgfpathmoveto{\pgfpoint{236.774155pt}{323.338776pt}}
\pgflineto{\pgfpoint{232.726791pt}{332.359161pt}}
\pgfusepath{stroke}
\pgfpathmoveto{\pgfpoint{240.821503pt}{313.915924pt}}
\pgflineto{\pgfpoint{236.774155pt}{323.338776pt}}
\pgfusepath{stroke}
\pgfpathmoveto{\pgfpoint{244.868835pt}{304.132996pt}}
\pgflineto{\pgfpoint{240.821503pt}{313.915924pt}}
\pgfusepath{stroke}
\pgfpathmoveto{\pgfpoint{248.916199pt}{294.032288pt}}
\pgflineto{\pgfpoint{244.868835pt}{304.132996pt}}
\pgfusepath{stroke}
\pgfpathmoveto{\pgfpoint{252.963562pt}{283.656250pt}}
\pgflineto{\pgfpoint{248.916199pt}{294.032288pt}}
\pgfusepath{stroke}
\pgfpathmoveto{\pgfpoint{257.010895pt}{273.047180pt}}
\pgflineto{\pgfpoint{252.963562pt}{283.656250pt}}
\pgfusepath{stroke}
\pgfpathmoveto{\pgfpoint{261.058258pt}{262.247528pt}}
\pgflineto{\pgfpoint{257.010895pt}{273.047180pt}}
\pgfusepath{stroke}
\pgfpathmoveto{\pgfpoint{265.105621pt}{251.299576pt}}
\pgflineto{\pgfpoint{261.058258pt}{262.247528pt}}
\pgfusepath{stroke}
\pgfpathmoveto{\pgfpoint{269.152985pt}{240.245743pt}}
\pgflineto{\pgfpoint{265.105621pt}{251.299576pt}}
\pgfusepath{stroke}
\pgfpathmoveto{\pgfpoint{273.200317pt}{229.128372pt}}
\pgflineto{\pgfpoint{269.152985pt}{240.245743pt}}
\pgfusepath{stroke}
\pgfpathmoveto{\pgfpoint{277.247681pt}{217.989853pt}}
\pgflineto{\pgfpoint{273.200317pt}{229.128372pt}}
\pgfusepath{stroke}
\pgfpathmoveto{\pgfpoint{281.295044pt}{206.872543pt}}
\pgflineto{\pgfpoint{277.247681pt}{217.989853pt}}
\pgfusepath{stroke}
\pgfpathmoveto{\pgfpoint{285.342377pt}{195.818802pt}}
\pgflineto{\pgfpoint{281.295044pt}{206.872543pt}}
\pgfusepath{stroke}
\pgfpathmoveto{\pgfpoint{289.389740pt}{184.870972pt}}
\pgflineto{\pgfpoint{285.342377pt}{195.818802pt}}
\pgfusepath{stroke}
\pgfpathmoveto{\pgfpoint{293.437103pt}{174.071442pt}}
\pgflineto{\pgfpoint{289.389740pt}{184.870972pt}}
\pgfusepath{stroke}
\pgfpathmoveto{\pgfpoint{297.484436pt}{163.462570pt}}
\pgflineto{\pgfpoint{293.437103pt}{174.071442pt}}
\pgfusepath{stroke}
\pgfpathmoveto{\pgfpoint{301.531799pt}{153.086700pt}}
\pgflineto{\pgfpoint{297.484436pt}{163.462570pt}}
\pgfusepath{stroke}
\pgfpathmoveto{\pgfpoint{305.579163pt}{142.986206pt}}
\pgflineto{\pgfpoint{301.531799pt}{153.086700pt}}
\pgfusepath{stroke}
\pgfpathmoveto{\pgfpoint{309.626495pt}{133.203430pt}}
\pgflineto{\pgfpoint{305.579163pt}{142.986206pt}}
\pgfusepath{stroke}
\pgfpathmoveto{\pgfpoint{313.673859pt}{123.780777pt}}
\pgflineto{\pgfpoint{309.626495pt}{133.203430pt}}
\pgfusepath{stroke}
\pgfpathmoveto{\pgfpoint{317.721222pt}{114.760559pt}}
\pgflineto{\pgfpoint{313.673859pt}{123.780777pt}}
\pgfusepath{stroke}
\pgfpathmoveto{\pgfpoint{321.768555pt}{106.185158pt}}
\pgflineto{\pgfpoint{317.721222pt}{114.760559pt}}
\pgfusepath{stroke}
\pgfpathmoveto{\pgfpoint{325.815918pt}{98.096931pt}}
\pgflineto{\pgfpoint{321.768555pt}{106.185158pt}}
\pgfusepath{stroke}
\pgfpathmoveto{\pgfpoint{329.863281pt}{90.532745pt}}
\pgflineto{\pgfpoint{325.815918pt}{98.096931pt}}
\pgfusepath{stroke}
\pgfpathmoveto{\pgfpoint{333.910614pt}{83.514206pt}}
\pgflineto{\pgfpoint{329.863281pt}{90.532745pt}}
\pgfusepath{stroke}
\pgfpathmoveto{\pgfpoint{337.957977pt}{77.060379pt}}
\pgflineto{\pgfpoint{333.910614pt}{83.514206pt}}
\pgfusepath{stroke}
\pgfpathmoveto{\pgfpoint{342.005310pt}{71.190353pt}}
\pgflineto{\pgfpoint{337.957977pt}{77.060379pt}}
\pgfusepath{stroke}
\pgfpathmoveto{\pgfpoint{346.052673pt}{65.923126pt}}
\pgflineto{\pgfpoint{342.005310pt}{71.190353pt}}
\pgfusepath{stroke}
\pgfpathmoveto{\pgfpoint{350.100037pt}{61.277740pt}}
\pgflineto{\pgfpoint{346.052673pt}{65.923126pt}}
\pgfusepath{stroke}
\pgfpathmoveto{\pgfpoint{354.147369pt}{57.273285pt}}
\pgflineto{\pgfpoint{350.100037pt}{61.277740pt}}
\pgfusepath{stroke}
\pgfpathmoveto{\pgfpoint{358.194733pt}{53.928787pt}}
\pgflineto{\pgfpoint{354.147369pt}{57.273285pt}}
\pgfusepath{stroke}
\pgfpathmoveto{\pgfpoint{362.242096pt}{51.263290pt}}
\pgflineto{\pgfpoint{358.194733pt}{53.928787pt}}
\pgfusepath{stroke}
\pgfpathmoveto{\pgfpoint{366.289459pt}{49.295837pt}}
\pgflineto{\pgfpoint{362.242096pt}{51.263290pt}}
\pgfusepath{stroke}
\pgfpathmoveto{\pgfpoint{370.336823pt}{48.045502pt}}
\pgflineto{\pgfpoint{366.289459pt}{49.295837pt}}
\pgfusepath{stroke}
\pgfpathmoveto{\pgfpoint{374.384155pt}{47.531296pt}}
\pgflineto{\pgfpoint{370.336823pt}{48.045502pt}}
\pgfusepath{stroke}
\pgfpathmoveto{\pgfpoint{378.431519pt}{47.769333pt}}
\pgflineto{\pgfpoint{374.384155pt}{47.531296pt}}
\pgfusepath{stroke}
\pgfpathmoveto{\pgfpoint{382.478882pt}{48.750015pt}}
\pgflineto{\pgfpoint{378.431519pt}{47.769333pt}}
\pgfusepath{stroke}
\pgfpathmoveto{\pgfpoint{386.526245pt}{50.450531pt}}
\pgflineto{\pgfpoint{382.478882pt}{48.750015pt}}
\pgfusepath{stroke}
\pgfpathmoveto{\pgfpoint{390.573578pt}{52.847931pt}}
\pgflineto{\pgfpoint{386.526245pt}{50.450531pt}}
\pgfusepath{stroke}
\pgfpathmoveto{\pgfpoint{394.620941pt}{55.919296pt}}
\pgflineto{\pgfpoint{390.573578pt}{52.847931pt}}
\pgfusepath{stroke}
\pgfpathmoveto{\pgfpoint{398.668304pt}{59.641663pt}}
\pgflineto{\pgfpoint{394.620941pt}{55.919296pt}}
\pgfusepath{stroke}
\pgfpathmoveto{\pgfpoint{402.715637pt}{63.992126pt}}
\pgflineto{\pgfpoint{398.668304pt}{59.641663pt}}
\pgfusepath{stroke}
\pgfpathmoveto{\pgfpoint{406.763000pt}{68.947754pt}}
\pgflineto{\pgfpoint{402.715637pt}{63.992126pt}}
\pgfusepath{stroke}
\pgfpathmoveto{\pgfpoint{410.810364pt}{74.485611pt}}
\pgflineto{\pgfpoint{406.763000pt}{68.947754pt}}
\pgfusepath{stroke}
\pgfpathmoveto{\pgfpoint{414.857697pt}{80.582733pt}}
\pgflineto{\pgfpoint{410.810364pt}{74.485611pt}}
\pgfusepath{stroke}
\pgfpathmoveto{\pgfpoint{418.905060pt}{87.216217pt}}
\pgflineto{\pgfpoint{414.857697pt}{80.582733pt}}
\pgfusepath{stroke}
\pgfpathmoveto{\pgfpoint{422.952423pt}{94.363144pt}}
\pgflineto{\pgfpoint{418.905060pt}{87.216217pt}}
\pgfusepath{stroke}
\pgfpathmoveto{\pgfpoint{426.999756pt}{102.000549pt}}
\pgflineto{\pgfpoint{422.952423pt}{94.363144pt}}
\pgfusepath{stroke}
\pgfpathmoveto{\pgfpoint{431.047119pt}{110.105515pt}}
\pgflineto{\pgfpoint{426.999756pt}{102.000549pt}}
\pgfusepath{stroke}
\pgfpathmoveto{\pgfpoint{435.094482pt}{118.655106pt}}
\pgflineto{\pgfpoint{431.047119pt}{110.105515pt}}
\pgfusepath{stroke}
\pgfpathmoveto{\pgfpoint{439.141785pt}{127.626389pt}}
\pgflineto{\pgfpoint{435.094482pt}{118.655106pt}}
\pgfusepath{stroke}
\pgfpathmoveto{\pgfpoint{443.189148pt}{136.996414pt}}
\pgflineto{\pgfpoint{439.141785pt}{127.626389pt}}
\pgfusepath{stroke}
\pgfpathmoveto{\pgfpoint{447.236511pt}{146.742279pt}}
\pgflineto{\pgfpoint{443.189148pt}{136.996414pt}}
\pgfusepath{stroke}
\pgfpathmoveto{\pgfpoint{451.283875pt}{156.841019pt}}
\pgflineto{\pgfpoint{447.236511pt}{146.742279pt}}
\pgfusepath{stroke}
\pgfpathmoveto{\pgfpoint{455.331238pt}{167.269745pt}}
\pgflineto{\pgfpoint{451.283875pt}{156.841019pt}}
\pgfusepath{stroke}
\pgfpathmoveto{\pgfpoint{459.378601pt}{178.005463pt}}
\pgflineto{\pgfpoint{455.331238pt}{167.269745pt}}
\pgfusepath{stroke}
\pgfpathmoveto{\pgfpoint{463.425964pt}{189.025299pt}}
\pgflineto{\pgfpoint{459.378601pt}{178.005463pt}}
\pgfusepath{stroke}
\pgfpathmoveto{\pgfpoint{467.473267pt}{200.306274pt}}
\pgflineto{\pgfpoint{463.425964pt}{189.025299pt}}
\pgfusepath{stroke}
\pgfpathmoveto{\pgfpoint{471.520630pt}{211.825485pt}}
\pgflineto{\pgfpoint{467.473267pt}{200.306274pt}}
\pgfusepath{stroke}
\pgfpathmoveto{\pgfpoint{475.567993pt}{223.559998pt}}
\pgflineto{\pgfpoint{471.520630pt}{211.825485pt}}
\pgfusepath{stroke}
\end{pgfscope}
\end{pgfpicture}
}

    One can easily notice while \verb|nearest| simply chooses the nearest
    neighbor, \verb|cubic| and \verb|spline| both try to \textit{smoothen}
    the curve.  This leads to the fact that \verb|nearest|'s approximations
    strays from \verb|linear|'s in the opposite dirrection when compared to
    the other two's.  It also explains why \verb|cubic|'s and \verb|spline|'s
    results are quite close to each other.

  \item Since we are already extrapolating (by providing the \verb|extrap|
    argument to \verb|interp1|), interpolating for \verb|f(10)| is rather
    straightforward:
\begin{verbatim}
octave> interpolate (10, "spline")
ans =  1.4499
octave> C = linspace (0, 10);
octave> interplot (10, C, "spline")
\end{verbatim}
\scalebox{0.39}{% Title: gl2ps_renderer figure
% Creator: GL2PS 1.4.0, (C) 1999-2017 C. Geuzaine
% For: Octave
% CreationDate: Fri Oct 25 16:22:36 2019
\begin{pgfpicture}
\color[rgb]{1.000000,1.000000,1.000000}
\pgfpathrectanglecorners{\pgfpoint{0pt}{0pt}}{\pgfpoint{576pt}{432pt}}
\pgfusepath{fill}
\begin{pgfscope}
\pgfpathrectangle{\pgfpoint{0pt}{0pt}}{\pgfpoint{576pt}{432pt}}
\pgfusepath{fill}
\pgfpathrectangle{\pgfpoint{0pt}{0pt}}{\pgfpoint{576pt}{432pt}}
\pgfusepath{clip}
\pgfpathmoveto{\pgfpoint{74.880005pt}{399.600006pt}}
\pgflineto{\pgfpoint{521.279968pt}{47.519974pt}}
\pgflineto{\pgfpoint{74.880005pt}{47.519974pt}}
\pgfpathclose
\pgfusepath{fill,stroke}
\pgfpathmoveto{\pgfpoint{74.880005pt}{399.600006pt}}
\pgflineto{\pgfpoint{521.279968pt}{399.600006pt}}
\pgflineto{\pgfpoint{521.279968pt}{47.519974pt}}
\pgfpathclose
\pgfusepath{fill,stroke}
\color[rgb]{0.150000,0.150000,0.150000}
\pgfsetlinewidth{0.500000pt}
\pgfpathmoveto{\pgfpoint{74.880005pt}{51.985001pt}}
\pgflineto{\pgfpoint{74.880005pt}{47.519974pt}}
\pgfusepath{stroke}
\pgfpathmoveto{\pgfpoint{74.880005pt}{395.135010pt}}
\pgflineto{\pgfpoint{74.880005pt}{399.600006pt}}
\pgfusepath{stroke}
\pgfpathmoveto{\pgfpoint{164.160004pt}{51.985001pt}}
\pgflineto{\pgfpoint{164.160004pt}{47.519974pt}}
\pgfusepath{stroke}
\pgfpathmoveto{\pgfpoint{164.160004pt}{395.135010pt}}
\pgflineto{\pgfpoint{164.160004pt}{399.600006pt}}
\pgfusepath{stroke}
\pgfpathmoveto{\pgfpoint{253.440002pt}{51.985001pt}}
\pgflineto{\pgfpoint{253.440002pt}{47.519974pt}}
\pgfusepath{stroke}
\pgfpathmoveto{\pgfpoint{253.440002pt}{395.135010pt}}
\pgflineto{\pgfpoint{253.440002pt}{399.600006pt}}
\pgfusepath{stroke}
\pgfpathmoveto{\pgfpoint{342.720001pt}{51.985001pt}}
\pgflineto{\pgfpoint{342.720001pt}{47.519974pt}}
\pgfusepath{stroke}
\pgfpathmoveto{\pgfpoint{342.720001pt}{395.135010pt}}
\pgflineto{\pgfpoint{342.720001pt}{399.600006pt}}
\pgfusepath{stroke}
\pgfpathmoveto{\pgfpoint{432.000000pt}{51.985001pt}}
\pgflineto{\pgfpoint{432.000000pt}{47.519974pt}}
\pgfusepath{stroke}
\pgfpathmoveto{\pgfpoint{432.000000pt}{395.135010pt}}
\pgflineto{\pgfpoint{432.000000pt}{399.600006pt}}
\pgfusepath{stroke}
\pgfpathmoveto{\pgfpoint{521.279968pt}{51.985001pt}}
\pgflineto{\pgfpoint{521.279968pt}{47.519974pt}}
\pgfusepath{stroke}
\pgfpathmoveto{\pgfpoint{521.279968pt}{395.135010pt}}
\pgflineto{\pgfpoint{521.279968pt}{399.600006pt}}
\pgfusepath{stroke}
{
\pgftransformshift{\pgfpoint{74.880005pt}{40.018295pt}}
\pgfnode{rectangle}{north}{\fontsize{10}{0}\selectfont\textcolor[rgb]{0.15,0.15,0.15}{{0}}}{}{\pgfusepath{discard}}}
{
\pgftransformshift{\pgfpoint{164.159988pt}{40.018295pt}}
\pgfnode{rectangle}{north}{\fontsize{10}{0}\selectfont\textcolor[rgb]{0.15,0.15,0.15}{{2}}}{}{\pgfusepath{discard}}}
{
\pgftransformshift{\pgfpoint{253.440002pt}{40.018295pt}}
\pgfnode{rectangle}{north}{\fontsize{10}{0}\selectfont\textcolor[rgb]{0.15,0.15,0.15}{{4}}}{}{\pgfusepath{discard}}}
{
\pgftransformshift{\pgfpoint{342.720001pt}{40.018295pt}}
\pgfnode{rectangle}{north}{\fontsize{10}{0}\selectfont\textcolor[rgb]{0.15,0.15,0.15}{{6}}}{}{\pgfusepath{discard}}}
{
\pgftransformshift{\pgfpoint{432.000000pt}{40.018295pt}}
\pgfnode{rectangle}{north}{\fontsize{10}{0}\selectfont\textcolor[rgb]{0.15,0.15,0.15}{{8}}}{}{\pgfusepath{discard}}}
{
\pgftransformshift{\pgfpoint{521.279968pt}{40.018295pt}}
\pgfnode{rectangle}{north}{\fontsize{10}{0}\selectfont\textcolor[rgb]{0.15,0.15,0.15}{{10}}}{}{\pgfusepath{discard}}}
\pgfpathmoveto{\pgfpoint{79.347992pt}{47.519974pt}}
\pgflineto{\pgfpoint{74.880005pt}{47.519974pt}}
\pgfusepath{stroke}
\pgfpathmoveto{\pgfpoint{516.812012pt}{47.519974pt}}
\pgflineto{\pgfpoint{521.279968pt}{47.519974pt}}
\pgfusepath{stroke}
\pgfpathmoveto{\pgfpoint{79.347992pt}{97.817131pt}}
\pgflineto{\pgfpoint{74.880005pt}{97.817131pt}}
\pgfusepath{stroke}
\pgfpathmoveto{\pgfpoint{516.812012pt}{97.817131pt}}
\pgflineto{\pgfpoint{521.279968pt}{97.817131pt}}
\pgfusepath{stroke}
\pgfpathmoveto{\pgfpoint{79.347992pt}{148.114273pt}}
\pgflineto{\pgfpoint{74.880005pt}{148.114273pt}}
\pgfusepath{stroke}
\pgfpathmoveto{\pgfpoint{516.812012pt}{148.114273pt}}
\pgflineto{\pgfpoint{521.279968pt}{148.114273pt}}
\pgfusepath{stroke}
\pgfpathmoveto{\pgfpoint{79.347992pt}{198.411423pt}}
\pgflineto{\pgfpoint{74.880005pt}{198.411423pt}}
\pgfusepath{stroke}
\pgfpathmoveto{\pgfpoint{516.812012pt}{198.411423pt}}
\pgflineto{\pgfpoint{521.279968pt}{198.411423pt}}
\pgfusepath{stroke}
\pgfpathmoveto{\pgfpoint{79.347992pt}{248.708572pt}}
\pgflineto{\pgfpoint{74.880005pt}{248.708572pt}}
\pgfusepath{stroke}
\pgfpathmoveto{\pgfpoint{516.812012pt}{248.708572pt}}
\pgflineto{\pgfpoint{521.279968pt}{248.708572pt}}
\pgfusepath{stroke}
\pgfpathmoveto{\pgfpoint{79.347992pt}{299.005707pt}}
\pgflineto{\pgfpoint{74.880005pt}{299.005707pt}}
\pgfusepath{stroke}
\pgfpathmoveto{\pgfpoint{516.812012pt}{299.005707pt}}
\pgflineto{\pgfpoint{521.279968pt}{299.005707pt}}
\pgfusepath{stroke}
\pgfpathmoveto{\pgfpoint{79.347992pt}{349.302856pt}}
\pgflineto{\pgfpoint{74.880005pt}{349.302856pt}}
\pgfusepath{stroke}
\pgfpathmoveto{\pgfpoint{516.812012pt}{349.302856pt}}
\pgflineto{\pgfpoint{521.279968pt}{349.302856pt}}
\pgfusepath{stroke}
\pgfpathmoveto{\pgfpoint{79.347992pt}{399.600006pt}}
\pgflineto{\pgfpoint{74.880005pt}{399.600006pt}}
\pgfusepath{stroke}
\pgfpathmoveto{\pgfpoint{516.812012pt}{399.600006pt}}
\pgflineto{\pgfpoint{521.279968pt}{399.600006pt}}
\pgfusepath{stroke}
{
\pgftransformshift{\pgfpoint{69.875519pt}{47.519989pt}}
\pgfnode{rectangle}{east}{\fontsize{10}{0}\selectfont\textcolor[rgb]{0.15,0.15,0.15}{{-1}}}{}{\pgfusepath{discard}}}
{
\pgftransformshift{\pgfpoint{69.875519pt}{97.817131pt}}
\pgfnode{rectangle}{east}{\fontsize{10}{0}\selectfont\textcolor[rgb]{0.15,0.15,0.15}{{-0.5}}}{}{\pgfusepath{discard}}}
{
\pgftransformshift{\pgfpoint{69.875519pt}{148.114273pt}}
\pgfnode{rectangle}{east}{\fontsize{10}{0}\selectfont\textcolor[rgb]{0.15,0.15,0.15}{{0}}}{}{\pgfusepath{discard}}}
{
\pgftransformshift{\pgfpoint{69.875519pt}{198.411423pt}}
\pgfnode{rectangle}{east}{\fontsize{10}{0}\selectfont\textcolor[rgb]{0.15,0.15,0.15}{{0.5}}}{}{\pgfusepath{discard}}}
{
\pgftransformshift{\pgfpoint{69.875519pt}{248.708572pt}}
\pgfnode{rectangle}{east}{\fontsize{10}{0}\selectfont\textcolor[rgb]{0.15,0.15,0.15}{{1}}}{}{\pgfusepath{discard}}}
{
\pgftransformshift{\pgfpoint{69.875519pt}{299.005707pt}}
\pgfnode{rectangle}{east}{\fontsize{10}{0}\selectfont\textcolor[rgb]{0.15,0.15,0.15}{{1.5}}}{}{\pgfusepath{discard}}}
{
\pgftransformshift{\pgfpoint{69.875519pt}{349.302856pt}}
\pgfnode{rectangle}{east}{\fontsize{10}{0}\selectfont\textcolor[rgb]{0.15,0.15,0.15}{{2}}}{}{\pgfusepath{discard}}}
{
\pgftransformshift{\pgfpoint{69.875519pt}{399.599976pt}}
\pgfnode{rectangle}{east}{\fontsize{10}{0}\selectfont\textcolor[rgb]{0.15,0.15,0.15}{{2.5}}}{}{\pgfusepath{discard}}}
\pgfsetrectcap
\pgfsetdash{{16pt}{0pt}}{0pt}
\pgfpathmoveto{\pgfpoint{521.279968pt}{47.519974pt}}
\pgflineto{\pgfpoint{74.880005pt}{47.519974pt}}
\pgfusepath{stroke}
\pgfpathmoveto{\pgfpoint{521.279968pt}{399.600006pt}}
\pgflineto{\pgfpoint{74.880005pt}{399.600006pt}}
\pgfusepath{stroke}
\pgfpathmoveto{\pgfpoint{74.880005pt}{399.600006pt}}
\pgflineto{\pgfpoint{74.880005pt}{47.519974pt}}
\pgfusepath{stroke}
\pgfpathmoveto{\pgfpoint{521.279968pt}{399.600006pt}}
\pgflineto{\pgfpoint{521.279968pt}{47.519974pt}}
\pgfusepath{stroke}
\color[rgb]{0.000000,0.447000,0.741000}
\pgfsetroundcap
\pgfsetroundjoin
\pgfsetdash{}{0pt}
\pgfpathmoveto{\pgfpoint{523.707031pt}{292.199280pt}}
\pgflineto{\pgfpoint{524.280029pt}{293.962646pt}}
\pgfusepath{stroke}
\pgfpathmoveto{\pgfpoint{522.207092pt}{291.109497pt}}
\pgflineto{\pgfpoint{523.707031pt}{292.199280pt}}
\pgfusepath{stroke}
\pgfpathmoveto{\pgfpoint{520.352966pt}{291.109497pt}}
\pgflineto{\pgfpoint{522.207092pt}{291.109497pt}}
\pgfusepath{stroke}
\pgfpathmoveto{\pgfpoint{518.852966pt}{292.199280pt}}
\pgflineto{\pgfpoint{520.352966pt}{291.109497pt}}
\pgfusepath{stroke}
\pgfpathmoveto{\pgfpoint{518.280029pt}{293.962646pt}}
\pgflineto{\pgfpoint{518.852966pt}{292.199280pt}}
\pgfusepath{stroke}
\pgfpathmoveto{\pgfpoint{518.852966pt}{295.726013pt}}
\pgflineto{\pgfpoint{518.280029pt}{293.962646pt}}
\pgfusepath{stroke}
\pgfpathmoveto{\pgfpoint{520.352966pt}{296.815826pt}}
\pgflineto{\pgfpoint{518.852966pt}{295.726013pt}}
\pgfusepath{stroke}
\pgfpathmoveto{\pgfpoint{522.207092pt}{296.815826pt}}
\pgflineto{\pgfpoint{520.352966pt}{296.815826pt}}
\pgfusepath{stroke}
\pgfpathmoveto{\pgfpoint{523.707031pt}{295.726013pt}}
\pgflineto{\pgfpoint{522.207092pt}{296.815826pt}}
\pgfusepath{stroke}
\pgfpathmoveto{\pgfpoint{524.280029pt}{293.962646pt}}
\pgflineto{\pgfpoint{523.707031pt}{295.726013pt}}
\pgfusepath{stroke}
\color[rgb]{0.850000,0.325000,0.098000}
\pgfsetbuttcap
\pgfpathmoveto{\pgfpoint{79.389099pt}{158.729248pt}}
\pgflineto{\pgfpoint{74.880005pt}{148.114273pt}}
\pgfusepath{stroke}
\pgfpathmoveto{\pgfpoint{83.898178pt}{169.012939pt}}
\pgflineto{\pgfpoint{79.389099pt}{158.729248pt}}
\pgfusepath{stroke}
\pgfpathmoveto{\pgfpoint{88.407272pt}{178.912537pt}}
\pgflineto{\pgfpoint{83.898178pt}{169.012939pt}}
\pgfusepath{stroke}
\pgfpathmoveto{\pgfpoint{92.916351pt}{188.375214pt}}
\pgflineto{\pgfpoint{88.407272pt}{178.912537pt}}
\pgfusepath{stroke}
\pgfpathmoveto{\pgfpoint{97.425446pt}{197.348129pt}}
\pgflineto{\pgfpoint{92.916351pt}{188.375214pt}}
\pgfusepath{stroke}
\pgfpathmoveto{\pgfpoint{101.934555pt}{205.778458pt}}
\pgflineto{\pgfpoint{97.425446pt}{197.348129pt}}
\pgfusepath{stroke}
\pgfpathmoveto{\pgfpoint{106.443634pt}{213.613373pt}}
\pgflineto{\pgfpoint{101.934555pt}{205.778458pt}}
\pgfusepath{stroke}
\pgfpathmoveto{\pgfpoint{110.952728pt}{220.800034pt}}
\pgflineto{\pgfpoint{106.443634pt}{213.613373pt}}
\pgfusepath{stroke}
\pgfpathmoveto{\pgfpoint{115.461807pt}{227.285645pt}}
\pgflineto{\pgfpoint{110.952728pt}{220.800034pt}}
\pgfusepath{stroke}
\pgfpathmoveto{\pgfpoint{119.970917pt}{233.017334pt}}
\pgflineto{\pgfpoint{115.461807pt}{227.285645pt}}
\pgfusepath{stroke}
\pgfpathmoveto{\pgfpoint{124.480011pt}{237.942291pt}}
\pgflineto{\pgfpoint{119.970917pt}{233.017334pt}}
\pgfusepath{stroke}
\pgfpathmoveto{\pgfpoint{128.989090pt}{242.007690pt}}
\pgflineto{\pgfpoint{124.480011pt}{237.942291pt}}
\pgfusepath{stroke}
\pgfpathmoveto{\pgfpoint{133.498184pt}{245.160706pt}}
\pgflineto{\pgfpoint{128.989090pt}{242.007690pt}}
\pgfusepath{stroke}
\pgfpathmoveto{\pgfpoint{138.007263pt}{247.348495pt}}
\pgflineto{\pgfpoint{133.498184pt}{245.160706pt}}
\pgfusepath{stroke}
\pgfpathmoveto{\pgfpoint{142.516357pt}{248.518250pt}}
\pgflineto{\pgfpoint{138.007263pt}{247.348495pt}}
\pgfusepath{stroke}
\pgfpathmoveto{\pgfpoint{147.025452pt}{248.618576pt}}
\pgflineto{\pgfpoint{142.516357pt}{248.518250pt}}
\pgfusepath{stroke}
\pgfpathmoveto{\pgfpoint{151.534546pt}{247.641327pt}}
\pgflineto{\pgfpoint{147.025452pt}{248.618576pt}}
\pgfusepath{stroke}
\pgfpathmoveto{\pgfpoint{156.043640pt}{245.627655pt}}
\pgflineto{\pgfpoint{151.534546pt}{247.641327pt}}
\pgfusepath{stroke}
\pgfpathmoveto{\pgfpoint{160.552719pt}{242.621460pt}}
\pgflineto{\pgfpoint{156.043640pt}{245.627655pt}}
\pgfusepath{stroke}
\pgfpathmoveto{\pgfpoint{165.061813pt}{238.666595pt}}
\pgflineto{\pgfpoint{160.552719pt}{242.621460pt}}
\pgfusepath{stroke}
\pgfpathmoveto{\pgfpoint{169.570892pt}{233.806931pt}}
\pgflineto{\pgfpoint{165.061813pt}{238.666595pt}}
\pgfusepath{stroke}
\pgfpathmoveto{\pgfpoint{174.080002pt}{228.086365pt}}
\pgflineto{\pgfpoint{169.570892pt}{233.806931pt}}
\pgfusepath{stroke}
\pgfpathmoveto{\pgfpoint{178.589081pt}{221.548737pt}}
\pgflineto{\pgfpoint{174.080002pt}{228.086365pt}}
\pgfusepath{stroke}
\pgfpathmoveto{\pgfpoint{183.098175pt}{214.240692pt}}
\pgflineto{\pgfpoint{178.589081pt}{221.548737pt}}
\pgfusepath{stroke}
\pgfpathmoveto{\pgfpoint{187.607269pt}{206.239822pt}}
\pgflineto{\pgfpoint{183.098175pt}{214.240692pt}}
\pgfusepath{stroke}
\pgfpathmoveto{\pgfpoint{192.116364pt}{197.643463pt}}
\pgflineto{\pgfpoint{187.607269pt}{206.239822pt}}
\pgfusepath{stroke}
\pgfpathmoveto{\pgfpoint{196.625458pt}{188.549194pt}}
\pgflineto{\pgfpoint{192.116364pt}{197.643463pt}}
\pgfusepath{stroke}
\pgfpathmoveto{\pgfpoint{201.134552pt}{179.054611pt}}
\pgflineto{\pgfpoint{196.625458pt}{188.549194pt}}
\pgfusepath{stroke}
\pgfpathmoveto{\pgfpoint{205.643631pt}{169.257324pt}}
\pgflineto{\pgfpoint{201.134552pt}{179.054611pt}}
\pgfusepath{stroke}
\pgfpathmoveto{\pgfpoint{210.152725pt}{159.254944pt}}
\pgflineto{\pgfpoint{205.643631pt}{169.257324pt}}
\pgfusepath{stroke}
\pgfpathmoveto{\pgfpoint{214.661804pt}{149.145050pt}}
\pgflineto{\pgfpoint{210.152725pt}{159.254944pt}}
\pgfusepath{stroke}
\pgfpathmoveto{\pgfpoint{219.170914pt}{139.025269pt}}
\pgflineto{\pgfpoint{214.661804pt}{149.145050pt}}
\pgfusepath{stroke}
\pgfpathmoveto{\pgfpoint{223.679993pt}{128.993179pt}}
\pgflineto{\pgfpoint{219.170914pt}{139.025269pt}}
\pgfusepath{stroke}
\pgfpathmoveto{\pgfpoint{228.189087pt}{119.146370pt}}
\pgflineto{\pgfpoint{223.679993pt}{128.993179pt}}
\pgfusepath{stroke}
\pgfpathmoveto{\pgfpoint{232.698181pt}{109.582405pt}}
\pgflineto{\pgfpoint{228.189087pt}{119.146370pt}}
\pgfusepath{stroke}
\pgfpathmoveto{\pgfpoint{237.207275pt}{100.398895pt}}
\pgflineto{\pgfpoint{232.698181pt}{109.582405pt}}
\pgfusepath{stroke}
\pgfpathmoveto{\pgfpoint{241.716370pt}{91.693428pt}}
\pgflineto{\pgfpoint{237.207275pt}{100.398895pt}}
\pgfusepath{stroke}
\pgfpathmoveto{\pgfpoint{246.225449pt}{83.563568pt}}
\pgflineto{\pgfpoint{241.716370pt}{91.693428pt}}
\pgfusepath{stroke}
\pgfpathmoveto{\pgfpoint{250.734543pt}{76.106903pt}}
\pgflineto{\pgfpoint{246.225449pt}{83.563568pt}}
\pgfusepath{stroke}
\pgfpathmoveto{\pgfpoint{255.243622pt}{69.408401pt}}
\pgflineto{\pgfpoint{250.734543pt}{76.106903pt}}
\pgfusepath{stroke}
\pgfpathmoveto{\pgfpoint{259.752716pt}{63.517975pt}}
\pgflineto{\pgfpoint{255.243622pt}{69.408401pt}}
\pgfusepath{stroke}
\pgfpathmoveto{\pgfpoint{264.261810pt}{58.479507pt}}
\pgflineto{\pgfpoint{259.752716pt}{63.517975pt}}
\pgfusepath{stroke}
\pgfpathmoveto{\pgfpoint{268.770905pt}{54.336884pt}}
\pgflineto{\pgfpoint{264.261810pt}{58.479507pt}}
\pgfusepath{stroke}
\pgfpathmoveto{\pgfpoint{273.279999pt}{51.133957pt}}
\pgflineto{\pgfpoint{268.770905pt}{54.336884pt}}
\pgfusepath{stroke}
\pgfpathmoveto{\pgfpoint{277.789093pt}{48.914642pt}}
\pgflineto{\pgfpoint{273.279999pt}{51.133957pt}}
\pgfusepath{stroke}
\pgfpathmoveto{\pgfpoint{282.298187pt}{47.722809pt}}
\pgflineto{\pgfpoint{277.789093pt}{48.914642pt}}
\pgfusepath{stroke}
\pgfpathmoveto{\pgfpoint{286.807251pt}{47.601624pt}}
\pgflineto{\pgfpoint{282.298187pt}{47.722809pt}}
\pgfusepath{stroke}
\pgfpathmoveto{\pgfpoint{291.316345pt}{48.557632pt}}
\pgflineto{\pgfpoint{286.807251pt}{47.601624pt}}
\pgfusepath{stroke}
\pgfpathmoveto{\pgfpoint{295.825439pt}{50.542450pt}}
\pgflineto{\pgfpoint{291.316345pt}{48.557632pt}}
\pgfusepath{stroke}
\pgfpathmoveto{\pgfpoint{300.334564pt}{53.503235pt}}
\pgflineto{\pgfpoint{295.825439pt}{50.542450pt}}
\pgfusepath{stroke}
\pgfpathmoveto{\pgfpoint{304.843628pt}{57.387177pt}}
\pgflineto{\pgfpoint{300.334564pt}{53.503235pt}}
\pgfusepath{stroke}
\pgfpathmoveto{\pgfpoint{309.352722pt}{62.141434pt}}
\pgflineto{\pgfpoint{304.843628pt}{57.387177pt}}
\pgfusepath{stroke}
\pgfpathmoveto{\pgfpoint{313.861816pt}{67.713196pt}}
\pgflineto{\pgfpoint{309.352722pt}{62.141434pt}}
\pgfusepath{stroke}
\pgfpathmoveto{\pgfpoint{318.370911pt}{74.049606pt}}
\pgflineto{\pgfpoint{313.861816pt}{67.713196pt}}
\pgfusepath{stroke}
\pgfpathmoveto{\pgfpoint{322.880005pt}{81.097839pt}}
\pgflineto{\pgfpoint{318.370911pt}{74.049606pt}}
\pgfusepath{stroke}
\pgfpathmoveto{\pgfpoint{327.389099pt}{88.805099pt}}
\pgflineto{\pgfpoint{322.880005pt}{81.097839pt}}
\pgfusepath{stroke}
\pgfpathmoveto{\pgfpoint{331.898193pt}{97.118500pt}}
\pgflineto{\pgfpoint{327.389099pt}{88.805099pt}}
\pgfusepath{stroke}
\pgfpathmoveto{\pgfpoint{336.407288pt}{105.985237pt}}
\pgflineto{\pgfpoint{331.898193pt}{97.118500pt}}
\pgfusepath{stroke}
\pgfpathmoveto{\pgfpoint{340.916382pt}{115.352501pt}}
\pgflineto{\pgfpoint{336.407288pt}{105.985237pt}}
\pgfusepath{stroke}
\pgfpathmoveto{\pgfpoint{345.425446pt}{125.167427pt}}
\pgflineto{\pgfpoint{340.916382pt}{115.352501pt}}
\pgfusepath{stroke}
\pgfpathmoveto{\pgfpoint{349.934540pt}{135.377197pt}}
\pgflineto{\pgfpoint{345.425446pt}{125.167427pt}}
\pgfusepath{stroke}
\pgfpathmoveto{\pgfpoint{354.443634pt}{145.928986pt}}
\pgflineto{\pgfpoint{349.934540pt}{135.377197pt}}
\pgfusepath{stroke}
\pgfpathmoveto{\pgfpoint{358.952728pt}{156.769958pt}}
\pgflineto{\pgfpoint{354.443634pt}{145.928986pt}}
\pgfusepath{stroke}
\pgfpathmoveto{\pgfpoint{363.461823pt}{167.847290pt}}
\pgflineto{\pgfpoint{358.952728pt}{156.769958pt}}
\pgfusepath{stroke}
\pgfpathmoveto{\pgfpoint{367.970917pt}{179.108124pt}}
\pgflineto{\pgfpoint{363.461823pt}{167.847290pt}}
\pgfusepath{stroke}
\pgfpathmoveto{\pgfpoint{372.479980pt}{190.499664pt}}
\pgflineto{\pgfpoint{367.970917pt}{179.108124pt}}
\pgfusepath{stroke}
\pgfpathmoveto{\pgfpoint{376.989075pt}{201.969070pt}}
\pgflineto{\pgfpoint{372.479980pt}{190.499664pt}}
\pgfusepath{stroke}
\pgfpathmoveto{\pgfpoint{381.498169pt}{213.463516pt}}
\pgflineto{\pgfpoint{376.989075pt}{201.969070pt}}
\pgfusepath{stroke}
\pgfpathmoveto{\pgfpoint{386.007263pt}{224.930145pt}}
\pgflineto{\pgfpoint{381.498169pt}{213.463516pt}}
\pgfusepath{stroke}
\pgfpathmoveto{\pgfpoint{390.516357pt}{236.316147pt}}
\pgflineto{\pgfpoint{386.007263pt}{224.930145pt}}
\pgfusepath{stroke}
\pgfpathmoveto{\pgfpoint{395.025452pt}{247.568695pt}}
\pgflineto{\pgfpoint{390.516357pt}{236.316147pt}}
\pgfusepath{stroke}
\pgfpathmoveto{\pgfpoint{399.534546pt}{258.634949pt}}
\pgflineto{\pgfpoint{395.025452pt}{247.568695pt}}
\pgfusepath{stroke}
\pgfpathmoveto{\pgfpoint{404.043640pt}{269.462067pt}}
\pgflineto{\pgfpoint{399.534546pt}{258.634949pt}}
\pgfusepath{stroke}
\pgfpathmoveto{\pgfpoint{408.552734pt}{279.997253pt}}
\pgflineto{\pgfpoint{404.043640pt}{269.462067pt}}
\pgfusepath{stroke}
\pgfpathmoveto{\pgfpoint{413.061798pt}{290.187653pt}}
\pgflineto{\pgfpoint{408.552734pt}{279.997253pt}}
\pgfusepath{stroke}
\pgfpathmoveto{\pgfpoint{417.570892pt}{299.980438pt}}
\pgflineto{\pgfpoint{413.061798pt}{290.187653pt}}
\pgfusepath{stroke}
\pgfpathmoveto{\pgfpoint{422.080017pt}{309.322754pt}}
\pgflineto{\pgfpoint{417.570892pt}{299.980438pt}}
\pgfusepath{stroke}
\pgfpathmoveto{\pgfpoint{426.589111pt}{318.161835pt}}
\pgflineto{\pgfpoint{422.080017pt}{309.322754pt}}
\pgfusepath{stroke}
\pgfpathmoveto{\pgfpoint{431.098145pt}{326.444794pt}}
\pgflineto{\pgfpoint{426.589111pt}{318.161835pt}}
\pgfusepath{stroke}
\pgfpathmoveto{\pgfpoint{435.607239pt}{334.118805pt}}
\pgflineto{\pgfpoint{431.098145pt}{326.444794pt}}
\pgfusepath{stroke}
\pgfpathmoveto{\pgfpoint{440.116364pt}{341.131042pt}}
\pgflineto{\pgfpoint{435.607239pt}{334.118805pt}}
\pgfusepath{stroke}
\pgfpathmoveto{\pgfpoint{444.625458pt}{347.428711pt}}
\pgflineto{\pgfpoint{440.116364pt}{341.131042pt}}
\pgfusepath{stroke}
\pgfpathmoveto{\pgfpoint{449.134552pt}{352.958923pt}}
\pgflineto{\pgfpoint{444.625458pt}{347.428711pt}}
\pgfusepath{stroke}
\pgfpathmoveto{\pgfpoint{453.643616pt}{357.668915pt}}
\pgflineto{\pgfpoint{449.134552pt}{352.958923pt}}
\pgfusepath{stroke}
\pgfpathmoveto{\pgfpoint{458.152710pt}{361.505798pt}}
\pgflineto{\pgfpoint{453.643616pt}{357.668915pt}}
\pgfusepath{stroke}
\pgfpathmoveto{\pgfpoint{462.661804pt}{364.416748pt}}
\pgflineto{\pgfpoint{458.152710pt}{361.505798pt}}
\pgfusepath{stroke}
\pgfpathmoveto{\pgfpoint{467.170898pt}{366.348969pt}}
\pgflineto{\pgfpoint{462.661804pt}{364.416748pt}}
\pgfusepath{stroke}
\pgfpathmoveto{\pgfpoint{471.679993pt}{367.249603pt}}
\pgflineto{\pgfpoint{467.170898pt}{366.348969pt}}
\pgfusepath{stroke}
\pgfpathmoveto{\pgfpoint{476.189087pt}{367.065796pt}}
\pgflineto{\pgfpoint{471.679993pt}{367.249603pt}}
\pgfusepath{stroke}
\pgfpathmoveto{\pgfpoint{480.698181pt}{365.744751pt}}
\pgflineto{\pgfpoint{476.189087pt}{367.065796pt}}
\pgfusepath{stroke}
\pgfpathmoveto{\pgfpoint{485.207275pt}{363.233704pt}}
\pgflineto{\pgfpoint{480.698181pt}{365.744751pt}}
\pgfusepath{stroke}
\pgfpathmoveto{\pgfpoint{489.716370pt}{359.479706pt}}
\pgflineto{\pgfpoint{485.207275pt}{363.233704pt}}
\pgfusepath{stroke}
\pgfpathmoveto{\pgfpoint{494.225433pt}{354.429962pt}}
\pgflineto{\pgfpoint{489.716370pt}{359.479706pt}}
\pgfusepath{stroke}
\pgfpathmoveto{\pgfpoint{498.734528pt}{348.031677pt}}
\pgflineto{\pgfpoint{494.225433pt}{354.429962pt}}
\pgfusepath{stroke}
\pgfpathmoveto{\pgfpoint{503.243622pt}{340.231964pt}}
\pgflineto{\pgfpoint{498.734528pt}{348.031677pt}}
\pgfusepath{stroke}
\pgfpathmoveto{\pgfpoint{507.752716pt}{330.978058pt}}
\pgflineto{\pgfpoint{503.243622pt}{340.231964pt}}
\pgfusepath{stroke}
\pgfpathmoveto{\pgfpoint{512.261780pt}{320.217072pt}}
\pgflineto{\pgfpoint{507.752716pt}{330.978058pt}}
\pgfusepath{stroke}
\pgfpathmoveto{\pgfpoint{516.770874pt}{307.896240pt}}
\pgflineto{\pgfpoint{512.261780pt}{320.217072pt}}
\pgfusepath{stroke}
\pgfpathmoveto{\pgfpoint{521.279968pt}{293.962646pt}}
\pgflineto{\pgfpoint{516.770874pt}{307.896240pt}}
\pgfusepath{stroke}
\end{pgfscope}
\end{pgfpicture}
}
\begin{verbatim}
octave> interpolate (10, "linear")
ans =  3.3463
octave> interplot (10, C, "linear")
\end{verbatim}
\scalebox{0.39}{% Title: gl2ps_renderer figure
% Creator: GL2PS 1.4.0, (C) 1999-2017 C. Geuzaine
% For: Octave
% CreationDate: Fri Oct 25 16:23:47 2019
\begin{pgfpicture}
\color[rgb]{1.000000,1.000000,1.000000}
\pgfpathrectanglecorners{\pgfpoint{0pt}{0pt}}{\pgfpoint{576pt}{432pt}}
\pgfusepath{fill}
\begin{pgfscope}
\pgfpathrectangle{\pgfpoint{0pt}{0pt}}{\pgfpoint{576pt}{432pt}}
\pgfusepath{fill}
\pgfpathrectangle{\pgfpoint{0pt}{0pt}}{\pgfpoint{576pt}{432pt}}
\pgfusepath{clip}
\pgfpathmoveto{\pgfpoint{74.880005pt}{399.599976pt}}
\pgflineto{\pgfpoint{521.279968pt}{47.519974pt}}
\pgflineto{\pgfpoint{74.880005pt}{47.519974pt}}
\pgfpathclose
\pgfusepath{fill,stroke}
\pgfpathmoveto{\pgfpoint{74.880005pt}{399.599976pt}}
\pgflineto{\pgfpoint{521.279968pt}{399.599976pt}}
\pgflineto{\pgfpoint{521.279968pt}{47.519974pt}}
\pgfpathclose
\pgfusepath{fill,stroke}
\color[rgb]{0.150000,0.150000,0.150000}
\pgfsetlinewidth{0.500000pt}
\pgfpathmoveto{\pgfpoint{74.880005pt}{51.985001pt}}
\pgflineto{\pgfpoint{74.880005pt}{47.519974pt}}
\pgfusepath{stroke}
\pgfpathmoveto{\pgfpoint{74.880005pt}{395.134979pt}}
\pgflineto{\pgfpoint{74.880005pt}{399.599976pt}}
\pgfusepath{stroke}
\pgfpathmoveto{\pgfpoint{164.160004pt}{51.985001pt}}
\pgflineto{\pgfpoint{164.160004pt}{47.519974pt}}
\pgfusepath{stroke}
\pgfpathmoveto{\pgfpoint{164.160004pt}{395.134979pt}}
\pgflineto{\pgfpoint{164.160004pt}{399.599976pt}}
\pgfusepath{stroke}
\pgfpathmoveto{\pgfpoint{253.440002pt}{51.985001pt}}
\pgflineto{\pgfpoint{253.440002pt}{47.519974pt}}
\pgfusepath{stroke}
\pgfpathmoveto{\pgfpoint{253.440002pt}{395.134979pt}}
\pgflineto{\pgfpoint{253.440002pt}{399.599976pt}}
\pgfusepath{stroke}
\pgfpathmoveto{\pgfpoint{342.720001pt}{51.985001pt}}
\pgflineto{\pgfpoint{342.720001pt}{47.519974pt}}
\pgfusepath{stroke}
\pgfpathmoveto{\pgfpoint{342.720001pt}{395.134979pt}}
\pgflineto{\pgfpoint{342.720001pt}{399.599976pt}}
\pgfusepath{stroke}
\pgfpathmoveto{\pgfpoint{432.000000pt}{51.985001pt}}
\pgflineto{\pgfpoint{432.000000pt}{47.519974pt}}
\pgfusepath{stroke}
\pgfpathmoveto{\pgfpoint{432.000000pt}{395.134979pt}}
\pgflineto{\pgfpoint{432.000000pt}{399.599976pt}}
\pgfusepath{stroke}
\pgfpathmoveto{\pgfpoint{521.279968pt}{51.985001pt}}
\pgflineto{\pgfpoint{521.279968pt}{47.519974pt}}
\pgfusepath{stroke}
\pgfpathmoveto{\pgfpoint{521.279968pt}{395.134979pt}}
\pgflineto{\pgfpoint{521.279968pt}{399.599976pt}}
\pgfusepath{stroke}
{
\pgftransformshift{\pgfpoint{74.880005pt}{40.018265pt}}
\pgfnode{rectangle}{north}{\fontsize{10}{0}\selectfont\textcolor[rgb]{0.15,0.15,0.15}{{0}}}{}{\pgfusepath{discard}}}
{
\pgftransformshift{\pgfpoint{164.159988pt}{40.018265pt}}
\pgfnode{rectangle}{north}{\fontsize{10}{0}\selectfont\textcolor[rgb]{0.15,0.15,0.15}{{2}}}{}{\pgfusepath{discard}}}
{
\pgftransformshift{\pgfpoint{253.440002pt}{40.018265pt}}
\pgfnode{rectangle}{north}{\fontsize{10}{0}\selectfont\textcolor[rgb]{0.15,0.15,0.15}{{4}}}{}{\pgfusepath{discard}}}
{
\pgftransformshift{\pgfpoint{342.720001pt}{40.018265pt}}
\pgfnode{rectangle}{north}{\fontsize{10}{0}\selectfont\textcolor[rgb]{0.15,0.15,0.15}{{6}}}{}{\pgfusepath{discard}}}
{
\pgftransformshift{\pgfpoint{432.000000pt}{40.018265pt}}
\pgfnode{rectangle}{north}{\fontsize{10}{0}\selectfont\textcolor[rgb]{0.15,0.15,0.15}{{8}}}{}{\pgfusepath{discard}}}
{
\pgftransformshift{\pgfpoint{521.279968pt}{40.018265pt}}
\pgfnode{rectangle}{north}{\fontsize{10}{0}\selectfont\textcolor[rgb]{0.15,0.15,0.15}{{10}}}{}{\pgfusepath{discard}}}
\pgfpathmoveto{\pgfpoint{79.347992pt}{47.519974pt}}
\pgflineto{\pgfpoint{74.880005pt}{47.519974pt}}
\pgfusepath{stroke}
\pgfpathmoveto{\pgfpoint{516.812012pt}{47.519974pt}}
\pgflineto{\pgfpoint{521.279968pt}{47.519974pt}}
\pgfusepath{stroke}
\pgfpathmoveto{\pgfpoint{79.347992pt}{117.935989pt}}
\pgflineto{\pgfpoint{74.880005pt}{117.935989pt}}
\pgfusepath{stroke}
\pgfpathmoveto{\pgfpoint{516.812012pt}{117.935989pt}}
\pgflineto{\pgfpoint{521.279968pt}{117.935989pt}}
\pgfusepath{stroke}
\pgfpathmoveto{\pgfpoint{79.347992pt}{188.351990pt}}
\pgflineto{\pgfpoint{74.880005pt}{188.351990pt}}
\pgfusepath{stroke}
\pgfpathmoveto{\pgfpoint{516.812012pt}{188.351990pt}}
\pgflineto{\pgfpoint{521.279968pt}{188.351990pt}}
\pgfusepath{stroke}
\pgfpathmoveto{\pgfpoint{79.347992pt}{258.768005pt}}
\pgflineto{\pgfpoint{74.880005pt}{258.768005pt}}
\pgfusepath{stroke}
\pgfpathmoveto{\pgfpoint{516.812012pt}{258.768005pt}}
\pgflineto{\pgfpoint{521.279968pt}{258.768005pt}}
\pgfusepath{stroke}
\pgfpathmoveto{\pgfpoint{79.347992pt}{329.183990pt}}
\pgflineto{\pgfpoint{74.880005pt}{329.183990pt}}
\pgfusepath{stroke}
\pgfpathmoveto{\pgfpoint{516.812012pt}{329.183990pt}}
\pgflineto{\pgfpoint{521.279968pt}{329.183990pt}}
\pgfusepath{stroke}
\pgfpathmoveto{\pgfpoint{79.347992pt}{399.599976pt}}
\pgflineto{\pgfpoint{74.880005pt}{399.599976pt}}
\pgfusepath{stroke}
\pgfpathmoveto{\pgfpoint{516.812012pt}{399.599976pt}}
\pgflineto{\pgfpoint{521.279968pt}{399.599976pt}}
\pgfusepath{stroke}
{
\pgftransformshift{\pgfpoint{69.875519pt}{47.519989pt}}
\pgfnode{rectangle}{east}{\fontsize{10}{0}\selectfont\textcolor[rgb]{0.15,0.15,0.15}{{-1}}}{}{\pgfusepath{discard}}}
{
\pgftransformshift{\pgfpoint{69.875519pt}{117.935989pt}}
\pgfnode{rectangle}{east}{\fontsize{10}{0}\selectfont\textcolor[rgb]{0.15,0.15,0.15}{{0}}}{}{\pgfusepath{discard}}}
{
\pgftransformshift{\pgfpoint{69.875519pt}{188.351990pt}}
\pgfnode{rectangle}{east}{\fontsize{10}{0}\selectfont\textcolor[rgb]{0.15,0.15,0.15}{{1}}}{}{\pgfusepath{discard}}}
{
\pgftransformshift{\pgfpoint{69.875519pt}{258.768005pt}}
\pgfnode{rectangle}{east}{\fontsize{10}{0}\selectfont\textcolor[rgb]{0.15,0.15,0.15}{{2}}}{}{\pgfusepath{discard}}}
{
\pgftransformshift{\pgfpoint{69.875519pt}{329.183990pt}}
\pgfnode{rectangle}{east}{\fontsize{10}{0}\selectfont\textcolor[rgb]{0.15,0.15,0.15}{{3}}}{}{\pgfusepath{discard}}}
{
\pgftransformshift{\pgfpoint{69.875519pt}{399.599976pt}}
\pgfnode{rectangle}{east}{\fontsize{10}{0}\selectfont\textcolor[rgb]{0.15,0.15,0.15}{{4}}}{}{\pgfusepath{discard}}}
\pgfsetrectcap
\pgfsetdash{{16pt}{0pt}}{0pt}
\pgfpathmoveto{\pgfpoint{521.279968pt}{47.519974pt}}
\pgflineto{\pgfpoint{74.880005pt}{47.519974pt}}
\pgfusepath{stroke}
\pgfpathmoveto{\pgfpoint{521.279968pt}{399.599976pt}}
\pgflineto{\pgfpoint{74.880005pt}{399.599976pt}}
\pgfusepath{stroke}
\pgfpathmoveto{\pgfpoint{74.880005pt}{399.599976pt}}
\pgflineto{\pgfpoint{74.880005pt}{47.519974pt}}
\pgfusepath{stroke}
\pgfpathmoveto{\pgfpoint{521.279968pt}{399.599976pt}}
\pgflineto{\pgfpoint{521.279968pt}{47.519974pt}}
\pgfusepath{stroke}
\color[rgb]{0.000000,0.447000,0.741000}
\pgfsetroundcap
\pgfsetroundjoin
\pgfsetdash{}{0pt}
\pgfpathmoveto{\pgfpoint{523.707031pt}{351.806549pt}}
\pgflineto{\pgfpoint{524.280029pt}{353.569916pt}}
\pgfusepath{stroke}
\pgfpathmoveto{\pgfpoint{522.207092pt}{350.716736pt}}
\pgflineto{\pgfpoint{523.707031pt}{351.806549pt}}
\pgfusepath{stroke}
\pgfpathmoveto{\pgfpoint{520.352966pt}{350.716736pt}}
\pgflineto{\pgfpoint{522.207092pt}{350.716736pt}}
\pgfusepath{stroke}
\pgfpathmoveto{\pgfpoint{518.852966pt}{351.806549pt}}
\pgflineto{\pgfpoint{520.352966pt}{350.716736pt}}
\pgfusepath{stroke}
\pgfpathmoveto{\pgfpoint{518.280029pt}{353.569916pt}}
\pgflineto{\pgfpoint{518.852966pt}{351.806549pt}}
\pgfusepath{stroke}
\pgfpathmoveto{\pgfpoint{518.852966pt}{355.333252pt}}
\pgflineto{\pgfpoint{518.280029pt}{353.569916pt}}
\pgfusepath{stroke}
\pgfpathmoveto{\pgfpoint{520.352966pt}{356.423096pt}}
\pgflineto{\pgfpoint{518.852966pt}{355.333252pt}}
\pgfusepath{stroke}
\pgfpathmoveto{\pgfpoint{522.207092pt}{356.423096pt}}
\pgflineto{\pgfpoint{520.352966pt}{356.423096pt}}
\pgfusepath{stroke}
\pgfpathmoveto{\pgfpoint{523.707031pt}{355.333252pt}}
\pgflineto{\pgfpoint{522.207092pt}{356.423096pt}}
\pgfusepath{stroke}
\pgfpathmoveto{\pgfpoint{524.280029pt}{353.569916pt}}
\pgflineto{\pgfpoint{523.707031pt}{355.333252pt}}
\pgfusepath{stroke}
\color[rgb]{0.850000,0.325000,0.098000}
\pgfsetbuttcap
\pgfpathmoveto{\pgfpoint{79.389099pt}{124.339706pt}}
\pgflineto{\pgfpoint{74.880005pt}{117.935989pt}}
\pgfusepath{stroke}
\pgfpathmoveto{\pgfpoint{83.898178pt}{130.743423pt}}
\pgflineto{\pgfpoint{79.389099pt}{124.339706pt}}
\pgfusepath{stroke}
\pgfpathmoveto{\pgfpoint{88.407272pt}{137.147141pt}}
\pgflineto{\pgfpoint{83.898178pt}{130.743423pt}}
\pgfusepath{stroke}
\pgfpathmoveto{\pgfpoint{92.916351pt}{143.550858pt}}
\pgflineto{\pgfpoint{88.407272pt}{137.147141pt}}
\pgfusepath{stroke}
\pgfpathmoveto{\pgfpoint{97.425446pt}{149.954590pt}}
\pgflineto{\pgfpoint{92.916351pt}{143.550858pt}}
\pgfusepath{stroke}
\pgfpathmoveto{\pgfpoint{101.934555pt}{156.358307pt}}
\pgflineto{\pgfpoint{97.425446pt}{149.954590pt}}
\pgfusepath{stroke}
\pgfpathmoveto{\pgfpoint{106.443634pt}{162.762024pt}}
\pgflineto{\pgfpoint{101.934555pt}{156.358307pt}}
\pgfusepath{stroke}
\pgfpathmoveto{\pgfpoint{110.952728pt}{168.323425pt}}
\pgflineto{\pgfpoint{106.443634pt}{162.762024pt}}
\pgfusepath{stroke}
\pgfpathmoveto{\pgfpoint{115.461807pt}{170.975891pt}}
\pgflineto{\pgfpoint{110.952728pt}{168.323425pt}}
\pgfusepath{stroke}
\pgfpathmoveto{\pgfpoint{119.970917pt}{173.628357pt}}
\pgflineto{\pgfpoint{115.461807pt}{170.975891pt}}
\pgfusepath{stroke}
\pgfpathmoveto{\pgfpoint{124.480011pt}{176.280823pt}}
\pgflineto{\pgfpoint{119.970917pt}{173.628357pt}}
\pgfusepath{stroke}
\pgfpathmoveto{\pgfpoint{128.989090pt}{178.933289pt}}
\pgflineto{\pgfpoint{124.480011pt}{176.280823pt}}
\pgfusepath{stroke}
\pgfpathmoveto{\pgfpoint{133.498184pt}{181.585754pt}}
\pgflineto{\pgfpoint{128.989090pt}{178.933289pt}}
\pgfusepath{stroke}
\pgfpathmoveto{\pgfpoint{138.007263pt}{184.238220pt}}
\pgflineto{\pgfpoint{133.498184pt}{181.585754pt}}
\pgfusepath{stroke}
\pgfpathmoveto{\pgfpoint{142.516357pt}{186.890701pt}}
\pgflineto{\pgfpoint{138.007263pt}{184.238220pt}}
\pgfusepath{stroke}
\pgfpathmoveto{\pgfpoint{147.025452pt}{187.160812pt}}
\pgflineto{\pgfpoint{142.516357pt}{186.890701pt}}
\pgfusepath{stroke}
\pgfpathmoveto{\pgfpoint{151.534546pt}{184.508362pt}}
\pgflineto{\pgfpoint{147.025452pt}{187.160812pt}}
\pgfusepath{stroke}
\pgfpathmoveto{\pgfpoint{156.043640pt}{181.855896pt}}
\pgflineto{\pgfpoint{151.534546pt}{184.508362pt}}
\pgfusepath{stroke}
\pgfpathmoveto{\pgfpoint{160.552719pt}{179.203415pt}}
\pgflineto{\pgfpoint{156.043640pt}{181.855896pt}}
\pgfusepath{stroke}
\pgfpathmoveto{\pgfpoint{165.061813pt}{176.550964pt}}
\pgflineto{\pgfpoint{160.552719pt}{179.203415pt}}
\pgfusepath{stroke}
\pgfpathmoveto{\pgfpoint{169.570892pt}{173.898499pt}}
\pgflineto{\pgfpoint{165.061813pt}{176.550964pt}}
\pgfusepath{stroke}
\pgfpathmoveto{\pgfpoint{174.080002pt}{171.246017pt}}
\pgflineto{\pgfpoint{169.570892pt}{173.898499pt}}
\pgfusepath{stroke}
\pgfpathmoveto{\pgfpoint{178.589081pt}{168.593567pt}}
\pgflineto{\pgfpoint{174.080002pt}{171.246017pt}}
\pgfusepath{stroke}
\pgfpathmoveto{\pgfpoint{183.098175pt}{163.414124pt}}
\pgflineto{\pgfpoint{178.589081pt}{168.593567pt}}
\pgfusepath{stroke}
\pgfpathmoveto{\pgfpoint{187.607269pt}{157.010315pt}}
\pgflineto{\pgfpoint{183.098175pt}{163.414124pt}}
\pgfusepath{stroke}
\pgfpathmoveto{\pgfpoint{192.116364pt}{150.606506pt}}
\pgflineto{\pgfpoint{187.607269pt}{157.010315pt}}
\pgfusepath{stroke}
\pgfpathmoveto{\pgfpoint{196.625458pt}{144.202713pt}}
\pgflineto{\pgfpoint{192.116364pt}{150.606506pt}}
\pgfusepath{stroke}
\pgfpathmoveto{\pgfpoint{201.134552pt}{137.798920pt}}
\pgflineto{\pgfpoint{196.625458pt}{144.202713pt}}
\pgfusepath{stroke}
\pgfpathmoveto{\pgfpoint{205.643631pt}{131.395111pt}}
\pgflineto{\pgfpoint{201.134552pt}{137.798920pt}}
\pgfusepath{stroke}
\pgfpathmoveto{\pgfpoint{210.152725pt}{124.991318pt}}
\pgflineto{\pgfpoint{205.643631pt}{131.395111pt}}
\pgfusepath{stroke}
\pgfpathmoveto{\pgfpoint{214.661804pt}{118.587517pt}}
\pgflineto{\pgfpoint{210.152725pt}{124.991318pt}}
\pgfusepath{stroke}
\pgfpathmoveto{\pgfpoint{219.170914pt}{112.183792pt}}
\pgflineto{\pgfpoint{214.661804pt}{118.587517pt}}
\pgfusepath{stroke}
\pgfpathmoveto{\pgfpoint{223.679993pt}{105.780067pt}}
\pgflineto{\pgfpoint{219.170914pt}{112.183792pt}}
\pgfusepath{stroke}
\pgfpathmoveto{\pgfpoint{228.189087pt}{99.376358pt}}
\pgflineto{\pgfpoint{223.679993pt}{105.780067pt}}
\pgfusepath{stroke}
\pgfpathmoveto{\pgfpoint{232.698181pt}{92.972633pt}}
\pgflineto{\pgfpoint{228.189087pt}{99.376358pt}}
\pgfusepath{stroke}
\pgfpathmoveto{\pgfpoint{237.207275pt}{86.568909pt}}
\pgflineto{\pgfpoint{232.698181pt}{92.972633pt}}
\pgfusepath{stroke}
\pgfpathmoveto{\pgfpoint{241.716370pt}{80.165192pt}}
\pgflineto{\pgfpoint{237.207275pt}{86.568909pt}}
\pgfusepath{stroke}
\pgfpathmoveto{\pgfpoint{246.225449pt}{73.761475pt}}
\pgflineto{\pgfpoint{241.716370pt}{80.165192pt}}
\pgfusepath{stroke}
\pgfpathmoveto{\pgfpoint{250.734543pt}{67.818420pt}}
\pgflineto{\pgfpoint{246.225449pt}{73.761475pt}}
\pgfusepath{stroke}
\pgfpathmoveto{\pgfpoint{255.243622pt}{65.165939pt}}
\pgflineto{\pgfpoint{250.734543pt}{67.818420pt}}
\pgfusepath{stroke}
\pgfpathmoveto{\pgfpoint{259.752716pt}{62.513489pt}}
\pgflineto{\pgfpoint{255.243622pt}{65.165939pt}}
\pgfusepath{stroke}
\pgfpathmoveto{\pgfpoint{264.261810pt}{59.861008pt}}
\pgflineto{\pgfpoint{259.752716pt}{62.513489pt}}
\pgfusepath{stroke}
\pgfpathmoveto{\pgfpoint{268.770905pt}{57.208542pt}}
\pgflineto{\pgfpoint{264.261810pt}{59.861008pt}}
\pgfusepath{stroke}
\pgfpathmoveto{\pgfpoint{273.279999pt}{54.556076pt}}
\pgflineto{\pgfpoint{268.770905pt}{57.208542pt}}
\pgfusepath{stroke}
\pgfpathmoveto{\pgfpoint{277.789093pt}{51.903610pt}}
\pgflineto{\pgfpoint{273.279999pt}{54.556076pt}}
\pgfusepath{stroke}
\pgfpathmoveto{\pgfpoint{282.298187pt}{49.251144pt}}
\pgflineto{\pgfpoint{277.789093pt}{51.903610pt}}
\pgfusepath{stroke}
\pgfpathmoveto{\pgfpoint{286.807251pt}{48.441284pt}}
\pgflineto{\pgfpoint{282.298187pt}{49.251144pt}}
\pgfusepath{stroke}
\pgfpathmoveto{\pgfpoint{291.316345pt}{51.093765pt}}
\pgflineto{\pgfpoint{286.807251pt}{48.441284pt}}
\pgfusepath{stroke}
\pgfpathmoveto{\pgfpoint{295.825439pt}{53.746216pt}}
\pgflineto{\pgfpoint{291.316345pt}{51.093765pt}}
\pgfusepath{stroke}
\pgfpathmoveto{\pgfpoint{300.334564pt}{56.398697pt}}
\pgflineto{\pgfpoint{295.825439pt}{53.746216pt}}
\pgfusepath{stroke}
\pgfpathmoveto{\pgfpoint{304.843628pt}{59.051163pt}}
\pgflineto{\pgfpoint{300.334564pt}{56.398697pt}}
\pgfusepath{stroke}
\pgfpathmoveto{\pgfpoint{309.352722pt}{61.703629pt}}
\pgflineto{\pgfpoint{304.843628pt}{59.051163pt}}
\pgfusepath{stroke}
\pgfpathmoveto{\pgfpoint{313.861816pt}{64.356079pt}}
\pgflineto{\pgfpoint{309.352722pt}{61.703629pt}}
\pgfusepath{stroke}
\pgfpathmoveto{\pgfpoint{318.370911pt}{67.008545pt}}
\pgflineto{\pgfpoint{313.861816pt}{64.356079pt}}
\pgfusepath{stroke}
\pgfpathmoveto{\pgfpoint{322.880005pt}{71.806290pt}}
\pgflineto{\pgfpoint{318.370911pt}{67.008545pt}}
\pgfusepath{stroke}
\pgfpathmoveto{\pgfpoint{327.389099pt}{78.209991pt}}
\pgflineto{\pgfpoint{322.880005pt}{71.806290pt}}
\pgfusepath{stroke}
\pgfpathmoveto{\pgfpoint{331.898193pt}{84.613724pt}}
\pgflineto{\pgfpoint{327.389099pt}{78.209991pt}}
\pgfusepath{stroke}
\pgfpathmoveto{\pgfpoint{336.407288pt}{91.017433pt}}
\pgflineto{\pgfpoint{331.898193pt}{84.613724pt}}
\pgfusepath{stroke}
\pgfpathmoveto{\pgfpoint{340.916382pt}{97.421158pt}}
\pgflineto{\pgfpoint{336.407288pt}{91.017433pt}}
\pgfusepath{stroke}
\pgfpathmoveto{\pgfpoint{345.425446pt}{103.824867pt}}
\pgflineto{\pgfpoint{340.916382pt}{97.421158pt}}
\pgfusepath{stroke}
\pgfpathmoveto{\pgfpoint{349.934540pt}{110.228592pt}}
\pgflineto{\pgfpoint{345.425446pt}{103.824867pt}}
\pgfusepath{stroke}
\pgfpathmoveto{\pgfpoint{354.443634pt}{116.632309pt}}
\pgflineto{\pgfpoint{349.934540pt}{110.228592pt}}
\pgfusepath{stroke}
\pgfpathmoveto{\pgfpoint{358.952728pt}{123.036034pt}}
\pgflineto{\pgfpoint{354.443634pt}{116.632309pt}}
\pgfusepath{stroke}
\pgfpathmoveto{\pgfpoint{363.461823pt}{129.439758pt}}
\pgflineto{\pgfpoint{358.952728pt}{123.036034pt}}
\pgfusepath{stroke}
\pgfpathmoveto{\pgfpoint{367.970917pt}{135.843475pt}}
\pgflineto{\pgfpoint{363.461823pt}{129.439758pt}}
\pgfusepath{stroke}
\pgfpathmoveto{\pgfpoint{372.479980pt}{142.247192pt}}
\pgflineto{\pgfpoint{367.970917pt}{135.843475pt}}
\pgfusepath{stroke}
\pgfpathmoveto{\pgfpoint{376.989075pt}{148.650909pt}}
\pgflineto{\pgfpoint{372.479980pt}{142.247192pt}}
\pgfusepath{stroke}
\pgfpathmoveto{\pgfpoint{381.498169pt}{155.054626pt}}
\pgflineto{\pgfpoint{376.989075pt}{148.650909pt}}
\pgfusepath{stroke}
\pgfpathmoveto{\pgfpoint{386.007263pt}{161.458344pt}}
\pgflineto{\pgfpoint{381.498169pt}{155.054626pt}}
\pgfusepath{stroke}
\pgfpathmoveto{\pgfpoint{390.516357pt}{167.862061pt}}
\pgflineto{\pgfpoint{386.007263pt}{161.458344pt}}
\pgfusepath{stroke}
\pgfpathmoveto{\pgfpoint{395.025452pt}{174.265778pt}}
\pgflineto{\pgfpoint{390.516357pt}{167.862061pt}}
\pgfusepath{stroke}
\pgfpathmoveto{\pgfpoint{399.534546pt}{180.669495pt}}
\pgflineto{\pgfpoint{395.025452pt}{174.265778pt}}
\pgfusepath{stroke}
\pgfpathmoveto{\pgfpoint{404.043640pt}{187.073212pt}}
\pgflineto{\pgfpoint{399.534546pt}{180.669495pt}}
\pgfusepath{stroke}
\pgfpathmoveto{\pgfpoint{408.552734pt}{193.476929pt}}
\pgflineto{\pgfpoint{404.043640pt}{187.073212pt}}
\pgfusepath{stroke}
\pgfpathmoveto{\pgfpoint{413.061798pt}{199.880661pt}}
\pgflineto{\pgfpoint{408.552734pt}{193.476929pt}}
\pgfusepath{stroke}
\pgfpathmoveto{\pgfpoint{417.570892pt}{206.284378pt}}
\pgflineto{\pgfpoint{413.061798pt}{199.880661pt}}
\pgfusepath{stroke}
\pgfpathmoveto{\pgfpoint{422.080017pt}{212.688095pt}}
\pgflineto{\pgfpoint{417.570892pt}{206.284378pt}}
\pgfusepath{stroke}
\pgfpathmoveto{\pgfpoint{426.589111pt}{219.091812pt}}
\pgflineto{\pgfpoint{422.080017pt}{212.688095pt}}
\pgfusepath{stroke}
\pgfpathmoveto{\pgfpoint{431.098145pt}{225.495529pt}}
\pgflineto{\pgfpoint{426.589111pt}{219.091812pt}}
\pgfusepath{stroke}
\pgfpathmoveto{\pgfpoint{435.607239pt}{231.899246pt}}
\pgflineto{\pgfpoint{431.098145pt}{225.495529pt}}
\pgfusepath{stroke}
\pgfpathmoveto{\pgfpoint{440.116364pt}{238.302963pt}}
\pgflineto{\pgfpoint{435.607239pt}{231.899246pt}}
\pgfusepath{stroke}
\pgfpathmoveto{\pgfpoint{444.625458pt}{244.706680pt}}
\pgflineto{\pgfpoint{440.116364pt}{238.302963pt}}
\pgfusepath{stroke}
\pgfpathmoveto{\pgfpoint{449.134552pt}{251.110413pt}}
\pgflineto{\pgfpoint{444.625458pt}{244.706680pt}}
\pgfusepath{stroke}
\pgfpathmoveto{\pgfpoint{453.643616pt}{257.514130pt}}
\pgflineto{\pgfpoint{449.134552pt}{251.110413pt}}
\pgfusepath{stroke}
\pgfpathmoveto{\pgfpoint{458.152710pt}{263.917847pt}}
\pgflineto{\pgfpoint{453.643616pt}{257.514130pt}}
\pgfusepath{stroke}
\pgfpathmoveto{\pgfpoint{462.661804pt}{270.321564pt}}
\pgflineto{\pgfpoint{458.152710pt}{263.917847pt}}
\pgfusepath{stroke}
\pgfpathmoveto{\pgfpoint{467.170898pt}{276.725281pt}}
\pgflineto{\pgfpoint{462.661804pt}{270.321564pt}}
\pgfusepath{stroke}
\pgfpathmoveto{\pgfpoint{471.679993pt}{283.128998pt}}
\pgflineto{\pgfpoint{467.170898pt}{276.725281pt}}
\pgfusepath{stroke}
\pgfpathmoveto{\pgfpoint{476.189087pt}{289.532715pt}}
\pgflineto{\pgfpoint{471.679993pt}{283.128998pt}}
\pgfusepath{stroke}
\pgfpathmoveto{\pgfpoint{480.698181pt}{295.936432pt}}
\pgflineto{\pgfpoint{476.189087pt}{289.532715pt}}
\pgfusepath{stroke}
\pgfpathmoveto{\pgfpoint{485.207275pt}{302.340149pt}}
\pgflineto{\pgfpoint{480.698181pt}{295.936432pt}}
\pgfusepath{stroke}
\pgfpathmoveto{\pgfpoint{489.716370pt}{308.743896pt}}
\pgflineto{\pgfpoint{485.207275pt}{302.340149pt}}
\pgfusepath{stroke}
\pgfpathmoveto{\pgfpoint{494.225433pt}{315.147583pt}}
\pgflineto{\pgfpoint{489.716370pt}{308.743896pt}}
\pgfusepath{stroke}
\pgfpathmoveto{\pgfpoint{498.734528pt}{321.551331pt}}
\pgflineto{\pgfpoint{494.225433pt}{315.147583pt}}
\pgfusepath{stroke}
\pgfpathmoveto{\pgfpoint{503.243622pt}{327.955017pt}}
\pgflineto{\pgfpoint{498.734528pt}{321.551331pt}}
\pgfusepath{stroke}
\pgfpathmoveto{\pgfpoint{507.752716pt}{334.358765pt}}
\pgflineto{\pgfpoint{503.243622pt}{327.955017pt}}
\pgfusepath{stroke}
\pgfpathmoveto{\pgfpoint{512.261780pt}{340.762451pt}}
\pgflineto{\pgfpoint{507.752716pt}{334.358765pt}}
\pgfusepath{stroke}
\pgfpathmoveto{\pgfpoint{516.770874pt}{347.166199pt}}
\pgflineto{\pgfpoint{512.261780pt}{340.762451pt}}
\pgfusepath{stroke}
\pgfpathmoveto{\pgfpoint{521.279968pt}{353.569916pt}}
\pgflineto{\pgfpoint{516.770874pt}{347.166199pt}}
\pgfusepath{stroke}
\end{pgfscope}
\end{pgfpicture}
}

    From the existing data, we can make a guess that \verb|f|
    is a cubic function and regression fits quite well:
\begin{verbatim}
octave> p = polyfit (x, fx, 3)
p =
   0.084488  -0.796282   1.681694  -0.043870
octave> polyval (p, 10)
ans =  21.633
octave> plot (x, fx, "o", C, polyval (p, C))
\end{verbatim}
\scalebox{0.62}{% Title: gl2ps_renderer figure
% Creator: GL2PS 1.4.0, (C) 1999-2017 C. Geuzaine
% For: Octave
% CreationDate: Fri Oct 25 15:43:26 2019
\begin{pgfpicture}
\color[rgb]{1.000000,1.000000,1.000000}
\pgfpathrectanglecorners{\pgfpoint{0pt}{0pt}}{\pgfpoint{576pt}{432pt}}
\pgfusepath{fill}
\begin{pgfscope}
\pgfpathrectangle{\pgfpoint{0pt}{0pt}}{\pgfpoint{576pt}{432pt}}
\pgfusepath{fill}
\pgfpathrectangle{\pgfpoint{0pt}{0pt}}{\pgfpoint{576pt}{432pt}}
\pgfusepath{clip}
\pgfpathmoveto{\pgfpoint{74.880005pt}{399.600037pt}}
\pgflineto{\pgfpoint{521.279968pt}{47.519974pt}}
\pgflineto{\pgfpoint{74.880005pt}{47.519974pt}}
\pgfpathclose
\pgfusepath{fill,stroke}
\pgfpathmoveto{\pgfpoint{74.880005pt}{399.600037pt}}
\pgflineto{\pgfpoint{521.279968pt}{399.600037pt}}
\pgflineto{\pgfpoint{521.279968pt}{47.519974pt}}
\pgfpathclose
\pgfusepath{fill,stroke}
\color[rgb]{0.150000,0.150000,0.150000}
\pgfsetlinewidth{0.500000pt}
\pgfpathmoveto{\pgfpoint{74.880005pt}{51.985001pt}}
\pgflineto{\pgfpoint{74.880005pt}{47.519974pt}}
\pgfusepath{stroke}
\pgfpathmoveto{\pgfpoint{74.880005pt}{395.134979pt}}
\pgflineto{\pgfpoint{74.880005pt}{399.600037pt}}
\pgfusepath{stroke}
\pgfpathmoveto{\pgfpoint{164.160004pt}{51.985001pt}}
\pgflineto{\pgfpoint{164.160004pt}{47.519974pt}}
\pgfusepath{stroke}
\pgfpathmoveto{\pgfpoint{164.160004pt}{395.134979pt}}
\pgflineto{\pgfpoint{164.160004pt}{399.600037pt}}
\pgfusepath{stroke}
\pgfpathmoveto{\pgfpoint{253.440002pt}{51.985001pt}}
\pgflineto{\pgfpoint{253.440002pt}{47.519974pt}}
\pgfusepath{stroke}
\pgfpathmoveto{\pgfpoint{253.440002pt}{395.134979pt}}
\pgflineto{\pgfpoint{253.440002pt}{399.600037pt}}
\pgfusepath{stroke}
\pgfpathmoveto{\pgfpoint{342.720001pt}{51.985001pt}}
\pgflineto{\pgfpoint{342.720001pt}{47.519974pt}}
\pgfusepath{stroke}
\pgfpathmoveto{\pgfpoint{342.720001pt}{395.134979pt}}
\pgflineto{\pgfpoint{342.720001pt}{399.600037pt}}
\pgfusepath{stroke}
\pgfpathmoveto{\pgfpoint{432.000000pt}{51.985001pt}}
\pgflineto{\pgfpoint{432.000000pt}{47.519974pt}}
\pgfusepath{stroke}
\pgfpathmoveto{\pgfpoint{432.000000pt}{395.134979pt}}
\pgflineto{\pgfpoint{432.000000pt}{399.600037pt}}
\pgfusepath{stroke}
\pgfpathmoveto{\pgfpoint{521.279968pt}{51.985001pt}}
\pgflineto{\pgfpoint{521.279968pt}{47.519974pt}}
\pgfusepath{stroke}
\pgfpathmoveto{\pgfpoint{521.279968pt}{395.134979pt}}
\pgflineto{\pgfpoint{521.279968pt}{399.600037pt}}
\pgfusepath{stroke}
{
\pgftransformshift{\pgfpoint{74.880005pt}{40.018265pt}}
\pgfnode{rectangle}{north}{\fontsize{10}{0}\selectfont\textcolor[rgb]{0.15,0.15,0.15}{{0}}}{}{\pgfusepath{discard}}}
{
\pgftransformshift{\pgfpoint{164.159988pt}{40.018265pt}}
\pgfnode{rectangle}{north}{\fontsize{10}{0}\selectfont\textcolor[rgb]{0.15,0.15,0.15}{{2}}}{}{\pgfusepath{discard}}}
{
\pgftransformshift{\pgfpoint{253.440002pt}{40.018265pt}}
\pgfnode{rectangle}{north}{\fontsize{10}{0}\selectfont\textcolor[rgb]{0.15,0.15,0.15}{{4}}}{}{\pgfusepath{discard}}}
{
\pgftransformshift{\pgfpoint{342.720001pt}{40.018265pt}}
\pgfnode{rectangle}{north}{\fontsize{10}{0}\selectfont\textcolor[rgb]{0.15,0.15,0.15}{{6}}}{}{\pgfusepath{discard}}}
{
\pgftransformshift{\pgfpoint{432.000000pt}{40.018265pt}}
\pgfnode{rectangle}{north}{\fontsize{10}{0}\selectfont\textcolor[rgb]{0.15,0.15,0.15}{{8}}}{}{\pgfusepath{discard}}}
{
\pgftransformshift{\pgfpoint{521.279968pt}{40.018265pt}}
\pgfnode{rectangle}{north}{\fontsize{10}{0}\selectfont\textcolor[rgb]{0.15,0.15,0.15}{{10}}}{}{\pgfusepath{discard}}}
\pgfpathmoveto{\pgfpoint{79.347992pt}{47.519974pt}}
\pgflineto{\pgfpoint{74.880005pt}{47.519974pt}}
\pgfusepath{stroke}
\pgfpathmoveto{\pgfpoint{516.812012pt}{47.519974pt}}
\pgflineto{\pgfpoint{521.279968pt}{47.519974pt}}
\pgfusepath{stroke}
\pgfpathmoveto{\pgfpoint{79.347992pt}{106.199989pt}}
\pgflineto{\pgfpoint{74.880005pt}{106.199989pt}}
\pgfusepath{stroke}
\pgfpathmoveto{\pgfpoint{516.812012pt}{106.199989pt}}
\pgflineto{\pgfpoint{521.279968pt}{106.199989pt}}
\pgfusepath{stroke}
\pgfpathmoveto{\pgfpoint{79.347992pt}{164.879990pt}}
\pgflineto{\pgfpoint{74.880005pt}{164.879990pt}}
\pgfusepath{stroke}
\pgfpathmoveto{\pgfpoint{516.812012pt}{164.879990pt}}
\pgflineto{\pgfpoint{521.279968pt}{164.879990pt}}
\pgfusepath{stroke}
\pgfpathmoveto{\pgfpoint{79.347992pt}{223.559998pt}}
\pgflineto{\pgfpoint{74.880005pt}{223.559998pt}}
\pgfusepath{stroke}
\pgfpathmoveto{\pgfpoint{516.812012pt}{223.559998pt}}
\pgflineto{\pgfpoint{521.279968pt}{223.559998pt}}
\pgfusepath{stroke}
\pgfpathmoveto{\pgfpoint{79.347992pt}{282.239990pt}}
\pgflineto{\pgfpoint{74.880005pt}{282.239990pt}}
\pgfusepath{stroke}
\pgfpathmoveto{\pgfpoint{516.812012pt}{282.239990pt}}
\pgflineto{\pgfpoint{521.279968pt}{282.239990pt}}
\pgfusepath{stroke}
\pgfpathmoveto{\pgfpoint{79.347992pt}{340.919983pt}}
\pgflineto{\pgfpoint{74.880005pt}{340.919983pt}}
\pgfusepath{stroke}
\pgfpathmoveto{\pgfpoint{516.812012pt}{340.919983pt}}
\pgflineto{\pgfpoint{521.279968pt}{340.919983pt}}
\pgfusepath{stroke}
\pgfpathmoveto{\pgfpoint{79.347992pt}{399.600037pt}}
\pgflineto{\pgfpoint{74.880005pt}{399.600037pt}}
\pgfusepath{stroke}
\pgfpathmoveto{\pgfpoint{516.812012pt}{399.600037pt}}
\pgflineto{\pgfpoint{521.279968pt}{399.600037pt}}
\pgfusepath{stroke}
{
\pgftransformshift{\pgfpoint{69.875519pt}{47.519989pt}}
\pgfnode{rectangle}{east}{\fontsize{10}{0}\selectfont\textcolor[rgb]{0.15,0.15,0.15}{{-5}}}{}{\pgfusepath{discard}}}
{
\pgftransformshift{\pgfpoint{69.875519pt}{106.199989pt}}
\pgfnode{rectangle}{east}{\fontsize{10}{0}\selectfont\textcolor[rgb]{0.15,0.15,0.15}{{0}}}{}{\pgfusepath{discard}}}
{
\pgftransformshift{\pgfpoint{69.875519pt}{164.879990pt}}
\pgfnode{rectangle}{east}{\fontsize{10}{0}\selectfont\textcolor[rgb]{0.15,0.15,0.15}{{5}}}{}{\pgfusepath{discard}}}
{
\pgftransformshift{\pgfpoint{69.875519pt}{223.559998pt}}
\pgfnode{rectangle}{east}{\fontsize{10}{0}\selectfont\textcolor[rgb]{0.15,0.15,0.15}{{10}}}{}{\pgfusepath{discard}}}
{
\pgftransformshift{\pgfpoint{69.875519pt}{282.239990pt}}
\pgfnode{rectangle}{east}{\fontsize{10}{0}\selectfont\textcolor[rgb]{0.15,0.15,0.15}{{15}}}{}{\pgfusepath{discard}}}
{
\pgftransformshift{\pgfpoint{69.875519pt}{340.919983pt}}
\pgfnode{rectangle}{east}{\fontsize{10}{0}\selectfont\textcolor[rgb]{0.15,0.15,0.15}{{20}}}{}{\pgfusepath{discard}}}
{
\pgftransformshift{\pgfpoint{69.875519pt}{399.599976pt}}
\pgfnode{rectangle}{east}{\fontsize{10}{0}\selectfont\textcolor[rgb]{0.15,0.15,0.15}{{25}}}{}{\pgfusepath{discard}}}
\pgfsetrectcap
\pgfsetdash{{16pt}{0pt}}{0pt}
\pgfpathmoveto{\pgfpoint{521.279968pt}{47.519974pt}}
\pgflineto{\pgfpoint{74.880005pt}{47.519974pt}}
\pgfusepath{stroke}
\pgfpathmoveto{\pgfpoint{521.279968pt}{399.600037pt}}
\pgflineto{\pgfpoint{74.880005pt}{399.600037pt}}
\pgfusepath{stroke}
\pgfpathmoveto{\pgfpoint{74.880005pt}{399.600037pt}}
\pgflineto{\pgfpoint{74.880005pt}{47.519974pt}}
\pgfusepath{stroke}
\pgfpathmoveto{\pgfpoint{521.279968pt}{399.600037pt}}
\pgflineto{\pgfpoint{521.279968pt}{47.519974pt}}
\pgfusepath{stroke}
\color[rgb]{0.000000,0.447000,0.741000}
\pgfsetroundcap
\pgfsetroundjoin
\pgfsetdash{}{0pt}
\pgfpathmoveto{\pgfpoint{77.307053pt}{104.436646pt}}
\pgflineto{\pgfpoint{77.880005pt}{106.200005pt}}
\pgfusepath{stroke}
\pgfpathmoveto{\pgfpoint{75.807053pt}{103.346832pt}}
\pgflineto{\pgfpoint{77.307053pt}{104.436646pt}}
\pgfusepath{stroke}
\pgfpathmoveto{\pgfpoint{73.952942pt}{103.346832pt}}
\pgflineto{\pgfpoint{75.807053pt}{103.346832pt}}
\pgfusepath{stroke}
\pgfpathmoveto{\pgfpoint{72.452942pt}{104.436646pt}}
\pgflineto{\pgfpoint{73.952942pt}{103.346832pt}}
\pgfusepath{stroke}
\pgfpathmoveto{\pgfpoint{71.879990pt}{106.200005pt}}
\pgflineto{\pgfpoint{72.452942pt}{104.436646pt}}
\pgfusepath{stroke}
\pgfpathmoveto{\pgfpoint{72.452942pt}{107.963356pt}}
\pgflineto{\pgfpoint{71.879990pt}{106.200005pt}}
\pgfusepath{stroke}
\pgfpathmoveto{\pgfpoint{73.952942pt}{109.053169pt}}
\pgflineto{\pgfpoint{72.452942pt}{107.963356pt}}
\pgfusepath{stroke}
\pgfpathmoveto{\pgfpoint{75.807053pt}{109.053169pt}}
\pgflineto{\pgfpoint{73.952942pt}{109.053169pt}}
\pgfusepath{stroke}
\pgfpathmoveto{\pgfpoint{77.307053pt}{107.963356pt}}
\pgflineto{\pgfpoint{75.807053pt}{109.053169pt}}
\pgfusepath{stroke}
\pgfpathmoveto{\pgfpoint{77.880005pt}{106.200005pt}}
\pgflineto{\pgfpoint{77.307053pt}{107.963356pt}}
\pgfusepath{stroke}
\pgfpathmoveto{\pgfpoint{112.367310pt}{112.735306pt}}
\pgflineto{\pgfpoint{112.940262pt}{114.498665pt}}
\pgfusepath{stroke}
\pgfpathmoveto{\pgfpoint{110.867310pt}{111.645493pt}}
\pgflineto{\pgfpoint{112.367310pt}{112.735306pt}}
\pgfusepath{stroke}
\pgfpathmoveto{\pgfpoint{109.013199pt}{111.645493pt}}
\pgflineto{\pgfpoint{110.867310pt}{111.645493pt}}
\pgfusepath{stroke}
\pgfpathmoveto{\pgfpoint{107.513199pt}{112.735306pt}}
\pgflineto{\pgfpoint{109.013199pt}{111.645493pt}}
\pgfusepath{stroke}
\pgfpathmoveto{\pgfpoint{106.940247pt}{114.498665pt}}
\pgflineto{\pgfpoint{107.513199pt}{112.735306pt}}
\pgfusepath{stroke}
\pgfpathmoveto{\pgfpoint{107.513199pt}{116.262016pt}}
\pgflineto{\pgfpoint{106.940247pt}{114.498665pt}}
\pgfusepath{stroke}
\pgfpathmoveto{\pgfpoint{109.013199pt}{117.351830pt}}
\pgflineto{\pgfpoint{107.513199pt}{116.262016pt}}
\pgfusepath{stroke}
\pgfpathmoveto{\pgfpoint{110.867310pt}{117.351830pt}}
\pgflineto{\pgfpoint{109.013199pt}{117.351830pt}}
\pgfusepath{stroke}
\pgfpathmoveto{\pgfpoint{112.367310pt}{116.262016pt}}
\pgflineto{\pgfpoint{110.867310pt}{117.351830pt}}
\pgfusepath{stroke}
\pgfpathmoveto{\pgfpoint{112.940262pt}{114.498665pt}}
\pgflineto{\pgfpoint{112.367310pt}{116.262016pt}}
\pgfusepath{stroke}
\pgfpathmoveto{\pgfpoint{147.427582pt}{116.172646pt}}
\pgflineto{\pgfpoint{148.000519pt}{117.936005pt}}
\pgfusepath{stroke}
\pgfpathmoveto{\pgfpoint{145.927582pt}{115.082832pt}}
\pgflineto{\pgfpoint{147.427582pt}{116.172646pt}}
\pgfusepath{stroke}
\pgfpathmoveto{\pgfpoint{144.073486pt}{115.082832pt}}
\pgflineto{\pgfpoint{145.927582pt}{115.082832pt}}
\pgfusepath{stroke}
\pgfpathmoveto{\pgfpoint{142.573471pt}{116.172646pt}}
\pgflineto{\pgfpoint{144.073486pt}{115.082832pt}}
\pgfusepath{stroke}
\pgfpathmoveto{\pgfpoint{142.000519pt}{117.936005pt}}
\pgflineto{\pgfpoint{142.573471pt}{116.172646pt}}
\pgfusepath{stroke}
\pgfpathmoveto{\pgfpoint{142.573471pt}{119.699356pt}}
\pgflineto{\pgfpoint{142.000519pt}{117.936005pt}}
\pgfusepath{stroke}
\pgfpathmoveto{\pgfpoint{144.073486pt}{120.789169pt}}
\pgflineto{\pgfpoint{142.573471pt}{119.699356pt}}
\pgfusepath{stroke}
\pgfpathmoveto{\pgfpoint{145.927582pt}{120.789169pt}}
\pgflineto{\pgfpoint{144.073486pt}{120.789169pt}}
\pgfusepath{stroke}
\pgfpathmoveto{\pgfpoint{147.427582pt}{119.699356pt}}
\pgflineto{\pgfpoint{145.927582pt}{120.789169pt}}
\pgfusepath{stroke}
\pgfpathmoveto{\pgfpoint{148.000519pt}{117.936005pt}}
\pgflineto{\pgfpoint{147.427582pt}{119.699356pt}}
\pgfusepath{stroke}
\pgfpathmoveto{\pgfpoint{182.487823pt}{112.735306pt}}
\pgflineto{\pgfpoint{183.060760pt}{114.498665pt}}
\pgfusepath{stroke}
\pgfpathmoveto{\pgfpoint{180.987823pt}{111.645493pt}}
\pgflineto{\pgfpoint{182.487823pt}{112.735306pt}}
\pgfusepath{stroke}
\pgfpathmoveto{\pgfpoint{179.133728pt}{111.645493pt}}
\pgflineto{\pgfpoint{180.987823pt}{111.645493pt}}
\pgfusepath{stroke}
\pgfpathmoveto{\pgfpoint{177.633713pt}{112.735306pt}}
\pgflineto{\pgfpoint{179.133728pt}{111.645493pt}}
\pgfusepath{stroke}
\pgfpathmoveto{\pgfpoint{177.060760pt}{114.498665pt}}
\pgflineto{\pgfpoint{177.633713pt}{112.735306pt}}
\pgfusepath{stroke}
\pgfpathmoveto{\pgfpoint{177.633713pt}{116.262016pt}}
\pgflineto{\pgfpoint{177.060760pt}{114.498665pt}}
\pgfusepath{stroke}
\pgfpathmoveto{\pgfpoint{179.133728pt}{117.351830pt}}
\pgflineto{\pgfpoint{177.633713pt}{116.262016pt}}
\pgfusepath{stroke}
\pgfpathmoveto{\pgfpoint{180.987823pt}{117.351830pt}}
\pgflineto{\pgfpoint{179.133728pt}{117.351830pt}}
\pgfusepath{stroke}
\pgfpathmoveto{\pgfpoint{182.487823pt}{116.262016pt}}
\pgflineto{\pgfpoint{180.987823pt}{117.351830pt}}
\pgfusepath{stroke}
\pgfpathmoveto{\pgfpoint{183.060760pt}{114.498665pt}}
\pgflineto{\pgfpoint{182.487823pt}{116.262016pt}}
\pgfusepath{stroke}
\pgfpathmoveto{\pgfpoint{217.547638pt}{104.436646pt}}
\pgflineto{\pgfpoint{218.120575pt}{106.200005pt}}
\pgfusepath{stroke}
\pgfpathmoveto{\pgfpoint{216.047638pt}{103.346832pt}}
\pgflineto{\pgfpoint{217.547638pt}{104.436646pt}}
\pgfusepath{stroke}
\pgfpathmoveto{\pgfpoint{214.193527pt}{103.346832pt}}
\pgflineto{\pgfpoint{216.047638pt}{103.346832pt}}
\pgfusepath{stroke}
\pgfpathmoveto{\pgfpoint{212.693527pt}{104.436646pt}}
\pgflineto{\pgfpoint{214.193527pt}{103.346832pt}}
\pgfusepath{stroke}
\pgfpathmoveto{\pgfpoint{212.120575pt}{106.200005pt}}
\pgflineto{\pgfpoint{212.693527pt}{104.436646pt}}
\pgfusepath{stroke}
\pgfpathmoveto{\pgfpoint{212.693527pt}{107.963356pt}}
\pgflineto{\pgfpoint{212.120575pt}{106.200005pt}}
\pgfusepath{stroke}
\pgfpathmoveto{\pgfpoint{214.193527pt}{109.053169pt}}
\pgflineto{\pgfpoint{212.693527pt}{107.963356pt}}
\pgfusepath{stroke}
\pgfpathmoveto{\pgfpoint{216.047638pt}{109.053169pt}}
\pgflineto{\pgfpoint{214.193527pt}{109.053169pt}}
\pgfusepath{stroke}
\pgfpathmoveto{\pgfpoint{217.547638pt}{107.963356pt}}
\pgflineto{\pgfpoint{216.047638pt}{109.053169pt}}
\pgfusepath{stroke}
\pgfpathmoveto{\pgfpoint{218.120575pt}{106.200005pt}}
\pgflineto{\pgfpoint{217.547638pt}{107.963356pt}}
\pgfusepath{stroke}
\pgfpathmoveto{\pgfpoint{252.607880pt}{96.138008pt}}
\pgflineto{\pgfpoint{253.180832pt}{97.901367pt}}
\pgfusepath{stroke}
\pgfpathmoveto{\pgfpoint{251.107880pt}{95.048203pt}}
\pgflineto{\pgfpoint{252.607880pt}{96.138008pt}}
\pgfusepath{stroke}
\pgfpathmoveto{\pgfpoint{249.253784pt}{95.048203pt}}
\pgflineto{\pgfpoint{251.107880pt}{95.048203pt}}
\pgfusepath{stroke}
\pgfpathmoveto{\pgfpoint{247.753784pt}{96.138008pt}}
\pgflineto{\pgfpoint{249.253784pt}{95.048203pt}}
\pgfusepath{stroke}
\pgfpathmoveto{\pgfpoint{247.180832pt}{97.901367pt}}
\pgflineto{\pgfpoint{247.753784pt}{96.138008pt}}
\pgfusepath{stroke}
\pgfpathmoveto{\pgfpoint{247.753784pt}{99.664726pt}}
\pgflineto{\pgfpoint{247.180832pt}{97.901367pt}}
\pgfusepath{stroke}
\pgfpathmoveto{\pgfpoint{249.253784pt}{100.754532pt}}
\pgflineto{\pgfpoint{247.753784pt}{99.664726pt}}
\pgfusepath{stroke}
\pgfpathmoveto{\pgfpoint{251.107880pt}{100.754532pt}}
\pgflineto{\pgfpoint{249.253784pt}{100.754532pt}}
\pgfusepath{stroke}
\pgfpathmoveto{\pgfpoint{252.607880pt}{99.664726pt}}
\pgflineto{\pgfpoint{251.107880pt}{100.754532pt}}
\pgfusepath{stroke}
\pgfpathmoveto{\pgfpoint{253.180832pt}{97.901367pt}}
\pgflineto{\pgfpoint{252.607880pt}{99.664726pt}}
\pgfusepath{stroke}
\pgfpathmoveto{\pgfpoint{287.668152pt}{92.700623pt}}
\pgflineto{\pgfpoint{288.241089pt}{94.463974pt}}
\pgfusepath{stroke}
\pgfpathmoveto{\pgfpoint{286.168152pt}{91.610809pt}}
\pgflineto{\pgfpoint{287.668152pt}{92.700623pt}}
\pgfusepath{stroke}
\pgfpathmoveto{\pgfpoint{284.314056pt}{91.610809pt}}
\pgflineto{\pgfpoint{286.168152pt}{91.610809pt}}
\pgfusepath{stroke}
\pgfpathmoveto{\pgfpoint{282.814056pt}{92.700623pt}}
\pgflineto{\pgfpoint{284.314056pt}{91.610809pt}}
\pgfusepath{stroke}
\pgfpathmoveto{\pgfpoint{282.241089pt}{94.463974pt}}
\pgflineto{\pgfpoint{282.814056pt}{92.700623pt}}
\pgfusepath{stroke}
\pgfpathmoveto{\pgfpoint{282.814056pt}{96.227333pt}}
\pgflineto{\pgfpoint{282.241089pt}{94.463974pt}}
\pgfusepath{stroke}
\pgfpathmoveto{\pgfpoint{284.314056pt}{97.317146pt}}
\pgflineto{\pgfpoint{282.814056pt}{96.227333pt}}
\pgfusepath{stroke}
\pgfpathmoveto{\pgfpoint{286.168152pt}{97.317146pt}}
\pgflineto{\pgfpoint{284.314056pt}{97.317146pt}}
\pgfusepath{stroke}
\pgfpathmoveto{\pgfpoint{287.668152pt}{96.227333pt}}
\pgflineto{\pgfpoint{286.168152pt}{97.317146pt}}
\pgfusepath{stroke}
\pgfpathmoveto{\pgfpoint{288.241089pt}{94.463974pt}}
\pgflineto{\pgfpoint{287.668152pt}{96.227333pt}}
\pgfusepath{stroke}
\pgfpathmoveto{\pgfpoint{322.728394pt}{96.138008pt}}
\pgflineto{\pgfpoint{323.301361pt}{97.901367pt}}
\pgfusepath{stroke}
\pgfpathmoveto{\pgfpoint{321.228394pt}{95.048203pt}}
\pgflineto{\pgfpoint{322.728394pt}{96.138008pt}}
\pgfusepath{stroke}
\pgfpathmoveto{\pgfpoint{319.374298pt}{95.048203pt}}
\pgflineto{\pgfpoint{321.228394pt}{95.048203pt}}
\pgfusepath{stroke}
\pgfpathmoveto{\pgfpoint{317.874298pt}{96.138008pt}}
\pgflineto{\pgfpoint{319.374298pt}{95.048203pt}}
\pgfusepath{stroke}
\pgfpathmoveto{\pgfpoint{317.301361pt}{97.901367pt}}
\pgflineto{\pgfpoint{317.874298pt}{96.138008pt}}
\pgfusepath{stroke}
\pgfpathmoveto{\pgfpoint{317.874298pt}{99.664726pt}}
\pgflineto{\pgfpoint{317.301361pt}{97.901367pt}}
\pgfusepath{stroke}
\pgfpathmoveto{\pgfpoint{319.374298pt}{100.754532pt}}
\pgflineto{\pgfpoint{317.874298pt}{99.664726pt}}
\pgfusepath{stroke}
\pgfpathmoveto{\pgfpoint{321.228394pt}{100.754532pt}}
\pgflineto{\pgfpoint{319.374298pt}{100.754532pt}}
\pgfusepath{stroke}
\pgfpathmoveto{\pgfpoint{322.728394pt}{99.664726pt}}
\pgflineto{\pgfpoint{321.228394pt}{100.754532pt}}
\pgfusepath{stroke}
\pgfpathmoveto{\pgfpoint{323.301361pt}{97.901367pt}}
\pgflineto{\pgfpoint{322.728394pt}{99.664726pt}}
\pgfusepath{stroke}
\pgfpathmoveto{\pgfpoint{357.788666pt}{104.436646pt}}
\pgflineto{\pgfpoint{358.361603pt}{106.200005pt}}
\pgfusepath{stroke}
\pgfpathmoveto{\pgfpoint{356.288666pt}{103.346832pt}}
\pgflineto{\pgfpoint{357.788666pt}{104.436646pt}}
\pgfusepath{stroke}
\pgfpathmoveto{\pgfpoint{354.434570pt}{103.346832pt}}
\pgflineto{\pgfpoint{356.288666pt}{103.346832pt}}
\pgfusepath{stroke}
\pgfpathmoveto{\pgfpoint{352.934570pt}{104.436646pt}}
\pgflineto{\pgfpoint{354.434570pt}{103.346832pt}}
\pgfusepath{stroke}
\pgfpathmoveto{\pgfpoint{352.361603pt}{106.200005pt}}
\pgflineto{\pgfpoint{352.934570pt}{104.436646pt}}
\pgfusepath{stroke}
\pgfpathmoveto{\pgfpoint{352.934570pt}{107.963356pt}}
\pgflineto{\pgfpoint{352.361603pt}{106.200005pt}}
\pgfusepath{stroke}
\pgfpathmoveto{\pgfpoint{354.434570pt}{109.053169pt}}
\pgflineto{\pgfpoint{352.934570pt}{107.963356pt}}
\pgfusepath{stroke}
\pgfpathmoveto{\pgfpoint{356.288666pt}{109.053169pt}}
\pgflineto{\pgfpoint{354.434570pt}{109.053169pt}}
\pgfusepath{stroke}
\pgfpathmoveto{\pgfpoint{357.788666pt}{107.963356pt}}
\pgflineto{\pgfpoint{356.288666pt}{109.053169pt}}
\pgfusepath{stroke}
\pgfpathmoveto{\pgfpoint{358.361603pt}{106.200005pt}}
\pgflineto{\pgfpoint{357.788666pt}{107.963356pt}}
\pgfusepath{stroke}
\color[rgb]{0.850000,0.325000,0.098000}
\pgfsetbuttcap
\pgfpathmoveto{\pgfpoint{79.389099pt}{107.584381pt}}
\pgflineto{\pgfpoint{74.880005pt}{105.685135pt}}
\pgfusepath{stroke}
\pgfpathmoveto{\pgfpoint{83.898178pt}{109.299057pt}}
\pgflineto{\pgfpoint{79.389099pt}{107.584381pt}}
\pgfusepath{stroke}
\pgfpathmoveto{\pgfpoint{88.407272pt}{110.835297pt}}
\pgflineto{\pgfpoint{83.898178pt}{109.299057pt}}
\pgfusepath{stroke}
\pgfpathmoveto{\pgfpoint{92.916351pt}{112.199234pt}}
\pgflineto{\pgfpoint{88.407272pt}{110.835297pt}}
\pgfusepath{stroke}
\pgfpathmoveto{\pgfpoint{97.425446pt}{113.397003pt}}
\pgflineto{\pgfpoint{92.916351pt}{112.199234pt}}
\pgfusepath{stroke}
\pgfpathmoveto{\pgfpoint{101.934555pt}{114.434723pt}}
\pgflineto{\pgfpoint{97.425446pt}{113.397003pt}}
\pgfusepath{stroke}
\pgfpathmoveto{\pgfpoint{106.443634pt}{115.318542pt}}
\pgflineto{\pgfpoint{101.934555pt}{114.434723pt}}
\pgfusepath{stroke}
\pgfpathmoveto{\pgfpoint{110.952728pt}{116.054581pt}}
\pgflineto{\pgfpoint{106.443634pt}{115.318542pt}}
\pgfusepath{stroke}
\pgfpathmoveto{\pgfpoint{115.461807pt}{116.648964pt}}
\pgflineto{\pgfpoint{110.952728pt}{116.054581pt}}
\pgfusepath{stroke}
\pgfpathmoveto{\pgfpoint{119.970917pt}{117.107841pt}}
\pgflineto{\pgfpoint{115.461807pt}{116.648964pt}}
\pgfusepath{stroke}
\pgfpathmoveto{\pgfpoint{124.480011pt}{117.437332pt}}
\pgflineto{\pgfpoint{119.970917pt}{117.107841pt}}
\pgfusepath{stroke}
\pgfpathmoveto{\pgfpoint{128.989090pt}{117.643570pt}}
\pgflineto{\pgfpoint{124.480011pt}{117.437332pt}}
\pgfusepath{stroke}
\pgfpathmoveto{\pgfpoint{133.498184pt}{117.732681pt}}
\pgflineto{\pgfpoint{128.989090pt}{117.643570pt}}
\pgfusepath{stroke}
\pgfpathmoveto{\pgfpoint{138.007263pt}{117.710815pt}}
\pgflineto{\pgfpoint{133.498184pt}{117.732681pt}}
\pgfusepath{stroke}
\pgfpathmoveto{\pgfpoint{142.516357pt}{117.584084pt}}
\pgflineto{\pgfpoint{138.007263pt}{117.710815pt}}
\pgfusepath{stroke}
\pgfpathmoveto{\pgfpoint{147.025452pt}{117.358627pt}}
\pgflineto{\pgfpoint{142.516357pt}{117.584084pt}}
\pgfusepath{stroke}
\pgfpathmoveto{\pgfpoint{151.534546pt}{117.040573pt}}
\pgflineto{\pgfpoint{147.025452pt}{117.358627pt}}
\pgfusepath{stroke}
\pgfpathmoveto{\pgfpoint{156.043640pt}{116.636055pt}}
\pgflineto{\pgfpoint{151.534546pt}{117.040573pt}}
\pgfusepath{stroke}
\pgfpathmoveto{\pgfpoint{160.552719pt}{116.151207pt}}
\pgflineto{\pgfpoint{156.043640pt}{116.636055pt}}
\pgfusepath{stroke}
\pgfpathmoveto{\pgfpoint{165.061813pt}{115.592155pt}}
\pgflineto{\pgfpoint{160.552719pt}{116.151207pt}}
\pgfusepath{stroke}
\pgfpathmoveto{\pgfpoint{169.570892pt}{114.965034pt}}
\pgflineto{\pgfpoint{165.061813pt}{115.592155pt}}
\pgfusepath{stroke}
\pgfpathmoveto{\pgfpoint{174.080002pt}{114.275978pt}}
\pgflineto{\pgfpoint{169.570892pt}{114.965034pt}}
\pgfusepath{stroke}
\pgfpathmoveto{\pgfpoint{178.589081pt}{113.531113pt}}
\pgflineto{\pgfpoint{174.080002pt}{114.275978pt}}
\pgfusepath{stroke}
\pgfpathmoveto{\pgfpoint{183.098175pt}{112.736572pt}}
\pgflineto{\pgfpoint{178.589081pt}{113.531113pt}}
\pgfusepath{stroke}
\pgfpathmoveto{\pgfpoint{187.607269pt}{111.898491pt}}
\pgflineto{\pgfpoint{183.098175pt}{112.736572pt}}
\pgfusepath{stroke}
\pgfpathmoveto{\pgfpoint{192.116364pt}{111.022987pt}}
\pgflineto{\pgfpoint{187.607269pt}{111.898491pt}}
\pgfusepath{stroke}
\pgfpathmoveto{\pgfpoint{196.625458pt}{110.116211pt}}
\pgflineto{\pgfpoint{192.116364pt}{111.022987pt}}
\pgfusepath{stroke}
\pgfpathmoveto{\pgfpoint{201.134552pt}{109.184280pt}}
\pgflineto{\pgfpoint{196.625458pt}{110.116211pt}}
\pgfusepath{stroke}
\pgfpathmoveto{\pgfpoint{205.643631pt}{108.233330pt}}
\pgflineto{\pgfpoint{201.134552pt}{109.184280pt}}
\pgfusepath{stroke}
\pgfpathmoveto{\pgfpoint{210.152725pt}{107.269493pt}}
\pgflineto{\pgfpoint{205.643631pt}{108.233330pt}}
\pgfusepath{stroke}
\pgfpathmoveto{\pgfpoint{214.661804pt}{106.298920pt}}
\pgflineto{\pgfpoint{210.152725pt}{107.269493pt}}
\pgfusepath{stroke}
\pgfpathmoveto{\pgfpoint{219.170914pt}{105.327698pt}}
\pgflineto{\pgfpoint{214.661804pt}{106.298920pt}}
\pgfusepath{stroke}
\pgfpathmoveto{\pgfpoint{223.679993pt}{104.362000pt}}
\pgflineto{\pgfpoint{219.170914pt}{105.327698pt}}
\pgfusepath{stroke}
\pgfpathmoveto{\pgfpoint{228.189087pt}{103.407928pt}}
\pgflineto{\pgfpoint{223.679993pt}{104.362000pt}}
\pgfusepath{stroke}
\pgfpathmoveto{\pgfpoint{232.698181pt}{102.471634pt}}
\pgflineto{\pgfpoint{228.189087pt}{103.407928pt}}
\pgfusepath{stroke}
\pgfpathmoveto{\pgfpoint{237.207275pt}{101.559235pt}}
\pgflineto{\pgfpoint{232.698181pt}{102.471634pt}}
\pgfusepath{stroke}
\pgfpathmoveto{\pgfpoint{241.716370pt}{100.676872pt}}
\pgflineto{\pgfpoint{237.207275pt}{101.559235pt}}
\pgfusepath{stroke}
\pgfpathmoveto{\pgfpoint{246.225449pt}{99.830666pt}}
\pgflineto{\pgfpoint{241.716370pt}{100.676872pt}}
\pgfusepath{stroke}
\pgfpathmoveto{\pgfpoint{250.734543pt}{99.026764pt}}
\pgflineto{\pgfpoint{246.225449pt}{99.830666pt}}
\pgfusepath{stroke}
\pgfpathmoveto{\pgfpoint{255.243622pt}{98.271294pt}}
\pgflineto{\pgfpoint{250.734543pt}{99.026764pt}}
\pgfusepath{stroke}
\pgfpathmoveto{\pgfpoint{259.752716pt}{97.570374pt}}
\pgflineto{\pgfpoint{255.243622pt}{98.271294pt}}
\pgfusepath{stroke}
\pgfpathmoveto{\pgfpoint{264.261810pt}{96.930145pt}}
\pgflineto{\pgfpoint{259.752716pt}{97.570374pt}}
\pgfusepath{stroke}
\pgfpathmoveto{\pgfpoint{268.770905pt}{96.356750pt}}
\pgflineto{\pgfpoint{264.261810pt}{96.930145pt}}
\pgfusepath{stroke}
\pgfpathmoveto{\pgfpoint{273.279999pt}{95.856300pt}}
\pgflineto{\pgfpoint{268.770905pt}{96.356750pt}}
\pgfusepath{stroke}
\pgfpathmoveto{\pgfpoint{277.789093pt}{95.434929pt}}
\pgflineto{\pgfpoint{273.279999pt}{95.856300pt}}
\pgfusepath{stroke}
\pgfpathmoveto{\pgfpoint{282.298187pt}{95.098770pt}}
\pgflineto{\pgfpoint{277.789093pt}{95.434929pt}}
\pgfusepath{stroke}
\pgfpathmoveto{\pgfpoint{286.807251pt}{94.853973pt}}
\pgflineto{\pgfpoint{282.298187pt}{95.098770pt}}
\pgfusepath{stroke}
\pgfpathmoveto{\pgfpoint{291.316345pt}{94.706650pt}}
\pgflineto{\pgfpoint{286.807251pt}{94.853973pt}}
\pgfusepath{stroke}
\pgfpathmoveto{\pgfpoint{295.825439pt}{94.662926pt}}
\pgflineto{\pgfpoint{291.316345pt}{94.706650pt}}
\pgfusepath{stroke}
\pgfpathmoveto{\pgfpoint{300.334564pt}{94.728958pt}}
\pgflineto{\pgfpoint{295.825439pt}{94.662926pt}}
\pgfusepath{stroke}
\pgfpathmoveto{\pgfpoint{304.843628pt}{94.910858pt}}
\pgflineto{\pgfpoint{300.334564pt}{94.728958pt}}
\pgfusepath{stroke}
\pgfpathmoveto{\pgfpoint{309.352722pt}{95.214760pt}}
\pgflineto{\pgfpoint{304.843628pt}{94.910858pt}}
\pgfusepath{stroke}
\pgfpathmoveto{\pgfpoint{313.861816pt}{95.646790pt}}
\pgflineto{\pgfpoint{309.352722pt}{95.214760pt}}
\pgfusepath{stroke}
\pgfpathmoveto{\pgfpoint{318.370911pt}{96.213104pt}}
\pgflineto{\pgfpoint{313.861816pt}{95.646790pt}}
\pgfusepath{stroke}
\pgfpathmoveto{\pgfpoint{322.880005pt}{96.919807pt}}
\pgflineto{\pgfpoint{318.370911pt}{96.213104pt}}
\pgfusepath{stroke}
\pgfpathmoveto{\pgfpoint{327.389099pt}{97.773048pt}}
\pgflineto{\pgfpoint{322.880005pt}{96.919807pt}}
\pgfusepath{stroke}
\pgfpathmoveto{\pgfpoint{331.898193pt}{98.778938pt}}
\pgflineto{\pgfpoint{327.389099pt}{97.773048pt}}
\pgfusepath{stroke}
\pgfpathmoveto{\pgfpoint{336.407288pt}{99.943626pt}}
\pgflineto{\pgfpoint{331.898193pt}{98.778938pt}}
\pgfusepath{stroke}
\pgfpathmoveto{\pgfpoint{340.916382pt}{101.273239pt}}
\pgflineto{\pgfpoint{336.407288pt}{99.943626pt}}
\pgfusepath{stroke}
\pgfpathmoveto{\pgfpoint{345.425446pt}{102.773918pt}}
\pgflineto{\pgfpoint{340.916382pt}{101.273239pt}}
\pgfusepath{stroke}
\pgfpathmoveto{\pgfpoint{349.934540pt}{104.451775pt}}
\pgflineto{\pgfpoint{345.425446pt}{102.773918pt}}
\pgfusepath{stroke}
\pgfpathmoveto{\pgfpoint{354.443634pt}{106.312950pt}}
\pgflineto{\pgfpoint{349.934540pt}{104.451775pt}}
\pgfusepath{stroke}
\pgfpathmoveto{\pgfpoint{358.952728pt}{108.363571pt}}
\pgflineto{\pgfpoint{354.443634pt}{106.312950pt}}
\pgfusepath{stroke}
\pgfpathmoveto{\pgfpoint{363.461823pt}{110.609779pt}}
\pgflineto{\pgfpoint{358.952728pt}{108.363571pt}}
\pgfusepath{stroke}
\pgfpathmoveto{\pgfpoint{367.970917pt}{113.057701pt}}
\pgflineto{\pgfpoint{363.461823pt}{110.609779pt}}
\pgfusepath{stroke}
\pgfpathmoveto{\pgfpoint{372.479980pt}{115.713455pt}}
\pgflineto{\pgfpoint{367.970917pt}{113.057701pt}}
\pgfusepath{stroke}
\pgfpathmoveto{\pgfpoint{376.989075pt}{118.583191pt}}
\pgflineto{\pgfpoint{372.479980pt}{115.713455pt}}
\pgfusepath{stroke}
\pgfpathmoveto{\pgfpoint{381.498169pt}{121.673042pt}}
\pgflineto{\pgfpoint{376.989075pt}{118.583191pt}}
\pgfusepath{stroke}
\pgfpathmoveto{\pgfpoint{386.007263pt}{124.989128pt}}
\pgflineto{\pgfpoint{381.498169pt}{121.673042pt}}
\pgfusepath{stroke}
\pgfpathmoveto{\pgfpoint{390.516357pt}{128.537582pt}}
\pgflineto{\pgfpoint{386.007263pt}{124.989128pt}}
\pgfusepath{stroke}
\pgfpathmoveto{\pgfpoint{395.025452pt}{132.324524pt}}
\pgflineto{\pgfpoint{390.516357pt}{128.537582pt}}
\pgfusepath{stroke}
\pgfpathmoveto{\pgfpoint{399.534546pt}{136.356125pt}}
\pgflineto{\pgfpoint{395.025452pt}{132.324524pt}}
\pgfusepath{stroke}
\pgfpathmoveto{\pgfpoint{404.043640pt}{140.638489pt}}
\pgflineto{\pgfpoint{399.534546pt}{136.356125pt}}
\pgfusepath{stroke}
\pgfpathmoveto{\pgfpoint{408.552734pt}{145.177734pt}}
\pgflineto{\pgfpoint{404.043640pt}{140.638489pt}}
\pgfusepath{stroke}
\pgfpathmoveto{\pgfpoint{413.061798pt}{149.980011pt}}
\pgflineto{\pgfpoint{408.552734pt}{145.177734pt}}
\pgfusepath{stroke}
\pgfpathmoveto{\pgfpoint{417.570892pt}{155.051437pt}}
\pgflineto{\pgfpoint{413.061798pt}{149.980011pt}}
\pgfusepath{stroke}
\pgfpathmoveto{\pgfpoint{422.080017pt}{160.398163pt}}
\pgflineto{\pgfpoint{417.570892pt}{155.051437pt}}
\pgfusepath{stroke}
\pgfpathmoveto{\pgfpoint{426.589111pt}{166.026306pt}}
\pgflineto{\pgfpoint{422.080017pt}{160.398163pt}}
\pgfusepath{stroke}
\pgfpathmoveto{\pgfpoint{431.098145pt}{171.942001pt}}
\pgflineto{\pgfpoint{426.589111pt}{166.026306pt}}
\pgfusepath{stroke}
\pgfpathmoveto{\pgfpoint{435.607239pt}{178.151398pt}}
\pgflineto{\pgfpoint{431.098145pt}{171.942001pt}}
\pgfusepath{stroke}
\pgfpathmoveto{\pgfpoint{440.116364pt}{184.660583pt}}
\pgflineto{\pgfpoint{435.607239pt}{178.151398pt}}
\pgfusepath{stroke}
\pgfpathmoveto{\pgfpoint{444.625458pt}{191.475739pt}}
\pgflineto{\pgfpoint{440.116364pt}{184.660583pt}}
\pgfusepath{stroke}
\pgfpathmoveto{\pgfpoint{449.134552pt}{198.602951pt}}
\pgflineto{\pgfpoint{444.625458pt}{191.475739pt}}
\pgfusepath{stroke}
\pgfpathmoveto{\pgfpoint{453.643616pt}{206.048386pt}}
\pgflineto{\pgfpoint{449.134552pt}{198.602951pt}}
\pgfusepath{stroke}
\pgfpathmoveto{\pgfpoint{458.152710pt}{213.818161pt}}
\pgflineto{\pgfpoint{453.643616pt}{206.048386pt}}
\pgfusepath{stroke}
\pgfpathmoveto{\pgfpoint{462.661804pt}{221.918396pt}}
\pgflineto{\pgfpoint{458.152710pt}{213.818161pt}}
\pgfusepath{stroke}
\pgfpathmoveto{\pgfpoint{467.170898pt}{230.355255pt}}
\pgflineto{\pgfpoint{462.661804pt}{221.918396pt}}
\pgfusepath{stroke}
\pgfpathmoveto{\pgfpoint{471.679993pt}{239.134842pt}}
\pgflineto{\pgfpoint{467.170898pt}{230.355255pt}}
\pgfusepath{stroke}
\pgfpathmoveto{\pgfpoint{476.189087pt}{248.263290pt}}
\pgflineto{\pgfpoint{471.679993pt}{239.134842pt}}
\pgfusepath{stroke}
\pgfpathmoveto{\pgfpoint{480.698181pt}{257.746735pt}}
\pgflineto{\pgfpoint{476.189087pt}{248.263290pt}}
\pgfusepath{stroke}
\pgfpathmoveto{\pgfpoint{485.207275pt}{267.591339pt}}
\pgflineto{\pgfpoint{480.698181pt}{257.746735pt}}
\pgfusepath{stroke}
\pgfpathmoveto{\pgfpoint{489.716370pt}{277.803162pt}}
\pgflineto{\pgfpoint{485.207275pt}{267.591339pt}}
\pgfusepath{stroke}
\pgfpathmoveto{\pgfpoint{494.225433pt}{288.388428pt}}
\pgflineto{\pgfpoint{489.716370pt}{277.803162pt}}
\pgfusepath{stroke}
\pgfpathmoveto{\pgfpoint{498.734528pt}{299.353149pt}}
\pgflineto{\pgfpoint{494.225433pt}{288.388428pt}}
\pgfusepath{stroke}
\pgfpathmoveto{\pgfpoint{503.243622pt}{310.703583pt}}
\pgflineto{\pgfpoint{498.734528pt}{299.353149pt}}
\pgfusepath{stroke}
\pgfpathmoveto{\pgfpoint{507.752716pt}{322.445770pt}}
\pgflineto{\pgfpoint{503.243622pt}{310.703583pt}}
\pgfusepath{stroke}
\pgfpathmoveto{\pgfpoint{512.261780pt}{334.585907pt}}
\pgflineto{\pgfpoint{507.752716pt}{322.445770pt}}
\pgfusepath{stroke}
\pgfpathmoveto{\pgfpoint{516.770874pt}{347.130066pt}}
\pgflineto{\pgfpoint{512.261780pt}{334.585907pt}}
\pgfusepath{stroke}
\pgfpathmoveto{\pgfpoint{521.279968pt}{360.084412pt}}
\pgflineto{\pgfpoint{516.770874pt}{347.130066pt}}
\pgfusepath{stroke}
\end{pgfscope}
\end{pgfpicture}
}

    In all these cases, due to the missing data, the value of \verb|f| at 10
    tends to go \textit{wild}, i.e. far away from the given data in \verb|fx|.
    If anything, the interpolated/extrapolated ones looks more harmonic,
    while regression simply fit the curve into the function of the given form.
    It is not obvious that either technique is better is this case,
    since the amount of given data is too small.
\end{enumerate}
\end{document}
