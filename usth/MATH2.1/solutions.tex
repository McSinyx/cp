\documentclass[a4paper,12pt]{article}
\usepackage[english,vietnamese]{babel}
\usepackage{amsmath}
\usepackage{amssymb}
\usepackage{enumerate}
\usepackage{lmodern}
\usepackage{mathtools}
\usepackage{pgfplots}
\usepackage{venndiagram}

\newcommand{\E}{\mathbf E}
\newcommand{\C}{\mathrm C}
\newcommand{\N}{\mathcal N}
\newcommand{\ud}{\,\mathrm{d}}
\newcommand{\var}{\mathrm{var}}
\newcommand{\cov}{\mathrm{cov}}
\newcommand{\exercise}[1]{\noindent\textbf{#1.}}
\renewcommand{\thesubsection}{\Alph{subsection}}
\renewcommand{\thefootnote}{\fnsymbol{footnote}}

\title{Probability Homework}
\author{Nguyễn Gia Phong}
\date{Fall 2019}

\begin{document}
\maketitle
\section{Basic Probability 1}

\exercise 1  Problems regarding de Morgan's law
\begin{enumerate}[(a)]
  \item Consider rolling a six-sided die, where
    \begin{align*}
      &\begin{cases}
        A = \{2, 4, 6\}\iff A^\C = \{1, 3, 5\}\\
        B = \{4, 5, 6\}\iff B^\C = \{1, 2, 3\}
      \end{cases}\\
      \Longrightarrow &\begin{cases}
        (A \cup B)^\C = \{2, 4, 5, 6\}^\C = \{1, 3\} = A^\C \cap B^\C\\
        (A \cap B)^\C = \{4, 6\}^\C = \{1, 2, 3, 5\} = A^\C \cup B^\C\\
      \end{cases}
    \end{align*}
  \item By de Morgan's law,
    \begin{align*}
      P\left(A^\C \cap B^\C\right) &= P\left((A \cup B)^\C\right)\\
      &= 1 - P\left(A \cup \left(A^\C \cap B\right)\right)\\
      &= 1 - P(A) - P\left(A^\C \cap B\right)
      \tag{since $A \cap \left(A^\C \cap B\right) = \varnothing$}\\
      &= 1 -P(A) -P\left((A\cap B)\cup\left(A^\C\cap B\right)\right) +P(A\cap B)
      \tag{since $(A\cap B)\cap\left(A^\C\cap B\right) = \varnothing$}\\
      &= 1 - P(A) - P(B) + P(A\cap B)
    \end{align*}\label{1.b}
  \item Consider events A and B such that $P(A) = 1/2$, $P(A\cup B) = 3/4$,
    $P\left(B^\C\right) = 5/8$.

    \begin{align*}
      & P\left(A^\C\cap B\right) = P(A\cup B) - P(A)
      = \frac 3 4 - \frac 1 2 = \frac 1 4\\
      & P\left(A^\C\cap B^\C\right) = P\left((A\cup B)^\C\right)
      = P(\Omega) - P(A\cup B) = 1 - \frac 3 4 = \frac 1 4\\
      & P\left(A\cap B^\C\right) = P\left(B^\C\right)-P\left((A\cup B)^\C\right)
      = \frac 5 8 - \frac 1 4 = \frac 3 8\\
      & P(A\cap B) = P(A) - P\left(A\cap B^\C\right)
      = \frac 1 2 - \frac 3 8 = \frac 1 8\\
      & P\left(A^\C\cup B^\C\right) = P\left((A\cap B)^\C\right)
      = P(\Omega) - P(A\cap B) = 1 - \frac 1 8 = \frac 7 8
    \end{align*}
\end{enumerate}

\exercise 2  A four-sided die is rolled repeatedly,
until the first time (if ever) that an even number is obtained.
What is the sample space for this experiment?

Let the outcome be a n-dimensional vector, whose elements are values
of each roll in chronological order.  The sample space would then be
\[\Omega = \{v \in \{1, 3\}^m\times\{2, 4\} \mid m \in \mathbb N\}\]

\exercise 3  A ball is drawn at random from a box containing 6 red balls,
4 white balls, and 5 blue balls.

Let $\Omega$ be the sample space then $n(\Omega) = 6 + 4 + 5 = 15$.
Let the R, W and B be the event where a red, white and blue ball is drawn
respectively, each of these events are mutually exclusive.  Suppose each ball
is equally likely to be drawn, we get
\[\begin{cases}
  n(R) = 6\\
  n(W) = 4\\
  n(B) = 5
\end{cases}
\Longrightarrow\begin{dcases}
  P(R) = \frac{n(R)}{n(\Omega)} = \frac{2}{5}\\
  P(W) = \frac{n(W)}{n(\Omega)} = \frac{4}{15}\\
  P(B) = \frac{n(B)}{n(\Omega)} = \frac{1}{3}
\end{dcases}\]

\begin{enumerate}[(a)]
  \item For a ball that is not red to be drawn, the probability is
    \[P\left(R^\C\right) = P(\Omega) - P(R) = 1 - \frac 2 5 = \frac 3 5\]
  \item For a ball that is either red or white to be drawn, the probability is
    \[P(R\cup W) = P(R) + P(W) = \frac{2}{5} + \frac{4}{15} = \frac 2 3\]
\end{enumerate}

\exercise 4  Given $P(C_a) = 0.8$, $P(C_b) = 0.6$ and $P(C_a\cap C_b) = 0.5$.

We can easily prove that $P(C_a\cup C_b) = P(C_a) + P(C_b) - P(C_a\cap C_b)$
(similar to what we did in exercise 1.b).  Thus the probability that the student
will get at least one offer from these two companies is $0.8 + 0.6 - 0.5 = 0.9$.

\exercise 5  Let G and C be the events that the selected student is a genius and
is a chocolate lover, respectively, then $P(G) = 0.6$, $P(C) = 0.7$ and
$P(G\cap C) = 0.4$.  The probability that a randomly selected student is
neither a genius nor a chocolate lover is
\[P\left((G\cup C)^\C\right) = 1 - P(G) - P(C) + P(G\cap C)
= 1 - 0.6 - 0.7 + 0.4 = 0.1\]

\exercise 6  First, consider Rick's choice of entrance.  We denote
the outcome that he chooses each gate as $R_A$, $R_B$, $R_C$ and $R_D$,
then $P\{R_A\} = 1/3$ and $P\{R_B\} = P\{R_C\} = P\{R_D\} = 2/9$.
The sample space is $\Omega_R = \{R_A, R_B, R_C, R_D\}$.

Similarly, denote Brenda's and Ali's choices as $B_Y$ and $A_X$ respectively,
where $X$, $Y$ (and later $Z$) are one of the four entrances
$\omega = \{A, B, C, D\}$, we get
\[\begin{dcases}
  P\{B_A\} = P\{B_B\} = P\{B_C\} = P\{B_D\} = \frac 1 4\\
  P\{A_A\} = P\{A_B\} = \frac{2}{35}\\
  P\{A_C\} = \frac 2 7\\
  P\{A_D\} = \frac 3 5
\end{dcases}\]
The sample spaces of these two models are $\Omega_B = \{B_A, B_B, B_C, B_D\}$
and $\Omega_A = \{A_A, A_B, A_C, A_D\}$.

Now consider the probability model of the choices of the three friends.
The sample space is $\Omega = \Omega_R\times\Omega_B\times\Omega_A$.
Since the three friends chooses their entrance independently,
for all $\mathbf v = \langle R_Z, B_Y, A_X\rangle$ in $\Omega$,
\[P\{\mathbf v\} = P\{R_Z\} \cdot P\{B_Y\} \cdot P\{A_X\}\]

\begin{enumerate}[(a)]
  \item The event that at least two friends choose entrance B is
    \[a = (\Omega_R \times \{B_B\} \times \{A_B\})
    \cup (\{R_B\} \times \Omega_B \times \{A_B\})
    \cup (\{R_B\} \times \{B_B\} \times \Omega_A)\]

    Notice that 
    \begin{align*}
      &(\Omega_R \times \{B_B\} \times \{A_B\})
      \cap (\{R_B\} \times \Omega_B \times \{A_B\})\\
      =\,&(\Omega_R \times \{B_B\} \times \{A_B\})
      \cap (\{R_B\} \times \Omega_B \times \{A_B\})
      \cap (\{R_B\} \times \{B_B\} \times \Omega_A)\\
      =\,&\{R_B, B_B, A_B\}
    \end{align*}

    Therefore the probability of this event is
    \begin{align*}
      P(a) &= P(\Omega_A \times \{B_B\} \times \{A_B\})\\
      &+ P(\{R_B\} \times \Omega_B \times \{A_B\})\\
      &- P\{R_B, B_B, A_B\}\\
      &+ P(\{R_B\} \times \{B_B\} \times \Omega_A)\\
      &- P\{R_B, B_B, A_B\}\\
      &= P\{B_B\} \cdot P\{A_B\}\\
      &+ P\{R_B\} \cdot P\{A_B\}\\
      &+ P\{R_B\} \cdot P\{B_B\}\\
      &- 2 \cdot P\{R_B\} \cdot P\{B_B\} \cdot P\{A_B\}\\
      &= \frac{1}{4} \cdot \frac{2}{35}
       + \frac{2}{9} \cdot \frac{2}{35}
       + \frac{2}{9} \cdot \frac{1}{4}
       - \frac{2}{9} \cdot \frac{1}{4} \cdot \frac{2}{35}
       = \frac{8}{105}
    \end{align*}
  \item The only four cases where all friends choose the same entrance are
    $\{\langle R_X, B_X, A_X\rangle \mid X = \omega\}$.
    Hence the probability of this event is
    \begin{align*}
      P(b) &= \sum_{X \in \omega} P\{R_X\}\cdot P\{B_X\}\cdot P\{A_X\}\\
      &= \frac{1}{3} \cdot \frac{1}{4} \cdot \frac{2}{35}
       + \frac{2}{9} \cdot \frac{1}{4} \cdot \frac{2}{35}
       + \frac{2}{9} \cdot \frac{1}{4} \cdot \frac{2}{7}
       + \frac{2}{9} \cdot \frac{1}{4} \cdot \frac{3}{5}
       = \frac{2}{35}
    \end{align*}
\end{enumerate}

\exercise 7  We roll two fair six-sided dice.
\begin{enumerate}[(a)]
  \item The event that doubles are rolled has six outcomes, thus its probability
    is 6/36 = 1/6.
  \item Among the six outcomes where the result is four or less
    (\{(1, 1), (1, 2), (1, 3), (2, 1), (2, 2), (3, 1)\}),
    there are two that are doubles, hence the probability would then be 1/3.
  \item Let $\omega = \{1, 2, 3, 4, 5, 6\}$,
    the sample space is $\Omega = \omega^2$.  For one die roll is a six,
    the event is $C = (\{6\}\times\omega) \cup (\omega\times\{6\})$.
    Since $(\{6\}\times\omega) \cap (\omega\times\{6\}) = \{(6, 6)\}$,
    \begin{align*}
      P(C) &= P(\{6\}\times\omega) + P(\omega\times\{6\}) - P\{(6, 6)\}\\
      &= \frac{n(\{6\}\times\omega) + n(\omega\times\{6\}) - n\{(6, 6)\}}
              {n(\Omega)}\\
      &= \frac{6 + 6 - 1}{36} = \frac{11}{36}
    \end{align*}
\end{enumerate}

\exercise 8  A baby rolls two six-sided dice.  Assumed that the dice are fair.
Let $\omega = \{1, 2, 3, 4, 5, 6\}$, the sample space is $\Omega = \omega^2$.
\begin{enumerate}[(a)]
  \item There are six outcomes where the result of seven:
    \[A = \{(m, 7 - m) \mid m \in \omega\}\]
    hence this event's probability is
    \[P(A) = \frac{n(A)}{n(\Omega)} = \frac{6}{36} = \frac{1}{6}\]
  \item There are two outcomes where the result of eleven:
    $B = \{(5, 6), (6, 5)\}$, thus $P(B) = 1/18$.
    As $A$ and $B$ are disjoint, the probability of not getting
    a sum of seven or eleven is
    \begin{align*}
      P\left((A\cup B)^\C\right) &= P(\Omega) - P(A\cup B)\\
      &= 1 - (P(A) + P(B))\\
      &= 1 - \frac{1}{6} - \frac{1}{18} = \frac{7}{9}
    \end{align*}
\end{enumerate}

\exercise 9  Given $n(\Omega) = 25$, $n(C) = 9$, $n(D) = 8$
and $n\left((C\cup D)^\C\right) = 10$.

By the Venn diagram, $a = C\cap D^\C$, $b = C\cap D$ and $c = C^\C\cap D$.
Let $E = C\cup D$ and $d = E^\C$, $n(d) = 10$ and $n(E) = n(\Omega)-n(d) = 15$.
Assume that the boy fairly randomly selected,
\begin{align*}
  P(E) &= \frac{n(E)}{n(\Omega)}
  = \frac{15}{25}
  = \frac{3}{5}\\
  P(a) &= P\left((D\cap d)^\C\right)
  = 1 - P(D\cap d)
  = 1 - P(D) - P(d)\\
  &= 1 - \frac{n(D) + n(d)}{n(\Omega)}
  = 1 - \frac{8 + 10}{25}
  = \frac{7}{25}\\
  P(c) &= P\left((C\cap d)^\C\right)
  = 1 - P(C\cap d)
  = 1 - P(C) - P(d)\\
  &= 1 - \frac{n(C) + n(d)}{n(\Omega)}
  = 1 - \frac{9 + 10}{25}
  = \frac{6}{25}\\
  P(b) &= \frac{n(b)}{n(\Omega)}
  = \frac{n(E) - n(a) - n(c)}{n(\Omega)}
  = P(E) - P(a) - P(c)\\
  &= \frac{3}{5} - \frac{7}{25} - \frac{6}{25}
  = \frac{2}{25}
\end{align*}
\pagebreak

\exercise{10}
\begin{enumerate}[(a)]
  \item Venn diagram:
    \begin{venndiagram3sets}[
      labelOnlyB={5},
      labelABC={2},
      labelNotABC={0},
      overlap=1cm]
      \setpostvennhook{
        \draw[-stealth] (labelA) -- ++(135:2.5cm) node[left]{15};
        \draw[-stealth] (labelB) -- ++(45:2.5cm) node[right]{8};
        \draw[-stealth] (labelC) -- ++(-90:2cm) node[below]{12};
        \draw[-stealth] (0,0) -- ++(-135:1.5cm) node[below]{27};
        \draw[-stealth] (1.4,1.9) -- ++(180:2.5cm) node[left]{4};
        \draw[-stealth] (2.4,3.6) -- ++(-157:3.8cm);
        \draw[-stealth] (4,3) -- ++(-70:4cm) node[below]{21};
        \draw[-stealth] (1,2.5) -- ++(-40:5.3cm);
        \draw[-stealth] (3.3,1.5) -- ++(-50:3cm);
      }
    \end{venndiagram3sets}
  \item The number of tourists who had not visited Burundi:
    \[n\left(B^\C\right) = n(\Omega) - n(B) = 27 - 8 = 19\]
  \item The number of tourists who had not visited Cameroon
    unless they had visited all three countries:
    \[n\left((A\cap B)\cup C^\C\right)
    = n(\Omega) - n(C) + n(A\cap B\cap C)
    = 27 - 12 + 2 = 17\]
  \item For the randomly selected tourist to have visited at least
    two countries, that person must not visited only one country.
    Thus the event can be denoted as
    \[d = \Omega\setminus((A\setminus B\setminus C)\cup
    (B\setminus C\setminus A)\cup(C\setminus A\setminus B))\]
    Since the selection is random, the event's probability can be calculated as
    \[P(d) = 1 - \frac{n((A\setminus B\setminus C)\cup
    (B\setminus C\setminus A)\cup(C\setminus A\setminus B))}{n(\Omega)}
    = 1 - \frac{21}{27} = \frac{2}{9}\]
\end{enumerate}
\pagebreak

\section{Basic Probability 2}
\exercise 1  Let $A$ be the event that the chosen transistor is defective,
$B$ be the event that the chosen one is partially defective
and $C$ be the event that the chosen one is acceptable.
$A$, $B$ and $C$ are disjoint and $A\cup B\cup C = \Omega$, thus
\[n(\Omega) = n(A) + n(B) + n(C) = 5 + 10 + 25 = 40\]

The probability that the chosen transistor does not immediately fail is
$P\left(A^\C\right) = 1 - P(A) = 1 - n(A)/n(\Omega) = 1 - 5/40 = 7/8$.

Given this condition, the probability the chosen transistor is acceptable is
\[P\left(C|A^\C\right) = \frac{P\left(C\cap A^\C\right)}{P\left(A^\C\right)}
= \frac{P(C)}{7/8}
= \frac{8n(C)}{7n(\Omega)} = \frac{8\cdot 25}{7\cdot 40} = \frac 5 7\]

\exercise 2  Denote the outcomes of tossing a coin as $H$ (head) and $T$ (tail).
\begin{enumerate}[(a)]
  \item Consider tossing a coin $n$ times, the sample space is
    $\Omega = \{H, T\}^n$.  Let $A$ be the event of getting at least a head,
    $A^\C$ would then be getting all tails ($\{T\}^n$).  Suppose the chance of getting
    head and tail are equal,
    \[P\left(A^\C\right) = \frac{n\left(A^\C\right)}{n(\Omega)} = \frac{1}{2^n}
    \Longrightarrow P(A) = 1 - P\left(A^\C\right) = \frac{2^n - 1}{2^n}\]
  \item For $n = 4$, $P(A) = \dfrac{2^4 - 1}{2^4} = \dfrac{15}{16}$.
  \item Consider rolling a six-sided die $n$ times, the sample space is
    \[\Omega = \{1, 2, 3, 4, 5, 6\}^n\]
    Let $B$ be the event of getting a six, $B^\C = \{1, 2, 3, 4, 5\}^n$.
    Therefore the probability of $B$ is
    \[P(B) = 1 - P\left(B^\C\right)
    = 1 - \frac{n\left(B^\C\right)}{n(\Omega)} = 1 - \frac{5^n}{6^n}\]
    For $n = 4$, $P(B) = 671/1296$.
  \item For $P(B) = 0.99$,
    \[1 - \left(\frac{5}{6}\right)^n = 0.99
    \iff \left(\frac{5}{6}\right)^n = 0.01
    \iff n = \log_{5/6}0.01 \approx 25\]
\end{enumerate}

\exercise 3  Let $B$ be the event that the woman rides the bicycle to work,
$B^\C$ would be that she ride the scooter.  Let $L$ be that she is late,
\begin{align*}
  P(B) &= 0.7\\
  P\left(B^\C\right) &= 0.3\\
  P(L|B) &= 0.03\\
  P\left(L|B^\C\right) &= 0.02
\end{align*}
\begin{enumerate}[(a)]
  \item By Total Probability Theorem, the probability the woman
    is late for work is
    \[P(L) = P(B)\cdot P(L|B) + P\left(B^\C\right)\cdot P\left(L|B^\C\right)
    = 0.7\cdot 0.03 + 0.3\cdot 0.02 = 0.027\]
  \item The probability she is not late for work is
    \[P\left(L^\C\right) = 1 - P(L) = 1 - 0.027 = 0.973\]
    Since the woman is expected to be on time roughly 223 days a year,
    she goes to work $223 / P\left(L^\C\right) \approx 229$ days a year.
\end{enumerate}

\exercise 4  Consider flipping the coin twice, the sample space is
\[\Omega = \{(H, H), (H, T), (T, H), (T, T)\}\]
where $H$ stands for head and $T$ stands for tail.

Denote getting a head from the first flip as $H_1$ and getting a head
from the second one as $H_2$.  Assume that $P(H_1) = P(H_2) = 0.6$.
It is obvious that these two events are independent, or in other words
\[P(H_1\cap H_2) = P(H_1)\cdot P(H_2)\]

Similarly,
\begin{align*}
  P\{(H, T)\} &= P\left(H_1\cap H_2^\C\right)
  = P\left(H_1\right)\cdot\left(1 - P\left(H_2\right)\right) = 0.24\\
  P\{(T, H)\} &= P\left(H_1^\C\cap H_2\right)
  = \left(1 - P\left(H_1\right)\right)\cdot P\left(H_2\right) = 0.24
\end{align*}

Therefore if Minh and Nam flip the coin twice for both head and tail
and choose K-pop when they get a head first and US music otherwise,
the genre would be chosen equally even.

\exercise 5  Place three maths, two history and four biology book on a shelf.
\begin{enumerate}[(a)]
  \item There would be $(3 + 2 + 4)! = 362880$ ways to do it
    without any further restriction.
  \item If each subject needs to stay together,
    there are $3! 2! 4! 3! = 1728$ ways.
  \item If only biology books must stay together,
    we can do it in $4!(3 + 2 + 1)!$ or 17280 ways.
\end{enumerate}

\exercise 6  Seat six people around a table.
\begin{enumerate}[(a)]
  \item If they can sit anywhere, there are $6!/6 = 120$ arrangements.
  \item If two particular people must sit next to each other,
    there are $2\cdot 5!/5$ or 48 arrangements; thus if those two cannot
    sit side-by-side, the figure is $120 - 48 = 72$.
\end{enumerate}

\exercise 7  Let $\omega$ be the set of cards in a standard 52-card deck.
Shuffle the deck an draw seven cards,
the sample space of this probability model is $\Omega = \binom{\omega}{7}$,
$n(\Omega) = \binom{52}{7}$.
\begin{enumerate}[(a)]
  \item Let $A$ be the event that exactly three of the drawn ones are aces,
    \[n(A) = \binom{4}{3}\binom{48}{4}
    \Longrightarrow P(A) = \frac{n(A)}{n(\Omega)} = \frac{9}{1547}\]
  \item Let $K$ be the event that exactly two of the drawn ones are kings,
    \[n(K) = \binom{4}{2}\binom{48}{5}
    \Longrightarrow P(K) = \frac{n(K)}{n(\Omega)} = \frac{594}{7735}\]
  \item The probability that exactly three aces and two kings are drawn is
    \[n(A\cap K) = \binom{4}{3}\binom{4}{2}\binom{44}{2}
    \Longrightarrow P(A\cap K) = \frac{n(A\cap K)}{n(\Omega)}
    = \frac{1419}{8361535}\]

    Thus probability that either exactly three aces or two kings are drawn is
    \[P(A\cup K) = P(A) + P(K) - P(A\cap K) = \frac{137868}{1672307}\]
\end{enumerate}

\exercise 8  Let $M$ be the event that a red marble is picked
and $C$ be the event of getting head from tossing the coin, we have
\begin{align*}
  P(M|C) &= 0.6\\
  P\left(M|C^\C\right) &= 0.2\\
  P(C) = P\left(C^\C\right) &= 0.5
\end{align*}

\begin{enumerate}[(a)]
  \item By Total Probability Theorem, the probability a red marble is picked is
    \[P(M) = P(C)\cdot P(M|C) + P\left(C^\C\right)\cdot P\left(M|C^\C\right)
    = 0.4\]
  \item The probability that a blue marble is picked is
    \[P\left(M^\C\right) = 1 - P(M) = 0.6\]
  \item The probability of getting a head if the red marble is picked is
    \[P(C|M) = \frac{P(C\cap M)}{P(M)} = \frac{P(C)\cdot P(M|C)}{P(M)}
    = \frac{0.5\cdot 0.6}{0.4} = 0.75\]
\end{enumerate}

\exercise 9  Consider $n$ random people and their birthdays, assuming
that all 366 birthdays are equally likely\footnote{If you are wondering
how one could be equally likely to be born on the leap day, then well,
the distribution of birthdays on other days is in fact not uniform either.
\textit{Don't complicate it, don't drive yourself insane!}}.
The size of the sample space is $n(\Omega_n) = 366^n$.

Let $A_n$ be the event that no two of these $n$ people to celebrate
their birthday on the same day, $n(A_n) = \prod_{i=0}^{n-1}(366-i)$.
Thus the probability of this is
\[P(A_n) = \frac{n(A_n)}{n(\Omega_n)} = \prod_{i=1}^{n-1}\frac{366 - i}{366}\]

Since $P(A_{23}) < 0.5 < P(A_{22})$, $n$ needs to be at least 23
for the probability to be less than 0.5.

\exercise{10} The reasoning is not correct because:
\begin{itemize}
  \item If he is not to be released, the answer from the guard will be
    both of other prisoners, and everyones' fate will be known.
  \item Otherwise, in case the guard only gives one name,
    our protagonist will sure be released.
\end{itemize}
\pagebreak

\section{Discrete Random Variable 1}
\subsection{Discrete Random Variable and PMF}
\exercise 1  Consider a fair coin.
\begin{enumerate}[(a)]
  \item Toss it twice and let $X$ be the number of heads,
    $X$ would be a binomial random variable
    \begin{align*}
      X\colon \Omega &\to \{0, 1, 2\}\\
      \omega &\mapsto x
    \end{align*}
    whose probability mass function is
    \[p_X(x) = \binom{2}{x}\cdot 0.5^x\cdot 0.5^{2-x}
    = \frac{1}{2x!(2-x)!}\]

    Therefore PMF of $X$ for each case is
    \begin{align*}
      p_X(0) = p_X(2) &= \frac{1}{2\cdot 0!2!} = \frac{1}{4}\\
      p_X(1) &= \frac{1}{2\cdot 1!1!} = \frac{1}{2}
    \end{align*}
  \item Toss it thrice and let $Y$ be the number of heads,
    $Y$ would be a binomial random variable
    \begin{align*}
      Y\colon \Omega &\to \{0, 1, 2, 3\}\\
      \omega &\mapsto y
    \end{align*}
    whose probability mass function is
    \[p_Y(y) = \binom{3}{y}\cdot 0.5^y\cdot 0.5^{3-y}
    = \frac{3}{4y!(3-y)!}\]

    Therefore PMF of $Y$ for each case is
    \begin{align*}
      p_Y(0) = p_Y(3) &= \frac{3}{4\cdot 0!3!} = \frac{1}{8}\\
      p_Y(1) = p_Y(2) &= \frac{3}{4\cdot 1!2!} = \frac{3}{8}
    \end{align*}
\end{enumerate}

\exercise 2  Toss a pair of fair siz-sided dice
and let $X$ be the sum of the points
\begin{align*}
  X\colon \Omega &\to [2, 12]\cap\mathbb Z\\
  \omega &\mapsto x
\end{align*}
with $\Omega = S^2 = \{1, 2, 3, 4, 5, 6\}^2 \Longrightarrow n(\Omega) = 36$.

\begin{enumerate}[(a)]
  \item $X$ is a random variable whose PMF is
    \begin{align*}
      p_X(2) &= P\{(1,1)\} = \frac{1}{36}\\
      p_X(3) &= P\{(1,2), (2,1)\} = \frac{1}{18}\\
      p_X(4) &= P\{(1,3), (2,2), (3,1)\} = \frac{1}{12}\\
      p_X(5) &= P\{(1,4), (2,3), (3,2), (4,1)\} = \frac{1}{9}\\
      p_X(6) &= P\{(1,5), (2,4), (3,3), (4,2), (5,1)\} = \frac{5}{36}\\
      p_X(7) &= P\{(1,6), (2,5), (3,4), (4,3), (5,2), (6,1)\} = \frac{1}{6}\\
      p_X(8) &= P\{(2,6), (3,5), (4,4), (5,3), (6,2)\} = \frac{5}{36}\\
      p_X(9) &= P\{(3,6), (4,5), (5,4), (6,3)\} = \frac{1}{9}\\
      p_X(10) &= P\{(4,6), (5,5), (6,4)\} = \frac{1}{12}\\
      p_X(11) &= P\{(5,6), (6,5)\} = \frac{1}{18}\\
      p_X(12) &= P\{(6,6)\} = \frac{1}{36}
    \end{align*}
  \item The graph of $p_X(x)$:

    \begin{tikzpicture}
      \begin{axis}[xlabel={$x$}, ylabel={$p_X(x)$}]
        \addplot[ycomb, samples at={2,3,...,12}]{1/6-abs(x-7)/36};
      \end{axis}
    \end{tikzpicture}
\end{enumerate}
\pagebreak

\exercise 3  Denote the event of winning, tying and losing the first game as
$A_2$, $A_1$ and $A_0$ respectively, we get $P(A_2) = P(A_1) = 0.2$
and $P(A_0) = 0.6$.  Similarly, let $B_2$, $B_1$ and $B_0$ in that order
be the event MIT soccer team winning, tying and losing the second game,
we get $P(B_2) = P(B_1) = 0.35$ and $P(B_0) = 0.3$.

Let $A$ and $B$ be the random variable satisfying
\begin{align*}
  A &= \begin{cases}
    2\text{ if }A_2\\
    1\text{ if }A_1\\
    0\text{ if }A_0
  \end{cases}
  \Longrightarrow\begin{cases}
    p_A(2) = p_A(1) = 0.2\\
    p_A(0) = 0.6
  \end{cases}\\
  B &= \begin{cases}
    2\text{ if }B_2\\
    1\text{ if }B_1\\
    0\text{ if }B_0
  \end{cases}
  \Longrightarrow\begin{cases}
    p_B(2) = p_B(1) = 0.35\\
    p_B(0) = 0.3
  \end{cases}
\end{align*}
then the number of points the team earns over the weekend is $X = A + B$.

Since the outcome of the two games are independent,
\begin{align*}
  p_X(0) &= p_A(0)\cdot p_B(0) = 0.18\\
  p_X(1) &= p_A(0)\cdot p_B(1) + p_A(1)\cdot p_B(0) = 0.27\\
  p_X(2) &= p_A(0)\cdot p_B(2) + p_A(1)\cdot p_B(1) + p_A(2)\cdot p_B(0)
          = 0.34\\
  p_X(3) &= p_A(1)\cdot p_B(2) + p_A(2)\cdot p_B(1) = 0.14\\
  p_X(4) &= p_A(2)\cdot p_B(2) = 0.07
\end{align*}

\subsection{Expectation of Random Variables}
\exercise 4  Given a random variable
\[X = \begin{cases}
  -2&\text{ with probability of 1/3}\\
  3&\text{ with probability of 1/2}\\
  1&\text{ with probability of 1/6}
\end{cases}\]
\begin{align*}
  \E[X] &= \sum_{x\in\{-2, 1, 3\}}xp_X(x) = 1\\
  \E[2X+5] &= \sum_{x\in\{-2, 1, 3\}}(2x + 5)p_X(x) = 7\\
  \E\left[X^2\right] &= \sum_{x\in\{-2, 1, 3\}}x^2 p_X(x) = 6
\end{align*}

\exercise 5  Consider the genders of the three children, and assume
that both genders\footnote{You SJWs really need to calm down.
This is just a mathematical problem.} are equally likely.

Let $X$ be the number of girls, $X$ is a binomial random variable
whose probability mass function is
\begin{multline*}
  p_X(x) = \binom{3}{x}\cdot 0.5^x\cdot 0.5^{3-x} = \frac{3}{4x!(3-x)!}\\
  \Longrightarrow \E[X] = \sum_{x=0}^3\frac{3x}{4x!(3-x)!} = \frac 3 2
\end{multline*}

\exercise 6  Consider rolling a fair six-sided die, the sample space is
$\Omega = \{1, 2, 3, 4, 5, 6\}$.  Let $X$ be a random variable given by
\begin{multline*}
  X(\omega) = \begin{cases}
    -1&\text{ if }\omega \in \{1, 2, 3\}\\
    2&\text{  if }\omega \in \{4, 5\}\\
    8&\text{  if }\omega = 6
  \end{cases}
  \Longrightarrow\begin{dcases}
    p_X(-1) = \frac{3}{6} = \frac{1}{2}\\
    p_X(2) = \frac{2}{6} = \frac{1}{3}\\
    p_X(8) = \frac{1}{6}
  \end{dcases}\\
  \Longrightarrow \E[X] = \frac{-1}{2} + \frac{2}{3} + \frac{4}{3} = \frac{3}{2}
\end{multline*}

Practically, this means that at the end of the day,
it is very unlikely that the house will win.

\exercise 7  Let $X$ be the prize in dollars on a randomly chosen
lottery ticket, its PMF is
\begin{align*}
  p_X(100) &= \frac{5}{10\,000} = \frac{1}{2\,000}\\
  p_X(25) &= \frac{20}{10\,000} = \frac{1}{5\,000}\\
  p_X(5) &= \frac{200}{10\,000} = \frac{1}{500}\\
  p_X(0) &= \frac{10\,000-200-20-5}{10\,000} = \frac{391}{400}
\end{align*}

Thus the expected value for a ticket's value in dollars is
\[\E[X] = \sum_x x\cdot p_X(x) = \frac{13}{200}\]
or 6.5 cents.

\exercise 8  Let $X$ be the prize in dollars on a randomly chosen
raffle ticket, its PMF is
\begin{align*}
  & p_X(1998) = p_X(999) = \frac{1}{5000}\\
  & p_X(498) = \frac{2}{5000} = \frac{1}{2500}\\
  & p_X(98) = \frac{5}{5000} = \frac{1}{1000}\\
  & p_X(-2) = \frac{5000-5-2-1-1}{5000} = \frac{4991}{5000}
\end{align*}

Thus the expected value in dollars to get when buying a ticket is
\[\E[X] = \sum_x x\cdot p_X(x) = \frac{-11}{10}\]
or to lose \$1.1.

\subsection{Variance and Standard Deviation}
\exercise{9} Given the outcome $X$ from rolling a fair six-sided die.
\begin{align*}
  \E[X] &= \frac{1+2+3+4+5+6}{6} = \frac{7}{2}\\
  \Longrightarrow \var(X) &= \E\left[(X - \E[X])^2\right]
  = \sum_x \frac{(x - \frac{7}{2})^2}{6} = \frac{35}{24}\\
  \Longrightarrow \sigma_X &= \sqrt{\var(X)} = \sqrt\frac{35}{24}
\end{align*}

\exercise{10} Based on the result of exercise 2,
\[E[X] = 7,\qquad\var(X) = \frac{35}{6},\qquad\sigma_X = \sqrt\frac{35}{6}\]

\exercise{11} Given the integral random variable $X$ with PMF
\[p_X(x) = \begin{dcases}
  \frac{1}{9} &\text{ if } x \in [-4, 4]\\
  0 &\text{ otherwise}
\end{dcases}\]

Let $S = \{-4, -3, -2, -1, 0, 1, 2, 3, 4\}$,
\begin{multline*}
  \E[X] = \sum_{x\in\mathbb Z}x\cdot p_X(x)
  = \sum_{x\in S}\frac{x}{9} + \sum_{x\in\mathbb Z\setminus S}x\cdot 0 = 0\\
  \Longrightarrow \var(X) = \E\left[X^2\right]
  = \sum_{x\in S}\frac{x^2}{9} = \frac{20}{3}
\end{multline*}

\exercise{12} Given the integral random variable $X$ with PMF
\[p_X(x) = \begin{dcases}
  \frac{x^2}{a} &\text{ if } x \in [-3, 3]\\
  0 &\text{ otherwise}
\end{dcases}\]
\begin{enumerate}[(a)]
  \item Let $S = \{-3, -2, -1, 0, 1, 2, 3\}$. Since
    \begin{multline*}
      \sum_{x\in\mathbb Z}p_X(x) = 1
      \iff \sum_{x\in S}\frac{x^2}{a} + \sum_{x\in\mathbb Z\setminus S}0 = 1
      \iff a = \sum_{x\in S}x^2 = 28\\
      \Longrightarrow \E[X] = \sum_{x\in S}\frac{x^3}{28} = 0
    \end{multline*}
  \item Let $Z = (X - \E[X])^2 = X^2$,
    the range of $Z$ is $\{z^2\mid z\in\mathbb Z\}$.
    For all $z > 9$, it is trivial that $p_Z(z) = 0$.  Otherwise,
    \begin{align*}
      p_Z(0) &= P(Z = 0) = P(X=0) = p_X(0) = \frac{0^2}{28} = 0\\
      p_Z(1) &= P(X = \pm 1) = p_X(-1) + p_X(1)
      = \frac{(-1)^2}{28} + \frac{1^2}{28} = \frac{1}{14}\\
      p_Z(4) &= P(X = \pm 2) = p_X(-2) + p_X(2)
      = \frac{(-2)^2}{28} + \frac{2^2}{28} = \frac{2}{7}\\
      p_Z(9) &= P(X = \pm 3) = p_X(-3) + p_X(3)
      = \frac{(-3)^2}{28} + \frac{3^2}{28} = \frac{9}{14}
    \end{align*}
  \item The variance of $X$ is
    \[\var(X) = \E\left[(X - \E[X])^2\right]
    = \E[Z] = 1\cdot\frac{1}{14} + 4\cdot\frac{2}{7} + 9\cdot\frac{9}{14} = 7\]
\end{enumerate}
\pagebreak

\section{Discrete Random Variable 2}
\subsection{Conditional PMF and Expectation}
\exercise 1  Compute conditional PMF:
\begin{enumerate}[(a)]
  \item Let $X$ be the roll if a fair six-sided die and $A$ be the event that
    the roll is an number greater or equal to 4, we have $A = \{X \ge 4\}$
    and $P(A) = 0.5$, thus
    \[p_{X|A}(x) = \frac{P(\{X = x\}\cap\{X \ge 4\})}{P(A)}\]

    For $x \in \{1, 2, 3\}$, $\{X = x\}\cap\{X \ge 4\} = \varnothing$
    so $p_{X|A}(x) = 0/0.5 = 0$.

    For $x \in \{4, 5, 6\}$, $\{X = x\}\cap\{X \ge 4\} = \{x\}$,
    \[p_{X|A}(x) = \frac{1/6}{0.5} = \frac{1}{3}\]
  \item Let $X$ represent number of heads from the three-time toss
    of a fair coin and $B = \{X \ge 2\}$,
    $P(B) = \binom{3}{2}0.5^3 + \binom{3}{3}0.5^3 = 0.5$.
    \begin{multline*}
      p_{X|B}(x) = \frac{P(\{X = x\}\cap B)}{P(B)}
      = \frac{P(\{X = x\}\cap\{X \ge 2\})}{0.5}\\
      \Longrightarrow\begin{dcases}
        p_{X|B}(0) = p_{X|B}(1) = 0\\
        p_{X|B}(2) = \frac{\binom{3}{2}0.5^3}{0.5} = \frac{3}{4}\\
        p_{X|B}(3) = \frac{\binom{3}{3}0.5^3}{0.5} = \frac{1}{4}
      \end{dcases}
    \end{multline*}
  \item Let $X$ be the roll of a pair of fair dice and $C = \{X = 7\}$.
    As shown in the previous section, $P(C) = 1/6$ and thus
    \[p_{X|C}(x) = \begin{cases}
      1\text{ if }x = 7\\
      0\text{ if }x \neq 7
    \end{cases}\]
\end{enumerate}

\exercise 2  Consider the destination of the message and denote the event
it arrives at Liberty City, Chicago and San Fierro as $B$, $C$ and $F$
respectively, we have $B\cup C\cup F = \Omega$.  The expected transit time is
\begin{align*}
  \E[X] &= P(B)\E[X|B] + P(C)\E[X|C] + P(F)\E[X|F]\\
  &= 0.5\cdot 0.05 + 0.3\cdot 0.1 + 0.2\cdot 0.3 = 0.115
\end{align*}

\exercise 3  Let $V$ and $T$ be respectively the speed (in mph) and time
(in hours) Alyssa get to class.  Denote the event she walk to class as $W$,
$P(W) = 0.6$,
\[\begin{cases}
  \E[V|W] = 5\\
  \E\left[V|W^\C\right] = 30
\end{cases}
\Longrightarrow\begin{dcases}
  \E[T|W] = \frac{2}{5}\\
\E\left[T|W^\C\right] = \frac{1}{15}
\end{dcases}\]
\begin{enumerate}[(a)]
  \item The expected value of Alyssa's speed is
    \[\E[V] = P(W)\E[V|W] + P\left(W^\C\right)\E\left[V|W^\C\right]
    = 0.6\cdot 5 + (1 - 0.6)30 = 15\]
  \item The expected value of the time Alyssa she takes to get to class is
    \[\E[T] = P(W)\E[T|W] + P\left(W^\C\right)\E\left[T|W^\C\right]
    = 0.6\cdot\frac{2}{5} + (1 - 0.6)\frac{1}{15} = \frac{4}{15}\]
\end{enumerate}

\exercise 4  With $X$ being the number of tries until the program works
correctly and $p$ being the probability each try succeed,
we have $\mathrm{range}(X) = \mathbb N^*$ and $p_X(x) = (1 - p)^{x - 1}p$.
The mean of $X$ is
\begin{align*}
  \E[X] &= \sum_{x\in\mathbb N^*}x\cdot p_X(x)\\
  &= \sum_{x\in\mathbb N^*}x(1 - p)^{x - 1}p\\
  &= -p\sum_{x\in\mathbb N^*}\frac{\ud (1 - p)^x}{\ud p}\\
  &= -p\frac{\ud}{\ud p}\left(\sum_{x\in\mathbb N}(1 - p)^{x} - 1\right)\\
  &= -p\frac{\ud}{\ud p}\left(\frac{1}{p} - 1\right)\\
  &= \frac{1}{p}
\end{align*}

Similarly,
\begin{align*}
  \E\left[X^2\right] &= p\sum_{x\in\mathbb N^*}x^2(1 - p)^{x - 1}\\
  &= -p\frac{\ud}{\ud p}
  \left(\frac{1 - p}{p}\sum_{x\in\mathbb N^*}x(1 - p)^{x - 1}p\right)\\
  &= -p\frac{\ud}{\ud p}\left(\frac{1}{p^2} - \frac{1}{p}\right)\\
  &= \frac{2}{p^2} - \frac{1}{p}
\end{align*}

Therefore the variance of $X$ is
\[\var(X) = \E\left[X^2\right] - (\E[X])^2 = \frac{1}{p^2} - \frac{1}{p}\]

\subsection{Joint PMF and independent variables}
\exercise 5  Consider two independent coin tosses,
each with a 3/4 probability of a head,
and let $X$ be the number of heads obtained,
$X$ is a binomial random variable.
\[\E[X] = 0p_X(0) + 1p_X(1) + 2p_X(2)
= \binom{2}{1}\frac{3}{4}\left(1 - \frac{3}{4}\right)
+ 2\binom{2}{2}\left(\frac{3}{4}\right)^2
= \frac{3}{2}\]

\exercise 6  Let $X$ be the number of red traffic lights Alyssa encounters,
$X$ is a binomial random variable whose PMF is
\[p_X(x) = \binom{4}{x}0.5^4\]

The mean of $X$ is
\[\E[X] = \frac{1}{16}\sum_{x=0}^4 x\binom{4}{x} = 2\]

The variance of $X$ is
\[\var(X) = \E\left[X^2\right] - (\E[X])^2
= \frac{1}{16}\sum_{x=0}^4 x^2\binom{4}{x} - 4 = 1\]

\exercise 7  Let $X_i$ be 1 if the $i$th person gets his or her own hat and 0
otherwise, then for all positive integer $i \le n$
\[\E[X_i] = p_{X_i}(1) = \frac{(n - 1)!}{n!} = \frac{1}{n}\]
since if we fix one hat to its owner, there are $(n - 1)!$ arrangements
for the rest.  Due to the linearity property of expectation,
\[\E[X] = \E\left[\sum_{i=1}^n X_i\right] = \sum_{i=1}^n\E[X_i] = 1\]

\exercise 8  Consider four independent rolls of a six-sided die.
Let $X$ and $Y$ be the number of ones and twos obtained respectively,
both are binomial random variables:
\[p_X(k) = p_Y(k)
= \binom{4}{k}\left(\frac{1}{6}\right)^k\left(\frac{5}{6}\right)^{4 - k}
= \binom{4}{k}\frac{5^{4 - k}}{1296}\]

Given $Y = y$, $X$ is the number of ones in the remaining $4 - y$ rolls,
each of which can take the values other than two equally likely:
\[p_{X|Y}(x|y)
= \binom{4 - y}{x}\left(\frac{1}{5}\right)^x\left(\frac{4}{5}\right)^{4 - y - x}
= \binom{4 - y}{k}\frac{4^{4 - y - x}}{625}\]

Thus the joint PMF of $X$ and $Y$ is
\[p_{X,Y}(x, y) = p_Y(y)p_{X|Y}(x|y)
= \binom{4}{x}\binom{4 - y}{k}\frac{5^{4 - x}4^{4 - y - x}}{810\,000}\]

\exercise 9  Given the joint PMF of two discrete random variables $X$ and $Y$
\[p_{X,Y}(x, y) = \begin{cases}
  c(2x + y) &\text{where }(x, y)\in\{0, 1, 2\}\times\{0, 1, 2, 3\}\\
  0 &\text{otherwise}
\end{cases}\]
\begin{enumerate}[(a)]
  \item Consider all cases:
    \begin{align*}
      \sum_x\sum_y p_{X,Y}(x, y) = 1
      &\iff \sum_{x=0}^2\sum_{y=0}^3 c(2x + y) = 1\\
      &\iff \sum_{x=0}^2 c(8x + 6) = 1\\
      &\iff c(24 + 18) = 1\\
      &\iff c = \frac{1}{42}
    \end{align*}
  \item $P(X = 2, Y = 1) = (2\cdot 2 + 1)/42 = 5/42$.
  \item Similarly, $P(X \ge 1, Y \le 2) = 4/7$.
  \item The marginal PMF of $X$:
    \[p_X(x) = \sum_y p_{X,Y}(x, y)
    = \sum_{y=0}^3\frac{2x + y}{42} = \frac{4x + 3}{21}\]
  \item The marginal PMF of $Y$:
    \[p_Y(y) = \sum_x p_{X,Y}(x, y)
    = \sum_{x=0}^2\frac{2x + y}{42} = \frac{2 + y}{14}\]
  \item Since $p_X(2)p_Y(1) \neq p_{X,Y}(2, 1)$,
    the two variables are dependent.
  \item Given $X = 2$,
    \[p_{Y|X}(y|2) = \frac{p_{X,Y}(2, y)}{p_X(2)} = \frac{4 + y}{22}
    \Longrightarrow p_{Y|X}(1|2) = \frac{5}{22}\]
  \item Given $Y = 2$,
    \[p_{X|Y}(x|2) = \frac{p_{X,Y}(x, 2)}{p_Y(2)} = \frac{x + 1}{6}
    \Longrightarrow p_{X|Y}(3|2) = \frac{2}{3}\]
\end{enumerate}

\exercise{10} Given the joint PMF of two discrete random variables $X$ and $Y$
\[p_{X,Y}(x, y) = \begin{cases}
  cxy &\text{where }(x, y)\in\{1, 2, 3\}\times\{1, 2, 3\}\\
  0 &\text{otherwise}
\end{cases}\]
\begin{enumerate}[(a)]
  \item Consider all cases:
    \begin{align*}
      \sum_x\sum_y p_{X,Y}(x, y) = 1
      &\iff \sum_{x=1}^3\sum_{y=1}^3 cxy = 1\\
      &\iff 36c = 1\\
      &\iff c = \frac{1}{36}
    \end{align*}
  \item $P(X = 2, Y = 3) = 1/6$.
  \item Similarly, $P(1\le X\le 2, Y\le 2) = 1/4$.
  \item By the result of (e), $P(X\ge 2) = 5/6$,
    $P(Y < 2) = P(Y = 1) = P(X = 1) = 1/6$ and $P(Y = 3) = 1/2$.
  \item The marginal PMF of $X$:
    \[p_X(x) = \sum_y p_{X,Y}(x, y)
    = \sum_{y=1}^3\frac{xy}{36} = \frac{x}{6}\]
    The marginal PMF of $Y$:
    \[p_Y(y) = \sum_x p_{X,Y}(x, y)
    = \sum_{x=1}^3\frac{xy}{36} = \frac{y}{6}\]
  \item Since $p_{X,Y}(x, y) = p_X(x)p_Y(y)$, $X$ and $Y$ are independent.
\end{enumerate}
\pagebreak

\section{Continuous Random Variable 1}
\subsection{PDF and CDF}
\exercise 1  Given a PDF such that
\[f_X(x) = \begin{cases}
  cx^2 &\text{if }0 < x < 3\\
  0 &\text{otherwise}
\end{cases}\]
\begin{enumerate}[(a)]
  \item By the normalization property,
    \[1 = \int_{-\infty}^{\infty}f_X(x)\ud x = \int_0^3 cx^2\ud x = 9c
    \Longrightarrow c = \frac{1}{9}\]
  \item $\displaystyle P(1 < X < 2) = \int_1^2\frac{x^2}{9}\ud x = \frac{7}{27}$
  \item The CDF of $X$:
    \[F_X(x) = \begin{dcases}
      0 &\text{if }x \le 0\\
      1 &\text{if }x \ge 3\\
      \int_0^x\frac{t^2}{9}\ud t = \frac{x^3}{27} &\text{otherwise}
    \end{dcases}\]
  \item $P(1 < X \le 2) = F_X(2) - F_X(1)
    = \dfrac{8}{27} - \dfrac{1}{27} = \dfrac{7}{27}$
\end{enumerate}

\exercise 2  Denote the event that the day is sunny as $A$, $P(A) = 2/3$.
\begin{align*}
  f_{X|A}(x) &= \begin{cases}
    b &\text{if }15 \le x \le 20\\
    0 &\text{otherwise}
  \end{cases}
  \Longrightarrow \int_{15}^{20}b\ud x = 1
  \iff b = \frac{1}{5}\\
  f_{X|A^\C}(x) &= \begin{cases}
    c &\text{if }20 \le x \le 25\\
    0 &\text{otherwise}
  \end{cases}
  \Longrightarrow \int_{20}^{25}c\ud x = 1
  \iff c = \frac{1}{5}
\end{align*}

By the total probability theorem,
\[f_X(x) = P(A)f_{X|A}(x) + P\left(A^\C\right)f_{X|A^\C}(x)
  = \begin{dcases}
    \frac{2}{15} &\text{if }15 \le x < 20\\
    \frac{1}{15} &\text{if }20 \le x < 25\\
    0 &\text{otherwise}
  \end{dcases}\]
\pagebreak

\exercise 3  Given a random variable $X$ with the PDF
$f_X(x) = \dfrac{c}{x^2 + 1}$.
\begin{enumerate}[(a)]
  \item By the normalization property,
    \[\int_{-\infty}^\infty\frac{c}{x^2 + 1}\ud x = 1
      \iff \left.c\arctan x\right|_{-\infty}^\infty = 1
      \iff c\pi = 1 \iff c = \frac{1}{\pi}\]
  \item The probability that $X^2$ to lie between 1/3 and 1 is
    \begin{align*}
      P\left(\frac{1}{3} < X^2 < 1\right)
      &= P\left(-1 < X < \frac{-1}{\sqrt 3}\right)
       + P\left(\frac{1}{\sqrt 3} < X < 1\right)\\
      &= \left.\frac{\arctan x}{\pi}\right|_{-1}^{\frac{-1}{\sqrt 3}}
       + \left.\frac{\arctan x}{\pi}\right|_{\frac{1}{\sqrt 3}}^{1}
       = \frac{1}{6}
    \end{align*}
  \item The CDF of $X$:
    \[F_X(x) = \int_{-\infty}^{x}\frac{1}{\pi\left(x^2 + 1\right)}\ud t
      = \frac{\arctan x}{\pi} + \frac{1}{2}\]
\end{enumerate}

\exercise 4 Given a random variable $X$ with the CDF
\[F_X(x) = \begin{cases}
    1 - \exp(-2x) &\text{if } x \ge 0\\
    0 &\text{otherwise}
  \end{cases}\]
\begin{enumerate}[(a)]
  \item Let $g$ be the antiderivative of $f_X$, $g$ is a constant function
    in $(-\infty, 0)$ and for all positive $x$
    \[g(x) = g(0) + 1 - \exp(-2x)
      \Longrightarrow f_X(x) = 2\exp(-2x)\]
  \item The probability that $X > 2$ is
    \[P(X > 2) = P(\Omega) - P(X \le 2) = 1 - F_X(2) = \frac{1}{e^4}\]
  \item The probability that $-3 < X \le 4$ is
    \[P(-3 < X \le 4) = F_X(4) - F_X(-3) = 1 - \frac{1}{e^8}\]
\end{enumerate}

\exercise 5  Given random variable with the following PDF:
\[f_X(x) = \begin{dcases}
  \frac{10}{x^2} &\text{if }x > 10\\
  0 &\text{otherwise}
\end{dcases}
\Longrightarrow F_X(x) = \begin{dcases}
  1 - \frac{10}{x} &\text{if }x > 10\\
  0 &\text{otherwise}
\end{dcases}\tag{b}\]

\[P(X > 20) = P(\Omega) - F_X(20) = 1 - 1 + \frac{10}{20} = \frac{1}{2}\tag{a}\]

Let $Y$ be the number out of six devices that will function for at least
15 hours, $Y$ is a binomial random variable whose PMF is
\begin{align*}
  p_Y(y) &= \binom{6}{y} P^y(X \ge 15) P^{6-y}(X < 15)\\
  &= \binom{6}{y} \left(1 - F_X(15)\right)^y F_X^{6-y}(15)\\
  &= \binom{6}{y} \left(\frac{10}{15}\right)^y
     \left(1 - \frac{10}{15}\right)^{6-y}\\
  &= \binom{6}{y} \frac{2^y}{3^6}
\end{align*}

Denote $A$ as the event that at least three out of six devices will function
for at least 15 hours,
\[P(A) = \sum_{y=3}^6 p_Y(y) = \frac{656}{729}\tag{c}\]

\subsection{Expectation, Variance and STD\protect\footnote{No, not that STD.}}
\exercise 6  Given a random variable $X$ with the PDF
$f_X(x) = \frac{\lambda}{2}\exp(-\lambda|x|)$.
\begin{align*}
  \int_{-\infty}^\infty f_X(x)\ud x
  &= \int_{-\infty}^0 \frac{\lambda}{2}\exp(\lambda x)\ud x
   + \int_0^\infty \frac{\lambda}{2}\exp(-\lambda x)\ud x\\
  &= \frac{1}{2}\left(\int_{-\infty}^0\ud\exp(\lambda x)
   - \int_0^\infty\ud\exp(-\lambda x)\right)\\
  &= \frac{1 - (-1)}{2} = 1
\end{align*}

Let $g(x) = x f_X(x)$, the mean of $X$ is
$\E[X] = \displaystyle\int_{-\infty}^\infty g(x)\ud x$.  Since $g(-x) = -g(x)$
for all $x$, $\E[X] = 0$.

The variance of $X$ can be calculated as
$\var(X) = \E\left[X^2\right] - \E^2[X] = \E\left[X^2\right]$
or $\var(X) = \displaystyle\int_{-\infty}^\infty x^2 f_X(x)\ud x$.

Let $h(x) = x^2 f_X(x)$, we have $h(-x) = h(x)$ for all $x$ and thus
\begin{align*}
  \E\left[X^2\right] &= 2\int_{-\infty}^\infty h(x)\ud x\\
  &= 2\int_0^\infty\frac{\lambda x^2}{2}\exp(-\lambda x)\ud x\\
  &= 2\int_0^\infty\frac{x^2}{-2}\ud\exp(-\lambda x)\\
  &= 2\int_0^\infty\exp(-\lambda x)\ud\frac{x^2}{2}
   - \int_0^\infty\ud x^2\exp(-\lambda x)\\
  &= 2\int_0^\infty x\exp(-\lambda x)\ud x\\
  &= \frac{-2}{\lambda}\int_0^\infty x\ud\exp(-\lambda x)\\
  &= \frac{2}{\lambda}\left(\int_0^\infty\exp(-\lambda x)\ud x
   - \int_0^\infty\ud x\exp(-\lambda x)\right)\\
  &= \frac{-2}{\lambda^2}\int_0^\infty\ud\exp(-\lambda x)
   = \frac{2}{\lambda^2}
\end{align*}

\exercise 7  Given a random variable $X$ with PDF
\[f_X(x) = \begin{cases}
  2\exp(-2 x) &\text{if }x > 0\\
  0 &\text{otherwise}
\end{cases}\]

Since $X$ is exponentially distributed with the parameter $\lambda = 2$,
\[\E[X] = \sigma_X = \frac{1}{\lambda} = \frac{1}{2},\,
\var(X) = \frac{1}{\lambda^2} = \frac{1}{4}\]

Thus $\E\left[X^2\right] = \var(X) + \E^2[X] = 1/2$.

\exercise 8  Given a random variable $X$ with PDF
\[f_X(x) = \begin{cases}
  a + bx^2 &\text{if }0 \le x \le 1\\
  0 &\text{otherwise}
\end{cases}\]

By the normalization probability,
\[\int_0^1\left(a + bx^2\right)\ud x = 1 \iff a + \frac{b}{3} = 1\]

Since $\E[X] = 3/5$,
$\displaystyle\int_0^1\left(ax + 3x^3\right)\ud x = \frac{3}{5}$
or $a/2 + b/4 = 0.6$
and thus $a = 0.6$ and $b = 1.2$.

\exercise 9  The expectation is
\begin{align*}
  \E[X] &= \int_0^\infty\frac{x^2}{e^x}\ud x\\
  &= -\int_0^\infty x^2\ud e^{-x}\\
  &= \int_0^\infty e^{-x}\ud x^2 - \int_0^\infty\ud\frac{x^2}{e^x}\\
  &= \int_0^\infty\frac{2x}{e^x}\ud x\\
  &= -2\int_0^\infty x\ud e^{-x}\\
  &= 2\int_0^\infty e^{-x}\ud x - 2\int_0^\infty\ud\frac{x}{e^x}\\
  &= -2\int_0^\infty\ud e^{-x} = 2
\end{align*}

\exercise{10} The $X$ is exponentially distributed by the PDF
$f_X(x) = \dfrac{1}{3}\exp\dfrac{-x}{3}$.
\begin{enumerate}[(a)]
  \item $\E[X] = \sigma_X = 3$ and $\var(X) = 9$.
  \item The CDF of $X$ is $F_X(x) = 1 - \exp(-x/3)$ and thus
    \[P(2 < X \le 4) = F_X(4) - F_X(2) 
    = \exp\frac{-2}{3} - \exp\frac{-4}{3}\]
\end{enumerate}
\pagebreak

\section{Continuous Random Variable 2}
\exercise 1  Let $X$ be the time to repair the machine,
\begin{align*}
  f_X(x) &= \begin{dcases}
    \frac{1}{2}\exp\frac{-x}{2} &\text{if }x \ge 0\\
    0 &\text{otherwise}
  \end{dcases}\\
  \Longrightarrow F_X(x) &= \begin{dcases}
    1 - \exp\frac{-x}{2} &\text{if }x \ge 0\\
    0 &\text{otherwise}
  \end{dcases}
\end{align*}
\begin{enumerate}[(a)]
  \item $P(X > 2) = 1 - F_X(2) = \exp\dfrac{-2}{2} = \dfrac{1}{e}$
  \item Let $A = \{X > 10\}$ and $B = \{X > 8\}$, we have
    $P(B) = \exp\dfrac{-8}{2} = \dfrac{1}{e^4}$ and
    $P(A\cap B) = P(A) = \exp\dfrac{-10}{2} = \dfrac{1}{e^5}$.
    Hence the conditional probability that a repair exceeding eight hours
    takes at least 10 hours is
    \[P(A|B) = \frac{P(A\cap B)}{P(B)} = \frac{1}{e}\]
\end{enumerate}

\exercise 2  Denote the event that the day is sunny as $A$, $P(A) = 2/3$.
\begin{align*}
  f_{X|A}(x) &= \begin{cases}
    b &\text{if }15 \le x \le 23\\
    0 &\text{otherwise}
  \end{cases}
  \Longrightarrow \int_{15}^{20}b\ud x = 1
  \iff b = \frac{1}{8}\\
  f_{X|A^\C}(x) &= \begin{cases}
    c &\text{if }20 \le x \le 25\\
    0 &\text{otherwise}
  \end{cases}
  \Longrightarrow \int_{20}^{25}c\ud x = 1
  \iff c = \frac{1}{5}
\end{align*}

By the total probability theorem,
\[f_X(x) = P(A)f_{X|A}(x) + P\left(A^\C\right)f_{X|A^\C}(x)
  = \begin{dcases}
    \frac{1}{12} &\text{if }15 \le x < 20\\
    \frac{3}{20} &\text{if }20 \le x < 23\\
    \frac{1}{15} &\text{if }23 \le x < 25\\
    0 &\text{otherwise}
  \end{dcases}\]

\exercise 3  Let $X$ be the waiting time
and $A$ be the event that one arrives at the station before 7:15,
we have $P(A) = 0.25$, $P\left(A^\C\right) = 0.75$.

\begin{align*}
  f_{X|A}(x) &= \begin{dcases}
    \frac{1}{5} &\text{if }0 \le x < 5\\
    0 &\text{otherwise}
  \end{dcases}\\
  f_{X|A^\C}(x) &= \begin{dcases}
    \frac{1}{15} &\text{if }5 \le x \le 15\\
    0 &\text{otherwise}
  \end{dcases}
\end{align*}

By the total probability theorem,
\[f_X(x) = P(A)f_{X|A}(x) + P\left(A^\C\right)f_{X|A^\C}(x)
  = \begin{dcases}
    \frac{1}{10} &\text{if }0 \le x < 5\\
    \frac{1}{20} &\text{if }5 \le x < 15\\
    0 &\text{otherwise}
  \end{dcases}\]

\exercise 4  Let $X$ be a random variable with PDF
\[f_X(x) = \begin{dcases}
  \frac{x}{4} &\text{if }1 < x \le 3\\
  0 &\text{otherwise}
\end{dcases}\]
and $A = \{X \ge 2\}$.
\begin{enumerate}[(a)]
  \item $X$ has the mean of
    $\displaystyle\E[X] = \int_1^3\frac{x^2}{4}\ud x = \frac{13}{6}$.
    The CDF of $X$ is
    \[F_X(x) = \begin{dcases}
      0 &\text{if }x \le 1\\
      \frac{x^2 - 1}{8} &\text{if }1 < x \le 3\\
      1 &\text{otherwise}
    \end{dcases}\]
    thus $P(A) = P(\Omega) - P(X < 2) = 1 - F_X(2) = \dfrac{5}{8}$.
    \[f_{X|A}(x) = \frac{P(\{X = x\}\cap A)}{P(A)}
    = \frac{8}{5}P(\{X = x\}\cap A)\]
    It is trivial that $\{X = x\}\cap A = \varnothing$ if $x < 2$
    and $P(\{X = x\}\cap A) = f_X(x)$ otherwise, so
    \[f_{X|A}(x) = \begin{dcases}
      \frac{2x}{5} &\text{if }2 \le x \le 3\\
      0 &\text{otherwise}
    \end{dcases}
    \Longrightarrow \E[X|A] = \int_2^3\frac{2x^2}{5}\ud x = \frac{38}{15}\]
  \item Let $Y = X^2$, the $Y$ has the expectation of
    \[\E[Y] = \E\left[X^2\right] = \int_1^3\frac{x^3}{4}\ud x = 5\]

    The variance of $Y$ is
    \[\var(Y) = \E\left[Y^2\right] - \E^2[Y] = \E\left[X^4\right] - 5^2
    = \int_1^3\frac{x^5}{4}\ud x - 25 = \frac{16}{3}\]
\end{enumerate}

\exercise 5  Let $X$ be Alyssa's waiting time and $A$ be the event there is
a customer ahead, then $P(A) = P\left(A^\C\right) = 0.5$ and
\begin{align*}
  f_{X|A}(x) &= \begin{cases}
    \lambda\exp(-\lambda x) &\text{if }x \ge 0\\
    0 &\text{otherwise}
  \end{cases}\\
  \Longrightarrow F_{X|A}(x) &= \begin{cases}
    1 - \exp(-\lambda x) &\text{if }x \ge 0\\
    0 &\text{otherwise}
  \end{cases}\\
  p_{X|A^\C}(x) &= \begin{cases}
    1 &\text{if }x = 0\\
    0 &\text{otherwise}
  \end{cases}\\
  \Longrightarrow F_{X|A^\C}(x) &= \begin{cases}
    1 &\text{if }x \ge 0\\
    0 &\text{otherwise}
  \end{cases}
\end{align*}

Therefore
\begin{align*}
  F_X(x) &= P(A)F_{X|A}(x) + P\left(A^\C\right)F_{X^\C}(x)\\
  &= \begin{dcases}
    1 - 0.5\exp(-\lambda x) &\text{if }x \ge 0\\
    0 &\text{otherwise}
  \end{dcases}
\end{align*}

\exercise 6  Given the joint PDF of $X$ and $Y$
\[f_{X,Y}(x, y) = \begin{cases}
  cxy &\text{if }0 < x < 4\text{ and }1 < y < 5\\
  0 &\text{otherwise}
\end{cases}\]
\begin{enumerate}[(a)]
  \item By the normalization probability
    \[\int_0^4\int_1^5 cxy\ud y\ud x = 1 \iff 96c = 1 \iff c = \frac{1}{96}\]
  \item $\displaystyle P(1 < X < 2, 2 < Y < 3)
    = \int_1^2\int_2^3\frac{xy}{96}\ud y\ud x = \frac{5}{128}$
  \item $\displaystyle P(X \ge 3, Y \le 2)
    = \int_3^4\int_1^2\frac{xy}{96}\ud y\ud x = \frac{7}{128}$
  \item The marginal PDF of $X$ is $\displaystyle f_X(x)
    = \int_1^5\frac{xy}{96}\ud y = \frac{x}{8}$ and that of $Y$ is
    $\displaystyle f_Y(y) = \int_0^4\frac{xy}{96}\ud x = \frac{y}{12}$.
  \item The region with nonzero probability where $X + Y < 3$ is
    $\{(x, y) \in \mathbb R^2 \mid 0 < x < 2, 1 < y < 3 - x\}$, thus
    \[P(X + Y < 3) = \int_0^2\int_1^{3-x}\frac{xy}{96}\ud y\ud x
    = \int_0^2\frac{x^3 - 6x^2 + 8x}{192}\ud x = \frac{1}{48}\]
  \item Let $C_u$ be the line $X + 2Y = u$, then the PDF of $U = X + 2Y$ is
    \[f_U(u) = P(X + 2Y = u) = \int_{C_u}f_{X,Y}(x, y)\ud s\]
    where $\ud s$ is the infinitesimal length of $C_u$.

    Let $t$ satisfy $x = u - 2t$ and $y = t$ we get
    \begin{align*}
      f_U(u) &= \int_{-\infty}^\infty f_{X,Y}(u - 2t, t)
      \sqrt{\left(\frac{\ud x}{\ud t}\right)^2
      + \left(\frac{\ud y}{\ud t}\right)^2}\ud t\\
      &= \int_{-\infty}^\infty f_{X,Y}(u - 2t, t)\sqrt 5\ud t
    \end{align*}

    For $u < 2$, with $x \in (0, 4)$,
    \[0 < u - 2t < 4 \Longrightarrow 2t < u < 2 \iff y = t < 1\]
    and thus $f_{X,Y}(u - 2t, t) = 0$.  Similarly, the joint PDF of $X$ and $Y$
    also equals zero when $u > 15$, so $f_U(u) = 0$
    for $u \in \mathbb R \setminus [2, 14]$.

    For $u \in [2, 6]$,
    \begin{align*}
      \begin{cases}
        0 < x < 4\\
        1 < y < 5
      \end{cases}
      &\iff \begin{cases}
        0 < u - 2t < 4\\
        1 < t < 5
      \end{cases}\\
      &\iff \frac{u}{2} - 2 \le 1 < t < \frac{u}{2} < 5\\
      &\iff 1 < t < \frac{u}{2}
    \end{align*}
    so $\displaystyle f_U(u) = \frac{\sqrt 5}{96}\int_1^{u/2}(ut - 2t^2)\ud t
    = \frac{\sqrt 5}{2304}(u^3 - 12u + 16)$.

    For $u \in (6, 10)$,
    \begin{align*}
      \begin{cases}
        0 < x < 4\\
        1 < y < 5
      \end{cases}
      &\iff 1 < \frac{u}{2} - 2 < t < \frac{u}{2} < 5\\
      &\iff \frac{u}{2} - 2 < t < \frac{u}{2}
    \end{align*}
    so $\displaystyle f_U(u) = \frac{\sqrt 5}{96}
    \int_{u/2-2}^{u/2}(ut - 2t^2)\ud t
    = \frac{\sqrt 5}{288}(u^2 - 2u)$.

    For $u \in [10, 14]$,
    \begin{align*}
      \begin{cases}
        0 < x < 4\\
        1 < y < 5
      \end{cases}
      &\iff 1 < \frac{u}{2} - 2 < t < 5 \le \frac{u}{2}\\
      &\iff \frac{u}{2} - 2 < t < 5
    \end{align*}
    so $\displaystyle f_U(u) = \frac{\sqrt 5}{96}\int_{u/2-2}^5(ut - 2t^2)\ud t
    = \frac{\sqrt 5}{2304}(348u - u^3 - 2128)$.
\end{enumerate}

\exercise 7  Given the joint PDF of $X$ and $Y$
\[f_{X,Y}(x, y) = \begin{cases}
  8xy &\text{if }0 \le y \le x \le 1\\
  0 &\text{otherwise}
\end{cases}\]
\begin{enumerate}[(a)]
  \item The marginal PDFs are
    \begin{align*}
      f_X(x) &= \begin{dcases}
        \int_0^x 8xy\ud y = 4x^3 &\text{if }0 \le x \le 1\\
        0 &\text{otherwise}
      \end{dcases}\\
      f_Y(y) &= \begin{dcases}
        \int_y^1 8xy\ud x = 4y - 4y^3 &\text{if }0 \le y \le 1\\
        0 &\text{otherwise}
      \end{dcases}
    \end{align*}
  \item The conditional PDFs are
    \begin{align*}
      f_{X|Y}(x|y) = \frac{f_{X,Y}(x, y)}{f_Y(y)} &= \begin{dcases}
        \frac{2x}{1 - y^2} &\text{if }0 \le y \le x \le 1\\
        0 &\text{otherwise}
      \end{dcases}\\
      f_{Y|X}(y|x) = \frac{f_{X,Y}(x, y)}{f_X(x)} &= \begin{dcases}
        \frac{2y}{x^2} &\text{if }0 \le y \le x \le 1\\
        0 &\text{otherwise}
      \end{dcases}
    \end{align*}
  \item The conditional expectations are
    \begin{align*}
      \E[X|Y=y] &= \int_{-\infty}^\infty xf_{X|Y}(x|y)\ud x\\
      &= \begin{dcases}
        \int_y^1\frac{2x^2}{1-y^2}\ud x = \frac{2x^2 + 2x + 2}{3x + 3}
        &\text{if }0 \le y \le 1\\
        0 &\text{otherwise}
      \end{dcases}\\
      \E[Y|X=x] &= \int_{-\infty}^\infty yf_{Y|X}(y|x)\ud y\\
      &= \begin{dcases}
        \int_0^x\frac{2y^2}{x^2}\ud y = \frac{2x}{3} &\text{if }0 \le x \le 1\\
        0 &\text{otherwise}
      \end{dcases}
    \end{align*}
  \item It is trivial that given $X = x \in \mathbb R \setminus [0, 1]$,
    $\var(Y|X=x) = 0$. Otherwise,
    \begin{align*}
      \var(Y|X=x) &= \E\left[Y^2|X=x\right] - \E^2[Y|X=x]\\
      &= \int_0^x\frac{2y^3}{x^2}\ud y - \left(\frac{2x}{3}\right)^2
      = \frac{x^2}{2} - \frac{4x^2}{9} = \frac{x^2}{16}
    \end{align*}
\end{enumerate}

\exercise 8  Given the joint PDF of $X$ and $Y$
\[f_{X,Y}(x, y) = \begin{cases}
  \exp(-x-y) &\text{if }x \ge 0\text{ and }y \ge 0\\
  0 &\text{otherwise}
\end{cases}\]

The marginal PDFs are
\begin{align*}
  f_X(x) &= \begin{dcases}
    \int_0^\infty\exp(-x-y)\ud y = \exp(-x) &\text{if }x \ge 0\\
    0 &\text{otherwise}
  \end{dcases}\\
  f_Y(y) &= \begin{dcases}
    \int_0^\infty\exp(-x-y)\ud x = \exp(-y) &\text{if }y \ge 0\\
    0 &\text{otherwise}
  \end{dcases}
\end{align*}

It is noticeable that $X$ and $Y$ are independent,
thus $f_{X|Y}(x|y) = f_X(x)$ and $f_{Y|X}(y|x) = f_Y(y)$.
\pagebreak

\section{Continuous Random Variable 3}
\subsection{PDF and CDF}
\exercise 1  Given $Z \sim \N(0, 1)$.
\begin{enumerate}[(a)]
  \item $P(Z > 1.2) = 1 - \Phi(1.2) = 1 - 0.8849 = 0.1101$
  \item $P(-2 < Z < 2) = \Phi(2) - \Phi(-2) = 2\Phi(2) - 1
    = 2\cdot 0.9772 - 1 = 0.9544$
  \item $P(-1.2 < Z < 1) = \Phi(1) + \Phi(1.2) - 1
    = 0.8413 + 0.8849 - 1 = 0.7262$
\end{enumerate}

\exercise 2  Given $X \sim \N(4, 9)$.  Let $Y = \dfrac{X - 4}{3}$,
$Y$ is a standard normal random variable
and $F_X(x) = \Phi\left(\dfrac{x - 4}{3}\right)$.
\begin{enumerate}[(a)]
  \item $P(X > 6) = 1 - F_X(6) = 1 - \Phi\left(\dfrac{6 - 4}{3}\right)
    \approx 1 - \Phi(0.67) = 0.2514$
  \item $P(X > 1) = 1 - F_X(6) = 1 - \Phi\left(\dfrac{1 - 4}{3}\right)
    = 1 - \Phi(-1) =  \Phi(1) = 0.8413$
\end{enumerate}

\exercise 3  Let $X$ be the annual snowfall in inches
and $Y = \dfrac{X - 60}{20}$, we have $Y \sim \N(0, 1)$
and $F_X(x) = \Phi\left(\dfrac{x - 60}{20}\right)$.
The probability that snowfall will be at least 80 inches is
\[P(X \ge 80) = 1 - F_X(80) = 1 - \Phi\left(\frac{80 - 60}{20}\right)
= 1 - \Phi(1) = 1 - 0.8413 = 0.1587\]

\exercise 4  Let $X$ be the number of customers arriving during an one-hour
period, $X$ is a Poisson random variable whose PMF is
\begin{multline*}
  p_X(x) = \frac{24^x}{e^{24}x!},\qquad x \in \mathbb N\\
  \Longrightarrow P(X < 15) = \sum_{x=0}^{14}p_X(x)
  = \sum_{x=0}^{14}\frac{24^x}{e^{24}x!} \approx 0.019825332823463673
\end{multline*}

\subsection{Covariance and Correlation Coefficient}
\exercise 5  Given the joint PMF of $X$ and $Y$
\[p_{X,Y}(x, y) = \begin{cases}
  c(2x + y) &\text{where }x \in \{0, 1, 2\}\text{ and }y \in \{0, 1, 2, 3\}\\
  0 &\text{otherwise}
\end{cases}\]

By the normalization property,
\[\sum_{x=0}^2\sum_{y=0}^3 c(2x + y) = 1 \iff c = \frac{1}{42}\]
\begin{enumerate}[(a)]
  \item The maginal PMFs are
    \begin{align*}
      p_X(x) &= \sum_{y=0}^3 p_{X,Y}(x, y) = \begin{dcases}
        \frac{4x + 3}{21} &\text{if }x \in \{0, 1, 2\}\\
        0 &\text{otherwise}
      \end{dcases}\\
      p_Y(y) &= \sum_{x=0}^2 p_{X,Y}(x, y) = \begin{dcases}
        \frac{y + 2}{14} &\text{if }y \in \{0, 1, 2, 3\}\\
        0 &\text{otherwise}
      \end{dcases}
    \end{align*}

    Therefore we can compute these expectations:
    \begin{align*}
      \E[X] &= \sum_{x=0}^2 x p_X(x)
      = \sum_{x=0}^2\frac{4x^2 + 3x}{21} = \frac{29}{21}\\
      \E[Y] &= \sum_{y=0}^3 y p_Y(y)
      = \sum_{y=0}^3\frac{y^2 + 2y}{14} = \frac{13}{7}\\
      \E[XY] &= \sum_{x=0}^2\sum_{y=0}^3 xy p_{X,Y}(x,y)
      = \sum_{x=0}^2\sum_{y=0}^3\frac{2x^2 y + xy^2}{42} = \frac{17}{7}
    \end{align*}
  \item The variances of these variables can be calculated as
    \begin{align*}
      \var(X) &= \E\left[X^2\right] - \E^2[X] = \frac{230}{21}\\
      \var(Y) &= \E\left[Y^2\right] - \E^2[Y] = \frac{25}{7}
    \end{align*}
    where
    \begin{align*}
      \E\left[X^2\right] &= \sum_{x=0}^2 x^2 p_X(x)
      = \sum_{x=0}^2\frac{4x^3 + 3x^2}{21} = \frac{17}{7}\\
      \E\left[Y^2\right] &= \sum_{y=0}^3 y^2 p_Y(y)
      = \sum_{y=0}^3\frac{y^3 + 2y^2}{14} = \frac{32}{7}
    \end{align*}
  \item $\cov(X, Y) = \E[XY] - \E[X]\E[Y] = \dfrac{-20}{147}$ so
    \[\rho(X, Y) = \dfrac{\cov(X, Y)}{\sqrt{\var(X)\var(Y)}} \approx -0.027\]
\end{enumerate}

\exercise 6  Given the joint PDF of $X$ and $Y$ as followed
\[f_{X,Y}(x, y) = \begin{cases}
  c(2x + y) &\text{where }(x, y) \in (2, 6) \times (0, 5)\\
  0 &\text{otherwise}
\end{cases}\]

By the normalization property,
\[\int_2^6\int_0^5 c(2x + y)\ud y\ud x = 1 \iff c = \frac{1}{210}\]
\begin{enumerate}[(a)]
  \item The maginal PMFs are
    \begin{align*}
      f_X(x) &= \int_0^5 f_{X,Y}(x, y)\ud y = \begin{dcases}
        \frac{4x + 5}{84} &\text{if }2 < x < 6\\
        0 &\text{otherwise}
      \end{dcases}\\
      f_Y(y) &= \int_2^6 f_{X,Y}(x, y)\ud x = \begin{dcases}
        \frac{2y + 16}{105} &\text{if }0 < y < 5\\
        0 &\text{otherwise}
      \end{dcases}
    \end{align*}

    Therefore we can compute these expectations:
    \begin{align*}
      \E[X] &= \int_2^6 x f_X(x)\ud x
      = \int_1^6\frac{4x^2 + 5x}{84}\ud x = \frac{268}{63}\\
      \E[Y] &= \int_0^5 y f_Y(y)\ud y
      = \int_0^5\frac{2y^2 + 16y}{105}\ud y = \frac{170}{63}\\
      \E[XY] &= \int_2^6\int_0^5 xy f_{X,Y}(x,y)\ud y\ud x
      = \int_2^6\int_0^5\frac{2x^2 y + xy^2}{210}\ud y\ud x = \frac{80}{7}
    \end{align*}
  \item The variances of these variables can be calculated as
    \begin{align*}
      \var(X) &= \E\left[X^2\right] - \E^2[X] = \frac{5036}{3969}\\
      \var(Y) &= \E\left[Y^2\right] - \E^2[Y] = \frac{16225}{7938}
    \end{align*}
    where
    \begin{align*}
      \E\left[X^2\right] &= \int_2^6 x^2 f_X(x)
      = \int_2^6\frac{4x^3 + 5x^2}{84} = \frac{1220}{63}\\
      \E\left[Y^2\right] &= \int_0^5 y^2 f_Y(y)
      = \int_0^5\frac{2y^3 + 16y^2}{105} = \frac{1175}{126}
    \end{align*}
  \item $\cov(X, Y) = \E[XY] - \E[X]\E[Y] = \dfrac{-200}{3969}$ so
    \[\rho(X, Y) = \dfrac{\cov(X, Y)}{\sqrt{\var(X)\var(Y)}} \approx -0.0313\]
\end{enumerate}

\subsection{Derived Distribution}
\exercise 7  With $X$ being uniform on $[0, 1]$, by the normalization property,
$f_X(x) = 1$ $\Longrightarrow F_X(x) = x$ on this interval.  Given $Y = \sqrt X$,
\[F_Y(y) = P(Y \le y) = P\left(\sqrt X \le y\right) = F_X\left(y^2\right) = y^2
\Longrightarrow f_y(y) = \frac{\ud F_Y}{\ud y} = 2y\]
if $0 < Y < 1$, otherwise $f_Y(y) = 0$.

\exercise 8  Let $X$ be the speed in miles per hour,
\[f_X(x) = \begin{dcases}
  \frac{1}{30} &\text{if }30 \le x \le 60\\
  0 &\text{otherwise}
\end{dcases}
\Longrightarrow F_X(x) = \begin{dcases}
  0 &\text{if }x < 30\\
  \frac{x}{30} - 1 &\text{if }30 \le x < 60\\
  1 &\text{otherwise}
\end{dcases}\]
then the duration of the trip is $Y = 180/X$.  Where $3 \le Y \le 6$,
\begin{multline*}
  F_Y(y) = P\left(\frac{180}{X} \le y\right)
  = P\left(X \ge \frac{180}{y}\right)
  = 1 - F_X\left(\frac{180}{y}\right) = 2 - \frac{6}{y}\\
  \Longrightarrow f_Y(y) = \frac{\ud F_Y}{\ud y} = \frac{6}{y^2}
\end{multline*}

Otherwise, it is obvious that $f_Y(y) = 0$.

\exercise{9}\footnote{IMHO this is a really poor example to demonstrate
the usefulness of this method.} Let $A$ be the event that one arrives at
the station before 7:15, we have $P(A) = 0.25$ and $P\left(A^\C\right) = 0.75$.
The probability that $X = x$ is 0.05 for all $x \in [0, 20)$
and is 0 otherwise, thus
\begin{align*}
  F_{X|A}(x) &= \begin{dcases}
    0 &\text{if }x < 0\\
    \frac{\int_0^x 0.05\ud t}{0.25} = \frac{0.05x}{0.25}
    &\text{if }0 \le x < 5\\
    1 &\text{otherwise}
  \end{dcases}\\
  F_{X|A^\C}(x) &= \begin{dcases}
    0 &\text{if }x < 5\\
    \frac{\int_5^x 0.05\ud t}{0.25} = \frac{0.05x - 0.25}{0.75}
    &\text{if }5 \le x < 20\\
    1 &\text{otherwise}
  \end{dcases}
\end{align*}

With $Y = 5 - X$ if $A$ and $Y = 20 - X$ otherwise,
\begin{multline*}
  \begin{aligned}
    F_Y(y) &= P(Y \le y)\\
    &= 1 - P(Y > y)\\
    &= P(A)P(5 - X > y|A) + P\left(A^\C\right)P\left(20 - X > y|A^\C\right)\\
    &= 1 - 0.25P(X < 5 - y|A) - 0.75P\left(X < 20 - y|A^\C\right)\\
    &= 1 - 0.25F_{X|A}(5 - y) - 0.75F_{X|A^\C}(20 - y)\\
    &= \begin{dcases}
      0 &\text{if }y < 0\\
      1 - 0.05(5 - y) - 0.05(20 - y) + 0.25 = 0.1y &\text{if }0 \le y < 5\\
      1 - 0.05(20 - y) + 0.25 = 0.05y + 0.25 &\text{if }5 \le y < 15\\
      1 &\text{otherwise}
    \end{dcases}
  \end{aligned}\\
  \Longrightarrow f_Y(y) = \frac{\ud F_Y}{\ud y} = \begin{dcases}
    0.1 &\text{if }0 \le y < 5\\
    0.05 &\text{if }5 \le y < 15\\
    0 &\text{otherwise}
  \end{dcases}
\end{multline*}
\pagebreak

\setcounter{section}{8}
\section{Limit Theorem}
\exercise 1  Let $S_{100} = \sum_{i=1}^{100}X_i$, $M_{100} = S_{100}/100$
is the sample mean.  We have
\[Z_{100} = \frac{S_{100} - 100\cdot 10}{4\sqrt{100}} = 2.5M_{100} - 25\]

Since 100 is large, we can use the approximation
$P(Z_{100} \le z) \approx \Phi(z)$:
\begin{multline*}
  P(S_{100} \le 900) = P(M_{100} \le 9) = P(2.5M_{100} - 25 \le 22.5 - 25)\\
  = P(Z_n \le -2.5) \approx \Phi(-2.5) = 1 - \Phi(2.5) = 0.0062
\end{multline*}

\exercise 2  Let $X$ be the weight of a box in lbs, and denote
\[Z_{49} = \frac{\sum_{i=1}^{49}X_i - 49\cdot 205}{15\sqrt{49}}
= \frac{S_{49}}{105} - \frac{287}{3}\]

Since 49 is large, we can use the following approximation
to compute the probability that all 49 boxes can be
safely loaded onto the freight elevator and transported
\[P(S_{49} \le 9800) = P\left(Z_{49} \le \frac{-7}{3}\right)
\approx 1 - \Phi\left(\frac 7 3\right) = 0.0099\]

\exercise 3  Let $X$ be the number of tickets to be purchased by a student,
and denote
\[Z_{100} = \frac{\sum_{i=1}^{100}X_i - 100\cdot 2.4}{2\sqrt{100}}
= \frac{S_{100}}{20} - 12\]

Since 100 is large, we can use the following approximation
to compute the probability that all 100 students will be able to purchase
the tickets they desire from the 250 that is left:
\[P(S_{100} \le 250) = P(Z_{100} \le 0.5)
\approx \Phi(0.5) = 0.6915\]

\exercise 4  Let $X$ be the time in minutes to complete one problem,
and denote
\[Z_{40} = \frac{\sum_{i=1}^{40}X_i - 40\cdot 5}{2\sqrt{40}}
= \frac{S_{40} - 200}{4\sqrt{10}}\]

Since 40 is large, we can use the following approximation
to compute the probability that all 40 problems within 3 hours
\[P(S_{40} \le 180) = P\left(Z_{49} \le -\sqrt\frac{5}{2}\right)
\approx 1 - \Phi(1.58) = 0.0571\]

\exercise 5  Let $X$ be the size in MB of an image and denote
\[Z_{80} = \frac{\sum_{i=1}^{80}X_i - 80\cdot 0.6}{0.4\sqrt{80}}
= \frac{5S_{40} - 240}{8\sqrt{5}}\]

Since 80 is large, we can use the following formula to approximate
the probability that the total size is between 47 and 56 MB
\[P(47 \le S_{80} \le 56)
= P\left(\frac{\sqrt 5}{-8} \le Z_{49} \le \sqrt 5\right)
\approx \Phi(2.23) - 1 + \Phi(0.28) = 0.5974\]

\exercise 6  After 11 weeks, the station is supplied
\[74\,000 + 47\,000 \cdot 11 = 591\,000\text{ (gallons)}\]

Let $X$ be the gasoline consumption in gallons a week and denote
\[Z_{11} = \frac{\sum_{i=1}^{11}X_i - 11\cdot 50\,000}{10\,000\sqrt{11}}
= \frac{S_{11} - 550\,000}{10\,000\sqrt{11}}\]

While 11 is not exactly large, for the ease of calculation,
we still use the approximation $P(Z_{11} \le z) \approx \Phi(z)$.

\begin{enumerate}[(a)]
  \item The probability that the remain will be below 20\,000 gallons is
    \begin{multline*}
      P(591\,000 - S_{11} < 20\,000)
      = P(S_{11} > 571\,000)
      = P\left(Z_{11} > \frac{21}{10\sqrt{11}}\right)\\
      \approx 1 - \Phi(0.63) = 0.2643
    \end{multline*}
  \item Let $w$ be the weekly delivery satisfying the probability
    that below 20\,000 gallons will be remained is 0.5\%, we have
    \begin{align*}
      &P(74\,000 + 11w - S_{11} < 20\,000) = 0.005\\
      \iff &P(S_{11} > 54\,000 + 11w) = 0.005\\
      \iff &P\left(Z_{11}>\frac{11w-496\,000}{10\,000\sqrt{11}}\right) = 0.005\\
      \iff &1 - \Phi\left(\frac{11w-496\,000}{10\,000\sqrt{11}}\right) = 0.005\\
      \iff &\Phi\left(\frac{11w-496\,000}{10\,000\sqrt{11}}\right) = 0.995\\
      \iff &\frac{11w-496\,000}{10\,000\sqrt{11}} = 2.57\\
      \iff &w = 52840
    \end{align*}
\end{enumerate}
\end{document}
